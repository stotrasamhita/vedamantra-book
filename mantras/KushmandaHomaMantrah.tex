% !TeX program = XeLaTeX
% !TeX root = ../vedamantrabook.tex
\chapt{कूष्माण्ड-होमः}

\centerline{\scriptsize (तैत्तिरीयारण्यकम्/प्रपाठकः – २/अनुवाकः – ३)}

यद्दे॑वा देव॒हेड॑नम्।
देवा॑सश्चकृ॒मा व॒यम्।
आदि॑त्या॒\-स्तस्मा᳚न्मा मुञ्चत।
ऋ॒तस्य॒र्तेन॒ मामु॒त।
देवा॑ जीवनका॒म्या यत्।
वा॒चा\-ऽनृ॑तमूदि॒म।
अ॒ग्निर्मा॒ तस्मा॒देन॑सः।
गार्‌\mbox{}ह॑पत्यः॒ प्रमु॑ञ्चतु।
दु॒रि॒ता यानि॑ चकृ॒म।
क॒रोतु॒ माम॑ने॒नसम्᳚॥११६॥

ऋ॒तेन॑ द्यावापृथिवी।
ऋ॒तेन॒ त्वꣳ स॑रस्वति।
ऋ॒तान्मा॑ मुञ्च॒ताꣳह॑सः।
यद॒न्यकृ॑तमारि॒म।
स॒जा॒त॒श॒ꣳ॒सादु॒त वा॑ जामिश॒ꣳ॒सात्।
ज्याय॑सः॒ शꣳसा॑दु॒त वा॒ कनी॑यसः।
अना᳚ज्ञातं दे॒वकृ॑तं॒ यदेनः॑।
तस्मा॒त्त्वम॒स्माञ्जा॑तवेदो मुमुग्धि।
यद्वा॒चा यन्मन॑सा।
बा॒हुभ्या॑मू॒रुभ्या॑मष्ठी॒वद्भ्या᳚म्॥११७॥

%3.7.12.3
शि॒श्ञैर्यदनृ॑तं चकृ॒मा व॒यम्।
अ॒ग्निर्मा॒ तस्मा॒देन॑सः।
यद्धस्ता᳚भ्यां च॒कर॒ किल्बि॑षाणि।द्
अ॒क्षाणां᳚ व॒ग्नुमु॑प॒जिघ्न॑मानः।
दू॒रे॒प॒श्या च॑ राष्ट्र॒भृच्च॑।
तान्य॑फ्स॒रसा॒वनु॑दत्तामृ॒णानि॑।
अदी᳚व्यन्नृ॒णं यद॒हं च॒कार॑।
यद्वादा᳚स्यन्थ्सञ्ज॒गारा॒ जने᳚भ्यः।
अ॒ग्निर्मा॒ तस्मा॒देन॑सः।
यन्मयि॑ मा॒ता गर्भे॑ स॒ति॥११८॥

%3.7.12.4
एन॑श्च॒कार॒ यत्पि॒ता।
अ॒ग्निर्मा॒ तस्मा॒देन॑सः।
यदा॑ पि॒पेष॑ मा॒तरं॑ पि॒तरम्᳚।
पु॒त्रः प्रमु॑दितो॒ धयन्।
अहिꣳ॑सितौ पि॒तरौ॒ मया॒ तत्।
तद॑ग्ने अनृ॒णो भ॑वामि।
यद॒न्तरि॑क्षं पृथि॒वीमु॒त द्याम्।
यन्मा॒तरं॑ पि॒तरं॑ वा जिहिꣳसि॒म।
अ॒ग्निर्मा॒ तस्मा॒देन॑सः।
यदा॒शसा॑ नि॒शसा॒ यत्प॑रा॒शसा᳚॥११९॥

%3.7.12.5
यदेन॑श्चकृ॒मा नूत॑नं॒ यत्पु॑रा॒णम्।
अ॒ग्निर्मा॒ तस्मा॒देन॑सः।
अति॑ क्रामामि दुरि॒तं यदेनः॑।
जहा॑मि रि॒प्रं प॑र॒मे स॒धस्थे᳚।
यत्र॒ यन्ति॑ सु॒कृतो॒ नापि॑ दु॒ष्कृतः॑।
तमा रो॑हामि सु॒कृतां॒ नु लो॒कम्।
त्रि॒ते दे॒वा अ॑मृजतै॒तदेनः॑।
त्रि॒त ए॒तन्म॑नु॒ष्ये॑षु मामृजे।
ततो॑ मा॒ यदि॒ किञ्चि॑दान॒शे।
अ॒ग्निर्मा॒ तस्मा॒देन॑सः॥१२०॥

%3.7.12.6
गार्‌\mbox{}ह॑पत्यः॒ प्रमु॑ञ्चतु।
दु॒रि॒ता यानि॑ चकृ॒म।
क॒रोतु॒ माम॑ने॒नसम्᳚।
दि॒वि जा॒ता अ॒फ्सु जा॒ताः।
या जा॒ता ओष॑धीभ्यः।
अथो॒ या अ॑ग्नि॒जा आपः॑।
ता नः॑ शुन्धन्तु॒ शुन्ध॑नीः।
यदापो॒ नक्तं॑ दुरि॒तं चरा॑म।
यद्वा॒ दिवा॒ नूत॑नं॒ यत्पु॑रा॒णम्।
हिर॑ण्यवर्णा॒स्तत॒ उत्पु॑नीत नः।
इ॒मं मे॑ वरुण॒ तत्त्वा॑ यामि।
त्वन्नो॑ अग्ने॒ स त्वन्नो॑ अग्ने।
त्वम॑ग्ने अ॒यासि॑॥१२१॥


यद्दे॑वा देव॒हेळ॑नं॒ देवा॑सश्चकृ॒मा व॒यम्। 
आदि॑त्या॒स्तस्मा᳚न्मा मुञ्चत॒र्तस्य॒र्तेन॒ मामि॒त। 
देवा॑ जीवनका॒म्या यद्वा॒चाऽनृ॑त\-मूदि॒म। 
तस्मा᳚न्न इ॒ह मु॑ञ्चत॒ विश्वे॑ देवाः स॒जोष॑सः। 
ऋ॒तेन॑ द्यावापृथिवी ऋ॒तेन॒ त्वꣳ स॑रस्वति। 
कृ॒तान्नः॑ पा॒ह्येन॑सो॒ यत्किं चानृ॑त\-मूदि॒म। 
इ॒न्द्रा॒ग्नी मि॒त्रावरु॑णौ॒ सोमो॑ धा॒ता बृह॒स्पतिः॑। 
ते नो॑ मुञ्च॒न्त्वेन॑सो॒ यद॒न्यकृ॑तमारि॒म। 
स॒जा॒त॒श॒ꣳ॒सादु॒त जा॑मिश॒ꣳ॒साज्ज्याय॑सः॒ शꣳसा॑दु॒त वा॒ कनी॑यसः। 
अना॑धृष्टं दे॒वकृ॑तं॒ यदेन॒स्तस्मा॒त् त्वम॒स्माञ्जा॑तवेदो मुमुग्धि॥३॥

यद्वा॒चा यन्मन॑सा बा॒हुभ्या॑मू॒रुभ्या॑मष्ठी॒वद्भ्याꣳ॑ शि॒श्नैर्यदनृ॑तं चकृ॒मा व॒यम्। 
अ॒ग्निर्मा॒ तस्मा॒देन॑सो॒ गार्\mbox{}ह॑पत्यः॒ प्रमु॑ञ्चतु चकृ॒म यानि॑ दुष्कृ॒ता। 
येन॑ त्रि॒तो अ॑र्ण॒वान्नि॑र्ब॒भूव॒ येन॒ सूर्यं॒ तम॑सो निर्मु॒मोच॑। 
येनेन्द्रो॒ विश्वा॒ अज॑हा॒दरा॑ती॒स्तेना॒हं ज्योति॑षा॒ ज्योति॑रानशा॒न आ᳚क्षि। 
यत्कुसी॑द॒मप्र॑तीत्तं॒ मये॒ह येन॑ य॒मस्य॑ नि॒धिना॒ चरा॑मि। 
ए॒तत्तद॑ग्ने अनृ॒णो भ॑वामि॒ जीव॑न्ने॒व प्रति॒ तत्ते॑ दधामि। 


\centerline{\scriptsize (तैत्तिरीय-ब्राह्मणम् ३.७.१२)}


यन्मयि॑ मा॒ता गर्भे॑ स॒ति॥११८॥

%3.7.12.4
एन॑श्च॒कार॒ यत्पि॒ता।
अ॒ग्निर्मा॒ तस्मा॒देन॑सः।
यदा॑ पि॒पेष॑ मा॒तरं॑ पि॒तरम्᳚।
पु॒त्रः प्रमु॑दितो॒ धयन्।
अहिꣳ॑सितौ पि॒तरौ॒ मया॒ तत्।
तद॑ग्ने अनृ॒णो भ॑वामि।
यद॒न्तरि॑क्षं पृथि॒वीमु॒त द्याम्।
यन्मा॒तरं॑ पि॒तरं॑ वा जिहिꣳसि॒म।
अ॒ग्निर्मा॒ तस्मा॒देन॑सः।
यदा॒शसा॑ नि॒शसा॒ यत्प॑रा॒शसा᳚॥११९॥

%3.7.12.5
यदेन॑श्चकृ॒मा नूत॑नं॒ यत्पु॑रा॒णम्।
अ॒ग्निर्मा॒ तस्मा॒देन॑सः।
अति॑ क्रामामि दुरि॒तं यदेनः॑।
जहा॑मि रि॒प्रं प॑र॒मे स॒धस्थे᳚।
यत्र॒ यन्ति॑ सु॒कृतो॒ नापि॑ दु॒ष्कृतः॑।
तमा रो॑हामि सु॒कृतां॒ नु लो॒कम्।
त्रि॒ते दे॒वा अ॑मृजतै॒तदेनः॑।
त्रि॒त ए॒तन्म॑नु॒ष्ये॑षु मामृजे।
ततो॑ मा॒ यदि॒ किञ्चि॑दान॒शे।
अ॒ग्निर्मा॒ तस्मा॒देन॑सः॥१२०॥

%3.7.12.6
गार्‌\mbox{}ह॑पत्यः॒ प्रमु॑ञ्चतु।
दु॒रि॒ता यानि॑ चकृ॒म।
क॒रोतु॒ माम॑ने॒नसम्᳚।
दि॒वि जा॒ता अ॒फ्सु जा॒ताः।
या जा॒ता ओष॑धीभ्यः।
अथो॒ या अ॑ग्नि॒जा आपः॑।
ता नः॑ शुन्धन्तु॒ शुन्ध॑नीः।
यदापो॒ नक्तं॑ दुरि॒तं चरा॑म।
यद्वा॒ दिवा॒ नूत॑नं॒ यत्पु॑रा॒णम्।
हिर॑ण्यवर्णा॒स्तत॒ उत्पु॑नीत नः।
इ॒मं मे॑ वरुण॒ तत्त्वा॑ यामि।
त्वन्नो॑ अग्ने॒ स त्वन्नो॑ अग्ने।
त्वम॑ग्ने अ॒यासि॑॥१२१॥


इ॒मं मे॑ वरुण श्रुधी॒ हव॑म॒द्या च॑ मृडय। त्वाम॑व॒स्युराच॑के। तत्त्वा॑ यामि॒ ब्रह्म॑णा॒ वन्द॑मान॒स्तदाऽऽशा᳚स्ते॒ यज॑मानो ह॒विर्भिः॑। अहे॑डमानो वरुणे॒ह बो॒द्ध्युरु॑शꣳस॒ मा न॒ आयुः॒ प्रमो॑षीः॥

त्वं नो॑ अग्ने॒ वरु॑णस्य वि॒द्वां दे॒वस्य॒ हेडो\-ऽव॑ यासिसीष्ठाः। यजि॑ष्ठो॒ वह्नि॑तमः॒ शोशु॑चानो॒ विश्वा॒ द्वेषाꣳ॑सि॒ प्र मु॑मुग्ध्य॒स्मत्। स त्वं नो॑ अग्ने\-ऽव॒मो भ॑वो॒ती नेदि॑ष्ठो अ॒स्या उ॒षसो॒ व्यु॑ष्टौ। अव॑ यक्ष्व नो॒ वरु॑ण॒ꣳ॒ ररा॑णो वी॒हि मृ॑डी॒कꣳ सु॒हवो॑ न एधि।



यददी᳚व्यन्नृ॒णम॒हं ब॒भूवादि॑थ्सन्वा सञ्ज॒गर॒ जने᳚भ्यः। 
अ॒ग्निर्मा॒ तस्मा॒दिन्द्र॑श्च संविदा॒नौ प्रमु॑ञ्चताम्। 
यद्धस्ता᳚भ्यां च॒कर॒ किल्बि॑षाण्य॒क्षाणां᳚ व॒ग्नुमु॑प॒जिघ्न॑मानः। 
उ॒ग्रं॒ प॒श्या च॑ राष्ट्र॒भृच्च॒ तान्य॑फ्स॒रसा॒वनु॑दत्तामृ॒णानि॑। 
उग्रं॑ पश्ये॒ राष्ट्र॑भृ॒त्किल्बि॑षाणि॒ यद॒क्षवृ॑त्त॒मनु॑दत्तमे॒तत्। 
नेन्न॑ ऋ॒णानृ॒णव॒ इथ्स॑मानो य॒मस्य॑ लो॒के अधि॑रज्जु॒राय॑। 

अव॑ ते॒ हेळ॒ उदु॑त्त॒ममि॒मं मे॑ वरुण॒ तत्त्वा॑ यामि॒ त्वं नो॑ अग्ने॒ स त्वं नो अग्ने। 
सङ्कु॑सुको॒ विकु॑सुको निर्\mbox{}ऋ॒थो यश्च॑ निस्व॒नः। 
तेऽ(१\char"E009)स्मद्यक्ष्म॒मना॑गसो दू॒राद्दू॒रम॑चीचतम्। 
निर्य॑क्ष्ममचीचते कृ॒त्यां निर्\mbox{}ऋ॑तिं च। 
तेन॒ योऽ(१\char"E009)स्मथ्समृ॑च्छातै॒ तम॑स्मै॒ प्रसु॑वामसि। 
दुः॒श॒ꣳ॒सा॒नु॒श॒ꣳ॒साभ्यां᳚ घ॒णेना॑नुघ॒णेन॑ च। 
तेना॒न्योऽ(१\char"E009)स्मथ्समृ॑च्छातै॒ तम॑स्मै॒ प्रसु॑वामसि। 
सं वर्च॑सा॒ पय॑सा॒ सन्त॒नूभि॒रग॑न्महि॒ मन॑सा॒ सꣳ शि॒वेन॑। 
त्वष्टा॑ नो॒ अत्र॒ विद॑धातु रा॒योऽनु॑मार्ष्टु त॒न्वो(१\char"E009) यद्विलि॑ष्टम्॥५॥\anuvakamend


आयु॑ष्टे वि॒श्वतो॑ दधद॒यम॒ग्निर्वरे᳚ण्यः। 
पुन॑स्ते प्रा॒ण आया॑ति॒ परा॒यक्ष्मꣳ॑ सुवामि ते। 
आ॒यु॒र्दा अ॑ग्ने ह॒विषो॑ जुषा॒णो घृ॒तप्र॑तीको घृ॒तयो॑निरेधि। 
घृ॒तं पी॒त्वा मधु॒ चारु॒ गव्यं॑ पि॒तेव॑ पु॒त्रम॒भिर॑क्षतादि॒मम्। 
इ॒मम॑ग्न॒ आयु॑षे॒ वर्च॑से कृधि ति॒ग्ममोजो॑ वरुण॒ सꣳशि॑शाधि। 
मा॒तेवा᳚स्मा अदिते॒ शर्म॑ यच्छ॒ विश्वे॑ देवा॒ जर॑दष्टि॒र्यथाऽस॑त्। 
अग्न॒ आयूꣳ॑षि पवस॒ आ सु॒वोर्ज॒मिषं॑ च नः। 
आ॒रे बा॑धस्व दु॒च्छुना᳚म्। 
अग्ने॒ पव॑स्व॒ स्वपा॑ अ॒स्मे वर्चः॑ सु॒वीर्यम्᳚। 
दध॑द्र॒यिं मयि॒ पोषम्᳚॥६॥

अ॒ग्निर्\mbox{}ऋषिः॒ पव॑मानः॒ पाञ्च॑जन्यः पु॒रोहि॑तः। 
तमी॑महे महाग॒यम्। 
अग्ने॑ जा॒तान्प्रणु॑दा नः स॒पत्ना॒न्प्रत्यजा॑ताञ्जातवेदो नुदस्व। 
अ॒स्मे दी॑दिहि सु॒मना॒ अहे॑ळ॒ञ्छर्म॑न्ते स्याम त्रि॒वरू॑थ उ॒द्भौ। 
सह॑सा जा॒तान्प्रणु॑दा नः स॒पत्ना॒न्प्रत्यजा॑ताञ्जातवेदो नुदस्व। 
अधि॑ नो ब्रूहि सुमन॒स्यमा॑नो व॒यꣴ स्या॑म॒ प्रणु॑दा नः स॒पत्नान्। 
अग्ने॒ यो नो॒ऽभितो॒ जनो॒ वृको॒ वारो॒ जिघाꣳ॑सति। 
ताꣴस्त्वं वृ॑त्रहं जहि॒ वस्व॒स्मभ्य॒माभ॑र। 
अग्ने॒ यो नो॑ऽभि॒दास॑ति समा॒नो यश्च॒ निष्ट्यः॑। 
तं व॒यꣳ स॒मिधं॑ कृ॒त्वा तुभ्य॑म॒ग्नेऽपि॑ दध्मसि॥७॥

यो नः॒ शपा॒दश॑पतो॒ यश्च॑ नः॒ शप॑तः॒ शपा᳚त्। 
उ॒षाश्च॒ तस्मै॑ नि॒म्रुक्च॒ सर्वं॑ पा॒पꣳ समू॑हताम्। 
यो नः॑ स॒पत्नो॒ यो रणो॒ मर्तो॑ऽभि॒दास॑ति देवाः। 
इ॒ध्मस्ये॑व प्र॒क्षाय॑तो॒ मा तस्योच्छे॑षि॒ किं च॒न। 
यो मां द्वेष्टि॑ जातवेदो॒ यं चा॒हं द्वेष्मि॒ यश्च॒ माम्। 
सर्वा॒ꣴ॒स्तान॑ग्ने॒ सन्द॑ह॒ याꣴश्चा॒हं द्वेष्मि॒ ये च॒ माम्। 
यो अ॒स्मभ्य॑मराती॒याद्यश्च॑ नो॒ द्वेष॑ते॒ जनः॑। 
निन्दा॒द्यो अ॒स्मान्दिफ्सा᳚च्च॒ सर्वा॒ꣴ॒स्तान्म॑ष्म॒षा कु॑रु। 
सꣳशि॑तं मे॒ ब्रह्म॒ सꣳशि॑तं वी॒र्या(१\char"E009)म्बलम्᳚। 
सꣳशि॑तं क्ष॒त्रं मे॑ जि॒ष्णु यस्या॒हमस्मि॑ पु॒रोहि॑तः। 
उदे॑षां बा॒हू अ॑तिर॒मुद्वर्चो॒ अथो॒ बलम्᳚। 
क्षि॒णोमि॒ ब्रह्म॑णा॒ऽमित्रा॒नुन्न॑यामि॒ स्वा(१)म् अ॒हम्। 
पुन॒र्मनः॒ पुन॒रायु॑र्म॒ आगा॒त्पुन॒श्चक्षुः॒ पुनः॒ श्रोत्रं॑ म॒ आगा॒त्पुनः॑ प्रा॒णः पुन॒राकू॑तं म॒ आगा॒त्पुन॑श्चि॒त्तं पुन॒राधी॑तं म॒ आगा᳚त्। 
वै॒श्वा॒न॒रो मेऽद॑ब्धस्तनू॒पा अव॑बाधतां दुरि॒तानि॒ विश्वा᳚॥८॥\anuvakamend

