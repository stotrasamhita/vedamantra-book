% !TeX program = XeLaTeX
% !TeX root = ../BrahmanamBook-kindle.tex
%ꣳ॒ ꣳ॑ ꣳ
%ꣴ ꣴ॒ ꣴ॑
%Find and replace
% ॑ः -> ः॑ (used word to get the second samyuktakshara!)
%ः᳚ -> ः᳚
\chapt{अष्टकम् १}
\sect{प्रथमः प्रश्नः}
\setcounter{anuvakam}{0}
\dnsub{तैत्तिरीयब्राह्मणे प्रथमाष्टके प्रथमः प्रपाठकः}

%1.1.1.1
ब्रह्म॒ सन्ध॑त्तं॒ तन्मे॑ जिन्वतम्।
क्ष॒त्रꣳ सन्ध॑त्तं॒ तन्मे॑ जिन्वतम्।
इष॒ꣳ॒ सन्ध॑त्तं॒ तां मे॑ जिन्वतम्।
ऊर्ज॒ꣳ॒ सन्ध॑त्तं॒ तां मे॑ जिन्वतम्।
र॒यिꣳ सन्ध॑त्तं॒ तां मे॑ जिन्वतम्।
पुष्टि॒ꣳ॒ सन्ध॑त्तं॒ तां मे॑ जिन्वतम्।
प्र॒जाꣳ सन्ध॑त्तं॒ तां मे॑ जिन्वतम्।
प॒शून्थ्सन्ध॑त्तं॒ तान्मे॑ जिन्वतम्।
स्तु॒तो॑ऽसि॒ जन॑धाः।
दे॒वास्त्वा॑ शुक्र॒पाः प्रण॑यन्तु॥१॥

%1.1.1.2
सु॒वीराः᳚ प्र॒जाः प्र॑ज॒नय॒न्परी॑हि।
शु॒क्रः शु॒क्रशो॑चिषा।
स्तु॒तो॑ऽसि॒ जन॑धाः।
दे॒वास्त्वा॑ मन्थि॒पाः प्रण॑यन्तु।
सु॒प्र॒जाः प्र॒जाः प्र॑ज॒नय॒न्परी॑हि।
म॒न्थी म॒न्थिशो॑चिषा।
स॒ञ्ज॒ग्मा॒नौ दि॒व आपृ॑थि॒व्यायुः॑।
सन्ध॑त्तं॒ तन्मे॑ जिन्वतम्।
प्रा॒णꣳ सन्ध॑त्तं॒ तं मे॑ जिन्वतम्।
अ॒पा॒नꣳ सन्ध॑त्तं॒ तं मे॑ जिन्वतम्॥२॥

%1.1.1.3
व्या॒नꣳ सन्ध॑त्तं॒ तं मे॑ जिन्वतम्।
चक्षुः॒ सन्ध॑त्तं॒ तन्मे॑ जिन्वतम्।
श्रोत्र॒ꣳ॒ सन्ध॑त्तं॒ तन्मे॑ जिन्वतम्।
मनः॒ सन्ध॑त्तं॒ तन्मे॑ जिन्वतम्।
वाच॒ꣳ॒ सन्ध॑त्तं॒ तां मे॑ जिन्वतम्।
आयुः॑ स्थ॒ आयु॑र्मे धत्तम्।
आयु॑र्य॒ज्ञाय॑ धत्तम्।
आयु॑र्य॒ज्ञप॑तये धत्तम्।
प्रा॒णः स्थः॑ प्रा॒णं मे॑ धत्तम्।
प्रा॒णं य॒ज्ञाय॑ धत्तम्॥३॥

%1.1.1.4
प्रा॒णं य॒ज्ञप॑तये धत्तम्।
चक्षुः॑ स्थ॒श्चक्षु॑र्मे धत्तम्।
चक्षु॑र्य॒ज्ञाय॑ धत्तम्।
चक्षु॑र्य॒ज्ञप॑तये धत्तम्।
श्रोत्रꣴ॑ स्थः॒ श्रोत्रं॑ मे धत्तम्।
श्रोत्रं॑ य॒ज्ञाय॑ धत्तम्।
श्रोत्रं॑ य॒ज्ञप॑तये धत्तम्।
तौ दे॑वौ शुक्रामन्थिनौ।
क॒ल्पय॑तं॒ दैवी॒र्विशः॑।
क॒ल्पय॑तं॒ मानु॑षीः॥४॥

%1.1.1.5
इष॒मूर्ज॑म॒स्मासु॑ धत्तम्।
प्रा॒णान्प॒शुषु॑।
प्र॒जां मयि॑ च॒ यज॑माने च।
निर॑स्तः॒ शण्डः॑।
निर॑स्तो॒ मर्कः॑।
अप॑नुत्तौ॒ शण्डा॒मर्कौ॑ स॒हामुना᳚।
शु॒क्रस्य॑ स॒मिद॑सि।
म॒न्थिनः॑ स॒मिद॑सि।
स प्र॑थ॒मः सङ्कृ॑तिर्वि॒श्वक॑र्मा।
स प्र॑थ॒मो मि॒त्रो वरु॑णो अ॒ग्निः।
स प्र॑थ॒मो बृह॒स्पति॑श्चिकि॒त्वान्।
तस्मा॒ इन्द्रा॑य सु॒तमा जु॑होमि॥५॥\anuvakamend[न॒य॒न्त्व॒पा॒नꣳ सन्ध॑त्तं॒ तं मे॑ जिन्वतं प्रा॒णं य॒ज्ञाय॑ धत्तं॒ मानु॑षीर॒ग्निर्द्वे च॑॥ (ब्रह्म॑ क्ष॒त्रं तदिष॒मूर्जꣳ॑ र॒यिं पुष्टिं॑ प्र॒जां तां प॒शून्तान्थ्सन्ध॑त्तं॒ तत्प्रा॒णम॑पा॒नं व्या॒नं तं चक्षुः॒ श्रोत्रं॒ मन॒स्तद्वाचं॒ ताम्।
इ॒षादि॒पञ्च॑के॒ वाचं॒ तां मे॑ प॒शून्थ्सन्ध॑त्तं॒ तान्मे᳚ प्रा॒णादि॒त्रित॑ये॒ तं मे॒\-ऽन्यत्र॒ तन्मे᳚)]

%1.1.2.1
कृत्ति॑कास्व॒ग्निमाद॑धीत।
ए॒तद्वा अ॒ग्नेर्नक्ष॑त्रम्।
यत्कृत्ति॑काः।
स्वाया॑मे॒वैनं॑ दे॒वता॑यामा॒धाय॑।
ब्र॒ह्म॒व॒र्च॒सी भ॑वति।
मुखं॒ वा ए॒तन्नक्ष॑त्राणाम्।
यत्कृत्ति॑काः।
यः कृत्ति॑कास्व॒ग्निमा॑ध॒त्ते।
मुख्य॑ ए॒व भ॑वति।
अथो॒ खलु॑॥६॥

%1.1.2.2
अ॒ग्नि॒न॒क्ष॒त्रमित्यप॑चायन्ति।
गृ॒हान् ह॒ दाहु॑को भवति।
प्र॒जा\-प॑ती रोहि॒ण्याम॒ग्निम॑\-सृजत।
तं दे॒वा रो॑हि॒ण्यामाद॑धत।
ततो॒ वै ते सर्वा॒न्रोहा॑नरोहन्।
तद्रो॑हि॒ण्यै रो॑हिणि॒त्वम्।
यो रो॑हि॒ण्याम॒ग्निमा॑ध॒त्ते।
ऋ॒ध्नोत्ये॒व।
सर्वा॒न्रोहा᳚न्रोहति।
दे॒वा वै भ॒द्राः सन्तो॒\-ऽग्निमाधि॑थ्सन्त॥७॥

%1.1.2.3
तेषा॒मना॑हितो॒\-ऽग्निरासी᳚त्।
अथै᳚भ्यो वा॒मं वस्वपा᳚क्रामत्।
ते पुन॑र्वस्वो॒राद॑धत।
ततो॒ वै तान् वा॒मं वसू॒पाव॑र्तत।
यः पु॒राऽभ॒द्रः सन्पापी॑या॒न्थ्स्यात्।
स पुन॑र्वस्वोर॒ग्निमाद॑धीत।
पुन॑रे॒वैनं॑ वा॒मं वसू॒पाव॑र्तते।
भ॒द्रो भ॑वति।
यः का॒मये॑त॒ दानका॑मा मे प्र॒जाः स्यु॒रिति॑।
स पूर्व॑योः॒ फल्गु॑न्योर॒ग्निमाद॑धीत॥८॥

%1.1.2.4
अ॒र्य॒म्णो वा ए॒तन्नक्ष॑त्रम्।
यत्पूर्वे॒ फल्गु॑नी।
अ॒र्य॒मेति॒ तमा॑हु॒र्यो ददा॑ति।
दान॑कामा अस्मै प्र॒जा भ॑वन्ति।
यः का॒मये॑त भ॒गी स्या॒मिति॑।
स उत्त॑रयोः॒ फल्गु॑न्योर॒ग्निमाद॑धीत।
भग॑स्य॒ वा ए॒तन्नक्ष॑त्रम्।
यदुत्त॑रे॒ फल्गु॑नी।
भ॒ग्ये॑व भ॑वति।
का॒ल॒क॒ञ्जा वै नामासु॑रा आसन्॥९॥

%1.1.2.5
ते सु॑व॒र्गाय॑ लो॒काया॒ग्निम॑चिन्वत।
पुरु॑ष॒ इष्ट॑का॒मुपा॑दधा॒त्\-पुरु॑ष॒ इष्ट॑काम्।
स इन्द्रो᳚ ब्राह्म॒णो ब्रुवा॑ण॒ इष्ट॑का॒मुपा॑धत्त।
ए॒षा मे॑ चि॒त्रा नामेति॑।
ते सु॑व॒र्गं लो॒कमा प्रारो॑हन्।
स इन्द्र॒ इष्ट॑का॒मावृ॑हत्।
तेऽवा॑कीर्यन्त।
ये॑ऽवाकी᳚र्यन्त।
त ऊर्णा॒वभ॑यो\-ऽभवन्।
द्वावुद॑पतताम्॥१०॥

%1.1.2.6
तौ दि॒व्यौ श्वाना॑वभवताम्।
यो भ्रातृ॑व्यवा॒न्थ्स्यात्।
स चि॒त्राया॑म॒ग्निमाद॑धीत।
अ॒व॒कीर्यै॒व भ्रातृ॑व्यान्।
ओजो॒ बल॑मिन्द्रि॒यं वी॒र्य॑मा॒त्मन्ध॑त्ते।
व॒सन्ता᳚ ब्राह्म॒णो᳚\-ऽग्निमाद॑धीत।
व॒स॒न्तो वै ब्रा᳚ह्म॒णस्य॒र्तुः।
स्व ए॒वैन॑मृ॒तावा॒धाय॑।
ब्र॒ह्म॒व॒र्च॒सी भ॑वति।
मुखं॒ वा ए॒तदृ॑तू॒नाम्॥११॥

%1.1.2.7
यद्व॑स॒न्तः।
यो व॒सन्ता॒\-ऽग्निमा॑ध॒त्ते।
मुख्य॑ ए॒व भ॑वति।
अथो॒ योनि॑मन्तमे॒वैनं॒ प्रजा॑त॒माध॑त्ते।
ग्री॒ष्मे रा॑ज॒न्य॑ आद॑धीत।
ग्री॒ष्मो वै रा॑ज॒न्य॑स्य॒र्तुः।
स्व ए॒वैन॑मृ॒तावा॒धाय॑।
इ॒न्द्रि॒या॒वी भ॑वति।
श॒रदि॒ वैश्य॒ आद॑धीत।
श॒रद्वै वैश्य॑स्य॒र्तुः॥१२॥

%1.1.2.8
स्व ए॒वैन॑मृ॒तावा॒धाय॑।
प॒शु॒मान्भ॑वति।
न पूर्व॑योः॒ फल्गु॑न्योर॒ग्निमाद॑धीत।
ए॒षा वै ज॑घ॒न्या॑ रात्रिः॑ संवथ्स॒रस्य॑।
यत्पूर्वे॒ फल्गु॑नी।
पृ॒ष्टि॒त ए॒व सं॑वथ्स॒रस्या॒ग्निमा॒धाय॑।
पापी॑यान्भवति।
उत्त॑रयो॒रा द॑धीत।
ए॒षा वै प्र॑थ॒मा रात्रिः॑ संवथ्स॒रस्य॑।
यदुत्त॑रे॒ फल्गु॑नी।
मु॒ख॒त ए॒व सं॑वथ्स॒रस्या॒ग्निमा॒धाय॑।
वसी॑यान्भवति।
अथो॒ खलु॑।
य॒दैवैनं॑ य॒ज्ञ उ॑प॒नमे᳚त्।
अथाद॑धीत।
सैवास्यर्द्धिः॑॥१३॥\anuvakamend[खल्वा॑धिथ्सन्त॒ फल्गु॑न्योर॒ग्निमाद॑धीतासन्नपततामृतू॒नां वैश्य॑स्य॒र्तुरुत्त॑रे॒ फल्गु॑नी॒ षट्च॑]

%1.1.3.1
उद्ध॑न्ति।
यदे॒वास्या॑ अमे॒ध्यम्।
तदप॑हन्ति।
अ॒पोऽवो᳚क्षति॒ शान्त्यै᳚।
सिक॑ता॒ निव॑पति।
ए॒तद्वा अ॒ग्नेर्वै᳚श्वान॒रस्य॑ रू॒पम्।
रू॒पेणै॒व वै᳚श्वान॒रमव॑ रुन्धे।
ऊषां॒ निव॑पति।
पुष्टि॒र्वा ए॒षा प्र॒जन॑नम्।
यदूषाः᳚॥१४॥

%1.1.3.2
पुष्ट्या॑मे॒व प्र॒जन॑ने॒\-ऽग्निमाध॑त्ते।
अथो॑ सं॒ज्ञान॑ ए॒व।
सं॒ज्ञान॒ꣴ॒ ह्ये॑तत्प॑शू॒नाम्।
यदूषाः᳚।
द्यावा॑पृथि॒वी स॒हास्ता᳚म्।
ते वि॑य॒ती अ॑ब्रूताम्।
अस्त्वे॒व नौ॑ स॒ह य॒ज्ञिय॒मिति॑।
यद॒मुष्या॑ य॒ज्ञिय॒मासी᳚त्।
तद॒स्याम॑दधात्।
त ऊषा॑ अभवन्॥१५॥

%1.1.3.3
यद॒स्या य॒ज्ञिय॒मासी᳚त्।
तद॒मुष्या॑मदधात्।
तद॒दश्च॒न्द्रम॑सि कृ॒ष्णम्।
ऊषा᳚न्नि॒वप॑न्न॒दो ध्या॑येत्।
द्यावा॑पृथि॒व्योरे॒व य॒ज्ञिये॒\-ऽग्निमाध॑त्ते।
अ॒ग्निर्दे॒वेभ्यो॒ निला॑यत।
आ॒खू रू॒पं कृ॒त्वा।
स पृ॑थि॒वीं प्रावि॑शत्।
स ऊ॒तीः कु॑र्वा॒णः पृ॑थि॒वीमनु॒ सम॑चरत्।
तदा॑खुकरी॒षम॑भवत्॥१६॥

%1.1.3.4
यदा॑खुकरी॒षꣳ स॑म्भा॒रो भव॑ति।
यदे॒वास्य॒ तत्र॒ न्य॑क्तम्।
तदे॒वाव॑ रुन्धे।
ऊर्जं॒ वा ए॒तꣳ रसं॑ पृथि॒व्या उ॑प॒दीका॒ उद्दि॑हन्ति।
यद्व॒ल्मीकम्᳚।
यद्व॑ल्मीकव॒पा स॑म्भा॒रो भव॑ति।
ऊर्ज॑मे॒व रसं॑ पृथि॒व्या अव॑ रुन्धे।
अथो॒ श्रोत्र॑मे॒व।
श्रोत्र॒ꣴ॒ ह्ये॑तत्पृ॑थि॒व्याः।
यद्व॒ल्मीकः॑॥१७॥

%1.1.3.5
अब॑धिरो भवति।
य ए॒वं वेद॑।
प्र॒जा\-प॑तिः प्र॒जा अ॑\-सृजत।
तासा॒मन्न॒मुपा᳚क्षीयत।
ताभ्यः॒ सूद॒मुप॒प्राभि॑नत्।
ततो॒ वै तासा॒मन्नं॒ नाक्षी॑यत।
यस्य॒ सूदः॑ सम्भा॒रो भव॑ति।
नास्य॑ गृ॒हेऽन्नं॑ क्षीयते।
आपो॒ वा इ॒दमग्रे॑ सलि॒लमा॑सीत्।
तेन॑ प्र॒जा\-प॑तिरश्राम्यत्॥१८॥

%1.1.3.6
क॒थमि॒दꣴ स्या॒दिति॑।
सो॑ऽपश्यत्पुष्करप॒र्णं तिष्ठ॑त्।
सो॑ऽमन्यत।
अस्ति॒ वै तत्।
यस्मि॑न्नि॒दमधि॒ तिष्ठ॒तीति॑।
स व॑रा॒हो रू॒पं कृ॒त्वोप॒ न्य॑मज्जत्।
स पृ॑थि॒वीम॒ध आ᳚र्च्छत्।
तस्या॑ उप॒हत्योद॑मज्जत्।
तत्पु॑ष्करप॒र्णे᳚\-ऽप्रथयत्।
यदप्र॑थयत्॥१९॥

%1.1.3.7
तत्पृ॑थि॒व्यै पृ॑थिवि॒त्वम्।
अभू॒द्वा इ॒दमिति॑।
तद्भूम्यै॑ भूमि॒त्वम्।
तां दिशोऽनु॒ वातः॒ सम॑वहत्।
ताꣳ शर्क॑राभिरदृꣳहत्।
शं वै नो॑ऽभू॒दिति॑।
तच्छर्क॑राणाꣳ शर्कर॒त्वम्।
यद्व॑रा॒हवि॑हतꣳ सम्भा॒रो भव॑ति।
अ॒स्यामे॒वा\-छ॑म्बट्कारम॒ग्निमाध॑त्ते।
शर्क॑रा भवन्ति॒ धृत्यै᳚॥२०॥

%1.1.3.8
अथो॑ श॒न्त्वाय॑।
सरे॑ता अ॒ग्निरा॒धेय॒ इत्या॑हुः।
आपो॒ वरु॑णस्य॒ पत्न॑य आसन्।
ता अ॒ग्निर॒भ्य॑ध्यायत्।
ताः सम॑भवत्।
तस्य॒ रेतः॒ परा॑ऽपतत्।
तद्धिर॑ण्यमभवत्।
यद्धिर॑ण्यमु॒पास्य॑ति।
सरे॑तसमे॒वाग्निमाध॑त्ते।
पुरु॑ष॒ इन्न्वै स्वाद्रेत॑सो बीभथ्सत॒ इत्या॑हुः॥२१॥

%1.1.3.9
उ॒त्त॒र॒त उपा᳚स्य॒त्यबी॑भथ्सायै।
अति॒ प्रय॑च्छति।
आर्ति॑मे॒वाति॒ प्रय॑च्छति।
अ॒ग्निर्दे॒वेभ्यो॒ निला॑यत।
अश्वो॑ रू॒पं कृ॒त्वा।
सो᳚ऽश्व॒त्थे सं॑वथ्स॒रम॑तिष्ठत्।
तद॑श्व॒त्थस्या᳚श्वत्थ॒त्वम्।
यदाश्व॑त्थः सम्भा॒रो भव॑ति।
यदे॒वास्य॒ तत्र॒ न्य॑क्तम्।
तदे॒वाव॑ रुन्धे॥२२॥

%1.1.3.10
दे॒वा वा ऊर्जं॒ व्य॑भजन्त।
तत॑ उदु॒म्बर॒ उद॑तिष्ठत्।
ऊर्ग्वा उ॑दु॒म्बरः॑।
यदौदु॑म्बरः सम्भा॒रो भव॑ति।
ऊर्ज॑मे॒वाव॑ रुन्धे।
तृ॒तीय॑स्यामि॒तो दि॒वि सोम॑ आसीत्।
तं गा॑य॒त्र्या\-ऽह॑रत्।
तस्य॑ प॒र्णम॑च्छिद्यत।
तत्प॒र्णो॑\-ऽभवत्।
तत्प॒र्णस्य॑ पर्ण॒त्वम्॥२३॥

%1.1.3.11
यस्य॑ पर्ण॒मयः॑ सम्भा॒रो भव॑ति।
सो॒म॒पी॒थमे॒वाव॑ रुन्धे।
दे॒वा वै ब्रह्म॑न्नवदन्त।
तत्प॒र्ण उपा॑शृणोत्।
सु॒श्रवा॒ वै नाम॑।
यत्प॑र्ण॒मयः॑ सम्भा॒रो भव॑ति।
ब्र॒ह्म॒व॒र्च॒समे॒वाव॑ रुन्धे।
प्र॒जा\-प॑तिर॒ग्निम॑\-सृजत।
सो॑ऽबिभे॒त्प्र मा॑ धक्ष्य॒तीति॑।
तꣳ श॒म्या॑\-ऽशमयत्॥२४॥

%1.1.3.12
तच्छ॒म्यै॑ शमि॒त्वम्।
यच्छ॑मी॒मयः॑ सम्भा॒रो भव॑ति।
शान्त्या॒ अप्र॑दाहाय।
अ॒ग्नेः सृ॒ष्टस्य॑ य॒तः।
विक॑ङ्कतं॒ भा आ᳚र्च्छत्।
यद्वैक॑ङ्कतः सम्भा॒रो भव॑ति।
भा ए॒वाव॑ रुन्धे।
सहृ॑दयो॒\-ऽग्निरा॒धेय॒ इत्या॑हुः।
म॒रुतो॒\-ऽद्भिर॒ग्निम॑तमयन्।
तस्य॑ ता॒न्तस्य॒ हृद॑य॒माच्छि॑न्दन्।
साऽशनि॑रभवत्।
यद॒शनि॑हतस्य वृ॒क्षस्य॑ सम्भा॒रो भव॑ति।
सहृ॑दयमे॒वाग्निमा ध॑त्ते॥२५॥\anuvakamend[ऊषा॑ अभवन्नभवद्व॒ल्मीको᳚\-ऽश्राम्य॒दप्र॑थय॒द्धृत्यै॑ बीभथ्सत॒ इत्या॑हू रुन्धे पर्ण॒त्वम॑शमयदच्छिन्द॒ꣴ॒स्त्रीणि॑ च]

%1.1.4.1
द्वा॒द॒शसु॑ विक्रा॒मेष्व॒ग्निमा द॑धीत।
द्वाद॑श॒ मासाः᳚ संवथ्स॒रः।
सं॒व॒थ्स॒रादे॒वैन॑मव॒रुद्ध्या ध॑त्ते।
यद्द्वा॑द॒शसु॑ विक्रा॒मेष्वा॒ दधी॑त।
परि॑मित॒मव॑ रुन्धीत।
चक्षु॑र्निमित॒ आद॑धीत।
इय॒द्द्वाद॑श विक्रा॒मा(३) इति॑।
परि॑मितं चै॒वाप॑रिमितं॒ चाव॑ रुन्धे।
अनृ॑तं॒ वै वा॒चा व॑दति।
अनृ॑तं॒ मन॑सा ध्यायति॥२६॥

%1.1.4.2
चक्षु॒र्वै स॒त्यम्।
अद्रा(३)गित्या॑ह।
अद॑र्\mbox{}श॒मिति॑।
तथ्स॒त्यम्।
यश्चक्षु॑र्निमिते॒\-ऽग्निमा॑ध॒त्ते।
स॒त्य ए॒वैन॒मा ध॑त्ते।
तस्मा॒दाहि॑ताग्नि॒र्नानृ॑तं वदेत्।
नास्य॑ ब्राह्म॒णो\-ऽना᳚श्वान्गृ॒हे व॑सेत्।
स॒त्ये ह्य॑स्या॒ग्निराहि॑तः।
आ॒ग्ने॒यी वै रात्रिः॑॥२७॥

%1.1.4.3
आ॒ग्ने॒याः प॒शवः॑।
ऐ॒न्द्रमहः॑।
नक्तं॒ गार्\mbox{}ह॑पत्य॒मा द॑धाति।
प॒शूने॒वाव॑ रुन्धे।
दिवा॑ऽऽहव॒नीयम्᳚।
इ॒न्द्रि॒यमे॒वाव॑ रुन्धे।
अ॒र्धोदि॑ते॒ सूर्य॑ आहव॒नीय॒मा द॑धाति।
ए॒तस्मि॒न्वै लो॒के प्र॒जा\-प॑तिः प्र॒जा अ॑\-सृजत।
प्र॒जा ए॒व तद्यज॑मानः \-सृजते।
अथो॑ भू॒तं चै॒व भ॑वि॒ष्यच्चाव॑ रुन्धे॥२८॥

%1.1.4.4
इडा॒ वै मा॑न॒वी य॑ज्ञानूका॒शिन्या॑सीत्।
साऽशृ॑णोत्।
असु॑रा अ॒ग्निमाद॑धत॒ इति॑।
तद॑गच्छत्।
त आ॑हव॒नीय॒मग्र॒ आद॑धत।
अथ॒ गार्\mbox{}ह॑पत्यम्।
अथा᳚न्वाहार्य॒पच॑नम्।
साऽब्र॑वीत्।
प्र॒तीच्ये॑षा॒ꣴ॒ श्रीर॑गात्।
भ॒द्रा भू॒त्वा परा॑ भविष्य॒न्तीति॑॥२९॥

%1.1.4.5
यस्यै॒वम॒ग्निरा॑धी॒यते᳚।
प्र॒तीच्य॑स्य॒ श्रीरे॑ति।
भ॒द्रो भू॒त्वा परा॑भवति।
साऽशृ॑णोत्।
दे॒वा अ॒ग्निमाद॑धत॒ इति॑।
तद॑गच्छत्।
ते᳚ऽन्वाहार्य॒पच॑न॒मग्र॒ आद॑धत।
अथ॒ गार्\mbox{}ह॑पत्यम्।
अथा॑ऽऽहव॒नी\-यम्᳚।
साऽब्र॑वीत्॥३०॥

%1.1.4.6
प्राच्ये॑षा॒ꣴ॒ श्रीर॑गात्।
भ॒द्रा भू॒त्वा सु॑व॒र्गं लो॒कमे᳚ष्यन्ति।
प्र॒जां तु न वे᳚थ्स्यन्त॒ इति॑।
यस्यै॒वम॒ग्निरा॑धी॒यते᳚।
प्राच्य॑स्य॒ श्रीरे॑ति।
भ॒द्रो भू॒त्वा सु॑व॒र्गं लो॒कमे॑ति।
प्र॒जां तु न वि॑न्दते।
साऽब्र॑वी॒दिडा॒ मनुम्᳚।
तथा॒ वा अ॒हं तवा॒ग्निमाधा᳚स्यामि।
यथा॒ प्र प्र॒जया॑ प॒शुभि॑र्मिथु॒नैर्ज॑नि॒ष्यसे᳚॥३१॥

%1.1.4.7
प्रत्य॒स्मिँल्लो॒के स्था॒स्यसि॑।
अ॒भि सु॑व॒र्गं लो॒कं जे॒ष्यसीति॑।
गार्\mbox{}ह॑पत्य॒मग्र॒ आद॑धात्।
गार्\mbox{}ह॑पत्यं॒ वा अनु॑ प्र॒जाः प॒शवः॒ प्रजा॑यन्ते।
गार्\mbox{}ह॑पत्येनै॒वास्मै᳚ प्र॒जां प॒शून्प्राज॑नयत्।
अथा᳚न्वाहार्य॒\-पच॑नम्।
ति॒र्यङ्ङि॑व॒ वा अ॒यं लो॒कः।
अ॒स्मिन्ने॒व तेन॑ लो॒के प्रत्य॑तिष्ठत्।
अथा॑ऽऽहव॒नीयम्᳚।
तेनै॒व सु॑व॒र्गं लो॒कम॒भ्य॑जयत्॥३२॥

%1.1.4.8
यस्यै॒वम॒ग्निरा॑धी॒यते᳚।
प्र प्र॒जया॑ प॒शुभि॑र्मिथु॒नैर्जा॑यते।
प्रत्य॒स्मिँल्लो॒के ति॑ष्ठति।
अ॒भि सु॑व॒र्गं लो॒कं ज॑यति।
यस्य॒ वा अय॑थादेवतम॒ग्निरा॑धी॒यते᳚।
आ दे॒वता᳚भ्यो वृश्च्यते।
पापी॑यान्भवति।
यस्य॑ यथादेव॒तम्।
न दे॒वता᳚भ्य॒ आवृ॑श्च्यते।
वसी॑यान्भवति॥३३॥

भृगू॑णां॒ त्वा\-ऽङ्गि॑रसां व्रतपते व्र॒तेनाद॑धा॒मीति॑ भृग्वङ्गि॒रसा॒\-माद॑ध्यात्।
आ॒दि॒त्यानां᳚ त्वा दे॒वानां᳚ व्रतपते व्र॒तेनाद॑धा॒मी\-त्य॒न्यासां॒ ब्राह्म॑णीनां प्र॒जाना᳚म्।
वरु॑णस्य त्वा॒ राज्ञो᳚ व्रतपते व्र॒तेनाद॑धा॒मीति॒ राज्ञः॑।
इन्द्र॑स्य त्वेन्द्रि॒येण॑ व्रतपते व्र॒तेनाद॑धा॒मीति॑ राज॒न्य॑स्य।
मनो᳚स्त्वा ग्राम॒ण्यो᳚ व्रतपते व्र॒तेनाद॑धा॒मीति॒ वैश्य॑स्य।
ऋ॒भू॒णां त्वा॑ दे॒वानां᳚ व्रतपते व्र॒तेनाद॑धा॒मीति॑ रथका॒रस्य॑।
य॒था॒दे॒व॒तम॒ग्निराधी॑यते।
न दे॒वता᳚भ्य॒ आवृ॑श्च्यते।
वसी॑यान्भवति॥३४॥\anuvakamend[ध्या॒य॒ति॒ वै रात्रि॒श्चाव॑ रुन्धे भविष्य॒न्तीत्य॑ब्रवीज्जनि॒ष्यसे॑\-ऽजय॒द्वसी॑यान्भवति॒ नव॑ च]

%1.1.5.1
प्र॒जा\-प॑तिर्वा॒चः स॒त्यम॑पश्यत्।
तेना॒ग्निमाध॑त्त।
तेन॒ वै स आ᳚र्ध्नोत्।
भूर्भुवः॒ सुव॒रित्या॑ह।
ए॒तद्वै वा॒चः स॒त्यम्।
य ए॒तेना॒ग्निमा॑ध॒त्ते।
ऋ॒ध्नोत्ये॒व।
अथो॑ स॒त्यप्रा॑शूरे॒व भ॑वति।
अथो॒ य ए॒वं वि॒द्वान॑भि॒चर॑ति।
स्तृ॒णु॒त ए॒वैनम्᳚॥३५॥

%1.1.5.2
भूरित्या॑ह।
प्र॒जा ए॒व तद्यज॑मानः \-सृजते।
भुव॒ इत्या॑ह।
अ॒स्मिन्ने॒व लो॒के प्रति॑ तिष्ठति।
सुव॒रित्या॑ह।
सु॒व॒र्ग ए॒व लो॒के प्रति॑ तिष्ठति।
त्रि॒भिर॒क्षरै॒र्गार्\mbox{}ह॑पत्य॒मा द॑धाति।
त्रय॑ इ॒मे लो॒काः।
ए॒ष्वे॑वैनं॑ लो॒केषु॒ प्रति॑\-ष्ठित॒माध॑त्ते।
सर्वैः᳚ प॒ञ्चभि॑राहव॒नीयम्᳚॥३६॥

%1.1.5.3
सु॒व॒र्गाय॒ वा ए॒ष लो॒कायाधी॑यते।
यदा॑हव॒नीयः॑।
सु॒व॒र्ग ए॒वास्मै॑ लो॒के वा॒चः स॒त्यꣳ सर्व॑माप्नोति।
त्रि॒भिर्गार्\mbox{}ह॑पत्य॒मा द॑धाति।
प॒ञ्चभि॑राहव॒नीयम्᳚।
अ॒ष्टौ सम्प॑द्यन्ते।
अ॒ष्टाक्ष॑रा गाय॒त्री।
गा॒य॒त्रो᳚\-ऽग्निः।
यावा॑ने॒वाग्निः।
तमाध॑त्ते॥३७॥

%1.1.5.4
प्र॒जा\-प॑तिः प्र॒जा अ॑\-सृजत।
ता अ॑स्माथ्सृ॒ष्टाः परा॑चीरायन्।
ताभ्यो॒ ज्योति॒रुद॑गृह्णात्।
तं ज्योतिः॒ पश्य॑न्तीः प्र॒जा अ॒भि स॒माव॑र्तन्त।
उ॒परी॑वा॒ग्निमुद्गृ॑ह्णीयादु॒द्धरन्॑।
ज्योति॑रे॒व पश्य॑न्तीः प्र॒जा यज॑मानम॒भि स॒माव॑र्तन्ते।
प्र॒जा\-प॑ते॒रक्ष्य॑श्वयत्।
तत्परा॑\-ऽपतत्।
तदश्वो॑\-ऽभवत्।
तदश्व॑स्याश्व॒त्वम्॥३८॥

%1.1.5.5
ए॒ष वै प्र॒जा\-प॑तिः।
यद॒ग्निः।
प्रा॒जा॒प॒त्यो\-ऽश्वः॑।
यदश्वं॑ पु॒रस्ता॒न्नय॑ति।
स्वमे॒व चक्षुः॒ पश्य॑न्प्र॒जा\-प॑ति॒रनूदे॑ति।
व॒ज्री वा ए॒षः।
यदश्वः॑।
यदश्वं॑ पु॒रस्ता॒न्नय॑ति।
जा॒ताने॒व भ्रातृ॑व्या॒न्प्रणु॑दते।
पुन॒रा व॑र्तयति॥३९॥

%1.1.5.6
ज॒नि॒ष्यमा॑णाने॒व प्रति॑\-नुदते।
न्या॑हव॒नीयो॒ गार्\mbox{}ह॑पत्य\-मकामयत।
निगार्\mbox{}ह॑पत्य आहव॒नीयम्᳚।
तौ वि॒भाजं॒ नाश॑क्नोत्।
सोऽश्वः॑ पूर्व॒वाड्भू॒त्वा।
प्राञ्चं॒ पूर्व॒मुद॑वहत्।
तत्पू᳚र्व॒वाहः॑ पूर्ववा॒ट्त्वम्।
यदश्वं॑ पु॒रस्ता॒न्नय॑ति।
विभ॑क्ति\-रे॒वैन॑योः॒ सा।
अथो॒ नाना॑वीर्यावे॒वैनौ॑ कुरुते॥४०॥

%1.1.5.7
यदु॒पर्यु॑परि॒ शिरो॒ हरे᳚त्।
प्रा॒णान्‌ वि\-च्छि॑न्द्यात्।
अ॒धो॑ऽधः॒ शिरो॑ हरति।
प्रा॒णानां᳚ गोपी॒थाय॑।
इय॒त्यग्रे॑ हरति।
अथेय॒त्यथेय॑ति।
त्रय॑ इ॒मे लो॒काः।
ए॒ष्वे॑वैनं॑ लो॒केषु॒ प्रति॑\-ष्ठित॒माध॑त्ते।
प्र॒जा\-प॑तिर॒ग्निम॑\-सृजत।
सो॑ऽबिभे॒त्प्र मा॑ धक्ष्य॒तीति॑॥४१॥

%1.1.5.8
तस्य॑ त्रे॒धा म॑हि॒मानं॒ व्यौ॑हत्।
शान्त्या॒ अप्र॑दाहाय।
यत्त्रे॒धा\-ऽग्निरा॑धी॒यते᳚।
म॒हि॒मान॑मे॒वास्य॒ तद्व्यू॑हति।
शान्त्या॒ अप्र॑दाहाय।
पुन॒रा व॑र्तयति।
म॒हि॒मान॑मे॒वास्य॒ सन्द॑धाति।
प॒शुर्वा ए॒षः।
यदश्वः॑।
ए॒ष रु॒द्रः॥४२॥

%1.1.5.9
यद॒ग्निः।
यदश्व॑स्य प॒दे᳚\-ऽग्निमा॑द॒ध्यात्।
रु॒द्राय॑ प॒शूनपि॑\-दध्यात्।
अ॒प॒शुर्यज॑मानः स्यात्।
यन्नाक्र॒मये᳚त्।
अन॑वरुद्धा अस्य प॒शवः॑ स्युः।
पा॒र्श्व॒त आक्र॑मयेत्।
यथाऽऽहि॑तस्या॒ग्नेरङ्गा॑रा अभ्यव॒वर्ते॑रन्।
अव॑रुद्धा अस्य प॒शवो॒ भव॑न्ति।
न रु॒द्रायापि॑दधाति॥४३॥

%1.1.5.10
त्रीणि॑ ह॒वीꣳषि॒ निर्व॑पति।
वि॒राज॑ ए॒व विक्रा᳚न्तं॒ यज॑मा॒नोऽनु॒ विक्र॑मते।
अ॒ग्नये॒ पव॑मानाय।
अ॒ग्नये॑ पाव॒काय॑।
अ॒ग्नये॒ शुच॑ये।
यद॒ग्नये॒ पव॑मानाय नि॒र्वप॑ति।
पु॒नात्ये॒वैनम्᳚।
यद॒ग्नये॑ पाव॒काय॑।
पू॒त ए॒वास्मि॑न्न॒न्नाद्यं॑ दधाति।
यद॒ग्नये॒ शुच॑ये।
ब्र॒ह्म॒व॒र्च॒समे॒वास्मि॑न्नु॒परि॑ष्टाद्दधाति॥४४॥\anuvakamend[ए॒न॒मा॒ह॒व॒नीयं॑ धत्ते\-ऽश्व॒त्वं व॑र्तयति कुरुत॒ इति॑ रु॒द्रो द॑धाति॒ यद॒ग्नये॒ शुच॑य॒ एकं॑ च]

%1.1.6.1
दे॒वा॒सु॒राः संय॑त्ता आसन्।
ते दे॒वा वि॑ज॒यमु॑प॒यन्तः॑।
अ॒ग्नौ वा॒मं वसु॒ सं न्य॑दधत।
इ॒दमु॑ नो भविष्यति।
यदि॑ नो जे॒ष्यन्तीति॑।
तद॒ग्निर्नोथ्सह॑मशक्नोत्।
तत् त्रे॒धा विन्य॑दधात्।
प॒शुषु॒ तृती॑यम्।
अ॒फ्सु तृती॑यम्।
आ॒दि॒त्ये तृती॑यम्॥४५॥

%1.1.6.2
तद्दे॒वा वि॒जित्य॑।
पुन॒रवा॑रुरुथ्सन्त।
ते᳚ऽग्नये॒ पव॑मानाय पुरो॒डाश॑म॒ष्टा\-क॑पालं॒ निर॑वपन्।
प॒शवो॒ वा अ॒ग्निः पव॑मानः।
यदे॒व प॒शुष्वासी᳚त्।
तत्तेनावा॑रुन्धत।
ते᳚ऽग्नये॑ पाव॒काय॑।
आपो॒ वा अ॒ग्निः पा॑व॒कः।
यदे॒वाफ्स्वासी᳚त्।
तत्तेनावा॑रुन्धत॥४६॥

%1.1.6.3
ते᳚ऽग्नये॒ शुच॑ये।
अ॒सौ वा आ॑दि॒त्यो᳚\-ऽग्निः शुचिः॑।
यदे॒वाऽऽदि॒त्य आसी᳚त्।
तत्तेनावा॑रुन्धत।
ब्र॒ह्म॒वा॒दिनो॑ वदन्ति।
त॒नुवो॒ वावैता अ॑ग्न्या॒धेय॑स्य।
आ॒ग्ने॒यो वा अ॒ष्टा\-क॑पालो\-ऽग्न्या॒धेय॒मिति॑।
यत्तं नि॒र्वपे᳚त्।
नैतानि॑।
यथा॒ऽऽत्मा स्यात्॥४७॥

%1.1.6.4
नाङ्गा॑नि।
ता॒दृगे॒व तत्।
यदे॒तानि॑ नि॒र्वपे᳚त्।
न तम्।
यथाऽङ्गा॑नि॒ स्युः।
नाऽऽत्मा।
ता॒दृगे॒व तत्।
उ॒भया॑नि स॒ह नि॒रुप्या॑णि।
य॒ज्ञस्य॑ सात्म॒त्वाय॑।
उ॒भयं॒ वा ए॒तस्ये᳚न्द्रि॒यं वी॒र्य॑माप्यते॥४८॥

%1.1.6.5
यो᳚ऽग्निमा॑ध॒त्ते।
ऐ॒न्द्रा॒ग्नमेका॑\-दश\-कपाल॒मनु॒ निर्व॑पेत्।
आ॒दि॒त्यं च॒रुम्।
इ॒न्द्रा॒ग्नी वै दे॒वाना॒मया॑तयामानौ।
ये ए॒व दे॒वते॒ अया॑तयाम्नी।
ताभ्या॑मे॒वास्मा॑ इन्द्रि॒यं वी॒र्य॑मव॑ रुन्धे।
आ॒दि॒त्यो भ॑वति।
इ॒यं वा अदि॑तिः।
अ॒स्यामे॒व प्रति॑ तिष्ठति।
धे॒न्वै वा ए॒तद्रेतः॑॥४९॥

%1.1.6.6
यदाज्यम्᳚।
अ॒न॒डुह॑स्तण्डु॒लाः।
मि॒थु॒नमे॒वाव॑ रुन्धे।
घृ॒ते भ॑वति।
य॒ज्ञस्यालू᳚क्षान्तत्वाय।
च॒त्वार॑ आर्\mbox{}षे॒याः प्राश्ञ॑न्ति।
दि॒शामे॒व ज्योति॑षि जुहोति।
प॒शवो॒ वा ए॒तानि॑ ह॒वीꣳषि॑।
ए॒ष रु॒द्रः।
यद॒ग्निः॥५०॥

%1.1.6.7
यथ्स॒द्य ए॒तानि॑ ह॒वीꣳषि॑ नि॒र्वपे᳚त्।
रु॒द्राय॑ प॒शूनपि॑ दध्यात्।
अ॒प॒शुर्यज॑मानः स्यात्।
यन्नानु॑नि॒र्वपे᳚त्।
अन॑वरुद्धा अस्य प॒शवः॑ स्युः।
द्वा॒द॒शसु॒ रात्री॒ष्वनु॒ निर्व॑पेत्।
सं॒व॒थ्स॒रप्र॑तिमा॒ वै द्वाद॑श॒ रात्र॑यः।
सं॒व॒थ्स॒रेणै॒वास्मै॑ रु॒द्रꣳ श॑मयि॒त्वा।
प॒शूनव॑ रुन्धे।
यदेक॑मेकमे॒तानि॑ ह॒वीꣳषि॑ नि॒र्वपे᳚त्॥५१॥

%1.1.6.8
यथा॒ त्रीण्या॒वप॑नानि पू॒रये᳚त्।
ता॒दृक्तत्।
न प्र॒जन॑न॒\-मुच्छिꣳ॑षेत्।
एकं॑ नि॒रुप्य॑।
उत्त॑रे॒ सम॑स्येत्।
तृ॒तीय॑मे॒वास्मै॑ लो॒कमुच्छिꣳ॑षति प्र॒जन॑नाय।
तं प्र॒जया॑ प॒शुभि॒रनु॒\- प्रजा॑यते।
अथो॑ य॒ज्ञस्यै॒वैषा\-ऽभिक्रा᳚न्तिः।
र॒थ॒च॒क्रं प्रव॑र्तयति।
म॒नु॒ष्य॒र॒थेनै॒व दे॑वर॒थं प्र॒त्यव॑रोहति॥५२॥

%1.1.6.9
ब्र॒ह्म॒वा॒दिनो॑ वदन्ति।
हो॒त॒व्य॑मग्निहो॒त्राँ(३) न हो॑त॒व्या(३) मिति॑।
यद्यजु॑षा जुहु॒यात्।
अय॑था\-पूर्व॒माहु॑ती जुहुयात्।
यन्न जु॑हु॒यात्।
अ॒ग्निः परा॑ भवेत्।
तू॒ष्णीमे॒व हो॑त॒व्यम्᳚।
य॒था॒पू॒र्वमाहु॑ती जु॒होति॑।
नाग्निः परा॑भवति।
अ॒ग्नीधे॑ ददाति॥५३॥

%1.1.6.10
अ॒ग्निमु॑खाने॒वर्तून्प्री॑णाति।
उ॒प॒बर्\mbox{}ह॑णं ददाति।
रू॒पाणा॒मव॑\-रुद्ध्यै।
अश्वं॑ ब्र॒ह्मणे᳚।
इ॒न्द्रि॒यमे॒वाव॑ रुन्धे।
धे॒नुꣳ होत्रे᳚।
आ॒शिष॑ ए॒वाव॑ रुन्धे।
अ॒न॒ड्वाह॑मध्व॒र्यवे᳚।
वह्नि॒र्वा अ॑न॒ड्वान्।
वह्नि॑रध्व॒र्युः॥५४॥

%1.1.6.11
वह्नि॑नै॒व वह्नि॑ य॒ज्ञस्याव॑ रुन्धे।
मि॒थु॒नौ गावौ॑ ददाति।
मि॒थु॒नस्या\-व॑\-रुद्ध्यै।
वासो॑ ददाति।
स॒र्व॒दे॒व॒त्यं॑ वै वासः॑।
सर्वा॑ ए॒व दे॒वताः᳚ प्रीणाति।
आ द्वा॑द॒शभ्यो॑ ददाति।
द्वाद॑श॒ मासाः᳚ संवथ्स॒रः।
सं॒व॒थ्स॒र ए॒व प्रति॑ तिष्ठति।
काम॑मू॒र्ध्वं देयम्᳚।
अप॑रिमित॒स्या\-व॑\-रुद्ध्यै॥५५॥\anuvakamend[आ॒दि॒त्ये तृती॑यम॒फ्स्वासी॒त्तत्तेनावा॑रुन्धत॒ स्यादा᳚प्यते॒ रेतो॒\-ऽग्निरेक॑मेकमे॒तानि॑ ह॒वीꣳषि॑ नि॒र्वपे᳚त्प्र॒त्यव॑रोहति ददात्यध्व॒र्युर्देय॒मेकं॑ च]

%1.1.7.1
घ॒र्मः शिर॒स्तद॒यम॒ग्निः।
सम्प्रि॑यः प॒शुभि॑र्भुवत्।
छ॒र्दिस्तो॒काय॒ तन॑याय यच्छ।
वातः॑ प्रा॒णस्तद॒यम॒ग्निः।
सम्प्रि॑यः प॒शुभि॑र्भुवत्।
स्व॒दि॒तं तो॒काय॒ तन॑याय पि॒तुं प॑च।
प्राची॒मनु॑ प्र॒दिशं॒ प्रेहि॑ वि॒द्वान्।
अ॒ग्नेर॑ग्ने पु॒रो अ॑ग्निर्भवे॒ह।
विश्वा॒ आशा॒ दीद्या॑नो॒ विभा॑हि।
ऊर्जं॑ नो धेहि द्वि॒पदे॒ चतु॑ष्पदे॥५६॥

%1.1.7.2
अ॒र्कश्चक्षु॒स्तद॒सौ सूर्य॒स्तद॒यम॒ग्निः।
सम्प्रि॑यः प॒शुभि॑र्भुवत्।
यत्ते॑ शुक्र शु॒क्रं वर्चः॑ शु॒क्रा त॒नूः।
शु॒क्रं ज्योति॒रज॑स्रम्।
तेन॑ मे दीदिहि॒ तेन॒ त्वाऽऽद॑धे।
अ॒ग्निना᳚\-ऽग्ने॒ ब्रह्म॑णा।
आ॒न॒शे व्या॑नशे॒ सर्व॒मायु॒र्व्या॑नशे।
ये ते॑ अग्ने शि॒वे त॒नुवौ᳚।
वि॒राट्च॑ स्व॒राट्च॑।
ते मावि॑शतां॒ ते मा॑ जिन्वताम्॥५७॥

%1.1.7.3
ये ते॑ अग्ने शि॒वे त॒नुवौ᳚।
स॒म्राट्चा॑भि॒भूश्च॑।
ते मावि॑शतां॒ ते मा॑ जिन्वताम्।
ये ते॑ अग्ने शि॒वे त॒नुवौ᳚।
वि॒भूश्च॑ परि॒भूश्च॑।
ते मा वि॑शतां॒ ते मा॑ जिन्वताम्।
ये ते॑ अग्ने शि॒वे त॒नुवौ᳚।
प्र॒भ्वी च॒ प्रभू॑तिश्च।
ते मा वि॑शतां॒ ते मा॑ जिन्वताम्।
यास्ते॑ अग्ने शि॒वास्त॒नुवः॑।
ताभि॒स्त्वाऽऽद॑धे।
यास्ते॑ अग्ने घो॒रास्त॒नुवः॑।
ताभि॑र॒मुं ग॑च्छ॥५८॥\anuvakamend[चतु॑ष्पदे जिन्वतां त॒नुव॒स्त्रीणि॑ च]

%1.1.8.1
इ॒मे वा ए॒ते लो॒का अ॒ग्नयः॑।
ते यदव्या॑वृत्ता आधी॒येरन्॑।
शो॒चये॑यु॒र्यज॑मानम्।
घ॒र्मः शिर॒ इति॒ गार्\mbox{}ह॑पत्य॒मा द॑धाति।
वातः॑ प्रा॒ण इत्य॑न्वाहार्य॒पच॑नम्।
अ॒र्कश्चक्षु॒रित्या॑हव॒नीयम्᳚।
तेनै॒वै\-ना॒न्व्या\-व॑र्तयति।
तथा॒ न शो॑चयन्ति॒ यज॑मानम्।
र॒थ॒न्त॒रम॒भिगा॑यते॒ गार्\mbox{}ह॑पत्य आधी॒यमा॑ने।
राथ॑न्तरो॒ वा अ॒यं लो॒कः॥५९॥

%1.1.8.2
अ॒स्मिन्ने॒वैनं॑ लो॒के प्रति॑\-ष्ठित॒मा ध॑त्ते।
वा॒म॒दे॒व्यम॒भिगा॑यत उद्ध्रि॒यमा॑णे।
अ॒न्तरि॑क्षं॒ वै वा॑मदे॒व्यम्।
अ॒न्तरि॑क्ष ए॒वैनं॒ प्रति॑\-ष्ठित॒माध॑त्ते।
अथो॒ शान्ति॒र्वै वा॑मदे॒व्यम्।
शा॒न्तमे॒वैनं॑ पश॒व्य॑मुद्ध॑रते।
बृ॒हद॒भिगा॑यत आहव॒नीय॑ आधी॒यमा॑ने।
बार्\mbox{}ह॑तो॒ वा अ॒सौ लो॒कः।
अ॒मुष्मि॑न्ने॒वैनं॑ लो॒के प्रति॑\-ष्ठित॒माध॑त्ते।
प्र॒जा\-प॑तिर॒ग्निम॑\-सृजत॥६०॥

%1.1.8.3
सोऽश्वो॒\-ऽवारो॑ भू॒त्वा परा॑ङैत्।
तं वा॑रव॒न्तीये॑नावारयत।
तद्वा॑रव॒न्तीय॑स्य वारवन्तीय॒त्वम्।
श्यै॒तेन॑ श्ये॒ती अ॑कुरुत।
तच्छ्यै॒तस्य॑ श्यैत॒त्वम्।
यद्वा॑रव॒न्तीय॑मभि॒ गाय॑ते।
वा॒र॒यि॒त्वैवैनं॒ प्रति॑\-ष्ठित॒मा ध॑त्ते।
श्यै॒तेन॑ श्ये॒ती कु॑रुते।
घ॒र्मः शिर॒ इति॒ गार्\mbox{}ह॑पत्य॒माद॑धाति।
सशी॑र्\mbox{}षाणमे॒वैन॒मा ध॑त्ते॥६१॥

%1.1.8.4
उपै॑न॒मुत्त॑रो य॒ज्ञो न॑मति।
रु॒द्रो वा ए॒षः।
यद॒ग्निः।
स आ॑धी॒यमा॑न ईश्व॒रो यज॑मानस्य प॒शून् हिꣳसि॑तोः।
सम्प्रि॑यः प॒शुभि॑र्भुव॒दित्या॑ह।
प॒शुभि॑रे॒वैन॒ꣳ॒ सम्प्रि॑यं करोति।
प॒शू॒नामहिꣳ॑सायै।
छ॒र्दिस्तो॒काय॒ तन॑याय य॒च्छेत्या॑ह।
आ॒\-मे॒वैतामा शा᳚स्ते।
वातः॑ प्रा॒ण इत्य॑न्वाहार्य॒पच॑नम्॥६२॥

%1.1.8.5
सप्रा॑णमे॒वैन॒मा ध॑त्ते।
स्व॒दि॒तं तो॒काय॒ तन॑याय पि॒तुं प॒चेत्या॑ह।
अन्न॑मे॒वास्मै᳚ स्वदयति।
प्राची॒मनु॑ प्र॒दिशं॒ प्रेहि॑ वि॒द्वानित्या॑ह।
विभ॑क्तिरे॒वैन॑योः॒ सा।
अथो॒ नाना॑वीर्यावे॒वैनौ॑ कुरुते।
ऊर्जं॑ नो धेहि द्वि॒पदे॒ चतु॑ष्पद॒ इत्या॑ह।
आ॒\-मे॒वैतामा शा᳚स्ते।
अ॒र्कश्चक्षु॒रित्या॑हव॒नीयम्᳚।
अ॒र्को वै दे॒वाना॒मन्नम्᳚॥६३॥

%1.1.8.6
अन्न॑मे॒वाव॑ रुन्धे।
तेन॑ मे दीदि॒हीत्या॑ह।
समि॑न्ध ए॒वैनम्᳚।
आ॒न॒शे व्या॑नश॒ इति॒ त्रिरुदि॑ङ्गयति।
त्रय॑ इ॒मे लो॒काः।
ए॒ष्वे॑वैनं॑ लो॒केषु॒ प्रति॑\-ष्ठित॒मा ध॑त्ते।
तत्तथा॒ न का॒र्यम्᳚।
वीङ्गि॑त॒मप्र॑तिष्ठित॒मा द॑धीत।
उ॒द्धृत्यै॒वाधाया॑भि॒मन्त्रियः॑।
अवी᳚ङ्गितमे॒वैनं॒ प्रति॑\-ष्ठित॒माध॑त्ते।
वि॒राट्च॑ स्व॒राट्च॒ यास्ते॑ अग्ने शि॒वास्त॒नुव॒स्ताभि॒स्त्वा\-ऽऽद॑ध॒ इत्या॑ह।
ए॒ता वा अ॒ग्नेः शि॒वास्त॒नुवः॑।
ताभि॑रे॒वैन॒ꣳ॒ सम॑र्धयति।
यास्ते॑ अग्ने घो॒रास्त॒नुव॒स्ताभि॑र॒मुं ग॒च्छेति॑ ब्रूया॒द्यं द्वि॒ष्यात्।
ताभि॑रे॒वैनं॒ परा॑भावयति॥६४॥\anuvakamend[लो॒को॑\-ऽ\-सृजतैन॒माध॑त्ते\-ऽन्वाहार्य॒पच॑नं दे॒वाना॒मन्न॑मेनं॒ प्रति॑\-ष्ठित॒माध॑त्ते॒ पञ्च॑ च]

%1.1.9.1
श॒मी॒ग॒र्भाद॒ग्निं म॑न्थति।
ए॒षा वा अ॒ग्नेर्य॒ज्ञिया॑ त॒नूः।
तामे॒वास्मै॑ जनयति।
अदि॑तिः पु॒त्रका॑मा।
सा॒ध्येभ्यो॑ दे॒वेभ्यो᳚ ब्रह्मौद॒नम॑पचत्।
तस्या॑ उ॒च्छेष॑णमददुः।
तत्प्राश्ञा᳚त्।
सा रेतो॑\-ऽधत्त।
तस्यै॑ धा॒ता चा᳚र्य॒मा चा॑जाये\-ताम्।
सा द्वि॒तीय॑मपचत्॥६५॥

%1.1.9.2
तस्या॑ उ॒च्छेष॑णमददुः।
तत्प्राश्ञा᳚त्।
सा रेतो॑\-ऽधत्त।
तस्यै॑ मि॒त्रश्च॒ वरु॑णश्चाजाये\-ताम्।
सा तृ॒तीय॑मपचत्।
तस्या॑ उ॒च्छेष॑णमददुः।
तत्प्राश्ञा᳚त्।
सा रेतो॑\-ऽधत्त।
तस्या॒ अꣳश॑श्च॒ भग॑श्चाजाये\-ताम्।
सा च॑तु॒र्थम॑पचत्॥६६॥

%1.1.9.3
तस्या॑ उ॒च्छेष॑णमददुः।
तत्प्राश्ञा᳚त्।
सा रेतो॑\-ऽधत्त।
तस्या॒ इन्द्र॑श्च॒ विव॑स्वाꣴश्चाजाये\-ताम्।
ब्र॒ह्मौ॒द॒नं प॑चति।
रेत॑ ए॒व तद्द॑धाति।
प्राश्ञ॑न्ति ब्राह्म॒णा ओ॑द॒नम्।
यदाज्य॑मु॒च्छिष्य॑ते।
तेन॑ स॒मिधो॒\-ऽभ्यज्या द॑धाति।
उ॒च्छेष॑णा॒द्वा अदि॑ती॒ रेतो॑\-ऽधत्त॥६७॥

%1.1.9.4
उ॒च्छेष॑णादे॒व तद्रेतो॑ धत्ते।
अस्थि॒ वा ए॒तत्।
यथ्स॒मिधः॑।
ए॒तद्रेतः॑।
यदाज्यम्᳚।
यदाज्ये॑न स॒मिधो॒\-ऽभ्यज्या॒दधा॑ति।
अस्थ्ये॒व तद्रेत॑सि दधाति।
ति॒स्र आद॑धाति मिथुन॒त्वाय॑।
इय॑तीर्भवन्ति।
प्र॒जा\-प॑तिना यज्ञमु॒खेन॒ सम्मि॑ताः॥६८॥

%1.1.9.5
इय॑तीर्भवन्ति।
य॒ज्ञ॒प॒रुषा॒ सम्मि॑ताः।
इय॑तीर्भवन्ति।
ए॒ताव॒द्वै पुरु॑षे वी॒र्यम्᳚।
वी॒र्य॑सम्मिताः।
आ॒र्द्रा भ॑वन्ति।
आ॒र्द्रमि॑व॒ हि रेतः॑ सि॒च्यते᳚।
चित्रि॑यस्याश्व॒त्थस्याद॑धाति।
चि॒त्रमे॒व भ॑वति।
घृ॒तव॑तीभि॒रा द॑धाति॥६९॥

%1.1.9.6
ए॒तद्वा अ॒ग्नेः प्रि॒यं धाम॑।
यद्\mbox{}घृ॒तम्।
प्रि॒येणै॒वैनं॒ धाम्ना॒ सम॑र्ध\-यति।
अथो॒ तेज॑सा।
गा॒य॒त्रीभि॑र्ब्राह्म॒णस्याद॑ध्यात्।
गा॒य॒त्र\-छ॑न्दा॒ वै ब्रा᳚ह्म॒णः।
स्वस्य॒ छन्द॑सः प्रत्ययन॒स्त्वाय॑।
त्रि॒ष्टुग्भी॑ राज॒न्य॑स्य।
त्रि॒ष्टुप्छ॑न्दा॒ वै रा॑ज॒न्यः॑।
स्वस्य॒ छन्द॑सः प्रत्ययन॒स्त्वाय॑॥७०॥

%1.1.9.7
जग॑तीभि॒र्वैश्य॑स्य।
जग॑तीछन्दा॒ वै वैश्यः॑।
स्वस्य॒ छन्द॑सः प्रत्ययन॒स्त्वाय॑।
तꣳ सं॑वथ्स॒रं गो॑पायेत्।
सं॒व॒थ्स॒रꣳ हि रेतो॑ हि॒तं वर्ध॑ते।
यद्ये॑नꣳ संवथ्स॒रे नोप॒नमे᳚त्।
स॒मिधः॒ पुन॒राद॑ध्यात्।
रेत॑ ए॒व तद्धि॒तं वर्ध॑मानमेति।
न मा॒ꣳ॒सम॑श्ञीयात्।
न स्त्रिय॒मुपे॑यात्॥७१॥

%1.1.9.8
यन्मा॒ꣳ॒सम॑श्ञी॒यात्।
यथ्स्त्रिय॑मुपे॒यात्।
निर्वी᳚र्यः स्यात्।
नैन॑म॒ग्निरुप॑नमेत्।
श्व आ॑धा॒स्यमा॑नो ब्रह्मौद॒नं प॑चति।
आ॒दि॒त्या वा इ॒त उ॑त्त॒माः सु॑व॒र्गं लो॒कमा॑यन्।
ते वा इ॒तो यन्तं॒ प्रति॑\-नुदन्ते।
ए॒ते खलु॒ वावा\-ऽऽदि॒त्याः।
यद्ब्रा᳚ह्म॒णाः।
तैरे॒व स॒न्त्वं ग॑च्छति॥७२॥

%1.1.9.9
नैनं॒ प्रति॑\-नुदन्ते।
ब्र॒ह्म॒वा॒दिनो॑ वदन्ति।
क्वा॑ सः।
अ॒ग्निः का॒र्यः॑।
यो᳚ऽस्मै प्र॒जां प॒शून्प्र॑ज॒नय॒तीति॑।
शल्कै॒स्ताꣳरात्रि॑म॒ग्निमि॑न्धीत।
तस्मि॑न्नुपव्यु॒षम॒रणी॒ निष्ट॑पेत्।
यथ॑र्\mbox{}ष॒भाय॑ वाशि॒ता न्या॑विच्छा॒यति॑।
ता॒दृगे॒व तत्।
अ॒पो॒दूह्य॒ भस्मा॒ग्निं म॑न्थति॥७३॥

%1.1.9.10
सैव साऽग्नेः सन्त॑तिः।
तं म॑थि॒त्वा प्राञ्च॒मुद्ध॑रति।
सं॒व॒थ्स॒रमे॒व तद्रेतो॑ हि॒तं प्रज॑नयति।
अना॑हित॒स्तस्या॒ग्नि\-रित्या॑हुः।
यः स॒मिधो\-ऽना॑धाया॒ग्निमा॑ध॒त्त इति॑।
ताः सं॑वथ्स॒रे पु॒रस्ता॒दाद॑ध्यात्।
सं॒व॒थ्स॒रादे॒वैन॑मव॒रुध्याध॑त्ते।
यदि॑ संवथ्स॒रे\-ऽनाद॒ध्यात्।
द्वा॒द॒श्यां᳚ पु॒रस्ता॒दाद॑ध्यात्।
सं॒व॒थ्स॒रप्र॑तिमा॒ वै द्वाद॑श॒ रात्र॑यः।
सं॒व॒थ्स॒रमे॒वास्याऽऽहि॑ता भवन्ति।
यदि॑ द्वाद॒श्यां᳚ नाद॒ध्यात्।
त्र्य॒हे पु॒रस्ता॒दाद॑ध्यात्।
आहि॑ता ए॒वास्य॑ भवन्ति॥७४॥\anuvakamend[द्वि॒तीय॑मपचच्चतु॒र्थम॑पच॒ददि॑ती॒ रेतो॑\-ऽधत्त॒ सम्मि॑ता घृ॒तव॑तीभि॒राद॑धाति राज॒न्यः॑ स्वस्य॒ छन्द॑सः प्रत्ययन॒स्त्वाये॑याद्गच्छति मन्थति॒ रात्र॑यश्च॒त्वारि॑ च]

%1.1.10.1
प्र॒जा\-प॑तिः प्र॒जा अ॑\-सृजत।
स रि॑रिचा॒नो॑\-ऽमन्यत।
स तपो॑\-ऽतप्यत।
स आ॒त्मन्वी॒र्य॑मपश्यत्।
तद॑वर्धत।
तद॑स्मा॒थ्सह॑सो॒र्ध्वम॑सृज्यत।
सा वि॒राड॑भवत्।
तां दे॑वासु॒रा व्य॑गृह्णत।
सो᳚ऽब्रवीत्प्र॒जा\-प॑तिः।
मम॒ वा ए॒षा॥७५॥

%1.1.10.2
दोहा॑ ए॒व यु॒ष्माक॒मिति॑।
सा ततः॒ प्राच्युद॑क्रामत्।
तत्प्र॒जा\-प॑तिः॒ पर्य॑गृह्णात्।
अथ॑र्व पि॒तुं मे॑ गोपा॒येति॑।
सा द्वि॒तीय॒मुद॑क्रामत्।
तत्प्र॒जा\-प॑तिः॒ पर्य॑गृह्णात्।
नर्य॑ प्र॒जां मे॑ गोपा॒येति॑।
सा तृ॒तीय॒\-मुद॑\-क्रामत्।
तत्प्र॒जा\-प॑तिः॒ पर्य॑गृह्णात्।
शꣴस्य॑ प॒शून्मे॑ गोपा॒येति॑॥७६॥

%1.1.10.3
सा च॑तु॒र्थमुद॑क्रामत्।
तत्प्र॒जा\-प॑तिः॒ पर्य॑गृह्णात्।
सप्र॑थ स॒भां मे॑ गोपा॒येति॑।
सा प॑ञ्च॒ममुद॑क्रामत्।
तत्प्र॒जा\-प॑तिः॒ पर्य॑गृह्णात्।
अहे॑ बुध्निय॒ मन्त्रं॑ मे गोपा॒येति॑।
अ॒ग्नीन् वाव सा तान्व्य॑क्रमत।
तान्प्र॒जा\-प॑तिः॒ पर्य॑गृह्णात्।
अथो॑ प॒ङ्क्तिमे॒व।
प॒ङ्क्तिर्वा ए॒षा ब्रा᳚ह्म॒णे प्रवि॑ष्टा॥७७॥

%1.1.10.4
तामा॒त्मनोऽधि॒ निर्मि॑मीते।
यद॒ग्निरा॑धी॒यते᳚।
तस्मा॑दे॒ताव॑न्तो॒\-ऽग्नय॒ आधी॑यन्ते।
पाङ्क्तं॒ वा इ॒दꣳ सर्वम्᳚।
पाङ्क्ते॑नै॒व पाङ्क्तꣴ॑ स्पृणोति।
अथ॑र्व पि॒तुं मे॑ गोपा॒येत्या॑ह।
अन्न॑मे॒वैतेन॑ स्पृणोति।
नर्य॑ प्र॒जां मे॑ गोपा॒येत्या॑ह।
प्र॒जामे॒वैतेन॑ स्पृणोति।
शꣴस्य॑ प॒शून्मे॑ गोपा॒येत्या॑ह॥७८॥

%1.1.10.5
प॒शूने॒वैतेन॑ स्पृणोति।
सप्र॑थ स॒भां मे॑ गोपा॒येत्या॑ह।
स॒भामे॒वैतेने᳚न्द्रि॒यꣴ स्पृ॑णोति।
अहे॑ बुध्निय॒ मन्त्रं॑ मे गोपा॒येत्या॑ह।
मन्त्र॑मे॒वैतेन॒ श्रियꣴ॑ स्पृणोति।
यद॑न्वाहार्य॒पच॑ने\-ऽन्वाहा॒र्यं॑ पच॑न्ति।
तेन॒ सो᳚\-ऽस्या॒भीष्टः॑ प्री॒तः।
यद्गार्\mbox{}ह॑पत्य॒ आज्य॑मधि॒श्रय॑न्ति॒ सम्पत्नी᳚र्या॒जय॑न्ति।
तेन॒ सो᳚\-ऽस्या॒भीष्टः॑ प्री॒तः।
यदा॑हव॒नीये॒ जुह्व॑ति॥७९॥

%1.1.10.6
तेन॒ सो᳚\-ऽस्या॒भीष्टः॑ प्री॒तः।
यथ्स॒भायां᳚ वि॒जय॑न्ते।
तेन॒ सो᳚\-ऽस्या॒भीष्टः॑ प्री॒तः।
यदा॑वस॒थे\-ऽन्न॒ꣳ॒ हर॑न्ति।
तेन॒ सो᳚\-ऽस्या॒भीष्टः॑ प्री॒तः।
तथा᳚ऽस्य॒ सर्वे᳚ प्री॒ता अ॒भीष्टा॒ आधी॑यन्ते।
प्र॒व॒स॒थमे॒ष्यन्ने॒वमुप॑तिष्ठे॒तैक॑मेकम्।
यथा᳚ ब्राह्म॒णाय॑ गृहेवा॒सिने॑ परि॒दाय॑ गृ॒हानेति॑।
ता॒दृगे॒व तत्।
पुन॑रा॒गत्योप॑तिष्ठते।
सा भा॑गेयमे॒वैषां॒ तत्।
सा तत॑ ऊ॒र्ध्वारो॑हत्।
सा रो॑हि॒ण्य॑भवत्।
तद्रो॑हि॒ण्यै रो॑हिणि॒त्वम्।
रो॒हि॒ण्याम॒ग्निमाद॑धीत।
स्व ए॒वैनं॒ योनौ॒ प्रति॑\-ष्ठित॒माध॑त्ते।
ऋ॒ध्नोत्ये॑नेन॥८०॥\anuvakamend[ए॒षा प॒शून्मे॑ गोपा॒येति॒ प्रवि॑ष्टा प॒शून्मे॑ गोपा॒येत्या॑ह॒ जुह्व॑ति तिष्ठते स॒प्त च॑]





\prashnaend{ब्रह्म॒ सन्ध॑त्तं॒ कृत्ति॑का॒सूद्ध॑न्ति द्वाद॒शसु॑ प्र॒जा\-प॑तिर्वा॒चो दे॑वासु॒रास्तद॒ग्निर्नोद्{}घर्मः शिर॑ इ॒मे वै श॑मीग॒र्भात्प्र॒जा\-प॑तिः॒ स रि॑रिचा॒नः स तपः॒ स आ॒त्मन्वी॒र्यं॑ दश॑॥१०॥}{ब्रह्म॒ सन्ध॑त्तं॒ तौ दि॒व्यावथो॑ श॒न्त्वाय॒ प्राच्ये॑षां॒ यदु॒पर्यु॑परि॒ यथ्स॒द्यः सोऽश्वो॒\-ऽवारो॑ भू॒त्वा जग॑तीभि॒रशी॑तिः॥८०॥}{ब्रह्म॒ सन्ध॑त्तमृ॒ध्नोत्ये॑नेन॥}{हरिः॑ ओम्॥}{इति श्रीकृष्णयजुर्वेदीयतैत्तिरीयब्राह्मणे प्रथमाष्टके प्रथमः प्रपाठकः समाप्तः॥}
\clearpage
\sect{द्वितीयः प्रश्नः}
\setcounter{anuvakam}{0}
\dnsub{तैत्तिरीयब्राह्मणे प्रथमाष्टके द्वितीयः प्रपाठकः}

%1.2.1.1
उ॒द्ध॒न्यमा॑नम॒स्या अ॑मे॒ध्यम्।
अप॑ पा॒प्मानं॒ यज॑मानस्य हन्तु।
शि॒वा नः॑ सन्तु प्र॒दिश॒श्चत॑स्रः।
शं नो॑ मा॒ता पृ॑थि॒वी तोक॑साता।
शं नो॑ दे॒वीर॒भिष्ट॑ये।
आपो॑ भवन्तु पी॒तये᳚।
शं योर॒भि स्र॑वन्तु नः।
वै॒श्वा॒न॒रस्य॑ रू॒पम्।
पृ॒थि॒व्यां प॑रि॒स्रसा᳚।
स्यो॒नमा वि॑शन्तु नः॥१॥

%1.2.1.2
यदि॒दं दि॒वो यद॒दः पृ॑थि॒व्याः।
स॒ञ्ज॒ज्ञा॒ने रोद॑सी सम्बभू॒वतुः॑।
ऊषा᳚न्कृ॒ष्णम॑वतु कृ॒ष्णमूषाः᳚।
इ॒होभयो᳚र्य॒ज्ञिय॒\-माग॑मिष्ठाः।
ऊ॒तीः कु॑र्वा॒णो यत्पृ॑थि॒वीमच॑रः।
गु॒हा॒कार॑माखुरू॒पं प्र॒तीत्य॑।
तत्ते॒ न्य॑क्तमि॒ह स॒म्भर॑न्तः।
श॒तं जी॑वेम श॒रदः॒ सवी॑राः।
ऊर्जं॑ पृथि॒व्या रस॑मा॒भर॑न्तः।
श॒तं जी॑वेम श॒रदः॑ पुरू॒चीः॥२॥

%1.2.1.3
व॒म्रीभि॒रनु॑\-वित्तं॒ गुहा॑सु।
श्रोत्रं॑ त उ॒र्व्यब॑धिरा भवामः।
प्र॒जा\-प॑तिसृष्टानां प्र॒जाना᳚म्।
क्षु॒धो\-ऽप॑हत्यै सुवि॒तं नो॑ अस्तु।
उप॒ प्रभि॑न्न॒मिष॒मूर्जं॑ प्र॒जाभ्यः॑।
सूदं॑ गृ॒हेभ्यो॒ रस॒माभ॑रामि।
यस्य॑ रू॒पं बिभ्र॑दि॒मामवि॑न्दत्।
गुहा॒ प्रवि॑ष्टाꣳ सरि॒रस्य॒ मध्ये᳚।
तस्ये॒दं विह॑तमा॒भर॑न्तः।
अछ॑म्बट्कारम॒स्यां वि॑धेम॥३॥

%1.2.1.4
यत्प॒र्यप॑श्यथ्सरि॒रस्य॒ मध्ये᳚।
उ॒र्वीमप॑श्य॒ज्जग॑तः प्रति॒ष्ठाम्।
तत्पुष्क॑रस्या॒ऽऽयत॑ना॒द्धि जा॒तम्।
प॒र्णं पृ॑थि॒व्याः प्रथ॑नꣳ हरामि।
याभि॒रदृꣳ॑ह॒ज्जग॑तः प्रति॒ष्ठाम्।
उ॒र्वीमि॒मां वि॑श्वज॒नस्य॑ भ॒र्त्रीम्।
ता नः॑ शि॒वाः शर्क॑राः सन्तु॒ सर्वाः᳚।
अ॒ग्ने रेत॑श्च॒न्द्रꣳ हिर॑ण्यम्।
अ॒द्भ्यः सम्भू॑तम॒मृतं॑ प्र॒जासु॑।
तथ्स॒म्भर॑न्नुत्तर॒तो नि॒धाय॑॥४॥

%1.2.1.5
अ॒ति॒प्र॒यच्छं॒ दुरि॑तिं तरेयम्।
अश्वो॑ रू॒पं कृ॒त्वा यद॑श्व॒त्थे\-ऽति॑ष्ठः।
सं॒व॒थ्स॒रं दे॒वेभ्यो॑ नि॒लाय॑।
तत्ते॒ न्य॑क्तमि॒ह स॒म्भर॑न्तः।
श॒तं जी॑वेम श॒रदः॒ सवी॑राः।
ऊ॒र्जः पृ॑थि॒व्या अध्युत्थि॑तोऽसि।
वन॑स्पते श॒तव॑ल्‌शो॒ विरो॑ह।
त्वया॑ व॒यमिष॒मूर्जं॒ मद॑न्तः।
रा॒यस्पोषे॑ण॒ समि॒षा म॑देम।
गा॒य॒त्रि॒या ह्रि॒यमा॑णस्य॒ यत्ते᳚॥५॥

%1.2.1.6
प॒र्णमप॑तत्तृ॒तीय॑स्यै दि॒वोऽधि॑।
सो॑ऽयं प॒र्णः सो॑मप॒र्णाद्धि जा॒तः।
ततो॑ हरामि सोमपी॒थस्या\-व॑\-रुद्ध्यै।
दे॒वानां᳚ ब्रह्मवा॒दं वद॑तां॒ यत्।
उ॒पाशृ॑णोः सु॒श्रवा॒ वै श्रु॒तो॑ऽसि।
ततो॒ मामावि॑शतु ब्रह्मवर्च॒सम्।
तथ्स॒म्भर॒ꣴ॒स्तदव॑रुन्धीय सा॒क्षात्।
यया॑ ते सृ॒ष्टस्या॒ग्नेः।
हे॒तिमश॑मयत्प्र॒जा\-प॑तिः।
तामि॒मामप्र॑दाहाय॥६॥

%1.2.1.7
श॒मीꣳ शान्त्यै॑ हराम्य॒हम्।
यत्ते॑ सृ॒ष्टस्य॑ य॒तः।
विक॑ङ्कतं॒ भा आ᳚र्च्छज्जातवेदः।
तया॑ भा॒सा सम्मि॑तः।
उ॒रुं नो॑ लो॒कमनु॒ प्रभा॑हि।
यत्ते॑ ता॒न्तस्य॒ हृद॑य॒माच्छि॑न्दञ्जातवेदः।
म॒रुतो॒\-ऽद्भिस्त॑मयि॒त्वा।
ए॒तत्ते॒ तद॑श॒नेः सम्भ॑रामि।
सात्मा॑ अग्ने॒ सहृ॑दयो भवे॒ह।
चित्रि॑यादश्व॒त्थाथ्सम्भृ॑ता बृह॒त्यः॑॥७॥

%1.2.1.8
शरी॑रम॒भि सꣴस्कृ॑ताः स्थ।
प्र॒जा\-प॑तिना यज्ञमु॒खेन॒ सम्मि॑ताः।
ति॒स्रस्त्रि॒वृद्भि॑र्मिथु॒नाः प्रजा᳚त्यै।
अ॒श्व॒त्थाद्ध॑व्य\-वा॒हाद्धि जा॒ताम्।
अ॒ग्नेस्त॒नूं य॒ज्ञिया॒ꣳ॒ सम्भ॑रामि।
शा॒न्तयो॑निꣳ शमीग॒र्भम्।
अ॒ग्नये॒ प्रज॑नयि॒तवे᳚।
यो अ॑श्व॒त्थः श॑मीग॒र्भः।
आ॒रु॒रोह॒ त्वे सचा᳚।
तं ते॑ हरामि॒ ब्रह्म॑णा॥८॥

%1.2.1.9
य॒ज्ञियैः᳚ के॒तुभिः॑ स॒ह।
यं त्वा॑ स॒मभ॑रञ्जातवेदः।
य॒था॒श॒री॒रं भू॒तेषु॒ न्य॑क्तम्।
स सम्भृ॑तः सीद शि॒वः प्र॒जाभ्यः॑।
उ॒रुं नो॑ लो॒कमनु॑नेषि वि॒द्वान्।
प्रवे॒धसे॑ क॒वये॒ मेध्या॑य।
वचो॑ व॒न्दारु॑ वृष॒भाय॒ वृष्णे᳚।
यतो॑ भ॒यमभ॑यं॒ तन्नो॑ अस्तु।
अव॑ दे॒वान् य॑जे॒ हेड्यान्॑।
स॒मिधा॒\-ऽग्निं दु॑वस्यत॥९॥

\dnsub{घृत-सूक्तम्}

%1.2.1.10
घृ॒तैर्बो॑धय॒ताति॑थिम्।
आऽस्मि॑न् ह॒व्या जु॑होतन।
उप॑ त्वा\-ऽग्ने ह॒विष्म॑तीः।
घृ॒ताची᳚र्यन्तु हर्यत।
जु॒षस्व॑ स॒मिधो॒ मम॑।
तं त्वा॑ स॒मिद्भि॑रङ्गिरः।
घृ॒तेन॑ वर्धयामसि।
बृ॒हच्छो॑चा यविष्ठ्य।
स॒मि॒ध्यमा॑नः प्रथ॒मो नु धर्मः॑।
सम॒क्तुभि॑रज्यते वि॒श्ववा॑रः॥१०॥

%1.2.1.11
शो॒चिष्के॑शो घृ॒तनि॑र्णिक्पाव॒कः।
सु॒य॒ज्ञो अ॒ग्निर्य॒जथा॑य दे॒वान्।
घृ॒तप्र॑तीको घृ॒तयो॑निर॒ग्निः।
घृ॒तैः समि॑द्धो घृ॒तम॒स्यान्नम्᳚।
घृ॒त॒प्रुष॑स्त्वा स॒रितो॑ वहन्ति।
घृ॒तं पिब᳚न्थ्सु॒यजा॑ यक्षि दे॒वान्।
आ॒यु॒र्दा अ॑ग्ने ह॒विषो॑ जुषा॒णः।
घृ॒तप्र॑तीको घृ॒तयो॑निरेधि।
घृ॒तं पी॒त्वा मधु॒ चारु॒ गव्यम्᳚।
पि॒तेव॑ पु॒त्रम॒भिर॑क्षतादि॒मम्॥११॥

{\small \closesub{}}

%1.2.1.12
त्वाम॑ग्ने समिधा॒नं य॑विष्ठ।
दे॒वा दू॒तं च॑क्रिरे हव्य॒वाहम्᳚।
उ॒रु॒ज्रय॑सं घृ॒तयो॑नि॒माहु॑तम्।
त्वे॒षं चक्षु॑र्दधिरे चोद॒यन्व॑ति।
त्वाम॑ग्ने प्र॒दिव॒ आहु॑तं घृ॒तेन॑।
सु॒म्ना॒यवः॑ सुष॒मिधा॒ समी॑धिरे।
स वा॑वृधा॒न ओष॑धीभिरुक्षि॒तः।
उ॒रु ज्रयाꣳ॑सि॒ पार्थि॑वा॒ विति॑ष्ठसे।
घृ॒तप्र॑तीकं व ऋ॒तस्य॑ धूर्॒षदम्᳚।
अ॒ग्निं मि॒त्रं न स॑मिधा॒न ऋ॑ञ्जते॥१२॥

%1.2.1.13
इन्धा॑नो अ॒क्रो वि॒दथे॑षु॒ दीद्य॑त्।
शु॒क्रव॑र्णा॒मुदु॑ नो यꣳसते॒ धियम्᳚।
प्र॒जा अ॑ग्ने॒ संवा॑सय।
आशा᳚श्च प॒शुभिः॑ स॒ह।
रा॒ष्ट्राण्य॑स्मा॒ आधे॑हि।
यान्यास᳚न्थ्सवि॒तुः स॒वे।
म॒ही वि॒श्पत्नी॒ सद॑ने ऋ॒तस्य॑।
अ॒र्वाची॒ एतं॑ धरुणे रयी॒णाम्।
अ॒न्तर्व॑त्नी॒ जन्यं॑ जा॒तवे॑दसम्।
अ॒ध्व॒राणां᳚ जनयथः पुरो॒गाम्॥१३॥

%1.2.1.14
आरो॑हतं द॒शत॒ꣳ॒ शक्व॑री॒र्मम॑।
ऋ॒तेना᳚ग्न॒ आयु॑षा॒ वर्च॑सा स॒ह।
ज्योग्जीव॑न्त॒ उत्त॑रामुत्तरा॒ꣳ॒ समा᳚म्।
दर्\mbox{}श॑म॒हं पू॒र्णमा॑सं य॒ज्ञं यथा॒ यजै᳚।
ऋत्वि॑यवती स्थो अ॒ग्निरे॑तसौ।
गर्भं॑ दधाथां॒ ते वा॑म॒हं द॑दे।
तथ्स॒त्यं यद्वी॒रं बि॑भृथः।
वी॒रं ज॑नयि॒ष्यथः॑।
ते मत्प्रा॒तः प्रज॑निष्येथे।
ते मा॒ प्रजा॑ते॒ प्रज॑नयि॒ष्यथः॑॥१४॥

%1.2.1.15
प्र॒जया॑ प॒शुभि॑र्ब्रह्मवर्च॒सेन॑ सुव॒र्गे लो॒के।
अनृ॑ताथ्स॒त्य\-मुपै॑मि।
मा॒नु॒षाद्दैव्य॒मुपै॑मि।
दैवीं॒ वाचं॑ यच्छामि।
शल्कै॑र॒ग्निमि॑न्धा॒नः।
उ॒भौ लो॒कौ स॑नेम॒हम्।
उ॒भयो᳚र्लो॒कयोर्॑ ऋ॒ध्वा।
अति॑ मृ॒त्युं त॑राम्य॒हम्।
जात॑वेदो॒ भुव॑नस्य॒ रेतः॑।
इ॒ह सि॑ञ्च॒ तप॑सो॒ यज्ज॑नि॒ष्यते᳚॥१५॥

%1.2.1.16
अ॒ग्निम॑श्व॒त्थादधि॑ हव्य॒वाहम्᳚।
श॒मी॒ग॒र्भाज्ज॒नय॒न्॒ यो म॑यो॒भूः।
अ॒यं ते॒ योनि॑र्‌\mbox{}ऋ॒त्वियः॑।
यतो॑ जा॒तो अरो॑चथाः।
तं जा॒नन्न॑ग्न॒ आरो॑ह।
अथा॑ नो वर्धया र॒यिम्।
अपे॑त॒ वीत॒ वि च॑ सर्प॒तातः॑।
येऽत्र॒ स्थ पु॑रा॒णा ये च॒ नूत॑नाः।
अदा॑दि॒दं य॒मो॑\-ऽव॒सानं॑ पृथि॒व्याः।
अक्र॑न्नि॒मं पि॒तरो॑ लो॒कम॑स्मै॥१६॥

%1.2.1.17
अ॒ग्नेर्भस्मा᳚स्य॒ग्नेः पुरी॑षमसि।
सं॒ज्ञान॑मसि काम॒धर॑णम्।
मयि॑ ते काम॒धर॑णं भूयात्।
संवः॑ सृजामि॒ हृद॑यानि।
सꣳसृ॑ष्टं॒ मनो॑ अस्तु वः।
सꣳसृ॑ष्टः प्रा॒णो अ॑स्तु वः।
सं या वः॑ प्रि॒यास्त॒नुवः॑।
सं प्रि॒या हृद॑यानि वः।
आ॒त्मा वो॑ अस्तु॒ सम्प्रि॑यः।
सम्प्रि॑यास्त॒नुवो॒ मम॑॥१७॥

%1.2.1.18
कल्पे॑तां॒ द्यावा॑पृथि॒वी।
कल्प॑न्ता॒माप॒ ओष॑धीः।
कल्प॑न्ताम॒ग्नयः॒ पृथ॑क्।
मम॒ ज्यैष्ठ्या॑य॒ सव्र॑ताः।
ये᳚ऽग्नयः॒ सम॑नसः।
अ॒न्त॒रा द्यावा॑पृथि॒वी।
वास॑न्तिकावृ॒तू अ॒भि कल्प॑मानाः।
इन्द्र॑मिव दे॒वा अ॒भि सं वि॑शन्तु।
दि॒वस्त्वा॑ वी॒र्ये॑ण।
पृ॒थि॒व्यै म॑हि॒म्ना॥१८॥

%1.2.1.19
अ॒न्तरि॑क्षस्य॒ पोषे॑ण।
स॒र्वप॑शु॒माद॑धे।
अजी॑जनन्न॒मृतं॒ मर्त्या॑सः।
अ॒स्रे॒माणं॑ त॒रणिं॑ वी॒डुज॑म्भम्।
दश॒ स्वसा॑रो अ॒ग्रुवः॑ समी॒चीः।
पुमाꣳ॑सं जा॒तम॒भि सꣳर॑भन्ताम्।
प्र॒जा\-प॑तेस्त्वा प्रा॒णेनाभि॒ प्राणि॑मि।
पू॒ष्णः पोषे॑ण॒ मह्यम्᳚।
दी॒र्घा॒यु॒त्वाय॑ श॒तशा॑रदाय।
श॒तꣳ श॒रद्भ्य॒ आयु॑षे॒ वर्च॑से॥१९॥

%1.2.1.20
जी॒वात्वै पुण्या॑य।
अ॒हं त्वद॑स्मि॒ मद॑सि॒ त्वमे॒तत्।
ममा॑सि॒ योनि॒स्तव॒ योनि॑रस्मि।
ममै॒व सन्वह॑ ह॒व्यान्य॑ग्ने।
पु॒त्रः पि॒त्रे लो॑क॒कृज्जा॑तवेदः।
प्रा॒णे त्वा॒\-ऽमृत॒माद॑धामि।
अ॒न्ना॒दम॒न्नाद्या॑य।
गो॒प्तारं॒ गुप्त्यै᳚।
सु॒गा॒र्॒ह॒प॒त्यो वि॒दह॒न्नरा॑तीः।
उ॒षसः॒ श्रेय॑सीः श्रेयसी॒र्दध॑त्॥२०॥

%1.2.1.21
अग्ने॑ स॒पत्नाꣳ॑ अप॒ बाध॑मानः।
रा॒यस्पोष॒मिष॒मूर्ज॑म॒स्मासु॑ धेहि।
इ॒मा उ॒ मामुप॑तिष्ठन्तु॒ रायः॑।
आ॒भिः प्र॒जाभि॑रि॒ह संव॑सेय।
इ॒हो इडा॑ तिष्ठतु विश्वरू॒पी।
मध्ये॒ वसो᳚र्दीदिहि जातवेदः।
ओज॑से॒ बला॑य॒ त्वोद्य॑च्छे।
वृष॑णे॒ शुष्मा॒या\-ऽऽ\-यु॑षे॒ वर्च॑से।
स॒प॒त्न॒तूर॑सि वृत्र॒तूः।
यस्ते॑ दे॒वेषु॑ महि॒मा सु॑व॒र्गः॥२१॥

%1.2.1.22
यस्त॑ आ॒त्मा प॒शुषु॒ प्रवि॑ष्टः।
पुष्टि॒र्या ते॑ मनु॒ष्ये॑षु पप्र॒थे।
तया॑ नो अग्ने जु॒षमा॑ण॒ एहि॑।
दि॒वः पृ॑थि॒व्याः पर्य॒न्तरि॑क्षात्।
वाता᳚त्प॒शुभ्यो॒ अध्योष॑धीभ्यः।
यत्र॑ यत्र जातवेदः सम्ब॒भूथ॑।
ततो॑ नो अग्ने जु॒षमा॑ण॒ एहि॑।
प्राची॒मनु॑ प्र॒दिशं॒ प्रेहि॑ वि॒द्वान्।
अ॒ग्नेर॑ग्ने पु॒रो अ॑ग्निर्भवे॒ह।
विश्वा॒ आशा॒ दीद्या॑नो॒ वि भा॑हि॥२२॥

%1.2.1.23
ऊर्जं॑ नो धेहि द्वि॒पदे॒ चतु॑ष्पदे।
अन्व॒ग्निरु॒षसा॒मग्र॑मख्यत्।
अन्वहा॑नि प्रथ॒मो जा॒तवे॑दाः।
अनु॒ सूर्य॑स्य पुरु॒त्रा च॑ र॒श्मीन्।
अनु॒ द्यावा॑पृथि॒वी आत॑तान।
विक्र॑मस्व म॒हाꣳ अ॑सि।
वे॒दि॒षन्मानु॑षेभ्यः।
त्रि॒षु लो॒केषु॑ जागृहि।
यदि॒दं दि॒वो यद॒दः पृ॑थि॒व्याः।
सं॒वि॒दा॒ने रोद॑सी सं बभू॒वतुः॑॥२३॥

%1.2.1.24
तयोः᳚ पृ॒ष्ठे सी॑दतु जा॒तवे॑दाः।
श॒म्भूः प्र॒जाभ्य॑स्त॒नुवे᳚ स्यो॒नः।
प्रा॒णं त्वा॒\-ऽमृत॒ आ द॑धामि।
अ॒न्ना॒दम॒न्नाद्या॑य।
गो॒प्तारं॒ गुप्त्यै᳚।
यत्ते॑ शुक्र शु॒क्रं वर्चः॑ शु॒क्रा त॒नूः।
शु॒क्रं ज्योति॒रज॑स्रम्।
तेन॑ मे दीदिहि॒ तेन॒ त्वाऽऽद॑धे।
अ॒ग्निना᳚\-ऽग्ने॒ ब्रह्म॑णा।
आ॒न॒शे व्या॑नशे॒ सर्व॒मायु॒र्व्या॑नशे॥२४॥

%1.2.1.25
नर्य॑ प्र॒जां मे॑ गोपाय।
अ॒मृ॒त॒त्वाय॑ जी॒वसे᳚।
जा॒तां ज॑नि॒ष्यमा॑णां च।
अ॒मृते॑ स॒त्ये प्रति॑\-ष्ठिताम्।
अथ॑र्व पि॒तुं मे॑ गोपाय।
रस॒मन्न॑मि॒हायु॑षे।
अद॑ब्धा॒यो\-ऽशी॑ततनो।
अवि॑षन्नः पि॒तुं कृ॑णु।
शꣴस्य॑ प॒शून्मे॑ गोपाय।
द्वि॒पादो॒ ये चतु॑ष्पदः॥२५॥

%1.2.1.26
अ॒ष्टाश॑फाश्च॒ य इ॒हाग्ने᳚।
ये चैक॑शफा आशु॒गाः।
सप्र॑थ स॒भां मे॑ गोपाय।
ये च॒ सभ्याः᳚ सभा॒सदः॑।
तानि॑न्द्रि॒याव॑तः कुरु।
सर्व॒मायु॒रुपा॑सताम्।
अहे॑ बुध्निय॒ मन्त्रं॑ मे गोपाय।
यमृष॑यस्त्रैवि॒दा वि॒दुः।
ऋचः॒ सामा॑नि॒ यजूꣳ॑षि।
सा हि श्रीर॒मृता॑ स॒ताम्॥२६॥

%1.2.1.27
चतुः॑ शिखण्डा युव॒तिः सु॒पेशाः᳚।
घृ॒तप्र॑तीका॒ भुव॑नस्य॒ मध्ये᳚।
म॒र्मृ॒ज्यमा॑ना मह॒ते सौभ॑गाय।
मह्यं॑ धुक्ष्व॒ यज॑मानाय॒ कामान्॑।
इ॒हैव सन्तत्र॑ स॒तो वो॑ अग्नयः।
प्रा॒णेन॑ वा॒चा मन॑सा बिभर्मि।
ति॒रो मा॒ सन्त॒मायु॒र्मा प्रहा॑सीत्।
ज्योति॑षा वो वैश्वान॒रेणोप॑तिष्ठे।
प॒ञ्च॒धा\-ऽग्नीन्व्य॑क्रामत्।
वि॒राट्थ्सृ॒ष्टा प्र॒जा\-प॑तेः।
ऊ॒र्ध्वा\-ऽऽरो॑हद्रोहि॒णी।
योनि॑र॒ग्नेः प्रति॑\-ष्ठितिः॥२७॥\anuvakamend[वि॒श॒न्तु॒ नः॒ पु॒रू॒चीर्वि॑धेम नि॒धाय॒ यत्ते\-ऽप्र॑दाहाय बृह॒त्यो᳚ ब्रह्म॑णा दुवस्यत वि॒श्ववा॑र इ॒ममृ॑ञ्जते पुरो॒गां प्रज॑नयि॒ष्यथो॑ जनि॒ष्यते᳚\-ऽस्मै॒ मम॑ महि॒म्ना वर्च॑से॒ दध॑थ्सुव॒र्गो भा॑हि सम्बभू॒वतु॒रायु॒र्व्या॑नशे॒ चतु॑ष्पदः स॒तां प्र॒जा\-प॑ते॒र्द्वे च॑]

%1.2.2.1
नवै॒तान्यहा॑नि भवन्ति।
नव॒ वै सु॑व॒र्गा लो॒काः।
यदे॒तान्यहा᳚न्युप॒यन्ति॑।
न॒वस्वे॒व तथ्सु॑व॒र्गेषु॑ लो॒केषु॑ स॒त्रिणः॑ प्रति॒तिष्ठ॑न्तो यन्ति।
अ॒ग्नि॒ष्टो॒माः परः॑ सामानः का॒र्या॑ इत्या॑हुः।
अ॒ग्नि॒ष्टो॒मस॑म्मितः सुव॒र्गो लो॒क इति॑।
द्वाद॑शाग्निष्टो॒मस्य॑ स्तो॒त्राणि॑।
द्वाद॑श॒ मासाः᳚ संवथ्स॒रः।
तत्तन्न सूर्क्ष्यम्᳚।
उ॒क्थ्या॑ ए॒व स॑प्तद॒शाः परः॑ सामानः का॒र्याः᳚॥२८॥

%1.2.2.2
प॒शवो॒ वा उ॒क्थानि॑।
प॒शू॒नामव॑रुद्ध्यै।
वि॒श्व॒जि॒द॒भि॒जिता॑\-वग्निष्टो॒मौ।
उ॒क्थ्याः᳚ सप्तद॒शाः परः॑ सामानः।
ते सꣴस्तु॑ता वि॒राज॑म॒भि सम्प॑द्यन्ते।
द्वे चर्चा॒वति॑रिच्येते।
एक॑या॒ गौरति॑रिक्तः।
एक॒या\-ऽऽयु॑रू॒नः।
सु॒व॒र्गो वै लो॒को ज्योतिः॑।
ऊर्ग्वि॒राट्॥२९॥

%1.2.2.3
सु॒व॒र्गमे॒व तेन॑ लो॒कम॒भि ज॑यन्ति।
यत्पर॒ꣳ॒ राथ॑न्तरम्।
तत्प्र॑थ॒मे\-ऽह॑न्का॒र्यम्᳚।
बृ॒हद्द्वि॒तीये᳚।
वै॒रू॒पं तृ॒तीये᳚।
वै॒रा॒जं च॑तु॒र्थे।
शा॒क्व॒रं प॑ञ्च॒मे।
रै॒व॒तꣳ ष॒ष्ठे।
तदु॑ पृ॒ष्ठेभ्यो॒ नय॑न्ति।
स॒न्तन॑य ए॒ते ग्रहा॑ गृह्यन्ते॥३०॥

%1.2.2.4
अ॒ति॒ग्रा॒ह्याः᳚ परः॑ सामसु।
इ॒माने॒वैतैर्लो॒कान्थ्सन्त॑न्वन्ति।
मि॒थु॒ना ए॒ते ग्रहा॑ गृह्यन्ते।
अ॒ति॒ग्रा॒ह्याः᳚ परः॑ सामसु।
मि॒थु॒नमे॒व तैर्यज॑माना॒ अव॑रुन्धते।
बृ॒हत्पृ॒ष्ठं भ॑वति।
बृ॒हद्वै सु॑व॒र्गो लो॒कः।
बृ॒ह॒तैव सु॑व॒र्गं लो॒कं य॑न्ति।
त्र॒य॒स्त्रि॒ꣳ॒शि  नाम॒ साम॑।
माध्यं॑ दिने॒ पव॑माने भवति॥३१॥

%1.2.2.5
त्रय॑स्त्रिꣳश॒द्वै दे॒वताः᳚।
दे॒वता॑ ए॒वाव॑रुन्धते।
ये वा इ॒तः परा᳚ञ्चꣳ संवथ्स॒रमु॑प॒यन्ति॑।
न है॑नं॒ ते स्व॒स्ति सम॑श्ञुवते।
अथ॒ ये॑\-ऽमुतो॒\-ऽर्वाञ्च॑मुप॒यन्ति॑।
ते है॑नꣴ स्व॒स्ति सम॑श्ञुवते।
ए॒तद्वा अ॒मुतो॒\-ऽर्वाञ्च॒मुप॑यन्ति।
यदे॒वम्।
यो ह॒ खलु॒ वाव प्र॒जा\-प॑तिः।
स उ॑वे॒वेन्द्रः॑।
तदु॑ दे॒वेभ्यो॒ नय॑न्ति॥३२॥\anuvakamend[का॒र्या॑ वि॒राड्गृ॑ह्यन्ते॒ पव॑माने भव॒तीन्द्र॒ एकं॑ च]

%1.2.3.1
सन्त॑ति॒र्वा ए॒ते ग्रहाः᳚।
यत्परः॑ सामानः।
वि॒षू॒वान्दि॑वा\-की॒र्त्यम्᳚।
यथा॒ शाला॑यै॒ पक्ष॑सी।
ए॒वꣳ सं॑वथ्स॒रस्य॒ पक्ष॑सी।
यदे॒तेन गृ॒ह्येरन्॑।
विषू॑ची संवथ्स॒रस्य॒ पक्ष॑सी॒ व्यव॑स्रꣳसेयाताम्।
आर्ति॒मार्च्छे॑युः।
यदे॒ते गृ॒ह्यन्ते᳚।
यथा॒ शाला॑यै॒ पक्ष॑सी मध्य॒मं व॒ꣳ॒शम॒भि स॑मा॒यच्छ॑ति॥३३॥

%1.2.3.2
ए॒वꣳ सं॑वथ्स॒रस्य॒ पक्ष॑सी दिवाकी॒र्त्य॑म॒भि सं त॑न्वन्ति।
नार्ति॒मार्च्छ॑न्ति।
ए॒क॒वि॒ꣳ॒शमह॑र्भवति।
शु॒क्राग्रा॒ ग्रहा॑ गृह्यन्ते।
प्रत्युत्त॑ब्ध्यै सय॒त्वाय॑।
सौ॒र्य॑ ए॒तदहः॑ प॒शुराल॑भ्यते।
सौ॒र्यो॑\-ऽतिग्रा॒ह्यो॑ गृह्यते।
अह॑रे॒व रू॒पेण॒ सम॑र्धयन्ति।
अथो॒ अह्न॑ ए॒वैष ब॒लिर्\mbox{}ह्रि॑यते।
स॒प्तैतदह॑रतिग्रा॒ह्या॑ गृह्यन्ते॥३४॥

%1.2.3.3
स॒प्त वै शी॑र्\mbox{}ष॒ण्याः᳚ प्रा॒णाः।
अ॒सावा॑दि॒त्यः शिरः॑ प्र॒जाना᳚म्।
शी॒र्॒षन्ने॒व प्र॒जानां᳚ प्रा॒णान्द॑धाति।
तस्मा᳚थ्स॒प्त शी॒र्॒षन्प्रा॒णाः।
इन्द्रो॑ वृ॒त्रꣳ ह॒त्वा।
असु॑रान्परा॒भाव्य॑।
स इ॒माँल्लो॒कान॒भ्य॑जयत्।
तस्या॒सौ लो॒को\-ऽन॑भिजित आसीत्।
तं वि॒श्वक॑र्मा भू॒त्वा\-ऽभ्य॑जयत्।
यद्वै᳚श्वकर्म॒णो गृ॒ह्यते᳚॥३५॥

%1.2.3.4
सु॒व॒र्गस्य॑ लो॒कस्या॒भिजि॑त्यै।
प्र वा ए॒ते᳚\-ऽस्माल्लो॒काच्च्य॑वन्ते।
ये वै᳚श्वकर्म॒णं गृ॒ह्णते᳚।
आ॒दि॒त्यः श्वो गृ॑ह्यते।
इ॒यं वा अदि॑तिः।
अ॒स्यामे॒व प्रति॑ तिष्ठन्ति।
अ॒न्यो᳚न्यो गृह्येते।
विश्वा᳚न्ये॒वान्येन॒ कर्मा॑णि कुर्वा॒णा य॑न्ति।
अ॒स्याम॒न्येन॒ प्रति॑ तिष्ठन्ति।
तावाऽप॑रा॒र्धाथ्सं॑वथ्स॒रस्या॒न्यो᳚न्यो गृह्येते।
तावु॒भौ स॒ह म॑हाव्र॒ते गृ॑ह्येते।
य॒ज्ञस्यै॒वान्तं॑ ग॒त्वा।
उ॒भयो᳚र्लो॒कयोः॒ प्रति॑ तिष्ठन्ति।
अ॒र्क्य॑मु॒क्थं भ॑वति।
अ॒न्नाद्य॒स्या\-व॑\-रुद्ध्यै॥३६॥\anuvakamend[स॒मा॒यच्छ॑त्यतिग्रा॒ह्या॑ गृह्यन्ते गृ॒ह्यते॑ संवथ्स॒रस्या॒न्यो᳚न्यो गृह्येते॒ पञ्च॑ च]

%1.2.4.1
ए॒क॒वि॒ꣳ॒श ए॒ष भ॑वति।
ए॒तेन॒ वै दे॒वा ए॑कवि॒ꣳ॒शेन॑।
आ॒दि॒त्यमि॒त उ॑त्त॒मꣳ सु॑व॒र्गं लो॒कमारो॑हयन्।
स वा ए॒ष इ॒त ए॑कवि॒ꣳ॒शः।
तस्य॒ दशा॒वस्ता॒दहा॑नि।
दश॑ प॒रस्ता᳚त्।
स वा ए॒ष वि॒राज्यु॑भ॒यतः॒ प्रति॑\-ष्ठितः।
वि॒राजि॒ हि वा ए॒ष उ॑भ॒यतः॒ प्रति॑\-ष्ठितः।
तस्मा॑दन्त॒रेमौ लो॒कौ यन्।
सर्वे॑षु सुव॒र्गेषु॑ लो॒केष्व॑भि॒तप॑न्नेति॥३७॥

%1.2.4.2
दे॒वा वा आ॑दि॒त्यस्य॑ सुव॒र्गस्य॑ लो॒कस्य॑।
परा॑चो\-ऽतिपा॒दा\-द॑बिभयुः।
तं छन्दो॑भिरदृꣳहं॒ धृत्यै᳚।
दे॒वा वा आ॑दि॒त्यस्य॑ सुव॒र्गस्य॑ लो॒कस्य॑।
अवा॑चो\-ऽवपा॒दाद॑बिभयुः।
तं प॒ञ्चभी॑ र॒श्मिभि॒रुद॑वयन्।
तस्मा॑देकवि॒ꣳ॒शे\-ऽह॒न्पञ्च॑ दिवाकी॒र्त्या॑नि क्रियन्ते।
र॒श्मयो॒ वै दि॑वाकी॒र्त्या॑नि।
ये गा॑य॒त्रे।
ते गा॑य॒त्रीषूत्त॑रयोः॒ पव॑मानयोः॥३८॥

%1.2.4.3
म॒हादि॑वाकीर्त्य॒ꣳ॒ होतुः॑ पृ॒ष्ठम्।
वि॒क॒र्णं ब्र॑ह्मसा॒मम्।
भा॒सो᳚\-ऽग्निष्टो॒मः।
अथै॒तानि॒ परा॑णि।
परै॒र्वै दे॒वा आ॑दि॒त्यꣳ सु॑व॒र्गं लो॒कम॑पारयन्।
यदपा॑रयन्।
तत्परा॑णां पर॒त्वम्।
पा॒रय॑न्त्येनं॒ परा॑णि।
य ए॒वं वेद॑।
अथै॒तानि॒ स्परा॑णि।
स्परै॒र्वै दे॒वा आ॑दि॒त्यꣳ सु॑व॒र्गं लो॒कम॑स्पारयन्।
यदस्पा॑रयन्।
तथ्स्परा॑णाꣴ स्पर॒त्वम्।
स्पा॒रय॑न्त्यैन॒ꣴ॒ स्परा॑णि।
य ए॒वं वेद॑॥३९॥\anuvakamend[ए॒ति॒ पव॑मानयोः॒ स्परा॑णि॒ पञ्च॑ च]

%1.2.5.1
अप्र॑तिष्ठां॒ वा ए॒ते ग॑च्छन्ति।
येषाꣳ॑ संवथ्स॒रे\-ऽना॒प्तेऽथ॑।
ए॒का॒द॒शिन्या॒प्यते᳚।
वै॒ष्ण॒वं वा॑म॒नमाल॑भन्ते।
य॒ज्ञो वै विष्णुः॑।
य॒ज्ञमे॒वाल॑भन्ते॒ प्रति॑\-ष्ठित्यै।
ऐ॒न्द्रा॒ग्नमाल॑भन्ते।
इ॒न्द्रा॒ग्नी वै दे॒वाना॒मया॑तयामानौ।
ये ए॒व दे॒वते॒ अया॑तयाम्नी।
ते ए॒वाऽऽल॑भन्ते॥४०॥

%1.2.5.2
वै॒श्व॒दे॒वमाल॑भन्ते।
दे॒वता॑ ए॒वाव॑रुन्धते।
द्या॒वा॒पृ॒थिव्यां᳚ धे॒नुमाल॑भन्ते।
द्यावा॑पृथि॒व्योरे॒व प्रति॑ तिष्ठन्ति।
वा॒य॒व्यं॑ व॒थ्समाल॑भन्ते।
वा॒युरे॒वैभ्यो॑ यथा\-ऽऽयत॒नाद्दे॒वता॒ अव॑ रुन्धे।
आ॒दि॒त्यामविं॑ व॒शामाल॑भन्ते।
इ॒यं वा अदि॑तिः।
अ॒स्यामे॒व प्रति॑ तिष्ठन्ति।
मै॒त्रा॒व॒रु॒णीमाल॑भन्ते॥४१॥

%1.2.5.3
मि॒त्रेणै॒व य॒ज्ञस्य॒ स्वि॑ष्टꣳ शमयन्ति।
वरु॑णेन॒ दुरि॑ष्टम्।
प्रा॒जा॒प॒त्यं तू॑प॒रं म॑हाव्र॒त आल॑भन्ते।
प्रा॒जा॒प॒त्यो॑\-ऽतिग्रा॒ह्यो॑ गृह्यते।
अह॑रे॒व रू॒पेण॒ सम॑र्धयन्ति।
अथो॒ अह्न॑ ए॒वैष ब॒लिर्\mbox{}ह्रि॑यते।
आ॒ग्ने॒यमा ल॑भन्ते॒ प्रति॒ प्रज्ञा᳚त्यै।
अ॒ज॒पे॒त्वान् वा ए॒ते पूर्वै॒र्मासै॒रव॑ रुन्धते।
यदे॒ते ग॒व्याः प॒शव॑ आल॒भ्यन्ते᳚।
उ॒भये॑षां पशू॒नामव॑रुद्ध्यै॥४२॥

%1.2.5.4
यदति॑रिक्तामेकाद॒शिनी॑मा॒लभे॑रन्।
अप्रि॑यं॒ भ्रातृ॑व्यम॒भ्यति॑\-रिच्येत।
यद्द्वौ द्वौ॑ प॒शू स॒मस्ये॑युः।
कनी॑य॒ आयुः॑ कुर्वीरन्।
यदे॒ते ब्राह्म॑णवन्तः प॒शव॑ आल॒भ्यन्ते᳚।
नाप्रि॑यं॒ भ्रातृ॑व्यम॒भ्य॑ति॒\-रिच्य॑ते।
न कनी॑य॒ आयुः॑ कुर्वते॥४३॥\anuvakamend[ते ए॒वाल॑भन्ते मैत्रावरु॒णीमाल॑भ॒न्ते\-ऽव॑रुद्ध्यै स॒प्त च॑]

%1.2.6.1
प्र॒जा\-प॑तिः प्र॒जाः सृ॒ष्ट्वा वृ॒त्तो॑\-ऽशयत्।
तं दे॒वा भू॒ताना॒ꣳ॒ रसं॒ तेजः॑ स॒म्भृत्य॑।
तेनै॑नमभिषज्यन्।
म॒हान॑वव॒र्तीति॑।
तन्म॑हाव्र॒तस्य॑ महाव्रत॒त्वम्।
म॒हद्व्र॒तमिति॑।
तन्म॑हाव्र॒तस्य॑ महाव्रत॒त्वम्।
म॒ह॒तो व्र॒तमिति॑।
तन्म॑हाव्र॒तस्य॑ महाव्रत॒त्वम्।
प॒ञ्च॒वि॒ꣳ॒शः स्तोमो॑ भवति॥४४॥

%1.2.6.2
चतु॑र्विꣳशत्यर्धमासः संवथ्स॒रः।
यद्वा ए॒तस्मि᳚न्थ्संवथ्स॒रेऽधि॒ प्राजा॑यत।
तदन्नं॑ पञ्चवि॒ꣳ॒शम॑भवत्।
म॒ध्य॒तः क्रि॑यते।
म॒ध्य॒तो ह्यन्न॑मशि॒तं धि॒नोति॑।
अथो॑ मध्य॒त ए॒व प्र॒जाना॒मूर्ग्धी॑यते।
अथ॒ यद्वा इ॒दम॑न्त॒तः क्रि॒यते᳚।
तस्मा॑दुद॒न्ते प्र॒जाः समे॑धन्ते।
अ॒न्त॒तः क्रि॑यते प्र॒जन॑नायै॒व।
त्रि॒वृच्छिरो॑ भवति॥४५॥

%1.2.6.3
त्रे॒धा॒वि॒हि॒तꣳ हि शिरः॑।
लोम॑ छ॒वीरस्थि॑।
परा॑चा स्तुवन्ति।
तस्मा॒त्तथ्स॒दृगे॒व।
न मेद्य॒तोऽनु॑ मेद्यति।
न कृश्य॒तोऽनु॑ कृश्यति।
प॒ञ्च॒द॒शो᳚\-ऽन्यः प॒क्षो भ॑वति।
स॒प्त॒द॒शो᳚\-ऽन्यः।
तस्मा॒द्वयाꣴ॑स्यन्यत॒रम॒र्धम॒भि प॒र्याव॑र्तन्ते।
अ॒न्य॒त॒रतो॒ हि तद्गरी॑यः क्रि॒यते᳚॥४६॥

%1.2.6.4
प॒ञ्च॒वि॒ꣳ॒श आ॒त्मा भ॑वति।
तस्मा᳚न्मध्य॒तः प॒शवो॒ वरि॑ष्ठाः।
ए॒क॒वि॒ꣳ॒शं पुच्छम्᳚।
द्वि॒पदा॑सु स्तुवन्ति॒ प्रति॑\-ष्ठित्यै।
सर्वे॑ण स॒ह स्तु॑वन्ति।
सर्वे॑ण॒ ह्या᳚त्मना᳚\-ऽऽत्म॒न्वी।
स॒होत्पत॑न्ति।
एकै॑का॒मुच्छिꣳ॑षन्ति।
आ॒त्मन्न् ह्यङ्गा॑नि ब॒द्धानि॑।
न वा ए॒तेन॒ सर्वः॒ पुरु॑षः॥४७॥

%1.2.6.5
यदि॒त इ॑तो॒ लोमा॑नि द॒तो न॒खान्।
प॒रि॒मादः॑ क्रियन्ते।
तान्ये॒व तेन॒ प्रत्यु॑प्यन्ते।
औदु॑म्बर॒स्तल्पो॑ भवति।
ऊर्ग्वा अन्न॑मुदु॒म्बरः॑।
ऊ॒र्ज ए॒वान्नाद्य॒स्या\-व॑\-रुद्ध्यै।
यस्य॑ तल्प॒सद्य॒मन॑भिजित॒ꣴ॒ स्यात्।
स दे॒वाना॒ꣳ॒ साम्य॑क्षे।
त॒ल्प॒सद्य॑म॒भिज॑या॒नीति॒ तल्प॑मा॒रुह्योद्गा॑येत्।
त॒ल्प॒सद्य॑मे॒वाभि ज॑यति॥४८॥

%1.2.6.6
यस्य॑ तल्प॒सद्य॑म॒भिजि॑त॒ꣴ॒ स्यात्।
स दे॒वाना॒ꣳ॒ साम्य॑क्षे।
त॒ल्प॒सद्यं॒ मा परा॑जे॒षीति॒ तल्प॑मा॒रुह्योद्गा॑येत्।
न त॑ल्प॒सद्यं॒ परा॑जयते।
प्ले॒ङ्खे शꣳ॑सति।
महो॒ वै प्ले॒ङ्खः।
मह॑स ए॒वान्नाद्य॒स्या\-व॑\-रुद्ध्यै।
दे॒वा॒सु॒राः संय॑त्ता आसन्।
त आ॑दि॒त्ये व्याय॑च्छन्त।
तं दे॒वाः सम॑जयन्॥४९॥

%1.2.6.7
ब्रा॒ह्म॒णश्च॑ शू॒द्रश्च॑ चर्मक॒र्ते व्याय॑च्छेते।
दैव्यो॒ वै वर्णो᳚ ब्राह्म॒णः।
अ॒सु॒र्यः॑ शू॒द्रः।
इ॒मे॑\-ऽराथ्सुरि॒मे सु॑भू॒तम॑क्र॒न्नित्य॑न्यत॒रो ब्रू॑यात्।
इ॒म उ॑द्वासीका॒रिण॑ इ॒मे दु॑र्भू॒तम॑क्र॒न्नित्य॑न्यत॒रः।
यदे॒वैषाꣳ॑ सुकृ॒तं या राद्धिः॑।
तद॑न्यत॒रो॑ऽभि श्री॑णाति।
यदे॒वैषां᳚ दुष्कृ॒तं याऽरा᳚द्धिः।
तद॑न्यत॒रोऽप॑ हन्ति।
ब्रा॒ह्म॒णः सं ज॑यति।
अ॒मुमे॒वाऽऽदि॒त्यं भ्रातृ॑व्यस्य॒ संवि॑न्दन्ते॥५०॥\anuvakamend[भ॒व॒ति॒ भ॒व॒ति॒ क्रि॒यते॒ पुरु॑षो जयत्यजयञ्जय॒त्येकं॑ च]




\prashnaend{उ॒द्ध॒न्यमा॑नं॒ नवै॒तानि॒ सन्त॑तिरेकवि॒ꣳ॒श ए॒षो\-ऽप्र॑तिष्ठां प्र॒जा\-प॑तिर्वृ॒त्तः षट्॥६॥}{उ॒द्ध॒न्यमा॑नꣳ शो॒चिष्के॒शो\-ऽग्ने॑ स॒पत्ना॑नतिग्रा॒ह्या॑ वैश्वदे॒वमाल॑भन्ते पञ्चा॒शत्॥५०॥}{उ॒द्ध॒न्यमा॑न॒ꣳ॒ संवि॑न्दन्ते॥}{हरिः॑ ओम्॥}{इति श्रीकृष्णयजुर्वेदीयतैत्तिरीयब्राह्मणे प्रथमाष्टके द्वितीयः प्रपाठकः समाप्तः॥}
\clearpage
\sect{तृतीयः प्रश्नः}
\setcounter{anuvakam}{0}
\dnsub{तैत्तिरीयब्राह्मणे प्रथमाष्टके तृतीयः प्रपाठकः}

%1.3.1.1
दे॒वा॒सु॒राः संय॑त्ता आसन्।
ते दे॒वा वि॑ज॒यमु॑प॒यन्तः॑।
अ॒ग्नीषोम॑योस्तेज॒स्विनी᳚स्त॒नूः सन्न्य॑दधत।
इ॒दमु॑ नो भविष्यति।
यदि॑ नो जे॒ष्यन्तीति॑।
तेना॒ग्नीषोमा॒वपा᳚क्रामताम्।
ते दे॒वा वि॒जित्य॑।
अ॒ग्नीषोमा॒वन्वै᳚च्छन्।
ते᳚ऽग्निमन्व॑\-विन्दन्नृ॒तुषूथ्स॑न्नम्।
तस्य॒ विभ॑क्तीभिस्तेज॒स्विनी᳚स्त॒नू\-रवा॑रुन्धत॥१॥

%1.3.1.2
ते सोम॒मन्व॑विन्दन्।
तम॑घ्नन्।
तस्य॑ यथा\-ऽभि॒ज्ञायं॑ त॒नूर्व्य॑गृह्णत।
ते ग्रहा॑ अभवन्।
तद्ग्रहा॑णां ग्रह॒त्वम्।
यस्यै॒वं वि॒दुषो॒ ग्रहा॑ गृ॒ह्यन्ते᳚।
तस्य॒ त्वे॑व गृ॑ही॒ताः।
नाना᳚\-ऽऽग्नेयं पुनरा॒धेये॑ कुर्यात्।
यदना᳚ग्नेयं पुनरा॒धेये॑ कु॒र्यात्।
व्यृ॑द्धमे॒व तत्॥२॥

%1.3.1.3
अना᳚ग्नेयं॒ वा ए॒तत्क्रि॑यते।
यथ्स॒मिध॒स्तनू॒नपा॑तमि॒डो ब॒र्॒\mbox{}हिर्य॑जति।
उ॒भावा᳚ग्ने॒यावाज्य॑\-भागौ स्याताम्।
अना᳚ज्यभागौ भवत॒ इत्या॑हुः।
यदु॒भावा᳚ग्ने॒याव॒न्वञ्चा॒विति॑।
अ॒ग्नये॒ पव॑माना॒योत्त॑रः स्यात्।
यत्पव॑मानाय।
तेनाऽऽज्य॑भागः।
तेन॑ सौ॒म्यः।
बुध॑न्वत्याग्ने॒यस्या\-ऽ\-ऽ\-ज्य॑भागस्य पुरो\-ऽनुवा॒क्या॑ भवति॥३॥

%1.3.1.4
यथा॑ सु॒प्तं बो॒धय॑ति।
ता॒दृगे॒व तत्।
अ॒ग्निन्य॑क्ताः पत्नीसंया॒जाना॒मृचः॑ स्युः।
तेना᳚ऽऽग्ने॒यꣳ सर्वं॑ भवति।
ए॒क॒धा ते॑ज॒स्विनीं᳚ दे॒वता॒मुपै॒तीत्या॑हुः।
सैन॑मीश्व॒रा प्र॒दह॒ इति॑।
नेति॑ ब्रूयात्।
प्र॒जन॑नं॒ वा अ॒ग्निः।
प्र॒जन॑नमे॒वोपै॒तीति॑।
कृ॒तय॑जुः॒ सम्भृ॑तसम्भार॒ इत्या॑हुः॥४॥

%1.3.1.5
न स॒म्भृत्याः᳚ सम्भा॒राः।
न यजुः॑ का॒र्य॑मिति॑।
अथो॒ खलु॑।
स॒म्भृत्या॑ ए॒व स॑म्भा॒राः।
का॒र्यं॑ यजुः॑।
पु॒न॒रा॒धेय॑स्य॒ समृ॑द्ध्यै।
तेनो॑पा॒ꣳ॒शु प्रच॑रति।
एष्य॑ इव॒ वा ए॒षः।
यत्पु॑नरा॒धेयः॑।
यथो॑पा॒ꣳ॒शु न॒ष्टमि॒च्छति॑॥५॥

%1.3.1.6
ता॒दृगे॒व तत्।
उ॒च्चैः स्वि॑ष्ट॒कृत॒मुथ्सृ॑जति।
यथा॑ न॒ष्टं वि॒त्त्वा प्राहा॒यमिति॑।
ता॒दृगे॒व तत्।
ए॒क॒धा ते॑ज॒स्विनीं᳚ दे॒वता॒मुपै॒तीत्या॑हुः।
सैन॑मीश्व॒रा प्र॒दह॒ इति॑।
तत्तथा॒ नोपै॑ति।
प्र॒या॒जा॒नू॒या॒जेष्वे॒व विभ॑क्तीः कुर्यात्।
य॒था॒पू॒र्वमाज्य॑\-भागौ॒ स्याता᳚म्।
ए॒वं प॑त्नीसंया॒जाः॥६॥

%1.3.1.7
तद्वै᳚श्वान॒रव॑त्प्र॒जन॑नवत्तर॒मुपै॒तीति॑।
तदा॑हुः।
व्यृ॑द्धं॒ वा ए॒तत्।
अना᳚ग्नेयं॒ वा ए॒तत्क्रि॑यत॒ इति॑।
नेति॑ ब्रूयात्।
अ॒ग्निं प्र॑थ॒मं विभ॑क्तीनां यजति।
अ॒ग्निमु॑त्त॒मं प॑त्नीसंया॒जाना᳚म्।
तेना᳚ऽऽग्ने॒यम्।
तेन॒ समृ॑द्धं क्रियत॒ इति॑॥७॥\anuvakamend[अ॒रु॒न्ध॒तै॒व तद्भ॑वति॒ सम्भृ॑तसम्भार॒ इत्या॑हुरि॒च्छति॑ पत्नीसंया॒जा नव॑ च]

%1.3.2.1
दे॒वा वै यथा॒दर्\mbox{}शं॑ य॒ज्ञानाह॑रन्त।
यो᳚ऽग्निष्टो॒मम्।
य उ॒क्थ्यम्᳚।
यो॑ऽतिरा॒त्रम्।
ते स॒हैव सर्वे॑ वाज॒पेय॑मपश्यन्।
ते।
अ॒न्यो᳚\-ऽन्यस्मै॒ नाति॑ष्ठन्त।
अ॒हम॒नेन॑ यजा॒ इति॑।
ते᳚ऽब्रुवन्।
आ॒जिम॒स्य धा॑वा॒मेति॑॥८॥

%1.3.2.2
तस्मि॑न्ना॒जिम॑धावन्।
तं बृह॒स्पति॒रुद॑जयत्।
तेना॑यजत।
स स्वारा᳚ज्यमगच्छत्।
तमिन्द्रो᳚\-ऽब्रवीत्।
माम॒नेन॑ याज॒येति॑।
तेनेन्द्र॑मयाजयत्।
सोऽग्रं॑ दे॒वता॑नां॒ पर्यै᳚त्।
अग॑च्छ॒थ्स्वारा᳚ज्यम्।
अति॑ष्ठन्तास्मै॒ ज्यैष्ठ्या॑य॥९॥

%1.3.2.3
य ए॒वं वि॒द्वान् वा॑ज॒पेये॑न॒ यज॑ते।
गच्छ॑ति॒ स्वारा᳚ज्यम्।
अग्रꣳ॑ समा॒नानां॒ पर्ये॑ति।
तिष्ठ॑न्ते\-ऽस्मै॒ ज्यैष्ठ्या॑य।
स वा ए॒ष ब्रा᳚ह्म॒णस्य॑ चै॒व रा॑ज॒न्य॑स्य च य॒ज्ञः।
तं वा ए॒तं वा॑ज॒पेय॒ इत्या॑हुः।
वा॒जाप्यो॒ वा ए॒षः।
वाज॒ꣴ॒ ह्ये॑तेन॑ दे॒वा ऐफ्सन्॑।
सोमो॒ वै वा॑ज॒पेयः॑।
यो वै सोमं॑ वाज॒पेयं॒ वेद॑॥१०॥

%1.3.2.4
वा॒ज्ये॑वैनं॑ पी॒त्वा भ॑वति।
आऽस्य॑ वा॒जी जा॑यते।
अन्नं॒ वै वा॑ज॒पेयः॑।
य ए॒वं वेद॑।
अत्यन्नम्᳚।
आऽस्या᳚न्ना॒दो जा॑यते।
ब्रह्म॒ वै वा॑ज॒पेयः॑।
य ए॒वं वेद॑।
अत्ति॒ ब्रह्म॒णा\-ऽन्नम्᳚।
आऽस्य॑ ब्र॒ह्मा जा॑यते॥११॥

%1.3.2.5
वाग्वै वाज॑स्य प्रस॒वः।
य ए॒वं वेद॑।
क॒रोति॑ वा॒चा वी॒र्यम्᳚।
ऐनं॑ वा॒चा ग॑च्छति।
अपि॑वतीं॒ वाचं॑ वदति।
प्र॒जा\-प॑तिर्दे॒वेभ्यो॑ य॒ज्ञान्व्यादि॑शत्।
स आ॒त्मन्वा॑ज॒पेय॑मधत्त।
तं दे॒वा अ॑ब्रुवन्।
ए॒ष वाव य॒ज्ञः।
यद्वा॑ज॒पेयः॑॥१२॥

%1.3.2.6
अप्ये॒व नोऽत्रा॒स्त्विति॑।
तेभ्य॑ ए॒ता उज्जि॑तीः॒ प्राय॑च्छत्।
ता वा ए॒ता उज्जि॑तयो॒ व्याख्या॑यन्ते।
य॒ज्ञस्य॑ सर्व॒त्वाय॑।
दे॒वता॑ना॒मनि॑र्भागाय।
दे॒वा वै ब्रह्म॑ण॒श्चान्न॑स्य च॒ शम॑ल॒मपा᳚घ्नन्।
यद्ब्रह्म॑णः॒ शम॑ल॒मासी᳚त्।
सा गाथा॑ नाराश॒ꣴ॒स्य॑भवत्।
यदन्न॑स्य।
सा सुरा᳚॥१३॥

%1.3.2.7
तस्मा॒द्गाय॑तश्च म॒त्तस्य॑ च॒ न प्र॑ति॒गृह्यम्᳚।
यत्प्र॑तिगृह्णी॒यात्।
शम॑लं॒ प्रति॑\-गृह्णीयात्।
सर्वा॒ वा ए॒तस्य॒ वाचो\-ऽव॑रुद्धाः।
यो वा॑जपेयया॒जी।
या पृ॑थि॒व्यां याऽग्नौ या र॑थन्त॒रे।
याऽन्तरि॑क्षे॒ या वा॒यौ या वा॑मदे॒व्ये।
या दि॒वि याऽऽदि॒त्ये या बृ॑ह॒ति।
याऽफ्सु यौष॑धीषु॒ या वन॒स्पति॑षु।
तस्मा᳚द्वाजपेयया॒ज्यार्त्वि॑जीनः।
सर्वा॒ ह्य॑स्य॒ वाचो\-ऽव॑रुद्धाः॥१४॥\anuvakamend[धा॒वा॒मेति॒ ज्यैष्ठ्या॑य॒ वेद॑ ब्र॒ह्मा जा॑यते वाज॒पेयः॒ सुरा\-ऽऽर्त्वि॑जीन॒ एकं॑ च]

%1.3.3.1
दे॒वा वै यद॒न्यैर्ग्रहै᳚र्य॒ज्ञस्य॒ नावारु॑न्धत।
तद॑तिग्रा॒ह्यै॑रति॒\-गृह्या\-वा॑रुन्धत।
तद॑तिग्रा॒ह्या॑णामति\-ग्राह्य॒त्वम्।
यद॑तिग्रा॒ह्या॑ गृ॒ह्यन्ते᳚।
यदे॒वान्यैर्ग्रहै᳚र्य॒ज्ञस्य॒ नाव॑ रु॒न्धे।
तदे॒व तैर॑ति॒गृह्या\-व॑ रुन्धे।
पञ्च॑ गृह्यन्ते।
पाङ्क्तो॑ य॒ज्ञः।
यावा॑ने॒व य॒ज्ञः।
तमा॒प्त्वा\-ऽव॑ रुन्धे॥१५॥

%1.3.3.2
सर्व॑ ऐ॒न्द्रा भ॑वन्ति।
ए॒क॒धैव यज॑मान इन्द्रि॒यं द॑धति।
स॒प्तद॑श प्राजाप॒त्या ग्रहा॑ गृह्यन्ते।
स॒प्त॒द॒शः प्र॒जा\-प॑तिः।
प्र॒जा\-प॑ते॒राप्त्यै᳚।
एक॑य॒र्चा गृ॑ह्णाति।
ए॒क॒धैव यज॑माने वी॒र्यं॑ दधाति।
सो॒म॒ग्र॒हाꣴश्च॑ सुराग्र॒हाꣴश्च॑ गृह्णाति।
ए॒तद्वै दे॒वानां᳚ पर॒ममन्नम्᳚।
यथ्सोमः॑॥१६॥

%1.3.3.3
ए॒तन्म॑नु॒ष्या॑णाम्।
यथ्सुरा᳚।
प॒र॒मेणै॒वास्मा॑ अ॒न्नाद्ये॒नाव॑र\-म॒न्नाद्य॒मव॑ रुन्धे।
सो॒म॒ग्र॒हान्गृ॑ह्णाति।
ब्रह्म॑णो॒ वा ए॒तत्तेजः॑।
यथ्सोमः॑।
ब्रह्म॑ण ए॒व तेज॑सा॒ तेजो॒ यज॑माने दधाति।
सु॒रा॒ग्र॒हान्गृ॑ह्णाति।
अन्न॑स्य॒ वा ए॒तच्छम॑लम्।
यथ्सुरा᳚॥१७॥

%1.3.3.4
अन्न॑स्यै॒व शम॑लेन॒ शम॑लं॒ यज॑माना॒दप॑हन्ति।
सो॒म॒ग्र॒हाꣴश्च॑ सुराग्र॒हाꣴश्च॑ गृह्णाति।
पुमा॒न्॒ वै सोमः॑।
स्त्री सुरा᳚।
तन्मि॑थु॒नम्।
मि॒थु॒नमे॒वास्य॒ तद्य॒ज्ञे क॑रोति प्र॒जन॑नाय।
आ॒त्मान॑मे॒व सो॑मग्र॒हैः स्पृ॑णोति।
जा॒याꣳ सु॑राग्र॒हैः।
तस्मा᳚द्वाजपेयया॒ज्य॑मुष्मिँ॑ल्लो॒के स्त्रिय॒ꣳ॒ सम्भ॑वति।
वा॒ज॒पेया॑भिजित॒ꣴ॒ ह्य॑स्य॥१८॥

%1.3.3.5
पूर्वे॑ सोमग्र॒हा गृ॑ह्यन्ते।
अप॑रे सुराग्र॒हाः।
पु॒रो॒\-ऽक्षꣳ सो॑मग्र॒हान्थ्सा॑दयति।
प॒श्चा॒द॒क्षꣳ सु॑राग्र॒हान्।
पा॒प॒व॒स्य॒सस्य॒ विधृ॑त्यै।
ए॒ष वै यज॑मानः।
यथ्सोमः॑।
अन्न॒ꣳ॒ सुरा᳚।
सो॒म॒ग्र॒हाꣴश्च॑ सुराग्र॒हाꣴश्च॒ व्यति॑षजति।
अ॒न्नाद्ये॑नै॒वैनं॒ व्यति॑षजति॥१९॥

%1.3.3.6
स॒म्पृचः॑ स्थ॒ सं मा॑ भ॒द्रेण॑ पृ॒ङ्क्तेत्या॑ह।
अन्नं॒ वै भ॒द्रम्।
अ॒न्ना\-द्ये॑नै॒वैन॒ꣳ॒ सꣳसृ॑जति।
अन्न॑स्य॒ वा ए॒तच्छम॑लम्।
यथ्सुरा᳚।
पा॒प्मेव॒ खलु॒ वै शम॑लम्।
पा॒प्मना॒ वा ए॑नमे॒तच्छम॑लेन॒ व्यति॑षजति।
यथ्सो॑मग्र॒हाꣴश्च॑ सुराग्र॒हाꣴश्च॑ व्यति॒षज॑ति।
वि॒पृचः॑ स्थ॒ वि मा॑ पा॒प्मना॑ पृ॒ङ्क्तेत्या॑ह।
पा॒प्मनै॒वैन॒ꣳ॒ शम॑लेन॒ व्याव॑र्तयति॥२०॥

%1.3.3.7
तस्मा᳚द्वाजपेयया॒जी पू॒तो मेध्यो॑ दक्षि॒ण्यः॑।
प्राङुद्द्र॑वति सोम\-ग्र॒हैः।
अ॒मुमे॒व तैर्लो॒क\-म॒भि\-ज॑यति।
प्र॒त्यङ्ख्सु॑राग्र॒हैः।
इ॒ममे॒व तैर्लो॒कम॒भि\-ज॑यति।
प्रति॑\-ष्ठन्ति सोमग्र॒हैः।
याव॑दे॒व स॒त्यम्।
तेन॑ सूयते।
वा॒ज॒सृद्भ्यः॑ सुराग्र॒हान् ह॑रन्ति।
अनृ॑तेनै॒व विश॒ꣳ॒ सꣳ\-सृ॑जति।
हि॒र॒ण्य॒पा॒त्रं मधोः᳚ पू॒र्णं द॑दाति।
म॒ध॒व्यो॑\-ऽसा॒नीति॑।
ए॒क॒धा ब्र॒ह्मण॒ उप॑ हरति।
ए॒क॒धैव यज॑मान॒ आयु॒स्तेजो॑ दधाति॥२१॥\anuvakamend[आ॒प्त्वा\-ऽव॑ रुन्धे॒ सोमः॒ शम॑लं॒ यथ्सुरा॒ ह्य॑स्यैनं॒ व्यति॑षजति॒ व्याव॑र्तयति \-सृजति च॒त्वारि॑ च]

%1.3.4.1
ब्र॒ह्म॒वा॒दिनो॑ वदन्ति।
नाग्नि॑ष्टो॒मो नोक्थ्यः॑।
न षो॑ड॒शी नाति॑\-रा॒त्रः।
अथ॒ कस्मा᳚द्वाज॒पेये॒ सर्वे॑ यज्ञक्र॒तवो\-ऽव॑रुध्यन्त॒ इति॑।
प॒शुभि॒\-रिति॑ ब्रूयात्।
आ॒ग्ने॒यं प॒शुमाल॑भते।
अ॒ग्नि॒ष्टो॒ममे॒व तेनाव॑ रुन्धे।
ऐ॒न्द्रा॒ग्नेनो॒क्थ्यम्᳚।
ऐ॒न्द्रेण॑ षोड॒शिनः॑ स्तो॒त्रम्।
सा॒र॒स्व॒त्या\-ऽति॑रा॒त्रम्॥२२॥

%1.3.4.2
मा॒रु॒त्या बृ॑ह॒तः स्तो॒त्रम्।
ए॒ताव॑न्तो॒ वै य॑ज्ञक्र॒तवः॑।
तान्प॒शुभि॑रे॒वाव॑ रुन्धे।
आ॒त्मान॑मे॒व स्पृ॑णोत्यग्निष्टो॒मेन॑।
प्रा॒णा॒पा॒नावु॒क्थ्ये॑न।
वी॒र्यꣳ॑ षोड॒शिनः॑ स्तो॒त्रेण॑।
वाच॑मतिरा॒त्रेण॑।
प्र॒जां बृ॑ह॒तः स्तो॒त्रेण॑।
इ॒ममे॒व लो॒कम॒भिज॑यत्यग्निष्टो॒मेन॑।
अ॒न्तरि॑क्षमु॒क्थ्ये॑न॥२३॥

%1.3.4.3
सु॒व॒र्गं लो॒कꣳ षो॑ड॒शिनः॑ स्तो॒त्रेण॑।
दे॒व॒याना॑ने॒व प॒थ आरो॑हत्यतिरा॒त्रेण॑।
नाकꣳ॑ रोहति बृह॒तः स्तो॒त्रेण॑।
तेज॑ ए॒वाऽऽत्मन्ध॑त्त आग्ने॒येन॑ प॒शुना᳚।
ओजो॒ बल॑मैन्द्रा॒ग्नेन॑।
इ॒न्द्रि॒यमै॒न्द्रेण॑।
वाचꣳ॑ सारस्व॒त्या।
उ॒भावे॒व दे॑वलो॒कं च॑ मनुष्यलो॒कं चा॒भि\-ज॑यति मारु॒त्या व॒शया᳚।
स॒प्तद॑श प्राजाप॒त्यान्प॒शूनाल॑भते।
स॒प्त॒द॒शः प्र॒जा\-प॑तिः॥२४॥

%1.3.4.4
प्र॒जा\-प॑ते॒राप्त्यै᳚।
श्या॒मा एक॑रूपा भवन्ति।
ए॒वमि॑व॒ हि प्र॒जा\-प॑तिः॒ समृ॑द्ध्यै।
तान्पर्य॑ग्निकृता॒नुथ्सृ॑जति।
म॒रुतो॑ य॒ज्ञ\-म॑जिघाꣳ\-सन्प्र॒जा\-प॑तेः।
तेभ्य॑ ए॒तां मा॑रु॒तीं व॒शामाल॑भत।
तयै॒वैना॑नशमयत्।
मा॒रु॒त्या प्र॒चर्य॑।
ए॒तान्थ्संज्ञ॑पयेत्।
म॒रुत॑ ए॒व श॑मयि॒त्वा॥२५॥

%1.3.4.5
ए॒तैः प्रच॑रति।
य॒ज्ञस्याघा॑ताय।
ए॒क॒धा व॒पा जु॑होति।
ए॒क॒दे॒व॒त्या॑ हि।
ए॒ते।
अथो॑ एक॒धैव यज॑माने वी॒र्यं॑ दधाति।
नै॒वा॒रेण॑ स॒प्तद॑शशरावेणै॒तर्\mbox{}हि॒ प्रच॑रति।
ए॒तत्पु॑रोडाशा॒ ह्ये॑ते।
अथो॑ पशू॒नामे॒व छि॒द्रमपि॑दधाति।
सा॒र॒स्व॒त्योत्त॒मया॒ प्रच॑रति।
वाग्वै सर॑स्वती।
तस्मा᳚त्प्रा॒णानां॒ वागु॑त्त॒मा।
अथो᳚ प्र॒जा\-प॑तावे॒व य॒ज्ञं प्रति॑\-ष्ठापयति।
प्र॒जा\-प॑ति॒र्‌॒\mbox{}हि वाक्।
अप॑न्नदती भवति।
तस्मा᳚न्मनु॒ष्याः᳚ सर्वां॒ वाचं॑ वदन्ति॥२६॥\anuvakamend[अ॒ति॒रा॒त्रम॒न्तरि॑क्षमु॒क्थ्ये॑न प्र॒जा\-प॑तिः शमयि॒त्वोत्त॒मया॒ प्रच॑रति॒ षट् च॑]

%1.3.5.1
सा॒वि॒त्रं जु॑होति॒ कर्म॑णः कर्मणः पु॒रस्ता᳚त्।
कस्तद्वे॒देत्या॑हुः।
यद्वा॑ज॒पेय॑स्य॒ पूर्वं॒ यदप॑र॒मिति॑।
स॒वि॒तृप्र॑सूत ए॒व य॑थापू॒र्वं कर्मा॑णि करोति।
सव॑नेसवने जुहोति।
आ॒क्रम॑णमे॒व तथ्सेतुं॒ यज॑मानः कुरुते।
सु॒व॒र्गस्य॑ लो॒कस्य॒ सम॑ष्ट्यै।
वा॒चस्पति॒र्वाच॑म॒द्य स्व॑दाति न॒ इत्या॑ह।
वाग्वै दे॒वानां᳚ पु॒रा\-ऽन्न॑मासीत्।
वाच॑मे॒वास्मा॒ अन्नꣴ॑ स्वदयति॥२७॥

%1.3.5.2
इन्द्र॑स्य॒ वज्रो॑ऽसि॒ वार्त्र॑घ्न॒ इति॒ रथ॑मु॒पाव॑\-हरति॒ विजि॑त्यै।
वाज॑स्य॒ नु प्र॑स॒वे मा॒तरं॑ म॒हीमित्या॑ह।
यच्चै॒वेयम्।
यच्चा॒स्यामधि॑।
तदे॒वाव॑ रुन्धे।
अथो॒ तस्मि॑न्ने॒वोभये॒\-ऽभि\-षि॑च्यते।
अ॒फ्स्व॑न्तर॒मृत॑\-म॒फ्सु भे॑ष॒जमित्यश्वा᳚न्पल्पूलयति।
अ॒फ्सु वा अश्व॑स्य॒ तृती॑यं॒ प्रवि॑ष्टम्।
तद॑नु॒वेन॒न्वव॑प्लवते।
यद॒फ्सु प॑ल्पू॒लय॑ति॥२८॥

%1.3.5.3
यदे॒वास्या॒फ्सु प्रवि॑ष्टम्।
तदे॒वाव॑ रुन्धे।
ब॒हु वा अश्वो॑\-ऽमे॒ध्यमुप॑\-गच्छति।
यद॒फ्सु प॑ल्पू॒लय॑ति।
मेध्या॑ने॒वै\-ना᳚न्करोति।
वा॒युर्वा᳚ त्वा॒ मनु॑र्वा॒ त्वेत्या॑ह।
ए॒ता वा ए॒तं दे॒वता॒ अग्रे॒ अश्व॑मयुञ्जन्।
ताभि॑रे॒वैनान्॑ युनक्ति।
स॒वस्योज्जि॑त्यै।
यजु॑षा युनक्ति॒ व्यावृ॑त्त्यै॥२९॥

%1.3.5.4
अपा᳚न्नपादाशुहेम॒न्निति॒ सम्मा᳚र्ष्टि।
मेध्या॑ने॒वैना᳚न्करोति।
अथो॒ स्तौत्ये॒वैना॑ना॒जिꣳ स॑रिष्य॒तः।
वि॒ष्णु॒क्र॒मान्क्र॑मते।
विष्णु॑रे॒व भू॒त्वेमाँल्लो॒कान॒भि\-ज॑यति।
वै॒श्व॒दे॒वो वै रथः॑।
अ॒ङ्कौ न्य॒ङ्काव॒भितो॒ रथं॒ यावित्या॑ह।
या ए॒व दे॒वता॒ रथे॒ प्रवि॑ष्टाः।
ताभ्य॑ ए॒व नम॑स्करोति।
आ॒त्मनो\-ऽना᳚र्त्यै।
अश॑मरथं भावुको\-ऽस्य॒ रथो॑ भवति।
य ए॒वं वेद॑॥३०॥\anuvakamend[स्व॒द॒य॒ति॒ प॒ल्पू॒लय॑ति॒ व्यावृ॑त्त्या॒ अना᳚र्त्यै॒ द्वे च॑]

%1.3.6.1
दे॒वस्या॒हꣳ स॑वि॒तुः प्र॑स॒वे बृह॒स्पति॑ना वाज॒जिता॒ वाजं॑ जेष॒मित्या॑ह।
स॒वि॒तृप्र॑सूत ए॒व ब्रह्म॑णा॒ वाज॒मुज्ज॑यति।
दे॒वस्या॒हꣳ स॑वि॒तुः प्र॑स॒वे बृह॒स्पति॑ना वाज॒जिता॒ वर्\mbox{}षि॑ष्ठं॒ नाकꣳ॑ रुहेय॒मित्या॑ह।
स॒वि॒तृप्र॑सूत ए॒व ब्रह्म॑णा॒ वर्\mbox{}षि॑ष्ठं॒ नाकꣳ॑ रोहति।
चात्वा॑ले रथच॒क्रं निमि॑तꣳ रोहति।
अतो॒ वा अङ्गि॑रस उत्त॒माः सु॑व॒र्गं लो॒कमा॑यन्।
सा॒क्षादे॒व यज॑मानः सुव॒र्गं लो॒कमे॑ति।
आवे᳚ष्टयति।
वज्रो॒ वै रथः॑।
वज्रे॑णै॒व दिशो॒\-ऽभि\-ज॑यति॥३१॥

%1.3.6.2
वा॒जिना॒ꣳ॒ साम॑ गायते।
अन्नं॒ वै वाजः॑।
अन्न॑मे॒वाव॑ रुन्धे।
वा॒चो वर्ष्म॑ दे॒वेभ्यो\-ऽपा᳚क्रामत्।
तद्वन॒स्पती॒न्प्रावि॑शत्।
सैषा वाग्वन॒स्पति॑षु वदति।
या दु॑न्दु॒भौ।
तस्मा᳚द्दुन्दु॒भिः सर्वा॒ वाचो\-ऽति॑वदति।
दु॒न्दु॒भीन्थ्स॒माघ्न॑न्ति।
प॒र॒मा वा ए॒षा वाक्॥३२॥

%1.3.6.3
या दु॑न्दु॒भौ।
प॒र॒मयै॒व वा॒चा\-ऽव॑रां॒ वाच॒म॑व रुन्धे।
अथो॑ वा॒च ए॒व वर्ष्म॒ यज॑मा॒नो\-ऽव॑ रुन्धे।
इन्द्रा॑य॒ वाचं॑ वद॒तेन्द्रं॒ वाजं॑ जापय॒तेन्द्रो॒ वाज॑मजयि॒दित्या॑ह।
ए॒ष वा ए॒तर्\mbox{}हीन्द्रः॑।
यो यज॑ते।
यज॑मान ए॒व वाज॒मुज्ज॑यति।
स॒प्तद॑श प्रव्या॒धाना॒जिं धा॑वन्ति।
स॒प्त॒द॒शꣴ स्तो॒त्रं भ॑वति।
स॒प्तद॑शसप्तदश दीयन्ते॥३३॥

%1.3.6.4
स॒प्त॒द॒शः प्र॒जा\-प॑तिः।
प्र॒जा॑पते॒राप्त्यै᳚।
अर्वा॑ऽसि॒ सप्ति॑रसि वा॒ज्य॑सीत्या॑ह।
अ॒ग्निर्वा अर्वा᳚।
वा॒युः सप्तिः॑।
आ॒दि॒त्यो वा॒जी।
ए॒ताभि॑रे॒वास्मै॑ दे॒वता॑भिर्देवर॒थं यु॑नक्ति।
प्र॒ष्टि॒वा॒हिनं॑ युनक्ति।
प्र॒ष्टि॒वा॒ही वै दे॑वर॒थः।
दे॒व॒र॒थमे॒वास्मै॑ युनक्ति॥३४॥

%1.3.6.5
वाजि॑नो॒ वाजं॑ धावत॒ काष्ठां᳚ गच्छ॒तेत्या॑ह।
सु॒व॒र्गो वै लो॒कः काष्ठा᳚।
सु॒व॒र्गमे॒व लो॒कं य॑न्ति।
सु॒व॒र्गं वा ए॒ते लो॒कं य॑न्ति।
य आ॒जिं धाव॑न्ति।
प्राञ्चो॑ धावन्ति।
प्राङि॑व॒ हि सु॑व॒र्गो लो॒कः।
च॒त॒सृभि॒रनु॑\- मन्त्रयते।
च॒त्वारि॒ छन्दाꣳ॑सि।
छन्दो॑भिरे॒वैना᳚न्थ्सुव॒र्गं लो॒कं ग॑मयति॥३५॥

%1.3.6.6
प्र वा ए॒ते᳚\-ऽस्माल्लो॒काच्च्य॑वन्ते।
य आ॒जिं धाव॑न्ति।
उदं॑ च॒ आव॑र्तन्ते।
अ॒स्मादे॒व तेन॑ लो॒कान्नय॑न्ति।
र॒थ॒वि॒मो॒च॒नीयं॑ जुहोति॒ प्रति॑\-ष्ठित्यै।
आ मा॒ वाज॑स्य प्रस॒वो ज॑गम्या॒दित्या॑ह।
अन्नं॒ वै वाजः॑।
अन्न॑मे॒वाव॑ रुन्धे।
य॒था॒लो॒कं वा ए॒त उज्ज॑यन्ति।
य आ॒जिं धाव॑न्ति॥३६॥

%1.3.6.7
कृ॒ष्णलं॑ कृष्णलं वाज॒सृद्भ्यः॒ प्रय॑च्छति।
यमे॒व ते वाजं॑ लो॒कमु॒ज्जय॑न्ति।
तं प॑रि॒क्रीयाव॑ रुन्धे।
ए॒क॒धा ब्र॒ह्मण॒ उप॑हरति।
ए॒क॒धैव यज॑माने वी॒र्यं॑ दधाति।
दे॒वा वा ओष॑धीष्वा॒जिम॑युः।
ता बृह॒स्पति॒रुद॑जयत्।
स नी॒वारा॒न्निर॑वृणीत।
तन्नी॒वारा॑णां नीवार॒त्वम्।
नै॒वा॒रश्च॒रुर्भ॑वति॥३७॥

%1.3.6.8
ए॒तद्वै दे॒वानां᳚ पर॒ममन्नम्᳚।
यन्नी॒वाराः᳚।
प॒र॒मेणै॒वास्मा॑ अ॒न्नाद्ये॒नाव॑रम॒न्नाद्य॒मव॑ रुन्धे।
स॒प्तद॑शशरावो भवति।
स॒प्त॒द॒शः प्र॒जा\-प॑तिः।
प्र॒जा\-प॑ते॒राप्त्ये᳚।
क्षी॒रे भ॑वति।
रुच॑मे॒वास्मि॑न्दधाति।
स॒र्पिष्वा᳚न्भवति मेध्य॒त्वाय॑।
बा॒र्॒ह॒स्प॒त्यो वा ए॒ष दे॒वत॑या॥३८॥

%1.3.6.9
यो वा॑ज॒पेये॑न॒ यज॑ते।
बा॒र्॒ह॒स्प॒त्य ए॒ष च॒रुः।
अश्वा᳚न्थ्सरिष्य॒तः स॒स्रुष॒श्चाव॑ घ्रापयति।
यमे॒व ते वाजं॑ लो॒कमु॒ज्जय॑न्ति।
तमे॒वाव॑ रुन्धे।
अजी॑जिपत वनस्पतय॒ इन्द्रं॒ वाजं॒ विमु॑च्यध्व॒मिति॑ दुन्दु॒भीन् विमु॑ञ्चति।
यमे॒व ते वाजं॑ लो॒कमि॑न्द्रि॒यं दु॑न्दु॒भय॑ उ॒ज्जय॑न्ति।
तमे॒वाव॑ रुन्धे॥३९॥\anuvakamend[अ॒भि\-ज॑यति॒ वा ए॒षा वाग्दी॑यन्ते\-ऽस्मै युनक्ति गमयति॒ य आ॒जिं धाव॑न्ति भवति दे॒वत॑या॒\-ऽष्टौ च॑]

%1.3.7.1
ता॒र्प्यं यज॑मानं॒ परि॑धापयति।
य॒ज्ञो वै ता॒र्प्यम्।
य॒ज्ञेनै॒वैन॒ꣳ॒ सम॑र्धयति।
द॒र्भ॒मयं॒ परि॑धापयति।
प॒वित्रं॒ वै द॒र्भाः।
पु॒नात्ये॒वैनम्᳚।
वाजं॒ वा ए॒षो\-ऽव॑रुरुथ्सते।
यो वा॑ज॒पेये॑न॒ यज॑ते।
ओष॑धयः॒ खलु॒ वै वाजः॑।
यद्द॑र्भ॒मयं॑ परिधा॒पय॑ति॥४०॥

%1.3.7.2
वाज॒स्या\-व॑\-रुद्ध्यै।
जाय॒ एहि॒ सुवो॒ रोहा॒वेत्या॑ह।
पत्नि॑या ए॒वैष य॒ज्ञस्या᳚न्वार॒म्भो\-ऽन॑वच्छित्त्यै।
स॒प्तद॑शारत्नि॒र्यूपो॑ भवति।
स॒प्त॒द॒शः प्र॒जा\-प॑तिः।
प्र॒जा\-प॑ते॒राप्त्यै᳚।
तू॒प॒रश्चतु॑रश्रिर्भवति।
गौ॒धू॒मं च॒षालम्᳚।
न वा ए॒ते व्री॒हयो॒ न यवाः᳚।
यद्गो॒धूमाः᳚॥४१॥

%1.3.7.3
ए॒वमि॑व॒ हि प्र॒जा\-प॑तिः॒ समृ॑द्ध्यै।
अथो॑ अ॒मुमे॒वास्मै॑ लो॒कमन्न॑वन्तं करोति।
वासो॑भिर्वेष्टयति।
ए॒ष वै यज॑मानः।
यद्यूपः॑।
स॒र्व॒दे॒व॒त्यं॑ वासः॑।
सर्वा॑भिरे॒वैनं॑ दे॒वता॑भिः॒ सम॑र्धयति।
अथो॑ आ॒क्रम॑णमे॒व तथ्सेतुं॒ यज॑मानः कुरुते।
सु॒व॒र्गस्य॑ लो॒कस्य॒ सम॑ष्ट्यै।
द्वाद॑श वाजप्रस॒वीया॑नि जुहोति॥४२॥

%1.3.7.4
द्वाद॑श॒ मासाः᳚ संवथ्स॒रः।
सं॒व॒थ्स॒रमे॒व प्री॑णाति।
अथो॑ संवथ्स॒र\-मे॒वास्मा॒ उप॑दधाति।
सु॒व॒र्गस्य॑ लो॒कस्य॒ सम॑ष्ट्यै।
द॒शभिः॒ कल्पै॑ रोहति।
नव॒ वै पुरु॑षे प्रा॒णाः।
नाभि॑र्दश॒मी।
प्रा॒णाने॒व य॑था\-स्था॒नं क॑ल्पयि॒त्वा।
सु॒व॒र्गं लो॒कमे॑ति।
ए॒ताव॒द्वै पुरु॑षस्य॒ स्वम्॥४३॥

%1.3.7.5
याव॑त्प्रा॒णाः।
याव॑दे॒वास्यास्ति॑।
तेन॑ स॒ह सु॑व॒र्गं लो॒कमे॑ति।
सुव॑र्दे॒वाꣳ अ॑ग॒न्मेत्या॑ह।
सु॒व॒र्गमे॒व लो॒कमे॑ति।
अ॒मृता॑ अभू॒मेत्या॑ह।
अ॒मृत॑मिव॒ हि सु॑व॒र्गो लो॒कः।
प्र॒जा\-प॑तेः प्र॒जा अ॑भू॒मेत्या॑ह।
प्रा॒जा॒प॒त्यो वा अ॒यं लो॒कः।
अ॒स्मादे॒व तेन॑ लो॒कान्नैति॑॥४४॥

%1.3.7.6
सम॒हं प्र॒जया॒ सं मया᳚ प्र॒जेत्या॑ह।
आ॒\-मे॒वैतामा शा᳚स्ते।
आ॒स॒पु॒टैर्घ्न॑न्ति।
अन्नं॒ वा इ॒यम्।
अ॒न्नाद्ये॑नै॒वैन॒ꣳ॒ सम॑र्धयन्ति।
ऊषै᳚र्घ्नन्ति।
ए॒ते हि सा॒क्षादन्नम्᳚।
यदूषाः᳚।
सा॒क्षादे॒वैन॑म॒न्नाद्ये॑न॒ सम॑र्धयन्ति।
पु॒रस्ता᳚त्प्र॒त्यञ्चं᳚ घ्नन्ति॥४५॥

%1.3.7.7
पु॒रस्ता॒द्धि प्र॑ती॒चीन॒मन्न॑म॒द्यते᳚।
शी॒र्॒ष॒तो घ्न॑न्ति।
शी॒र्॒ष॒तो ह्यन्न॑म॒द्यते᳚।
दि॒ग्भ्यो घ्न॑न्ति।
दि॒ग्भ्य ए॒वास्मा॑ अ॒न्नाद्य॒मव॑रुन्धते।
ई॒श्व॒रो वा ए॒ष परा᳚ङ्प्र॒दघः॑।
यो यूप॒ꣳ॒ रोह॑ति।
हिर॑ण्यम॒ध्यव॑रोहति।
अ॒मृतं॒ वै हिर॑ण्यम्।
अ॒मृतꣳ॑ सुव॒र्गो लो॒कः॥४६॥

%1.3.7.8
अ॒मृत॑ ए॒व सु॑व॒र्गे लो॒के प्रति॑ तिष्ठति।
श॒तमा॑नं भवति।
श॒तायुः॒ पुरु॑षः श॒तेन्द्रि॑यः।
आयु॑ष्ये॒वेन्द्रि॒ये प्रति॑ तिष्ठति।
पुष्ट्यै॒ वा ए॒तद्रू॒पम्।
यद॒जा।
त्रिः सं॑वथ्स॒रस्या॒न्यान्प॒शून्परि॒ प्रजा॑यते।
ब॒स्ता॒जि॒नम॒ध्यव॑ रोहति।
पुष्ट्या॑मे॒व प्र॒जन॑ने॒ प्रति॑ तिष्ठति॥४७॥\anuvakamend[प॒रि॒धा॒पय॑ति गो॒धूमा॑ जुहोति॒ स्वं नैति॑ प्र॒त्यञ्चं᳚ घ्नन्ति लो॒को नव॑ च]

%1.3.8.1
स॒प्तान्न॑हो॒माञ्जु॑होति।
स॒प्त वा अन्ना॑नि।
याव॑न्त्ये॒वान्ना॑नि।
तान्ये॒वाव॑ रुन्धे।
स॒प्त ग्रा॒म्या ओष॑धयः।
स॒प्तार॒ण्याः।
उ॒भयी॑षा॒मव॑रुद्ध्यै।
अन्न॑स्यान्नस्य जुहोति।
अन्न॑स्यान्न॒स्या\-व॑रुद्ध्यै।
यद्वा॑जपेयया॒ज्यन॑वरुद्धस्याश्ञी॒यात्॥४८॥

%1.3.8.2
अव॑रुद्धेन॒ व्यृ॑द्ध्येत।
सर्व॑स्य समव॒दाय॑ जुहोति।
अन॑वरुद्ध॒स्या\-व॑\-रुद्ध्यै।
औदु॑म्बरेण स्रु॒वेण॑ जुहोति।
ऊर्ग्वा अन्न॑मुदु॒म्बरः॑।
ऊ॒र्ज ए॒वान्नाद्य॒स्या\-व॑\-रुद्ध्यै।
दे॒वस्य॑ त्वा सवि॒तुः प्र॑स॒व इत्या॑ह।
स॒वि॒तृप्र॑सूत ए॒वैनं॒ ब्रह्म॑णा दे॒वता॑भिर॒भिषि॑ञ्चति।
अन्न॑स्यान्नस्या॒भिषि॑ञ्चति।
अन्न॑स्यान्न॒स्या\-व॑\-रुद्ध्यै॥४९॥

%1.3.8.3
पु॒रस्ता᳚त्प्र॒त्यञ्च॑म॒भिषि॑ञ्चति।
पु॒रस्ता॒द्धि प्र॑ती॒चीन॒मन्न॑म॒द्यते᳚।
शी॒र्॒ष॒तो॑\-ऽभिषि॑ञ्चति।
शी॒र्॒ष॒तो ह्यन्न॑म॒द्यते᳚।
आ मुखा॑द॒न्वव॑\-स्रावयति।
मु॒ख॒त ए॒वास्मा॑ अ॒न्नाद्यं॑ दधाति।
अ॒ग्नेस्त्वा॒ साम्रा᳚ज्येना॒\-भिषि॑ञ्चा॒मी\-त्या॑ह।
ए॒ष वा अ॒ग्नेः स॒वः।
तेनै॒वैन॑म॒भि\-षि॑ञ्चति।
इन्द्र॑स्य त्वा॒ साम्रा᳚ज्येना॒\-भिषि॑ञ्चा॒मी\-त्या॑ह॥५०॥

%1.3.8.4
इ॒न्द्रि॒यमे॒वास्मि॑न्ने॒तेन॑ दधाति।
बृह॒स्पते᳚स्त्वा॒ साम्रा᳚ज्येना॒भि\-षि॑ञ्चा॒मीत्या॑ह।
ब्रह्म॒ वै दे॒वानां॒ बृह॒स्पतिः॑।
ब्रह्म॑णै॒वैन॑म॒भि\-षि॑ञ्चति।
सो॒म॒ग्र॒हाꣴश्चा॑वदानी॒यानि॑ च॒र्त्विग्भ्य॒ उप॑हरन्ति।
अ॒मुमे॒व तैर्लो॒कमन्न॑वन्तं करोति।
सु॒रा॒ग्र॒हाꣴश्चा॑नवदानी॒\-यानि॑ च वाज॒सृद्भ्यः॑।
इ॒ममे॒व तैर्लो॒कमन्न॑वन्तं करोति।
अथो॑ उ॒भयी᳚ष्वे॒वाभिषि॑च्यते।
वि॒मा॒थं कु॑र्वते वाज॒सृतः॑॥५१॥

%1.3.8.5
इ॒न्द्रि॒यस्या\-व॑\-रुद्ध्यै।
अनि॑रुक्ताभिः प्रातः सव॒ने स्तु॑वते।
अनि॑रुक्तः प्र॒जा\-प॑तिः।
प्र॒जा\-प॑ते॒राप्त्यै᳚।
वाज॑वतीभि॒र्माध्यं॑ दिने।
अन्नं॒ वै वाजः॑।
अन्न॑मे॒वाव॑ रुन्धे।
शि॒पि॒वि॒ष्ट\-व॑तीभिस्तृतीयसव॒ने।
य॒ज्ञो वै विष्णुः॑।
प॒शवः॒ शिपिः॑।
य॒ज्ञ ए॒व प॒शुषु॒ प्रति॑ तिष्ठति।
बृ॒हदन्त्यं॑ भवति।
अन्त॑मे॒वैनꣴ॑ श्रि॒यै ग॑मयति॥५२॥\anuvakamend[अ॒श्ञी॒यादन्न॑स्यान्न॒स्याव॑रुद्ध्या॒ इन्द्र॑स्य त्वा॒ साम्रा᳚ज्येना॒भिषि॑ञ्चा॒मीत्या॑ह वाज॒सृतः॒ शिपि॒स्त्रीणि॑ च]

%1.3.9.1
नृ॒षदं॒ त्वेत्या॑ह।
प्र॒जा वै नॄन्।
प्र॒जाना॑मे॒वैतेन॑ सूयते।
द्रु॒षद॒मित्या॑ह।
वन॒स्पत॑यो॒ वै द्रु।
वन॒स्पती॑नामे॒वैतेन॑ सूयते।
भु॒व॒न॒सद॒मित्या॑ह।
य॒दा वै वसी॑या॒न्भव॑ति।
भुव॑नमग॒न्निति॒ वै तमा॑हुः।
भुव॑नमे॒वैतेन॑ गच्छति॥५३॥

%1.3.9.2
अ॒फ्सु॒षदं॑ त्वा घृत॒सद॒मित्या॑ह।
अ॒पामे॒वैतेन॑ घृ॒तस्य॑ सूयते।
व्यो॒म॒सद॒मित्या॑ह।
य॒दा वै वसी॑या॒न्भव॑ति।
व्यो॑माग॒न्निति॒ वै तमा॑हुः।
व्यो॑मै॒वैतेन॑ गच्छति।
पृ॒थि॒वि॒षदं॑ त्वा\-ऽन्तरिक्ष॒सद॒मित्या॑ह।
ए॒षामे॒वैतेन॑ लो॒कानाꣳ॑ सूयते।
तस्मा᳚द्वाजपेयया॒जी न कञ्च॒न प्र॒त्यव॑रोहति।
अपी॑व॒ हि दे॒वता॑नाꣳ सू॒यते᳚॥५४॥

%1.3.9.3
ना॒क॒सद॒मित्या॑ह।
य॒दा वै वसी॑या॒न्भव॑ति।
नाक॑मग॒न्निति॒ वै तमा॑हुः।
नाक॑मे॒वैतेन॑ गच्छति।
ये ग्रहाः᳚ पञ्चज॒नीना॒ इत्या॑ह।
प॒ञ्च॒ज॒नाना॑मे॒वैतेन॑ सूयते।
अ॒पाꣳ रस॒मुद्व॑यस॒मि\-त्या॑ह।
अ॒पामे॒वैतेन॒ रस॑स्य सूयते।
सूर्य॑रश्मिꣳ स॒माभृ॑त॒मित्या॑ह सशुक्र॒त्वाय॑॥५५॥\anuvakamend[ग॒च्छ॒ति॒ सू॒यते॒ नव॑ च]

%1.3.10.1
इन्द्रो॑ वृ॒त्रꣳ ह॒त्वा।
असु॑रान्परा॒भाव्य॑।
सो॑ऽमावा॒स्यां᳚ प्रत्याग॑च्छत्।
ते पि॒तरः॑ पूर्वे॒द्युराग॑च्छन्।
पि॒तॄन् य॒ज्ञो॑\-ऽगच्छत्।
तं दे॒वाः पुन॑रयाचन्त।
तमे᳚भ्यो॒ न पुन॑रददुः।
ते᳚ऽब्रुव॒न्वरं॑ वृणामहै।
अथ॑ वः॒ पुन॑र्दास्यामः।
अ॒स्मभ्य॑मे॒व पू᳚र्वे॒द्युः क्रि॑याता॒ इति॑॥५६॥

%1.3.10.2
तमे᳚भ्यः॒ पुन॑रददुः।
तस्मा᳚त्पि॒तृभ्यः॑ पूर्वे॒द्युः क्रि॑यते।
यत्पि॒तृभ्यः॑ पूर्वे॒द्युः क॒रोति॑।
पि॒तृभ्य॑ ए॒व तद्य॒ज्ञं नि॒ष्क्रीय॒ यज॑मानः॒ प्रत॑नुते।
सोमा॑य पि॒तृपी॑ताय स्व॒धा नम॒ इत्या॑ह।
पि॒तुरे॒वाधि॑ सोमपी॒थमव॑ रुन्धे।
न हि पि॒ता प्र॒मीय॑माण॒ आहै॒ष सो॑मपी॒थ इति॑।
इ॒न्द्रि॒यं वै सो॑मपी॒थः।
इ॒न्द्रि॒यमे॒व सो॑मपी॒थमव॑ रुन्धे।
तेने᳚न्द्रि॒येण॑ द्वि॒तीयां᳚ जा॒याम॒भ्य॑श्ञुते॥५७॥

%1.3.10.3
ए॒तद्वै ब्राह्म॑णं पु॒रा वा॑जवश्रव॒सा वि॒दाम॑क्रन्।
तस्मा॒त्ते द्वेद्वे॑ जा॒ये अ॒भ्या᳚क्षत।
य ए॒वं वेद॑।
अ॒भि द्वि॒तीयां᳚ जा॒याम॑श्ञुते।
अ॒ग्नये॑ कव्य॒वाह॑नाय स्व॒धा नम॒ इत्या॑ह।
य ए॒व पि॑तृ॒णाम॒ग्निः।
तं प्री॑णाति।
ति॒स्र आहु॑तीर्जुहोति।
त्रिर्निद॑धाति।
षट्थ्सम्प॑द्यन्ते॥५८॥

%1.3.10.4
षड्वा ऋ॒तवः॑।
ऋ॒तूने॒व प्री॑णाति।
तू॒ष्णीं मेक्ष॑ण॒माद॑धाति।
अस्ति॑ वा॒ हि ष॒ष्ठ ऋ॒तुर्न वा᳚।
दे॒वान् वै पि॒तॄन्प्री॒तान्।
म॒नु॒ष्याः᳚ पि॒तरोऽनु॒ प्रपि॑पते।
ति॒स्र आहु॑तीर्जुहोति।
त्रिर्निद॑धाति।
षट्थ्सम्प॑द्यन्ते।
षड्वा ऋ॒तवः॑॥५९॥

%1.3.10.5
ऋ॒तवः॒ खलु॒ वै दे॒वाः पि॒तरः॑।
ऋ॒तूने॒व दे॒वान्पि॒तॄन्प्री॑णाति।
तान्प्री॒तान्।
म॒नु॒ष्याः᳚ पि॒तरोऽनु॒ प्रपि॑पते।
स॒कृ॒दा॒च्छि॒न्नं ब॒र्॒\mbox{}हिर्भ॑वति।
स॒कृदि॑व॒ हि पि॒तरः॑।
त्रिर्निद॑धाति।
तृ॒तीये॒ वा इ॒तो लो॒के पि॒तरः॑।
ताने॒व प्री॑णाति।
परा॒ङाव॑र्तते॥६०॥

%1.3.10.6
ह्लीका॒ हि पि॒तरः॑।
ओष्मणो᳚ व्या॒वृत॒ उपा᳚स्ते।
ऊ॒ष्मभा॑गा॒ हि पि॒तरः॑।
ब्र॒ह्म॒वा॒दिनो॑ वदन्ति।
प्राश्या (३) न्न प्राश्या (३) मिति॑।
यत्प्रा᳚श्ञी॒यात्।
जन्य॒मन्न॑मद्यात्।
प्र॒मायु॑कः स्यात्।
यन्न प्रा᳚श्ञी॒यात्।
अह॑विः स्यात्॥६१॥

%1.3.10.7
पि॒तृभ्य॒ आवृ॑श्च्येत।
अ॒व॒घ्रेय॑मे॒व।
तन्नेव॒ प्राशि॑तं॒ नेवाप्रा॑शितम्।
वी॒रं वा॒ वै पि॒तरः॑ प्र॒यन्तो॒ हर॑न्ति।
वी॒रं वा॑ ददति।
द॒शां छि॑नत्ति।
हर॑णभागा॒ हि पि॒तरः॑।
पि॒तॄने॒व नि॒रव॑दयते।
उत्त॑र॒ आयु॑षि॒ लोम॑ छिन्दीत।
पि॒तृ॒णाꣴ ह्ये॑तर्\mbox{}हि॒ नेदी॑यः॥६२॥

%1.3.10.8
नम॑स्करोति।
न॒म॒स्का॒रो हि पि॑तृ॒णाम्।
नमो॑ वः पितरो॒ रसा॑य।
नमो॑ वः पितरः॒ शुष्मा॑य।
नमो॑ वः पितरो जी॒वाय॑।
नमो॑ वः पितरः स्व॒धायै᳚।
नमो॑ वः पितरो म॒न्यवे᳚।
नमो॑ वः पितरो घो॒राय॑।
पित॑रो॒ नमो॑ वः।
य ए॒तस्मिँ॑ल्लो॒के स्थ॥६३॥

%1.3.10.9
यु॒ष्माꣴस्तेऽनु॑।
ये᳚ऽस्मिँल्लो॒के।
मां तेऽनु॑।
य ए॒तस्मिँ॑ल्लो॒के स्थ।
यू॒यं तेषां॒ वसि॑ष्ठा भूयास्त।
ये᳚ऽस्मिँल्लो॒के।
अ॒हं तेषां॒ वसि॑ष्ठो भूयास॒मित्या॑ह।
वसि॑ष्ठः समा॒नानां᳚ भवति।
य ए॒वं वि॒द्वान्पि॒तृभ्यः॑ क॒रोति॑।
ए॒ष वै म॑नु॒ष्या॑णां य॒ज्ञः॥६४॥

%1.3.10.10
दे॒वानां॒ वा इत॑रे य॒ज्ञाः।
तेन॒ वा ए॒तत्पि॑तृलो॒के च॑रति।
यत्पि॒तृभ्यः॑ क॒रोति॑।
स ई᳚श्व॒रः प्रमे॑तोः।
प्रा॒जा॒प॒त्यय॒र्चा पुन॒रैति॑।
य॒ज्ञो वै प्र॒जा\-प॑तिः।
य॒ज्ञेनै॒व स॒ह पुन॒रैति॑।
न प्र॒मायु॑को भवति।
पि॒तृ॒लो॒के वा ए॒तद्यज॑मानश्चरति।
यत्पि॒तृभ्यः॑ क॒रोति॑।
स ई᳚श्व॒र आर्ति॒मार्तोः᳚।
प्र॒जा\-प॑ति॒स्त्वावैनं॒ तत॒ उन्ने॑तुमर्\mbox{}ह॒तीत्या॑हुः।
यत्प्रा॑जाप॒त्यय॒र्चा पुन॒रैति॑।
प्र॒जा\-प॑तिरे॒वैनं॒ तत॒ उन्न॑यति।
नार्ति॒मार्च्छ॑ति॒ यज॑मानः॥६५॥\anuvakamend[इत्य॑श्ञुते पद्यन्ते पद्यन्ते॒ षड्वा ऋ॒तवो॑ वर्त॒ते\-ऽह॑विः स्या॒न्नेदी॑यः॒ स्थ य॒ज्ञो यज॑मानश्चरति॒ यत्पि॒तृभ्यः॑ क॒रोति॒ पञ्च॑ च]




\prashnaend{दे॒वा॒सु॒रा अ॒ग्नीषोम॑योर्दे॒वा वै यथा॒दर्\mbox{}शं॑ दे॒वा वै यद॒न्यैर्ग्रहै᳚र्ब्रह्मवा॒दिनो॒ नाग्नि॑ष्टो॒मो न सा॑वि॒त्रं दे॒वस्या॒हं ता॒र्प्यꣳ स॒प्तान्न॑हो॒मान्नृ॒षदं॒ त्वेन्द्रो॑ वृ॒त्रꣳ ह॒त्वा दश॑॥१०॥}{दे॒वा॒सु॒रा वा॒ज्ये॑वैनं॒ तस्मा᳚द्वाजपेयया॒जी दे॒वस्या॒हं वाज॒स्याव॑रुद्ध्या इन्द्रि॒यमे॒वास्मि॒न्॒ ह्लीका॒ हि पि॒तरः॒ पञ्च॑षष्टिः॥६५॥}{दे॒वा॒सु॒रा यज॑मानः॥}{हरिः॑ ओम्॥}{इति श्रीकृष्णयजुर्वेदीयतैत्तिरीयब्राह्मणे प्रथमाष्टके तृतीयः प्रपाठकः समाप्तः॥}
\clearpage
\sect{चतुर्थः प्रश्नः}
\setcounter{anuvakam}{0}
\dnsub{तैत्तिरीयब्राह्मणे प्रथमाष्टके चतुर्थः प्रपाठकः}

%1.4.1.1
उ॒भये॒ वा ए॒ते प्र॒जा\-प॑ते॒रध्य॑सृज्यन्त।
दे॒वाश्चासु॑राश्च।
तान्न व्य॑जानात्।
इ॒मे᳚ऽन्य इ॒मे᳚ऽन्य इति॑।
स दे॒वान॒ꣳ॒शून॑करोत्।
तान॒भ्य॑षुणोत्।
तान्प॒वित्रे॑णापुनात्।
तान्प॒रस्ता᳚त्प॒वित्र॑स्य॒ व्य॑गृह्णात्।
ते ग्रहा॑ अभवन्।
तद्ग्रहा॑णां ग्रह॒त्वम्॥१॥

%1.4.1.2
दे॒वता॒ वा ए॒ता यज॑मानस्य गृ॒हे गृ॑ह्यन्ते।
यद्ग्रहाः᳚।
वि॒दुरे॑नं दे॒वाः।
यस्यै॒वं वि॒दुष॑ ए॒ते ग्रहा॑ गृ॒ह्यन्ते᳚।
ए॒षा वै सोम॒स्या\-ऽऽहु॑तिः।
यदु॑पा॒ꣳ॒शुः।
सोमे॑न दे॒वाꣴस्त॑र्पया॒णीति॒ खलु॒ वै सोमे॑न यजते।
यदु॑पा॒ꣳ॒शुं जु॒होति॑।
सोमे॑नै॒व तद्दे॒वाꣴस्त॑र्पयति।
यद्ग्रहां᳚ जु॒होति॑॥२॥

%1.4.1.3
दे॒वा ए॒व तद्दे॒वान्ग॑च्छन्ति।
यच्च॑म॒सां जु॒होति॑।
तेनै॒वानु॑रूपेण॒ यज॑मानः सुव॒र्गं लो॒कमे॑ति।
किं न्वे॑तदग्र॑ आसी॒दित्या॑हुः।
यत्पात्रा॒णीति॑।
इ॒यं वा ए॒तदग्र॑ आसीत्।
मृ॒न्मया॑नि॒ वा ए॒तान्या॑सन्।
तैर्दे॒वा न व्या॒वृत॑मगच्छन्।
त ए॒तानि॑ दारु॒मया॑णि॒ पात्रा᳚ण्यपश्यन्।
तान्य॑कुर्वत॥३॥

%1.4.1.4
तैर्वै ते व्या॒वृत॑मगच्छन्।
यद्दा॑रु॒मया॑णि॒ पात्रा॑णि॒ भव॑न्ति।
व्या॒वृत॑मे॒व तैर्यज॑मानो गच्छति।
यानि॑ दारु॒मया॑णि॒ पात्रा॑णि॒ भव॑न्ति।
अ॒मुमे॒व तैर्लो॒कम॒भि\-ज॑यति।
यानि॑ मृ॒न्मया॑नि।
इ॒ममे॒व तैर्लो॒कम॒भि\-ज॑यति।
ब्र॒ह्म॒वा॒दिनो॑ वदन्ति।
काश्चत॑स्रः स्था॒लीर्वा॑य॒व्याः᳚ सोम॒ग्रह॑णी॒रिति॑।
दे॒वा वै पृश्ञि॑मदुह्रन्॥४॥

%1.4.1.5
तस्या॑ ए॒ते स्तना॑ आसन्।
इ॒यं वै पृश्ञिः॑।
तामा॑दि॒त्या आ॑दित्यस्था॒ल्या चतु॑ष्पदः प॒शून॑दुह्रन्।
यदा॑दित्यस्था॒ली भव॑ति।
चतु॑ष्पद ए॒व तया॑ प॒शून् यज॑मान इ॒मां दु॑हे।
तामिन्द्र॑ उक्थ्यस्था॒ल्येन्द्रि॒यम॑दुहत्।
यदु॑क्थ्यस्था॒ली भव॑ति।
इ॒न्द्रि॒यमे॒व तया॒ यज॑मान इ॒मां दु॑हे।
तां विश्वे॑ दे॒वा आ᳚ग्रयणस्था॒ल्योर्ज॑मदुह्रन्।
यदा᳚ग्रयणस्था॒ली भव॑ति॥५॥

%1.4.1.6
ऊर्ज॑मे॒व तया॒ यज॑मान इ॒मां दु॑हे।
तां म॑नु॒ष्या᳚ ध्रुवस्था॒ल्या\-ऽऽयु॑रदुह्रन्।
यद्ध्रु॑वस्था॒ली भव॑ति।
आयु॑रे॒व तया॒ यज॑मान इ॒मां दु॑हे।
स्था॒ल्या गृ॒ह्णाति॑।
वा॒य॒व्ये॑न जुहोति।
तस्मा॑द॒न्येन॒ पात्रे॑ण प॒शून्दु॒हन्ति॑।
अ॒न्येन॒ प्रति॑\-गृह्णन्ति।
अथो᳚ व्या॒वृत॑मे॒व तद्यज॑मानो गच्छति॥६॥\anuvakamend[ग्र॒ह॒त्वं ग्रहां᳚ जु॒होत्य॑कुर्वतादुह्रन्नाग्रयणस्था॒ली भव॑ति॒ नव॑ च]

%1.4.2.1
यु॒वꣳ सु॒राम॑मश्विना।
नमु॑चावासु॒रे सचा᳚।
वि॒पि॒पा॒ना शु॑भस्पती।
इन्द्रं॒ कर्म॑ स्वावतम्।
पु॒त्रमि॑व पि॒तरा॑व॒श्विनो॒भा।
इन्द्राव॑तं॒ कर्म॑णा द॒ꣳ॒सना॑भिः।
यथ्सु॒रामं॒ व्यपि॑बः॒ शची॑भिः।
सर॑स्वती त्वा मघवन्नभीष्णात्।
अहा᳚व्यग्ने ह॒विरा॒स्ये॑ते।
स्रु॒चीव॑ घृ॒तं च॒मू इ॑व॒ सोमः॑॥७॥

%1.4.2.2
वा॒ज॒सनिꣳ॑ र॒यिम॒स्मे सु॒वीरम्᳚।
प्र॒श॒स्तं धे॑हि य॒शसं॑ बृ॒हन्तम्᳚।
यस्मि॒न्नश्वा॑स ऋष॒भास॑ उ॒क्षणः॑।
व॒शा मे॒षा अ॑वसृ॒ष्टास॒ आहु॑ताः।
की॒ला॒ल॒पे सोम॑पृष्ठाय वे॒धसे᳚।
हृ॒दा म॒तिं ज॑नय॒ चारु॑म॒ग्नये᳚।
नाना॒ हि वां᳚ दे॒वहि॑त॒ꣳ॒ सदो॑ मि॒तम्।
मा सꣳसृ॑क्षाथां पर॒मे व्यो॑मन्।
सुरा॒ त्वमसि॑ शु॒ष्मिणी॒ सोम॑ ए॒षः।
मा मा॑ हिꣳसीः॒ स्वां योनि॑मावि॒शन्॥८॥

%1.4.2.3
यदत्र॑ शि॒ष्टꣳ र॒सिनः॑ सु॒तस्य॑।
यदिन्द्रो॒ अपि॑ब॒च्छची॑भिः।
अ॒हं तद॑स्य॒ मन॑सा शि॒वेन॑।
सोम॒ꣳ॒ राजा॑नमि॒ह भ॑क्षयामि।
द्वे स्रु॒ती अ॑शृणवं पितृ॒णाम्।
अ॒हं दे॒वाना॑मु॒त मर्त्या॑नाम्।
ताभ्या॑मि॒दं विश्वं॒ भुव॑न॒ꣳ॒ समे॑ति।
अ॒न्त॒रा पूर्व॒मप॑रं च के॒तुम्।
यस्ते॑ देव वरुण गाय॒त्रछ॑न्दाः॒ पाशः॑।
तं त॑ ए॒तेनाव॑ यजे॥९॥

%1.4.2.4
यस्ते॑ देव वरुण त्रि॒ष्टुप्छ॑न्दाः॒ पाशः॑।
तं त॑ ए॒तेनाव॑ यजे।
यस्ते॑ देव वरुण॒ जग॑तीछन्दाः॒ पाशः॑।
तं त॑ ए॒तेनाव॑ यजे।
सोमो॒ वा ए॒तस्य॑ रा॒ज्यमाद॑त्ते।
यो राजा॒ सन्रा॒ज्यो वा॒ सोमे॑न॒ यज॑ते।
दे॒व॒सु॒वामे॒तानि॑ ह॒वीꣳषि॑ भवन्ति।
ए॒ताव॑न्तो॒ वै दे॒वानाꣳ॑ स॒वाः।
त ए॒वास्मै॑ स॒वान्प्रय॑च्छन्ति।
त ए॑नं॒ पुनः॑ सुवन्ते रा॒ज्याय॑।
दे॒व॒सू राजा॑ भवति॥१०॥\anuvakamend[सोम॑ आवि॒शन् य॑जे रा॒ज्यायैकं॑ च]

%1.4.3.1
उद॑स्थाद्दे॒व्यदि॑तिर्विश्वरू॒पी।
आयु॑र्य॒ज्ञप॑तावधात्।
इन्द्रा॑य कृण्व॒ती भा॒गम्।
मि॒त्राय॒ वरु॑णाय च।
इ॒यं वा अ॑ग्निहो॒त्री।
इ॒यं वा ए॒तस्य॒ निषी॑दति।
यस्या᳚ग्निहो॒त्री नि॒षीद॑ति।
तामुत्था॑पयेत्।
उद॑स्थाद्दे॒व्यदि॑ति॒रिति॑।
इ॒यं वै दे॒व्यदि॑तिः॥११॥

%1.4.3.2
इ॒मामे॒वास्मा॒ उत्था॑पयति।
आयु॑र्य॒ज्ञप॑तावधा॒दित्या॑ह।
आयु॑रे॒वा\-स्मि॑न्दधाति।
इन्द्रा॑य कृण्व॒ती भा॒गं मि॒त्राय॒ वरु॑णाय॒ चेत्या॑ह।
य॒था॒\-य॒जु\-रे॒वै\-तत्।
अव॑र्तिं॒ वा ए॒षैतस्य॑ पा॒प्मानं॑ प्रति॒ख्याय॒ निषी॑दति।
यस्या᳚ग्निहो॒त्र्युप॑सृष्टा नि॒षीद॑ति।
तां दु॒ग्ध्वा ब्रा᳚ह्म॒णाय॑ दद्यात्।
यस्यान्नं॒ नाद्यात्।
अव॑र्तिमे॒वास्मि॑न्पा॒प्मानं॒ प्रति॑\-मुञ्चति॥१२॥

%1.4.3.3
दु॒ग्ध्वा द॑दाति।
न ह्यदृ॑ष्टा॒ दक्षि॑णा दी॒यते᳚।
पृ॒थि॒वीं वा ए॒तस्य॒ पयः॒ प्रवि॑शति।
यस्या᳚ग्निहो॒त्रं दु॒ह्यमा॑न॒ꣴ॒ स्कन्द॑ति।
यद॒द्य दु॒ग्धं पृ॑थि॒वीमस॑क्त।
यदोष॑धीर॒प्यस॑र॒द्यदापः॑।
पयो॑ गृ॒हेषु॒ पयो॑ अघ्नि॒यासु॑।
पयो॑ व॒थ्सेषु॒ पयो॑ अस्तु॒ तन्मयीत्या॑ह।
पय॑ ए॒वाऽऽत्मन्गृ॒हेषु॑ प॒शुषु॑ धत्ते।
अ॒प उप॑\-सृजति॥१३॥

%1.4.3.4
अ॒द्भिरे॒वैन॑दाप्नोति।
यो वै य॒ज्ञस्यार्ते॒ नाना᳚र्तꣳ सꣳसृ॒जति॑।
उ॒भे वै ते तर्ह्यार्च्छ॑तः।
आर्च्छ॑ति॒ खलु॒ वा ए॒तद॑ग्निहो॒त्रम्।
यद्दु॒ह्यमा॑न॒ꣴ॒ स्कन्द॑ति।
यद॑भिदु॒ह्यात्।
आर्ते॒ नाना᳚र्तं य॒ज्ञस्य॒ सꣳसृ॑जेत्।
तदे॒व या॒दृक्की॒दृक्च॑ होत॒व्यम्᳚।
अथा॒न्यां दु॒ग्ध्वा पुन॑र्\mbox{}होत॒व्यम्᳚।
अना᳚र्तेनै॒वार्तं॑ य॒ज्ञस्य॒ निष्क॑रोति॥१४॥

%1.4.3.5
यद्युद्द्रु॑तस्य॒ स्कन्दे᳚त्।
यत्ततो\-ऽहु॑त्वा॒ पुन॑रे॒यात्।
य॒ज्ञं वि\-च्छि॑न्द्यात्।
यत्र॒ स्कन्दे᳚त्।
तन्नि॒षद्य॒ पुन॑र्गृह्णीयात्।
यत्रै॒व स्कन्द॑ति।
तत॑ ए॒वैन॒त्पुन॑र्गृह्णाति।
तदे॒व या॒दृक्की॒दृक्च॑ होत॒व्यम्᳚।
अथा॒न्यां दु॒ग्ध्वा पुन॑र्\mbox{}होत॒व्यम्᳚।
अना᳚र्तेनै॒वार्तं॑ य॒ज्ञस्य॒ निष्क॑रोति॥१५॥

%1.4.3.6
वि वा ए॒तस्य॑ य॒ज्ञश्छि॑द्यते।
यस्या᳚ग्निहो॒त्रे॑\-ऽधिश्रि॑ते॒ श्वा\-ऽन्त॒रा धाव॑ति।
रु॒द्रः खलु॒ वा ए॒षः।
यद॒ग्निः।
यद्गाम॑न्वत्या व॒र्तये᳚त्।
रु॒द्राय॑ प॒शूनपि॑ दध्यात्।
अ॒प॒शुर्यज॑मानः स्यात्।
यद॒पो᳚\-ऽन्वतिषि॒ञ्चेत्।
अ॒ना॒द्यम॒ग्नेरापः॑।
अ॒ना॒द्यमा᳚भ्या॒मपि॑ दध्यात्।
गार्\mbox{}ह॑पत्या॒द्भस्मा॒दाय॑।
इ॒दं विष्णु॒र्विच॑क्रम॒ इति॑ वैष्ण॒व्यर्चा\-ऽऽह॑व॒नीया᳚द्‌ध्व॒ꣳ॒सय॒न्नुद्र॑वेत्।
य॒ज्ञो वै विष्णुः॑।
य॒ज्ञेनै॒व य॒ज्ञꣳ सं त॑नोति।
भस्म॑ना प॒दमपि॑ वपति॒ शान्त्यै᳚॥१६॥\anuvakamend[वै दे॒व्यदि॑तिर्मुञ्चति \-सृजति करोति करोत्याभ्या॒मपि॑ दध्या॒त् पञ्च॑ च]

%1.4.4.1
नि वा ए॒तस्या॑\-ऽऽहव॒नीयो॒ गार्\mbox{}ह॑पत्यं कामयते।
निगार्\mbox{}ह॑पत्य आहव॒नीयम्᳚।
यस्या॒ग्निमनु॑द्धृत॒ꣳ॒ सूर्यो॒ऽभि नि॒म्रोच॑ति।
द॒र्भेण॒ हिर॑ण्यं प्र॒बद्ध्य॑ पु॒रस्ता᳚द्धरेत्।
अथा॒ग्निम्।
अथा᳚ग्निहो॒त्रम्।
यद्धिर॑ण्यं पु॒रस्ता॒द्धर॑ति।
ज्योति॒र्वै हिर॑ण्यम्।
ज्योति॑रे॒वैनं॒ पश्य॒न्नुद्ध॑रति।
यद॒ग्निं पूर्व॒ꣳ॒ हर॒त्यथा᳚ग्निहो॒त्रम्॥१७॥

%1.4.4.2
भा॒ग॒धेये॑नै॒वैनं॒ प्रण॑यति।
ब्रा॒ह्म॒ण आ॑र्\mbox{}षे॒य उद्ध॑रेत्।
ब्रा॒ह्म॒णो वै सर्वा॑ दे॒वताः᳚।
सर्वा॑भिरे॒वैनं॑ दे॒वता॑भि॒रुद्ध॑रति।
अ॒ग्नि॒हो॒त्रमु॑प॒साद्यातमि॑तोरासीत।
व्र॒तमे॒व ह॒तमनु॑ म्रियते।
अन्तं॒ वा ए॒ष आ॒त्मनो॑ गच्छति।
यस्ताम्य॑ति।
अन्त॑मे॒ष य॒ज्ञस्य॑ गच्छति।
यस्या॒ग्निमनु॑द्धृत॒ꣳ॒ सूर्यो॒ऽभि नि॒म्रोच॑ति॥१८॥

%1.4.4.3
पुनः॑ स॒मन्य॑ जुहोति।
अन्ते॑नै॒वान्तं॑ य॒ज्ञस्य॒ निष्क॑रोति।
वरु॑णो॒ वा ए॒तस्य॑ य॒ज्ञं गृ॑ह्णाति।
यस्या॒ग्निमनु॑द्धृत॒ꣳ॒ सूर्यो॒ऽभि नि॒म्रोच॑ति।
वा॒रु॒णं च॒रुं निर्व॑पेत्।
तेनै॒व य॒ज्ञं निष्क्री॑णीते।
नि वा ए॒तस्या॑\-ऽऽहव॒नीयो॒ गार्\mbox{}ह॑पत्यं कामयते।
नि गार्\mbox{}ह॑पत्य आहव॒नीयम्᳚।
यस्या॒ग्निमनु॑द्धृत॒ꣳ॒ सूर्यो॒\-ऽभ्यु॑देति॑।
च॒तु॒र्गृ॒ही॒तमाज्यं॑ पु॒रस्ता᳚द्धरेत्॥१९॥

%1.4.4.4
अथा॒ग्निम्।
अथा᳚ग्निहो॒त्रम्।
यदाज्यं॑ पु॒रस्ता॒द्धर॑ति।
ए॒तद्वा अ॒ग्नेः प्रि॒यं धाम॑।
यदाज्यम्᳚।
प्रि॒येणै॒वैनं॒ धाम्ना॒ सम॑र्धयति।
यद॒ग्निं पूर्व॒ꣳ॒ हर॒त्यथा᳚ग्निहो॒त्रम्।
भा॒ग॒धेये॑नै॒वैनं॒ प्रण॑यति।
ब्रा॒ह्म॒ण आ॑र्\mbox{}षे॒य उद्ध॑रेत्।
ब्रा॒ह्म॒णो वै सर्वा॑ दे॒वताः᳚॥२०॥

%1.4.4.5
सर्वा॑भिरे॒वैनं॑ दे॒वता॑भि॒रुद्ध॑रति।
परा॑ची॒ वा ए॒तस्मै᳚ व्यु॒च्छन्ती॒ व्यु॑च्छति।
यस्या॒ग्निमनु॑द्धृत॒ꣳ॒ सूर्यो॒\-ऽभ्यु॑देति॑।
उ॒षाः के॒तुना॑ जुषताम्।
य॒ज्ञं दे॒वेभि॑रिन्वि॒तम्।
दे॒वेभ्यो॒ मधु॑मत्तम॒ꣴ॒ स्वाहेति॑ प्र॒त्यङ्नि॒षद्याज्ये॑न जुहुयात्।
प्र॒तीची॑मे॒वास्मै॒ विवा॑सयति।
अ॒ग्नि॒हो॒त्रमु॑प॒साद्यातमि॑तोरासीत।
व्र॒तमे॒व ह॒तमनु॑ म्रियते।
अन्तं॒ वा ए॒ष आ॒त्मनो॑ गच्छति॥२१॥

%1.4.4.6
यस्ताम्य॑ति।
अन्त॑मे॒ष य॒ज्ञस्य॑ गच्छति।
यस्या॒ग्निमनु॑द्धृत॒ꣳ॒ सूर्यो॒\-ऽभ्यु॑देति॑।
पुनः॑ स॒मन्य॑ जुहोति।
अन्ते॑नै॒वान्तं॑ य॒ज्ञस्य॒ निष्क॑रोति।
मि॒त्रो वा ए॒तस्य॑ य॒ज्ञं गृ॑ह्णाति।
यस्या॒ग्निमनु॑द्धृत॒ꣳ॒ सूर्यो॒\-ऽभ्यु॑देति॑।
मै॒त्रं च॒रुं निर्व॑पेत्।
तेनै॒व य॒ज्ञं निष्क्री॑णीते।
यस्या॑\-ऽऽहव॒नीये\-ऽनु॑द्वाते॒ गार्\mbox{}ह॑पत्य उ॒द्वाये᳚त्॥२२॥

%1.4.4.7
यदा॑हव॒नीय॒मनु॑द्वाप्य॒ गार्\mbox{}ह॑पत्यं॒ मन्थे᳚त्।
वि\-च्छि॑न्द्यात्।
भ्रातृ॑व्यमस्मै जनयेत्।
यद्वै य॒ज्ञस्य॑ वास्त॒व्यं॑ क्रि॒यते᳚।
तदनु॑ रु॒द्रो\-ऽव॑चरति।
यत्पूर्व॑मन्वव॒स्येत्।
वा॒स्त॒व्य॑म॒ग्निमुपा॑सीत।
रु॒द्रो᳚ऽस्य प॒शून्घातु॑कः स्यात्।
आ॒ह॒व॒नीय॑मु॒द्वाप्य॑।
गार्\mbox{}ह॑पत्यं मन्थेत्॥२३॥

%1.4.4.8
इ॒तः प्र॑थ॒मं ज॑ज्ञे अ॒ग्निः।
स्वाद्योने॒रधि॑ जा॒तवे॑दाः।
स गा॑यत्रि॒या त्रि॒ष्टुभा॒ जग॑त्या।
दे॒वेभ्यो॑ ह॒व्यं व॑हतु प्रजा॒नन्निति॑।
छन्दो॑भिरे॒वैन॒ꣴ॒ स्वाद्योनेः॒ प्रज॑नयति।
गार्\mbox{}ह॑पत्यं मन्थति।
गार्\mbox{}ह॑पत्यं॒ वा अन्वाहि॑ताग्नेः प॒शव॒ उप॑ तिष्ठन्ते।
स यदु॒द्वाय॑ति।
तदनु॑ प॒शवोऽप॑ क्रामन्ति।
इ॒षे र॒य्यै र॑मस्व॥२४॥

%1.4.4.9
सह॑से द्यु॒म्नाय॑।
ऊ॒र्जे\-ऽपत्या॒येत्या॑ह।
प॒शवो॒ वै र॒यिः।
प॒शूने॒वास्मै॑ रमयति।
सा॒र॒स्व॒तौ त्वोथ्सौ॒ समि॑न्धाता॒मित्या॑ह।
ऋ॒ख्सा॒मे वै सा॑रस्व॒तावुथ्सौ᳚।
ऋ॒ख्सा॒माभ्या॑मे॒वैन॒ꣳ॒ समि॑न्धे।
स॒म्राड॑सि वि॒राड॒सीत्या॑ह।
र॒थ॒न्त॒रं वै स॒म्राट्।
बृ॒हद्वि॒राट्॥२५॥

%1.4.4.10
ताभ्या॑मे॒वैन॒ꣳ॒ समि॑न्धे।
वज्रो॒ वै च॒क्रम्।
वज्रो॒ वा ए॒तस्य॑ य॒ज्ञं विच्छि॑नत्ति।
यस्यानो॑ वा॒ रथो॑ वाऽन्त॒रा\-ऽग्नी याति॑।
आ॒ह॒व॒नीय॑मु॒द्वाप्य॑।
गार्\mbox{}ह॑पत्या॒दुद्ध॑रेत्।
यद॑ग्ने॒ पूर्वं॒ प्रभृ॑तं प॒दꣳ हि ते᳚।
सूर्य॑स्य र॒श्मीनन्वा॑त॒तान॑।
तत्र॑ रयि॒ष्ठामनु॒ सं भ॑रै॒तम्।
सं नः॑ सृज सुम॒त्या वाज॑व॒त्येति॑॥२६॥

%1.4.4.11
पूर्वे॑णै॒वास्य॑ य॒ज्ञेन॑ य॒ज्ञमनु॒ सं त॑नोति।
त्वम॑ग्ने स॒प्रथा॑ अ॒सीत्या॑ह।
अ॒ग्निः सर्वा॑ दे॒वताः᳚।
दे॒वता॑भिरे॒व य॒ज्ञꣳ सं त॑नोति।
अ॒ग्नये॑ पथि॒कृते॑ पुरो॒डाश॑म॒ष्टा\-क॑पालं॒ निर्व॑पेत्।
अ॒ग्निमे॒व प॑थि॒कृत॒ꣴ॒ स्वेन॑ भाग॒धेये॒नोप॑धावति।
स ए॒वैनं॑ य॒ज्ञियं॒ पन्था॒मपि॑ नयति।
अ॒न॒ड्वान्दक्षि॑णा।
व॒ही ह्ये॑ष समृ॑द्ध्यै॥२७॥\anuvakamend[हर॒त्यथा᳚ग्निहो॒त्रं नि॒म्रोच॑ति हरेद्दे॒वता॑ गच्छत्यु॒द्वाये᳚न्मन्थेद्रमस्व बृ॒हद्वि॒राडिति॒ नव॑ च (नि वै पूर्वं॒ त्रीणि॑ नि॒म्रोच॑ति द॒र्भेण॒ यद्धिर॑ण्यमग्निहो॒त्रं पुन॒र्वरु॑णो वारु॒णं नि वा ए॒तस्या॒भ्यु॑देति॑ चतुर्गृही॒तमाज्यं॒ यदाज्यं॒ परा᳚च्यु॒षाः पुन॑र्मि॒त्रो मै॒त्रं यस्या॑\-ऽऽहव॒नीये\-ऽनु॑द्वाते॒ गार्\mbox{}ह॑पत्यो॒ यद्वै म॑न्थे॒दुद्ध॑रेत्॥)]

%1.4.5.1
यस्य॑ प्रातः सव॒ने सोमो॑\-ऽति॒\-रिच्य॑ते।
माध्यं॑ दिन॒ꣳ॒ सव॑नं का॒मय॑मानो॒\-ऽभ्यति॑रिच्यते।
गौर्ध॑यति म॒रुता॒मिति॒ धय॑द्वतीषु कुर्वन्ति।
हि॒नस्ति॒ वै स॒न्ध्यधी॑तम्।
स॒न्धीव॒ खलु॒ वा ए॒तत्।
यथ्सव॑नस्याति॒\-रिच्य॑ते।
यद्धय॑द्वतीषु कु॒र्वन्ति॑।
स॒न्धेः शान्त्यै᳚।
गा॒य॒त्रꣳ साम॑ भवति पञ्चद॒शः स्तोमः॑।
तेनै॒व प्रा॑तः सव॒नान्नय॑न्ति॥२८॥

%1.4.5.2
म॒रुत्व॑तीषु कुर्वन्ति।
तेनै॒व माध्यं॑ दिना॒थ्सव॑ना॒न्नय॑न्ति।
होतु॑श्चम॒समनून्न॑यन्ते।
होताऽनु॑ शꣳसति।
म॒ध्य॒त ए॒व य॒ज्ञꣳ स॒माद॑धाति।
यस्य॒ माध्यं॑ दिने॒ सव॑ने॒ सोमो॑\-ऽति॒\-रिच्य॑ते।
आ॒दि॒त्यं तृ॑तीयसव॒नं का॒मय॑मानो॒\-ऽभ्यति॑रिच्यते।
गौ॒रि॒वी॒तꣳ साम॑ भवति।
अति॑रिक्तं॒ वै गौ॑रिवी॒तम्।
अति॑रिक्तं॒ यथ्सव॑नस्याति॒\-रिच्य॑ते॥२९॥

%1.4.5.3
अति॑रिक्तस्य॒ शान्त्यै᳚।
बण्म॒हाꣳ अ॑सि सू॒र्येति॑ कुर्वन्ति।
यस्यै॒वा\-ऽऽदि॒त्यस्य॒ सव॑नस्य॒ कामे॑नाति॒\-रिच्य॑ते।
तेनै॒वैनं॒ कामे॑न॒ सम॑र्धयन्ति।
गौ॒रि॒वी॒तꣳ साम॑ भवति।
तेनै॒व माध्यं॑ दिना॒थ्सव॑ना॒न्नय॑न्ति।
स॒प्त॒द॒शः स्तोमः॑।
तेनै॒व तृ॑तीयसव॒नान्नय॑न्ति।
होतु॑श्चम॒समनून्न॑यन्ते।
होताऽनु॑ शꣳसति॥३०॥

%1.4.5.4
म॒ध्य॒त ए॒व य॒ज्ञꣳ स॒माद॑धाति।
यस्य॑ तृतीयसव॒ने सोमो॑\-ऽति॒रिच्ये॑त।
उ॒क्थ्यं॑ कुर्वीत।
यस्यो॒क्थ्ये॑\-ऽति॒रिच्ये॑त।
अ॒ति॒रा॒त्रं कु॑र्वीत।
यस्या॑तिरा॒त्रे॑\-ऽति॒\-रिच्य॑ते।
तत्त्वै दु॑ष्प्रज्ञा॒नम्।
यज॑मानं॒ वा ए॒तत्प॒शव॑ आ॒साह्य॑यन्ति।
बृ॒हथ्साम॑ भवति।
बृ॒हद्वा इ॒माँल्लो॒कान्दा॑धार।
बार्\mbox{}ह॑ताः प॒शवः॑।
बृ॒ह॒तैवास्मै॑ प॒शून्दा॑धार।
शि॒पि॒वि॒ष्टव॑तीषु कुर्वन्ति।
शि॒पि॒वि॒ष्टो वै दे॒वानां᳚ पु॒ष्टम्।
पुष्ट्यै॒वैन॒ꣳ॒ सम॑र्धयन्ति।
होतु॑श्चम॒समनून्न॑यन्ते।
होताऽनु॑शꣳसति।
म॒ध्य॒त ए॒व य॒ज्ञꣳ स॒माद॑धाति॥३१॥\anuvakamend[य॒न्ति॒ सव॑नस्याति॒\-रिच्य॑ते शꣳसति दाधारा॒ष्टौ च॑]

%1.4.6.1
एकै॑को॒ वै ज॒नता॑या॒मिन्द्रः॑।
एकं॒ वा ए॒ताविन्द्र॑म॒भि सꣳसु॑नुतः।
यौ द्वौ सꣳ॑सुनु॒तः।
प्र॒जा\-प॑ति॒र्वा ए॒ष विता॑यते।
यद्य॒ज्ञः।
तस्य॒ ग्रावा॑णो॒ दन्ताः᳚।
अ॒न्य॒त॒रं वा ए॒ते सꣳ॑सुन्व॒तोर्निर्ब॑फ्सति।
पूर्वे॑णोप॒सृत्या॑ दे॒वता॒ इत्या॑हुः।
पू॒र्वो॒प॒सृ॒तस्य॒ वै श्रेया᳚न्भवति।
एति॑व॒न्त्याज्या॑नि भवन्त्य॒भिजि॑त्यै॥३२॥

%1.4.6.2
म॒रुत्व॑तीः प्रति॒पदः॑।
म॒रुतो॒ वै दे॒वाना॒मप॑राजितमा॒यत॑नम्।
दे॒वाना॑मे॒वाप॑राजित आ॒यत॑ने यतते।
उ॒भे बृ॑हद्रथन्त॒रे भ॑वतः।
इ॒यं वाव र॑थन्त॒रम्।
अ॒सौ बृ॒हत्।
आ॒भ्यामे॒वैन॑म॒न्तरे॑ति।
वा॒चश्च॒ मन॑सश्च।
प्रा॒णाच्चा॑पा॒नाच्च॑।
दि॒वश्च॑ पृथि॒व्याश्च॑॥३३॥

%1.4.6.3
सर्व॑स्माद्वि॒त्ताद्वेद्या᳚त्।
अ॒भि॒व॒र्तो ब्र॑ह्मसा॒मं भ॑वति।
सु॒व॒र्गस्य॑ लो॒कस्या॒भिवृ॑त्त्यै।
अ॒भि॒जिद्भ॑वति।
सु॒व॒र्गस्य॑ लो॒कस्या॒भिजि॑त्यै।
वि॒श्व॒जिद्भ॑वति।
विश्व॑स्य॒ जित्यै᳚।
यस्य॒ भूयाꣳ॑सो यज्ञक्र॒तव॒ इत्या॑हुः।
स दे॒वता॑ वृङ्क्त॒ इति॑।
यद्य॑ग्निष्टो॒मः सोमः॑ प॒रस्ता॒थ्स्यात्॥३४॥

%1.4.6.4
उ॒क्थ्यं॑ कुर्वीत।
यद्यु॒क्थः॑ स्यात्।
अ॒ति॒रा॒त्रं कु॑र्वीत।
य॒ज्ञ॒क्र॒तुभि॑रे॒वास्य॑ दे॒वता॑ वृङ्क्ते।
यो वै छन्दो॑भिरभि॒भव॑ति।
स सꣳ॑सुन्व॒तोर॒भिभ॑वति।
सं॒वे॒शाय॑ त्वोपवे॒शाय॒ त्वेत्या॑ह।
छन्दाꣳ॑सि॒ वै सं॑वे॒श उ॑पवे॒शः।
छन्दो॑भिरे॒वास्य॒ छन्दाꣴ॑स्य॒भिभ॑वति।
इ॒ष्टर्गो॒ वा ऋ॒त्विजा॑मध्व॒र्युः॥३५॥

%1.4.6.5
इ॒ष्टर्गः॒ खलु॒ वै पूर्वो॒\-ऽर्ष्टुः क्षी॑यते।
प्राणा॑पानौ मृ॒त्योर्मा॑ पात॒मित्या॑ह।
प्रा॒णा॒पा॒नयो॑रे॒व श्र॑यते।
प्राणा॑पानौ॒ मा मा॑ हासिष्ट॒मित्या॑ह।
नैनं॑ पु॒रा\-ऽऽयु॑षः प्राणापा॒नौ ज॑हितः।
आर्तिं॒ वा ए॒ते निय॑न्ति।
येषां᳚ दीक्षि॒तानां᳚ प्र॒मीय॑ते।
तं यद॑व॒वर्जे॑युः।
क्रू॒र॒कृता॑मिवैषां लो॒कः स्या᳚त्।
आह॑र द॒हेति॑ ब्रूयात्॥३६॥

%1.4.6.6
तं द॑क्षिण॒तो वेद्यै॑ नि॒धाय॑।
स॒र्प॒रा॒ज्ञिया॑ ऋ॒ग्भिः स्तु॑युः।
इ॒यं वै सर्प॑तो॒ राज्ञी᳚।
अ॒स्या ए॒वैनं॒ परि॑ददति।
व्यृ॑द्धं॒ तदित्या॑हुः।
यथ्स्तु॒तमन॑नुशस्त॒मिति॑।
होता᳚ प्रथ॒मः प्रा॑चीनावी॒ती मा᳚र्जा॒लीयं॒ परी॑यात्।
या॒मीर॑नुब्रु॒वन्।
स॒र्प॒रा॒ज्ञीनां᳚ कीर्तयेत्।
उ॒भयो॑रे॒वैनं॑ लो॒कयोः॒ परि॑ददति॥३७॥

%1.4.6.7
अथो॑ धु॒वन्त्ये॒वैनम्᳚।
अथो॒ न्ये॑वास्मै᳚ ह्नुवते।
त्रिः परि॑यन्ति।
त्रय॑ इ॒मे लो॒काः।
ए॒भ्य ए॒वैनं॑ लो॒केभ्यो॑ धुवते।
त्रिः पुनः॒ परि॑यन्ति।
षट्थ्सम्प॑द्यन्ते।
षड्वा ऋ॒तवः॑।
ऋ॒तुभि॑रे॒वैनं॑ धुवते।
अग्न॒ आयूꣳ॑षि पवस॒ इति॑ प्रति॒पदं॑ कुर्वीरन्।
र॒थ॒न्त॒रसा॑मैषा॒ꣳ॒ सोमः॑ स्यात्।
आयु॑रे॒वाऽऽत्मन्द॑धते।
अथो॑ पा॒प्मान॑मे॒व वि॒जह॑तो यन्ति॥३८॥\anuvakamend[अ॒भिजि॑त्यै पृथि॒व्याश्च॒ स्याद॑ध्व॒र्युर्ब्रू॑याल्लो॒कयोः॒ परि॑ददति कुर्वीर॒ꣴ॒स्त्रीणि॑ च]

%1.4.7.1
अ॒सु॒र्यं॑ वा ए॒तस्मा॒द्वर्णं॑ कृ॒त्वा।
प॒शवो॑ वी॒र्य॑मप॑ क्रामन्ति।
यस्य॒ यूपो॑ वि॒रोह॑ति।
त्वा॒ष्ट्रं ब॑हुरू॒पमाल॑भेत।
त्वष्टा॒ वै रू॒पाणा॑मीशे।
य ए॒व रू॒पाणा॒मीशे᳚।
सो᳚ऽस्मिन्प॒शून् वी॒र्यं॑ यच्छति।
नास्मा᳚त्प॒शवो॑ वी॒र्य॑मप॑ क्रामन्ति।
आर्तिं॒ वा ए॒ते निय॑न्ति।
येषां᳚ दीक्षि॒ताना॑म॒ग्निरु॒द्वाय॑ति॥३९॥

%1.4.7.2
यदा॑हव॒नीय॑ उ॒द्वाये᳚त्।
यत्तं मन्थे᳚त्।
वि\-च्छि॑न्द्यात्।
भ्रातृ॑व्यमस्मै जनयेत्।
यदा॑हव॒नीय॑ उ॒द्वाये᳚त्।
आग्नी᳚द्ध्रा॒दुद्ध॑\-रेत्।
यदाग्नी᳚द्ध्र उ॒द्वाये᳚त्।
गार्\mbox{}ह॑पत्या॒दुद्ध॑रेत्।
यद्गार्\mbox{}ह॑पत्य उ॒द्वाये᳚त्।
अत॑ ए॒व पुन॑र्मन्थेत्॥४०॥

%1.4.7.3
अत्र॒ वाव स निल॑यते।
यत्र॒ खलु॒ वै निली॑नमुत्त॒मं पश्य॑न्ति।
तदे॑नमिच्छन्ति।
यस्मा॒द्दारो॑रु॒द्वाये᳚त्।
तस्या॒रणी॑ कुर्यात्।
क्रु॒मु॒कमपि॑ कुर्यात्।
ए॒षा वा अ॒ग्नेः प्रि॒या त॒नूः।
यत्क्रु॑मु॒कः।
प्रि॒ययै॒वैनं॑ त॒नुवा॒ सम॑र्धयति।
गार्\mbox{}ह॑पत्यं मन्थति॥४१॥

%1.4.7.4
गार्\mbox{}ह॑पत्यो॒ वा अ॒ग्नेर्योनिः॑।
स्वादे॒वैनं॒ योने᳚र्जनयति।
नास्मै॒ भ्रातृ॑व्यं जनयति।
यस्य॒ सोम॑ उप॒दस्ये᳚त्।
सु॒वर्ण॒ꣳ॒ हिर॑ण्यं द्वे॒धा वि॒च्छिद्य॑।
ऋ॒जी॒षे᳚\-ऽन्यदा॑धूनु॒यात्।
जु॒हु॒याद॒न्यत्।
सोम॑मे॒वाभि॑षु॒णोति॑।
सोमं॑ जुहोति।
सोम॑स्य॒ वा अ॑भिषू॒यमा॑णस्य प्रि॒या त॒नूरुद॑क्रामत्॥४२॥

%1.4.7.5
तथ्सु॒वर्ण॒ꣳ॒ हिर॑ण्यमभवत्।
यथ्सु॒वर्ण॒ꣳ॒ हिर॑ण्यं कु॒र्वन्ति॑।
प्रि॒ययै॒वैनं॑ त॒नुवा॒ सम॑र्धयन्ति।
यस्याक्री॑त॒ꣳ॒ सोम॑मप॒हरे॑युः।
क्री॒णी॒यादे॒व।
सैव ततः॒ प्राय॑श्चित्तिः।
यस्य॑ क्री॒तम॑प॒हरे॑युः।
आ॒दा॒राꣴश्च॑ फाल्गु॒नानि॑ चा॒भिषु॑णुयात्।
गा॒य॒त्री यꣳ सोम॒माह॑रत्।
तस्य॒ योऽꣳ॑शुः प॒रा\-ऽप॑तत्॥४३॥

%1.4.7.6
त आ॑दा॒रा अ॑भवन्।
इन्द्रो॑ वृ॒त्रम॑हन्।
तस्य॑ व॒ल्कः परा॑\-ऽपतत्।
तानि॑ फाल्गु॒नान्य॑भवन्।
प॒शवो॒ वै फा᳚ल्गु॒नानि॑।
प॒शवः॒ सोमो॒ राजा᳚।
यदा॑दा॒राꣴश्च॑ फाल्गु॒नानि॑ चाभिषु॒णोति॑।
सोम॑मे॒व राजा॑नम॒भिषु॑णोति।
शृ॒तेन॑ प्रातः सव॒ने श्री॑णीयात्।
द॒ध्ना म॒ध्यं दि॑ने॥४४॥

%1.4.7.7
नी॒त॒मि॒श्रेण॑ तृतीयसव॒ने।
अ॒ग्नि॒ष्टो॒मः सोमः॑ स्याद्रथन्त॒र\-सा॑मा।
य ए॒वर्त्विजो॑ वृ॒ताः स्युः।
त ए॑नं याजयेयुः।
एकां॒ गां दक्षि॑णां दद्या॒त्तेभ्य॑ ए॒व।
पुनः॒ सोमं॑ क्रीणीयात्।
य॒ज्ञेनै॒व तद्य॒ज्ञमि॑च्छति।
सैव ततः॒ प्राय॑श्चित्तिः।
सर्वा᳚भ्यो॒ वा ए॒ष दे॒वता᳚भ्यः॒ सर्वे᳚भ्यः पृ॒ष्ठेभ्य॑ आ॒त्मान॒मागु॑रते।
यः स॒त्राया॑गु॒रते᳚।
ए॒तावा॒न्खलु॒ वै पुरु॑षः।
याव॑दस्य वि॒त्तम्।
स॒र्व॒वे॒द॒सेन॑ यजेत।
सर्व॑पृष्ठो\-ऽस्य॒ सोमः॑ स्यात्।
सर्वा᳚भ्य ए॒व दे॒वता᳚भ्यः॒ सर्वे᳚भ्यः पृ॒ष्ठेभ्य॑ आ॒त्मानं॒ निष्क्री॑णीते॥४५॥\anuvakamend[उ॒द्वाय॑ति मन्थेन्मन्थत्यक्रामत्प॒रा\-ऽप॑तन्म॒ध्यन्दि॑न आगु॒रते॒ पञ्च॑ च]

%1.4.8.1
पव॑मानः॒ सुव॒र्जनः॑।
प॒वित्रे॑ण॒ विच॑र्\mbox{}षणिः।
यः पोता॒ स पु॑नातु मा।
पु॒नन्तु॑ मा देवज॒नाः।
पु॒नन्तु॒ मन॑वो धि॒या।
पु॒नन्तु॒ विश्व॑ आ॒यवः॑।
जात॑वेदः प॒वित्र॑वत्।
प॒वित्रे॑ण पुनाहि मा।
शु॒क्रेण॑ देव॒ दीद्य॑त्।
अग्ने॒ क्रत्वा॒ क्रतू॒ꣳ॒ रनु॑\-॥४६॥

%1.4.8.2
यत्ते॑ प॒वित्र॑म॒र्चिषि॑।
अग्ने॒ वित॑तमन्त॒रा।
ब्रह्म॒ तेन॑ पुनीमहे।
उ॒भाभ्यां᳚ देव सवितः।
प॒वित्रे॑ण स॒वेन॑ च।
इ॒दं ब्रह्म॑ पुनीमहे।
वै॒श्व॒दे॒वी पु॑न॒ती दे॒व्यागा᳚त्।
यस्यै॑ ब॒ह्वीस्त॒नुवो॑ वी॒तपृ॑ष्ठाः।
तया॒ मद॑न्तः सध॒माद्ये॑षु।
व॒यꣴ स्या॑म॒ पत॑यो रयी॒णाम्॥४७॥

%1.4.8.3
वै॒श्वा॒न॒रो र॒श्मिभि॑र्मा पुनातु।
वातः॑ प्रा॒णेने॑षि॒रो म॑यो॒भूः।
द्यावा॑पृथि॒वी पय॑सा॒ पयो॑भिः।
ऋ॒ताव॑री य॒ज्ञिये॑ मा पुनीताम्।
बृ॒हद्भिः॑ सवित॒स्तृभिः॑।
वर्\mbox{}षि॑ष्ठैर्देव॒ मन्म॑भिः।
अग्ने॒ दक्षैः᳚ पुनाहि मा।
येन॑ दे॒वा अपु॑नत।
येनाऽऽपो॑ दि॒व्यं कशः॑।
तेन॑ दि॒व्येन॒ ब्रह्म॑णा॥४८॥

%1.4.8.4
इ॒दं ब्रह्म॑ पुनीमहे।
यः पा॑वमा॒नीर॒ध्येति॑।
ऋषि॑भिः॒ सम्भृ॑त॒ꣳ॒ रसम्᳚।
सर्व॒ꣳ॒ स पू॒तम॑श्ञाति।
स्व॒दि॒तं मा॑त॒रिश्व॑ना।
पा॒व॒मा॒नीर्यो अ॒ध्येति॑।
ऋषि॑भिः॒ सम्भृ॑त॒ꣳ॒ रसम्᳚।
तस्मै॒ सर॑स्वती दुहे।
क्षी॒रꣳ स॒र्पिर्मधू॑द॒कम्।
पा॒व॒मा॒नीः स्व॒स्त्यय॑नीः॥४९॥

%1.4.8.5
सु॒दुघा॒ हि पय॑स्वतीः।
ऋषि॑भिः॒ सम्भृ॑तो॒ रसः॑।
ब्रा॒ह्म॒णेष्व॒मृतꣳ॑ हि॒तम्।
पा॒व॒मा॒नीर्दि॑शन्तु नः।
इ॒मं लो॒कमथो॑ अ॒मुम्।
कामा॒न्थ्सम॑र्धयन्तु नः।
दे॒वीर्दे॒वैः स॒माभृ॑ताः।
पा॒व॒मा॒नीः स्व॒स्त्यय॑नीः।
सु॒दुघा॒ हि घृ॑त॒श्चुतः॑।
ऋषि॑भिः॒ सम्भृ॑तो॒ रसः॑॥५०॥

%1.4.8.6
ब्रा॒ह्म॒णेष्व॒मृतꣳ॑ हि॒तम्।
येन॑ दे॒वाः प॒वित्रे॑ण।
आ॒त्मानं॑ पु॒नते॒ सदा᳚।
तेन॑ स॒हस्र॑धारेण।
पा॒व॒मा॒न्यः पु॑नन्तु मा।
प्रा॒जा॒प॒त्यं प॒वित्रम्᳚।
श॒तोद्या॑मꣳ हिर॒ण्मयम्᳚।
तेन॑ ब्रह्म॒विदो॑ व॒यम्।
पू॒तं ब्रह्म॑ पुनीमहे।
इन्द्रः॑ सुनी॒ती स॒ह मा॑ पुनातु।
सोमः॑ स्व॒स्त्या वरु॑णः स॒मीच्या᳚।
य॒मो राजा᳚ प्रमृ॒णाभिः॑ पुनातु मा।
जा॒तवे॑दा मो॒र्जय॑न्त्या पुनातु॥५१॥\anuvakamend[अनु॑ रयी॒णां ब्रह्म॑णा स्व॒स्त्यय॑नीः सु॒दुघा॒ हि घृ॑त॒श्चुत॒ ऋषि॑भिः॒ सम्भृ॑तो॒ रसः॑ पुनातु॒ त्रीणि॑ च]

%1.4.9.1
प्र॒जा वै स॒त्रमा॑सत॒ तप॒स्तप्य॑माना॒ अजु॑ह्वतीः।
दे॒वा अ॑पश्यञ्चम॒सं घृ॒तस्य॑ पू॒र्णꣴ स्व॒धाम्।
तमुपोद॑तिष्ठ॒न्तम॑\-जुहवुः।
तेना᳚र्धमा॒स ऊर्ज॒मवा॑रुन्धत।
तस्मा॑दर्धमा॒से दे॒वा इ॑ज्यन्ते।
पि॒तरो॑\-ऽपश्यञ्चम॒सं घृ॒तस्य॑ पू॒र्णꣴ स्व॒धाम्।
तमुपोद॑तिष्ठ॒न्तम॑जुहवुः।
तेन॑ मा॒स्यूर्ज॒मवा॑रुन्धत।
तस्मा᳚न्मा॒सि पि॒तृभ्यः॑ क्रियते।
म॒नु॒ष्या॑ अपश्यञ्चम॒सं घृ॒तस्य॑ पू॒र्णꣴ स्व॒धाम्॥५२॥

%1.4.9.2
तमुपोद॑तिष्ठ॒न्तम॑जुहवुः।
तेन॑ द्व॒यीमूर्ज॒मवा॑रुन्धत।
तस्मा॒द्द्विरह्नो॑ मनु॒ष्ये᳚भ्य॒ उप॑ह्रियते।
प्रा॒तश्च॑ सा॒यं च॑।
प॒शवो॑\-ऽपश्यञ्चम॒सं घृ॒तस्य॑ पू॒र्णꣴ स्व॒धाम्।
तमुपोद॑तिष्ठ॒न्त\-म॑जुहवुः।
तेन॑ त्र॒यीमूर्ज॒मवा॑रुन्धत।
तस्मा॒त्त्रिरह्नः॑ प॒शवः॒ प्रेर॑ते।
प्रा॒तः स॑ङ्ग॒वे सा॒यम्।
असु॑रा अपश्यञ्चम॒सं घृ॒तस्य॑ पू॒र्णꣴ स्व॒धाम्॥५३॥

%1.4.9.3
तमुपोद॑तिष्ठ॒न्तम॑जुहवुः।
तेन॑ संवथ्स॒र ऊर्ज॒मवा॑रुन्धत।
ते दे॒वा अ॑मन्यन्त।
अ॒मी वा इ॒दम॑भूवन्।
यद्व॒यꣴ स्म इति॑।
त ए॒तानि॑ चातुर्मा॒स्यान्य॑पश्यन्।
तानि॒ निर॑वपन्।
तैरे॒वैषां तामूर्ज॑मवृञ्जत।
ततो॑ दे॒वा अभ॑वन्।
पराऽसु॑राः॥५४॥

%1.4.9.4
यद्यज॑ते।
यामे॒व दे॒वा ऊर्ज॑म॒वारु॑न्धत।
तान्तेनाव॑ रुन्धे।
यत्पि॒तृभ्यः॑ क॒रोति॑।
यामे॒व पि॒तर॒ ऊर्ज॑म॒वारु॑न्धत।
तान्तेनाव॑ रुन्धे।
यदा॑वस॒थे\-ऽन्न॒ꣳ॒ हर॑न्ति।
यामे॒व म॑नु॒ष्या॑ ऊर्ज॑म॒वारु॑न्धत।
तान्तेनाव॑ रुन्धे।
यद्दक्षि॑णां॒ ददा॑ति॥५५॥

%1.4.9.5
यामे॒व प॒शव॒ ऊर्ज॑म॒वारु॑न्धत।
तान्तेनाव॑ रुन्धे।
यच्चा॑तुर्मा॒स्यैर्\-यज॑ते।
यामे॒वासु॑रा॒ ऊर्ज॑म॒वारु॑न्धत।
तान्तेनाव॑ रुन्धे।
भव॑त्या॒त्मना᳚।
परा᳚स्य॒ भ्रातृ॑व्यो भवति।
वि॒राजो॒ वा ए॒षा विक्रा᳚न्तिः।
यच्चा॑तुर्मा॒स्यानि॑।
वै॒श्व॒दे॒वेना॒स्मिँल्लो॒के प्रत्य॑तिष्ठत्।
व॒रु॒ण॒प्र॒घा॒सैर॒न्तरि॑क्षे।
सा॒क॒मे॒धैर॒मुष्मिँ॑ल्लो॒के।
ए॒ष ह॒ त्वावैतथ्सर्वं॑ भवति।
य ए॒वं वि॒द्वाꣴश्चा॑तुर्मा॒स्यैर्यज॑ते॥५६॥\anuvakamend[म॒नु॒ष्या॑ अपश्यञ्चम॒सं घृ॒तस्य॑ पू॒र्णꣴ स्व॒धामसु॑रा अपश्यञ्चम॒सं घृ॒तस्य॑ पू॒र्णꣴ स्व॒धामसु॑रा॒ ददा᳚त्यतिष्ठच्च॒त्वारि॑ च]

%1.4.10.1
अ॒ग्निर्वाव सं॑वथ्स॒रः।
आ॒दि॒त्यः प॑रिवथ्स॒रः।
च॒न्द्रमा॑ इदावथ्स॒रः।
वा॒युर॑नुवथ्स॒रः।
यद्वै᳚श्वदे॒वेन॒ यज॑ते।
अ॒ग्निमे॒व तथ्सं॑वथ्स॒रमा᳚प्नोति।
तस्मा᳚द्वैश्वदे॒वेन॒ यज॑मानः।
सं॒व॒थ्स॒रीणाꣴ॑ स्व॒स्तिमा शा᳚स्त॒ इत्याशा॑सीत।
यद्व॑रुण\-प्रघा॒सैर्यज॑ते।
आ॒दि॒त्यमे॒व तत्प॑रिवथ्स॒रमा᳚प्नोति॥५७॥

%1.4.10.2
तस्मा᳚द्वरुणप्रघा॒सैर्यज॑मानः।
प॒रि॒व॒थ्स॒रीणाꣴ॑ स्व॒स्तिमा शा᳚स्त॒ इत्याशा॑सीत।
यथ्सा॑कमे॒धैर्यज॑ते।
च॒न्द्रम॑समे॒व तदि॑दावथ्स॒र\-मा᳚प्नोति।
तस्मा᳚थ्साकमे॒धैर्यज॑मानः।
इ॒दा॒\-व॒थ्स॒रीणाꣴ॑ स्व॒स्तिमा शा᳚स्त॒ इत्याशा॑सीत।
यत्पि॑तृय॒ज्ञेन॒ यज॑ते।
दे॒वाने॒व तद॒न्वव॑स्यति।
अथ॒ वा अ॑स्य वा॒युश्चा॑नु\-वथ्स॒रश्चाप्री॑ता॒\-वुच्छि॑ष्येते।
यच्छु॑नासी॒रीये॑ण॒ यज॑ते॥५८॥

%1.4.10.3
वा॒युमे॒व तद॑नुवथ्स॒रमा᳚प्नोति।
तस्मा᳚च्छुनासी॒रीये॑ण॒ यज॑मानः।
अ॒नु॒व॒थ्स॒रीणाꣴ॑ स्व॒स्तिमा शा᳚स्त॒ इत्याशा॑सीत।
सं॒व॒थ्स॒रं वा ए॒ष ई᳚फ्स॒तीत्या॑हुः।
यश्चा॑तुर्मा॒स्यैर्यज॑त॒ इति॑।
ए॒ष ह॒ त्वै सं॑वथ्स॒रमा᳚प्नोति।
य ए॒वं वि॒द्वाꣴश्चा॑तुर्मा॒स्यैर्यज॑ते।
विश्वे॑ दे॒वाः सम॑यजन्त।
ते᳚ऽग्निमे॒वाय॑जन्त।
त ए॒तं लो॒कम॑जयन्॥५९॥

%1.4.10.4
यस्मि॑न्न॒ग्निः।
यद्वै᳚श्वदे॒वेन॒ यज॑ते।
ए॒तमे॒व लो॒कं ज॑यति।
यस्मि॑न्न॒ग्निः।
अ॒ग्नेरे॒व सायु॑ज्य॒मुपै॑ति।
य॒दा वै᳚श्वदे॒वेन॒ यज॑ते।
अथ॑ संवथ्स॒रस्य॑ गृ॒हप॑तिमाप्नोति।
य॒दा सं॑वथ्स॒रस्य॑ गृ॒हप॑तिमा॒प्नोति॑।
अथ॑ सहस्रया॒जिन॑माप्नोति।
य॒दा स॑हस्रया॒जिन॑मा॒प्नोति॑॥६०॥

%1.4.10.5
अथ॑ गृहमे॒धिन॑माप्नोति।
य॒दा गृ॑हमे॒धिन॑मा॒प्नोति॑।
अथा॒ग्निर्भ॑वति।
य॒दाग्निर्भव॑ति।
अथ॒ गौर्भ॑वति।
ए॒षा वै वै᳚श्वदे॒वस्य॒ मात्रा᳚।
ए॒तद्वा ए॒तेषा॑मव॒मम्।
अतो॑तो॒ वा उत्त॑राणि॒ श्रेयाꣳ॑सि भवन्ति।
यद्विश्वे॑ दे॒वाः स॒मय॑जन्त।
तद्वै᳚श्वदे॒वस्य॑ वैश्वदेव॒त्वम्॥६१॥

%1.4.10.6
अथा॑ऽऽदि॒त्यो वरु॑ण॒ꣳ॒ रा॑जानं वरुणप्रघा॒सैर॑यजत।
स ए॒तं लो॒कम॑जयत्।
यस्मि॑न्नादि॒त्यः।
यद्व॑रुणप्रघा॒सैर्यज॑ते।
ए॒तमे॒व लो॒कं ज॑यति।
यस्मि॑न्नादि॒त्यः।
आ॒दि॒त्यस्यै॒व सायु॑ज्य॒मुपै॑ति।
यदा॑दि॒त्यो वरु॑ण॒ꣳ॒ राजा॑नं वरुणप्रघा॒सै\-रय॑जत।
तद्व॑रुणप्रघा॒सानां᳚ वरुणप्रघास॒त्वम्।
अथ॒ सोमो॒ राजा॒ छन्दाꣳ॑सि साकमे॒धैर॑यजत॥६२॥

%1.4.10.7
स ए॒तं लो॒कम॑जयत्।
यस्मिꣴ॑श्च॒न्द्रमा॑ वि॒भाति॑।
यथ्सा॑कमे॒धैर्यज॑ते।
ए॒तमे॒व लो॒कं ज॑यति।
यस्मिꣴ॑श्च॒न्द्रमा॑ वि॒भाति॑।
च॒न्द्रम॑स ए॒व सायु॑ज्य॒मुपै॑ति।
सोमो॒ वै च॒न्द्रमाः᳚।
ए॒ष ह॒ त्वै सा॒क्षाथ्सोमं॑ भक्षयति।
य ए॒वं वि॒द्वान्थ्सा॑कमे॒धैर्यज॑ते।
यथ्सोम॑श्च॒ राजा॒ छन्दाꣳ॑सि च स॒मैध॑न्त॥६३॥

%1.4.10.8
तथ्सा॑कमे॒धानाꣳ॑ साकमेध॒त्वम्।
अथ॒र्तवः॑ पि॒तरः॑ प्र॒जा\-प॑तिं पि॒तरं॑ पितृय॒ज्ञेना॑यजन्त।
त ए॒तं लो॒कम॑जयन्।
यस्मि॑न्नृ॒तवः॑।
यत्पि॑तृय॒ज्ञेन॒ यज॑ते।
ए॒तमे॒व लो॒कं ज॑यति।
यस्मि॑न्नृ॒तवः॑।
ऋ॒तू॒नामे॒व सायु॑ज्य॒मुपै॑ति।
यदृ॒तवः॑ पि॒तरः॑ प्र॒जा\-प॑तिं पि॒तरं॑ पितृय॒ज्ञेनाय॑जन्त।
तत्पि॑तृय॒ज्ञस्य॑ पितृयज्ञ॒त्वम्॥६४॥

%1.4.10.9
अथौष॑धय इ॒मं दे॒वं त्र्य॑म्बकैरयजन्त॒ प्रथे॑म॒हीति॑।
ततो॒ वै ता अ॑प्रथन्त।
य ए॒वं वि॒द्वाꣴस्त्र्य॑म्बकै॒र्यज॑ते।
प्रथ॑ते प्र॒जया॑ प॒शुभिः॑।
अथ॑ वा॒युः प॑रमे॒ष्ठिनꣳ॑ शुनासी॒रीये॑णायजत।
स ए॒तं लो॒कम॑जयत्।
यस्मि॑न्वा॒युः।
यच्छु॑नासी॒रीये॑ण॒ यज॑ते।
ए॒तमे॒व लो॒कं ज॑यति।
यस्मि॑न्वा॒युः॥६५॥

%1.4.10.10
वा॒योरे॒व सायु॑ज्य॒मुपै॑ति।
ब्र॒ह्म॒वा॒दिनो॑ वदन्ति।
प्र चा॑तुर्मास्यया॒जी मी॑य॒ता (३) न प्रमी॑य॒ता (३) इति॑।
जीव॒न्वा ए॒ष ऋ॒तूनप्ये॑ति।
यदि॑ व॒सन्ता᳚ प्र॒मीय॑ते।
व॒स॒न्तो भ॑वति।
यदि॑ ग्री॒ष्मे ग्री॒ष्मः।
यदि॑ व॒र्॒षासु॑ व॒र्॒षाः।
यदि॑ श॒रदि॑ श॒रत्।
यदि॒ हेम॑न् हेम॒न्तः।
ऋ॒तुर्भू॒त्वा सं॑वथ्स॒रमप्ये॑ति।
सं॒व॒थ्स॒रः प्र॒जा\-प॑तिः।
प्र॒जा\-प॑ति॒र्वावैषः॥६६॥\anuvakamend[प॒रि॒व॒थ्स॒रमा᳚प्नोति शुनासी॒रीये॑ण॒ यज॑ते\-ऽजयन्थ्सहस्रया॒जिन॑मा॒प्नोति॑ वैश्वदेव॒त्वꣳ सा॑कमे॒धैर॑यजत स॒मैध॑न्त पितृयज्ञ॒त्वं ज॑यति॒ यस्मि॑न्वा॒युर्\mbox{}हे॑म॒न्तस्त्रीणि॑ च]






\prashnaend{उ॒भये॑ यु॒वꣳ सु॒राम॒मुद॑स्था॒न्नि वै यस्य॑ प्रातः सव॒न एकै॑को\-ऽसु॒र्यं॑ पव॑मानः प्र॒जा वै स॒त्रमा॑सता॒ग्निर्वाव सं॑वथ्स॒रो दश॑॥१०॥}{उ॒भये॒ वा उद॑स्था॒थ्सर्वा॑भिर्मध्य॒तो\-ऽत्र॒ वाव ब्रा᳚ह्म॒णेष्वथ॑ गृहमे॒धिन॒ꣳ॒ षट्थ्ष॑ष्टिः॥६६॥}{उ॒भये॒ वा वैषः॥}{हरिः॑ ओम्॥}{इति श्रीकृष्णयजुर्वेदीयतैत्तिरीयब्राह्मणे प्रथमाष्टके चतुर्थः प्रपाठकः समाप्तः॥}
\clearpage
\sect{पञ्चमः प्रश्नः}
\setcounter{anuvakam}{0}
\dnsub{तैत्तिरीयब्राह्मणे प्रथमाष्टके पञ्चमः प्रपाठकः}

%1.5.1.1
अ॒ग्नेः कृत्ति॑काः।
शु॒क्रं प॒रस्ता॒ज्ज्योति॑र॒वस्ता᳚त्।
प्र॒जा\-प॑ते रोहि॒णी।
आपः॑ प॒रस्ता॒दोष॑धयो॒\-ऽवस्ता᳚त्।
सोम॑स्येन्व॒का वित॑तानि।
प॒रस्ता॒द्वय॑न्तो॒\-ऽवस्ता᳚त्।
रु॒द्रस्य॑ बा॒हू।
मृ॒ग॒यवः॑ प॒रस्ता᳚द्विक्षा॒रो॑\-ऽवस्ता᳚त्।
अदि॑त्यै॒ पुन॑र्वसू।
वातः॑ प॒रस्ता॑दा॒र्द्रम॒वस्ता᳚त्॥१॥

%1.5.1.2
बृह॒स्पते᳚स्ति॒ष्यः॑।
जुह्व॑तः प॒रस्ता॒द्यज॑माना अ॒वस्ता᳚त्।
स॒र्पाणा॑माश्रे॒षाः।
अ॒भ्या॒गच्छ॑न्तः प॒रस्ता॑दभ्या॒नृत्य॑न्तो॒\-ऽवस्ता᳚त्।
पि॒तृ॒णां म॒घाः।
रु॒दन्तः॑ प॒रस्ता॑दपभ्र॒ꣳ॒शो॑\-ऽवस्ता᳚त्।
अ॒र्य॒म्णः पूर्वे॒ फल्गु॑नी।
जा॒या प॒रस्ता॑दृष॒भो॑\-ऽवस्ता᳚त्।
भग॒स्योत्त॑रे।
व॒ह॒तवः॑ प॒रस्ता॒द्वह॑माना अ॒वस्ता᳚त्॥२॥

%1.5.1.3
दे॒वस्य॑ सवि॒तुर्\mbox{}हस्तः॑।
प्र॒स॒वः प॒रस्ता᳚थ्स॒निर॒वस्ता᳚त्।
इन्द्र॑स्य चि॒त्रा।
ऋ॒तं प॒र\-स्ता᳚थ्\-स॒त्यम॒वस्ता᳚त्।
वा॒योर्निष्ट्या᳚ व्र॒ततिः॑।
प॒रस्ता॒दसि॑द्धिर॒वस्ता᳚त्।
इ॒न्द्रा॒ग्नि॒योर्विशा॑खे।
यु॒गानि॑ प॒रस्ता᳚त्कृ॒षमा॑णा अ॒वस्ता᳚त्।
मि॒त्रस्या॑नूरा॒धाः।
अ॒भ्या॒रोह॑त्प॒रस्ता॑\-द॒भ्यारू॑ढम॒वस्ता᳚त्॥३॥

%1.5.1.4
इन्द्र॑स्य रोहि॒णी।
शृ॒णत्प॒रस्ता᳚त्प्रतिशृ॒णद॒वस्ता᳚त्।
निर्\mbox{}ऋ॑त्यै मूल॒वर्\mbox{}ह॑णी।
प्र॒ति॒भ॒ञ्जन्तः॑ प॒रस्ता᳚त्प्रतिशृ॒णन्तो॒\-ऽवस्ता᳚त्।
अ॒पां पूर्वा॑ अषा॒ढाः।
वर्चः॑ प॒रस्ता॒थ्समि॑तिर॒वस्ता᳚त्।
विश्वे॑षां दे॒वाना॒मुत्त॑राः।
अ॒भि॒जय॑त्प॒रस्ता॑द॒भिजि॑तम॒वस्ता᳚त्।
विष्णोः᳚ श्रो॒णा पृ॒च्छमा॑नाः।
प॒रस्ता॒त्पन्था॑ अ॒वस्ता᳚त्॥४॥

%1.5.1.5
वसू॑ना॒ꣴ॒ श्रवि॑ष्ठाः।
भू॒तं प॒रस्ता॒द्भूति॑र॒वस्ता᳚त्।
इन्द्र॑स्य श॒तभि॑षक्।
वि॒श्वव्य॑चाः प॒रस्ता᳚द्वि॒श्वक्षि॑तिर॒वस्ता᳚त्।
अ॒जस्यैक॑पदः॒ पूर्वे᳚ प्रोष्ठप॒दाः।
वै॒श्वा॒न॒रं प॒रस्ता᳚द्वैश्वावस॒वम॒\-वस्ता᳚त्।
अहे᳚र्बु॒ध्निय॒स्यो\-त्त॑रे।
अ॒भि॒षि॒ञ्चन्तः॑ प॒रस्ता॑दभि\-षु॒ण्वन्तो॒\-ऽवस्ता᳚त्।
पू॒ष्णो रे॒वती᳚।
गावः॑ प॒रस्ता᳚द्व॒थ्सा अ॒वस्ता᳚त्।
अ॒श्विनो॑रश्व॒युजौ᳚।
ग्रामः॑ प॒रस्ता॒थ्सेना॒\-ऽवस्ता᳚त्।
य॒मस्या॑प॒भर॑णीः।
अ॒प॒कर्\mbox{}ष॑न्तः प॒रस्ता॑दप॒वह॑न्तो॒\-ऽवस्ता᳚त्।
पू॒र्णा प॒श्चाद्यत्ते॑ दे॒वा अद॑धुः॥५॥\anuvakamend[आ॒र्द्रम॒वस्ता॒द्वह॑माना अ॒वस्ता॑द॒भ्यारू॑ढम॒वस्ता॒त्पन्था॑ अ॒वस्ता᳚द्व॒थ्सा अ॒वस्ता॒त्पञ्च॑ च]

%1.5.2.1
यत्पुण्यं॒ नक्ष॑त्रम्।
तद्बट्कु॑र्वीतोपव्यु॒षम्।
य॒दा वै सूर्य॑ उ॒देति॑।
अथ॒ नक्ष॑त्रं॒ नैति॑।
याव॑ति॒ तत्र॒ सूर्यो॒ गच्छे᳚त्।
यत्र॑ जघ॒न्यं॑ पश्ये᳚त्।
ताव॑ति कुर्वीत यत्का॒री स्यात्।
पु॒ण्या॒ह ए॒व कु॑रुते।
ए॒वꣳ ह॒ वै य॒ज्ञेषुं॑ च श॒तद्यु॑म्नं च मा॒थ्स्यो नि॑रवसाय॒यां च॑कार॥६॥

%1.5.2.2
यो वै न॑क्ष॒त्रियं॑ प्र॒जा\-प॑तिं॒ वेद॑।
उ॒भयो॑रेनं लो॒कयो᳚र्विदुः।
हस्त॑ ए॒वास्य॒ हस्तः॑।
चि॒त्रा शिरः॑।
निष्ट्या॒ हृद॑यम्।
ऊ॒रू विशा॑खे।
प्र॒ति॒ष्ठा\-ऽनू॑रा॒धाः।
ए॒ष वै न॑क्ष॒त्रियः॑ प्र॒जा\-प॑तिः।
य ए॒वं वेद॑।
उ॒भयो॑रेनं लो॒कयो᳚र्विदुः॥७॥

%1.5.2.3
अ॒स्मिꣴश्चा॒मुष्मिꣴ॑श्च।
यां का॒मये॑त दुहि॒तरं॑ प्रि॒या स्या॒दिति॑।
तां निष्ट्या॑यां दद्यात्।
प्रि॒यैव भ॑वति।
नेव॒ तु पुन॒राग॑च्छति।
अ॒भि॒जिन्नाम॒ नक्ष॑त्रम्।
उ॒परि॑ष्टादषा॒ढाना᳚म्।
अ॒वस्ता᳚च्छ्रो॒णायै᳚।
दे॒वा॒सु॒राः संय॑त्ता आसन्।
ते दे॒वास्तस्मि॒न्नक्ष॑त्रे॒\-ऽभ्य॑जयन्॥८॥

%1.5.2.4
यद॒भ्यज॑यन्।
तद॑भि॒जितो॑\-ऽभिजि॒त्त्वम्।
यं का॒मये॑तानप\-ज॒य्यं ज॑ये॒दिति॑।
तमे॒तस्मि॒न्नक्ष॑त्रे यातयेत्।
अ॒न॒प॒ज॒य्यमे॒व ज॑यति।
पा॒पप॑राजितमिव॒ तु।
प्र॒जा\-प॑तिः प॒शून॑\-सृजत।
ते नक्ष॑त्रं नक्षत्र॒मुपा॑तिष्ठन्त।
ते स॒माव॑न्त ए॒वाभ॑वन्।
ते रे॒वती॒मुपा॑तिष्ठन्त॥९॥

%1.5.2.5
ते रे॒वत्यां॒ प्राभ॑वन्।
तस्मा᳚द्रे॒वत्यां᳚ पशू॒नां कु॑र्वीत।
यत्किं चा᳚र्वा॒चीन॒ꣳ॒ सोमा᳚त्।
प्रैव भ॑वन्ति।
स॒लि॒लं वा इ॒दम॑न्त॒रासी᳚त्।
यदत॑रन्।
तत्तार॑काणां तारक॒त्वम्।
यो वा इ॒ह यज॑ते।
अ॒मुꣳ स लो॒कं न॑क्षते।
तन्नक्ष॑त्राणां नक्षत्र॒त्वम्॥१०॥

%1.5.2.6
दे॒व॒गृ॒हा वै नक्ष॑त्राणि।
य ए॒वं वेद॑।
गृ॒ह्ये॑व भ॑वति।
यानि॒ वा इ॒मानि॑ पृथि॒व्याश्चि॒त्राणि॑।
तानि॒ नक्ष॑त्राणि।
तस्मा॑दश्ली॒लना॑मꣴश्चि॒त्रे।
नाव॑स्ये॒न्न य॑जेत।
यथा॑ पापा॒हे कु॑रु॒ते।
ता॒दृगे॒व तत्।
दे॒व॒न॒क्ष॒त्राणि॒ वा अ॒न्यानि॑॥११॥

%1.5.2.7
य॒म॒न॒क्ष॒त्राण्य॒न्यानि॑।
कृत्ति॑काः प्रथ॒मम्।
विशा॑खे उत्त॒मम्।
तानि॑ देवनक्ष॒त्राणि॑।
अ॒नू॒रा॒धाः प्र॑थ॒मम्।
अ॒प॒भर॑णीरुत्त॒मम्।
तानि॑ यमनक्ष॒त्राणि॑।
यानि॑ देवनक्ष॒त्राणि॑।
तानि॒ दक्षि॑णेन॒ परि॑यन्ति।
यानि॑ यमनक्ष॒त्राणि॑॥१२॥

%1.5.2.8
तान्युत्त॑रेण।
अन्वे॑षामरा॒थ्स्मेति॑।
तद॑नूरा॒धाः।
ज्ये॒ष्ठमे॑षाम\-वधि॒ष्मेति॑।
तज्ज्ये᳚ष्ठ॒घ्नी।
मूल॑मेषामवृक्षा॒मेति॑।
तन्मू॑ल॒वर्\mbox{}ह॑णी।
यन्नास॑हन्त।
तद॑षा॒ढाः।
यदश्लो॑णत्॥१३॥

%1.5.2.9
तच्छ्रो॒णा।
यदशृ॑णोत्।
तच्छ्रवि॑ष्ठाः।
यच्छ॒तमभि॑षज्यन्।
तच्छ॒तभि॑षक्।
प्रो॒ष्ठ॒प॒देषूद॑यच्छन्त।
रे॒वत्या॑मरवन्त।
अ॒श्व॒युजो॑र\-युञ्जत।
अ॒प॒भर॑णी॒ष्वपा॑वहन्।
तानि॒ वा ए॒तानि॑ यमनक्ष॒त्राणि॑।
यान्ये॒व दे॑वनक्ष॒त्राणि॑।
तेषु॑ कुर्वीत यत्का॒री स्यात्।
पु॒ण्या॒ह ए॒व कु॑रुते॥१४॥\anuvakamend[च॒का॒रै॒वं वेदो॒भयो॑रेनं लो॒कयो᳚र्विदुरजयन्रे॒वती॒मुपा॑तिष्ठन्त नक्षत्र॒त्वम॒न्यानि॒ यानि॑ यमनक्ष॒त्राण्यश्लो॑णद्यम\-नक्ष॒त्राणि॒ त्रीणि॑ च]

%1.5.3.1
दे॒वस्य॑ सवि॒तुः प्रा॒तः प्र॑स॒वः प्रा॒णः।
वरु॑णस्य सा॒यमा॑स॒वो॑\-ऽपा॒नः।
यत्प्र॑ती॒चीनं॑ प्रात॒स्तना᳚त्।
प्रा॒चीनꣳ॑ सङ्ग॒वात्।
ततो॑ दे॒वा अ॑ग्निष्टो॒मं निर॑मिमत।
तत्तदात्त॑वीर्यं निर्मा॒र्गः।
मि॒त्रस्य॑ सङ्ग॒वः।
तत्पुण्यं॑ तेज॒स्व्यहः॑।
तस्मा॒त्तर्\mbox{}हि॑ प॒शवः॑ स॒माय॑न्ति।
यत्प्र॑ती॒चीनꣳ॑ सङ्ग॒वात्॥१५॥

%1.5.3.2
प्रा॒चीनं॑ म॒ध्यं दि॑नात्।
ततो॑ दे॒वा उ॒क्थ्यं॑ निर॑मिमत।
तत्तदात्त॑वीर्यं निर्मा॒र्गः।
बृह॒स्पते᳚र्म॒ध्यं दि॑नः।
तत्पुण्यं॑ तेज॒स्व्यहः॑।
तस्मा॒त्तर्\mbox{}हि॒ तेक्ष्णि॑ष्ठं तपति।
यत्प्र॑ती॒चीनं॑ म॒ध्यं दि॑नात्।
प्रा॒चीन॑\-मपरा॒ह्णात्।
ततो॑ दे॒वाः षो॑ड॒शिनं॒ निर॑मिमत।
तत्तदात्त॑वीर्यं निर्मा॒र्गः॥१६॥

%1.5.3.3
भग॑स्यापरा॒ह्णः।
तत्पुण्यं॑ तेज॒स्व्यहः॑।
तस्मा॑दपरा॒ह्णे कु॑मा॒र्यो॑ भग॑मि॒च्छमा॑नाश्चरन्ति।
यत्प्र॑ती॒चीन॑मपरा॒ह्णात्।
प्रा॒चीनꣳ॑ सा॒यात्।
ततो॑ दे॒वा अ॑तिरा॒त्रं निर॑मिमत।
तत्तदात्त॑वीर्यं निर्मा॒र्गः।
वरु॑णस्य सा॒यम्।
तत्पुण्यं॑ तेज॒स्व्यहः॑।
तस्मा॒त्तर्\mbox{}हि॒ नानृ॑तं वदेत्॥१७॥

%1.5.3.4
ब्रा॒ह्म॒णो वा अ॑ष्टावि॒ꣳ॒शो नक्ष॑त्राणाम्।
स॒मा॒नस्याह्नः॒ पञ्च॒ पुण्या॑नि॒ नक्ष॑त्राणि।
च॒त्वार्य॑श्ली॒लानि॑।
तानि॒ नव॑।
यच्च॑ प॒रस्ता॒न्नक्ष॑त्राणां॒ यच्चा॒वस्ता᳚त्।
तान्येका॑दश।
ब्रा॒ह्म॒णो द्वा॑द॒शः।
य ए॒वं वि॒द्वान्थ्सं॑वथ्स॒रं व्र॒तं चर॑ति।
सं॒व॒थ्स॒रेणै॒वास्य॑ व्र॒तं गु॒प्तं भ॑वति।
स॒मा॒नस्याह्नः॒ पञ्च॒ पुण्या॑नि॒ नक्ष॑त्राणि।
च॒त्वार्य॑श्ली॒लानि॑।
तानि॒ नव॑।
आ॒ग्ने॒यी रात्रिः॑।
ऐ॒न्द्रमहः॑।
तान्येका॑दश।
आ॒दि॒त्यो द्वा॑द॒शः।
य ए॒वं वि॒द्वान्थ्सं॑वथ्स॒रं व्र॒तं चर॑ति।
सं॒व॒थ्स॒रेणै॒वास्य॑ व्र॒तं गु॒प्तं भ॑वति॥१८॥\anuvakamend[स॒ङ्ग॒वाथ्षो॑ड॒शिनं॒ निर॑मिमत॒ तत्तदात्त॑वीर्यं निर्मा॒र्गो व॑देद्भवति समा॒नस्याह्नः॒ पञ्च॒ पुण्या॑नि॒ नक्ष॑त्राण्य॒ष्टौ च॑]

%1.5.4.1
ब्र॒ह्म॒वा॒दिनो॑ वदन्ति।
कति॒ पात्रा॑णि य॒ज्ञं व॑ह॒न्तीति॑।
त्रयो॑द॒शेति॑ ब्रूयात्।
स यद्ब्रू॒यात्।
कस्तानि॒ निर॑मिमी॒तेति॑।
प्र॒जा\-प॑ति॒रिति॑ ब्रूयात्।
स यद्ब्रू॒यात्।
कुत॒स्तानि॒ निर॑मिमी॒तेति॑।
आ॒त्मन॒ इति॑।
प्रा॒णा॒पा॒नाभ्या॑मे॒वोपाꣴ॑\-श्वन्तर्या॒मौ निर॑मिमीत॥१९॥

%1.5.4.2
व्या॒नादु॑पाꣳशु॒सव॑नम्।
वा॒च ऐ᳚न्द्रवाय॒वम्।
द॒क्ष॒क्र॒तुभ्यां᳚ मैत्रावरु॒णम्।
श्रोत्रा॑दाश्वि॒नम्।
चक्षु॑षः शु॒क्राम॒न्थिनौ᳚।
आ॒त्मन॑ आग्रय॒णम्।
अङ्गे᳚भ्य उ॒क्थ्यम्᳚।
आयु॑षो ध्रु॒वम्।
प्र॒ति॒ष्ठाया॑ ऋतुपा॒त्रे।
य॒ज्ञं वाव तं प्र॒जा\-प॑ति॒र्निर॑मिमीत।
स निर्मि॑तो॒ नाद्ध्रि॑यत॒ सम॑व्लीयत।
स ए॒तान्प्र॒जा\-प॑तिरपिवा॒पान॑पश्यत्।
तां निर॑वपत्।
तैर्वै स य॒ज्ञमप्य॑वपत्।
यद॑पिवा॒पा भव॑न्ति।
य॒ज्ञस्य॒ धृत्या॒ असं॑व्लयाय॥२०॥\anuvakamend[उ॒पा॒ꣳ॒श्व॒न्त॒र्या॒मौ निर॑मिमीतामिमीत॒ षट्च॑]

%1.5.5.1
ऋ॒तमे॒व प॑रमे॒ष्ठि।
ऋ॒तं नात्ये॑ति॒ किञ्च॒न।
ऋ॒ते स॑मु॒द्र आहि॑तः।
ऋ॒ते भूमि॑रि॒यꣴश्रि॒ता।
अ॒ग्निस्ति॒ग्मेन॑ शो॒चिषा᳚।
तप॒ आक्रा᳚न्तमु॒ष्णिहा᳚।
शिर॒स्तप॒स्याहि॑तम्।
वै॒श्वा॒न॒रस्य॒ तेज॑सा।
ऋ॒तेना᳚स्य॒ नि व॑र्तये।
स॒त्येन॒ परि॑ वर्तये।
तप॑सा॒\-ऽस्यानु॑ वर्तये।
शि॒वेना॒स्योप॑ वर्तये।
श॒ग्मेना᳚स्या॒भि व॑र्तये।
तदृ॒तं तथ्स॒त्यम्।
तद्व्र॒तं तच्छ॑केयम्।
तेन॑ शकेयं॒ तेन॑ राध्यासम्॥२१॥

%1.5.5.2
यद्\mbox{}घ॒र्मः प॒र्यव॑र्तयत्।
अन्ता᳚न्पृथि॒व्या दि॒वः।
अ॒ग्निरीशा॑न॒ ओज॑सा।
वरु॑णो धी॒तिभिः॑ स॒ह।
इन्द्रो॑ म॒रुद्भिः॒ सखि॑भिः स॒ह।
अ॒ग्निस्ति॒ग्मेन॑ शो॒चिषा᳚।
तप॒ आक्रा᳚न्तमु॒ष्णिहा᳚।
शिर॒स्तप॒स्याहि॑तम्।
वै॒श्वा॒न॒रस्य॒ तेज॑सा।
ऋ॒तेना᳚स्य॒ नि व॑र्तये।
स॒त्येन॒ परि॑ वर्तये।
तप॑सा॒\-ऽस्यानु॑ वर्तये।
शि॒वेना॒स्योप॑ वर्तये।
श॒ग्मेना᳚स्या॒भि व॑र्तये।
तदृ॒तं तथ्स॒त्यम्।
तद्व्र॒तं तच्छ॑केयम्।
तेन॑ शकेयं॒ तेन॑ राध्यासम्॥२२॥

%1.5.5.3
यो अ॒स्याः पृ॑थि॒व्यास्त्व॒चि।
नि॒व॒र्तय॒त्योष॑धीः।
अ॒ग्निरीशा॑न॒ ओज॑सा।
वरु॑णो धी॒तिभिः॑ स॒ह।
इन्द्रो॑ म॒रुद्भिः॒ सखि॑भिः स॒ह।
अ॒ग्निस्ति॒ग्मेन॑ शो॒चिषा᳚।
तप॒ आक्रा᳚न्तमु॒ष्णिहा᳚।
शिर॒स्तप॒स्याहि॑तम्।
वै॒श्वा॒न॒रस्य॒ तेज॑सा।
ऋ॒तेना᳚स्य॒ नि व॑र्तये।
स॒त्येन॒ परि॑ वर्तये।
तप॑सा॒\-ऽस्यानु॑ वर्तये।
शि॒वेना॒स्योप॑ वर्तये।
श॒ग्मेना᳚स्या॒भि व॑र्तये।
तदृ॒तं तथ्स॒त्यम्।
तद्व्र॒तं तच्छ॑केयम्।
तेन॑ शकेयं॒ तेन॑ राध्यासम्॥२३॥

%1.5.5.4
एकं॒ मास॒मुद॑\-सृजत्।
प॒र॒मे॒ष्ठी प्र॒जाभ्यः॑।
तेना᳚भ्यो॒ मह॒ आव॑हत्।
अ॒मृतं॒ मर्त्या᳚भ्यः।
प्र॒जामनु॒ प्र जा॑यसे।
तदु॑ ते मर्त्या॒मृतम्᳚।
येन॒ मासा॑ अर्धमा॒साः।
ऋ॒तवः॑ परिवथ्स॒राः।
येन॒ ते ते᳚ प्रजापते।
ई॒जा॒नस्य॒ न्यव॑र्तयन्।
तेना॒हम॒स्य ब्रह्म॑णा।
निव॑र्तयामि जी॒वसे᳚।
अ॒ग्निस्ति॒ग्मेन॑ शो॒चिषा᳚।
तप॒ आक्रा᳚न्तमु॒ष्णिहा᳚।
शिर॒स्तप॒स्याहि॑तम्।
वै॒श्वा॒न॒रस्य॒ तेज॑सा।
ऋ॒तेना᳚स्य॒ नि व॑र्तये।
स॒त्येन॒ परि॑ वर्तये।
तप॑सा॒\-ऽस्यानु॑ वर्तये।
शि॒वेना॒स्योप॑ वर्तये।
श॒ग्मेना᳚स्या॒भि व॑र्तये।
तदृ॒तं तथ्स॒त्यम्।
तद्व्र॒तं तच्छ॑केयम्।
तेन॑ शकेयं॒ तेन॑ राध्यासम्॥२४॥\anuvakamend[परि॑वर्तये स॒हाभिव॑र्तय उ॒ष्णिहा॑ राध्यासं॒ न्यव॑र्तय॒न्नुप॑वर्तये च॒त्वारि॑ च।
(ऋ॒तमे॒व षोड॑श।
यद्\mbox{}घ॒र्मो यो अ॒स्याः सप्तद॑शसप्तदश।
एकं॒ मासं॒ चतु॑र्विꣳशतिः)]

%1.5.6.1
दे॒वा वै यद्य॒ज्ञे\-ऽकु॑र्वत।
तदसु॑रा अकुर्वत।
तेऽसु॑रा ऊ॒र्ध्वं पृ॒ष्ठेभ्यो॒ नाप॑श्यन्।
ते केशा॒नग्रे॑\-ऽवपन्त।
अथ॒ श्मश्रू॑णि।
अथो॑प\-प॒क्षौ।
तत॒स्ते\-ऽवा᳚ञ्च आयन्।
परा॑ऽभवन्।
यस्यै॒वं वप॑न्ति।
अवा॑ङेति॥२५॥

%1.5.6.2
अथो॒ परै॒व भ॑वति।
अथ॑ दे॒वा ऊ॒र्ध्वं पृ॒ष्ठेभ्यो॑\-ऽपश्यन्।
त उ॑पप॒क्षावग्रे॑\-ऽवपन्त।
अथ॒ श्मश्रू॑णि।
अथ॒ केशान्॑।
तत॒स्ते॑\-ऽभवन्।
सु॒व॒र्गं लो॒कमा॑यन्।
यस्यै॒वं वप॑न्ति।
भव॑त्या॒त्मना᳚।
अथो॑ सुव॒र्गं लो॒कमे॑ति॥२६॥

%1.5.6.3
अथै॒तन्मनु॑र्व॒प्त्रे मि॑थु॒नम॑पश्यत्।
स श्मश्रू॒ण्यग्रे॑ऽवपत।
अथो॑पप॒क्षौ।
अथ॒ केशान्॑।
ततो॒ वै स प्राजा॑यत प्र॒जया॑ प॒शुभिः॑।
यस्यै॒वं वप॑न्ति।
प्र प्र॒जया॑ प॒शुभि॑र्मिथु॒नैर्जा॑यते।
दे॒वा॒सु॒राः संय॑त्ता आसन्।
ते सं॑वथ्स॒रे व्याय॑च्छन्त।
तान्दे॒वाश्चा॑तुर्मा॒स्यैरे॒वाभि प्रायु॑ञ्जत॥२७॥

%1.5.6.4
वै॒श्व॒दे॒वेन॑ च॒तुरो॑ मा॒सो॑\-ऽवृञ्ज॒तेन्द्र॑राजानः।
ताञ्छी॒र्॒षं नि चा\-व॑र्तयन्त॒ परि॑ च।
व॒रु॒ण॒प्र॒घा॒सैश्च॒तुरो॑ मा॒सो॑\-ऽवृञ्जत॒ वरु॑ण\-राजानः।
ताञ्छी॒र्॒षं नि चा\-व॑र्तयन्त॒ परि॑ च।
सा॒क॒मे॒धैश्च॒तुरो॑ मा॒सो॑\-ऽवृञ्जत॒ सोम॑\-राजानः।
ताञ्छी॒र्॒षं नि चा\-व॑र्तयन्त॒ परि॑ च।
या सं॑वथ्स॒र उ॑पजी॒वा\-ऽऽसी᳚त्।
तामे॑षामवृञ्जत।
ततो॑ दे॒वा अभ॑वन्।
पराऽसु॑राः॥२८॥

%1.5.6.5
य ए॒वं वि॒द्वाꣴश्चा॑तुर्मा॒स्यैर्यज॑ते।
भ्रातृ॑व्यस्यै॒व मा॒सो वृ॒क्त्वा।
शी॒र्॒षं नि च॑ व॒र्तय॑ते॒ परि॑ च।
यैषा सं॑वथ्स॒र उ॑पजी॒वा।
वृ॒ङ्क्ते तां भ्रातृ॑व्यस्य।
क्षु॒धा\-ऽस्य॒ भ्रातृ॑व्यः॒ परा॑ भवति।
लो॒हि॒ता॒य॒सेन॒ नि व॑र्तयते।
यद्वा इ॒माम॒ग्निर्\mbox{}ऋ॒तावाग॑ते निव॒र्तय॑ति।
ए॒तदे॒वैनाꣳ॑ रू॒पं कृ॒त्वा निव॑र्तयति।
सा ततः॒ श्वश्वो॒ भूय॑सी॒ भव॑न्त्येति॥२९॥

%1.5.6.6
प्र जा॑यते।
य ए॒वं वि॒द्वाँल्लो॑हिताय॒सेन॑ निव॒र्तय॑ते।
ए॒तदे॒व रू॒पं कृ॒त्वा नि व॑र्तयते।
स ततः॒ श्वश्वो॒ भूया॒न्भव॑न्नेति।
प्रैव जा॑यते।
त्रे॒ण्या श॑ल॒ल्या नि व॑र्तयेत।
त्रीणि॑ त्रीणि॒ वै दे॒वाना॑मृ॒द्धानि॑।
त्रीणि॒ छन्दाꣳ॑सि।
त्रीणि॒ सव॑नानि।
त्रय॑ इ॒मे लो॒काः॥३०॥

%1.5.6.7
ऋ॒ध्यामे॒व तद्वी॒र्य॑ ए॒षु लो॒केषु॒ प्रति॑ तिष्ठति।
यच्चा॑तुर्मास्य\-या॒ज्या᳚त्मनो॒ नाव॒द्येत्।
दे॒वेभ्य॒ आवृ॑श्च्येत।
च॒तृ॒षु च॑तृषु॒ मासे॑षु॒ नि व॑र्तयेत।
प॒रोक्ष॑मे॒व तद्दे॒वेभ्य॑ आ॒त्मनो\-ऽव॑द्य॒त्यना᳚\-व्रस्काय।
दे॒वानां॒ वा ए॒ष आनी॑तः।
यश्चा॑तुर्मास्यया॒जी।
य ए॒वं वि॒द्वान्नि च॑ व॒र्तय॑ते॒ परि॑ च।
दे॒वता॑ ए॒वाप्ये॑ति।
नास्य॑ रु॒द्रः प्र॒जां प॒शून॒भि म॑न्यते॥३१॥\anuvakamend[ए॒त्ये॒त्य॒यु॒ञ्ज॒तासु॑रा एति लो॒का म॑न्यते]

%1.5.7.1
आयु॑षः प्रा॒णꣳ सन्त॑नु।
प्रा॒णाद॑पा॒नꣳ सन्त॑नु।
अ॒पा॒नाद्व्या॒नꣳ सन्त॑नु।
व्या॒नाच्चक्षुः॒ सन्त॑नु।
चक्षु॑षः॒ श्रोत्र॒ꣳ॒ सन्त॑नु।
श्रोत्रा॒न्मनः॒ सन्त॑नु।
मन॑सो॒ वाच॒ꣳ॒ सन्त॑नु।
वा॒च आ॒त्मान॒ꣳ॒ सन्त॑नु।
आ॒त्मनः॑ पृथि॒वीꣳ सन्त॑नु।
पृ॒थि॒व्या अ॒न्तरि॑क्ष॒ꣳ॒ सन्त॑नु।
अ॒न्तरि॑क्षा॒द्दिव॒ꣳ॒ सन्त॑नु।
दिवः॒ सुवः॒ सन्त॑नु॥३२॥\anuvakamend[अ॒न्तरि॑क्ष॒ꣳ॒ सन्त॑नु॒ द्वे च॑]

%1.5.8.1
इन्द्रो॑ दधी॒चो अ॒स्थभिः॑।
वृ॒त्राण्यप्र॑तिष्कुतः।
ज॒घान॑ नव॒तीर्नव॑।
इ॒च्छन्नश्व॑स्य॒ यच्छिरः॑।
पर्व॑ते॒ष्वप॑श्रितम्।
तद्वि॑दच्छर्य॒णाव॑ति।
अत्राह॒ गोरम॑न्वत।
नाम॒ त्वष्टु॑रपी॒च्यम्᳚।
इ॒त्था च॒न्द्रम॑सो गृ॒हे।
इन्द्र॒मिद्गा॒थिनो॑ बृ॒हत्॥३३॥

%1.5.8.2
इन्द्र॑म॒र्केभि॑र॒र्किणः॑।
इन्द्रं॒ वाणी॑रनूषत।
इन्द्र॒ इद्धर्योः॒ सचा᳚।
सम्मि॑श्ल॒ आव॑चो॒ युजा᳚।
इन्द्रो॑ व॒ज्री हि॑र॒ण्ययः॑।
इन्द्रो॑ दी॒र्घाय॒ चक्ष॑से।
आ सूर्यꣳ॑ रोहयद्दि॒वि।
वि गोभि॒रद्रि॑मैरयत्।
इन्द्र॒ वाजे॑षु नो अव।
स॒हस्र॑प्रधनेषु च॥३४॥

%1.5.8.3
उ॒ग्र उ॒ग्राभि॑रू॒तिभिः॑।
तमिन्द्रं॑ वाजयामसि।
म॒हे वृ॒त्राय॒ हन्त॑वे।
स वृषा॑ वृष॒भो भु॑वत्।
इन्द्रः॒ स दाम॑ने कृ॒तः।
ओजि॑ष्ठः॒ स बले॑ हि॒तः।
द्यु॒म्नी श्लो॒की स सौ॒म्यः॑।
गि॒रा वज्रो॒ न सम्भृ॑तः।
सब॑लो॒ अन॑पच्युतः।
व॒व॒क्षुरु॒ग्रो अस्तृ॑तः॥३५॥\anuvakamend[बृ॒हच्चास्तृ॑तः]

%1.5.9.1
दे॒वा॒सु॒राः संय॑त्ता आसन्।
स प्र॒जा\-प॑ति॒रिन्द्रं॑ ज्ये॒ष्ठं पु॒त्रमप॒ न्य॑धत्त।
नेदे॑न॒मसु॑रा॒ बली॑याꣳसो\-ऽहन॒न्निति॑।
प्र॒ह्रादो॑ ह॒ वै का॑याध॒वः।
वि॒रोच॑न॒ꣴ॒ स्वं पु॒त्रमप॒ न्य॑धत्त।
नेदे॑नं दे॒वा अ॑हन॒न्निति॑।
ते दे॒वाः प्र॒जा\-प॑तिमुपस॒मेत्यो॑चुः।
नारा॒जक॑स्य यु॒द्धम॑स्ति।
इन्द्र॒मन्वि॑च्छा॒मेति॑।
तं य॑ज्ञक्र॒तुभि॒रन्वै᳚च्छन्॥३६॥

%1.5.9.2
तं य॑ज्ञक्र॒तुभि॒र्नान्व॑विन्दन्।
तमिष्टि॑भि॒रन्वै᳚च्छन्।
तमिष्टि॑भि॒रन्व॑\-विन्दन्।
तदिष्टी॑नामिष्टि॒\-त्वम्।
एष्ट॑यो ह॒ वै नाम॑।
ता इष्ट॑य॒ इत्याच॑क्षते प॒रोक्षे॑ण।
प॒रोक्ष॑प्रिया इव॒ हि दे॒वाः।
तस्मा॑ ए॒तमा᳚ग्नावैष्ण॒वमेका॑\-दश\-कपालं दीक्ष॒णीयं॒ निर॑वपन्।
तद॑प॒द्रुत्या॑तन्वत।
तान्प॑त्नीसंया॒जान्त॒ उपा॑नयन्॥३७॥

%1.5.9.3
ते तद॑न्तमे॒व कृ॒त्वोद॑द्रवन्।
ते प्रा॑य॒णीय॑म॒भि स॒मारो॑हन्।
तद॑प॒द्रुत्या॑\-तन्वत।
ताञ्छ॒य्य्वँ॑न्त॒ उपा॑नयन्।
ते तद॑न्तमे॒व कृ॒त्वोद॑द्रवन्।
त आ॑ति॒थ्यम॒भि स॒मारो॑हन्।
तद॑प॒द्रुत्या॑\-तन्वत।
तानिडा᳚न्त॒ उपा॑नयन्।
ते तद॑न्तमे॒व कृ॒त्वोद॑द्रवन्।
तस्मा॑दे॒ता ए॒तद॑न्ता॒ इष्ट॑यः॒ सन्ति॑ष्ठन्ते॥३८॥

%1.5.9.4
ए॒वꣳ हि दे॒वा अकु॑र्वत।
इति॑ दे॒वा अ॑कुर्वत।
इत्यु॒ वै म॑नु॒ष्याः᳚ कुर्वते।
ते दे॒वा ऊ॑चुः।
यद्वा इ॒दमु॒च्चैर्य॒ज्ञेन॒ चरा॑म।
तन्नो\-ऽसु॑राः पा॒प्मा\-ऽनु॑विन्दन्ति।
उ॒पा॒ꣳ॒शू॑प॒सदा॑ चराम।
तथा॒ नोऽसु॑राः पा॒प्मा नानु॑वेथ्स्य॒न्तीति॑।
त उ॑पा॒ꣳ॒शू॑प॒सद॑मतन्वत।
ति॒स्र ए॒व सा॑मिधे॒नीर॒नूच्य॑॥३९॥

%1.5.9.5
स्रु॒वेणा॑घा॒रमा॒घार्य॑।
ति॒स्रः परा॑ची॒राहु॑तीर्\mbox{}हु॒त्वा।
स्रु॒वेणो॑प॒सदं॑ जुह॒वां च॑क्रुः।
उ॒ग्रं वचो॒ अपा॑वधीन्त्वे॒षं वचो॒ अपा॑वधी॒ꣴ॒ स्वाहेति॑।
अ॒श॒न॒या॒पि॒पा॒से ह॒ वा उ॒ग्रं वचः॑।
एन॑श्च॒ वैर॑हत्यं च त्वे॒षं वचः॑।
ए॒तꣳ ह॒ वाव तच्च॑तुर्धाविहि॒तं पा॒प्मानं॑ दे॒वा अप॑जघ्निरे।
तथो॑ ए॒वैतदे॑वं॒विद्यज॑मानः।
ति॒स्र ए॒व सा॑मिधे॒नीर॒नूच्य॑।
स्रु॒वेणा॑घा॒रमा॒घार्य॑॥४०॥

%1.5.9.6
ति॒स्रः परा॑ची॒राहु॑तीर्\mbox{}हु॒त्वा।
स्रु॒वेणो॑प॒सदं॑ जुहोति।
उ॒ग्रं वचो॒ अपा॑वधीन्त्वे॒षं वचो॒ अपा॑वधी॒ꣴ॒ स्वाहेति॑।
अ॒श॒न॒या॒पि॒पा॒से ह॒ वा उ॒ग्रं वचः॑।
एन॑श्च॒ वैर॑हत्यं च त्वे॒षं वचः॑।
ए॒तमे॒व तच्च॑तुर्धाविहि॒तं पा॒प्मानं॒ यज॑मा॒नोऽप॑ हते।
ते॑ऽभि॒नीयै॒वाहः॑ प॒शुमा\-ऽल॑भन्त।
अह्न॑ ए॒व तद्दे॒वा अव॑र्तिं पा॒प्मानं॑ मृ॒त्युमप॑जघ्निरे।
तेना॑भि॒नीये॑व॒ रात्रेः॒ प्राच॑रन्।
रात्रि॑या ए॒व तद्दे॒वा अव॑र्तिं पा॒प्मानं॑ मृ॒त्युमप॑जघ्निरे॥४१॥

%1.5.9.7
तस्मा॑दभि॒नीयै॒वाहः॑ प॒शुमा ल॑भेत।
अह्न॑ ए॒व तद्यज॑मा॒नो\-ऽव॑र्तिं पा॒प्मानं॒ भ्रातृ॑व्या॒नप॑ नुदते।
तेना॑भि॒नीये॑व॒ रात्रेः॒ प्रच॑रेत्।
रात्रि॑या ए॒व तद्यज॑मा॒नो\-ऽव॑र्तिं पा॒प्मानं॒ भ्रातृ॑व्या॒नप॑ नुदते।
स ए॒ष उ॑पवस॒थीयेऽहं॑ द्विदेव॒त्यः॑ प॒शुरा ल॑भ्यते।
द्व॒यं वा अ॒स्मिँल्लो॒के यज॑मानः।
अस्थि॑ च मा॒ꣳ॒सं च॑।
अस्थि॑ चै॒व तेन॑ मा॒ꣳ॒सं च॒ यज॑मानः॒ सꣴस्कु॑रुते।
ता वा ए॒ताः पञ्च॑ दे॒वताः᳚।
अ॒ग्नीषोमा॑व॒ग्निर्मि॒त्रावरु॑णौ॥४२॥

%1.5.9.8
प॒ञ्च॒प॒ञ्ची वै यज॑मानः।
त्वङ्मा॒ꣳ॒सꣴ स्नावा\-ऽस्थि॑ म॒ज्जा।
ए॒तमे॒व तत्प॑ञ्चधाविहि॒तमा॒त्मानं॑ वरुणपा॒शान्मु॑ञ्चति।
भे॒ष॒जता॑यै निर्वरुण॒त्वाय॑।
तꣳ स॒प्तभि॒श्छन्दो॑भिः प्रा॒तर॑ह्वयन्।
तस्मा᳚थ्स॒प्त च॑तुरुत्त॒राणि॒ छन्दाꣳ॑सि प्रातरनु\-वा॒के\-ऽनू᳚च्यन्ते।
तमे॒तयो॑पस॒मेत्योपा॑सीदन्।
उपा᳚स्मै गायता नर॒ इति॑।
तस्मा॑दे॒तया॑ बहिष्पवमा॒न उ॑प॒सद्यः॑॥४३॥\anuvakamend[ऐ॒च्छ॒न्न॒न॒य॒ꣴ॒स्ति॒ष्ठ॒न्ते॒\-ऽनूच्या॒नूच्य॑ स्रु॒वेणा॑घा॒रमा॒घार्य॒ रात्रि॑या ए॒व तद्दे॒वा अव॑र्तिं पा॒प्मानं॑ मृ॒त्युमप॑जघ्निरे मि॒त्रावरु॑णौ॒ नव॑ च (दे॒वा यज॑मानो दे॒वा दे॒वा यज॑मानो॒ यज॑मानः॒ प्राच॑रं॒ प्रच॑रे॒दाल॑भ॒न्ताल॑भेत मृ॒त्युमप॑जघ्निरे॒ भ्रातृ॑व्यान्॥)]

%1.5.10.1
स स॑मु॒द्र उ॑त्तर॒तः प्राज्व॑लद्भूम्य॒न्तेन॑।
ए॒ष वाव स स॑मु॒द्रः।
यच्चात्वा॑लः।
ए॒ष उ॑वे॒व स भू᳚म्य॒न्तः।
यद्वे᳚द्य॒न्तः।
तदे॒तत्त्रि॑श॒लं त्रि॑पूरु॒षम्।
तस्मा॒त्तं त्रि॑वित॒स्तं ख॑नन्ति।
स सु॑वर्णरज॒ताभ्यां᳚ कु॒शीभ्यां॒ परि॑गृहीत आसीत्।
तं यद॒स्या अध्य॒जन॑यन्।
तस्मा॑दादि॒त्यः॥४४॥

%1.5.10.2
अथ॒ यथ्सु॑वर्णरज॒ताभ्यां᳚ कु॒शीभ्यां॒ परि॑गृहीत॒ आसी᳚त्।
साऽस्य॑ कौशि॒कता᳚।
तं त्रि॒वृता॒ऽभि प्रास्तु॑वत।
तं त्रि॒वृता\-ऽद॑दत।
तं त्रि॒वृता\-ऽह॑रन्।
याव॑ती त्रि॒वृतो॒ मात्रा᳚।
तं प॑ञ्चद॒शेना॒भि प्रास्तु॑वत।
तं प॑ञ्चद॒शेनाद॑दत।
तं प॑ञ्चद॒शेनाह॑रन्।
याव॑ती पञ्चद॒शस्य॒ मात्रा᳚॥४५॥

%1.5.10.3
तꣳ स॑प्तद॒शेना॒भि प्रास्तु॑वत।
तꣳ स॑प्तद॒शेनाद॑दत।
तꣳ स॑प्तद॒शेनाह॑रन्।
याव॑ती सप्तद॒शस्य॒ मात्रा᳚।
तस्य॑ सप्तद॒शेन॑ ह्रि॒यमा॑णस्य॒ तेजो॒ हरो॑\-ऽपतत्।
तमे॑कवि॒ꣳ॒शेना॒भि प्रास्तु॑वत।
तमे॑कवि॒ꣳ॒शेनाद॑दत।
तमे॑कवि॒ꣳ॒शेनाह॑रन्।
याव॑त्येकवि॒ꣳ॒शस्य॒ मात्रा᳚।
ते यत्त्रि॒वृता᳚ स्तु॒वते᳚॥४६॥

%1.5.10.4
त्रि॒वृतै॒व तद्यज॑मान॒माद॑दते।
तं त्रि॒वृतै॒व ह॑रन्ति।
याव॑ती त्रि॒वृतो॒ मात्रा᳚।
अ॒ग्निर्वै त्रि॒वृत्।
याव॒द्वा अ॒ग्नेर्दह॑तो धू॒म उ॒देत्यानु॒ व्येति॑।
ताव॑ती त्रि॒वृतो॒ मात्रा᳚।
अ॒ग्नेरे॒वैनं॒ तत्।
मात्रा॒ꣳ॒ सायु॑ज्यꣳ सलो॒कतां᳚ गमयन्ति।
अथ॒ यत्प॑ञ्चद॒शेन॑ स्तु॒वते᳚।
प॒ञ्च॒द॒शेनै॒व तद्यज॑मान॒माद॑दते॥४७॥

%1.5.10.5
तं प॑ञ्चद॒शेनै॒व ह॑रन्ति।
याव॑ती पञ्चद॒शस्य॒ मात्रा᳚।
च॒न्द्रमा॒ वै प॑ञ्चद॒शः।
ए॒ष हि प॑ञ्चद॒श्याम॑पक्षी॒यते᳚।
प॒ञ्च॒द॒श्यामा॑पू॒र्यते᳚।
च॒न्द्रम॑स ए॒वैनं॒ तत्।
मात्रा॒ꣳ॒ सायु॑ज्यꣳ सलो॒कतां᳚ गमयन्ति।
अथ॒ यथ्स॑प्तद॒शेन॑ स्तु॒वते᳚।
स॒प्त॒द॒शेनै॒व तद्यज॑मान॒माद॑दते।
तꣳ स॑प्तद॒शेनै॒व ह॑रन्ति॥४८॥

%1.5.10.6
याव॑ती सप्तद॒शस्य॒ मात्रा᳚।
प्र॒जा\-प॑ति॒र्वै स॑प्तद॒शः।
प्र॒जा\-प॑तेरे॒वैनं॒ तत्।
मात्रा॒ꣳ॒ सायु॑ज्यꣳ सलो॒कतां᳚ गमयन्ति।
अथ॒ यदे॑कवि॒ꣳ॒शेन॑ स्तु॒वते᳚।
ए॒क॒वि॒ꣳ॒शेनै॒व तद्यज॑मान॒माद॑दते।
तमे॑कवि॒ꣳ॒शेनै॒व ह॑रन्ति।
याव॑त्येक\-वि॒ꣳ॒शस्य॒ मात्रा᳚।
अ॒सौ वा आ॑दि॒त्य ए॑कवि॒ꣳ॒शः।
आ॒दि॒त्यस्यै॒वैनं॒ तत्॥४९॥

%1.5.10.7
मात्रा॒ꣳ॒ सायु॑ज्यꣳ सलो॒कतां᳚ गमयन्ति।
ते कु॒श्यौ᳚।
व्य॑घ्नन्।
ते अ॑होरा॒त्रे अ॑भवताम्।
अह॑रे॒व सु॒वर्णा॑\-ऽभवत्।
र॒ज॒ता रात्रिः॑।
स यदा॑दि॒त्य उ॒देति॑।
ए॒तामे॒व तथ्सु॒वर्णां᳚ कु॒शीमनु॒ समे॑ति।
अथ॒ यद॑स्त॒मेति॑।
ए॒तामे॒व तद्र॑ज॒तां कु॒शीमनु॒संवि॑शति।
प्र॒ह्रादो॑ ह॒ वै का॑याध॒वः।
वि॒रोच॑न॒ꣴ॒ स्वं पु॒त्रमुदा᳚स्यत्।
स प्र॑द॒रो॑\-ऽभवत्।
तस्मा᳚त्प्रद॒रादु॑द॒कं नाचा॑मेत्॥५०॥\anuvakamend[आ॒दि॒त्यः प॑ञ्चद॒शस्य॒ मात्रा᳚ स्तु॒वते॑ पञ्चद॒शेनै॒व तद्यज॑मान॒माद॑दते सप्तद॒शेनै॒व ह॑रन्त्यादि॒त्यस्यै॒वैनं॒ तद्वि॑शति च॒त्वारि॑ च]

%1.5.11.1
ये वै च॒त्वारः॒ स्तोमाः᳚।
कृ॒तं तत्।
अथ॒ ये पञ्च॑।
कलिः॒ सः।
तस्मा॒च्चतु॑ष्टोमः।
तच्चतु॑ष्टोमस्य चतुष्टोम॒त्वम्।
तदा॑हुः।
क॒त॒मानि॒ तानि॒ ज्योतीꣳ॑षि।
य ए॒तस्य॒ स्तोमा॒ इति॑।
त्रि॒वृत्प॑ञ्चद॒शः स॑प्तद॒श ए॑कवि॒ꣳ॒शः॥५१॥

%1.5.11.2
ए॒तानि॒ वाव तानि॒ ज्योतीꣳ॑षि।
य ए॒तस्य॒ स्तोमाः᳚।
सो᳚ऽब्रवीत्।
स॒प्त॒द॒शेन॑ ह्रि॒यमा॑णो॒ व्य॑लेशिषि।
भि॒षज्य॑त॒ मेति॑।
तम॒श्विनौ॑ धा॒नाभि॑रभिषज्यताम्।
पू॒षा क॑र॒म्भेण॑।
भार॑ती परिवा॒पेण॑।
मि॒त्रावरु॑णौ पय॒स्य॑या।
तदा॑हुः॥५२॥

%1.5.11.3
यद॒श्विभ्यां᳚ धा॒नाः।
पू॒ष्णः क॑र॒म्भः।
भार॑त्यै परिवा॒पः।
मि॒त्रावरु॑णयोः पय॒स्याऽथ॑।
कस्मा॑दे॒तेषाꣳ॑ ह॒विषा॒मिन्द्र॑मे॒व य॑ज॒न्तीति॑।
ए॒ता ह्ये॑नं दे॒वता॒ इति॑ ब्रूयात्।
ए॒तैर्\mbox{}ह॒विर्भि॒\-रभि॑षज्य॒ꣴ॒स्तस्मा॒दिति॑।
तं वस॑वो॒\-ऽष्टा\-क॑पालेन प्रातः सव॒ने॑\-ऽभिषज्यन्।
रु॒द्रा एका॑\-दश\-कपालेन॒ माध्यं॑ दिने॒ सव॑ने।
विश्वे॑ दे॒वा द्वाद॑शकपालेन तृतीयसव॒ने॥५३॥

%1.5.11.4
स यद॒ष्टा\-क॑पालान्प्रातः सव॒ने कु॒र्यात्।
एका॑दश\-कपाला॒न्माध्यं॑ दिने॒ सव॑ने।
द्वाद॑श\-कपालाꣴस्तृतीयसव॒ने।
विलो॑म॒ तद्य॒ज्ञस्य॑ क्रियेत।
एका॑दश\-कपालाने॒व प्रा॑तः सव॒ने कु॑र्यात्।
एका॑दश\-कपाला॒न्माध्यं॑ दिने॒ सव॑ने।
एका॑दश\-कपालाꣴ\-स्तृतीयसव॒ने।
य॒ज्ञस्य॑ सलोम॒त्वाय॑।
तदा॑हुः।
यद्वसू॑नां प्रातः सव॒नम्।
रु॒द्राणां॒ माध्यं॑ दिन॒ꣳ॒ सव॑नम्।
विश्वे॑षां दे॒वानां᳚ तृतीयसव॒नम्।
अथ॒ कस्मा॑दे॒तेषाꣳ॑ ह॒विषा॒मिन्द्र॑मे॒व य॑ज॒न्तीति॑।
ए॒ता ह्ये॑नं दे॒वता॒ इति॑ ब्रूयात्।
ए॒तैर्\-\mbox{}ह॒विर्भि॒रभि॑\-षज्य॒ꣴ॒स्तस्मा॒दिति॑॥५४॥\anuvakamend[ए॒क॒वि॒ꣳ॒श आ॑हुस्तृतीयसव॒ने प्रा॑तः सव॒नं पञ्च॑ च]

%1.5.12.1
तस्यावा॑चो\-ऽवपा॒दाद॑बिभयुः।
तमे॒तेषु॑ स॒प्तसु॒ छन्दः॑ स्वश्रयन्।
यदश्र॑यन्।
तच्छ्रा॑य॒न्तीय॑स्य श्रायन्तीय॒त्वम्।
यदवा॑रयन्।
तद्वा॑र\-व॒न्तीय॑स्य वारवन्तीय॒त्वम्।
तस्यावा॑च ए॒वाव॑पा॒दाद॑बिभयुः।
तस्मा॑ ए॒तानि॑ स॒प्त च॑तुरुत्त॒राणि॒ छन्दा॒ꣴ॒स्युपा॑दधुः।
तेषा॒मति॒ त्रीण्य॑रिच्यन्त।
न त्रीण्युद॑\-भवन्॥५५॥

%1.5.12.2
स बृ॑ह॒तीमे॒वास्पृ॑शत्।
द्वाभ्या॑म॒क्षरा᳚भ्याम्।
अ॒हो॒रा॒त्राभ्या॑मे॒व।
तदा॑हुः।
क॒त॒मा सा दे॒वाक्ष॑रा बृह॒ती।
यस्या॒न्तत्प्र॒त्यति॑ष्ठत्।
द्वाद॑श पौर्णमा॒स्यः॑।
द्वाद॒शाष्ट॑काः।
द्वाद॑शामावा॒स्याः᳚।
ए॒षा वाव सा दे॒वाक्ष॑रा बृह॒ती॥५६॥

%1.5.12.3
यस्यां॒ तत्प्र॒त्यति॑ष्ठ॒दिति॑।
यानि॑ च॒ छन्दाꣴ॑स्य॒त्यरि॑च्यन्त।
यानि॑ च॒ नोदभ॑वन्।
तानि॒ निर्वी᳚र्याणि ही॒नान्य॑मन्यन्त।
साऽब्र॑वीद्बृह॒ती।
मामे॒व भू॒त्वा।
मामुप॒ सꣴश्र॑य॒तेति॑।
च॒तुर्भि॑र॒क्षरै॑रनु॒\-ष्टुग्बृ॑ह॒तीं नोद॑भवत्।
च॒तुर्भि॑र॒क्षरैः᳚ प॒ङ्क्तिर्बृ॑ह॒ती\-मत्य॑रिच्यत।
तस्या॑मे॒तानि॑ च॒त्वार्य॒क्षरा᳚ण्यप॒च्छिद्या॑\-दधात्॥५७॥

%1.5.12.4
ते बृ॑ह॒ती ए॒व भू॒त्वा।
बृ॒ह॒तीमुप॒ सम॑श्रयताम्।
अ॒ष्टा॒भि\-र॒क्षरै॑रु॒ष्णिग्बृ॑ह॒तीं नोद॑भवत्।
अ॒ष्टा॒भि\-र॒क्षरै᳚स्त्रि॒ष्टुग्बृ॑ह॒ती\-मत्य॑\-रिच्यत।
तस्या॑मे॒तान्य॒ष्टाव॒क्षरा᳚ण्यप॒च्छिद्या॑\-दधात्।
ते बृ॑ह॒ती ए॒व भू॒त्वा।
बृ॒ह॒तीमुप॒ सम॑श्रयताम्।
द्वा॒द॒शभि॑र॒क्षरै᳚र्गाय॒त्री बृ॑ह॒तीं नोद॑भवत्।
द्वा॒द॒शभि॑र॒क्षरै॒र्जग॑ती बृह॒तीमत्य॑रिच्यत।
तस्या॑मे॒तानि॒ द्वाद॑शा॒क्षरा᳚ण्यप॒च्छिद्या॑\-दधात्॥५८॥

%1.5.12.5
ते बृ॑ह॒ती ए॒व भू॒त्वा।
बृ॒ह॒तीमुप॒ सम॑श्रयताम्।
सो᳚ऽब्रवीत्प्र॒जा\-प॑तिः।
छन्दाꣳ॑सि॒ रथो॑ मे भवत।
यु॒ष्माभि॑र॒हमे॒तमध्वा॑न॒मनु॒ सञ्च॑रा॒णीति॑।
तस्य॑ गाय॒त्री च॒ जग॑ती च प॒क्षाव॑भवताम्।
उ॒ष्णिक्च॑ त्रि॒ष्टुप्च॒ प्रष्ट्यौ᳚।
अ॒नु॒ष्टुप्च॑ प॒ङ्क्तिश्च॒ धुर्यौ᳚।
बृ॒ह॒त्ये॑वोद्धिर॑भवत्।
स ए॒तं छ॑न्दोर॒थमा॒स्थाय॑।
ए॒तमध्वा॑न॒मनु॒ सम॑चरत्।
ए॒तꣳ ह॒ वै छ॑न्दोर॒थमा॒स्थाय॑।
ए॒तमध्वा॑न॒मनु॒ सञ्च॑रति।
येनै॒ष ए॒तथ्स॒ञ्चर॑ति।
य ए॒वं वि॒द्वान्थ्सोमे॑न॒ यज॑ते।
य उ॑ चैनमे॒वं वेद॑॥५९॥\anuvakamend[अ॒भ॒व॒न्वाव सा दे॒वाक्ष॑रा बृह॒त्य॑दधा॒द्द्वाद॑शा॒क्षरा᳚ण्यप॒च्छिद्या॑दधादा॒स्थाय॒ षट्च॑]






\prashnaend{अ॒ग्नेः कृत्ति॑का॒ यत्पुण्यं॑ दे॒वस्य॑ सवि॒तुर्ब्र॑ह्मवा॒दिनः॒ कत्यृ॒तमे॒व दे॒वा वा आयु॑षः प्रा॒णमिन्द्रो॑ दधी॒चो दे॑वासु॒राः स प्र॒जा\-प॑तिः॒ स स॑मु॒द्रो ये वै च॒त्वार॒स्तस्यावा॑चो॒ द्वाद॑श॥१२॥}{अ॒ग्नेः कृत्ति॑का देवगृ॒हा ऋ॒तमे॒वर्ध्यामे॒व ति॒स्रः परा॑ची॒र्ये वै च॒त्वारो॒ नव॑पञ्चा॒शत्॥५९॥}{अ॒ग्नेः कृत्ति॑का॒ य उ॑ चैनमे॒वं वेद॑॥}{हरिः॑ ओम्॥}{इति श्रीकृष्णयजुर्वेदीयतैत्तिरीयब्राह्मणे प्रथमाष्टके पञ्चमः प्रपाठकः समाप्तः॥}
\clearpage
\sect{षष्ठमः प्रश्नः}
\setcounter{anuvakam}{0}
\dnsub{तैत्तिरीयब्राह्मणे प्रथमाष्टके षष्ठः प्रपाठकः}

%1.6.1.1
अनु॑मत्यै पुरो॒डाश॑म॒ष्टा\-क॑पालं॒ निर्व॑पति।
ये प्र॒त्यञ्चः॒ शम्या॑या अव॒शीय॑न्ते।
तन्नैर्॑ऋ॒तमेक॑कपालम्।
इ॒यं वा अनु॑मतिः।
इ॒यं निर्\mbox{}ऋ॑तिः।
नै॒र्॒ऋ॒तेन॒ पूर्वे॑ण॒ प्रच॑रति।
पा॒प्मान॑मे॒व निर्\mbox{}ऋ॑तिं॒ पूर्वां᳚ नि॒रव॑दयते।
एक॑कपालो भवति।
ए॒क॒धैव निर्\mbox{}ऋ॑तिं नि॒रव॑दयते।
यदहु॑त्वा॒ गार्\mbox{}ह॑पत्य ई॒युः॥१॥

%1.6.1.2
रु॒द्रो भू॒त्वा\-ऽग्निर॑नू॒त्थाय॑।
अ॒ध्व॒र्युं च॒ यज॑मानं च हन्यात्।
वीहि॒ स्वाहा\-ऽऽहु॑तिं जुषा॒ण इत्या॑ह।
आहु॑त्यै॒वैनꣳ॑ शमयति।
नार्ति॒मार्च्छ॑त्यध्व॒र्युर्न यज॑मानः।
ए॒को॒ल्मु॒केन॑ यन्ति।
तद्धि निर्\mbox{}ऋ॑त्यै भाग॒धेयम्᳚।
इ॒मान्दिशं॑ यन्ति।
ए॒षा वै निर्\mbox{}ऋ॑त्यै॒ दिक्।
स्वाया॑मे॒व दि॒शि निर्\mbox{}ऋ॑तिं नि॒रव॑दयते॥२॥

%1.6.1.3
स्वकृ॑त॒ इरि॑णे जुहोति प्रद॒रे वा᳚।
ए॒तद्वै निर्\mbox{}ऋ॑त्या आ॒यत॑नम्।
स्व ए॒वाऽऽयत॑ने॒ निर्\mbox{}ऋ॑तिं नि॒रव॑दयते।
ए॒ष ते॑ निर्‌\mbox{}ऋते भा॒ग इत्या॑ह।
निर्दि॑शत्ये॒वैना᳚म्।
भूते॑ ह॒विष्म॑त्य॒सीत्या॑ह।
भूति॑मे॒वोपाव॑र्तते।
मु॒ञ्चेममꣳह॑स॒ इत्या॑ह।
अꣳह॑स ए॒वैनं॑ मुञ्चति।
अ॒ङ्गु॒ष्ठाभ्यां᳚ जुहोति॥३॥

%1.6.1.4
अ॒न्त॒त ए॒व निर्\mbox{}ऋ॑तिं नि॒रव॑दयते।
कृ॒ष्णं वासः॑ कृ॒ष्णतू॑षं॒ दक्षि॑णा।
ए॒तद्वै निर्\mbox{}ऋ॑त्यै रू॒पम्।
रू॒पेणै॒व निर्\mbox{}ऋ॑तिं नि॒रव॑दयते।
अप्र॑तीक्ष॒माय॑न्ति।
निर्\mbox{}ऋ॑त्या अ॒न्तर्\mbox{}हि॑त्यै।
स्वाहा॒ नमो॒ य इ॒दं च॒कारेति॒ पुन॒रेत्य॒ गार्\mbox{}ह॑पत्ये जुहोति।
आहु॑त्यै॒व न॑म॒स्यन्तो॒ गार्\mbox{}ह॑पत्यमु॒पाव॑र्तन्ते।
आ॒नु॒म॒तेन॒ प्रच॑रति।
इ॒यं वा अनु॑मतिः॥४॥

%1.6.1.5
इ॒यमे॒वास्मै॑ रा॒ज्यमनु॑ मन्यते।
धे॒नुर्दक्षि॑णा।
इ॒मामे॒व धे॒नुं कु॑रुते।
आ॒दि॒त्यं च॒रुं निर्व॑पति।
उ॒भयी᳚ष्वे॒व प्र॒जास्व॒भिषि॑च्यते।
दैवी॑षु च॒ मानु॑षीषु च।
वरो॒ दक्षि॑णा।
वरो॒ हि रा॒ज्यꣳ समृ॑द्ध्यै।
आ॒ग्ना॒वै॒ष्ण॒वमेका॑\-दश\-कपालं॒ निर्व॑पति।
अ॒ग्निः सर्वा॑ दे॒वताः᳚॥५॥

%1.6.1.6
विष्णु॑र्य॒ज्ञः।
दे॒वता᳚श्चै॒व य॒ज्ञं चाव॑ रुन्धे।
वा॒म॒नो व॒ही दक्षि॑णा।
यद्व॒ही।
तेना᳚ऽऽग्ने॒यः।
यद्वा॑म॒नः।
तेन॑ वैष्ण॒वः समृ॑द्ध्यै।
अ॒ग्नी॒षो॒मीय॒मेका॑\-दश\-कपालं॒ निर्व॑पति।
अ॒ग्नीषोमा᳚भ्यां॒ वा इन्द्रो॑ वृ॒त्रम॑ह॒न्निति॑।
यद॑ग्नीषो॒मीय॒मेका॑\-दश\-कपालं नि॒र्वप॑ति॥६॥

%1.6.1.7
वार्त्र॑घ्नमे॒व विजि॑त्यै।
हिर॑ण्यं॒ दक्षि॑णा॒ समृ॑द्ध्यै।
इन्द्रो॑ वृ॒त्रꣳ ह॒त्वा।
दे॒वता॑भिश्चेन्द्रि॒येण॑ च॒ व्या᳚र्ध्यत।
स ए॒तमै᳚न्द्रा॒ग्नमेका॑\-दश\-कपालमपश्यत्।
तन्निर॑वपत्।
तेन॒ वै स दे॒वता᳚श्चेन्द्रि॒यं चावा॑रुन्ध।
यदै᳚न्द्रा॒ग्नमेका॑\-दश\-कपालं नि॒र्वप॑ति।
दे॒वता᳚श्चै॒व तेने᳚न्द्रि॒यं च॒ यज॑मा॒नो\-ऽव॑ रुन्धे।
ऋ॒ष॒भो व॒ही दक्षि॑णा॥७॥

%1.6.1.8
यद्व॒ही।
तेना᳚ऽऽग्ने॒यः।
यदृ॑ष॒भः।
तेनै॒न्द्रः समृ॑द्ध्यै।
आ॒ग्ने॒यम॒ष्टा\-क॑पालं॒ निर्व॑पति।
ऐ॒न्द्रं दधि॑।
यदा᳚ग्ने॒यो भव॑ति।
अ॒ग्निर्वै य॑ज्ञमु॒खम्।
य॒ज्ञ॒\-मु॒खमे॒वर्द्धिं॑ पु॒रस्ता᳚द्धत्ते।
यदै॒न्द्रं दधि॑॥८॥

%1.6.1.9
इ॒न्द्रि॒यमे॒वाव॑ रुन्धे।
ऋ॒ष॒भो व॒ही दक्षि॑णा।
यद्व॒ही।
तेना᳚ऽऽग्ने॒यः।
यदृ॑ष॒भः।
तेनै॒न्द्रः समृ॑द्ध्यै।
याव॑ती॒र्वै प्र॒जा ओष॑धीना॒महु॑ताना॒माश्ञन्॑।
ताः परा॑\-ऽभवन्।
आ॒ग्र॒य॒णं भ॑वति हु॒ताद्या॑य।
यज॑मान॒स्याप॑रा\-भावाय॥९॥

%1.6.1.10
दे॒वा वा ओष॑धीष्वा॒जिम॑युः।
ता इ॑न्द्रा॒ग्नी उद॑जयताम्।
तावे॒तमै᳚न्द्रा॒ग्नं द्वाद॑शकपालं॒ निर॑वृणाताम्।
यदै᳚न्द्रा॒ग्नो भव॒त्युज्जि॑त्यै।
द्वाद॑शकपालो भवति।
द्वाद॑श॒ मासाः᳚ संवथ्स॒रः।
सं॒व॒थ्स॒रेणै॒वास्मा॒ अन्न॒मव॑ रुन्धे।
वै॒श्व॒दे॒व\-श्च॒रुर्भ॑वति।
वै॒श्व॒दे॒वं वा अन्नम्᳚।
अन्न॑मे॒वास्मै᳚ स्वदयति॥१०॥

%1.6.1.11
प्र॒थ॒म॒जो व॒थ्सो दक्षि॑णा॒ समृ॑द्ध्यै।
सौ॒म्यꣴ श्या॑मा॒कं च॒रुं निर्व॑पति।
सोमो॒ वा अ॑कृष्टप॒च्यस्य॒ राजा᳚।
अ॒कृ॒ष्ट॒प॒च्यमे॒वास्मै᳚ स्वदयति।
वासो॒ दक्षि॑णा।
सौ॒म्यꣳ हि दे॒वत॑या॒ वासः॒ समृ॑द्ध्यै।
सर॑स्वत्यै च॒रुं निर्व॑पति।
सर॑स्वते च॒रुम्।
मि॒थु॒नमे॒वाव॑ रुन्धे।
मि॒थु॒नौ गावौ॒ दक्षि॑णा॒ समृ॑द्ध्यै।
एति॒ वा ए॒ष य॑ज्ञमु॒खादृध्याः᳚।
यो᳚ऽग्नेर्दे॒वता॑या॒ एति॑।
अ॒ष्टावे॒तानि॑ ह॒वीꣳषि॑ भवन्ति।
अ॒ष्टाक्ष॑रा गाय॒त्री।
गा॒य॒त्रो᳚\-ऽग्निः।
तेनै॒व य॑ज्ञमु॒खादृध्या॑ अ॒ग्नेर्दे॒वता॑यै॒ नैति॑॥११॥\anuvakamend[ई॒युर्नि॒रव॑दयते\-ऽङ्गु॒ष्ठाभ्यां᳚ जुहो॒त्यनु॑मतिर्दे॒वता॑ नि॒र्वप॑ति व॒ही दक्षि॑णा॒ यदै॒न्द्रं दध्यप॑राभावाय स्वदयति॒ गावौ॒ दक्षि॑णा॒ समृ॑द्ध्यै॒ षट्च॑]

%1.6.2.1
वै॒श्व॒दे॒वेन॒ वै प्र॒जा\-प॑तिः प्र॒जा अ॑\-सृजत।
ताः सृ॒ष्टा न प्राजा॑यन्त।
सो᳚ऽग्निर॑कामयत।
अ॒हमि॒माः प्रज॑नयेय॒मिति॑।
स प्र॒जा\-प॑तये॒ शुच॑मदधात्।
सो॑ऽशोचत्प्र॒जामि॒च्छमा॑नः।
तस्मा॒द्यं च॑ प्र॒जा भु॒नक्ति॒ यं च॒ न।
तावु॒भौ शो॑चतः प्र॒जामि॒च्छमा॑नौ।
तास्व॒ग्निमप्य॑\-सृजत्।
ता अ॒ग्निरध्यै᳚त्॥१२॥

%1.6.2.2
सोमो॒ रेतो॑\-ऽदधात्।
स॒वि॒ता प्राज॑नयत्।
सर॑स्वती॒ वाच॑मदधात्।
पू॒षा\-ऽपो॑षयत्।
ते वा ए॒ते त्रिः सं॑वथ्स॒रस्य॒ प्रयु॑ज्यन्ते।
ये दे॒वाः पुष्टि॑पतयः।
सं॒व॒थ्स॒रो वै प्र॒जा\-प॑तिः।
सं॒व॒थ्स॒रेणै॒वास्मै᳚ प्र॒जाः प्राज॑नयत्।
ताः प्र॒जा जा॒ता म॒रुतो᳚\-ऽघ्नन्।
अ॒स्मानपि॒ न प्रायु॑क्ष॒तेति॑॥१३॥

%1.6.2.3
स ए॒तं प्र॒जा\-प॑तिर्मारु॒तꣳ स॒प्तक॑पालमपश्यत्।
तन्निर॑वपत्।
ततो॒ वै प्र॒जाभ्यो॑\-ऽकल्पत।
यन्मा॑रु॒तो नि॑रु॒प्यते᳚।
य॒ज्ञस्य॒ कॢप्त्यै᳚।
प्र॒जाना॒मघा॑ताय।
स॒प्तक॑पालो भवति।
स॒प्तग॑णा॒ वै म॒रुतः॑।
ग॒ण॒श ए॒वास्मै॒ विशं॑ कल्पयति।
स प्र॒जा\-प॑तिरशोचत्॥१४॥

%1.6.2.4
याः पूर्वाः᳚ प्र॒जा असृ॑क्षि।
म॒रुत॒स्ता अ॑वधिषुः।
क॒थमप॑राः सृजे॒येति॑।
तस्य॒ शुष्म॑ आ॒ण्डं भू॒तं निर॑वर्तत।
तद्व्युद॑हरत्।
तद॑पोषयत्।
तत्प्राजा॑यत।
आ॒ण्डस्य॒ वा ए॒तद्रू॒पम्।
यदा॒मिक्षा᳚।
यद्व्यु॒द्धर॑ति॥१५॥

%1.6.2.5
प्र॒जा ए॒व तद्यज॑मानः पोषयति।
वै॒श्व॒दे॒व्या॑मिक्षा॑ भवति।
वै॒श्व॒दे॒व्यो॑ वै प्र॒जाः।
प्र॒जा ए॒वास्मै॒ प्रज॑नयति।
वाजि॑न॒मान॑यति।
प्र॒जास्वे॒व प्रजा॑तासु॒ रेतो॑ दधाति।
द्या॒वा॒पृ॒थि॒व्य॑ एक॑कपालो भवति।
प्र॒जा ए॒व प्रजा॑ता॒ द्यावा॑पृथि॒वीभ्या॑मुभ॒यतः॒ परि॑ गृह्णाति।
दे॒वा॒सु॒राः संय॑त्ता आसन्।
सो᳚ऽग्निर॑ब्रवीत्॥१६॥

%1.6.2.6
मामग्रे॑ यजत।
मया॒ मुखे॒नासु॑राञ्जेष्य॒थेति॑।
मां द्वि॒तीय॒मिति॒ सोमो᳚\-ऽब्रवीत्।
मया॒ राज्ञा॑ जेष्य॒थेति॑।
मां तृ॒तीय॒मिति॑ सवि॒ता।
मया॒ प्रसू॑ता जेष्य॒थेति॑।
मां च॑तु॒र्थीमिति॒ सर॑स्वती।
इ॒न्द्रि॒यं वो॒ऽहं धा᳚स्या॒मीति॑।
मां प॑ञ्च॒ममिति॑ पू॒षा।
मया᳚ प्रति॒ष्ठया॑ जेष्य॒थेति॑॥१७॥

%1.6.2.7
ते᳚ऽग्निना॒ मुखे॒नासु॑रानजयन्।
सोमे॑न॒ राज्ञा᳚।
स॒वि॒त्रा प्रसू॑ताः।
सर॑स्वतीन्द्रि॒यम॑दधात्।
पू॒षा प्र॑ति॒ष्ठा\-ऽऽसी᳚त्।
ततो॒ वै दे॒वा व्य॑जयन्त।
यदे॒तानि॑ ह॒वीꣳषि॑ निरु॒प्यन्ते॒ विजि॑त्यै।
नोत्त॑रवे॒दिमुप॑वपति।
प॒शवो॒ वा उ॑त्तरवे॒दिः।
अजा॑ता इव॒ ह्ये॑तर्\mbox{}हि॑ प॒शवः॑॥१८॥\anuvakamend[ऐ॒दित्य॑शोचद्व्यु॒द्धर॑त्यब्रवीत्प्रति॒ष्ठया॑ जेष्य॒थेत्ये॒तर्\mbox{}हि॑ प॒शवः॑]

%1.6.3.1
त्रि॒वृद्ब॒र्॒हिर्भ॑वति।
मा॒ता पि॒ता पु॒त्रः।
तदे॒व तन्मि॑थु॒नम्।
उल्बं॒ गर्भो॑ ज॒रायु॑।
तदे॒व तन्मि॑थु॒नम्।
त्रे॒धा ब॒र्॒हिः सन्न॑द्धं भवति।
त्रय॑ इ॒मे लो॒काः।
ए॒ष्वे॑व लो॒केषु॒ प्रति॑ तिष्ठति।
ए॒क॒धा पुनः॒ सन्न॑द्धं भवति।
एक॑ इव॒ ह्य॑यं लो॒कः॥१९॥

%1.6.3.2
अ॒स्मिन्ने॒व तेन॑ लो॒के प्रति॑ तिष्ठति।
प्र॒सुवो॑ भवन्ति।
प्र॒थ॒म॒जामे॒व पुष्टि॒मव॑ रुन्धे।
प्र॒थ॒म॒जो व॒थ्सो दक्षि॑णा॒ समृ॑द्ध्यै।
पृ॒ष॒दा॒ज्यं गृ॑ह्णाति।
प॒शवो॒ वै पृ॑षदा॒ज्यम्।
प॒शूने॒वाव॑ रुन्धे।
प॒ञ्च॒गृ॒ही॒तं भ॑वति।
पाङ्क्ता॒ हि प॒शवः॑।
ब॒हु॒रू॒पं भ॑वति॥२०॥

%1.6.3.3
ब॒हु॒रू॒पा हि प॒शवः॒ समृ॑द्ध्यै।
अ॒ग्निं म॑न्थन्ति।
अ॒ग्निमु॑खा॒ वै प्र॒जा\-प॑तिः प्र॒जा अ॑\-सृजत।
यद॒ग्निं मन्थ॑न्ति।
अ॒ग्निमु॑खा ए॒व तत्प्र॒जा यज॑मानः \-सृजते।
नव॑ प्रया॒जा इ॑ज्यन्ते।
नवा॑नूया॒जाः।
अ॒ष्टौ ह॒वीꣳषि॑।
द्वावा॑घा॒रौ।
द्वावाज्य॑\-भागौ॥२१॥

%1.6.3.4
त्रि॒ꣳ॒शथ्सम्प॑द्यन्ते।
त्रि॒ꣳ॒शद॑क्षरा वि॒राट्।
अन्नं॑ वि॒राट्।
वि॒राजै॒वान्नाद्य॒मव॑ रुन्धे।
यज॑मानो॒ वा एक॑कपालः।
तेज॒ आज्यम्᳚।
यदेक॑कपाल॒ आज्य॑मा॒नय॑ति।
यज॑मानमे॒व तेज॑सा॒ सम॑र्धयति।
यज॑मानो॒ वा एक॑कपालः।
प॒शव॒ आज्यम्᳚॥२२॥

%1.6.3.5
यदेक॑कपाल॒ आज्य॑मा॒नय॑ति।
यज॑मानमे॒व प॒शुभिः॒ सम॑र्धयति।
यदल्प॑मा॒नये᳚त्।
अल्पा॑ एनं प॒शवो॑ भु॒ञ्जन्त॒ उप॑तिष्ठेरन्।
यद्ब॒ह्वा॑नये᳚त्।
ब॒हव॑ एनं प॒शवो\-ऽभु॑ञ्जन्त॒ उप॑तिष्ठेरन्।
ब॒ह्वा॑नीया॒विः पृ॑ष्ठं कुर्यात्।
ब॒हव॑ ए॒वैनं॑ प॒शवो॑ भु॒ञ्जन्त॒ उप॑तिष्ठन्ते।
यज॑मानो॒ वा एक॑कपालः।
यदेक॑कपालस्याव॒द्येत्॥२३॥

%1.6.3.6
यज॑मान॒स्याव॑द्येत्।
उद्वा॒ माद्ये॒द्यज॑मानः।
प्र वा॑ मीयेत।
स॒कृदे॒व हो॑त॒व्यः॑।
स॒कृदि॑व॒ हि सु॑व॒र्गो लो॒कः।
हु॒त्वाऽभि जु॑होति।
यज॑मानमे॒व सु॑व॒र्गं लो॒कं ग॑मयि॒त्वा।
तेज॑सा॒ सम॑र्धयति।
यज॑मानो॒ वा एक॑कपालः।
सु॒व॒र्गो लो॒क आ॑हव॒नीयः॑॥२४॥

%1.6.3.7
यदेक॑कपालमाहव॒नीये॑ जु॒होति॑।
यज॑मानमे॒व सु॑व॒र्गं लो॒कं ग॑मयति।
यद्धस्ते॑न जुहु॒यात्।
सु॒व॒र्गाल्लो॒काद्यज॑मान॒मव॑\-विध्येत्।
स्रु॒चा जु॑होति।
सु॒व॒र्गस्य॑ लो॒कस्य॒ सम॑ष्ट्यै।
यत्प्राङ्पद्ये॑त।
दे॒व॒लो॒कम॒भिज॑येत्।
यद्द॑क्षि॒णा पि॑तृलो॒कम्।
यत्प्र॒त्यक्॥२५॥

%1.6.3.8
रक्षाꣳ॑सि य॒ज्ञꣳ ह॑न्युः।
यदुदङ्ङ्॑।
म॒नु॒ष्य॒लो॒कम॒भिज॑येत्।
प्रति॑\-ष्ठितो होत॒व्यः॑।
एक॑कपालं॒ वै प्र॑ति॒तिष्ठ॑न्तं॒ द्यावा॑पृथि॒वी अनु॒ प्रति॑ तिष्ठतः।
द्यावा॑पृथि॒वी ऋ॒तवः॑।
ऋ॒तून् य॒ज्ञः।
य॒ज्ञं यज॑मानः।
यज॑मानं प्र॒जाः।
तस्मा॒त्प्रति॑\-ष्ठितो होत॒व्यः॑॥२६॥

%1.6.3.9
वा॒जिनो॑ यजति।
अ॒ग्निर्वा॒युः सूर्यः॑।
ते वै वा॒जिनः॑।
ताने॒व तद्य॑जति।
अथो॒ खल्वा॑हुः।
छन्दाꣳ॑सि॒ वै वा॒जिन॒ इति॑।
तान्ये॒व तद्य॑जति।
ऋ॒ख्सा॒मे वा इन्द्र॑स्य॒ हरी॑ सोम॒पानौ᳚।
तयोः᳚ परि॒धय॑ आ॒धानम्᳚।
वाजि॑नं भाग॒धेयम्᳚॥२७॥

%1.6.3.10
यदप्र॑हृत्य परि॒धीं जु॑हु॒यात्।
अ॒न्तरा॑धानाभ्यां घा॒सं प्रय॑च्छेत्।
प्र॒हृत्य॑ परि॒धीं जु॑होति।
निरा॑धानाभ्यामे॒व घा॒सं प्रय॑च्छति।
ब॒र्॒हिषि॑ विषि॒ञ्चन्वाजि॑न॒मा न॑यति।
प्र॒जा वै ब॒र्॒हिः।
रेतो॒ वाजि॑नम्।
प्र॒जास्वे॒व रेतो॑ दधाति।
स॒मु॒प॒हूय॑ भक्षयन्ति।
ए॒तथ्सो॑मपीथा॒ ह्ये॑ते।
अथो॑ आ॒त्मन्ने॒व रेतो॑ दधते।
यज॑मान उत्त॒मो भ॑क्षयति।
प॒शवो॒ वै वाजि॑नम्।
यज॑मान ए॒व प॒शून्प्रति॑\-ष्ठापयन्ति॥२८॥\anuvakamend[लो॒को ब॑हुरू॒पं भ॑व॒त्याज्य॑\-भागौ प॒शव॒ आज्य॑मव॒द्येदा॑हव॒नीयः॑ प्र॒त्यक्तस्मा॒त्प्रति॑\-ष्ठितो होत॒व्यो॑ भाग॒धेय॑मे॒ते च॒त्वारि॑ च]

%1.6.4.1
प्र॒जा\-प॑तिः सवि॒ता भू॒त्वा प्र॒जा अ॑\-सृजत।
ता ए॑न॒मत्य॑मन्यन्त।
ता अ॑स्मा॒दपा᳚क्रामन्।
ता वरु॑णो भू॒त्वा प्र॒जा वरु॑णेनाग्राहयत्।
ताः प्र॒जा वरु॑णगृहीताः।
प्र॒जा\-प॑तिं॒ पुन॒रुपा॑धावन्ना॒थमि॒च्छमा॑नाः।
स ए॒तान्प्र॒जा\-प॑तिर्वरुण\-प्रघा॒सान॑पश्यत्।
तां निर॑वपत्।
तैर्वै स प्र॒जा व॑रुणपा॒शाद॑मुञ्चत्।
यद्व॑रुणप्रघा॒सा नि॑रु॒प्यन्ते᳚॥२९॥

%1.6.4.2
प्र॒जाना॒मव॑रुणग्राहाय।
तासां॒ दक्षि॑णो बा॒हुर्न्य॑क्न॒ आसी᳚त्।
स॒व्यः प्रसृ॑तः।
स ए॒तां द्वि॒तीयां᳚ दक्षिण॒तो वेदि॒मुद॑हन्।
ततो॒ वै स प्र॒जानां॒ दक्षि॑णं बा॒हुं प्रासा॑रयत्।
यद्द्वि॒तीयां᳚ दक्षिण॒तो वेदि॑मु॒द्धन्ति॑।
प्र॒जाना॑मे॒व तद्यज॑मानो॒ दक्षि॑णं बा॒हुं प्रसा॑रयति।
तस्मा᳚च्चातुर्मास्यया॒ज्य॑मुष्मिँ॑ल्लो॒क उ॑भ॒याबा॑हुः।
य॒ज्ञाभि॑जित॒ꣴ॒ ह्य॑स्य।
पृ॒थ॒मा॒त्राद्वेदी॒ अस॑म्भिन्ने भवतः॥३०॥

%1.6.4.3
तस्मा᳚त्पृथमा॒त्रं व्यꣳसौ᳚।
उत्त॑रस्यां॒ वेद्या॑मुत्तरवे॒दिमुप॑ वपति।
प॒शवो॒ वा उ॑त्तरवे॒दिः।
प॒शूने॒वाव॑ रुन्धे।
अथो॑ यज्ञप॒रुषो\-ऽन॑न्तरित्यै।
ए॒तद्ब्रा᳚ह्मणान्ये॒व पञ्च॑ ह॒वीꣳषि॑।
अथै॒ष ऐ᳚न्द्रा॒ग्नो भ॑वति।
प्रा॒णा॒पा॒नौ वा ए॒तौ दे॒वाना᳚म्।
यदि॑न्द्रा॒ग्नी।
यदै᳚न्द्रा॒ग्नो भव॑ति॥३१॥

%1.6.4.4
प्रा॒णा॒पा॒नावे॒वाव॑ रुन्धे।
ओजो॒ बलं॒ वा ए॒तौ दे॒वाना᳚म्।
यदि॑न्द्रा॒ग्नी।
यदै᳚न्द्रा॒ग्नो भव॑ति।
ओजो॒ बल॑मे॒वाव॑ रुन्धे।
मा॒रु॒त्या॑\-मिक्षा॑ भवति।
वा॒रु॒ण्या॑मिक्षा᳚।
मे॒षी च॑ मे॒षश्च॑ भवतः।
मि॒थु॒ना ए॒व प्र॒जा व॑रुणपा॒शान्मु॑ञ्चति।
लो॒म॒शौ भ॑वतो मेध्य॒त्वाय॑॥३२॥

%1.6.4.5
श॒मी॒प॒र्णान्युप॑ वपति।
घा॒समे॒वाभ्या॒मपि॑ यच्छति।
प्र॒जा\-प॑तिम॒न्नाद्यं॒ नोपा॑नमत्।
स ए॒तेन॑ श॒तेध्मे॑न ह॒विषा॒\-ऽन्नाद्य॒मवा॑रुन्ध।
यत्प॑रः श॒तानि॑ शमीप॒र्णानि॒ भव॑न्ति।
अ॒न्नाद्य॒स्या\-व॑\-रुद्ध्यै।
सौ॒म्यानि॒ वै क॒रीरा॑णि।
सौ॒म्या खलु॒ वा आहु॑तिर्दि॒वो वृष्टिं॑ च्यावयति।
यत्क॒रीरा॑णि॒ भव॑न्ति।
सौ॒म्ययै॒वाऽऽहु॑त्या दि॒वो वृष्टि॒मव॑ रुन्धे।
का॒य एक॑कपालो भवति।
प्र॒जानां᳚ क॒न्त्वाय॑।
प्र॒ति॒पू॒रु॒षं क॑रम्भपा॒त्राणि॑ भवन्ति।
जा॒ता ए॒व प्र॒जा व॑रुणपा॒शान्मु॑ञ्चति।
एक॒मति॑रिक्तम्।
ज॒नि॒ष्यमा॑णा ए॒व प्र॒जा व॑रुणपा॒शान्मु॑ञ्चति॥३३॥\anuvakamend[नि॒रु॒प्यन्ते॑ भवतो॒ भव॑ति मेध्य॒त्वाय॑ रुन्धे॒ षट्च॑]

%1.6.5.1
उत्त॑रस्यां॒ वेद्या॑म॒न्यानि॑ ह॒वीꣳषि॑ सादयति।
दक्षि॑णायां मारु॒तीम्।
अ॒प॒धु॒रमे॒वैना॑ युनक्ति।
अथो॒ ओज॑ ए॒वासा॒मव॑ हरति।
तस्मा॒द्ब्रह्म॑णश्च क्ष॒त्राच्च॒ विशो᳚\-ऽन्यतो\-ऽपक्र॒मिणीः᳚।
मा॒रु॒त्या पूर्व॑या॒ प्रच॑रति।
अनृ॑तमे॒वाव॑ यजते।
वा॒रु॒ण्योत्त॑रया।
अ॒न्त॒त ए॒व वरु॑ण॒मव॑ यजते।
यदे॒वाध्व॒र्युः क॒रोति॑॥३४॥

%1.6.5.2
तत्प्र॑तिप्रस्था॒ता क॑रोति।
तस्मा॒द्यच्छ्रेया᳚न्क॒रोति॑।
तत्पापी॑\-यान्करोति।
पत्नीं᳚ वाचयति।
मेध्या॑मे॒वैनां᳚ करोति।
अथो॒ तप॑ ए॒वैना॒मुप॑ नयति।
यज्जा॒रꣳ सन्त॒न्न प्र॑ब्रू॒यात्।
प्रि॒यं ज्ञा॒तिꣳ रु॑न्ध्यात्।
अ॒सौ मे॑ जा॒र इति॒ निर्दि॑शेत्।
नि॒र्दिश्यै॒वैनं॑ वरुणपा॒शेन॑ ग्राहयति॥३५॥

%1.6.5.3
प्र॒घा॒स्यान्॑ हवामह॒ इति॒ पत्नी॑मु॒दान॑यति।
अह्व॑तै॒वैना᳚म्।
यत्पत्नी॑ पुरोनुवा॒क्या॑मनुब्रू॒यात्।
निर्वी᳚र्यो॒ यज॑मानः स्यात्।
यज॑मा॒नो\-ऽन्वा॑ह।
आ॒त्मन्ने॒व वी॒र्यं॑ धत्ते।
उ॒भौ या॒ज्याꣳ॑ सवीर्य॒त्वाय॑।
यद्ग्रामे॒ यदर॑ण्य॒ इत्या॑ह।
य॒थो॒दि॒तमे॒व वरु॑ण॒मव॑ यजते।
य॒ज॒मा॒न॒दे॒व॒त्यो॑ वा आ॑हव॒नीयः॑॥३६॥

%1.6.5.4
भ्रा॒तृ॒व्य॒दे॒व॒त्यो॑ दक्षि॑णः।
यदा॑हव॒नीये॑ जुहु॒यात्।
यज॑मानं वरुणपा॒शेन॑ ग्राहयेत्।
दक्षि॑णे॒\-ऽग्नौ जु॑होति।
भ्रातृ॑व्यमे॒व व॑रुणपा॒शेन॑ ग्राहयति।
शूर्पे॑ण जुहोति।
अन्य॑मे॒व वरु॑ण॒मव॑ यजते।
शी॒र्॒षन्न॑धि नि॒धाय॑ जुहोति।
शी॒र्॒\mbox{}ष॒त ए॒व वरु॑ण॒मव॑ यजते।
प्र॒त्यङ्तिष्ठं॑ जुहोति॥३७॥

%1.6.5.5
प्र॒त्यङ्ङे॒व व॑रुणपा॒शान्निर्मु॑च्यते।
अक्र॒न्कर्म॑ कर्म॒कृत॒ इत्या॑ह।
दे॒वा\-ऽनृ॒णं नि॑रव॒दाय॑।
अ॒नृ॒णा गृ॒हानुप॒ प्रेतेति॒ वावैतदा॑ह।
वरु॑णगृहीतं॒ वा ए॒तद्य॒ज्ञस्य॑।
यद्यजु॑षा गृही॒तस्या॑ति॒\-रिच्य॑ते।
तुषा᳚श्च निष्का॒सश्च॑।
तुषै᳚श्च निष्का॒सेन॑ चावभृ॒थमवै॑ति।
वरु॑णगृहीतेनै॒व वरु॑ण॒मव॑यजते।
अ॒पो॑\-ऽवभृ॒थमवै॑ति॥३८॥

%1.6.5.6
अ॒फ्सु वै वरु॑णः।
सा॒क्षादे॒व वरु॑ण॒मव॑यजते।
प्रति॑\-युतो॒ वरु॑णस्य॒ पाश॒ इत्या॑ह।
व॒रु॒ण॒पा॒शादे॒व निर्मु॑च्यते।
अप्र॑तीक्ष॒मा य॑न्ति।
वरु॑णस्या॒न्तर्\mbox{}हि॑त्यै।
एधो᳚ऽस्येधिषी॒मही\-त्या॑ह।
स॒मिधै॒वाग्निं न॑म॒स्यन्त॑ उ॒पाय॑न्ति।
तेजो॑ऽसि॒ तेजो॒ मयि॑ धे॒हीत्या॑ह।
तेज॑ ए॒वाऽऽत्मन्ध॑त्ते॥३९॥\anuvakamend[क॒रोति॑ ग्राहयत्याहव॒नीय॒स्तिष्ठं॑ जुहोत्य॒पो॑\-ऽवभृ॒थमवै॑ति धत्ते]

%1.6.6.1
दे॒वा॒सु॒राः संय॑त्ता आसन्।
सो᳚ऽग्निर॑ब्रवीत्।
ममे॒यमनी॑क\-वती त॒नूः।
तां प्री॑णीत।
अथासु॑रान॒भि भ॑विष्य॒थेति॑।
ते दे॒वा अ॒ग्नये\-ऽनी॑कवते पुरो॒डाश॑म॒ष्टा\-क॑पालं॒ निर॑वपन्।
सो᳚ऽग्निरनी॑कवा॒न्थ्स्वेन॑ भाग॒धेये॑न प्री॒तः।
च॒तु॒र्धा\-ऽनी॑कान्य\-जनयत।
ततो॑ दे॒वा अभ॑वन्।
पराऽसु॑राः॥४०॥

%1.6.6.2
यद॒ग्नये\-ऽनी॑कवते पुरो॒डाश॑म॒ष्टा\-क॑पालं नि॒र्वप॑ति।
अ॒ग्निमे॒वा\-नी॑क\-वन्त॒ꣴ॒ स्वेन॑ भाग॒धेये॑न प्रीणाति।
सो᳚ऽग्निरनी॑कवा॒न्थ्स्वेन॑ भाग॒धेये॑न प्री॒तः।
च॒तु॒र्धा\-ऽनी॑कानि जनयते।
अ॒सौ वा आ॑दि॒त्यो᳚\-ऽग्निरनी॑कवान्।
तस्य॑ र॒श्मयो\-ऽनी॑कानि।
सा॒कꣳ सूर्ये॑णोद्य॒ता निर्व॑पति।
सा॒क्षादे॒वास्मा॒ अनी॑कानि जनयति।
तेऽसु॑राः॒ परा॑जिता॒ यन्तः॑।
द्यावा॑पृथि॒वी उपा᳚श्रयन्॥४१॥

%1.6.6.3
ते दे॒वा म॒रुद्भ्यः॑ सान्तप॒नेभ्य॑श्च॒रुं निर॑वपन्।
तान्द्यावा॑पृथि॒वी\-भ्या॑मे॒वोभ॒यतः॒ सम॑तपन्।
यन्म॒रुद्भ्यः॑ सान्तप॒नेभ्य॑श्च॒रुं नि॒र्वप॑ति।
द्यावा॑पृथि॒वीभ्या॑मे॒व तदु॑भ॒यतो॒ यज॑मानो॒ भ्रातृ॑व्या॒न्थ्सन्त॑पति।
म॒ध्यन्दि॑ने॒ निर्व॑पति।
तर्\mbox{}हि॒ हि तेक्ष्णि॑ष्ठं॒ तप॑ति।
च॒रुर्भ॑वति।
स॒र्वत॑ ए॒वैना॒न्थ्सन्त॑पति।
ते दे॒वाः श्वो॑विज॒यिनः॒ सन्तः॑।
सर्वा॑सां दु॒ग्धे गृ॑हमे॒धीयं॑ च॒रुं निर॑वपन्॥४२॥

%1.6.6.4
आशि॑ता ए॒वाद्योप॑वसाम।
कस्य॒ वाऽहे॒दम्।
कस्य॑ वा॒ श्वो भ॑वि॒तेति॑।
स शृ॒तो॑\-ऽभवत्।
तस्याहु॑तस्य॒ नाश्ञन्॑।
न हि दे॒वा अहु॑तस्या॒श्ञन्ति॑।
ते᳚ऽब्रुवन्।
कस्मा॑ इ॒मꣳ हो᳚ष्याम॒ इति॑।
म॒रुद्भ्यो॑ गृहमे॒धिभ्य॒ इत्य॑ब्रुवन्।
तं म॒रुद्भ्यो॑ गृहमे॒धिभ्यो॑\-ऽजुहवुः॥४३॥

%1.6.6.5
ततो॑ दे॒वा अभ॑वन्।
पराऽसु॑राः।
यस्यै॒वं वि॒दुषो॑ म॒रुद्भ्यो॑ गृहमे॒धिभ्यो॑ गृ॒हे जुह्व॑ति।
भव॑त्या॒त्मना᳚।
परा᳚ऽस्य॒ भ्रातृ॑व्यो भवति।
यद्वै य॒ज्ञस्य॑ पाक॒त्रा क्रि॒यते᳚।
प॒श॒व्यं॑ तत्।
पा॒क॒त्रा वा ए॒तत्क्रि॑यते।
यन्नेध्माब॒र्॒हिर्भव॑ति।
न सा॑मिधे॒नीर॒न्वाह॑॥४४॥

%1.6.6.6
न प्र॑या॒जा इ॒ज्यन्ते᳚।
नानू॑या॒जाः।
य ए॒वं वेद॑।
प॒शु॒मान्भ॑वति।
आज्य॑\-भागौ यजति।
य॒ज्ञस्यै॒व चक्षु॑षी॒ नान्तरे॑ति।
म॒रुतो॑ गृहमे॒धिनो॑ यजति।
भा॒ग॒धेये॑नै॒वैना॒न्थ्सम॑र्ध\-यति।
अ॒ग्निꣴस्वि॑ष्ट॒कृतं॑ यजति॒ प्रति॑\-ष्ठित्यै।
इडा᳚न्तो भवति।
प॒शवो॒ वा इडा᳚।
प॒शुष्वे॒वोपरि॑ष्टा॒त्प्रति॑ तिष्ठति॥४५॥\anuvakamend[असु॑रा अश्रयन्गृहमे॒धीयं॑ च॒रुं निर॑वपन्नजुहवुर॒न्वाहेडा᳚न्तो भवति॒ द्वे च॑]

%1.6.7.1
यत्पत्नी॑ गृहमे॒धीय॑स्याश्ञी॒यात्।
गृ॒ह॒मे॒ध्ये॑व स्या᳚त्।
वि त्व॑स्य य॒ज्ञ ऋ॑ध्येत।
यन्नाश्ञी॒यात्।
अगृ॑हमेधी स्यात्।
नास्य॑ य॒ज्ञो व्यृ॑द्ध्येत।
प्रति॑\-वेशं पचेयुः।
तस्या᳚श्ञीयात्।
गृ॒ह॒मे॒ध्ये॑व भ॑वति।
नास्य॑ य॒ज्ञो व्यृ॑द्ध्यते॥४६॥

%1.6.7.2
ते दे॒वा गृ॑हमे॒धीये॑ने॒ष्ट्वा।
आशि॑ता अभवन्।
आञ्ज॑ता॒भ्य॑ञ्जत।
अनु॑ व॒थ्सान॑वासयन्।
तेभ्यो\-ऽसु॑राः॒ क्षुधं॒ प्राहि॑ण्वन्।
सा दे॒वेषु॑ लो॒कमवि॑त्वा।
असु॑रा॒न्पुन॑रगच्छत्।
गृ॒ह॒मे॒धीये॑ने॒ष्ट्वा।
आशि॑ता भवन्ति।
आञ्ज॑ते॒\-ऽभ्य॑ञ्जते॥४७॥

%1.6.7.3
अनु॑ व॒थ्सान् वा॑सयन्ति।
भ्रातृ॑व्यायै॒व तद्यज॑मानः॒ क्षुधं॒ प्रहि॑णोति।
ते दे॒वा गृ॑हमे॒धीये॑ने॒ष्ट्वा।
इन्द्रा॑य निष्का॒सं न्य॑दधुः।
अ॒स्माने॒व श्व इन्द्रो॒ निहि॑तभाग उपावर्ति॒तेति॑।
तानिन्द्रो॒ निहि॑तभाग उ॒पाव॑र्तत।
गृ॒ह॒मे॒धीये॑ने॒ष्ट्वा।
इन्द्रा॑य निष्का॒सं निद॑ध्यात्।
इन्द्र॑ ए॒वैनं॒ निहि॑तभाग उ॒पाव॑र्तते।
गार्\mbox{}ह॑पत्ये जुहोति॥४८॥

%1.6.7.4
भा॒ग॒धेये॑नै॒वैन॒ꣳ॒ सम॑र्धयति।
ऋ॒ष॒भमाह्व॑यति।
व॒ष॒ट्का॒र ए॒वास्य॒ सः।
अथो॑ इन्द्रि॒यमे॒व तद्वी॒र्यं॑ यज॑मानो॒ भ्रातृव्य॑स्य वृङ्क्ते।
इन्द्रो॑ वृ॒त्रꣳ ह॒त्वा।
परां᳚ परा॒वत॑मगच्छत्।
अपा॑राध॒मिति॒ मन्य॑मानः।
सो᳚ऽब्रवीत्।
क इ॒दं वे॑दिष्य॒तीति॑।
ते᳚ऽब्रुवन्म॒रुतो॒ वरं॑ वृणामहै॥४९॥

%1.6.7.5
अथ॑ व॒यं वे॑दाम।
अ॒स्मभ्य॑मे॒व प्र॑थ॒मꣳ ह॒विर्निरु॑प्याता॒ इति॑।
त ए॑न॒मध्य॑क्रीडन्।
तत्क्री॒डिनां᳚ क्रीडि॒त्वम्।
यन्म॒रुद्भ्यः॑ क्री॒डिभ्यः॑ प्रथ॒मꣳ ह॒विर्नि॑रु॒प्यते॒ विजि॑त्यै।
सा॒कꣳ सूर्ये॑णोद्य॒ता निर्व॑पति।
ए॒तस्मि॒न्वै लो॒क इन्द्रो॑ वृ॒त्रम॑ह॒न्थ्समृ॑द्ध्यै।
ए॒तद्ब्रा᳚ह्मणान्ये॒व पञ्च॑ ह॒वीꣳषि॑।
ए॒तद्ब्रा᳚ह्मण ऐन्द्रा॒ग्नः।
अथै॒ष ऐ॒न्द्रश्च॒रुर्भ॑वति॥५०॥

%1.6.7.6
उ॒द्धा॒रं वा ए॒तमिन्द्र॒ उद॑हरत।
वृ॒त्रꣳ ह॒त्वा।
अ॒न्यासु॑ दे॒वता॒स्वधि॑।
यदे॒ष ऐ॒न्द्रश्च॒रुर्भव॑ति।
उ॒द्धा॒रमे॒व तं यज॑मान॒ उद्ध॑रते।
अ॒न्यासु॑ प्र॒जास्वधि॑।
वै॒श्व॒क॒र्म॒ण एक॑कपालो भवति।
विश्वा᳚न्ये॒व तेन॒ कर्मा॑णि॒ यज॑मा॒नो\-ऽव॑ रुन्धे॥५१॥\anuvakamend[ऋ॒द्ध्य॒ते॒\-ऽभ्य॑ञ्जते जुहोति वृणामहै भवत्य॒ष्टौ च॑]

%1.6.8.1
वै॒श्व॒दे॒वेन॒ वै प्र॒जा\-प॑तिः प्र॒जा अ॑\-सृजत।
ता व॑रुण\-प्रघा॒सैर्व॑रुण\-पा॒शाद॑मुञ्चत्।
सा॒क॒मे॒धैः प्रत्य॑स्थापयत्।
त्र्य॑म्बकै रु॒द्रं नि॒रवा॑दयत।
पि॒तृ॒य॒ज्ञेन॑ सुव॒र्गं लो॒कम॑गमयत्।
यद्वै᳚श्वदे॒वेन॒ यज॑ते।
प्र॒जा ए॒व तद्यज॑मानः \-सृजते।
ता व॑रुणप्रघा॒सैर्व॑रुणपा॒शान्मु॑ञ्चति।
सा॒क॒मे॒धैः प्रति॑\-ष्ठापयति।
त्र्य॑म्बकै रु॒द्रं नि॒रव॑दयते॥५२॥

%1.6.8.2
पि॒तृ॒य॒ज्ञेन॑ सुव॒र्गं लो॒कं ग॑मयति।
द॒क्षि॒ण॒तः प्रा॑चीनावी॒ती निर्व॑पति।
द॒क्षि॒णावृ॒द्धि पि॑तृ॒णाम्।
अना॑दृत्य॒ तत्।
उ॒त्त॒र॒त ए॒वोप॒वीय॒ निर्व॑पेत्।
उ॒भये॒ हि दे॒वाश्च॑ पि॒तर॑श्चे॒ज्यन्ते᳚।
अथो॒ यदे॒व द॑क्षिणा॒र्धे॑ऽधि॒ श्रय॑ति।
तेन॑ दक्षि॒णावृ॑त्।
सोमा॑य पितृ॒मते॑ पुरो॒डाश॒ꣳ॒ षट्क॑पालं॒ निर्व॑पति।
सं॒व॒थ्स॒रो वै सोमः॑ पितृ॒मान्॥५३॥

%1.6.8.3
सं॒व॒थ्स॒रमे॒व प्री॑णाति।
पि॒तृभ्यो॑ बर्\mbox{}हि॒षद्भ्यो॑ धा॒नाः।
मासा॒ वै पि॒तरो॑ बर्\mbox{}हि॒षदः॑।
मासा॑ने॒व प्री॑णाति।
यस्मि॒न्वा ऋ॒तौ पुरु॑षः प्र॒मीय॑ते।
सो᳚ऽस्या॒मुष्मिँ॑ल्लो॒के भ॑वति।
ब॒हु॒रू॒पा धा॒ना भ॑वन्ति।
अ॒हो॒रा॒त्राणा॑म॒भिजि॑त्यै।
पि॒तृभ्यो᳚\-ऽग्निष्वा॒त्तेभ्यो॑ म॒न्थम्।
अ॒र्ध॒मा॒सा वै पि॒तरो᳚\-ऽग्निष्वा॒त्ताः॥५४॥

%1.6.8.4
अ॒र्ध॒मा॒साने॒व प्री॑णाति।
अ॒भि॒वा॒न्या॑यै दु॒ग्धे भ॑वति।
सा हि पि॑तृदेव॒त्यं॑ दु॒हे।
यत्पू॒र्णम्।
तन्म॑नु॒ष्या॑णाम्।
उ॒प॒र्य॒र्धो दे॒वाना᳚म्।
अ॒र्धः पि॑तृ॒णाम्।
अ॒र्ध उप॑मन्थति।
अ॒र्धो हि पि॑तृ॒णाम्।
एक॒योप॑मन्थति॥५५॥

%1.6.8.5
एका॒ हि पि॑तृ॒णाम्।
द॒क्षि॒णोप॑मन्थति।
द॒क्षि॒णावृ॒द्धि पि॑तृ॒णाम्।
अना॑र॒भ्योप॑मन्थति।
तद्धि पि॒तॄन्गच्छ॑ति।
इ॒मान्दिशं॒ वेदि॒मुद्ध॑न्ति।
उ॒भये॒ हि दे॒वाश्च॑ पि॒तर॑श्चे॒ज्यन्ते᳚।
चतुः॑ स्रक्तिर्भवति।
सर्वा॒ ह्यनु॒ दिशः॑ पि॒तरः॑।
अखा॑ता भवति॥५६॥

%1.6.8.6
खा॒ता हि दे॒वाना᳚म्।
म॒ध्य॒तो᳚\-ऽग्निराधी॑यते।
अ॒न्त॒तो हि दे॒वाना॑माधी॒यते᳚।
वर्\mbox{}षी॑यानि॒ध्म इ॒ध्माद्भ॑वति॒ व्यावृ॑त्त्यै।
परि॑श्रयति।
अ॒न्तर्\mbox{}हि॑तो॒ हि पि॑तृलो॒को म॑नुष्यलो॒कात्।
यत्परु॑षि दि॒नम्।
तद्दे॒वाना᳚म्।
यद॑न्त॒रा।
तन्म॑नु॒ष्या॑णाम्॥५७॥

%1.6.8.7
यथ्समू॑लम्।
तत्पि॑तृ॒णाम्।
समू॑लं ब॒र्॒हिर्भ॑वति॒ व्यावृ॑त्त्यै।
द॒क्षि॒णा स्तृ॑णाति।
द॒क्षि॒णावृ॒द्धि पि॑तृ॒णाम्।
त्रिः पर्ये॑ति।
तृ॒तीये॒ वा इ॒तो लो॒के पि॒तरः॑।
ताने॒व प्री॑णाति।
त्रिः पुनः॒ पर्ये॑ति।
षट्थ्सम्प॑द्यन्ते॥५८॥

%1.6.8.8
षड्वा ऋ॒तवः॑।
ऋ॒तूने॒व प्री॑णाति।
यत्प्र॑स्त॒रं यजु॑षा गृह्णी॒यात्।
प्र॒मायु॑को॒ यज॑मानः स्यात्।
यन्न गृ॑ह्णी॒यात्।
अ॒ना॒य॒त॒नः स्या᳚त्।
तू॒ष्णीमे॒व न्य॑स्येत्।
न प्र॒मायु॑को॒ भव॑ति।
नाना॑यत॒नः।
यत्त्रीन्प॑रि॒धीन्प॑रिद॒ध्यात्॥५९॥

%1.6.8.9
मृ॒त्युना॒ यज॑मानं॒ परि॑गृह्णीयात्।
यन्न प॑रिद॒ध्यात्।
रक्षाꣳ॑सि य॒ज्ञꣳ ह॑न्युः।
द्वौ प॑रि॒धी परि॑दधाति।
रक्ष॑सा॒मप॑हत्यै।
अथो॑ मृ॒त्योरे॒व यज॑मान॒मुथ्सृ॑जति।
यत्त्रीणि॑ त्रीणि ह॒वीꣴष्यु॑दा॒हरे॑युः।
त्रय॑स्त्रय एषाꣳ सा॒कं प्रमी॑येरन्।
एकै॑कमनू॒चीना᳚न्यु॒दाह॑रन्ति।
एकै॑क ए॒वैषा॑म॒न्वञ्चः॒ प्रमी॑यते।
क॒शिपु॑ कशिप॒व्या॑य।
उ॒प॒बर्\mbox{}ह॑णमुपबर्\mbox{}ह॒ण्या॑य।
आञ्ज॑नमाञ्ज॒न्या॑य।
अ॒भ्यञ्ज॑नमभ्यञ्ज॒न्या॑य।
य॒था॒भा॒गमे॒\-वैना᳚न्प्रीणाति॥६०॥\anuvakamend[नि॒रव॑दयते पितृ॒मान॑ग्निष्वा॒त्ता एक॒योप॑ मन्थ॒त्यखा॑ता भवति मनु॒ष्या॑णां पद्यन्ते परिद॒ध्यान्मी॑यते॒ पञ्च॑ च]

%1.6.9.1
अ॒ग्नये॑ दे॒वेभ्यः॑ पि॒तृभ्यः॑ समि॒ध्यमा॑ना॒यानु॑ ब्रू॒हीत्या॑ह।
उ॒भये॒ हि दे॒वाश्च॑ पि॒तर॑श्चे॒ज्यन्ते᳚।
एका॒मन्वा॑ह।
एका॒ हि पि॑तृ॒णाम्।
त्रिरन्वा॑ह।
त्रिर्\mbox{}हि दे॒वाना᳚म्।
आ॒घा॒रावाघा॑रयति।
य॒ज्ञ॒प॒रुषो॒रन॑न्तरित्यै।
नार्\mbox{}षे॒यं वृ॑णीते।
न होता॑रम्॥६१॥

%1.6.9.2
यदा॑र्\mbox{}षे॒यं वृ॑णी॒त।
यद्धोता॑रम्।
प्र॒मायु॑को॒ यज॑मानः स्यात्।
प्र॒मायु॑को॒ होता᳚।
तस्मा॒न्न वृ॑णीते।
यज॑मानस्य॒ होतु॑र्गोपी॒थाय॑।
अप॑ बर्\mbox{}हिषः प्रया॒जान् य॑जति।
प्र॒जा वै ब॒र्॒हिः।
प्र॒जा ए॒व मृ॒त्योरुथ्सृ॑जति।
आज्य॑\-भागौ यजति॥६२॥

%1.6.9.3
य॒ज्ञस्यै॒व चक्षु॑षी॒ नान्तरे॑ति।
प्रा॒ची॒ना॒वी॒ती सोमं॑ यजति।
पि॒तृ॒दे॒व॒त्या॑ हि।
ए॒षाऽऽहु॑तिः।
पञ्च॒कृत्वोऽव॑ द्यति।
पञ्च॒ ह्ये॑ता दे॒वताः᳚।
द्वे पु॑रो\-ऽनुवा॒क्ये᳚।
या॒ज्या॑ दे॒वता॑ वषट्का॒रः।
ता ए॒व प्री॑णाति।
सन्त॑त॒मव॑ द्यति॥६३॥

%1.6.9.4
ऋ॒तू॒नाꣳ सन्त॑त्यै।
प्रैवैभ्यः॒ पूर्व॑या पुरो\-ऽनुवा॒क्य॑याऽऽह।
प्रण॑यति द्वि॒तीय॑या।
ग॒मय॑ति या॒ज्य॑या।
तृ॒तीये॒ वा इ॒तो लो॒के पि॒तरः॑।
अह्न॑ ए॒वैना॒न्पूर्व॑या पुरो\-ऽनुवा॒क्य॑या॒\-ऽत्यान॑यति।
रात्रि॑यै द्वि॒तीय॑या।
ऐवैनान्॑ या॒ज्य॑या गमयति।
द॒क्षि॒ण॒तो॑\-ऽव॒दाय॑।
उद॒ङ्ङति॑ क्रामति॒ व्यावृ॑त्त्यै॥६४॥

%1.6.9.5
आ स्व॒धेत्याश्रा॑वयति।
अस्तु॑ स्व॒धेति॑ प्र॒त्याश्रा॑वयति।
स्व॒धा नम॒ इति॒ वष॑ट्करोति।
स्व॒धा॒का॒रो हि पि॑तृ॒णाम्।
सोम॒मग्रे॑ यजति।
सोम॑प्रयाजा॒ हि पि॒तरः॑।
सोमं॑ पितृ॒मन्तं॑ यजति।
सं॒व॒थ्स॒रो वै सोमः॑ पितृ॒मान्।
सं॒व॒थ्स॒रमे॒व तद्य॑जति।
पि॒तॄन्ब॑र्हि॒षदो॑ यजति॥६५॥

%1.6.9.6
ये वै यज्वा॑नः।
ते पि॒तरो॑ बर्\mbox{}हि॒षदः॑।
ताने॒व तद्य॑जति।
पि॒तॄन॑ग्निष्वा॒त्तान् य॑जति।
ये वा अय॑ज्वानो गृहमे॒धिनः॑।
ते पि॒तरो᳚\-ऽग्निष्वा॒त्ताः।
ताने॒व तद्य॑जति।
अ॒ग्निं क॑व्य॒वाह॑नं यजति।
य ए॒व पि॑तृ॒णाम॒ग्निः।
तमे॒व तद्य॑जति॥६६॥

%1.6.9.7
अथो॒ यथा॒\-ऽग्निꣴ स्वि॑ष्ट॒कृतं॒ यज॑ति।
ता॒दृगे॒व तत्।
ए॒तत्ते॑ तत॒ ये च॒ त्वामन्विति॑ ति॒सृषु॑ स्र॒क्तीषु॒ निद॑धाति।
तस्मा॒दा तृ॒तीया॒त्पुरु॑षा॒न्नाम॒ न गृ॑ह्णन्ति।
ए॒ताव॑न्तो॒ हीज्यन्ते᳚।
अत्र॑ पितरो यथाभा॒गं म॑न्दध्व॒मित्या॑ह।
ह्लीका॒ हि पि॒तरः॑।
उद॑ञ्चो॒ निष्क्रा॑मन्ति।
ए॒षा वै म॑नु॒ष्या॑णां॒ दिक्।
स्वामे॒व तद्दिश॒मनु॒ निष्क्रा॑मन्ति॥६७॥

%1.6.9.8
आ॒ह॒व॒नीय॒मुप॑तिष्ठन्ते।
न्ये॑वास्मै॒ तद्ध्नु॑वते।
यथ्स॒त्या॑हव॒नीये᳚।
अथा॒न्यत्र॒ चर॑न्ति।
आतमि॑तो॒रुप॑तिष्ठन्ते।
अ॒ग्निमे॒वोप॑द्र॒ष्टारं॑ कृ॒त्वा।
पि॒तॄन्नि॒रव॑दयन्ते।
अन्तं॒ वा ए॒ते प्रा॒णानां᳚ गच्छन्ति।
य आतमि॑तोरुप॒ तिष्ठ॑न्ते।
सु॒स॒न्दृशं॑ त्वा व॒यमित्या॑ह॥६८॥

%1.6.9.9
प्रा॒णो वै सु॑स॒न्दृक्।
प्रा॒णमे॒वाऽऽत्मन्द॑धते।
योजा॒ न्वि॑न्द्र ते॒ हरी॒ इत्या॑ह।
प्रा॒णमे॒व पुन॑रयुक्त।
अक्ष॒न्नमी॑मदन्त॒ हीति॒ गार्\mbox{}ह॑पत्य॒मुप॑तिष्ठन्ते।
अक्ष॒न्नमी॑मद॒न्ताथ॒ त्वोप॑तिष्ठामह॒ इति॒ वावैतदा॑ह।
अमी॑मदन्त पि॒तरः॑ सो॒म्या इत्य॒भि प्रप॑द्यन्ते।
अमी॑मदन्त पि॒तरोऽथ॑ त्वा॒ऽभि प्रप॑द्यामह॒ इति॒ वावैतदा॑ह।
अ॒पः परि॑षिञ्चति।
मा॒र्जय॑त्ये॒वैनान्॑॥६९॥

%1.6.9.10
अथो॑ त॒र्पय॑त्ये॒व।
तृप्य॑ति प्र॒जया॑ प॒शुभिः॑।
य ए॒वं वेद॑।
अप॑ बर्\mbox{}हिषावनूया॒जौ य॑जति।
प्र॒जा वै ब॒र्॒हिः।
प्र॒जा ए॒व मृ॒त्योरुथ्सृ॑जति।
च॒तुरः॑ प्रया॒जान् य॑जति।
द्वाव॑नूया॒जौ।
षट्थ्सम्प॑द्यन्ते।
षड्वा ऋ॒तवः॑।
ऋ॒तूने॒व प्री॑णाति।
न पत्न्यन्वा᳚स्ते।
न संया॑जयन्ति।
यत्पत्न्य॒न्वासी॑त।
यथ्सं॑या॒जये॑युः।
प्र॒मायु॑का स्यात्।
तस्मा॒न्नान्वा᳚स्ते।
न संया॑जयन्ति।
पत्नि॑यै गोपी॒थाय॑॥७०॥\anuvakamend[होता॑र॒माज्य॑\-भागौ यजति॒ सन्त॑त॒मव॑द्यति॒ व्यावृ॑त्त्यै बर्\mbox{}हि॒षदो॑ यजति॒ तमे॒व तद्य॑ज॒त्यनु॒ निष्क्रा॑मन्त्याहैनानृ॒तवो॒ नव॑ च]

%1.6.10.1
प्र॒ति॒पू॒रु॒षमेक॑कपालां॒ निर्व॑पति।
जा॒ता ए॒व प्र॒जा रु॒द्रान्नि॒रव॑दयते।
एक॒मति॑रिक्तम्।
ज॒नि॒ष्यमा॑णा ए॒व प्र॒जा रु॒द्रान्नि॒रव॑दयते।
एक॑कपाला भवन्ति।
ए॒क॒धैव रु॒द्रं नि॒रव॑दयते।
नाभिघा॑रयति।
यद॑भिघा॒रये᳚त्।
अ॒न्त॒र॒व॒चा॒रिणꣳ॑ रु॒द्रं कु॑र्यात्।
ए॒को॒ल्मु॒केन॑ यन्ति॥७१॥

%1.6.10.2
तद्धि रु॒द्रस्य॑ भाग॒धेयम्᳚।
इ॒मान्दिशं॑ यन्ति।
ए॒षा वै रु॒द्रस्य॒ दिक्।
स्वाया॑मे॒व दि॒शि रु॒द्रं नि॒रव॑दयते।
रु॒द्रो वा अ॑प॒शुका॑या॒ आहु॑त्यै॒ नाति॑ष्ठत।
अ॒सौ ते॑ प॒शुरिति॒ निर्दि॑शे॒द्यं द्वि॒ष्यात्।
यमे॒व द्वेष्टि॑।
तम॑स्मै प॒शुं निर्दि॑शति।
यदि॒ न द्वि॒ष्यात्।
आ॒खुस्ते॑ प॒शुरिति॑ ब्रूयात्॥७२॥

%1.6.10.3
न ग्रा॒म्यान्प॒शून् हि॒नस्ति॑।
नार॒ण्यान्।
च॒तु॒ष्प॒थे जु॑होति।
ए॒ष वा अ॑ग्नी॒नां पड्बी॑शो॒ नाम॑।
अ॒ग्नि॒वत्ये॒व जु॑होति।
म॒ध्य॒मेन॑ प॒र्णेन॑ जुहोति।
स्रुग्घ्ये॑षा।
अथो॒ खलु॑।
अ॒न्त॒मेनै॒व हो॑त॒व्यम्᳚।
अ॒न्त॒त ए॒व रु॒द्रं नि॒रव॑दयते॥७३॥

%1.6.10.4
ए॒ष ते॑ रुद्र भा॒गः स॒ह स्वस्रा\-ऽम्बि॑क॒येत्या॑ह।
श॒रद्वा अ॒स्याम्बि॑का॒ स्वसा᳚।
तया॒ वा ए॒ष हि॑नस्ति।
यꣳ हि॒नस्ति॑।
तयै॒वैनꣳ॑ स॒ह श॑मयति।
भे॒ष॒जङ्गव॒ इत्या॑ह।
याव॑न्त ए॒व ग्रा॒म्याः प॒शवः॑।
तेभ्यो॑ भेष॒जं क॑रोति।
अवा᳚म्ब रु॒द्रम॑दिम॒हीत्या॑ह।
आ॒\-मे॒वैतामा शा᳚स्ते॥७४॥

%1.6.10.5
त्र्य॑म्बकं यजामह॒ इत्या॑ह।
मृ॒त्योर्मु॑क्षीय॒ माऽमृता॒दिति॒ वावैतदा॑ह।
उत्कि॑रन्ति।
भग॑स्य लीफ्सन्ते।
मूते॑कृ॒त्वा\-ऽऽस॑जन्ति।
यथा॒ जनं॑ य॒ते॑\-ऽव॒सं क॒रोति॑।
ता॒दृगे॒व तत्।
ए॒ष ते॑ रुद्र भा॒ग इत्या॑ह नि॒रव॑त्त्यै।
अप्र॑तीक्ष॒मा य॑न्ति।
अ॒पः परि॑षिञ्चति।
रु॒द्रस्या॒न्तर्\mbox{}हि॑त्यै।
प्र वा ए॒ते᳚\-ऽस्माल्लो॒काच्च्य॑वन्ते।
ये त्र्य॑म्बकै॒श्चर॑न्ति।
आ॒दि॒त्यं च॒रुं पुन॒रेत्य॒ निर्व॑पति।
इ॒यं वा अदि॑तिः।
अ॒स्यामे॒व प्रति॑ तिष्ठन्ति॥७५॥\anuvakamend[य॒न्ति॒ ब्रू॒या॒न्नि॒रव॑दयते शास्ते सिञ्चति॒ षट्च॑]




\prashnaend{अनु॑मत्यै वैश्वदे॒वेन॒ ताः सृ॒ष्टास्त्रि॒वृत्प्र॒जा\-प॑तिः सवि॒तोत्त॑रस्यान्देवासु॒राः सो᳚\-ऽग्निर्यत्पत्नी॑ वैश्वदे॒वेन॒ ता व॑रुणप्रघा॒सैर॒ग्नये॑ दे॒वेभ्यः॑ प्रतिपूरु॒षं दश॑॥१०॥}{अनु॑मत्यै प्रथम॒जो व॒थ्सो ब॑हुरू॒पा हि प॒शव॒स्तस्मा᳚त्पृथमा॒त्रं यद॒ग्नये\-ऽनी॑कवत उद्धा॒रं वा अ॒ग्नये॑ दे॒वेभ्यः॑ प्रतिपूरु॒षं पञ्च॑सप्ततिः॥७५॥}{अनु॑मत्यै॒ प्रति॑ तिष्ठन्ति॥}{हरिः॑ ओम्॥}{इति श्रीकृष्णयजुर्वेदीयतैत्तिरीयब्राह्मणे प्रथमाष्टके षष्ठः प्रपाठकः समाप्तः॥}
\clearpage
\sect{सप्तमः प्रश्नः}
\setcounter{anuvakam}{0}
\dnsub{तैत्तिरीयब्राह्मणे प्रथमाष्टके सप्तमः प्रपाठकः}

%सं॒व॒थ्स॒        सं॒व॒थ्स॒
%1.7.1.1
ए॒तद्ब्रा᳚ह्मणान्ये॒व पञ्च॑ ह॒वीꣳषि॑।
अथेन्द्रा॑य॒ शुना॒सीरा॑य पुरो॒डाशं॒ द्वाद॑शकपालं॒ निर्व॑पति।
सं॒व॒थ्स॒रो वा इन्द्रा॒शुना॒सीरः॑।
सं॒व॒थ्स॒रेणै॒वास्मा॒ अन्न॒मव॑ रुन्धे।
वा॒य॒व्यं॑ पयो॑ भवति।
वा॒युर्वै वृष्ट्यै᳚ प्रदापयि॒ता।
स ए॒वास्मै॒ वृष्टिं॒ प्रदा॑पयति।
सौ॒र्य॑ एक॑कपालो भवति।
सूर्ये॑ण॒ वा अ॒मुष्मिँ॑ल्लो॒के वृष्टि॑र्धृ॒ता।
स ए॒वास्मै॒ वृष्टिं॒ निय॑च्छति॥१॥

%1.7.1.2
द्वा॒द॒श॒ग॒वꣳ सीरं॒ दक्षि॑णा॒ समृ॑द्ध्यै।
दे॒वा॒सु॒राः संय॑त्ता आसन्।
ते दे॒वा अ॒ग्निम॑ब्रुवन्।
त्वया॑ वी॒रेणासु॑रान॒भिभ॑वा॒मेति॑।
सो᳚ऽब्रवीत्।
त्रे॒धा\-ऽहमा॒त्मानं॒ विक॑रिष्य॒ इति॑।
स त्रे॒धा\-ऽऽत्मानं॒ व्य॑कुरुत।
अ॒ग्निं तृती॑यम्।
रु॒द्रं तृती॑यम्।
वरु॑णं॒ तृती॑यम्॥२॥

%1.7.1.3
सो᳚ऽब्रवीत्।
क इ॒दं तु॒रीय॒मिति॑।
अ॒हमितीन्द्रो᳚\-ऽब्रवीत्।
सन्तु सृ॑जावहा॒ इति॑।
तौ सम॑सृजेताम्।
स इन्द्र॑स्तु॒रीय॑मभवत्।
यदिन्द्र॑स्तु॒रीय॒मभ॑वत्।
तदि॑न्द्र\-तुरी॒यस्ये᳚न्द्र\-तुरीय॒त्वम्।
ततो॒ वै दे॒वा व्य॑जयन्त।
यदि॑न्द्रतुरी॒यं नि॑रु॒प्यते॒ विजि॑त्यै॥३॥

%1.7.1.4
व॒हिनी॑ धे॒नुर्दक्षि॑णा।
यद्व॒हिनी᳚।
तेना᳚ऽऽग्ने॒यी।
यद्गौः।
तेन॑ रौ॒द्री।
यद्धे॒नुः।
तेनै॒न्द्री।
यथ्स्त्री स॒ती दा॒न्ता।
तेन॑ वारु॒णी समृ॑द्ध्यै।
प्र॒जा\-प॑तिर्य॒ज्ञम॑\-सृजत॥४॥

%1.7.1.5
तꣳ सृ॒ष्टꣳ रक्षाꣴ॑स्यजिघाꣳसन्।
स ए॒ताः प्र॒जा\-प॑तिरा॒त्मनो॑ दे॒वता॒ निर॑मिमीत।
ताभि॒र्वै स दि॒ग्भ्यो रक्षाꣳ॑सि॒ प्राणु॑दत।
यत्प॑ञ्चाव॒त्तीयं॑ जु॒होति॑।
दि॒ग्भ्य ए॒व तद्यज॑मानो॒ रक्षाꣳ॑सि॒ प्रणु॑दते।
समू॑ढ॒ꣳ॒ रक्षः॒ सन्द॑ग्ध॒ꣳ॒ रक्ष॒ इत्या॑ह।
रक्षाꣴ॑स्ये॒व सन्द॑हति।
अ॒ग्नये॑ रक्षो॒घ्ने स्वाहेत्या॑ह।
दे॒वता᳚भ्य ए॒व वि॑जिग्या॒नाभ्यो॑ भाग॒धेयं॑ करोति।
प्र॒ष्टि॒वा॒ही रथो॒ दक्षि॑णा॒ समृ॑द्ध्यै॥५॥

%1.7.1.6
इन्द्रो॑ वृ॒त्रꣳ ह॒त्वा।
असु॑रान्परा॒भाव्य॑।
नमु॑चिमासु॒रं नाल॑भत।
तꣳ श॒च्या॑\-ऽगृह्णात्।
तौ सम॑लभेताम्।
सो᳚ऽस्माद॒भिशु॑नतरो\-ऽभवत्।
सो᳚ऽब्रवीत्।
स॒न्धाꣳ सन्द॑धावहै।
अथ॒ त्वाऽव॑ स्रक्ष्यामि।
न मा॒ शुष्के॑ण॒ नाऽऽर्द्रेण॑ हनः॥६॥

%1.7.1.7
न दिवा॒ न नक्त॒मिति॑।
स ए॒तम॒पां फेन॑मसिञ्चत्।
न वा ए॒ष शुष्को॒ नाऽऽर्द्रो व्यु॑ष्टा\-ऽऽसीत्।
अनु॑दितः॒ सूर्यः॑।
न वा ए॒तद्दिवा॒ न नक्तम्᳚।
तस्यै॒तस्मिँ॑ल्लो॒के।
अ॒पां फेने॑न॒ शिर॒ उद॑वर्तयत्।
तदे॑न॒मन्व॑वर्तत।
मित्र॑द्रु॒गिति॑॥७॥

%1.7.1.8
स ए॒तान॑पामा॒र्गान॑जनयत्।
तान॑जुहोत्।
तैर्वै स रक्षा॒ꣴ॒स्यपा॑हत।
यद॑पामार्गहो॒मो भव॑ति।
रक्ष॑सा॒मप॑हत्यै।
ए॒को॒ल्मु॒केन॑ यन्ति।
तद्धि रक्ष॑सां भाग॒धेयम्᳚।
इ॒मान्दिशं॑ यन्ति।
ए॒षा वै रक्ष॑सां॒ दिक्।
स्वाया॑मे॒व दि॒शि रक्षाꣳ॑सि हन्ति॥८॥

%1.7.1.9
स्वकृ॑त॒ इरि॑णे जुहोति प्रद॒रे वा᳚।
ए॒तद्वै रक्ष॑सामा॒यतन॑म्।
स्व ए॒वाऽऽयत॑ने॒ रक्षाꣳ॑सि हन्ति।
प॒र्ण॒मये॑न स्रु॒वेण॑ जुहोति।
ब्रह्म॒ वै प॒र्णः।
ब्रह्म॑णै॒व रक्षाꣳ॑सि हन्ति।
दे॒वस्य॑ त्वा सवि॒तुः प्र॑स॒व इत्या॑ह।
स॒वि॒तृप्र॑सूत ए॒व रक्षाꣳ॑सि हन्ति।
ह॒तꣳ रक्षो\-ऽव॑धिष्म॒ रक्ष॒ इत्या॑ह।
रक्ष॑सा॒ꣴ॒ स्तृत्यै᳚।
यद्वस्ते॒ तद्दक्षि॑णा नि॒रव॑त्यै।
अप्र॑तीक्ष॒माय॑न्ति।
रक्ष॑साम॒न्तर्‌\mbox{}हि॑त्यै॥९॥\anuvakamend[य॒च्छ॒ति॒ वरु॑णं॒ तृती॑यं॒ विजि॑त्या अ\-सृजत॒ समृ॑द्ध्यै हनो॒ मित्र॑द्रु॒गिति॑ हन्ति॒ स्तृत्यै॒ त्रीणि॑ च]

%1.7.2.1
धा॒त्रे पु॑रो॒डाशं॒ द्वाद॑शकपालं॒ निर्व॑पति।
सं॒व॒थ्स॒रो वै धा॒ता।
सं॒व॒थ्स॒रेणै॒वास्मै᳚ प्र॒जाः प्रज॑नयति।
अन्वे॒वास्मा॒ अनु॑मतिर्मन्यते।
रा॒ते रा॒का।
प्र सि॑नीवा॒ली ज॑नयति।
प्र॒जास्वे॒व प्रजा॑तासु कु॒ह्वा॑ वाचं॑ दधाति।
मि॒थु॒नौ गावौ॒ दक्षि॑णा॒ समृ॑द्ध्यै।
आ॒ग्ना॒वै॒ष्ण॒वमेका॑\-दश\-कपालं॒ निर्व॑पति।
ऐ॒न्द्रा॒वै॒ष्ण॒वमेका॑\-दश\-कपालम्॥१०॥

%1.7.2.2
वै॒ष्ण॒वं त्रि॑कपा॒लम्।
वी॒र्यं॑ वा अ॒ग्निः।
वी॒र्य॑मिन्द्रः॑।
वी॒र्यं॑ विष्णुः॑।
प्र॒जा ए॒व प्रजा॑ता वी॒र्ये᳚ प्रति॑\-ष्ठापयति।
तस्मा᳚त्प्र॒जा वी॒र्या॑वतीः।
वा॒म॒न ऋ॑ष॒भो व॒ही दक्षि॑णा।
यद्व॒ही।
तेना᳚ऽऽग्ने॒यः।
यदृ॑ष॒भः॥११॥

%1.7.2.3
तेनै॒न्द्रः।
यद्वा॑म॒नः।
तेन॑ वैष्ण॒वः समृ॑द्ध्यै।
अ॒ग्नी॒षो॒मीय॒मेका॑\-दश\-कपालं॒ निर्व॑पति।
इ॒न्द्रा॒सो॒मीय॒मेका॑\-दश\-कपालम्।
सौ॒म्यं च॒रुम्।
सोमो॒ वै रे॑तो॒धाः।
अ॒ग्निः प्र॒जानां᳚ प्रजनयि॒ता।
वृ॒द्धाना॒मिन्द्रः॑ प्रदापयि॒ता।
सोम॑ ए॒वास्मै॒ रेतो॒ दधा॑ति॥१२॥

%1.7.2.4
अ॒ग्निः प्र॒जां प्रज॑नयति।
वृ॒द्धामिन्द्रः॒ प्रय॑च्छति।
ब॒भ्रुर्दक्षि॑णा॒ समृ॑द्ध्यै।
सो॒मा॒पौ॒ष्णं च॒रुं निर्व॑पति।
ऐ॒न्द्रा॒पौ॒ष्णं च॒रुम्।
सोमो॒ वै रे॑तो॒धाः।
पू॒षा प॑शू॒नां प्र॑जनयि॒ता।
वृ॒द्धाना॒मिन्द्रः॑ प्रदापयि॒ता।
सोम॑ ए॒वास्मै॒ रेतो॒ दधा॑ति।
पू॒षा प॒शून्प्रज॑नयति॥१३॥

%1.7.2.5
वृ॒द्धानिन्द्रः॒ प्रय॑च्छति।
पौ॒ष्णश्च॒रुर्भ॑वति।
इ॒यं वै पू॒षा।
अ॒स्यामे॒व प्रति॑ तिष्ठति।
श्या॒मो दक्षि॑णा॒ समृ॑द्ध्यै।
ब॒हु वै पुरु॑षो मे॒ध्यमुप॑गच्छति।
वै॒श्वा॒न॒रं द्वाद॑शकपालं॒ निर्व॑पति।
सं॒व॒थ्स॒रो वा अ॒ग्निर्वै᳚श्वान॒रः।
सं॒व॒थ्स॒रेणै॒वैनꣴ॑ स्वदयति।
हिर॑ण्यं॒ दक्षि॑णा॥१४॥

%1.7.2.6
प॒वित्रं॒ वै हिर॑ण्यम्।
पु॒नात्ये॒वैनम्᳚।
ब॒हु वै रा॑ज॒न्यो\-ऽनृ॑तं करोति।
उप॑ जा॒म्यै हर॑ते।
जि॒नाति॑ ब्राह्म॒णम्।
वद॒त्यनृ॑तम्।
अनृ॑ते॒ खलु॒ वै क्रि॒यमा॑णे॒ वरु॑णो गृह्णाति।
वा॒रु॒णं य॑व॒मयं॑ च॒रुं निर्व॑पति।
व॒रु॒ण॒पा॒शादे॒वैनं॑ मुञ्चति।
अश्वो॒ दक्षि॑णा।
वा॒रु॒णो हि दे॒वत॒या\-ऽश्वः॒ समृ॑द्ध्यै॥१५॥\anuvakamend[ऐ॒न्द्रा॒वै॒ष्ण॒वमेका॑\-दश\-कपालं॒ यदृ॑ष॒भो दधा॑ति पू॒षा प॒शून्प्रज॑नयति॒ हिर॑ण्यं॒ दक्षि॑णा॒ दक्षि॒णैकं॑ च]

%1.7.3.1
र॒त्निना॑मे॒तानि॑ ह॒वीꣳषि॑ भवन्ति।
ए॒ते वै रा॒ष्ट्रस्य॑ प्रदा॒तारः॑।
ए॒ते॑\-ऽपादा॒तारः॑।
य ए॒व रा॒ष्ट्रस्य॑ प्रदा॒तारः॑।
ये॑ऽपादा॒तारः॑।
त ए॒वास्मै॑ रा॒ष्ट्रं प्रय॑च्छन्ति।
रा॒ष्ट्रमे॒व भ॑वति।
यथ्स॑मा॒हृत्य॑ नि॒र्वपे᳚त्।
अर॑त्निनः स्युः।
य॒था॒य॒थं निर्व॑पति रत्नि॒त्वाय॑॥१६॥

%1.7.3.2
यथ्स॒द्यो नि॒र्वपे᳚त्।
याव॑ती॒मेके॑न ह॒विषा॒\-ऽऽशिष॑मव रु॒न्धे।
ताव॑ती॒मव॑रुन्धीत।
अ॒न्व॒हन्निर्व॑पति।
भूय॑सीमे॒वाशिष॒मव॑ रुन्धे।
भूय॑सो यज्ञक्र॒तूनुपै॑ति।
बा॒र्॒ह॒स्प॒त्यं च॒रुं निर्व॑पति ब्र॒ह्मणो॑ गृ॒हे।
मु॒ख॒त ए॒वास्मै॒ ब्रह्म॒ सꣴश्य॑ति।
अथो॒ ब्रह्म॑न्ने॒व क्ष॒त्रम॒न्वार॑म्भयति।
शि॒ति॒पृ॒ष्ठो दक्षि॑णा॒ समृ॑द्ध्यै॥१७॥

%1.7.3.3
ऐ॒न्द्रमेका॑\-दश\-कपालꣳ राज॒न्य॑स्य गृ॒हे।
इ॒न्द्रि॒यमे॒वाव॑ रुन्धे।
ऋ॒ष॒भो दक्षि॑णा॒ समृ॑द्ध्यै।
आ॒दि॒त्यं च॒रुं महि॑ष्यै गृ॒हे।
इ॒यं वा अदि॑तिः।
अ॒स्यामे॒व प्रति॑तिष्ठति।
धे॒नुर्दक्षि॑णा॒ समृ॑द्ध्यै।
भगा॑य च॒रुं वा॒वाता॑यै गृ॒हे।
भग॑मे॒वास्मि॑न्दधाति।
विचि॑त्तगर्भा पष्ठौ॒ही दक्षि॑णा॒ समृ॑द्ध्यै॥१८॥

%1.7.3.4
नै॒र्॒ऋ॒तं च॒रुं प॑रिवृ॒क्त्यै॑ गृ॒हे कृ॒ष्णानां᳚ व्रीही॒णां न॒खनि॑र्भिन्नम्।
पा॒प्मान॑मे॒व निर्\mbox{}ऋ॑तिं नि॒रव॑दयते।
कृ॒ष्णा कू॒टा दक्षि॑णा॒ समृ॑द्ध्यै।
आ॒ग्ने॒यम॒ष्टा\-क॑पालꣳ सेना॒न्यो॑ गृ॒हे।
सेना॑मे॒वास्य॒ सꣴश्य॑ति।
हिर॑ण्यं॒ दक्षि॑णा॒ समृ॑द्ध्यै।
वा॒रु॒णं दश॑कपालꣳ सू॒तस्य॑ गृ॒हे।
व॒रु॒ण॒स॒वमे॒वाव॑ रुन्धे।
म॒हानि॑रष्टो॒ दक्षि॑णा॒ समृ॑द्ध्यै।
मा॒रु॒तꣳ स॒प्तक॑पालं ग्राम॒ण्यो॑ गृ॒हे॥१९॥

%1.7.3.5
अन्नं॒ वै म॒रुतः॑।
अन्न॑मे॒वाव॑ रुन्धे।
पृश्ञि॒र्दक्षि॑णा॒ समृ॑द्ध्यै।
सा॒वि॒त्रं द्वाद॑शकपालं क्ष॒त्तुर्गृ॒हे प्रसू᳚त्यै।
उ॒प॒ध्व॒स्तो दक्षि॑णा॒ समृ॑द्ध्यै।
आ॒श्वि॒नं द्वि॑कपा॒लꣳ स॑ङ्ग्रही॒तुर्गृ॒हे।
अ॒श्विनौ॒ वै दे॒वानां᳚ भि॒षजौ᳚।
ताभ्या॑मे॒वास्मै॑ भेष॒जं क॑रोति।
स॒वा॒त्यौ॑ दक्षि॑णा॒ समृ॑द्ध्यै।
पौ॒ष्णं च॒रुं भा॑गदु॒घस्य॑ गृ॒हे॥२०॥

%1.7.3.6
अन्नं॒ वै पू॒षा।
अन्न॑मे॒वाव॑ रुन्धे।
श्या॒मो दक्षि॑णा॒ समृ॑द्ध्यै।
रौ॒द्रं गा॑वीधु॒कं च॒रुम॑क्षावा॒पस्य॑ गृ॒हे।
अ॒न्त॒त ए॒व रु॒द्रं नि॒रव॑दयते।
श॒बल॒ उद्वा॑रो॒ दक्षि॑णा॒ समृ॑द्ध्यै।
द्वाद॑शै॒तानि॑ ह॒वीꣳषि॑ भवन्ति।
द्वाद॑श॒ मासाः᳚ संवथ्स॒रः।
सं॒व॒थ्स॒रेणै॒वास्मै॑ रा॒ष्ट्रमव॑ रुन्धे।
रा॒ष्ट्रमे॒व भ॑वति॥२१॥

%1.7.3.7
यन्न प्र॑ति नि॒र्वपे᳚त्।
र॒त्निन॑ आ॒शिषो\-ऽव॑रुन्धीरन्।
प्रति॒निर्व॑पति।
इन्द्रा॑य सु॒त्राम्णे॑ पुरो॒डाश॒मेका॑\-दश\-कपालम्।
इन्द्रा॑याꣳहो॒मुचे᳚।
आ॒शिष॑ ए॒वाव॑ रुन्धे।
अ॒यं नो॒ राजा॑ वृत्र॒हा राजा॑ भू॒त्वा वृ॒त्रं व॑ध्या॒दित्या॑ह।
आ॒\-मे॒वैतामा शा᳚स्ते।
मै॒त्रा॒बा॒र्॒ह॒स्प॒त्यं भ॑वति।
श्वे॒तायै᳚ श्वे॒तव॑थ्सायै दु॒ग्धे॥२२॥

%1.7.3.8
बा॒र्॒ह॒स्प॒त्ये मै॒त्रमपि॑ दधाति।
ब्रह्म॑ चै॒वास्मै᳚ क्ष॒त्रं च॑ स॒मीची॑ दधाति।
अथो॒ ब्रह्म॑न्ने॒व क्ष॒त्रं प्रति॑\-ष्ठापयति।
बा॒र्॒ह॒स्प॒त्येन॒ पूर्वे॑ण॒ प्रच॑रति।
मु॒ख॒त ए॒वास्मै॒ ब्रह्म॒ सꣴश्य॑ति।
अथो॒ ब्रह्म॑न्ने॒व क्ष॒त्रम॒न्वार॑म्भयति।
स्व॒यं॒ कृ॒ता वेदि॑र्भवति।
स्व॒यं॒ दि॒नं ब॒र्॒हिः।
स्व॒यं॒ कृ॒त इ॒ध्मः।
अन॑भिजितस्या॒भिजि॑त्यै।
तस्मा॒द्राज्ञा॒मर॑ण्यम॒भिजि॑तम्।
सैव श्वे॒ता श्वे॒तव॑थ्सा॒ दक्षि॑णा॒ समृ॑द्ध्यै॥२३॥\anuvakamend[र॒त्नि॒त्वाय॒ समृ॑द्ध्यै पष्ठौ॒ही दक्षि॑णा॒ समृ॑द्ध्यै ग्राम॒ण्यो॑ गृ॒हे भा॑गदु॒घस्य॑ गृ॒हे भ॑वति दु॒ग्धे॑\-ऽभिजि॑त्यै॒ द्वे च॑]

%1.7.4.1
दे॒व॒सु॒वामे॒तानि॑ ह॒वीꣳषि॑ भवन्ति।
ए॒ताव॑न्तो॒ वै दे॒वानाꣳ॑ स॒वाः।
त ए॒वास्मै॑ स॒वान्प्रय॑च्छन्ति।
त ए॑नꣳ सुवन्ते।
अ॒ग्निरे॒वैनं॑ गृ॒हप॑तीनाꣳ सुवते।
सोमो॒ वन॒स्पती॑नाम्।
रु॒द्रः प॑शू॒नाम्।
बृह॒स्पति॑र्वा॒चाम्।
इन्द्रो᳚ ज्ये॒ष्ठाना᳚म्।
मि॒त्रः स॒त्याना᳚म्॥२४॥

%1.7.4.2
वरु॑णो॒ धर्म॑पतीनाम्।
ए॒तदे॒व सर्वं॑ भवति।
स॒वि॒ता त्वा᳚ प्रस॒वानाꣳ॑ सुवता॒मिति॒ हस्तं॑ गृह्णाति॒ प्रसू᳚त्यै।
ये दे॑वा देवः॒ सुवः॒ स्थेत्या॑ह।
य॒था॒\-य॒जु\-रे॒वै\-तत्।
म॒ह॒ते क्ष॒त्राय॑ मह॒त आधि॑पत्याय मह॒ते जान॑राज्या॒येत्या॑ह।
आ॒\-मे॒वैतामा शा᳚स्ते।
ए॒ष वो॑ भरता॒ राजा॒ सोमो॒\-ऽस्माकं॑ ब्राह्म॒णाना॒ꣳ॒ राजेत्या॑ह।
तस्मा॒थ्सोम॑राजानो ब्राह्म॒णाः।
प्रति॒ त्यन्नाम॑ रा॒ज्यम॑धा॒यीत्या॑ह॥२५॥

%1.7.4.3
रा॒ज्यमे॒वास्मि॒न्प्रति॑\-दधाति।
स्वां त॒नुवं॒ वरु॑णो अशिश्रे॒दि\-त्या॑ह।
व॒रु॒ण॒स॒वमे॒वाव॑ रुन्धे।
शुचे᳚र्मि॒त्रस्य॒ व्रत्या॑ अभू॒मेत्या॑ह।
शुचि॑मे॒वैनं॒ व्रत्यं॑ करोति।
अम॑न्महि मह॒त ऋ॒तस्य॒ नामेत्या॑ह।
म॒नु॒त ए॒वैनम्᳚।
सर्वे॒ व्राता॒ वरु॑णस्याभूव॒न्नित्या॑ह।
सर्व॑व्रातमे॒वैनं॑ करोति।
वि मि॒त्र एवै॒ररा॑तिमतारी॒दित्या॑ह॥२६॥

%1.7.4.4
अरा॑तिमे॒वैनं॑ तारयति।
असू॑षुदन्त य॒ज्ञिया॑ ऋ॒तेनेत्या॑ह।
स्व॒दय॑त्ये॒वैनम्᳚।
व्यु॑ त्रि॒तो ज॑रि॒माणं॑ न आन॒डित्या॑ह।
आयु॑रे॒वास्मि॑न्दधाति।
द्वाभ्यां॒ विमृ॑ष्टे।
द्वि॒पाद्यज॑मानः॒ प्रति॑\-ष्ठित्यै।
अ॒ग्नी॒षो॒मीय॑स्य॒ चैका॑\-दश\-कपालस्य देवसु॒वां च॑ ह॒विषा॑म॒ग्नये᳚ स्विष्ट॒कृते॑ स॒मव॑द्यति।
दे॒वता॑भिरे॒वैन॑मुभ॒यतः॒ परि॑गृह्णाति।
वि॒ष्णु॒क्र॒मान्क्र॑मते।
विष्णु॑रे॒व भू॒त्वेमाँल्लो॒कान॒भि\-ज॑यति॥२७॥\anuvakamend[स॒त्याना॑मधा॒यीत्या॑हातारी॒दित्या॑ह क्रमत॒ एकं॑ च]

%1.7.5.1
अ॒र्थेतः॒ स्थेति॑ जुहोति।
आहु॑त्यै॒वैना॑ नि॒ष्क्रीय॑ गृह्णाति।
अथो॑ ह॒विष्कृ॑तानामे॒वाभिघृ॑तानां गृह्णाति।
वह॑न्तीनां गृह्णाति।
ए॒ता वा अ॒पाꣳ रा॒ष्ट्रम्।
रा॒ष्ट्रमे॒वास्मै॑ गृह्णाति।
अथो॒ श्रिय॑मे॒वैन॑म॒भिव॑हन्ति।
अ॒पां पति॑र॒सीत्या॑ह।
मि॒थु॒नमे॒वाकः॑।
वृषा᳚\-ऽस्यू॒र्मिरित्या॑ह॥२८॥

%1.7.5.2
ऊ॒र्मि॒मन्त॑मे॒वैनं॑ करोति।
वृ॒ष॒से॒नो॑\-ऽसीत्या॑ह।
सेना॑मे॒वास्य॒ सꣴश्य॑ति।
व्र॒ज॒क्षितः॒ स्थेत्या॑ह।
ए॒ता वा अ॒पां विशः॑।
विश॑मे॒वास्मै॒ पर्यू॑हति।
म॒रुता॒मोजः॒ स्थेत्या॑ह।
अन्नं॒ वै म॒रुतः॑।
अन्न॑मे॒वाव॑ रुन्धे।
सूर्य॑वर्चसः॒ स्थेत्या॑ह॥२९॥

%1.7.5.3
रा॒ष्ट्रमे॒व व॑र्च॒स्व्य॑कः।
सूर्य॑त्वचसः॒ स्थेत्या॑ह।
स॒त्यं वा ए॒तत्।
यद्वर्\mbox{}ष॑ति।
अनृ॑तं॒ यदा॒तप॑ति॒ वर्\mbox{}ष॑ति।
स॒त्या॒नृ॒ते ए॒वाव॑ रुन्धे।
नैनꣳ॑ सत्यानृ॒ते उ॑दि॒ते हिꣴ॑स्तः।
य ए॒वं वेद॑।
मान्दाः॒ स्थेत्या॑ह।
रा॒ष्ट्रमे॒व ब्र॑ह्म\-वर्च॒स्य॑कः॥३०॥

%1.7.5.4
वाशाः॒ स्थेत्या॑ह।
रा॒ष्ट्रमे॒व व॒श्य॑कः।
शक्व॑रीः॒ स्थेत्या॑ह।
प॒शवो॒ वै शक्व॑रीः।
प॒शूने॒वाव॑ रुन्धे।
वि॒श्व॒भृतः॒ स्थेत्या॑ह।
रा॒ष्ट्रमे॒व प॑य॒स्व्य॑कः।
ज॒न॒भृतः॒ स्थेत्या॑ह।
रा॒ष्ट्रमे॒वेन्द्रि॑या॒व्य॑कः।
अ॒ग्नेस्ते॑ज॒स्याः᳚ स्थेत्या॑ह॥३१॥

%1.7.5.5
रा॒ष्ट्रमे॒व ते॑ज॒स्व्य॑कः।
अ॒पामोष॑धीना॒ꣳ॒ रसः॒ स्थेत्या॑ह।
रा॒ष्ट्रमे॒व म॑ध॒व्य॑मकः।
सा॒र॒स्व॒तं ग्रहं॑ गृह्णाति।
ए॒षा वा अ॒पां पृ॒ष्ठम्।
यथ्सर॑स्वती।
पृ॒ष्ठमे॒वैनꣳ॑ समा॒नानां᳚ करोति।
षो॒ड॒शभि॑र्गृह्णाति।
षोड॑शकलो॒ वै पुरु॑षः।
यावा॑ने॒व पुरु॑षः।
तस्मि॑न्वी॒र्यं॑ दधाति।
षो॒ड॒शभि॑र्जु॒होति॑ षोड॒शभि॑र्गृह्णाति।
द्वात्रिꣳ॑श॒थ्सम्प॑द्यन्ते।
द्वात्रिꣳ॑शदक्षरा\-ऽनु॒ष्टुक्।
वाग॑नु॒ष्टुफ्सर्वा॑णि॒ छन्दाꣳ॑सि।
वा॒चैवैन॒ꣳ॒ सर्वे॑भि॒श्छन्दो॑भिर॒भिषि॑ञ्चति॥३२॥\anuvakamend[ऊ॒र्मिरित्या॑ह॒ सूर्य॑वर्चसः॒ स्थेत्या॑ह ब्रह्मवर्च॒स्य॑कस्तेज॒स्याः᳚ स्थेत्या॑है॒व पुरु॑षः॒ षट् च॑]

%1.7.6.1
देवी॑रापः॒ सं मधु॑मती॒र्मधु॑मतीभिः सृज्यध्व॒मित्या॑ह।
ब्रह्म॑णै॒वैनाः॒ सꣳसृ॑जति।
अना॑धृष्टाः सीद॒तेत्या॑ह।
ब्रह्म॑णै॒वैनाः᳚ सादयति।
अ॒न्त॒रा होतु॑श्च॒ धिष्णि॑यं ब्राह्मणाच्छ॒ꣳ॒सिन॑श्च सादयति।
आ॒ग्ने॒यो वै होता᳚।
ऐ॒न्द्रो ब्रा᳚ह्मणाच्छ॒ꣳ॒सी।
तेज॑सा चै॒वेन्द्रि॒येण॑ चोभ॒यतो॑ रा॒ष्ट्रं परि॑गृह्णाति।
हिर॑ण्ये॒नोत्पु॑नाति।
आहु॑त्यै॒ हि प॒वित्रा᳚भ्यामुत्पु॒नन्ति॒ व्यावृ॑त्त्यै॥३३॥

%1.7.6.2
श॒तमा॑नं भवति।
श॒तायुः॒ पुरु॑षः श॒तेन्द्रि॑यः।
आयु॑ष्ये॒वेन्द्रि॒ये प्रति॑ तिष्ठति।
अनि॑भृष्टम॒सीत्या॑ह।
अनि॑भृष्ट॒ꣴ॒ ह्ये॑तत्।
वा॒चो बन्धु॒रित्या॑ह।
वा॒चो ह्ये॑ष बन्धुः॑।
त॒पो॒जा इत्या॑ह।
त॒पो॒जा ह्ये॑तत्।
सोम॑स्य दा॒त्रम॒सीत्या॑ह॥३४॥

%1.7.6.3
सोम॑स्य॒ ह्ये॑तद्दा॒त्रम्।
शु॒क्रा वः॑ शु॒क्रेणोत्पु॑ना॒मीत्या॑ह।
शु॒क्रा ह्यापः॑।
शु॒क्रꣳ हिर॑ण्यम्।
च॒न्द्राश्च॒न्द्रेणेत्या॑ह।
च॒न्द्रा ह्यापः॑।
च॒न्द्रꣳ हिर॑ण्यम्।
अ॒मृता॑ अ॒मृते॒नेत्या॑ह।
अ॒मृता॒ ह्यापः॑।
अ॒मृत॒ꣳ॒ हिर॑ण्यम्॥३५॥

%1.7.6.4
स्वाहा॑ राज॒सूया॒येत्या॑ह।
रा॒ज॒सूया॑य॒ ह्ये॑ना उत्पु॒नाति॑।
स॒ध॒मादो᳚ द्यु॒म्निनी॒रूर्ज॑ ए॒ता इति॑ वारु॒ण्यर्चा गृ॑ह्णाति।
व॒रु॒ण॒स॒वमे॒वाव॑ रुन्धे।
एक॑या गृह्णाति।
ए॒क॒धैव यज॑माने वी॒र्यं॑ दधाति।
क्ष॒त्रस्योल्ब॑मसि क्ष॒त्रस्य॒ योनि॑र॒सीति॑ ता॒र्प्यं चो॒ष्णीषं॑ च॒ प्रय॑च्छति सयोनि॒त्वाय॑।
एक॑शतेन दर्भपुञ्जी॒लैः प॑वयति।
श॒तायु॒र्वै पुरु॑षः श॒तवी᳚र्यः।
आ॒त्मैक॑श॒तः॥३६॥

%1.7.6.5
यावा॑ने॒व पुरु॑षः।
तस्मि॑न्वी॒र्यं॑ दधाति।
दध्या॑शयति।
इ॒न्द्रि॒यमे॒वाव॑ रुन्धे।
उ॒दु॒म्बर॑माशयति।
अ॒न्नाद्य॒स्या\-व॑\-रुद्ध्यै।
शष्पा᳚ण्याशयति।
सुरा॑बलिमे॒वैनं॑ करोति।
आ॒विद॑ ए॒ता भ॑वन्ति।
आ॒विद॑मे॒वैनं॑ गमयन्ति॥३७॥

%1.7.6.6
अ॒ग्निरे॒वैनं॒ गार्\mbox{}ह॑पत्येनावति।
इन्द्र॑ इन्द्रि॒येण॑।
पू॒षा प॒शुभिः॑।
मि॒त्रावरु॑णौ प्राणापा॒नाभ्या᳚म्।
इन्द्रो॑ वृ॒त्राय॒ वज्र॒मुद॑यच्छत्।
स दिव॑मलिखत्।
सो᳚ऽर्य॒म्णः पन्था॑ अभवत्।
स आवि॑न्ने॒ द्यावा॑पृथि॒वी धृ॒तव्र॑ते॒ इति॒ द्यावा॑पृथि॒वी उपा॑धावत्।
स आ॒भ्यामे॒व प्रसू॑त॒ इन्द्रो॑ वृ॒त्राय॒ वज्रं॒ प्राह॑रत्।
आवि॑न्ने॒ द्यावा॑पृथि॒वी धृ॒तव्र॑ते॒ इति॒ यदाह॑॥३८॥

%1.7.6.7
आ॒भ्यामे॒व प्रसू॑तो॒ यज॑मानो॒ वज्रं॒ भ्रातृ॑व्याय॒ प्रह॑रति।
आवि॑न्ना दे॒व्यदि॑तिर्विश्वरू॒पीत्या॑ह।
इ॒यं वै दे॒व्यदि॑तिर्विश्व\-रू॒पी।
अ॒स्यामे॒व प्रति॑ तिष्ठति।
आवि॑न्नो॒\-ऽयम॒सावा॑मुष्या\-य॒णो᳚\-ऽस्यां वि॒श्य॑स्मिन्रा॒ष्ट्र इत्या॑ह।
वि॒शैवैनꣳ॑ रा॒ष्ट्रेण॒ सम॑र्धयति।
म॒ह॒ते क्ष॒त्राय॑ मह॒त आधि॑पत्याय मह॒ते जान॑राज्या॒येत्या॑ह।
आ॒\-मे॒वैतामा शा᳚स्ते।
ए॒ष वो॑ भरता॒ राजा॒ सोमो॒\-ऽस्माकं॑ ब्राह्म॒णाना॒ꣳ॒ राजेत्या॑ह।
तस्मा॒थ्सोम॑राजानो ब्राह्म॒णाः॥३९॥

%1.7.6.8
इन्द्र॑स्य॒ वज्रो॑ऽसि॒ वार्त्र॑घ्न॒ इति॒ धनुः॒ प्रय॑च्छति॒ विजि॑त्यै।
श॒त्रु॒बाध॑नाः॒ स्थेतीषून्॑।
शत्रू॑ने॒वास्य॑ बाधन्ते।
पा॒त मा᳚ प्र॒त्यञ्चं॑ पा॒त मा॑ ति॒र्यञ्च॑म॒न्वञ्चं॑ मा पा॒तेत्या॑ह।
ति॒स्रो वै श॑र॒व्याः᳚।
प्र॒तीची॑ ति॒रश्च्य॒नूची᳚।
ताभ्य॑ ए॒वैनं॑ पान्ति।
दि॒ग्भ्यो मा॑ पा॒तेत्या॑ह।
दि॒ग्भ्य ए॒वैनं॑ पान्ति।
विश्वा᳚भ्यो मा ना॒ष्ट्राभ्यः॑ पा॒तेत्या॑ह।
अप॑रिमितादे॒वैनं॑ पान्ति।
हिर॑ण्यवर्णावु॒षसां᳚ विरो॒क इति॑ त्रि॒ष्टुभा॑ बा॒हू उद्गृ॑ह्णाति।
इ॒न्द्रि॒यं वै वी॒र्यं॑ त्रि॒ष्टुक्।
इ॒न्द्रि॒यमे॒व वी॒र्य॑मु॒परि॑ष्टादा॒त्मन्ध॑त्ते॥४०॥\anuvakamend[व्यावृ॑त्त्यै दा॒त्रम॒सीत्या॑हा॒मृत॒ꣳ॒ हिर॑ण्यमेकश॒तो ग॑मय॒न्त्याह॑ ब्राह्म॒णा ना॒ष्ट्राभ्यः॑ पा॒तेत्या॑ह च॒त्वारि॑ च]

%1.7.7.1
दिशो॒ व्यास्था॑पयति।
दि॒शाम॒भिजि॑त्त्यै।
यद॑नु प्र॒क्रामे᳚त्।
अ॒भि दिशो॑ जयेत्।
उत्तु मा᳚द्येत्।
मन॒साऽनु॒ प्रक्रा॑मति।
अ॒भि दिशो॑ जयति।
नोन्मा᳚द्यति।
स॒मिध॒मा ति॒ष्ठेत्या॑ह।
तेज॑ ए॒वाव॑ रुन्धे॥४१॥

%1.7.7.2
उ॒ग्रामा ति॒ष्ठेत्या॑ह।
इ॒न्द्रि॒यमे॒वाव॑ रुन्धे।
वि॒राज॒माति॒ष्ठेत्या॑ह।
अ॒न्नाद्य॑मे॒वाव॑ रुन्धे।
उदी॑ची॒मा ति॒ष्ठेत्या॑ह।
प॒शूने॒वाव॑ रुन्धे।
ऊ॒र्ध्वामाति॒ष्ठेत्या॑ह।
सु॒व॒र्गमे॒व लो॒कम॒भि\-ज॑यति।
अनूज्जि॑हीते।
सु॒व॒र्गस्य॑ लो॒कस्य॒ सम॑ष्ट्यै॥४२॥

%1.7.7.3
मा॒रु॒त ए॒ष भ॑वति।
अन्नं॒ वै म॒रुतः॑।
अन्न॑मे॒वाव॑ रुन्धे।
एक॑विꣳशतिकपालो भवति॒ प्रति॑\-ष्ठित्यै।
यो॑ऽरण्ये\-ऽनुवा॒क्यो॑ ग॒णः।
तं म॑ध्य॒त उप॑दधाति।
ग्रा॒म्यैरे॒व प॒शुभि॑रार॒ण्यान्प॒शू\-न्परि॑ गृह्णाति।
तस्मा᳚द्ग्रा॒म्यैः प॒शुभि॑रार॒ण्याः प॒शवः॒ परि॑गृहीताः।
पृथि॑र्वै॒न्यः।
अ॒भ्य॑षिच्यत॥४३॥

%1.7.7.4
स रा॒ष्ट्रं नाभ॑वत्।
स ए॒तानि॑ पा॒र्थान्य॑पश्यत्।
तान्य॑जुहोत्।
तैर्वै स रा॒ष्ट्रम॑भवत्।
यत्पा॒र्थानि॑ जु॒होति॑।
रा॒ष्ट्रमे॒व भ॑वति।
बा॒र्॒ह॒स्प॒त्यं पूर्वे॑षामुत्त॒मं भ॑वति।
ऐ॒न्द्रमुत्त॑रेषां प्रथ॒मम्।
ब्रह्म॑ चै॒वास्मै᳚ क्ष॒त्रं च॑ स॒मीची॑ दधाति।
अथो॒ ब्रह्म॑न्ने॒व क्ष॒त्रं प्रति॑\-ष्ठापयति॥४४॥

%1.7.7.5
षट्पु॒रस्ता॑दभिषे॒कस्य॑ जुहोति।
षडु॒परि॑ष्टात्।
द्वाद॑श॒ सम्प॑द्यन्ते।
द्वाद॑श॒ मासाः᳚ संवथ्स॒रः।
सं॒व॒थ्स॒रः खलु॒ वै दे॒वानां॒ पूः।
दे॒वाना॑मे॒व पुरं॑ मध्य॒तो व्यव॑सर्पति।
तस्य॒ न कुत॑श्च॒नोपा᳚व्या॒धो भ॑वति।
भू॒ताना॒मवे᳚ष्टीर्जुहोति।
अत्रा᳚त्र॒ वै मृ॒त्युर्जा॑यते।
यत्र॑यत्रै॒व मृ॒त्युर्जाय॑ते।
तत॑ ए॒वैन॒मव॑यजते।
तस्मा᳚द्राज॒सूये॑नेजा॒नो नाभिच॑रित॒वै।
प्र॒त्यगे॑नमभिचा॒रः स्तृ॑णुते॥४५॥\anuvakamend[रु॒न्धे॒ सम॑ष्ट्या असिच्यत स्थापयति॒ जाय॑ते॒ पञ्च॑ च]

%1.7.8.1
सोम॑स्य॒ त्विषि॑रसि॒ तवे॑व मे॒ त्विषि॑र्भूया॒दिति॑ शार्दूल\-च॒र्मोप॑स्तृणाति।
यैव सोमे॒ त्विषिः॑।
या शा᳚र्दू॒ले।
तामे॒वाव॑ रुन्धे।
मृ॒त्योर्वा ए॒ष वर्णः॑।
यच्छा᳚र्दू॒लः।
अ॒मृत॒ꣳ॒ हिर॑ण्यम्।
अ॒मृत॑मसि मृ॒त्योर्मा॑ पा॒हीति॒ हिर॑ण्य॒मुपा᳚स्यति।
अ॒मृत॑मे॒व मृ॒त्योर॒न्तर्ध॑त्ते।
श॒तमा॑नं भवति॥४६॥

%1.7.8.2
श॒तायुः॒ पुरु॑षः श॒तेन्द्रि॑यः।
आयु॑ष्ये॒वेन्द्रि॒ये प्रति॑ तिष्ठति।
दि॒द्योन्मा॑ पा॒हीत्यु॒परि॑ष्टा॒दधि॒ निद॑धाति।
उ॒भ॒यत॑ ए॒वास्मै॒ शर्म॑ दधाति।
अवे᳚ष्टा दन्द॒शूका॒ इति॑ क्ली॒बꣳ सीसे॑न विध्यति।
द॒न्द॒शूका॑ने॒वाव॑यजते।
तस्मा᳚त्क्ली॒बं द॑न्द॒शूका॒ दꣳशु॑काः।
निर॑स्तं॒ नमु॑चेः॒ शिर॒ इति॑ लोहिताय॒सं निर॑स्यति।
पा॒प्मान॑मे॒व नमु॑चिं नि॒रव॑दयते।
प्रा॒णा आ॒त्मनः॒ पूर्वे॑\-ऽभि॒षिच्या॒ इत्या॑हुः॥४७॥

%1.7.8.3
सोमो॒ राजा॒ वरु॑णः।
दे॒वा ध॑र्म॒सुव॑श्च॒ ये।
ते ते॒ वाचꣳ॑ सुवन्तां॒ ते ते᳚ प्रा॒णꣳ सु॑वन्ता॒मित्या॑ह।
प्रा॒णाने॒वा\-ऽऽत्मनः॒ पूर्वा॑न॒भिषि॑ञ्चति।
यद्ब्रू॒यात्।
अ॒ग्नेस्त्वा॒ तेज॑सा॒\-ऽभिषि॑ञ्चा॒मीति॑।
ते॒ज॒स्व्ये॑व स्या᳚त्।
दु॒श्चर्मा॒ तु भ॑वेत्।
सोम॑स्य त्वा द्यु॒म्नेना॒भिषि॑ञ्चा॒मीत्या॑ह।
सौ॒म्यो वै दे॒वत॑या॒ पुरु॑षः॥४८॥

%1.7.8.4
स्वयै॒वैनं॑ दे॒वत॑या॒\-ऽभिषि॑ञ्चति।
अ॒ग्नेस्तेज॒सेत्या॑ह।
तेज॑ ए॒वास्मि॑न्दधाति।
सूर्य॑स्य॒ वर्च॒सेत्या॑ह।
वर्च॑ ए॒वास्मि॑न्दधाति।
इन्द्र॑स्येन्द्रि॒येणेत्या॑ह।
इ॒न्द्रि॒यमे॒वास्मि॑न्दधाति।
मि॒त्रावरु॑ण\-योर्वी॒र्ये॑णेत्या॑ह।
वी॒र्य॑मे॒वास्मि॑न्दधाति।
म॒रुता॒मोज॒सेत्या॑ह॥४९॥

%1.7.8.5
ओज॑ ए॒वास्मि॑न्दधाति।
क्ष॒त्राणां᳚ क्ष॒त्रप॑तिर॒सीत्या॑ह।
क्ष॒त्राणा॑मे॒वैनं॑ क्ष॒त्रप॑तिं करोति।
अति॑ दि॒वस्पा॒हीत्या॑ह।
अत्य॒न्यान्पा॒हीति॒ वावैतदा॑ह।
स॒माव॑वृत्रन्नध॒रागुदी॑ची॒\-रित्या॑ह।
रा॒ष्ट्रमे॒वास्मि॑न्ध्रु॒वम॑कः।
उ॒च्छेष॑णेन जुहोति।
उ॒च्छेष॑णभागो॒ वै रु॒द्रः।
भा॒ग॒धेये॑नै॒व रु॒द्रं नि॒रव॑दयते॥५०॥

%1.7.8.6
उद॑ङ्प॒रेत्याग्नी᳚द्ध्रे जुहोति।
ए॒षा वै रु॒द्रस्य॒ दिक्।
स्वाया॑मे॒व दि॒शि रु॒द्रं नि॒रव॑दयते।
रुद्र॒ यत्ते॒ क्रयी॒ परं॒ नामेत्या॑ह।
यद्वा अ॑स्य॒ क्रयी॒ परं॒ नाम॑।
तेन॒ वा ए॒ष हि॑नस्ति।
यꣳ हि॒नस्ति॑।
तेनै॒वैनꣳ॑ स॒ह श॑मयति।
तस्मै॑ हु॒तम॑सि य॒मेष्ट॑म॒सीत्या॑ह।
य॒मादे॒वास्य॑ मृ॒त्युमव॑यजते॥५१॥

%1.7.8.7
प्रजा॑पते॒ न त्वदे॒तान्य॒न्य इति॒ तस्यै॑ गृ॒हे जु॑हुयात्।
यां का॒मये॑त रा॒ष्ट्रम॑स्यै प्र॒जा स्या॒दिति॑।
रा॒ष्ट्रमे॒वास्यै᳚ प्र॒जा भ॑वति।
प॒र्ण॒मये॑नाध्व॒र्युर॒भिषि॑ञ्चति।
ब्र॒ह्म॒व॒र्च॒समे॒वा\-स्मि॒न्त्विषिं॑ दधाति।
औदु॑म्बरेण राज॒न्यः॑।
ऊर्ज॑मे॒वा\-स्मि॑न्न॒न्नाद्यं॑ दधाति।
आश्व॑त्थेन॒ वैश्यः॑।
विश॑मे॒वास्मि॒न्पुष्टिं॑ दधाति।
नैय॑ग्रोधेन॒ जन्यः॑।
मि॒त्राण्ये॒वास्मै॑ कल्पयति।
अथो॒ प्रति॑\-ष्ठित्यै॥५२॥\anuvakamend[भ॒व॒त्या॒हुः॒ पुरु॑ष॒ ओज॒सेत्या॑ह नि॒रव॑दयते यजते॒ जन्यो॒ द्वे च॑]

%1.7.9.1
इन्द्र॑स्य॒ वज्रो॑ऽसि॒ वार्त्र॑घ्न॒ इति॒ रथ॑मु॒पाव॑\-हरति॒ विजि॑त्यै।
मि॒त्रावरु॑णयोस्त्वा प्रशा॒स्त्रोः प्र॒शिषा॑ युन॒ज्मीत्या॑ह।
ब्रह्म॑णै॒वैनं॑ दे॒वता᳚भ्यां युनक्ति।
प्र॒ष्टि॒वा॒हिनं॑ युनक्ति।
प्र॒ष्टि॒वा॒ही वै दे॑वर॒थः।
दे॒व॒र॒थमे॒वास्मै॑ युनक्ति।
त्रयो\-ऽश्वा॑ भवन्ति।
रथ॑श्चतु॒र्थः।
द्वौ स॑व्येष्ठसार॒थी।
षट्थ्सम्प॑द्यन्ते॥५३॥

%1.7.9.2
षड्वा ऋ॒तवः॑।
ऋ॒तुभि॑रे॒वैनं॑ युनक्ति।
वि॒ष्णु॒क्र॒मान्क्र॑मते।
विष्णु॑रे॒व भू॒त्वेमाँल्लो॒कान॒भि\-ज॑यति।
यः क्ष॒त्रियः॒ प्रति॑\-हितः।
सो᳚ऽन्वार॑भते।
रा॒ष्ट्रमे॒व भ॑वति।
त्रि॒ष्टुभा॒\-ऽन्वार॑भते।
इ॒न्द्रि॒यं वै त्रि॒ष्टुक्।
इ॒न्द्रि॒यमे॒व यज॑माने दधाति॥५४॥

%1.7.9.3
म॒रुतां᳚ प्रस॒वे जे॑ष॒मित्या॑ह।
म॒रुद्भि॑रे॒व प्रसू॑त॒ उज्ज॑यति।
आ॒प्तं मन॒ इत्या॑ह।
यदे॒व मन॒सैफ्सी᳚त्।
तदा॑पत्।
रा॒ज॒न्यं॑ जिनाति।
अना᳚क्रान्त ए॒वाक्र॑मते।
वि वा ए॒ष इ॑न्द्रि॒येण॑ वी॒र्ये॑णर्ध्यते।
यो रा॑ज॒न्यं॑ जि॒नाति॑।
सम॒हमि॑न्द्रि॒येण॑ वी॒र्ये॑णेत्या॑ह॥५५॥

%1.7.9.4
इ॒न्द्रि॒यमे॒व वी॒र्य॑मा॒त्मन्ध॑त्ते।
प॒शू॒नां म॒न्युर॑सि॒ तवे॑व मे म॒न्युर्भू॑या॒दिति॒ वारा॑ही उपा॒नहा॒वुप॑ मुञ्चते।
प॒शू॒नां वा ए॒ष म॒न्युः।
यद्व॑रा॒हः।
तेनै॒व प॑शू॒नां म॒न्युमा॒त्मन्ध॑त्ते।
अ॒भि वा इ॒यꣳ सु॑षुवा॒णं का॑मयते।
तस्ये᳚श्व॒रेन्द्रि॒यं वी॒र्य॑मादा॑तोः।
वारा॑ही उपा॒नहा॒वुप॑मुञ्चते।
अ॒स्या ए॒वान्तर्ध॑त्ते।
इ॒न्द्रि॒यस्य॑ वी॒र्य॑स्याना᳚त्यै॥५६॥

%1.7.9.5
नमो॑ मा॒त्रे पृ॑थि॒व्या इत्या॒हाहिꣳ॑सायै।
इय॑द॒स्यायु॑र॒स्यायु॑र्मे धे॒हीत्या॑ह।
आयु॑रे॒वाऽऽत्मन्ध॑त्ते।
ऊर्ग॒स्यूर्जं॑ मे धे॒हीत्या॑ह।
ऊर्ज॑मे॒वाऽऽत्मन्ध॑त्ते।
युङ्ङ॑सि॒ वर्चो॑ऽसि॒ वर्चो॒ मयि॑ धे॒हीत्या॑ह।
वर्च॑ ए॒वाऽऽत्मन्ध॑त्ते।
ए॒क॒धा ब्र॒ह्मण॒ उप॑हरति।
ए॒क॒धैव यज॑मान॒ आयु॒रूर्जं॒ वर्चो॑ दधाति।
र॒थ॒वि॒मो॒च॒नीया॑ जुहोति॒ प्रति॑\-ष्ठित्यै॥५७॥

%1.7.9.6
त्रयो\-ऽश्वा॑ भवन्ति।
रथ॑श्चतु॒र्थः।
तस्मा᳚च्च॒तुर्जु॑होति।
यदु॒भौ स॒हाव॒तिष्ठे॑ताम्।
स॒मा॒नं लो॒कमि॑याताम्।
स॒ह स॑ङ्ग्रही॒त्रा र॑थ॒वाह॑ने॒ रथ॒माद॑धाति।
सु॒व॒र्गादे॒वैनं॑ लो॒काद॒न्तर्द॑धाति।
ह॒ꣳ॒सः शु॑चि॒षदित्याद॑धाति।
ब्रह्म॑णै॒वैन॑मुपाव॒हर॑ति।
ब्रह्म॒णा\-ऽऽद॑धाति।
अति॑च्छन्द॒सा\-ऽऽद॑धाति।
अति॑च्छन्दा॒ वै सर्वा॑णि॒ छन्दाꣳ॑सि।
सर्वे॑भिरे॒वैनं॒ छन्दो॑भि॒राद॑धाति।
वर्ष्म॒ वा ए॒षा छन्द॑साम्।
यदति॑च्छन्दाः।
यदति॑च्छन्दसा॒ दधा॑ति।
वर्ष्मै॒वैनꣳ॑ समा॒नानां᳚ करोति॥५८॥\anuvakamend[प॒द्य॒न्ते॒ द॒धा॒ति॒ वी॒र्ये॑णेत्या॒हाना᳚त्यै॒ प्रति॑\-ष्ठित्यै॒ ब्रह्म॒णा\-ऽऽद॑धाति स॒प्त च॑]

%1.7.10.1
मि॒त्रो॑ऽसि॒ वरु॑णो॒\-ऽसीत्या॑ह।
मै॒त्रं वा अहः॑।
वा॒रु॒णी रात्रिः॑।
अ॒हो॒रा॒त्राभ्या॑मे॒वैन॑मु॒पाव॑\-हरति।
मि॒त्रो॑ऽसि॒ वरु॑णो॒\-ऽसीत्या॑ह।
मै॒त्रो वै दक्षि॑णः।
वा॒रु॒णः स॒व्यः।
वै॒श्व॒दे॒व्या॑मिक्षा᳚।
स्वमे॒वैनौ॑ भाग॒धेय॑मु॒पाव॑\-हरति।
सम॒हं विश्वै᳚र्दे॒वैरित्या॑ह॥५९॥

%1.7.10.2
वै॒श्व॒दे॒व्यो॑ वै प्र॒जाः।
ता ए॒वाद्याः᳚ कुरुते।
क्ष॒त्रस्य॒ नाभि॑रसि क्ष॒त्रस्य॒ योनि॑र॒सीत्य॑धीवा॒समास्तृ॑णाति सयोनि॒त्वाय॑।
स्यो॒नामा सी॑द सु॒षदा॒मा सी॒देत्या॑ह।
य॒था॒\-य॒जु\-रे॒वै\-तत्।
मा त्वा॑ हिꣳसी॒न्मा मा॑ हिꣳसी॒दित्या॒हाहिꣳ॑सायै।
निष॑साद धृ॒तव्र॑तो॒ वरु॑णः प॒स्त्या᳚स्वा साम्रा᳚ज्याय सु॒क्रतु॒रित्या॑ह।
साम्रा᳚ज्यमे॒वैनꣳ॑ सु॒क्रतुं॑ करोति।
ब्रह्मा(३)न्त्वꣳ रा॑जन्ब्र॒ह्माऽसि॑ सवि॒ताऽसि॑ स॒त्यस॑व॒ इत्या॑ह।
स॒वि॒तार॑मे॒वैनꣳ॑ स॒त्यस॑वं करोति॥६०॥

%1.7.10.3
ब्रह्मा(३)न्त्वꣳ रा॑जन्ब्र॒ह्मा\-ऽसीन्द्रो॑ऽसि स॒त्यौजा॒ इत्या॑ह।
इन्द्र॑मे॒वैनꣳ॑ स॒त्यौज॑सं करोति।
ब्रह्मा(३)न्त्वꣳ रा॑जन्ब्र॒ह्माऽसि॑ मि॒त्रो॑ऽसि सु॒शेव॒ इत्या॑ह।
मि॒त्रमे॒वैनꣳ॑ सु॒शेवं॑ करोति।
ब्रह्मा(३)न्त्वꣳ रा॑जन्ब्र॒ह्मासि॒ वरु॑णोऽसि स॒त्यध॒र्मेत्या॑ह।
वरु॑णमे॒वैनꣳ॑ स॒त्यध॑र्माणं करोति।
स॒वि॒ताऽसि॑ स॒त्यस॑व॒ इत्या॑ह।
गा॒य॒त्रीमे॒वैतेना॑भि॒ व्याह॑रति।
इन्द्रो॑ऽसि स॒त्यौजा॒ इत्या॑ह।
त्रि॒ष्टुभ॑मे॒वैतेना॑भि॒ व्याह॑रति॥६१॥

%1.7.10.4
मि॒त्रो॑ऽसि सु॒शेव॒ इत्या॑ह।
जग॑तीमे॒वैतेना॑भि॒ व्याह॑रति।
स॒त्यमे॒ता दे॒वताः᳚।
स॒त्यमे॒तानि॒ छन्दाꣳ॑सि।
स॒त्यमे॒वाव॑\- रुन्धे।
वरु॑णोऽसि स॒त्यध॒र्मेत्या॑ह।
अ॒नु॒ष्टुभ॑मे॒वैतेना॑भि॒ व्याह॑रति।
स॒त्या॒नृ॒ते वा अ॑नु॒ष्टुप्।
स॒त्या॒नृ॒ते वरु॑णः।
स॒त्या॒नृ॒ते ए॒वाव॑ रुन्धे॥६२॥

%1.7.10.5
नैनꣳ॑ सत्यानृ॒ते उ॑दि॒ते हिꣴ॑स्तः।
य ए॒वं वेद॑।
इन्द्र॑स्य॒ वज्रो॑ऽसि॒ वार्त्र॑घ्न॒ इति॒ स्फ्यं प्रय॑च्छति।
वज्रो॒ वै स्फ्यः।
वज्रे॑णै॒वास्मा॑ अवरप॒रꣳ र॑न्धयति।
ए॒वꣳ हि तच्छ्रेयः॑।
यद॑स्मा ए॒ते रध्ये॑युः।
दिशो॒\-ऽभ्य॑यꣳ राजा॑\-ऽभू॒दिति॒ पञ्चा॒क्षान्प्रय॑च्छति।
ए॒ते वै सर्वे\-ऽयाः᳚।
अप॑राजायिनमे॒वैनं॑ करोति॥६३॥

%1.7.10.6
ओ॒द॒नमुद्ब्रु॑वते।
प॒र॒मे॒ष्ठी वा ए॒षः।
यदो॑द॒नः।
प॒र॒मामे॒वैन॒ꣴ॒ श्रियं॑ गमयति।
सुश्लो॒काँ(४) सुम॑ङ्ग॒लाँ(४) सत्य॑रा॒जा(३)\-नित्या॑ह।
आ॒\-मे॒वैतामा शा᳚स्ते।
शौ॒नः॒ शे॒पमाख्या॑पयते।
व॒रु॒ण॒पा॒शादे॒वैनं॑ मुञ्चति।
प॒रः॒ श॒तं भ॑वति।
श॒तायुः॒ पुरु॑षः श॒तेन्द्रि॑यः।
आयु॑ष्ये॒वेन्द्रि॒ये प्रति॑ तिष्ठति।
मा॒रु॒तस्य॒ चैक॑विꣳशतिकपालस्य वैश्वदे॒व्यै चा॒मिक्षा॑या अ॒ग्नये᳚ स्विष्ट॒कृते॑ स॒मव॑द्यति।
दे॒वता॑भिरे॒वैन॑मुभ॒यतः॒ परि॑ गृह्णाति।
अ॒पान्नप्त्रे॒ स्वाहो॒र्जो नप्त्रे॒ स्वाहा॒\-ऽग्नये॑ गृ॒हप॑तये॒ स्वाहेति॑ ति॒स्र आहु॑तीर्जुहोति।
त्रय॑ इ॒मे लो॒काः।
ए॒ष्वे॑व लो॒केषु॒ प्रति॑ तिष्ठति॥६४॥\anuvakamend[दे॒वैरित्या॑ह स॒त्यस॑वं करोति त्रि॒ष्टुभ॑मे॒वैतेना॑भि॒ व्याह॑रति सत्यानृ॒ते ए॒वाव॑ रुन्धे करोति श॒तेन्द्रि॑यः॒ षट् च॑]




\prashnaend{ए॒तद्ब्रा᳚ह्मणानि धा॒त्रे र॒त्निना᳚न्देवसु॒वाम॒र्थेतो॒ देवी॒र्दिशः॒ सोम॒स्येन्द्र॑स्य मि॒त्रो दश॑॥१०॥}{ए॒तद्ब्रा᳚ह्मणानि वैष्ण॒वं त्रि॑कपा॒लमन्नं॒ वै पू॒षा वाशाः॒ स्थेत्या॑ह॒ दिशो॒ व्यास्था॑पय॒त्युद॑ङ्प॒रेत्य॒ ब्रह्मा(३)न्त्वꣳ रा॑ज॒ञ्चतु॑ष्षष्टिः॥६४॥}{ए॒तद्ब्रा᳚ह्मणानि॒ प्रति॑ तिष्ठति॥}{हरिः॑ ओम्॥}{इति श्रीकृष्णयजुर्वेदीयतैत्तिरीयब्राह्मणे प्रथमाष्टके सप्तमः प्रपाठकः समाप्तः॥}
\clearpage
\sect{अष्टमः प्रश्नः}
\setcounter{anuvakam}{0}
\dnsub{तैत्तिरीयब्राह्मणे प्रथमाष्टके अष्टमः प्रपाठकः}

%1.8.1.1
वरु॑णस्य सुषुवा॒णस्य॑ दश॒धेन्द्रि॒यं वी॒र्यं॑ परा॑\-ऽपतत्।
तथ्स॒ꣳ॒सृद्भि॒रनु॒\- सम॑सर्पत्।
तथ्स॒ꣳ॒सृपाꣳ॑ सꣳसृ॒त्त्वम्।
अ॒ग्निना॑ दे॒वेन॑ प्रथ॒मे\-ऽह॒न्ननु॒ प्रायु॑ङ्क्त।
सर॑स्वत्या वा॒चा द्वि॒तीये᳚।
स॒वि॒त्रा प्र॑स॒वेन॑ तृ॒तीये᳚।
पू॒ष्णा प॒शुभि॑श्चतु॒र्थे।
बृह॒स्पति॑ना॒ ब्रह्म॑णा पञ्च॒मे।
इन्द्रे॑ण दे॒वेन॑ ष॒ष्ठे।
वरु॑णेन॒ स्वया॑ दे॒वत॑या सप्त॒मे॥१॥

%1.8.1.2
सोमे॑न॒ राज्ञा᳚\-ऽष्ट॒मे।
त्वष्ट्रा॑ रू॒पेण॑ नव॒मे।
विष्णु॑ना य॒ज्ञेना᳚\-ऽ\-ऽ\-प्नोत्।
यथ्स॒ꣳ॒सृपो॒ भव॑न्ति।
इ॒न्द्रि॒यमे॒व तद्वी॒र्यं॑ यज॑मान आप्नोति।
पूर्वा॑पूर्वा॒ वेदि॑र्भवति।
इ॒न्द्रि॒यस्य॑ वी॒र्य॑स्या\-व॑\-रुद्ध्यै।
पु॒रस्ता॑\-दुप॒\-सदाꣳ॑ सौ॒म्येन॒ प्रच॑रति।
सोमो॒ वै रे॑तो॒धाः।
रेत॑ ए॒व तद्द॑धाति।
अ॒न्त॒रा त्वा॒ष्ट्रेण॑।
रेत॑ ए॒व हि॒तं त्वष्टा॑ रू॒पाणि॒ विक॑रोति।
उ॒परि॑ष्टाद्वैष्ण॒वेन॑।
य॒ज्ञो वै विष्णुः॑।
य॒ज्ञ ए॒वान्त॒तः प्रति॑ तिष्ठति॥२॥\anuvakamend[स॒प्त॒मे द॑धाति॒ पञ्च॑ च]

%1.8.2.1
जा॒मि वा ए॒तत्कु॑र्वन्ति।
यथ्स॒द्यो दी॒क्षय॑न्ति स॒द्यः सोमं॑ क्री॒णन्ति॑।
पु॒ण्ड॒रि॒स्र॒जां प्रय॑च्छ॒त्यजा॑मित्वाय।
अङ्गि॑रसः सुव॒र्गं लो॒कं यन्तः॑।
अ॒फ्सु दी᳚क्षात॒पसी॒ प्रावे॑शयन्।
तत्पु॒ण्डरी॑कमभवत्।
यत्पु॑ण्डरिस्र॒जां प्र॒यच्छ॑ति।
सा॒क्षादे॒व दी᳚क्षात॒पसी॒ अव॑ रुन्धे।
द॒शभि॑र्वथ्सत॒रैः सोमं॑ क्रीणाति।
दशा᳚क्षरा वि॒राट्॥३॥

%1.8.2.2
अन्नं॑ वि॒राट्।
वि॒राजै॒वान्नाद्य॒मव॑ रुन्धे।
मु॒ष्क॒रा भ॑वन्ति सेन्द्र॒त्वाय॑।
द॒श॒पेयो॑ भवति।
अ॒न्नाद्य॒स्या\-व॑\-रुद्ध्यै।
श॒तं ब्रा᳚ह्म॒णाः पि॑बन्ति।
श॒तायुः॒ पुरु॑षः श॒तेन्द्रि॑यः।
आयु॑ष्ये॒वेन्द्रि॒ये प्रति॑ तिष्ठति।
स॒प्त॒द॒शꣴ स्तो॒त्रं भ॑वति।
स॒प्त॒द॒शः प्र॒जा\-प॑तिः॥४॥

%1.8.2.3
प्र॒जा\-प॑ते॒राप्त्यै᳚।
प्रा॒का॒शाव॑ध्व॒र्यवे॑ ददाति।
प्र॒का॒शमे॒वैनं॑ गमयति।
स्रज॑मुद्गा॒त्रे।
व्ये॑वास्मै॑ वासयति।
रु॒क्मꣳ होत्रे᳚।
आ॒दि॒त्यमे॒वास्मा॒ उन्न॑यति।
अश्वं॑ प्रस्तोतृप्रतिह॒र्तृभ्या᳚म्।
प्रा॒जा॒प॒त्यो वा अश्वः॑।
प्र॒जा\-प॑ते॒राप्त्यै᳚॥५॥

%1.8.2.4
द्वाद॑श पष्ठौ॒हीर्ब्र॒ह्मणे᳚।
आयु॑रे॒वाव॑ रुन्धे।
व॒शां मै᳚त्रावरु॒णाय॑।
रा॒ष्ट्रमे॒व व॒श्य॑कः।
ऋ॒ष॒भं ब्रा᳚ह्मणाच्छ॒ꣳ॒सिने᳚।
रा॒ष्ट्रमे॒वेन्द्रि॑या॒\-व्य॑कः।
वास॑सी नेष्टापो॒तृभ्या᳚म्।
प॒वित्रे॑ ए॒वास्यै॒ते।
स्थूरि॑ यवाचि॒तम॑च्छावा॒काय॑।
अ॒न्त॒त ए॒व वरु॑ण॒मव॑ यजते॥६॥

%1.8.2.5
अ॒न॒ड्वाह॑म॒ग्नीधे᳚।
वह्नि॒र्वा अ॑न॒ड्वान्।
वह्नि॑र॒ग्नीत्।
वह्नि॑नै॒व वह्नि॑ य॒ज्ञस्याव॑ रुन्धे।
इन्द्र॑स्य सुषुवा॒णस्य॑ त्रे॒धेन्द्रि॒यं वी॒र्यं॑ परा॑\-ऽपतत्।
भृगु॒स्तृती॑यमभवत्।
श्रा॒य॒न्तीयं॒ तृती॑यम्।
सर॑स्वती॒ तृती॑यम्।
भा॒र्ग॒वो होता॑ भवति।
श्रा॒य॒न्तीयं॑ ब्रह्मसा॒मं भ॑वति।
वा॒र॒व॒न्तीय॑\-मग्नि\-ष्टोम\-सा॒मम्।
सा॒र॒स्व॒तीर॒पो गृ॑ह्णाति।
इ॒न्द्रि॒यस्य॑ वी॒र्य॑स्या\-व॑\-रुद्ध्यै।
श्रा॒य॒न्तीयं॑ ब्रह्मसा॒मं भ॑वति।
इ॒न्द्रि॒यमे॒वास्मि॑न्वी॒र्यꣴ॑ श्रयति।
वा॒र॒व॒न्तीय॑मग्निष्टोमसा॒मम्।
इ॒न्द्रि॒यमे॒वास्मि॑न्वी॒र्यं॑ वारयति॥७॥\anuvakamend[वि॒राट्प्र॒जा\-प॑ति॒रश्वः॑ प्र॒जा\-प॑ते॒राप्त्यै॑ यजते ब्रह्मसा॒मं भ॑वति स॒प्त च॑]

%1.8.3.1
ई॒श्व॒रो वा ए॒ष दिशो\-ऽनून्म॑दितोः।
यं दिशोऽनु॑ व्यास्था॒पय॑न्ति।
दि॒शामवे᳚ष्टयो भवन्ति।
दि॒क्ष्वे॑व प्रति॑ तिष्ठ॒त्यनु॑न्मादाय।
पञ्च॑ दे॒वता॑ यजति।
पञ्च॒ दिशः॑।
दि॒क्ष्वे॑व प्रति॑ तिष्ठति।
ह॒विषो॑हविष इ॒ष्ट्वा बा॑र्\mbox{}हस्प॒त्यम॒भिघा॑रयति।
य॒ज॒मा॒न॒दे॒व॒त्यो॑ वै बृह॒स्पतिः॑।
यज॑मानमे॒व तेज॑सा॒ सम॑र्धयति॥८॥

%1.8.3.2
आ॒दि॒त्यां म॒ल्॒हां ग॒र्भिणी॒मा ल॑भते।
मा॒रु॒तीं पृश्ञिं॑ पष्ठौ॒हीम्।
विशं॑ चै॒वास्मै॑ रा॒ष्ट्रं च॑ स॒मीची॑ दधाति।
आ॒दि॒त्यया॒ पूर्व॑या॒ प्रच॑रति।
मा॒रु॒त्योत्त॑रया।
रा॒ष्ट्र ए॒व विश॒मनु॑बध्नाति।
उ॒च्चैरा॑दि॒त्याया॒ आश्रा॑वयति।
उ॒पा॒ꣳ॒शु मा॑रु॒त्यै।
तस्मा᳚द्रा॒ष्ट्रं विश॒मति॑वदति।
ग॒र्भिण्या॑दि॒त्या भ॑वति॥९॥

%1.8.3.3
इ॒न्द्रि॒यं वै गर्भः॑।
रा॒ष्ट्रमे॒वेन्द्रि॑या॒व्य॑कः।
अ॒ग॒र्भा मा॑रु॒ती।
विड्वै म॒रुतः॑।
विश॑मे॒व निरि॑न्द्रियामकः।
दे॒वा॒सु॒राः संय॑त्ता आसन्।
ते दे॒वा अ॒श्विनोः᳚ पू॒षन्वा॒चः स॒त्यꣳ स॑न्नि॒धाय॑।
अनृ॑ते॒नासु॑रान॒भ्य॑भवन्।
ते᳚ऽश्विभ्यां᳚ पू॒ष्णे पु॑रो॒डाशं॒ द्वाद॑शकपालं॒ निर॑वपन्।
ततो॒ वै ते वा॒चः स॒त्यमवा॑रुन्धत॥१०॥

%1.8.3.4
यद॒श्विभ्यां᳚ पू॒ष्णे पु॑रो॒डाशं॒ द्वाद॑शकपालं नि॒र्वप॑ति।
अनृ॑तेनै॒व भ्रातृ॑व्यानभि॒भूय॑।
वा॒चः स॒त्यमव॑ रुन्धे।
सर॑स्वते सत्य॒वाचे॑ च॒रुम्।
पूर्व॑मे॒वोदि॒तम्।
उत्त॑रेणा॒भि गृ॑णाति।
स॒वि॒त्रे स॒त्यप्र॑सवाय पुरो॒डाशं॒ द्वाद॑शकपालं॒ प्रसू᳚त्यै।
दू॒तान्प्रहि॑णोति।
आ॒विद॑ ए॒ता भ॑वन्ति।
आ॒विद॑मे॒वैनं॑ गमयन्ति।
अथो॑ दू॒तेभ्य॑ ए॒व न छि॑द्यते।
ति॒सृ॒ध॒न्वꣳ शु॑ष्कदृ॒तिर्दक्षि॑णा॒ समृ॑द्ध्यै॥११॥\anuvakamend[अ॒र्ध॒य॒ति॒ भ॒व॒त्य॒रु॒न्ध॒त॒ ग॒म॒य॒न्ति॒ द्वे च॑]

%1.8.4.1
आ॒ग्ने॒यम॒ष्टा\-क॑पालं॒ निर्व॑पति।
तस्मा॒च्छिशि॑रे कुरुपञ्चा॒लाः प्राञ्चो॑ यान्ति।
सौ॒म्यं च॒रुम्।
तस्मा᳚द्वस॒न्तं व्य॑व॒साया॑दयन्ति।
सा॒वि॒त्रं द्वाद॑शकपालम्।
तस्मा᳚त्पु॒रस्ता॒द्यवा॑नाꣳ सवि॒त्रा विरु॑न्धते।
बा॒र्॒ह॒स्प॒त्यं च॒रुम्।
स॒वि॒त्रैव वि॒रुध्य॑।
ब्रह्म॑णा॒ यवा॒नाद॑धते।
त्वा॒ष्ट्रम॒ष्टा\-क॑पालम्॥१२॥

%1.8.4.2
रू॒पाण्ये॒व तेन॑ कुर्वते।
वै॒श्वा॒न॒रं द्वाद॑शकपालम्।
तस्मा᳚ज्जघ॒न्ये॑ नैदा॑घे प्र॒त्यञ्चः॑ कुरुपञ्चा॒ला या᳚न्ति।
सा॒र॒स्व॒तं च॒रुं निर्व॑पति।
तस्मा᳚त्प्रा॒वृषि॒ सर्वा॒ वाचो॑ वदन्ति।
पौ॒ष्णेन॒ व्यव॑स्यन्ति।
मै॒त्रेण॑ कृषन्ते।
वा॒रु॒णेन॒ विधृ॑ता आसते।
क्षै॒त्र॒प॒त्येन॑ पाचयन्ते।
आ॒दि॒त्येनाद॑धते॥१३॥

%1.8.4.3
मा॒सिमा᳚स्ये॒तानि॑ ह॒वीꣳषि॑ नि॒रुप्या॒णीत्या॑हुः।
तेनै॒वर्तून्प्रयु॑ङ्क्त॒ इति॑।
अथो॒ खल्वा॑हुः।
कः सं॑वथ्स॒रं जी॑विष्य॒तीति॑।
षडे॒व पू᳚र्वे॒द्युर्नि॒रुप्या॑णि।
षडु॑त्तरे॒द्युः।
तेनै॒वर्तून्प्रयु॑ङ्क्ते।
दक्षि॑णो रथवाहनवा॒हः पूर्वे॑षां॒ दक्षि॑णा।
उत्त॑र॒ उत्त॑रेषाम्।
सं॒व॒थ्स॒रस्यै॒वान्तौ॑ युनक्ति।
सु॒व॒र्गस्य॑ लो॒कस्य॒ सम॑ष्ट्यै॥१४॥\anuvakamend[त्वा॒ष्ट्रम॒ष्टा\-क॑पालं दधते युन॒क्त्येकं॑ च]

%1.8.5.1
इन्द्र॑स्य सुषुवा॒णस्य॑ दश॒धेन्द्रि॒यं वी॒र्यं॑ परा॑\-ऽपतत्।
स यत्प्र॑थ॒मं नि॒रष्ठी॑वत्।
तत्क्व॑लमभवत्।
यद्द्वि॒तीयम्᳚।
तद्बद॑रम्।
यत्तृ॒तीयम्᳚।
तत्क॒र्कन्धु॑।
यन्न॒स्तः।
स सि॒ꣳ॒हः।
यदक्ष्योः᳚॥१५॥

%1.8.5.2
स शा᳚र्दू॒लः।
यत्कर्ण॑योः।
स वृकः॑।
य ऊ॒र्ध्वः।
स सोमः॑।
याऽवा॑ची।
सा सुरा᳚।
त्र॒याः सक्त॑वो भवन्ति।
इ॒न्द्रि॒यस्या\-व॑\-रुद्ध्यै।
त्र॒याणि॒ लोमा॑नि॥१६॥

%1.8.5.3
त्विषि॑मे॒वाव॑ रुन्धे।
त्रयो॒ ग्रहाः᳚।
वी॒र्य॑मे॒वाव॑ रुन्धे।
नाम्ना॑ दश॒मी।
नव॒ वै पुरु॑षे प्रा॒णाः।
नाभि॑र्दश॒मी।
प्रा॒णा इ॑न्द्रि॒यं वी॒र्यम्᳚।
प्रा॒णाने॒वेन्द्रि॒यं वी॒र्यं॑ यज॑मान आ॒त्मन्ध॑त्ते।
सीसे॑न क्ली॒बाच्छष्पा॑णि क्रीणाति।
न वा ए॒तदयो॒ न हिर॑ण्यम्॥१७॥

%1.8.5.4
यथ्सीसम्᳚।
न स्त्री न पुमान्॑।
यत्क्ली॒बः।
न सोमो॒ न सुरा᳚।
यथ्सौ᳚त्राम॒णी समृ॑द्ध्यै।
स्वा॒द्वीं त्वा᳚ स्वा॒दुनेत्या॑ह।
सोम॑मे॒वैनां᳚ करोति।
सोमो᳚\-ऽस्य॒श्विभ्यां᳚ पच्यस्व॒ सर॑स्वत्यै पच्य॒स्वेन्द्रा॑य सु॒त्राम्णे॑ पच्य॒स्वेत्या॑ह।
ए॒ताभ्यो॒ ह्ये॑षा दे॒वता᳚भ्यः॒ पच्य॑ते।
ति॒स्रः सꣳसृ॑ष्टा वसति॥१८॥

%1.8.5.5
ति॒स्रो हि रात्रीः᳚ क्री॒तः सोमो॒ वस॑ति।
पु॒नातु॑ ते परि॒स्रुत॒मिति॒ यजु॑षा पुनाति॒ व्यावृ॑त्त्यै।
प॒वित्रे॑ण पुनाति।
प॒वित्रे॑ण॒ हि सोमं॑ पु॒नन्ति॑।
वारे॑ण॒ शश्व॑ता॒ तनेत्या॑ह।
वारे॑ण॒ हि सोमं॑ पु॒नन्ति॑।
वा॒युः पू॒तः प॒वित्रे॒णेति॒ नैतया॑ पुनीयात्।
व्यृ॑द्धा॒ ह्ये॑षा।
अ॒ति॒प॒वि॒तस्यै॒तया॑ पुनीयात्।
कु॒विद॒ङ्गेत्यनि॑रुक्तया प्राजाप॒त्यया॑ गृह्णाति॥१९॥

%1.8.5.6
अनि॑रुक्तः प्र॒जा\-प॑तिः।
प्र॒जा\-प॑ते॒राप्त्यै᳚।
एक॑य॒र्चा गृ॑ह्णाति।
ए॒क॒धैव यज॑माने वी॒र्यं॑ दधाति।
आ॒श्वि॒नं धू॒म्रमाल॑भते।
अ॒श्विनौ॒ वै दे॒वानां᳚ भि॒षजौ᳚।
ताभ्या॑मे॒वास्मै॑ भेष॒जं क॑रोति।
सा॒र॒स्व॒तं मे॒षम्।
वाग्वै सर॑स्वती।
वा॒चैवैनं॑ भिषज्यति।
ऐ॒न्द्रमृ॑ष॒भꣳ से᳚न्द्र॒त्वाय॑॥२०॥\anuvakamend[अक्ष्यो॒र्लोमा॑नि॒ हिर॑ण्यं वसति गृह्णाति भिषज्य॒त्येकं॑ च]

%1.8.6.1
यत्त्रि॒षु यूपे᳚ष्वा॒लभे॑त।
ब॒हि॒र्धा\-ऽस्मा॑दिन्द्रि॒यं वी॒र्यं॑ दध्यात्।
भ्रातृ॑व्यमस्मै जनयेत्।
ए॒क॒यू॒प आल॑भते।
ए॒क॒धैवास्मि॑न्निन्द्रि॒यं वी॒र्यं॑ दधाति।
नास्मै॒ भ्रातृ॑व्यं जनयति।
नैतेषां᳚ पशू॒नां पु॑रो॒डाशा॑ भवन्ति।
ग्रह॑पुरोडाशा॒ ह्ये॑ते।
यु॒वꣳ सु॒राम॑मश्वि॒नेति॑ सर्वदेव॒त्ये॑ याज्यानुवा॒क्ये॑ भवतः।
सर्वा॑ ए॒व दे॒वताः᳚ प्रीणाति॥२१॥

%1.8.6.2
ब्रा॒ह्म॒णं परि॑क्रीणीयादु॒च्छेष॑णस्य पा॒तारम्᳚।
ब्रा॒ह्म॒णो ह्याहु॑त्या उ॒च्छेष॑णस्य पा॒ता।
यदि॑ ब्राह्म॒णं न वि॒न्देत्।
व॒ल्मी॒क॒व॒पाया॒मव॑ नयेत्।
सैव ततः॒ प्राय॑श्चित्तिः।
यद्वै सौ᳚त्राम॒ण्यै व्यृ॑द्धम्।
तद॑स्यै॒ समृ॑द्धम्।
ना॒ना॒दे॒व॒त्याः᳚ प॒शव॑श्च पुरो॒डाशा᳚श्च भवन्ति॒ समृ॑द्ध्यै।
ऐ॒न्द्रः प॑शू॒नामु॑त्त॒मो भ॑वति।
ऐ॒न्द्रः पु॑रो॒डाशा॑नां प्रथ॒मः॥२२॥

%1.8.6.3
इ॒न्द्रि॒ये ए॒वास्मै॑ स॒मीची॑ दधाति।
पु॒रस्ता॑दनूया॒जानां᳚ पुरो॒डाशैः॒ प्रच॑रति।
प॒शवो॒ वै पु॑रो॒डाशाः᳚।
प॒शूने॒वाव॑ रुन्धे।
ऐ॒न्द्रमेका॑\-दश\-कपालं॒ निर्व॑पति।
इ॒न्द्रि॒यमे॒वाव॑ रुन्धे।
सा॒वि॒त्रं द्वाद॑शकपालं॒ प्रसू᳚त्यै।
वा॒रु॒णं दश॑कपालम्।
अ॒न्त॒त ए॒व वरु॑ण॒मव॑ यजते।
वड॑बा॒ दक्षि॑णा॥२३॥

%1.8.6.4
उ॒त वा ए॒षा\-ऽश्वꣳ॑ सू॒ते।
उ॒ताऽश्व॑त॒रम्।
उ॒त सोम॑ उ॒त सुरा᳚।
यथ्सौ᳚त्राम॒णी समृ॑द्ध्यै।
बा॒र्॒ह॒स्प॒त्यं प॒शुं च॑तु॒र्थम॑तिपवि॒तस्या ल॑भते।
ब्रह्म॒ वै दे॒वानां॒ बृह॒स्पतिः॑।
ब्रह्म॑णै॒व य॒ज्ञस्य॒ व्यृ॑द्ध॒मपि॑ वपति।
पु॒रो॒डाश॑वाने॒ष प॒शुर्भ॑वति।
न ह्ये॑तस्य॒ ग्रहं॑ गृ॒ह्णन्ति॑।
सोम॑प्रतीकाः पितरस्तृप्णु॒तेति॑ शतातृ॒ण्णायाꣳ॑ स॒मव॑नयति॥२४॥

%1.8.6.5
श॒तायुः॒ पुरु॑षः श॒तेन्द्रि॑यः।
आयु॑ष्ये॒वेन्द्रि॒ये प्रति॑ तिष्ठति।
दक्षि॑णे॒\-ऽग्नौ जु॑होति।
पा॒प॒व॒स्य॒सस्य॒ व्यावृ॑त्त्यै।
हिर॑ण्यमन्त॒रा धा॑रयति।
पू॒तामे॒वैनां᳚ जुहोति।
श॒तमा॑नं भवति।
श॒तायुः॒ पुरु॑षः श॒तेन्द्रि॑यः।
आयु॑ष्ये॒वेन्द्रि॒ये प्रति॑ तिष्ठति।
यत्रै॒व श॑तातृ॒ण्णां धा॒रय॑ति॥२५॥

%1.8.6.6
तन्निद॑धाति॒ प्रति॑\-ष्ठित्यै।
पि॒तॄन् वा ए॒तस्ये᳚न्द्रि॒यं वी॒र्यं॑ गच्छति।
यꣳ सोमो॑ऽति॒ पव॑ते।
पि॒तृ॒णां या᳚ज्यानुवा॒क्या॑भि॒रुप॑ तिष्ठते।
यदे॒वास्य॑ पि॒तॄनि॑न्द्रि॒यं वी॒र्यं॑ गच्छ॑ति।
तदे॒वाव॑ रुन्धे।
ति॒सृभि॒रुप॑ तिष्ठते।
तृ॒तीये॒ वा इ॒तो लो॒के पि॒तरः॑।
ताने॒व प्री॑णाति।
अथो॒ त्रीणि॒ वै य॒ज्ञस्ये᳚न्द्रि॒याणि॑।
अ॒ध्व॒र्युर्\mbox{}होता᳚ ब्र॒ह्मा।
त उप॑तिष्ठन्ते।
यान्ये॒व य॒ज्ञस्ये᳚न्द्रि॒याणि॑।
तैरे॒वास्मै॑ भेष॒जं क॑रोति॥२६॥\anuvakamend[प्री॒णा॒ति॒ प्र॒थ॒मो दक्षि॑णा स॒मव॑नयति धा॒रय॑तीन्द्रि॒याणि॑ च॒त्वारि॑ च]

%1.8.7.1
अ॒ग्नि॒ष्टो॒ममग्र॒ आह॑रति।
य॒ज्ञ॒\-मु॒खं वा अ॑ग्निष्टो॒मः।
य॒ज्ञ॒\-मु॒खमे॒वारभ्य॑ स॒वमा क्र॑मते।
अथै॒षो॑\-ऽभिषेच॒नीय॑श्चतु\-स्त्रि॒ꣳ॒शप॑वमानो भवति।
त्रय॑स्त्रिꣳश॒द्वै दे॒वताः᳚।
ता ए॒वाऽऽप्नो॑ति।
प्र॒जा\-प॑तिश्चतुस्त्रि॒ꣳ॒शः।
तमे॒वाऽऽप्नो॑ति।
स॒ꣳ॒श॒र ए॒ष स्तोमा॑ना॒मय॑था\-पूर्वम्।
यद्विष॑माः॒ स्तोमाः᳚॥२७॥

%1.8.7.2
ए॒तावा॒न् वै य॒ज्ञः।
यावा॒न्पव॑मानाः।
अ॒न्तः॒ श्लेष॑णं॒ त्वा अ॒न्यत्।
यथ्स॒माः पव॑मानाः।
तेनाऽसꣳ॑शरः।
तेन॑ यथापू॒र्वम्।
आ॒त्मनै॒वाग्नि॑ष्टो॒मेन॒र्ध्नोति॑।
आ॒त्मना॒ पुण्यो॑ भवति।
प्र॒जा वा उ॒क्थानि॑।
प॒शव॑ उ॒क्थानि॑।
यदु॒क्थ्यो॑ भव॒त्यनु॒ सन्त॑त्त्यै॥२८॥\anuvakamend[स्तोमाः᳚ प॒शव॑ उ॒क्थान्येकं॑ च]

%1.8.8.1
उप॑ त्वा जा॒मयो॒ गिर॒ इति॑ प्रति॒पद्भ॑वति।
वाग्वै वा॒युः।
वा॒च ए॒वैषो॑\-ऽभिषे॒कः।
सर्वा॑सामे॒व प्र॒जानाꣳ॑ सूयते।
सर्वा॑ एनं प्र॒जा राजेति॑ वदन्ति।
ए॒तमु॒ त्यन्दश॒ क्षिप॒ इत्या॑ह।
आ॒दि॒त्या वै प्र॒जाः।
प्र॒जाना॑मे॒वैतेन॑ सूयते।
यन्ति॒ वा ए॒ते य॑ज्ञमु॒खात्।
ये स॑म्भा॒र्या॑ अक्रन्॑॥२९॥

%1.8.8.2
यदाह॒ पव॑स्व वा॒चो अ॑ग्रिय॒ इति॑।
तेनै॒व य॑ज्ञमु॒खान्नय॑न्ति।
अ॒नु॒ष्टुक्प्र॑थ॒मा भ॑वति।
अ॒नु॒ष्टुगु॑त्त॒मा।
वाग्वा अ॑नु॒ष्टुक्।
वा॒चैव प्र॒यन्ति॑।
वा॒चोद्य॑न्ति।
उद्व॑तीर्भवन्ति।
उद्व॒द्वा अ॑नु॒ष्टुभो॑ रू॒पम्।
आनु॑ष्टुभो राज॒न्यः॑॥३०॥

%1.8.8.3
तस्मा॒दुद्व॑तीर्भवन्ति।
सौ॒र्य॑नु॒ष्टुगु॑त्त॒मा भ॑वति।
सु॒व॒र्गस्य॑ लो॒कस्य॒ सन्त॑त्यै।
यो वै स॒वादेति॑।
नैनꣳ॑ स॒व उप॑नमति।
यः साम॑भ्य॒ एति॑।
पापी॑यान्थ्सुषुवा॒णो भ॑वति।
ए॒तानि॒ खलु॒ वै सामा॑नि।
यत्पृ॒ष्ठानि॑।
यत्पृ॒ष्ठानि॒ भव॑न्ति॥३१॥

%1.8.8.4
तैरे॒व स॒वान्नैति॑।
यानि॑ देवरा॒जाना॒ꣳ॒ सामा॑नि।
तैर॒मुष्मिँ॑ल्लो॒क ऋ॑ध्नोति।
यानि॑ मनुष्यरा॒जाना॒ꣳ॒ सामा॑नि।
तैर॒स्मिँल्लो॒क ऋ॑ध्नोति।
उ॒भयो॑रे॒व लो॒कयोर्॑ ऋध्नोति।
दे॒व॒लो॒के च॑ मनुष्यलो॒के च॑।
ए॒क॒वि॒ꣳ॒शो॑\-ऽभिषेच॒नीय॑स्योत्त॒मो भ॑वति।
ए॒क॒वि॒ꣳ॒शः के॑शवप॒नीय॑स्य प्रथ॒मः।
स॒प्त॒द॒शो द॑श॒पेयः॑॥३२॥

%1.8.8.5
विड्वा ए॑कवि॒ꣳ॒शः।
रा॒ष्ट्रꣳ स॑प्तद॒शः।
विश॑ ए॒वैतन्म॑ध्य॒तो॑\-ऽभिषि॑च्यते।
तस्मा॒द्वा ए॒ष वि॒शां प्रि॒यः।
वि॒शो हि म॑ध्य॒तो॑\-ऽभिषि॒च्यते᳚।
यद्वा ए॑नम॒दो दिशोऽनु॑ व्यास्था॒पय॑न्ति।
तथ्सु॑व॒र्गं लो॒कम॒भ्या रो॑हति।
यदि॒मं लो॒कं न प्र॑त्यव॒रोहे᳚त्।
अ॒ति॒ज॒नं वे॒यात्।
उद्वा॑ माद्येत्।
यदे॒ष प्र॑ती॒चीनः॑ स्तोमो॒ भव॑ति।
इ॒ममे॒व तेन॑ लो॒कं प्र॒त्यव॑रोहति।
अथो॑ अ॒स्मिन्ने॒व लो॒के प्रति॑ तिष्ठ॒त्यनु॑न्मादाय॥३३॥\anuvakamend[अक्र॑न्राज॒न्यो॑ भव॑न्ति दश॒पेयो॑ माद्ये॒त्त्रीणि॑ च]

%1.8.9.1
इ॒यं वै र॑ज॒ता।
अ॒सौ हरि॑णी।
यद्रु॒क्मौ भव॑तः।
आ॒भ्यामे॒वैन॑मुभ॒यतः॒ परि॑ गृह्णाति।
वरु॑णस्य॒ वा अ॑भिषि॒च्यमा॑न॒स्या\-ऽऽपः॑।
इ॒न्द्रि॒यं वी॒र्यं॑ निर॑घ्नन्।
तथ्सु॒वर्ण॒ꣳ॒ हिर॑ण्यमभवत्।
यद्रु॒क्मम॑न्त॒र्द\-धा॑ति।
इ॒न्द्रि॒यस्य॑ वी॒र्य॑स्या\-नि॑र्घाताय।
श॒तमा॑नो भवति श॒तक्ष॑रः।
श॒तायुः॒ पुरु॑षः श॒तेन्द्रि॑यः।
आयु॑ष्ये॒वेन्द्रि॒ये प्रति॑ तिष्ठति।
आयु॒र्वै हिर॑ण्यम्।
आ॒यु॒ष्या॑ ए॒वैन॑म॒भ्यति॑ क्षरन्ति।
तेजो॒ वै हिर॑ण्यम्।
ते॒ज॒स्या॑ ए॒वैन॑म॒भ्यति॑ क्षरन्ति।
वर्चो॒ वै हिर॑ण्यम्।
व॒र्च॒स्या॑ ए॒वैन॑म॒भ्यति॑ क्षरन्ति॥३४॥\anuvakamend[श॒तक्ष॑रो॒\-ऽष्टौ च॑]

%1.8.10.1
अप्र॑तिष्ठितो॒ वा ए॒ष इत्या॑हुः।
यो रा॑ज॒सूये॑न॒ यज॑त॒ इति॑।
य॒दा वा ए॒ष ए॒तेन॑ द्विरा॒त्रेण॒ यज॑ते।
अथ॑ प्रति॒ष्ठा।
अथ॑ संवथ्स॒रमा᳚प्नोति।
याव॑न्ति संवथ्स॒रस्या॑होरा॒त्राणि॑।
ताव॑तीरे॒तस्य॑ स्तो॒त्रीयाः᳚।
अ॒हो॒रा॒त्रेष्वे॒व प्रति॑ तिष्ठति।
अ॒ग्नि॒ष्टो॒मः पूर्व॒मह॑र्भवति।
अ॒ति॒रा॒त्र उत्त॑रम्॥३५॥

%1.8.10.2
नानै॒वाहो॑रा॒त्रयोः॒ प्रति॑ तिष्ठति।
पौ॒र्ण॒मा॒स्यां पूर्व॒मह॑र्भवति।
व्य॑ष्टकाया॒मुत्त॑रम्।
नानै॒वार्ध॑मा॒सयोः॒ प्रति॑ तिष्ठति।
अ॒मा॒वा॒स्या॑यां॒ पूर्व॒मह॑र्भवति।
उद्दृ॑ष्ट॒ उत्त॑रम्।
नानै॒व मास॑योः॒ प्रति॑ तिष्ठति।
अथो॒ खलु॑।
ये ए॒व स॑मानप॒क्षे पु॑ण्या॒हे स्याता᳚म्।
तयोः᳚ का॒र्यं॑ प्रति॑\-ष्ठित्यै॥३६॥

%1.8.10.3
अ॒प॒श॒व्यो द्वि॑रा॒त्र इत्या॑हुः।
द्वे ह्ये॑ते छन्द॑सी।
गा॒य॒त्रं च॒ त्रैष्टु॑भं च।
जग॑तीम॒न्तर्य॑न्ति।
न तेन॒ जग॑ती कृ॒तेत्या॑हुः।
यदे॑नान्तृतीयसव॒ने कु॒र्वन्तीति॑।
य॒दा वा ए॒षा\-ऽहीन॒स्याह॒र्भज॑ते।
सा॒ह्नस्य॑ वा॒ सव॑नम्।
अथै॒व जग॑ती कृ॒ता।
अथ॑ पश॒व्यः॑।
व्यु॑ष्टि॒र्वा ए॒ष द्वि॑रा॒त्रः।
य ए॒वं वि॒द्वान्द्वि॑रा॒त्रेण॒ यज॑ते।
व्ये॑वास्मा॑ उच्छति।
अथो॒ तम॑ ए॒वाप॑ हते।
अ॒ग्नि॒ष्टो॒मम॑न्त॒त आ ह॑रति।
अ॒ग्निः सर्वा॑ दे॒वताः᳚।
दे॒वता᳚स्वे॒व प्रति॑ तिष्ठति॥३७॥\anuvakamend[उत्त॑रं॒ प्रति॑\-ष्ठित्यै पश॒व्यः॑ स॒प्त च॑]

\prashnaend{वरु॑णस्य जा॒मि वा ई᳚श्व॒र आ᳚ग्ने॒यमिन्द्र॑स्य॒ यत्त्रि॒ष्व॑ग्निष्टो॒ममुप॑ त्वे॒यं वै र॑ज॒ता\-ऽप्र॑तिष्ठितो॒ दश॑॥१०॥}{वरु॑णस्य॒ यद॒श्विभ्यां॒ यत्त्रि॒षु तस्मा॒दुद्व॑तीः स॒प्तत्रिꣳ॑शत्॥३७॥}{वरु॑णस्य॒ प्रति॑ तिष्ठति॥}{हरिः॑ ओम्॥}{इति श्रीकृष्णयजुर्वेदीयतैत्तिरीयब्राह्मणे प्रथमाष्टके अष्टमः प्रपाठकः समाप्तः॥}
\clearpage
%%% END ASHTAKAM

\chapt{अष्टकम् २}
\sect{प्रथमः प्रश्नः}
\setcounter{anuvakam}{0}
\dnsub{तैत्तिरीयब्राह्मणे द्वितीयाष्टके प्रथमः प्रपाठकः}

%2.1.1.1
अङ्गि॑रसो॒ वै स॒त्रमा॑सत।
तेषां॒ पृश्ञि॑र्घर्म॒धुगा॑सीत्।
सर्जी॒षेणा॑जीवत्।
ते᳚ऽब्रुवन्।
कस्मै॒ नु स॒त्रमा᳚स्महे।
ये᳚ऽस्या ओष॑धी॒र्न ज॒नया॑म॒ इति॑।
ते दि॒वो वृष्टि॑मसृजन्त।
याव॑न्तः स्तो॒का अ॒वाप॑द्यन्त।
ताव॑ती॒रोष॑धयो\-ऽजायन्त।
ता जा॒ताः पि॒तरो॑ वि॒षेणा॑लिम्पन्॥१॥

%2.1.1.2
तासां᳚ ज॒ग्ध्वा रुप्य॒न्त्यैत्।
ते᳚ऽब्रुवन्।
क इ॒दमि॒त्थम॑क॒रिति॑।
व॒यं भा॑ग॒धेय॑मि॒च्छमा॑ना॒ इति॑ पि॒तरो᳚\-ऽब्रुवन्।
किं वो॑ भाग॒धेय॒मिति॑।
अ॒ग्नि॒हो॒त्र ए॒व नोऽप्य॒स्त्वित्य॑ब्रुवन्।
तेभ्य॑ ए॒तद्भा॑ग॒धेयं॒ प्राय॑च्छन्।
यद्धु॒त्वा नि॒मार्ष्टि॑।
ततो॒ वै त ओष॑धीरस्वदयन्।
य ए॒वं वेद॑॥२॥

%2.1.1.3
स्वद॑न्ते\-ऽस्मा॒ ओष॑धयः।
ते व॒थ्समु॒पावा॑सृजन्।
इ॒दं नो॑ ह॒व्यं प्रदा॑प॒येति॑।
सो᳚ऽब्रवी॒द्वरं॑ वृणै।
दश॑ मा॒ रात्री᳚र्जा॒तं न दो॑हन्।
आ॒स॒ङ्ग॒वं मा॒त्रा स॒ह च॑रा॒णीति॑।
तस्मा᳚द्व॒थ्सं जा॒तं दश॒ रात्री॒र्न दु॑हन्ति।
आ॒स॒ङ्ग॒वं मा॒त्रा स॒ह च॑रति।
वारे॑वृत॒ꣴ॒ ह्य॑स्य।
तस्मा᳚द्व॒थ्सꣳ सꣳ॑सृष्टध॒यꣳ रु॒द्रो घातु॑कः।
अति॒ हि स॒न्धान्धय॑ति॥३॥\anuvakamend[अ॒लि॒म्प॒न्वेद॒ घातु॑क॒ एकं॑ च]

%2.1.2.1
प्र॒जा\-प॑तिर॒ग्निम॑\-सृजत।
तं प्र॒जा अन्व॑सृज्यन्त।
तम॑भा॒ग उपा᳚स्त।
सो᳚ऽस्य प्र॒जाभि॒रपा᳚क्रामत्।
तम॑व॒रुरु॑थ्समा॒नो\-ऽन्वै᳚त्।
तम॑व॒रुधं॒ नाश॑क्नोत्।
स तपो॑\-ऽतप्यत।
सो᳚ऽग्निरुपा॑रम॒ताता॑पि॒ वै स्य प्र॒जा\-प॑ति॒रिति॑।
स र॒राटा॒दुद॑मृष्ट॥४॥

%2.1.2.2
तद्\mbox{}घृ॒तम॑भवत्।
तस्मा॒द्यस्य॑ दक्षिण॒तः केशा॒ उन्मृ॑ष्टाः।
ताञ्ज्ये᳚ष्ठल॒क्ष्मी प्रा॑जाप॒त्येत्या॑हुः।
यद्र॒राटा॑दु॒दमृ॑ष्ट।
तस्मा᳚द्र॒राटे॒ केशा॒ न स॑न्ति।
तद॒ग्नौ प्रागृ॑ह्णात्।
तद्व्य॑चिकिथ्सत्।
जु॒हवा॒नी(३) मा हौ॒षा(३)मिति॑।
तद्वि॑चिकि॒थ्सायै॒ जन्म॑।
य ए॒वं वि॒द्वान् वि॑चि॒किथ्स॑ति॥५॥

%2.1.2.3
वसी॑य ए॒व चे॑तयते।
तं वाग॒भ्य॑वदज्जु॒हुधीति॑।
सो᳚ऽब्रवीत्।
कस्त्वम॒सीति॑।
स्वैव ते॒ वागित्य॑ब्रवीत्।
सो॑ऽजुहो॒थ्स्वाहेति॑।
तथ्स्वा॑हाका॒रस्य॒ जन्म॑।
य ए॒वꣴस्वा॑हाका॒रस्य॒ जन्म॒ वेद॑।
क॒रोति॑ स्वाहाका॒रेण॑ वी॒र्यम्᳚।
यस्यै॒वं वि॒दुषः॑ स्वाहाका॒रेण॒ जुह्व॑ति॥६॥

%2.1.2.4
भोगा॑यै॒वास्य॑ हु॒तं भ॑वति।
तस्या॒ आहु॑त्यै॒ पुरु॑षम\-सृजत।
द्वि॒तीय॑मजुहोत्।
सोऽश्व॑म\-सृजत।
तृ॒तीय॑मजुहोत्।
स गाम॑\-सृजत।
च॒तु॒र्थम॑जुहोत्।
सोऽवि॑म\-सृजत।
प॒ञ्च॒मम॑जुहोत्।
सो॑ऽजाम॑\-सृजत॥७॥

%2.1.2.5
सो᳚ऽग्निर॑बिभेत्।
आहु॑तीभि॒र्वै मा᳚\-ऽऽप्नो॒तीति॑।
स प्र॒जा\-प॑तिं॒ पुनः॒ प्रावि॑शत्।
तं प्र॒जा\-प॑तिरब्रवीत्।
जाय॒स्वेति॑।
सो᳚ऽब्रवीत्।
किं भा॑ग॒धेय॑म॒भि ज॑निष्य॒ इति॑।
तुभ्य॑मे॒वेदꣳ हू॑याता॒ इत्य॑ब्रवीत्।
स ए॒तद्भा॑ग॒धेय॑म॒भ्य॑जायत।
यद॑ग्निहो॒त्रम्॥८॥

%2.1.2.6
तस्मा॑दग्निहो॒त्रमु॑च्यते।
तद्धू॒यमा॑नमादि॒त्यो᳚\-ऽब्रवीत्।
मा हौ॑षीः।
उ॒भयो॒र्वै ना॑वे॒तदिति॑।
सो᳚ऽग्निर॑ब्रवीत्।
क॒थं नौ॑ होष्य॒न्तीति॑।
सा॒यमे॒व तुभ्यं॑ जु॒हवन्॑।
प्रा॒तर्मह्य॒मित्य॑ब्रवीत्।
तस्मा॑द॒ग्नये॑ सा॒यꣳ हू॑यते।
सूर्या॑य प्रा॒तः॥९॥

%2.1.2.7
आ॒ग्ने॒यी वै रात्रिः॑।
ऐ॒न्द्रमहः॑।
यदनु॑दिते॒ सूर्ये᳚ प्रा॒तर्जु॑हु॒यात्।
उ॒भय॑मे॒वाग्ने॒यꣴ स्या᳚त्।
उदि॑ते॒ सूर्ये᳚ प्रा॒तर्जु॑होति।
तथा॒ऽग्नये॑ सा॒यꣳ हू॑यते।
सूर्या॑य प्रा॒तः।
रात्रिं॒ वा अनु॑ प्र॒जाः प्र जा॑यन्ते।
अह्ना॒ प्रति॑ तिष्ठन्ति।
यथ्सा॒यं जु॒होति॑॥१०॥

%2.1.2.8
प्रैव तेन॑ जायते।
उदि॑ते॒ सूर्ये᳚ प्रा॒तर्जु॑होति।
प्रत्ये॒व तेन॑ तिष्ठति।
प्र॒जा\-प॑तिरकामयत॒ प्रजा॑ये॒येति॑।
स ए॒तद॑ग्निहो॒त्रं मि॑थु॒नम॑पश्यत्।
तदुदि॑ते॒ सूर्ये॑\-ऽजुहोत्।
यजु॑षा॒\-ऽन्यत्।
तू॒ष्णीम॒न्यत्।
ततो॒ वै स प्राजा॑यत।
यस्यै॒वं वि॒दुष॒ उदि॑ते॒ सूर्ये᳚\-ऽग्निहो॒त्रं जुह्व॑ति॥११॥

%2.1.2.9
प्रैव जा॑यते।
अथो॒ यथा॒ दिवा᳚ प्रजा॒नन्नेति॑।
ता॒दृगे॒व तत्।
अथो॒ खल्वा॑हुः।
यस्य॒ वै द्वौ पुण्यौ॑ गृ॒हे वस॑तः।
यस्तयो॑र॒न्यꣳ रा॒धय॑त्य॒न्यं न।
उ॒भौ वाव स तावृ॑च्छ॒तीति॑।
अ॒ग्निं वावा\-ऽऽदि॒त्यः सा॒यं प्र वि॑शति।
तस्मा॑द॒ग्निर्दू॒रान्नक्तं॑ ददृशे।
उ॒भे हि तेज॑सी स॒म्पद्ये॑ते॥१२॥

%2.1.2.10
उ॒द्यन्तं॒ वावा\-ऽऽदि॒त्यम॒ग्निरनु॑\- स॒मारो॑हति।
तस्मा᳚द्धू॒म ए॒वाग्नेर्दिवा॑ ददृशे।
यद॒ग्नये॑ सा॒यं जु॑हु॒यात्।
आ सूर्या॑य वृश्च्येत।
यथ्सूर्या॑य प्रा॒तर्जु॑हु॒यात्।
आऽग्नये॑ वृश्च्येत।
दे॒वता᳚भ्यः स॒मदं॑ दध्यात्।
अ॒ग्निर्ज्योति॒र्ज्योतिः॒ सूर्यः॒ स्वाहेत्ये॒व सा॒यꣳ हो॑त॒व्यम्᳚।
सूर्यो॒ ज्योति॒र्ज्योति॑र॒ग्निः स्वाहेति॑ प्रा॒तः।
तथो॒भाभ्याꣳ॑ सा॒यꣳ हू॑यते॥१३॥

%2.1.2.11
उ॒भाभ्यां᳚ प्रा॒तः।
न दे॒वता᳚भ्यः स॒मदं॑ दधाति।
अ॒ग्निर्ज्योति॒\-रित्या॑ह।
अ॒ग्निर्वै रे॑तो॒धाः।
प्र॒जा ज्योति॒रित्या॑ह।
प्र॒जा ए॒वास्मै॒ प्र ज॑नयति।
सूर्यो॒ ज्योति॒रित्या॑ह।
प्र॒जास्वे॒व प्रजा॑तासु॒ रेतो॑ दधाति।
ज्योति॑र॒ग्निः स्वाहेत्या॑ह।
प्र॒जा ए॒व प्रजा॑ता अ॒स्यां प्रति॑\-ष्ठापयति॥१४॥

%2.1.2.12
तू॒ष्णीमुत्त॑रा॒माहु॑तिं जुहोति।
मि॒थु॒न॒त्वाय॒ प्रजा᳚त्यै।
यदुदि॑ते॒ सूर्ये᳚ प्रा॒तर्जु॑हु॒यात्।
यथाऽति॑थये॒ प्रद्रु॑ताय शू॒न्याया॑वस॒थाया॑हा॒र्यꣳ॑ हर॑न्ति।
ता॒दृगे॒व तत्।
क्वाऽऽह॒ तत॒स्तद्भव॒तीत्या॑हुः।
यथ्स न वेद॑।
यस्मै॒ तद्धर॒न्तीति॑।
तस्मा॒द्यदौ॑ष॒सं जु॒होति॑।
तदे॒व स॑म्प्र॒ति।
अथो॒ यथा॒ प्रार्थ॑मौष॒सं प॑रि॒वेवे᳚ष्टि।
ता॒दृगे॒व तत्॥१५॥\anuvakamend[अ॒मृ॒ष्ट॒ वि॒चि॒किथ्स॑ति॒ जुह्व॑त्य॒जाम॑\-सृजताग्निहो॒त्रꣳ सूर्या॑य प्रा॒तर्जु॒होति॒ जुह्व॑ति स॒म्पद्ये॑ते हूयते स्थापयति सम्प्र॒ति द्वे च॑]

%2.1.3.1
रु॒द्रो वा ए॒षः।
यद॒ग्निः।
पत्नी᳚ स्था॒ली।
यन्मध्ये॒\-ऽग्नेर॑धि॒श्रये᳚त्।
रु॒द्राय॒ पत्नी॒मपि॑ दध्यात्।
प्र॒मायु॑का स्यात्।
उदी॒चो\-ऽङ्गा॑रान्नि॒रूह्याधि॑ श्रयति।
पत्नि॑यै गोपी॒थाय॑।
व्य॑न्तान्करोति।
तथा॒ पत्न्यप्र॑मायुका भवति॥१६॥

%2.1.3.2
घ॒र्मो वा ए॒षो\-ऽशा᳚न्तः।
अह॑रहः॒ प्र वृ॑ज्यते।
यद॑ग्निहो॒त्रम्।
प्रति॑\-षिञ्चेत्प॒शुका॑मस्य।
शा॒न्तमि॑व॒ हि प॑श॒व्यम्᳚।
न प्रति॑\-षिञ्चेद्ब्रह्मवर्च॒सका॑मस्य।
समि॑द्धमिव॒ हि ब्र॑ह्म\-वर्च॒सम्।
अथो॒ खलु॑।
प्र॒ति॒षिच्य॑मे॒व।
यत्प्र॑तिषि॒ञ्चति॑॥१७॥

%2.1.3.3
तत्प॑श॒व्यम्᳚।
यज्जु॒होति॑।
तद्ब्र॑ह्म\-वर्च॒सि।
उ॒भय॑मे॒वाकः॑।
प्रच्यु॑तं॒ वा ए॒तद॒स्माल्लो॒कात्।
अग॑तं देवलो॒कम्।
यच्छृ॒तꣳ ह॒विरन॑भिघारितम्।
अ॒भि द्यो॑तयति।
अ॒भ्ये॑वैन॑द्\mbox{}घारयति।
अथो॑ देव॒त्रैवैन॑द्गमयति॥१८॥

%2.1.3.4
पर्य॑ग्नि करोति।
रक्ष॑सा॒मप॑हत्यै।
त्रिः पर्य॑ग्नि करोति।
त्र्या॑वृ॒द्धि य॒ज्ञः।
अथो॑ मेध्य॒त्वाय॑।
यत्प्रा॒चीन॑मुद्वा॒सये᳚त्।
यज॑मानꣳ शु॒चा\-ऽर्प॑येत्।
यद्द॑क्षि॒णा।
पि॒तृ॒दे॒व॒त्यꣴ॑ स्यात्।
यत्प्र॒त्यक्॥१९॥

%2.1.3.5
पत्नीꣳ॑ शु॒चा\-ऽर्प॑येत्।
उ॒दी॒चीन॒मुद्वा॑सयति।
ए॒षा वै दे॑वमनु॒ष्याणाꣳ॑ शा॒न्ता दिक्।
तामे॒वैन॒दनूद्वा॑सयति॒ शान्त्यै᳚।
वर्त्म॑ करोति।
य॒ज्ञस्य॒ सन्त॑त्यै।
निष्ट॑पति।
उपै॒व तथ्स्तृ॑णाति।
च॒तुरुन्न॑यति।
चतु॑ष्पादः प॒शवः॑॥२०॥

%2.1.3.6
प॒शूने॒वाव॑ रुन्धे।
सर्वा᳚न्पू॒र्णानुन्न॑यति।
सर्वे॒ हि पुण्या॑ रा॒द्धाः।
अ॒नूच॒ उन्न॑यति।
प्र॒जाया॑ अनूचीन॒त्वाय॑।
अ॒नूच्ये॒वास्य॑ प्र॒जा\-ऽर्धु॑का भवति।
सम्मृ॑शति॒ व्यावृ॑त्त्यै।
नाहो᳚ष्य॒न्नुप॑ सादयेत्।
यदहो᳚ष्यन्नुपसा॒दये᳚त्।
यथा॒ऽन्यस्मा॑ उपनि॒धाय॑॥२१॥

%2.1.3.7
अ॒न्यस्मै᳚ प्र॒यच्छ॑ति।
ता॒दृगे॒व तत्।
आऽस्मै॑ वृश्च्येत।
यदे॒व गार्\mbox{}ह॑पत्येऽधि॒ श्रय॑ति।
तेन॒ गार्\mbox{}ह॑पत्यं प्रीणाति।
अ॒ग्निर॑बिभेत्।
आहु॑तयो॒ माऽत्ये᳚ष्य॒न्तीति॑।
स ए॒ताꣳ स॒मिध॑मपश्यत्।
तामाऽध॑त्त।
ततो॒ वा अ॒ग्नावाहु॑तयो\-ऽध्रियन्त॥२२॥

%2.1.3.8
यदे॑नꣳ स॒मय॑च्छत्।
तथ्स॒मिधः॑ समि॒त्त्वम्।
स॒मिध॒मा द॑धाति।
समे॒वैनं॑ यच्छति।
आहु॑तीनां॒ धृत्यै᳚।
अथो॑ अग्निहो॒त्रमे॒वेध्मव॑त्करोति।
आहु॑तीनां॒ प्रति॑\-ष्ठित्यै।
ब्र॒ह्म॒वा॒दिनो॑ वदन्ति।
यदेकाꣳ॑ स॒मिध॑मा॒धाय॒ द्वे आहु॑ती जु॒होति॑।
अथ॒ कस्याꣳ॑ स॒मिधि॑ द्वि॒तीया॒माहु॑तिं जुहो॒तीति॑॥२३॥

%2.1.3.9
यद्द्वे स॒मिधा॑वा द॒ध्यात्।
भ्रातृ॑व्यमस्मै जनयेत्।
एकाꣳ॑ स॒मिध॑मा॒धाय॑।
यजु॑षा॒\-ऽन्यामाहु॑तिं जुहोति।
उ॒भे ए॒व स॒मिद्व॑ती॒ आहु॑ती जुहोति।
नास्मै॒ भ्रातृ॑व्यं जनयति।
आदी᳚प्तायां जुहोति।
समि॑द्धमिव॒ हि ब्र॑ह्म\-वर्च॒सम्।
अथो॒ यथा\-ऽति॑थिं॒ ज्योति॑ष्कृ॒त्वा प॑रि॒ वेवे᳚ष्टि।
ता॒दृगे॒व तत्।
च॒तुरुन्न॑यति।
द्विर्जु॑होति।
तस्मा᳚द्द्वि॒पाच्चतु॑ष्पादमत्ति।
अथो᳚ द्वि॒पद्ये॒व चतु॑ष्पदः॒ प्रति॑\-ष्ठापयति॥२४॥\anuvakamend[भ॒व॒ति॒ प्र॒ति॒षि॒ञ्चति॑ गमयति प्र॒त्यक्प॒शव॑ उपनि॒धाया᳚ध्रिय॒न्तेति॒ तच्च॒त्वारि॑ च]

%2.1.4.1
उ॒त्त॒राव॑तीं॒ वै दे॒वा आहु॑ति॒मजु॑हवुः।
अवा॑ची॒मसु॑राः।
ततो॑ दे॒वा अभ॑वन्।
पराऽसु॑राः।
यं का॒मये॑त॒ वसी॑यान्थ्स्या॒दिति॑।
कनी॑य॒स्तस्य॒ पूर्वꣳ॑ हु॒त्वा।
उत्त॑रं॒ भूयो॑ जुहुयात्।
ए॒षा वा उ॑त्त॒राव॒त्याहु॑तिः।
तान्दे॒वा अ॑जुहवुः।
तत॒स्ते॑\-ऽभवन्॥२५॥

%2.1.4.2
यस्यै॒वं जुह्व॑ति।
भव॑त्ये॒व।
यं का॒मये॑त॒ पापी॑यान्थ्स्या॒दिति॑।
भूय॒स्तस्य॒ पूर्वꣳ॑ हु॒त्वा।
उत्त॑रं॒ कनी॑यो जुहुयात्।
ए॒षा वा अवा॒च्याहु॑तिः।
तामसु॑रा अजुहवुः।
तत॒स्ते परा॑\-ऽभवन्।
यस्यै॒वं जुह्व॑ति।
परै॒व भ॑वति॥२६॥

%2.1.4.3
हु॒त्वोप॑ सादय॒त्यजा॑मित्वाय।
अथो॒ व्यावृ॑त्त्यै।
गार्\mbox{}ह॑पत्यं॒ प्रती᳚क्षते।
अन॑नुध्यायिनमे॒वैनं॑ करोति।
अ॒ग्नि॒हो॒त्रस्य॒ वै स्था॒णुर॑स्ति।
तं य ऋ॒च्छेत्।
य॒ज्ञ॒स्था॒णुमृ॑च्छेत्।
ए॒ष वा अ॑ग्निहो॒त्रस्य॑ स्था॒णुः।
यत्पूर्वा\-ऽऽहु॑तिः।
तां यदुत्त॑रया॒ऽभि जु॑हु॒यात्॥२७॥

%2.1.4.4
य॒ज्ञ॒स्था॒णुमृ॑च्छेत्।
अ॒ति॒हाय॒ पूर्वा॒माहु॑तिं जुहोति।
य॒ज्ञ॒स्था॒णुमे॒व परि॑ वृणक्ति।
अथो॒ भ्रातृ॑व्यमे॒वाप्त्वाऽति॑ क्रामति।
अ॒वा॒चीनꣳ॑ सा॒यमुप॑मार्ष्टि।
रेत॑ ए॒व तद्द॑धाति।
ऊ॒र्ध्वं प्रा॒तः।
प्र ज॑नयत्ये॒व तत्।
ब्र॒ह्म॒वा॒दिनो॑ वदन्ति।
च॒तुरुन्न॑यति॥२८॥

%2.1.4.5
द्विर्जु॑होति।
अथ॒ क्व॑ द्वे आहु॑ती भवत॒ इति॑।
अ॒ग्नौ वै᳚श्वान॒र इति॑ ब्रूयात्।
ए॒ष वा अ॒ग्निर्वै᳚श्वान॒रः।
यद्ब्रा᳚ह्म॒णः।
हु॒त्वा द्विः प्राश्ञा॑ति।
अ॒ग्नावे॒व वै᳚श्वान॒रे द्वे आहु॑ती जुहोति।
द्विर्जु॒होति॑।
द्विर्निमा᳚र्ष्टि।
द्विः प्राश्ञा॑ति॥२९॥

%2.1.4.6
षट्थ्सम्प॑द्यन्ते।
षड्वा ऋ॒तवः॑।
ऋ॒तूने॒व प्री॑णाति।
ब्र॒ह्म॒वा॒दिनो॑ वदन्ति।
किं॒ दे॒व॒त्य॑मग्निहो॒त्रमिति॑।
वै॒श्व॒दे॒वमिति॑ ब्रूयात्।
यद्यजु॑षा जु॒होति॑।
तदै᳚न्द्रा॒ग्नम्।
यत्तू॒ष्णीम्।
तत्प्रा॑जाप॒त्यम्॥३०॥

%2.1.4.7
यन्नि॒मार्ष्टि॑।
तदोष॑धीनाम्।
यद्द्वि॒तीयम्᳚।
तत्पि॑तृ॒णाम्।
यत्प्राश्ञा॑ति।
तद्गर्भा॑णाम्।
तस्मा॒द्गर्भा॒ अन॑श्ञन्तो वर्धन्ते।
यदा॒चाम॑ति।
तन्म॑नु॒ष्या॑णाम्।
उद॑ङ्पर्या॒वृत्याचा॑मति॥३१॥

%2.1.4.8
आ॒त्मनो॑ गोपी॒थाय॑।
निर्णे॑नेक्ति॒ शुद्ध्यै᳚।
निष्ट॑पति स्व॒गाकृ॑त्यै।
उद्दि॑शति।
स॒प्त॒र्॒षीने॒व प्री॑णाति।
द॒क्षि॒णा प॒र्याव॑र्तते।
स्वमे॒व वी॒र्य॑मनु॑ प॒र्याव॑र्तते।
तस्मा॒द्दक्षि॒णो\-ऽर्ध॑ आ॒त्मनो॑ वी॒र्या॑वत्तरः।
अथो॑ आदि॒त्यस्यै॒वावृत॒मनु॑ प॒र्याव॑र्तते।
हु॒त्वोप॒ समि॑न्धे॥३२॥

%2.1.4.9
ब्र॒ह्म॒व॒र्च॒सस्य॒ समि॑द्ध्यै।
न ब॒र्॒हिरनु॒\- प्र ह॑रेत्।
असꣴ॑स्थितो॒ वा ए॒ष य॒ज्ञः।
यद॑ग्निहो॒त्रम्।
यद॑नु प्र॒हरे᳚त्।
य॒ज्ञं वि\-च्छि॑न्द्यात्।
तस्मा॒न्नानु॑ प्र॒हृत्यम्᳚।
य॒ज्ञस्य॒ सन्त॑त्यै।
अ॒पो नि न॑यति।
अ॒व॒भृ॒थस्यै॒व रू॒पम॑कः॥३३॥\anuvakamend[अ॒भ॒व॒न्भ॒व॒ति॒ जु॒हु॒यान्न॑यति मार्ष्टि॒ द्विः प्राश्ञा॑ति प्राजाप॒त्यमाचा॑मतीन्धे\-ऽकः]

%2.1.5.1
ब्र॒ह्म॒वा॒दिनो॑ वदन्ति।
अ॒ग्नि॒हो॒त्रप्रा॑यणा य॒ज्ञाः।
किं प्रा॑यणमग्निहो॒त्रमिति॑।
व॒थ्सो वा अ॑ग्निहो॒त्रस्य॒ प्राय॑णम्।
अ॒ग्नि॒हो॒त्रं य॒ज्ञाना᳚म्।
तस्य॑ पृथि॒वी सदः॑।
अ॒न्तरि॑क्ष॒माग्नी᳚द्ध्रम्।
द्यौर्\mbox{}ह॑वि॒र्धानम्᳚।
दि॒व्या आपः॒ प्रोक्ष॑णयः।
ओष॑धयो ब॒र्॒हिः॥३४॥

%2.1.5.2
वन॒स्पत॑य इ॒ध्मः।
दिशः॑ परि॒धयः॑।
आ॒दि॒त्यो यूपः॑।
यज॑मानः प॒शुः।
स॒मु॒द्रो॑\-ऽवभृ॒थः।
सं॒व॒थ्स॒रः स्व॑गाका॒रः।
तस्मा॒दाहि॑ताग्नेः॒ सर्व॑मे॒व ब॑र्हि॒ष्यं॑ द॒त्तं भ॑वति।
यथ्सा॒यं जु॒होति॑।
रात्रि॑मे॒व तेन॑ दक्षि॒ण्यां᳚ कुरुते।
यत्प्रा॒तः॥३५॥

%2.1.5.3
अह॑रे॒व तेन॑ दक्षि॒ण्यं॑ कुरुते।
यत्ततो॒ ददा॑ति।
सा दक्षि॑णा।
याव॑न्तो॒ वै दे॒वा अहु॑त॒मादन्॑।
ते परा॑\-ऽभवन्।
त ए॒तद॑ग्निहो॒त्रꣳ सर्व॑स्यै॒व स॑मव॒दाया॑जुहवुः।
तस्मा॑दाहुः।
अ॒ग्नि॒हो॒त्रं वै दे॒वा गृ॒हाणां॒ निष्कृ॑तिमपश्य॒न्निति॑।
यथ्सा॒यं जु॒होति॑।
रात्रि॑या ए॒व तद्धु॒ताद्या॑य॥३६॥

%2.1.5.4
यज॑मान॒स्याप॑रा\-भावाय।
यत्प्रा॒तः।
अह्न॑ ए॒व तद्धु॒ताद्या॑य।
यज॑मान॒स्याप॑राभावाय।
यत्ततो॒\-ऽश्ञाति॑।
हु॒तमे॒व तत्।
द्वयोः॒ पय॑सा जुहुयात्प॒शुका॑मस्य।
ए॒तद्वा अ॑ग्निहो॒त्रं मि॑थु॒नम्।
य ए॒वं वेद॑।
प्र प्र॒जया॑ प॒शुभि॑र्मिथु॒नैर्जा॑यते॥३७॥

%2.1.5.5
इ॒मामे॒व पूर्व॑या दु॒हे।
अ॒मूमुत्त॑रया।
अ॒धि॒श्रित्योत्त॑र॒मा न॑यति।
योना॑वे॒व तद्रेतः॑ सिञ्चति प्र॒जन॑ने।
आज्ये॑न जुहुया॒त्तेज॑स्कामस्य।
तेजो॒ वा आज्यम्᳚।
ते॒ज॒स्व्ये॑व भ॑वति।
पय॑सा प॒शुका॑मस्य।
ए॒तद्वै प॑शू॒नाꣳ रू॒पम्।
रू॒पेणै॒वास्मै॑ प॒शूनव॑ रुन्धे॥३८॥

%2.1.5.6
प॒शु॒माने॒व भ॑वति।
द॒ध्नेन्द्रि॒यका॑मस्य।
इ॒न्द्रि॒यं वै दधि॑।
इ॒न्द्रि॒या॒व्ये॑व भ॑वति।
य॒वा॒ग्वा᳚ ग्राम॑कामस्यौष॒धा वै म॑नु॒ष्याः᳚।
भा॒ग॒धेये॑नै॒वास्मै॑ सजा॒तानव॑ रुन्धे।
ग्रा॒म्ये॑व भ॑वति।
अय॑ज्ञो॒ वा ए॒षः।
यो॑ऽसा॒मा॥३९॥

%2.1.5.7
च॒तुरुन्न॑यति।
चतु॑रक्षरꣳ रथन्त॒रम्।
र॒थ॒न्त॒रस्यै॒ष वर्णः॑।
उ॒परी॑व हरति।
अ॒न्तरि॑क्षं वामदे॒व्यम्।
वा॒म॒दे॒व्यस्यै॒ष वर्णः॑।
द्विर्जु॑होति।
द्व्य॑क्षरं बृ॒हत्।
बृ॒ह॒त ए॒ष वर्णः॑।
अ॒ग्नि॒हो॒त्रमे॒व तथ्साम॑न्वत्करोति॥४०॥

%2.1.5.8
यो वा अ॑ग्निहो॒त्रस्यो॑प॒सदो॒ वेद॑।
उपै॑नमुप॒सदो॑ नमन्ति।
वि॒न्दत॑ उपस॒त्तारम्᳚।
उ॒न्नीयोप॑ सादयति।
पृ॒थि॒वीमे॒व प्री॑णाति।
हो॒ष्यन्नुप॑सादयति।
अ॒न्तरि॑क्षमे॒व प्री॑णाति।
हु॒त्वोप॑ सादयति।
दिव॑मे॒व प्री॑णाति।
ए॒ता वा अ॑ग्निहो॒त्रस्यो॑प॒सदः॑॥४१॥

%2.1.5.9
य ए॒वं वेद॑।
उपै॑नमुप॒सदो॑ नमन्ति।
वि॒न्दत॑ उपस॒त्तारम्᳚।
यो वा अ॑ग्निहो॒त्रस्याश्रा॑वितं प्र॒त्याश्रा॑वित॒ꣳ॒ होता॑रं ब्र॒ह्माणं॑ वषट्का॒रं वेद॑।
तस्य॒ त्वे॑व हु॒तम्।
प्रा॒णो वा अ॑ग्निहो॒त्रस्याश्रा॑वितम्।
अ॒पा॒नः प्र॒त्याश्रा॑वितम्।
मनो॒ होता᳚।
चक्षु॑र्ब्र॒ह्मा।
नि॒मे॒षो व॑षट्का॒रः॥४२॥

%2.1.5.10
य ए॒वं वेद॑।
तस्य॒ त्वे॑व हु॒तम्।
सा॒यं॒ यावा॑नश्च॒ वै दे॒वाः प्रा॑त॒र्यावा॑णश्चाग्निहो॒त्रिणो॑ गृ॒हमाग॑च्छन्ति।
तान् यन्न त॒र्पये᳚त्।
प्र॒जया᳚ऽस्य प॒शुभि॒र्वि ति॑ष्ठेरन्।
यत्त॒र्पये᳚त्।
तृ॒प्ता ए॑नं प्र॒जया॑ प॒शुभि॑स्तर्पयेयुः।
स॒जूर्दे॒वैः सा॒यं याव॑भि॒रिति॑ सा॒यꣳ सम्मृ॑शति।
स॒जूर्दे॒वैः प्रा॒तर्याव॑भि॒रिति॑ प्रा॒तः।
ये चै॒व दे॒वाः सा॑यं॒ यावा॑नो॒ ये च॑ प्रात॒र्यावा॑णः॥४३॥

%2.1.5.11
ताने॒वोभयाꣴ॑स्तर्पयति।
त ए॑नं तृ॒प्ताः प्र॒जया॑ प॒शुभि॑स्तर्प\-यन्ति।
अ॒रु॒णो ह॑ स्मा॒हौप॑वेशिः।
अ॒ग्नि॒हो॒त्र ए॒वाहꣳ सा॒यं प्रा॑त॒र्वज्रं॒ भ्रातृ॑व्येभ्यः॒ प्र ह॑रामि।
तस्मा॒न्मत्पापी॑याꣳसो॒ भ्रातृ॑व्या॒ इति॑।
च॒तुरुन्न॑यति।
द्विर्जु॑होति।
स॒मिथ्स॑प्त॒मी।
स॒प्तप॑दा॒ शक्व॑री।
शा॒क्व॒रो वज्रः॑।
अ॒ग्नि॒हो॒त्र ए॒व तथ्सा॒यं प्रा॑त॒र्वज्रं॒ यज॑मानो॒ भ्रातृ॑व्याय॒ प्र ह॑रति।
भव॑त्या॒त्मना᳚।
परा᳚ऽस्य॒ भ्रातृ॑व्यो भवति॥४४॥\anuvakamend[ब॒र्॒हिः प्रा॒तर्\mbox{}हु॒ताद्या॑य जायते रुन्धे\-ऽसा॒मा क॑रोत्ये॒ता वा अ॑ग्निहो॒त्रस्यो॑प॒सदो॑ वषट्का॒रश्च॑ प्रात॒र्यावा॑णो॒ वज्र॒स्त्रीणि॑ च]

%2.1.6.1
प्र॒जा\-प॑तिरकामयता\-ऽऽत्म॒न्वन्मे॑ जाये॒तेति॑।
सो॑ऽजुहोत्।
तस्या᳚\-ऽऽत्म॒न्वद॑जायत।
अ॒ग्निर्वा॒युरा॑दि॒त्यः।
ते᳚ऽब्रुवन्।
प्र॒जा\-प॑तिर\-हौषी\-दात्म॒न्वन्मे॑ जाये॒तेति॑।
तस्य॑ व॒यम॑जनिष्महि।
जाय॑तान्न आत्म॒न्वदिति॒ ते॑\-ऽजुहवुः।
प्रा॒णाना॑म॒ग्निः।
त॒नुवै॑ वा॒युः॥४५॥

%2.1.6.2
चक्षु॑ष आदि॒त्यः।
तेषाꣳ॑ हु॒ताद॑जायत॒ गौरे॒व।
तस्यै॒ पय॑सि॒ व्याय॑च्छन्त।
मम॑ हु॒ताद॑जनि॒ ममेति॑।
ते प्र॒जा\-प॑तिं प्र॒श्ञमा॑यन्।
स आ॑दि॒त्यो᳚\-ऽग्निम॑ब्रवीत्।
य॒त॒रो नौ॒ जया᳚त्।
तन्नौ॑ स॒हास॒दिति॑।
कस्यै कोऽहौ॑षी॒दिति॑ प्र॒जा\-प॑तिरब्रवी॒त्कस्यै क॒ इति॑।
प्रा॒णाना॑म॒हमित्य॒ग्निः॥४६॥

%2.1.6.3
त॒नुवा॑ अ॒हमिति॑ वा॒युः।
चक्षु॑षो॒\-ऽहमित्या॑दि॒त्यः।
य ए॒व प्रा॒णाना॒महौ॑षीत्।
तस्य॑ हु॒ताद॑ज॒नीति॑।
अ॒ग्नेर्\mbox{}हु॒ताद॑ज॒नीति॑।
तद॑ग्निहो॒त्रस्या᳚ग्निहोत्र॒त्वम्।
गौर्वा अ॑ग्निहो॒त्रम्।
य ए॒वं वेद॒ गौर॑ग्निहो॒त्रमिति॑।
प्रा॒णा॒पा॒नाभ्या॑मे॒वाग्निꣳ सम॑र्धयति।
अव्य॑र्धुकः प्राणापा॒नाभ्यां᳚ भवति॥४७॥

%2.1.6.4
य ए॒वं वेद॑।
तौ वा॒युर॑ब्रवीत्।
अनु॒ मा भ॑जत॒मिति॑।
यदे॒व गार्\mbox{}ह॑पत्ये\-ऽधि॒श्रित्या॑हव॒नीय॑म॒भ्यु॑द्द्रवान्॑।
तेन॒ त्वां प्री॑णा॒नित्य॑ब्रूताम्।
तस्मा॒द्यद्गार्\mbox{}ह॑पत्ये\-ऽधि॒श्रित्या॑हव॒नीय॑\-म॒भ्यु॑द्द्रव॑ति।
वा॒युमे॒व तेन॑ प्रीणाति।
प्र॒जा\-प॑तिर्दे॒वताः᳚ सृ॒जमा॑नः।
अ॒ग्निमे॒व दे॒वता॑नां प्रथ॒मम॑\-सृजत।
सो᳚ऽन्यदा॑\-ल॒म्भ्य॑मवि॑त्वा॥४८॥

%2.1.6.5
प्र॒जा\-प॑तिम॒भि प॒र्याव॑र्तत।
स मृ॒त्योर॑बिभेत्।
सो॑ऽमुमा॑दि॒त्य\-मा॒त्मनो॒ निर॑मिमीत।
तꣳ हु॒त्वा परा᳚ङ्प॒र्याव॑र्तत।
ततो॒ वै स मृ॒त्युमपा॑जयत्।
अप॑ मृ॒त्युं ज॑यति।
य ए॒वं वेद॑।
तस्मा॒द्यस्यै॒वं वि॒दुषः॑।
उ॒तैका॒हमु॒त द्व्य॒हं न जुह्व॑ति।
हु॒तमे॒वास्य॑ भवति।
अ॒सौ ह्या॑दि॒त्यो᳚\-ऽग्निहो॒त्रम्॥४९॥\anuvakamend[त॒नुवै॑ वा॒युर॒ग्निर्भ॑व॒त्यवि॑त्वा भव॒त्येकं॑ च]

%2.1.7.1
रौ॒द्रं गवि॑।
वा॒य॒व्य॑मुप॑सृष्टम्।
आ॒श्वि॒नं दु॒ह्यमा॑नम्।
सौ॒म्यं दु॒ग्धम्।
वा॒रु॒णमधि॑ श्रितम्।
वै॒श्व॒दे॒वा भि॒न्दवः॑।
पौ॒ष्णमुद॑न्तम्।
सा॒र॒स्व॒तं वि॒ष्यन्द॑मानम्।
मै॒त्रꣳ शरः॑।
धा॒तुरुद्वा॑सितम्।
बृह॒स्पते॒रुन्नी॑तम्।
स॒वि॒तुः प्र क्रा᳚न्तम्।
द्या॒वा॒पृ॒थि॒व्यꣴ॑ ह्रि॒यमा॑णम्।
ऐ॒न्द्रा॒ग्नमुप॑सन्नम्।
अ॒ग्नेः पूर्वा\-ऽऽहु॑तिः।
प्र॒जा\-प॑ते॒रुत्त॑रा।
ऐ॒न्द्रꣳ हु॒तम्॥५०॥\anuvakamend[उद्वा॑सितꣳ स॒प्त च॑]

%2.1.8.1
द॒क्षि॒ण॒त उप॑ \-सृजति।
पि॒तृ॒लो॒कमे॒व तेन॑ जयति।
प्राची॒मा व॑र्तयति।
दे॒व॒लो॒कमे॒व तेन॑ जयति।
उदी॑चीमा॒वृत्य॑ दोग्धि।
म॒नु॒ष्य॒लो॒कमे॒व तेन॑ जयति।
पूर्वौ॑ दुह्याज्ज्ये॒ष्ठस्य॑ ज्यैष्ठिने॒यस्य॑।
यो वा॑ ग॒तश्रीः॒ स्यात्।
अप॑रौ दुह्यात्कनि॒ष्ठस्य॑ कानिष्ठिने॒यस्य॑।
यो वा॒ बुभू॑षेत्॥५१॥

%2.1.8.2
न सं मृ॑शति।
पा॒प॒व॒स्य॒सस्य॒ व्यावृ॑त्त्यै।
वा॒य॒व्यं॑ वा ए॒तदुप॑सृष्टम्।
आ॒श्वि॒नं दु॒ह्यमा॑नम्।
मै॒त्रं दु॒ग्धम्।
अ॒र्य॒म्ण उ॑द्वा॒स्यमा॑नम्।
त्वा॒ष्ट्रमु॑न्नी॒यमा॑नम्।
बृह॒स्पते॒रुन्नी॑तम्।
स॒वि॒तुः प्रक्रा᳚न्तम्।
द्या॒वा॒पृ॒थि॒व्यꣴ॑ ह्रि॒यमा॑णम्॥५२॥

%2.1.8.3
ऐ॒न्द्रा॒ग्नमुप॑ सादितम्।
सर्वा᳚भ्यो॒ वा ए॒ष दे॒वता᳚भ्यो जुहोति।
यो᳚ऽग्निहो॒त्रं जु॒होति॑।
यथा॒ खलु॒ वै धे॒नुं ती॒र्थे त॒र्पय॑ति।
ए॒वम॑ग्निहो॒त्री यज॑मानं तर्पयति।
तृप्य॑ति प्र॒जया॑ प॒शुभिः॑।
प्र सु॑व॒र्गं लो॒कं जा॑नाति।
पश्य॑ति पु॒त्रम्।
पश्य॑ति॒ पौत्रम्᳚।
प्र प्र॒जया॑ प॒शुभि॑र्मिथु॒नैर्जा॑यते।
यस्यै॒वं वि॒दुषो᳚\-ऽग्निहो॒त्रं जुह्व॑ति।
य उ॑ चैनदे॒वं वेद॑॥५३॥\anuvakamend[बुभू॑षेद्ध्रि॒यमा॑णञ्जायते॒ द्वे च॑]

%2.1.9.1
त्रयो॒ वै प्रै॑यमे॒धा आ॑सन्।
तेषां॒ त्रिरेको᳚\-ऽग्निहो॒त्रम॑जुहोत्।
द्विरेकः॑।
स॒कृदेकः॑।
तेषां॒ यस्त्रिरजु॑होत्।
स ऋ॒चा\-ऽजु॑होत्।
यो द्विः।
स यजु॑षा।
यः स॒कृत्।
स तू॒ष्णीम्॥५४॥

%2.1.9.2
यश्च॒ यजु॒षा\-ऽजु॑हो॒द्यश्च॑ तू॒ष्णीम्।
तावु॒भावा᳚र्ध्नुताम्।
तस्मा॒द्यजु॒षा\-ऽऽहु॑तिः॒ पूर्वा॑ होत॒व्या᳚।
तू॒ष्णीमुत्त॑रा।
उ॒भे ए॒वर्धी अव॑ रुन्धे।
अ॒ग्निर्ज्योति॒र्ज्योति॑र॒ग्निः स्वाहेति॑ सा॒यं जु॑होति।
रेत॑ ए॒व तद्द॑धाति।
सूर्यो॒ ज्योति॒र्ज्योतिः॒ सूर्यः॒ स्वाहेति॑ प्रा॒तः।
रेत॑ ए॒व हि॒तं प्र ज॑नयति।
रेतो॒ वा ए॒तस्य॑ हि॒तं न प्र जा॑यते॥५५॥

%2.1.9.3
यस्या᳚ग्निहो॒त्रमहु॑त॒ꣳ॒ सूर्यो॒\-ऽभ्यु॑देति॑।
यद्यन्ते॒ स्यात्।
उ॒न्नीय॒ प्राङु॒दाद्र॑वेत्।
स उ॑प॒साद्यातमि॑तोरासीत।
स य॒दा ताम्ये᳚त्।
अथ॒ भूः स्वाहेति॑ जुहुयात्।
प्र॒जा\-प॑ति॒र्वै भू॒तः।
तमे॒वोपा॑सरत्।
स ए॒वैनं॒ तत॒ उन्न॑यति।
नार्ति॒मार्च्छ॑ति॒ यज॑मानः॥५६॥\anuvakamend[तू॒ष्णीं जा॑यते॒ यज॑मानः]

%2.1.10.1
यद॒ग्निमु॒द्धर॑ति।
वस॑व॒स्तर्ह्य॒ग्निः।
तस्मि॒न्॒ यस्य॒ तथा॑विधे॒ जुह्व॑ति।
वसु॑ष्वे॒वास्या᳚ग्निहो॒त्रꣳ हु॒तं भ॑वति।
निहि॑तो धूपा॒यञ्छे॑ते।
रु॒द्रास्तर्ह्य॒ग्निः।
तस्मि॒न्॒ यस्य॒ तथा॑विधे॒ जुह्व॑ति।
रु॒द्रेष्वे॒वास्या᳚ग्निहो॒त्रꣳ हु॒तं भ॑वति।
प्र॒थ॒ममि॒ध्मम॒र्चिरा ल॑भते।
आ॒दि॒त्यास्तर्ह्य॒ग्निः॥५७॥

%2.1.10.2
तस्मि॒न्॒ यस्य॒ तथा॑विधे॒ जुह्व॑ति।
आ॒दि॒त्येष्वे॒वास्या᳚ग्निहो॒त्रꣳ हु॒तं भ॑वति।
सर्व॑ ए॒व स॑र्व॒श इ॒ध्म आदी᳚प्तो भवति।
विश्वे॑ दे॒वास्तर्ह्य॒ग्निः।
तस्मि॒न्॒ यस्य॒ तथा॑विधे॒ जुह्व॑ति।
विश्वे᳚ष्वे॒वास्य॑ दे॒वेष्व॑ग्निहो॒त्रꣳ हु॒तं भ॑वति।
नि॒त॒राम॒र्चिरु॒पावै॑ति लोहि॒नीके॑व भवति।
इन्द्र॒स्तर्ह्य॒ग्निः।
तस्मि॒न्॒ यस्य॒ तथा॑विधे॒ जुह्व॑ति।
इन्द्र॑ ए॒वास्या᳚ग्निहो॒त्रꣳ हु॒तं भ॑वति॥५८॥

%2.1.10.3
अङ्गा॑रा भवन्ति।
तेभ्यो\-ऽङ्गा॑रेभ्यो॒\-ऽर्चिरुदे॑ति।
प्र॒जा\-प॑ति॒\-स्तर्ह्य॒ग्निः।
तस्मि॒न्॒ यस्य॒ तथा॑विधे॒ जुह्व॑ति।
प्र॒जा\-प॑तावे॒वास्या᳚ग्नि\-हो॒त्रꣳ हु॒तं भ॑वति।
शरोऽङ्गा॑रा॒ अध्यू॑हन्ते।
ब्रह्म॒ तर्ह्य॒ग्निः।
तस्मि॒न्॒ यस्य॒ तथा॑विधे॒ जुह्व॑ति।
ब्रह्म॑न्ने॒वास्या᳚ग्निहो॒त्रꣳ हु॒तं भ॑वति।
वसु॑षु रु॒द्रेष्वा॑दि॒त्येषु॒ विश्वे॑षु दे॒वेषु॑।
इन्द्रे᳚ प्र॒जा\-प॑तौ॒ ब्रह्मन्॑।
अप॑रिवर्गमे॒वास्यै॒तासु॑ दे॒वता॑सु हु॒तं भ॑वति।
यस्यै॒वं वि॒दुषो᳚\-ऽग्निहो॒त्रं जुह्व॑ति।
य उ॑ चैनदे॒वं वेद॑॥५९॥\anuvakamend[आ॒दि॒त्यास्तर्ह्य॒ग्निरिन्द्र॑ ए॒वास्या᳚ग्निहो॒त्रꣳ हु॒तं भ॑वति दे॒वेषु॑ च॒त्वारि॑ च (यद॒ग्निन्निहि॑तः प्रथ॒मꣳ सर्व॑ ए॒व नि॑त॒रामङ्गा॑राः॒ शरो\-ऽङ्गा॑रा॒ ब्रह्म॒ वसु॑ष्व॒ष्टौ॥)]

%2.1.11.1
ऋ॒तं त्वा॑ स॒त्येन॒ परि॑षिञ्चा॒मीति॑ सा॒यं परि॑षिञ्चति।
स॒त्यं त्व॒र्तेन॒ परि॑षिञ्चा॒मीति॑ प्रा॒तः।
अ॒ग्निर्वा ऋ॒तम्।
अ॒सावा॑दि॒त्यः स॒त्यम्।
अ॒ग्निमे॒व तदा॑दि॒त्येन॑ सा॒यं परि॑षिञ्चति।
अ॒ग्निना॑\-ऽऽदि॒त्यं प्रा॒तः सः।
याव॑दहोरा॒त्रे भव॑तः।
ताव॑दस्य लो॒कस्य॑।
नार्ति॒र्न रिष्टिः॑।
नान्तो॒ न प॑र्य॒न्तो᳚\-ऽस्ति।
यस्यै॒वं वि॒दुषो᳚\-ऽग्निहो॒त्रं जुह्व॑ति।
य उ॑चैनदे॒वं वेद॑॥६०॥\anuvakamend[अ॒स्ति॒ द्वे च॑]




\prashnaend{अङ्गि॑रसः प्र॒जा\-प॑तिर॒ग्निꣳ रु॒द्र उ॑त्त॒राव॑तीं ब्रह्मवा॒दिनो᳚\-ऽग्निहो॒त्रप्रा॑यणा य॒ज्ञाः प्र॒जा\-प॑तिरकामयता\-ऽऽत्म॒न्वद्रौ॒द्रङ्गवि॑ दक्षिण॒तस्त्रयो॒ वै यद॒ग्निमृ॒तं त्वा॑ स॒त्येनैका॑दश॥११॥}{अङ्गि॑रसः॒ प्रैव तेन॑ प॒शूने॒व यन्नि॒मार्ष्टि॒ यो वा अ॑ग्निहो॒त्रस्यो॑प॒सदो॑ दक्षिण॒तः ष॒ष्टिः॥६०॥}{अङ्गि॑रसो॒ य उ॑चैनदे॒वं वेद॑॥}{हरिः॑ ओम्॥}{इति श्रीकृष्णयजुर्वेदीयतैत्तिरीयब्राह्मणे द्वितीयाष्टके प्रथमः प्रपाठकः समाप्तः॥}
\clearpage
\sect{द्वितीयः प्रश्नः}
\setcounter{anuvakam}{0}
\dnsub{तैत्तिरीयब्राह्मणे द्वितीयाष्टके द्वितीयः प्रपाठकः}

%2.2.1.1
प्र॒जा\-प॑तिरकामयत प्र॒जाः सृ॑जे॒येति॑।
स ए॒तं दश॑होतारम\-पश्यत्।
तं मन॑सा\-ऽनु॒द्रुत्य॑ दर्भस्त॒म्बे॑\-ऽजुहोत्।
ततो॒ वै स प्र॒जा अ॑\-सृजत।
ता अ॑स्माथ्सृ॒ष्टा अपा᳚क्रामन्।
ता ग्रहे॑णागृह्णात्।
तद्ग्रह॑स्य ग्रह॒त्वम्।
यः का॒मये॑त॒ प्रजा॑ये॒येति॑।
स दश॑होतारं॒ मन॑सा\-ऽनु॒द्रुत्य॑ दर्भस्त॒म्बे जु॑हुयात्।
प्र॒जा\-प॑ति॒र्वै दश॑होता॥१॥

%2.2.1.2
प्र॒जा\-प॑तिरे॒व भू॒त्वा प्रजा॑यते।
मन॑सा जुहोति।
मन॑ इव॒ हि प्र॒जा\-प॑तिः।
प्र॒जा\-प॑ते॒राप्त्यै᳚।
पू॒र्णया॑ जुहोति।
पू॒र्ण इ॑व॒ हि प्र॒जा\-प॑तिः।
प्र॒जा\-प॑ते॒राप्त्यै᳚।
न्यू॑नया जुहोति।
न्यू॑ना॒द्धि प्र॒जा\-प॑तिः प्र॒जा असृ॑जत।
प्र॒जाना॒ꣳ॒ सृष्ट्यै᳚॥२॥

%2.2.1.3
द॒र्भ॒स्त॒म्बे जु॑होति।
ए॒तस्मा॒द्वै योनेः᳚ प्र॒जा\-प॑तिः प्र॒जा अ॑\-सृजत।
यस्मा॑दे॒व योनेः᳚ प्र॒जा\-प॑तिः प्र॒जा असृ॑जत।
तस्मा॑दे॒व योनेः॒ प्रजा॑यते।
ब्रा॒ह्म॒णो द॑क्षिण॒त उपा᳚स्ते।
ब्रा॒ह्म॒णो वै प्र॒जाना॑मुपद्र॒ष्टा।
उ॒प॒द्र॒ष्टु॒मत्ये॒व प्रजा॑यते।
ग्रहो॑ भवति।
प्र॒जानाꣳ॑ सृ॒ष्टानां॒ धृत्यै᳚।
यं ब्रा᳚ह्म॒णं वि॒द्यां वि॒द्वाꣳसं॒ यशो॒ नर्च्छेत्॥३॥

%2.2.1.4
सोऽर॑ण्यं प॒रेत्य॑।
द॒र्भ॒स्त॒म्बमु॒द्ग्रथ्य॑।
ब्रा॒ह्म॒णं द॑क्षिण॒तो नि॒षाद्य॑।
चतु॑र्\mbox{}होतॄ॒न्व्याच॑क्षीत।
ए॒तद्वै दे॒वानां᳚ पर॒मं गुह्यं॒ ब्रह्म॑।
यच्चतु॑र्‌\mbox{}होतारः।
तदे॒व प्र॑का॒शं ग॑मयति।
तदे॑नं प्रका॒शं ग॒तम्।
प्र॒का॒शं प्र॒जानां᳚ गमयति।
द॒र्भ॒स्त॒म्बमु॒द्ग्रथ्य॒ व्याच॑ष्टे॥४॥

%2.2.1.5
अ॒ग्नि॒वान् वै द॑र्भस्त॒म्बः।
अ॒ग्नि॒वत्ये॒व व्याच॑ष्टे।
ब्रा॒ह्म॒णो द॑क्षिण॒त उपा᳚स्ते।
ब्रा॒ह्म॒णो वै प्र॒जाना॑मुपद्र॒ष्टा।
उ॒प॒द्र॒ष्टु॒मत्ये॒वैनं॒ यश॑ ऋच्छति।
ई॒श्व॒रन्तं यशोर्तो॒रित्या॑हुः।
यस्यान्ते᳚ व्या॒चष्ट॒ इति॑।
वर॒स्तस्मै॒ देयः॑।
यदे॒वैनं॒ तत्रो॑प॒नम॑ति।
तदे॒वाव॑ रुन्धे॥५॥

%2.2.1.6
अ॒ग्निमा॒दधा॑नो॒ दश॑होत्रा॒\-ऽरणि॒मव॑ दध्यात्।
प्रजा॑तमे॒वैन॒मा ध॑त्ते।
तेनै॒वोद्द्रुत्या᳚ग्निहो॒त्रं जु॑हुयात्।
प्रजा॑तमे॒वैन॑ज्जुहोति।
ह॒विर्नि॑र्व॒फ्स्यं दश॑होतारं॒ व्याच॑क्षीत।
प्रजा॑तमे॒वैनं॒ निर्व॑पति।
सा॒मि॒धे॒नीर॑नुव॒क्ष्यं दश॑होतारं॒ व्याच॑क्षीत।
सा॒मि॒धे॒नीरे॒व सृ॒ष्ट्वा\-ऽऽरभ्य॒ प्रत॑नुते।
अथो॑ य॒ज्ञो वै दश॑होता।
य॒ज्ञमे॒व त॑नुते॥६॥

%2.2.1.7
अ॒भि॒चरं॒ दश॑होतारं जुहुयात्।
नव॒ वै पुरु॑षे प्रा॒णाः।
नाभि॑र्दश॒मी।
सप्रा॑णमे॒वैन॑म॒भि च॑रति।
ए॒ताव॒द्वै पुरु॑षस्य॒ स्वम्।
याव॑त्प्रा॒णाः।
याव॑दे॒वास्यास्ति॑।
तद॒भि च॑रति।
स्वकृ॑त॒ इरि॑णे जुहोति प्रद॒रे वा᳚।
ए॒तद्वा अ॒स्यै निर्‌\mbox{}ऋ॑तिगृहीतम्।
निर्‌\mbox{}ऋ॑तिगृहीत ए॒वैनं॒ निर्‌\mbox{}ऋ॑त्या ग्राहयति।
यद्वा॒चः क्रू॒रम्।
तेन॒ वष॑ट्करोति।
वा॒च ए॒वैनं॑ क्रू॒रेण॒ प्र वृ॑श्चति।
ता॒जगार्ति॒मार्च्छ॑ति॥७॥\anuvakamend[दश॑होता॒ सृष्ट्या॑ ऋ॒च्छेद्व्याच॑प्टे रुन्ध ए॒व त॑नुते॒ निर्‌\mbox{}ऋ॑तिगृहीतं॒ पञ्च॑ च]

%2.2.2.1
प्र॒जा\-प॑तिरकामयत दर्‌\mbox{}शपूर्णमा॒सौ सृ॑जे॒येति॑।
स ए॒तं  चतु॑र्‌\mbox{}होतारमपश्यत्।
तं मन॑सा\-ऽनु॒द्रुत्या॑\-ऽऽहव॒नीये॑\-ऽजुहोत्।
ततो॒ वै स द॑र्‌\mbox{}शपूर्णमा॒साव॑\-सृजत।
ताव॑स्माथ्सृ॒ष्टावपा᳚\-क्रामताम्।
तौ ग्रहे॑णागृह्णात्।
तद्ग्रह॑स्य ग्रह॒त्वम्।
द॒र्श॒पू॒र्ण॒मा॒सावा॒लभ॑मानः।
चतु॑र्‌\mbox{}होतारं॒ मन॑सा\-ऽनु॒द्रुत्या॑\-हव॒नीये॑ जुहुयात्।
द॒र्श॒पू॒र्ण॒मा॒सावे॒व सृ॒ष्ट्वा\-ऽऽरभ्य॒ प्रत॑नुते॥८॥

%2.2.2.2
ग्रहो॑ भवति।
द॒र्‌॒\mbox{}श॒पू॒र्ण॒मा॒सयोः᳚ सृ॒ष्टयो॒र्धृत्यै᳚।
सो॑ऽकामयत चातुर्मा॒स्यानि॑ सृजे॒येति॑।
स ए॒तं पञ्च॑होतारमपश्यत्।
तं मन॑सा\-ऽनु॒द्रुत्या॑\-ऽऽहव॒नीये॑\-ऽजुहोत्।
ततो॒ वै स चा॑तुर्मा॒स्यान्य॑\-सृजत।
तान्य॑स्माथ्सृ॒ष्टान्यपा᳚क्रामन्।
तानि॒ ग्रहे॑णागृह्णात्।
तद्ग्रह॑स्य ग्रह॒त्वम्।
चा॒तु॒र्मा॒स्यान्या॒लभ॑मानः॥९॥

%2.2.2.3
पञ्च॑होतारं॒ मन॑सा\-ऽनु॒द्रुत्या॑\-ऽऽहव॒नीये॑ जुहुयात्।
चा॒तु॒र्मा॒स्या\-न्ये॒व सृ॒ष्ट्वा\-ऽऽरभ्य॒ प्रत॑नुते।
ग्रहो॑ भवति।
चा॒तु॒र्मा॒स्यानाꣳ॑ सृ॒ष्टानां॒ धृत्यै᳚।
सो॑ऽकामयत पशुब॒न्धꣳ सृ॑जे॒येति॑।
स ए॒तꣳ षड्ढो॑तारमपश्यत्।
तं मन॑सा\-ऽनु॒द्रुत्या॑\-ऽऽहव॒नीये॑\-ऽजुहोत्।
ततो॒ वै स प॑शुब॒न्धम॑\-सृजत।
सो᳚स्माथ्सृ॒ष्टो\-ऽपा᳚क्रामत्।
तं ग्रहे॑णागृह्णात्॥१०॥

%2.2.2.4
तद्ग्रह॑स्य ग्रह॒त्वम्।
प॒शु॒ब॒न्धेन॑ य॒क्ष्यमा॑णः।
षड्ढो॑तारं॒ मन॑सा\-ऽनु॒द्रुत्या॑\-ऽऽहव॒नीये॑ जुहुयात्।
प॒शु॒ब॒न्धमे॒व सृ॒ष्ट्वा\-ऽऽरभ्य॒ प्र त॑नुते।
ग्रहो॑ भवति।
प॒शु॒ब॒न्धस्य॑ सृ॒ष्टस्य॒ धृत्यै᳚।
सो॑ऽकामयत सौ॒म्यम॑ध्व॒रꣳ सृ॑जे॒येति॑।
स ए॒तꣳ स॒प्तहो॑तारमपश्यत्।
तं मन॑सा\-ऽनु॒द्रुत्या॑\-ऽऽहव॒नीये॑\-ऽजुहोत्।
ततो॒ वै स सौ॒म्यम॑ध्व॒रम॑\-सृजत॥११॥

%2.2.2.5
सो᳚ऽस्माथ्सृ॒ष्टो\-ऽपा᳚क्रामत्।
तं ग्रहे॑णागृह्णात्।
तद्ग्रह॑स्य ग्रह॒त्वम्।
दी॒क्षि॒ष्यमा॑णः।
स॒प्तहो॑तारं॒ मन॑सा\-ऽनु॒द्रुत्या॑\-ऽऽहव॒नीये॑ जुहुयात्।
सौ॒म्यमे॒वाध्व॒रꣳ सृ॒ष्ट्वा\-ऽऽरभ्य॒ प्र त॑नुते।
ग्रहो॑ भवति।
सौ॒म्यस्या᳚ध्व॒रस्य॑ सृ॒ष्टस्य॒ धृत्यै᳚।
दे॒वेभ्यो॒ वै य॒ज्ञो न प्राभ॑वत्।
तमे॑ताव॒च्छः सम॑भरन्॥१२॥

%2.2.2.6
यथ्स॑म्भा॒राः।
ततो॒ वै तेभ्यो॑ य॒ज्ञः प्राभ॑वत्।
यथ्स॑म्भा॒रा भव॑न्ति।
य॒ज्ञस्य॒ प्रभू᳚त्यै।
आ॒ति॒थ्यमा॒साद्य॒ व्याच॑ष्टे।
य॒ज्ञ॒\-मु॒खं वा आ॑ति॒थ्यम्।
मु॒ख॒त ए॒व य॒ज्ञꣳ स॒म्भृत्य॒ प्र त॑नुते।
अय॑ज्ञो॒ वा ए॒षः।
यो॑ऽप॒त्नीकः॑।
न प्र॒जाः प्रजा॑येरन्।
पत्नी॒र्व्याच॑ष्टे।
य॒ज्ञमे॒वाकः॑।
प्र॒जानां᳚ प्र॒जन॑नाय।
उ॒प॒सथ्सु॒ व्याच॑ष्टे।
ए॒तद्वै पत्नी॑नामा॒यत॑नम्।
स्व ए॒वैना॑ आ॒यत॒ने\-ऽव॑कल्पयति॥१३॥\anuvakamend[त॒नु॒त॒ आ॒लभ॑मानो\-ऽगृह्णाद\-सृजताभरञ्जायेर॒न्थ्षट्च॑]

%2.2.3.1
प्र॒जा\-प॑तिरकामयत॒ प्रजा॑ये॒येति॑।
स तपो॑\-ऽतप्यत।
स त्रि॒वृत॒ꣴ॒ स्तोम॑म\-सृजत।
तं प॑ञ्चद॒शः स्तोमो॑ मध्य॒त उद॑तृणत्।
तौ पू᳚र्वप॒क्षश्चा॑परप॒क्षश्चा॑भवताम्।
पू॒र्व॒प॒क्षं दे॒वा अन्वसृ॑ज्यन्त।
अ॒प॒र॒प॒क्षमन्वसु॑राः।
ततो॑ दे॒वा अभ॑वन्।
पराऽसु॑राः।
यं का॒मये॑त॒ वसी॑यान्थ्स्या॒दिति॑॥१४॥

%2.2.3.2
तं पू᳚र्वप॒क्षे या॑जयेत्।
वसी॑याने॒व भ॑वति।
यं का॒मये॑त॒ पापी॑यान्थ्स्या॒दिति॑।
तम॑परप॒क्षे या॑जयेत्।
पापी॑याने॒व भ॑वति।
तस्मा᳚त्पूर्वप॒क्षो॑\-ऽपरप॒क्षात्क॑रु॒ण्य॑तरः।
प्र॒जा\-प॑ति॒र्वै दश॑होता।
चतु॑र्‌\mbox{}होता॒ पञ्च॑होता।
षड्ढो॑ता स॒प्तहो॑ता।
ऋ॒तवः॑ संवथ्स॒रः॥१५॥

%2.2.3.3
प्र॒जाः प॒शव॑ इ॒मे लो॒काः।
य ए॒वं प्र॒जा\-प॑तिं ब॒होर्भूयाꣳ॑सं॒ वेद॑।
ब॒होरे॒व भूया᳚न्भवति।
प्र॒जा\-प॑तिर्देवासु॒रान॑\-सृजत।
स इन्द्र॒मपि॒ नासृ॑जत।
तं दे॒वा अ॑ब्रुवन्।
इन्द्रं॑ नो जन॒येति॑।
सो᳚ऽब्रवीत्।
यथा॒ऽहं यु॒ष्माꣴस्तप॒सा\-ऽसृ॑क्षि।
ए॒वमिन्द्रं॑ जनयध्व॒मिति॑॥१६॥

%2.2.3.4
ते तपो॑\-ऽतप्यन्त।
त आ॒त्मन्निन्द्र॑मपश्यन्।
तम॑ब्रुवन्।
जाय॒स्वेति॑।
सो᳚ऽब्रवीत्।
किं भा॑ग॒धेय॑म॒भि ज॑निष्य॒ इति॑।
ऋ॒तून्थ्सं॑वथ्स॒रम्।
प्र॒जाः प॒शून्।
इ॒माँल्लो॒कानित्य॑ब्रुवन्।
तं वै माऽऽहु॑त्या॒ प्र ज॑नय॒तेत्य॑ब्रवीत्॥१७॥

%2.2.3.5
तं चतु॑र्‌\mbox{}होत्रा॒ प्राज॑नयन्।
यः का॒मये॑त वी॒रो म॒ आजा॑ये॒तेति॑।
स चतु॑र्‌\mbox{}होतारं जुहुयात्।
प्र॒जा\-प॑ति॒र्वै चतु॑र्\mbox{}होता।
प्र॒जा\-प॑तिरे॒व भू॒त्वा प्रजा॑यते।
ज॒जन॒दिन्द्र॑मिन्द्रि॒याय॒ स्वाहेति॒ ग्रहे॑ण जुहोति।
आऽस्य॑ वी॒रो जा॑यते।
वी॒रꣳ हि दे॒वा ए॒तया\-ऽऽहु॑त्या॒ प्राज॑नयन्।
आ॒दि॒त्याश्चाङ्गि॑रसश्च सुव॒र्गे लो॒के᳚\-ऽस्पर्धन्त।
व॒यं पूर्वे॑ सुव॒र्गं लो॒कमि॑याम व॒यं पूर्व॒ इति॑॥१८॥

%2.2.3.6
त आ॑दि॒त्या ए॒तं पञ्च॑होतारमपश्यन्।
तं पु॒रा प्रा॑तरनु\-वा॒कादाग्नी᳚ध्रे\-ऽजुहवुः।
ततो॒ वै ते पूर्वे॑ सुव॒र्गं लो॒कमा॑यन्।
यः सु॑व॒र्गका॑मः॒ स्यात्।
स पञ्च॑होतारं पु॒रा प्रा॑तरनु\-वा॒का\-दाग्नी᳚ध्रे जुहुयात्।
सं॒व॒थ्स॒रो वै पञ्च॑होता।
सं॒व॒थ्स॒रः सु॑व॒र्गो लो॒कः।
सं॒व॒थ्स॒र ए॒वर्तुषु॑ प्रति॒ष्ठाय॑।
सु॒व॒र्गं लो॒कमे॑ति।
ते᳚ऽब्रुव॒न्नङ्गि॑रस आदि॒त्यान्॥१९॥

%2.2.3.7
क्व॑ स्थ।
क्व॑ वः स॒द्भ्यो ह॒व्यं व॑क्ष्याम॒ इति॑।
छन्दः॒ स्वित्य॑ब्रुवन्।
गा॒य॒त्रि॒यां त्रि॒ष्टुभि॒ जग॑त्या॒मिति॑।
तस्मा॒च्छन्दः॑ सु स॒द्भ्य आ॑दि॒त्येभ्यः॑।
आ॒ङ्गी॒र॒सीः प्र॒जा ह॒व्यं व॑हन्ति।
वह॑न्त्यस्मै प्र॒जा ब॒लिम्।
ऐन॒मप्र॑तिख्यातं गच्छति।
य ए॒वं वेद॑।
द्वाद॑श॒ मासाः॒ पञ्च॒र्तवः॑।
त्रय॑ इ॒मे लो॒काः।
अ॒सावा॑दि॒त्य ए॑कवि॒ꣳ॒शः।
ए॒तस्मि॒न्वा ए॒ष श्रि॒तः।
ए॒तस्मि॒न्प्रति॑\-ष्ठितः।
य ए॒वमे॒तꣴ श्रि॒तं प्रति॑\-ष्ठितं॒ वेद॑।
प्रत्ये॒व ति॑ष्ठति॥२०॥\anuvakamend[स्या॒दिति॑ संवथ्स॒रो ज॑नयध्व॒मितीत्य॑ब्रवी॒त्पूर्व॒ इत्या॑दि॒त्यानृ॒तवः॒ षट्च॑]

%2.2.4.1
प्र॒जा\-प॑तिरकामयत॒ प्रजा॑ये॒येति॑।
स ए॒तं दश॑होतारमपश्यत्।
तेन॑ दश॒धा\-ऽऽत्मानं॑ वि॒धाय॑।
दश॑होत्रा\-ऽतप्यत।
तस्य॒ चित्तिः॒ स्रुगासी᳚त्।
चि॒त्तमाज्यम्᳚।
तस्यै॒ताव॑त्ये॒व वागासी᳚त्।
ए॒तावान्॑ यज्ञक्र॒तुः।
स चतु॑र्‌\mbox{}होतारम\-सृजत।
सो॑ऽनन्दत्॥२१॥

%2.2.4.2
असृ॑क्षि॒ वा इ॒ममिति॑।
तस्य॒ सोमो॑ ह॒विरासी᳚त्।
स चतु॑र्‌\mbox{}होत्रा\-ऽतप्यत।
सो॑ऽताम्यत्।
स भूरिति॒ व्याह॑रत्।
स भूमि॑म\-सृजत।
अ॒ग्नि॒हो॒त्रं द॑र्‌\mbox{}शपूर्णमा॒सौ यजूꣳ॑षि।
स द्वि॒तीय॑मतप्यत।
सो॑ऽताम्यत्।
स भुव॒ इति॒ व्याह॑रत्॥२२॥

%2.2.4.3
सो᳚ऽन्तरि॑क्षम\-सृजत।
चा॒तु॒र्मा॒स्यानि॒ सामा॑नि।
स तृ॒तीय॑मतप्यत।
सो॑ऽताम्यत्।
स सुव॒रिति॒ व्याह॑रत्।
स दिव॑म\-सृजत।
अ॒ग्नि॒ष्टो॒ममु॒क्थ्य॑मतिरा॒त्रमृचः॑।
ए॒ता वै व्याहृ॑तय इ॒मे लो॒काः।
इ॒मान्खलु॒ वै लो॒काननु॑ प्र॒जाः प॒शव॒श्छन्दाꣳ॑सि॒ प्राजा॑यन्त।
य ए॒वमे॒ताः प्र॒जा\-प॑तेः प्रथ॒मा व्याहृ॑तीः॒ प्रजा॑ता॒ वेद॑॥२३॥

%2.2.4.4
प्र प्र॒जया॑ प॒शुभि॑र्मिथु॒नैर्जा॑यते।
स पञ्च॑होतारम\-सृजत।
स ह॒विर्नावि॑न्दत।
तस्मै॒ सोम॑स्त॒नुवं॒ प्राय॑च्छत्।
ए॒तत्ते॑ ह॒विरिति॑।
स पञ्च॑होत्रा\-ऽतप्यत।
सो॑ऽताम्यत्।
स प्र॒त्यङ्ङ॑बाधत।
सोऽसु॑रान\-सृजत।
तद॒स्याप्रि॑यमासीत्॥२४॥

%2.2.4.5
तद्दु॒र्वर्ण॒ꣳ॒ हिर॑ण्यमभवत्।
तद्दु॒र्वर्ण॑स्य॒ हिर॑ण्यस्य॒ जन्म॑।
स द्वि॒तीय॑मतप्यत।
सो॑ऽताम्यत्।
स प्राङ॑बाधत।
स दे॒वान॑\-सृजत।
तद॑स्य प्रि॒यमा॑सीत्।
तथ्सु॒वर्ण॒ꣳ॒ हिर॑ण्यमभवत्।
तथ्सु॒वर्ण॑स्य॒ हिर॑ण्यस्य॒ जन्म॑।
य ए॒वꣳ सु॒वर्ण॑स्य॒ हिर॑ण्यस्य॒ जन्म॒ वेद॑॥२५॥

%2.2.4.6
सु॒वर्ण॑ आ॒त्मना॑ भवति।
दु॒र्वर्णो᳚\-ऽस्य॒ भ्रातृ॑व्यः।
तस्मा᳚थ्सु॒वर्ण॒ꣳ॒ हिर॑ण्यं भा॒र्यम्᳚।
सु॒वर्ण॑ ए॒व भ॑वति।
ऐनं॑ प्रि॒यं ग॑च्छति॒ नाप्रि॑यम्।
स स॒प्तहो॑तारम\-सृजत।
स स॒प्तहो᳚त्रै॒व सु॑व॒र्गं लो॒कमै᳚त्।
त्रि॒ण॒वेन॒ स्तोमे॑नै॒भ्यो लो॒केभ्यो\-ऽसु॑रा॒न्प्राणु॑दत।
त्र॒य॒स्त्रि॒ꣳ॒शेन॒ प्रत्य॑तिष्ठत्।
ए॒क॒वि॒ꣳ॒शेन॒ रुच॑मधत्त॥२६॥

%2.2.4.7
स॒प्त॒द॒शेन॒ प्राजा॑यत।
य ए॒वं वि॒द्वान्थ्सोमे॑न॒ यज॑ते।
स॒प्तहो᳚त्रै॒व सु॑व॒र्गं लो॒कमे॑ति।
त्रि॒ण॒वेन॒ स्तोमे॑नै॒भ्यो लो॒केभ्यो॒ भ्रातृ॑व्या॒न्प्रणु॑दते।
त्र॒य॒स्त्रि॒ꣳ॒शेन॒ प्रति॑ तिष्ठति।
ए॒क॒वि॒ꣳ॒शेन॒ रुचं॑ धत्ते।
स॒प्त॒द॒शेन॒ प्र जा॑यते।
तस्मा᳚थ्सप्तद॒शः स्तोमो॒ न नि॒र्॒हृत्यः॑।
प्र॒जा\-प॑ति॒र्वै स॑प्तद॒शः।
प्र॒जा\-प॑तिमे॒व म॑ध्य॒तो ध॑त्ते॒ प्रजा᳚त्यै॥२७॥\anuvakamend[अ॒न॒न्द॒द्भुव॒ इति॒ व्याह॑र॒द्वेदा॑सी॒द्वेदा॑धत्त॒ प्रजा᳚त्यै]

%2.2.5.1
दे॒वा वै वरु॑णमयाजयन्।
स यस्यै॑यस्यै दे॒वता॑यै॒ दक्षि॑णा॒मन॑यत्।
ताम॑व्लीनात्।
ते᳚ऽब्रुवन्।
व्या॒वृत्य॒ प्रति॑\-गृह्णाम।
तथा॑ नो॒ दक्षि॑णा॒ न व्ले᳚ष्य॒तीति॑।
ते व्या॒वृत्य॒ प्रत्य॑गृह्णन्।
ततो॒ वै तान्दक्षि॑णा॒ नाव्ली॑नात्।
य ए॒वं वि॒द्वान्व्या॒वृत्य॒ दक्षि॑णां प्रति\-गृ॒ह्णाति॑।
नैनं॒ दक्षि॑णा व्लीनाति॥२८॥

%2.2.5.2
राजा᳚ त्वा॒ वरु॑णो नयतु देवि दक्षिणे॒\-ऽग्नये॒ हिर॑ण्य॒मित्या॑ह।
आ॒ग्ने॒यं वै हिर॑ण्यम्।
स्वयै॒वैन॑द्दे॒वत॑या॒ प्रति॑\-गृह्णाति।
सोमा॑य॒ वास॒ इत्या॑ह।
सौ॒म्यं वै वासः॑।
स्वयै॒वैन॑द्दे॒वत॑या॒ प्रति॑\-गृह्णाति।
रु॒द्राय॒ गामित्या॑ह।
रौ॒द्री वै गौः।
स्वयै॒वैनां᳚ दे॒वत॑या॒ प्रति॑\-गृह्णाति।
वरु॑णा॒याश्व॒मित्या॑ह॥२९॥

%2.2.5.3
वा॒रु॒णो वा अश्वः॑।
स्वयै॒वैनं॑ दे॒वत॑या॒ प्रति॑\-गृह्णाति।
प्र॒जा\-प॑तये॒ पुरु॑ष॒मित्या॑ह।
प्रा॒जा॒प॒त्यो वै पुरु॑षः।
स्वयै॒वैनं॑ दे॒वत॑या॒ प्रति॑ गृह्णाति।
मन॑वे॒ तल्प॒मित्या॑ह।
मा॒न॒वो वै तल्पः॑।
स्वयै॒वैनं॑ दे॒वत॑या॒ प्रति॑ गृह्णाति।
उ॒त्ता॒नाया᳚ङ्गीर॒सायान॒ इत्या॑ह।
इ॒यं वा उ॑त्ता॒न आ᳚ङ्गीर॒सः॥३०॥

%2.2.5.4
अ॒नयै॒वैन॒त्प्रति॑ गृह्णाति।
वै॒श्वा॒न॒र्यर्चा रथं॒ प्रति॑ गृह्णाति।
वै॒श्वा॒न॒रो वै दे॒वत॑या॒ रथः॑।
स्वयै॒वैनं॑ दे॒वत॑या॒ प्रति॑ गृह्णाति।
तेना॑मृत॒त्वम॑श्या॒मित्या॑ह।
अ॒मृत॑मे॒वाऽऽत्मन्ध॑त्ते।
वयो॑ दा॒त्र इत्या॑ह।
वय॑ ए॒वैनं॑ कृ॒त्वा।
सु॒व॒र्गं लो॒कं ग॑मयति।
मयो॒ मह्य॑मस्तु प्रतिग्रही॒त्र इत्या॑ह॥३१॥

%2.2.5.5
यद्वै शि॒वम्।
तन्मयः॑।
आ॒त्मन॑ ए॒वैषा परी᳚त्तिः।
क इ॒दं कस्मा॑ अदा॒दित्या॑ह।
प्र॒जा\-प॑ति॒र्वै कः।
स प्र॒जा\-प॑तये ददाति।
कामः॒ कामा॒येत्या॑ह।
कामे॑न॒ हि ददा॑ति।
कामे॑न प्रति\-गृ॒ह्णाति॑।
कामो॑ दा॒ता कामः॑ प्रतिग्रही॒तेत्या॑ह॥३२॥

%2.2.5.6
कामो॒ हि दा॒ता।
कामः॑ प्रतिग्रही॒ता।
कामꣳ॑ समु॒द्रमावि॒शे\-त्या॑ह।
स॒मु॒द्र इ॑व॒ हि कामः॑।
नेव॒ हि काम॒स्यान्तो\-ऽस्ति॑।
न स॑मु॒द्रस्य॑।
कामे॑न त्वा॒ प्रति॑\-गृह्णा॒मीत्या॑ह।
येन॒ कामे॑न प्रति\-गृ॒ह्णाति॑।
स ए॒वैन॑म॒मुष्मिँ॑ल्लो॒के काम॒ आग॑च्छति।
कामै॒तत्त॑ ए॒षा ते॑ काम॒ दक्षि॒णेत्या॑ह।
काम॑ ए॒व तद्यज॑मानो॒\-ऽमुष्मिँ॑ल्लो॒के दक्षि॑णामिच्छति।
न प्र॑तिग्रही॒तरि॑।
य ए॒वं वि॒द्वान्दक्षि॑णां प्रति\-गृ॒ह्णाति॑।
अ॒नृ॒णामे॒वैनां॒ प्रति॑ गृह्णाति॥३३॥\anuvakamend[व्ली॒ना॒त्यश्व॒मित्या॑हाङ्गीर॒सः प्र॑तिग्रही॒त्र इत्या॑ह प्रतिग्रही॒तेत्या॑ह॒ दक्षि॒णेत्या॑ह च॒त्वारि॑ च]

%2.2.6.1
अन्तो॒ वा ए॒ष य॒ज्ञस्य॑।
यद्द॑श॒ममहः॑।
द॒श॒मे\-ऽह᳚न्थ्सर्परा॒ज्ञिया॑ ऋ॒ग्भिः स्तु॑वन्ति।
य॒ज्ञस्यै॒वान्तं॑ ग॒त्वा।
अ॒न्नाद्य॒मव॑ रुन्धते।
ति॒सृभिः॑ स्तुवन्ति।
त्रय॑ इ॒मे लो॒काः।
ए॒भ्य ए॒व लो॒केभ्यो॒\-ऽन्नाद्य॒मव॑ रुन्धते।
पृश्ञि॑वतीर्भवन्ति।
अन्नं॒ वै पृश्ञि॑॥३४॥

%2.2.6.2
अन्न॑मे॒वाव॑ रुन्धते।
मन॑सा॒ प्रस्तौ॑ति।
मन॒सोद्गा॑यति।
मन॑सा॒ प्रति॑ हरति।
मन॑ इव॒ हि प्र॒जा\-प॑तिः।
प्र॒जा\-प॑ते॒राप्त्यै᳚।
दे॒वा वै स॒र्पाः।
तेषा॑मि॒यꣳ राज्ञी᳚।
यथ्स॑र्परा॒ज्ञिया॑ ऋ॒ग्भिः स्तु॒वन्ति॑।
अ॒स्यामे॒व प्रति॑ तिष्ठन्ति॥३५॥

%2.2.6.3
चतु॑र्‌\mbox{}होतॄ॒न्॒ होता॒ व्याच॑ष्टे।
स्तु॒तमनु॑शꣳसति॒ शान्त्यै᳚।
अन्तो॒ वा ए॒ष य॒ज्ञस्य॑।
यद्द॑श॒ममहः॑।
ए॒तत्खलु॒ वै दे॒वानां᳚ पर॒मं गुह्यं॒ ब्रह्म॑।
यच्चतु॑र्\mbox{}होतारः।
द॒श॒मे\-ऽह॒ꣴ॒श्चतु॑र्\-‌होतॄ॒न्व्याच॑ष्टे।
य॒ज्ञस्यै॒वान्तं॑ ग॒त्वा।
प॒र॒मं दे॒वानां॒ गुह्यं॒ ब्रह्माव॑ रुन्धे।
तदे॒व प्र॑का॒शं ग॑मयति॥३६॥

%2.2.6.4
तदे॑नं प्रका॒शं ग॒तम्।
प्र॒का॒शं प्र॒जानां᳚ गमयति।
वाचं॑ यच्छति।
य॒ज्ञस्य॒ धृत्यै᳚।
य॒ज॒मा॒न॒दे॒व॒त्यं॑ वा अहः॑।
भ्रा॒तृ॒व्य॒दे॒व॒त्या॑ रात्रिः॑।
अह्ना॒ रात्रिं॑ ध्यायेत्।
भ्रातृ॑व्यस्यै॒व तल्लो॒कं वृ॑ङ्क्ते।
यद्दिवा॒ वाचं॑ विसृ॒जेत्।
अह॒र्भ्रातृ॑व्या॒योच्छिꣳ॑षेत्।
यन्नक्तं॑ विसृ॒जेत्।
रात्रिं॒ भ्रातृ॑व्या॒योच्छिꣳ॑षेत्।
अ॒धि॒वृ॒क्ष॒सू॒र्ये वाचं॒ विसृ॑जति।
ए॒ताव॑न्तमे॒वास्मै॑ लो॒कमुच्छिꣳ॑षति।
याव॑दादि॒त्यो᳚\-ऽस्त॒मेति॑॥३७॥\anuvakamend[पृश्ञि॑ तिष्ठन्ति गमयति शिꣳषे॒त्पञ्च॑ च]

%2.2.7.1
प्र॒जा\-प॑तिः प्र॒जा अ॑\-सृजत।
ताः सृ॒ष्टाः सम॑श्लिष्यन्।
ता रू॒पेणानु॒प्रावि॑शत्।
तस्मा॑दाहुः।
रू॒पं वै प्र॒जा\-प॑ति॒रिति॑।
ता नाम्नाऽनु॒ प्रावि॑शत्।
तस्मा॑दाहुः।
नाम॒ वै प्र॒जा\-प॑ति॒रिति॑।
तस्मा॒दप्या॑मि॒त्रौ स॒ङ्गत्य॑।
नाम्ना॒ चेद्‌ध्वये॑ते॥३८॥

%2.2.7.2
मि॒त्रमे॒व भ॑वतः।
प्र॒जा\-प॑तिर्देवासु॒रान॑\-सृजत।
स इन्द्र॒मपि॒ नासृ॑जत।
तं दे॒वा अ॑ब्रुवन्।
इन्द्रं॑ नो जन॒येति॑।
स आ॒त्मन्निन्द्र॑मपश्यत्।
तम॑\-सृजत।
तं त्रि॒ष्टुग्वी॒र्यं॑ भू॒त्वाऽनु॒ प्रावि॑शत्।
तस्य॒ वज्रः॑ पञ्चद॒शो हस्त॒ आप॑द्यत।
तेनो॒दय्यासु॑रान॒भ्य॑भवत्॥३९॥

%2.2.7.3
य ए॒वं वेद॑।
अ॒भि भ्रातृ॑व्यान्भवति।
ते दे॒वा असु॑रैर्वि॒जित्य॑।
सु॒व॒र्गं लो॒कमा॑यन्।
ते॑ऽमुष्मिँ॑ल्लो॒के व्य॑क्षुध्यन्।
ते᳚ऽब्रुवन्।
अ॒मुतः॑ प्रदानं॒ वा उप॑जिजीवि॒मेति॑।
ते स॒प्तहो॑तारं य॒ज्ञं वि॒धाया॒यास्यम्᳚।
आ॒ङ्गी॒र॒सं प्राहि॑ण्वन्।
ए॒तेना॒मुत्र॑ कल्प॒येति॑॥४०॥

%2.2.7.4
तस्य॒ वा इ॒यं कॢप्तिः॑।
यदि॒दं किं च॑।
य ए॒वं वेद॑।
कल्प॑ते\-ऽस्मै।
स वा अ॒यं म॑नु॒ष्ये॑षु य॒ज्ञः स॒प्तहो॑ता।
अ॒मुत्र॑ स॒द्भ्यो दे॒वेभ्यो॑ ह॒व्यं व॑हति।
य ए॒वं वेद॑।
उपै॑नं य॒ज्ञो न॑मति।
सो॑ऽमन्यत।
अ॒भि वा इ॒मे᳚\-ऽस्माल्लो॒काद॒मुं लो॒कं क॑मिष्यन्त॒ इति॑।
स वाच॑स्पते॒ हृदिति॒ व्याह॑रत्।
तस्मा᳚त्पु॒त्रो हृद॑यम्।
तस्मा॑द॒स्माल्लो॒काद॒मुं लो॒कं नाभि का॑मयन्ते।
पु॒त्रो हि हृद॑यम्॥४१॥\anuvakamend[ह्वये॑ते अभवत्कल्प॒येतीति॑ च॒त्वारि॑ च]

%2.2.8.1
दे॒वा वै चतु॑र्‌\mbox{}होतृभिर्य॒ज्ञम॑तन्वत।
ते वि पा॒प्मना॒ भ्रातृ॑व्ये॒णाज॑यन्त।
अ॒भि सु॑व॒र्गं लो॒कम॑जयन्।
य ए॒वं वि॒द्वाꣴश्चतु॑र्\mbox{}होतृभिर्य॒ज्ञं त॑नु॒ते।
वि पा॒प्मना॒ भ्रातृ॑व्येण जयते।
अ॒भि सु॑व॒र्गं लो॒कं ज॑यति।
षड्ढो᳚त्रा प्राय॒णीय॒मा सा॑दयति।
अ॒मुष्मै॒ वै लो॒काय॒ षड्ढो॑ता।
घ्नन्ति॒ खलु॒ वा ए॒तथ्सोमम्᳚।
यद॑भिषु॒ण्वन्ति॑॥४२॥

%2.2.8.2
ऋ॒जु॒धैवैन॑म॒मुं लो॒कं ग॑मयति।
चतु॑र्‌\mbox{}होत्रा\-ऽऽति॒थ्यम्।
यशो॒ वै चतु॑र्‌\mbox{}होता।
यश॑ ए॒वाऽऽत्मन्ध॑त्ते।
पञ्च॑होत्रा प॒शुमुप॑सादयति।
सु॒व॒र्ग्यो॑ वै पञ्च॑होता।
यज॑मानः प॒शुः।
यज॑मानमे॒व सु॑व॒र्गं लो॒कं ग॑मयति।
ग्रहा᳚न्गृही॒त्वा स॒प्तहो॑तारं जुहोति।
इ॒न्द्रि॒यं वै स॒प्तहो॑ता॥४३॥

%2.2.8.3
इ॒न्द्रि॒यमे॒वाऽऽत्मन्ध॑त्ते।
यो वै चतु॑र्‌\mbox{}होतॄननुसव॒नं त॒र्पय॑ति।
तृप्य॑ति प्र॒जया॑ प॒शुभिः॑।
उपै॑नꣳ सोमपी॒थो न॑मति।
ब॒हि॒ष्प॒व॒मा॒ने दश॑होतारं॒ व्याच॑क्षीत।
माध्यं॑ दिने॒ पव॑माने॒ चतु॑र्‌\mbox{}होतारम्।
आर्भ॑वे॒ पव॑माने॒ पञ्च॑होतारम्।
पि॒तृ॒य॒ज्ञे षड्ढो॑तारम्।
य॒ज्ञा॒य॒ज्ञिय॑स्य स्तो॒त्रे स॒प्तहो॑तारम्।
अ॒नु॒स॒व॒नमे॒वैनाꣴ॑स्तर्पयति॥४४॥

%2.2.8.4
तृप्य॑ति प्र॒जया॑ प॒शुभिः॑।
उपै॑नꣳ सोमपी॒थो न॑मति।
दे॒वा वै चतु॑र्‌\mbox{}होतृभिः स॒त्रमा॑सत।
ऋद्धि॑परिमितं॒ यश॑स्कामाः।
ते᳚ऽब्रुवन्।
यन्नः॑ प्रथ॒मं यश॑ ऋ॒च्छात्।
सर्वे॑षान्न॒स्तथ्स॒हास॒दिति॑।
सोम॒श्चतु॑र्‌\mbox{}होत्रा।
अ॒ग्निः पञ्च॑होत्रा।
धा॒ता षड्ढो᳚त्रा॥४५॥

%2.2.8.5
इन्द्रः॑ स॒प्तहो᳚त्रा।
प्र॒जा\-प॑ति॒र्दश॑होत्रा।
तेषा॒ꣳ॒ सोम॒ꣳ॒ राजा॑नं॒ यश॑ आर्च्छत्।
तन्न्य॑कामयत।
तेनापा᳚क्रामत्।
तेन॑ प्र॒लाय॑मचरत्।
तं दे॒वाः प्रै॒षैः प्रैष॑मैच्छन्।
तत्प्रै॒षाणां᳚ प्रैष॒त्वम्।
नि॒विद्भि॒र्न्य॑वेदयन्।
तन्नि॒विदा᳚न्निवि॒त्त्वम्॥४६॥

%2.2.8.6
आ॒प्रीभि॑राप्नुवन्।
तदा॒प्रीणा॑माप्रि॒त्वम्।
तम॑घ्नन्।
तस्य॒ यशो॒ व्य॑गृह्णत।
ते ग्रहा॑ अभवन्।
तद्ग्रहा॑णां ग्रह॒त्वम्।
यस्यै॒वं वि॒दुषो॒ ग्रहा॑ गृ॒ह्यन्ते᳚।
तस्य॒ त्वे॑व गृ॑ही॒ताः।
ते᳚ऽब्रुवन्।
यो वै नः॒ श्रेष्ठो\-ऽभू᳚त्॥४७॥

%2.2.8.7
तम॑वधिष्म।
पुन॑रि॒मꣳ सु॑वामहा॒ इति॑।
तं छन्दो॑भिरसुवन्त।
तच्छन्द॑सां छन्द॒स्त्वम्।
साम्ना॒ समान॑यन्।
तथ्साम्नः॑ साम॒त्वम्।
उ॒क्थैरुद॑स्थापयन्।
तदु॒क्थाना॑मुक्थ॒त्वम्।
य ए॒वं वेद॑।
प्रत्ये॒व ति॑ष्ठति॥४८॥

%2.2.8.8
सर्व॒मायु॑रेति।
सोमो॒ वै यशः॑।
य ए॒वं वि॒द्वान्थ्सोम॑मा॒गच्छ॑ति।
यश॑ ए॒वैन॑मृच्छति।
तस्मा॑दाहुः।
यश्चै॒वं वेद॒ यश्च॒ न।
तावु॒भौ सोम॒माग॑च्छतः।
सोमो॒ हि यशः॑।
तं त्वाऽव यश॑ ऋच्छ॒तीत्या॑हुः।
यः सोमे॒ सोमं॒ प्राहेति॑।
तस्मा॒थ्सोमे॒ सोमः॒ प्रोच्यः॑।
यश॑ ए॒वैन॑मृच्छति॥४९॥\anuvakamend[अ॒भि॒षु॒ण्वन्ति॑ स॒प्तहो॑ता तर्पयति॒ षड्ढो᳚त्रा निवि॒त्त्वमभू᳚त्तिष्ठति॒ प्राहेति॒ द्वे च॑]

%2.2.9.1
इ॒दं वा अग्रे॒ नैव किं च॒ नाऽऽसी᳚त्।
न द्यौरा॑सीत्।
न पृ॑थि॒वी।
नान्तरि॑क्षम्।
तदस॑दे॒व सन्मनो॑\-ऽकुरुत॒ स्यामिति॑।
तद॑तप्यत।
तस्मा᳚त्तेपा॒नाद्धू॒मो॑\-ऽजायत।
तद्भूयो॑\-ऽतप्यत।
तस्मा᳚त्तेपा॒ना\-द॒ग्निर॑जायत।
तद्भूयो॑\-ऽतप्यत॥५०॥

%2.2.9.2
तस्मा᳚त्तेपा॒नाज्ज्योति॑रजायत।
तद्भूयो॑\-ऽतप्यत।
तस्मा᳚त्तेपा॒ना\-द॒र्चिर॑जायत।
तद्भूयो॑\-ऽतप्यत।
तस्मा᳚त्तेपा॒नान्मरी॑चयो\-ऽजायन्त।
तद्भूयो॑\-ऽतप्यत।
तस्मा᳚त्तेपा॒नादु॑दा॒रा अ॑जायन्त।
तद्भूयो॑\-ऽतप्यत।
तद॒भ्रमि॑व॒ सम॑हन्यत।
तद्व॒स्तिम॑भिनत्॥५१॥

%2.2.9.3
स स॑मु॒द्रो॑\-ऽभवत्।
तस्मा᳚थ्समु॒द्रस्य॒ न पि॑बन्ति।
प्र॒जन॑नमिव॒ हि मन्य॑न्ते।
तस्मा᳚त्प॒शोर्जाय॑माना॒दापः॑ पु॒रस्ता᳚द्यन्ति।
तद्दश॑हो॒ता\-ऽन्व॑सृज्यत।
प्र॒जा\-प॑ति॒र्वै दश॑होता।
य ए॒वं तप॑सो वी॒र्यं॑ वि॒द्वाꣴस्तप्य॑ते।
भव॑त्ये॒व।
तद्वा इ॒दमापः॑ सलि॒लमा॑सीत्।
सो॑ऽरोदीत्प्र॒जा\-प॑तिः॥५२॥

%2.2.9.4
स कस्मा॑ अज्ञि।
यद्य॒स्या अप्र॑तिष्ठाया॒ इति॑।
यद॒फ्स्व॑वाप॑द्यत।
सा पृ॑थि॒व्य॑भवत्।
यद्व्यमृ॑ष्ट।
तद॒न्तरि॑क्षम\-भवत्।
यदू॒र्ध्वमु॒दमृ॑ष्ट।
सा द्यौर॑भवत्।
यदरो॑दीत्।
तद॒नयो॑ रोद॒स्त्वम्॥५३॥

%2.2.9.5
य ए॒वं वेद॑।
नास्य॑ गृ॒हे रु॑दन्ति।
ए॒तद्वा ए॒षां लो॒कानां॒ जन्म॑।
य ए॒वमे॒षां लो॒कानां॒ जन्म॒ वेद॑।
नैषु लो॒केष्वार्ति॒मार्च्छ॑ति।
स इ॒मां प्र॑ति॒ष्ठाम॑विन्दत।
स इ॒मां प्र॑ति॒ष्ठां वि॒त्वा\-ऽका॑मयत॒ प्रजा॑ये॒येति॑।
स तपो॑\-ऽतप्यत।
सो᳚ऽन्तर्वा॑नभवत्।
स ज॒घना॒दसु॑रान\-सृजत॥५४॥

%2.2.9.6
तेभ्यो॑ मृ॒न्मये॒ पात्रे\-ऽन्न॑मदुहत्।
याऽस्य॒ सा त॒नूरासी᳚त्।
तामपा॑हत।
सा तमि॑स्रा\-ऽभवत्।
सो॑ऽकामयत॒ प्रजा॑ये॒येति॑।
स तपो॑\-ऽतप्यत।
सो᳚न्तर्वा॑नभवत्।
स प्र॒जन॑नादे॒व प्र॒जा अ॑\-सृजत।
तस्मा॑दि॒मा भूयि॑ष्ठाः।
प्र॒जन॑ना॒द्ध्ये॑ना॒ असृ॑जत॥५५॥

%2.2.9.7
ताभ्यो॑ दारु॒मये॒ पात्रे॒ पयो॑\-ऽदुहत्।
याऽस्य॒ सा त॒नूरासी᳚त्।
तामपा॑हत।
सा जोथ्स्ना॑\-ऽभवत्।
सो॑ऽकामयत॒ प्रजा॑ये॒येति॑।
स तपो॑\-ऽतप्यत।
सो᳚ऽन्तर्वा॑नभवत्।
स उ॑पप॒क्षाभ्या॑मे॒वर्तून॑\-सृजत।
तेभ्यो॑ रज॒ते पात्रे॑ घृ॒तम॑दुहत्।
याऽस्य॒ सा त॒नूरासी᳚त्॥५६॥

%2.2.9.8
तामपा॑हत।
सो॑ऽहोरा॒त्रयोः᳚ स॒न्धिर॑भवत्।
सो॑ऽकामयत॒ प्रजा॑ये॒येति॑।
स तपो॑\-ऽतप्यत।
सो᳚ऽन्तर्वा॑नभवत्।
स मुखा᳚द्दे॒वान॑\-सृजत।
तेभ्यो॒ हरि॑ते॒ पात्रे॒ सोम॑मदुहत्।
याऽस्य॒ सा त॒नूरासी᳚त्।
तामपा॑हत।
तदह॑रभवत्॥५७॥

%2.2.9.9
ए॒ते वै प्र॒जा\-प॑ते॒र्दोहाः᳚।
य ए॒वं वेद॑।
दु॒ह ए॒व प्र॒जाः।
दिवा॒ वै नो॑\-ऽभू॒दिति॑।
तद्दे॒वानां᳚ देव॒त्वम्।
य ए॒वं दे॒वानां᳚ देव॒त्वं वेद॑।
दे॒ववा॑ने॒व भ॑वति।
ए॒तद्वा अ॑होरा॒त्राणां॒ जन्म॑।
य ए॒वम॑होरा॒त्राणां॒ जन्म॒ वेद॑।
नाहो॑रा॒त्रेष्वार्ति॒मार्च्छ॑ति॥५८॥

%2.2.9.10
अस॒तोऽधि॒ मनो॑\-ऽसृज्यत।
मनः॑ प्र॒जा\-प॑तिम\-सृजत।
प्र॒जा\-प॑तिः प्र॒जा अ॑\-सृजत।
तद्वा इ॒दं मन॑स्ये॒व प॑र॒मं प्रति॑\-ष्ठितम्।
यदि॒दं किं च॑।
तदे॒तच्छ्वो॑वस्य॒सन्नाम॒ ब्रह्म॑।
व्यु॒च्छन्ती᳚व्युच्छन्त्यस्मै॒ वस्य॑सीवस्यसी॒ व्यु॑च्छति।
प्रजा॑यते प्र॒जया॑ प॒शुभिः॑।
प्र प॑रमे॒ष्ठिनो॒ मात्रा॑माप्नोति।
य ए॒वं वेद॑॥५९॥\anuvakamend[अ॒ग्निर॑जायत॒ तद्भूयो॑\-ऽतप्यताभिनदरोदीत्प्र॒जा\-प॑तीरोद॒स्त्वम॑सृज॒तासृ॑जत घृ॒तम॑दुह॒द्या\-ऽस्य॒ सा त॒नूरासी॒दह॑रभवदृच्छति॒ वेद॑ (इ॒दं धू॒मो᳚\-ऽग्निर्ज्योति॑र॒र्चिर्मरी॑चय उदा॒रास्तद॒ब्भ्रꣳ स ज॒घना॒थ्सा तमि॑स्रा॒ स प्र॒जन॑ना॒थ्सा जोथ्स्ना॒ स उ॑पप॒क्षाभ्या॒ꣳ॒ सो॑\-ऽहोरा॒त्रयोः᳚ स॒न्धिः स मुखा॒त्तदह॑र्दे॒ववा᳚न्मृ॒न्मये॑ दारु॒मये॑ रज॒ते हरि॑ते॒ तेभ्य॒स्ताभ्यो॒ द्वे तेऽन्नं॒ पयो॑ घृ॒तꣳ सोमम्᳚॥)]

%2.2.10.1
प्र॒जा\-प॑ति॒रिन्द्र॑म\-सृजतानुजाव॒रं दे॒वाना᳚म्।
तं प्राहि॑णोत्।
परे॑हि।
ए॒तेषां᳚ दे॒वाना॒मधि॑पतिरे॒धीति॑।
तं दे॒वा अ॑ब्रुवन्।
कस्त्वमसि॑।
व॒यं वै त्वच्छ्रेयाꣳ॑सः स्म॒ इति॑।
सो᳚ऽब्रवीत्।
कस्त्वमसि॑ व॒यं वै त्वच्छ्रेयाꣳ॑सः स्म॒ इति॑ मा दे॒वा अ॑वोच॒न्निति॑।
अथ॒ वा इ॒दं तर्‌\mbox{}हि॑ प्र॒जा\-प॑तौ॒ हर॑ आसीत्॥६०॥

%2.2.10.2
यद॒स्मिन्ना॑दि॒त्ये।
तदे॑नमब्रवीत्।
ए॒तन्मे॒ प्रय॑च्छ।
अथा॒हमे॒तेषां᳚ दे॒वाना॒मधि॑पतिर्भविष्या॒\-मीति॑।
को॑ऽहꣴ स्या॒मित्य॑ब्रवीत्।
ए॒तत्प्र॒दायेति॑।
ए॒तथ्स्या॒ इत्य॑ब्रवीत्।
यदे॒तद्ब्रवी॒षीति॑।
को ह॒ वै नाम॑ प्र॒जा\-प॑तिः।
य ए॒वं वेद॑॥६१॥

%2.2.10.3
वि॒दुरे॑नं॒ नाम्ना᳚।
तद॑स्मै रु॒क्मं कृ॒त्वा प्रत्य॑मुञ्चत्।
ततो॒ वा इन्द्रो॑ दे॒वाना॒मधि॑पतिरभवत्।
य ए॒वं वेद॑।
अधि॑पतिरे॒व स॑मा॒नानां᳚ भवति।
सो॑ऽमन्यत।
किं किं॒ वा अ॑कर॒मिति॑।
स च॒न्द्रं म॒ आह॒रेति॒ प्राल॑पत्।
तच्च॒न्द्रम॑सश्चन्द्रम॒स्त्वम्।
य ए॒वं वेद॑॥६२॥

%2.2.10.4
च॒न्द्रवा॑ने॒व भ॑वति।
तं दे॒वा अ॑ब्रुवन्।
सु॒वीर्यो॑ मर्या॒ यथा॑ गोपा॒यत॒ इति॑।
तथ्सूर्य॑स्य सूर्य॒त्वम्।
य ए॒वं वेद॑।
नैनं॑ दभ्नोति।
कश्च॒ नास्मि॒न्वा इ॒दमि॑न्द्रि॒यं प्रत्य॑स्था॒दिति॑।
तदिन्द्र॑स्येन्द्र॒त्वम्।
य ए॒वं वेद॑।
इ॒न्द्रि॒या॒व्ये॑व भ॑वति॥६३॥

%2.2.10.5
अ॒यं वा इ॒दं प॑र॒मो॑\-ऽभू॒दिति॑।
तत्प॑रमे॒ष्ठिनः॑ परमेष्ठि॒त्वम्।
य ए॒वं वेद॑।
प॒र॒मामे॒व काष्ठां᳚ गच्छति।
तं दे॒वाः स॑म॒न्तं पर्य॑विशन्।
वस॑वः पु॒रस्ता᳚त्।
रु॒द्रा द॑क्षिण॒तः।
आ॒दि॒त्याः प॒श्चात्।
विश्वे॑ दे॒वा उ॑त्तर॒तः।
अङ्गि॑रसः प्र॒त्यञ्चम्᳚॥६४॥

%2.2.10.6
सा॒ध्याः परा᳚ञ्चम्।
य ए॒वं वेद॑।
उपै॑नꣳ समा॒नाः संवि॑शन्ति।
स प्र॒जा\-प॑तिरे॒व भू॒त्वा प्र॒जा आव॑यत्।
ता अ॑स्मै॒ नाति॑ष्ठन्ता॒न्नाद्या॑य।
ता मुखं॑ पु॒रस्ता॒त्पश्य॑न्तीः।
द॒क्षि॒ण॒तः पर्या॑यन्।
स द॑क्षिण॒तः पर्य॑वर्तयत।
ता मुखं॑ पु॒रस्ता॒त्पश्य॑न्तीः।
मुखं॑ दक्षिण॒तः॥६५॥

%2.2.10.7
प॒श्चात्पर्या॑यन्।
स प॒श्चात्पर्य॑वर्तयत।
ता मुखं॑ पु॒रस्ता॒त्पश्य॑न्तीः।
मुखं॑ दक्षिण॒तः।
मुखं॑ प॒श्चात्।
उ॒त्त॒र॒तः पर्या॑यन्।
स उ॑त्तर॒तः पर्य॑वर्तयत।
ता मुखं॑ पु॒रस्ता॒त्पश्य॑न्तीः।
मुखं॑ दक्षिण॒तः।
मुखं॑ प॒श्चात्॥६६॥

%2.2.10.8
मुख॑मुत्तर॒तः।
ऊ॒र्ध्वा उदा॑यन्।
स उ॒परि॑ष्टा॒न्न्य॑वर्तयत।
ताः स॒र्वतो॑मुखो भू॒त्वा\-ऽऽव॑यत्।
ततो॒ वै तस्मै᳚ प्र॒जा अति॑ष्ठन्ता॒न्नाद्या॑य।
य ए॒वं वि॒द्वान्परि॑ च व॒र्तय॑ते॒ नि च॑।
प्र॒जा\-प॑तिरे॒व भू॒त्वा प्र॒जा अ॑त्ति।
तिष्ठ॑न्ते\-ऽस्मै प्र॒जा अ॒न्नाद्या॑य।
अ॒न्ना॒द ए॒व भ॑वति॥६७॥\anuvakamend[आ॒सी॒द्वेद॑ चन्द्रम॒स्त्वं य ए॒वं वेदे᳚न्द्रिया॒व्ये॑व भ॑वति प्र॒त्यञ्चं॒ मुखं॑ दक्षिण॒तो मुखं॑ प॒श्चान्नव॑ च]

%2.2.11.1
प्र॒जा\-प॑तिरकामयत ब॒होर्भूया᳚न्थ्स्या॒मिति॑।
स ए॒तं दश॑होतारमपश्यत्।
तं प्रायु॑ङ्क्त।
तस्य॒ प्रयु॑क्ति ब॒होर्भूया॑नभवत्।
यः का॒मये॑त ब॒होर्भूया᳚न्थ्स्या॒मिति॑।
स दश॑होतारं॒ प्रयु॑ञ्जीत।
ब॒होरे॒व भूया᳚न्भवति।
सो॑ऽकामयत वी॒रो म॒ आजा॑ये॒तेति॑।
स दश॑होतु॒श्चतु॑र्‌\mbox{}होतारं॒ निर॑मिमीत।
तं प्रायु॑ङ्क्त॥६८॥

%2.2.11.2
तस्य॒ प्रयु॒क्तीन्द्रो॑\-ऽजायत।
यः का॒मये॑त वी॒रो म॒ आजा॑ये॒तेति॑।
स चतु॑र्‌\mbox{}होतारं॒ प्रयु॑ञ्जीत।
आऽस्य॑ वी॒रो जा॑यते।
सो॑ऽकामयत पशु॒मान्थ्स्या॒मिति॑।
स चतु॑र्‌\mbox{}होतुः॒ पञ्च॑होतारं॒ निर॑मिमीत।
तं प्रायु॑ङ्क्त।
तस्य॒ प्रयु॑क्ति पशु॒मान॑भवत्।
यः का॒मये॑त पशु॒मान्थ्स्या॒मिति॑।
स पञ्च॑होतारं॒ प्रयु॑ञ्जीत॥६९॥

%2.2.11.3
प॒शु॒माने॒व भ॑वति।
सो॑ऽकामयत॒र्तवो॑ मे कल्पेर॒न्निति॑।
स पञ्च॑होतुः॒ षड्ढो॑तारं॒ निर॑मिमीत।
तं प्रायु॑ङ्क्त।
तस्य॒ प्रयु॑क्त्यृ॒तवो᳚\-ऽस्मा अकल्पन्त।
यः का॒मये॑त॒र्तवो॑ मे कल्पेर॒न्निति॑।
स षड्ढो॑तारं॒ प्रयु॑ञ्जीत।
कल्प॑न्ते\-ऽस्मा ऋ॒तवः॑।
सो॑ऽकामयत सोम॒पः सो॑मया॒जी स्या᳚म्।
आ मे॑ सोम॒पः सो॑मया॒जी जा॑ये॒तेति॑॥७०॥

%2.2.11.4
स षड्ढो॑तुः स॒प्तहो॑तारं॒ निर॑मिमीत।
तं प्रायु॑ङ्क्त।
तस्य॒ प्रयु॑क्ति सोम॒पः सो॑मया॒ज्य॑भवत्।
आऽस्य॑ सोम॒पः सो॑मया॒ज्य॑जायत।
यः का॒मये॑त सोम॒पः सो॑मया॒जी स्या᳚म्।
आ मे॑ सोम॒पः सो॑मया॒जी जा॑ये॒तेति॑।
स स॒प्तहो॑तारं॒ प्रयु॑ञ्जीत।
सो॒म॒प ए॒व सो॑मया॒जी भ॑वति।
आऽस्य॑ सोम॒पः सो॑मया॒जी जा॑यते।
स वा ए॒ष प॒शुः प॑ञ्च॒धा प्रति॑ तिष्ठति॥७१॥

%2.2.11.5
प॒द्भिर्मुखे॑न।
ते दे॒वाः प॒शून् वि॒त्वा।
सु॒व॒र्गं लो॒कमा॑यन्।
ते॑ऽमुष्मिँ॑ल्लो॒के व्य॑क्षुध्यन्।
ते᳚ऽब्रुवन्।
अ॒मुतः॑ प्रदानं॒ वा उप॑जिजीवि॒मेति॑।
ते स॒प्तहो॑तारं य॒ज्ञं वि॒धाया॒यास्यम्᳚।
आ॒ङ्गी॒र॒सं प्राहि॑ण्वन्।
ए॒तेना॒मुत्र॑ कल्प॒येति॑।
तस्य॒ वा इ॒यं कॢप्तिः॑॥७२॥

%2.2.11.6
यदि॒दं किं च॑।
य ए॒वं वेद॑।
कल्प॑ते\-ऽस्मै।
स वा अ॒यं म॑नु॒ष्ये॑षु य॒ज्ञः स॒प्तहो॑ता।
अ॒मुत्र॑ स॒द्भ्यो दे॒वेभ्यो॑ ह॒व्यं व॑हति।
य ए॒वं वेद॑।
उपै॑नं य॒ज्ञो न॑मति।
यो वै चतु॑र्‌\mbox{}होतृणां नि॒दानं॒ वेद॑।
नि॒दान॑वान्भवति।
अ॒ग्नि॒हो॒त्रं वै दश॑होतुर्नि॒दानम्᳚।
द॒र्‌॒\mbox{}श॒पू॒र्ण॒मा॒सौ चतु॑र्‌\mbox{}होतुः।
चा॒तु॒र्मा॒स्यानि॒ पञ्च॑होतुः।
प॒शु॒ब॒न्धः षड्ढो॑तुः।
सौ॒म्यो᳚\-ऽध्व॒रः स॒प्तहो॑तुः।
ए॒तद्वै चतु॑र्‌\mbox{}होतृणां नि॒दानम्᳚।
य ए॒वं वेद॑।
नि॒दान॑वान्भवति॥७३॥\anuvakamend[अ॒मि॒मी॒त॒ तं प्रायु॑ङ्क्त॒ पञ्च॑होतारं॒ प्र यु॑ञ्जीत जाये॒तेति॑ तिष्ठति॒ कॢप्ति॒र्दश॑होतुर्नि॒दानꣳ॑ स॒प्त च॑]




\prashnaend{प्र॒जा\-प॑तिरकामयत प्र॒जाः सृ॑जे॒येति॑ प्र॒जा\-प॑तिरकामयत दर्‌\mbox{}शपूर्णमा॒सौ सृ॑जे॒येति॑ प्र॒जा\-प॑तिरकामयत॒ प्रजा॑ये॒येति॒ स तपः॒ स त्रि॒वृतं॑ प्र॒जा\-प॑तिरकामयत॒ दश॑होतारं॒ तेन॑ दश॒धा\-ऽऽत्मानं॑ दे॒वा वै वरु॑ण॒मन्तो॒ वै प्र॒जा\-प॑ति॒स्ताः सृ॒ष्टाः सम॑श्लिष्यं दे॒वा वै चतु॑र्‌\mbox{}होतृभिरि॒दं वा अग्रे᳚ प्र॒जा\-प॑ति॒रिन्द्रं॑ प्र॒जा\-प॑तिरकामयत ब॒होर्भूया॒नेका॑दश॥११॥}{प्र॒जा\-प॑ति॒स्तद्ग्रह॑स्य प्र॒जा\-प॑तिरकामयता॒नयै॒वैन॒त्तस्य॒ वा इ॒यं कॢप्ति॒स्तस्मा᳚त्तेपा॒नाज्ज्योति॒र्॒\-यद॒स्मिन्ना॑दि॒त्ये स षड्ढो॑तुः स॒प्तहो॑तारं॒ त्रिस॑प्ततिः॥७३॥}{प्र॒जा\-प॑तिरकामयत नि॒दान॑वान्भवति॥}{हरिः॑ ओम्॥}{इति श्रीकृष्णयजुर्वेदीयतैत्तिरीयब्राह्मणे द्वितीयाष्टके द्वितीयः प्रपाठकः समाप्तः॥}
\clearpage
\sect{तृतीयः प्रश्नः}
\setcounter{anuvakam}{0}
\dnsub{तैत्तिरीयब्राह्मणे द्वितीयाष्टके तृतीयः प्रपाठकः}

%2.3.1.1
ब्र॒ह्म॒वा॒दिनो॑ वदन्ति।
किं चतु॑र्‌\mbox{}होतृणां चतुर्‌\mbox{}होतृ॒त्वमिति॑।
यदे॒वैषु च॑तु॒र्धा होता॑रः।
तेन॒ चतु॑र्‌\mbox{}होतारः।
तस्मा॒च्चतु॑र्‌\mbox{}होतार उच्यन्ते।
तच्चतुर्॑होतृणां चतुर्‌\mbox{}होतृ॒त्वम्।
सोमो॒ वै चतु॑र्‌\mbox{}होता।
अ॒ग्निः पञ्च॑होता।
धा॒ता षड्ढो॑ता।
इन्द्रः॑ स॒प्तहो॑ता॥१॥

%2.3.1.2
प्र॒जा\-प॑ति॒र्दश॑होता।
य ए॒वं चतु॑र्‌\mbox{}होतृणा॒मृद्धिं॒ वेद॑।
ऋ॒ध्नोत्ये॒व।
य ए॑षामे॒वं ब॒न्धुतां॒ वेद॑।
बन्धु॑मान्भवति।
य ए॑षामे॒वं कॢप्तिं॒ वेद॑।
कल्प॑ते\-ऽस्मै।
य ए॑षामे॒वमा॒यत॑नं॒ वेद॑।
आ॒यत॑नवान्भवति।
य ए॑षामे॒वं प्र॑ति॒ष्ठां वेद॑॥२॥

%2.3.1.3
प्रत्ये॒व ति॑ष्ठति।
ब्र॒ह्म॒वा॒दिनो॑ वदन्ति।
दश॑होता॒ चतु॑र्‌\mbox{}होता।
पञ्च॑होता॒ षड्ढो॑ता स॒प्तहो॑ता।
अथ॒ कस्मा॒च्चतु॑र्‌\mbox{}होतार उच्यन्त॒ इति॑।
इन्द्रो॒ वै चतु॑र्‌\mbox{}होता।
इन्द्रः॒ खलु॒ वै श्रेष्ठो॑ दे॒वता॑नामुप॒\-देश॑नात्।
य ए॒वमिन्द्र॒ꣴ॒ श्रेष्ठं॑ दे॒वता॑नामुप॒\-देश॑ना॒द्वेद॑।
वसि॑ष्ठः समा॒नानां᳚ भवति।
तस्मा॒च्छ्रेष्ठ॑मा॒यन्तं॑ प्रथ॒मेनै॒वानु॑ बुध्यन्ते।
अ॒यमागन्॑।
अ॒यमवा॑सा॒दिति॑।
की॒र्तिर॑स्य॒ पूर्वा\-ऽऽग॑च्छति ज॒नता॑माय॒तः।
अथो॑ एनं प्रथ॒मेनै॒वानु॑ बुध्यन्ते।
अ॒यमागन्॑।
अ॒यमवा॑सा॒दिति॑॥३॥\anuvakamend[स॒प्तहो॑ता प्रति॒ष्ठां वेद॑ बुध्यन्ते॒ षट्च॑]

%2.3.2.1
दक्षि॑णां प्रतिग्रही॒ष्यन्थ्स॒प्तद॑श॒कृत्वो\-ऽपा᳚न्यात्।
आ॒त्मान॑मे॒व समि॑न्धे।
तेज॑से वी॒र्या॑य।
अथो᳚ प्र॒जा\-प॑तिरे॒वैनां᳚ भू॒त्वा प्रति॑\-गृह्णाति।
आ॒त्मनो\-ऽना᳚र्त्यै।
यद्ये॑न॒मार्त्वि॑ज्याद्वृ॒तꣳ सन्तं॑ नि॒र्‌॒\mbox{}हरे॑रन्।
आग्नी᳚ध्रे जुहुया॒द्दश॑होतारम्।
च॒तु॒र्गृ॒ही॒तेनाऽऽज्ये॑न।
पु॒रस्ता᳚त्प्र॒त्यङ्तिष्ठन्॑।
प्र॒ति॒लो॒मं वि॒ग्राहम्᳚॥४॥

%2.3.2.2
प्रा॒णाने॒वास्योप॑ दासयति।
यद्ये॑नं॒ पुन॑रुप॒ शिक्षे॑युः।
आग्नी᳚ध्र ए॒व जु॑हुया॒द्दश॑होतारम्।
च॒तु॒र्गृ॒ही॒तेनाऽऽज्ये॑न।
प॒श्चात्प्राङासी॑नः।
अ॒नु॒लो॒ममवि॑ग्राहम्।
प्रा॒णाने॒वास्मै॑ कल्पयति।
प्राय॑श्चित्ती॒ वाग्घोतेत्यृ॑तुमु॒खऋ॑तुमुखे जुहोति।
ऋ॒तूने॒वास्मै॑ कल्पयति।
कल्प॑न्ते\-ऽस्मा ऋ॒तवः॑॥५॥

%2.3.2.3
कॢ॒प्ता अ॑स्मा ऋ॒तव॒ आय॑न्ति।
षड्ढो॑ता॒ वै भू॒त्वा प्र॒जा\-प॑तिरि॒दꣳ सर्व॑म\-सृजत।
स मनो॑\-ऽ\-सृजत।
मन॒सोऽधि॑ गाय॒त्रीम॑\-सृजत।
तद्गा॑य॒त्रीं यश॑ आर्च्छत्।
तामाऽल॑भत।
गा॒य॒त्रि॒या अधि॒ छन्दाꣴ॑स्य\-सृजत।
छन्दो॒भ्योऽधि॒ साम॑।
तथ्साम॒ यश॑ आर्च्छत्।
तदाऽल॑भत॥६॥

%2.3.2.4
साम्नोऽधि॒ यजूꣴ॑ष्य\-सृजत।
यजु॒र्भ्योऽधि॒ विष्णुम्᳚।
तद्विष्णुं॒ यश॑ आर्च्छत्।
तमाऽल॑भत।
विष्णो॒रध्योष॑धीर\-सृजत।
ओष॑धी॒भ्योऽधि॒ सोमम्᳚।
तथ्सोमं॒ यश॑ आर्च्छत्।
तमाऽल॑भत।
सोमा॒दधि॑ प॒शून॑\-सृजत।
प॒शुभ्यो\-ऽधीन्द्रम्᳚॥७॥

%2.3.2.5
तदिन्द्रं॒ यश॑ आर्च्छत्।
तदे॑नं॒ नाति॒ प्राच्य॑वत।
इन्द्र॑ इव यश॒स्वी भ॑वति।
य ए॒वं वेद॑।
नैनं॒ यशोऽति॒ प्रच्य॑वते।
यद्वा इ॒दं किं च॑।
तथ्सर्व॑मुत्ता॒न ए॒वा\-ऽऽङ्गी॑र॒सः प्रत्य॑गृह्णात्।
तदे॑नं॒ प्रति॑\-गृहीतं॒ नाहि॑नत्।
यत्किं च॑ प्रति\-गृह्णी॒यात्।
तथ्सर्व॑मुत्ता॒नस्त्वा᳚\-ऽऽङ्गीर॒सः प्रति॑\-गृह्णा॒त्वित्ये॒व प्रति॑\-गृह्णीयात्।
इ॒यं वा उ॑त्ता॒न आ᳚ङ्गीर॒सः।
अ॒नयै॒वैन॒त्प्रति॑\-गृह्णाति।
नैनꣳ॑ हिनस्ति।
ब॒र्॒हिषा॒ प्रती॑या॒द्गां वाऽश्वं॑ वा।
ए॒तद्वै प॑शू॒नां प्रि॒यं धाम॑।
प्रि॒येणै॒वैनं॒ धाम्ना॒ प्रत्ये॑ति॥८॥\anuvakamend[वि॒ग्राह॑मृ॒तव॒स्तदा\-ऽल॑भ॒तेन्द्रं॑ गृह्णीया॒थ्षट्च॑]

%2.3.3.1
यो वा अवि॑द्वान्निव॒र्तय॑ते।
विशी॑र्‌\mbox{}षा॒ सपा᳚प्मा॒\-ऽमुष्मिँ॑ल्लो॒के भ॑वति।
अथ॒ यो वि॒द्वान्नि॑व॒र्तय॑ते।
सशी॑र्‌\mbox{}षा॒ विपा᳚प्मा॒\-ऽमुष्मिँ॑ल्लो॒के भ॑वति।
दे॒वता॒ वै स॒प्त पुष्टि॑कामा॒ न्य॑वर्तयन्त।
अ॒ग्निश्च॑ पृथि॒वी च॑।
वा॒युश्चा॒न्तरि॑क्षं च।
आ॒दि॒त्यश्च॒ द्यौश्च॑ च॒न्द्रमाः᳚।
अ॒ग्निर्न्य॑वर्तयत।
स सा॑ह॒स्रम॑पुष्यत्॥९॥

%2.3.3.2
पृ॒थि॒वी न्य॑वर्तयत।
सौष॑धीभि॒र्वन॒स्पति॑भि\-र\-पुष्यत्।
वा॒युर्न्य॑\-वर्तयत।
स मरी॑चीभिरपुष्यत्।
अ॒न्तरि॑क्षं॒ न्य॑वर्तयत।
तद्वयो॑भिर\-पुष्यत्।
आ॒दि॒त्यो न्य॑वर्तयत।
स र॒श्मिभि॑रपुष्यत्।
द्यौर्न्य॑\-वर्तयत।
सा नक्ष॑त्रैरपुष्यत्।
च॒न्द्रमा॒ न्य॑वर्तयत।
सो॑ऽहोरा॒त्रैर॑र्ध\-मा॒सैर्मासैर्॑\-ऋ॒तुभिः॑ संवथ्स॒रेणा॑पुष्यत्।
तान्पोषा᳚न्पुष्यति।
याꣴस्ते\-ऽपु॑ष्यन्।
य ए॒वं वि॒द्वान्नि च॑ व॒र्तय॑ते॒ परि॑ च॥१०॥\anuvakamend[अ॒पु॒ष्य॒न्नक्ष॑त्रैरपुष्य॒त्पञ्च॑ च]

%2.3.4.1
तस्य॒ वा अ॒ग्नेर्‌\mbox{}हिर॑ण्यं प्रतिजग्र॒हुषः॑।
अ॒र्धमि॑न्द्रि॒य\-स्यापा᳚\-क्रामत्।
तदे॒तेनै॒व प्रत्य॑गृह्णात्।
तेन॒ वै सो᳚\-ऽर्धमि॑न्द्रि॒य\-स्या॒\-ऽऽत्मन्नु॒पा\-ध॑त्त।
अ॒र्धमि॑न्द्रि॒य\-स्या॒\-ऽऽत्मन्नु॒पा\-ध॑त्ते।
य ए॒वं वि॒द्वान् हिर॑ण्यं प्रति\-गृ॒ह्णाति॑।
अथ॒ योऽवि॑द्वान्प्रति\-गृ॒ह्णाति॑।
अ॒र्धम॑स्येन्द्रि॒य\-स्याप॑\-क्रामति।
तस्य॒ वै सोम॑स्य॒ वासः॑ प्रतिजग्र॒हुषः॑।
तृती॑यमिन्द्रि॒य\-स्यापा᳚क्रामत्॥११॥

%2.3.4.2
तदे॒तेनै॒व प्रत्य॑गृह्णात्।
तेन॒ वै स तृती॑यमिन्द्रि॒य\-स्या॒\-ऽऽत्मन्नु॒पा\-ध॑त्त।
तृती॑यमिन्द्रि॒य\-स्या॒\-ऽऽत्मन्नु॒पा\-ध॑त्ते।
य ए॒वं वि॒द्वान् वासः॑ प्रति\-गृ॒ह्णाति॑।
अथ॒ योऽवि॑द्वान्प्रति\-गृ॒ह्णाति॑।
तृती॑यमस्येन्द्रि॒य\-स्याप॑\-क्रामति।
तस्य॒ वै रु॒द्रस्य॒ गां प्र॑तिजग्र॒हुषः॑।
च॒तु॒र्थमि॑न्द्रि॒य\-स्यापा᳚\-क्रामत्।
तामे॒तेनै॒व प्रत्य॑गृह्णात्।
तेन॒ वै स च॑तु॒र्थमि॑न्द्रि॒य\-स्या॒\-ऽऽत्मन्नु॒पा\-ध॑त्त॥१२॥

%2.3.4.3
च॒तु॒र्थमि॑न्द्रि॒य\-स्या॒\-ऽऽत्मन्नु॒पा\-ध॑त्ते।
य ए॒वं वि॒द्वान्गां प्र॑तिगृ॒ह्णाति॑।
अथ॒ योऽवि॑द्वान्प्रति\-गृ॒ह्णाति॑।
च॒तु॒र्थम॑स्येन्द्रि॒य\-स्याप॑\-क्रामति।
तस्य॒ वै वरु॑ण॒स्याश्वं॑ प्रतिजग्र॒हुषः॑।
प॒ञ्च॒ममि॑\-न्द्रि॒य\-स्यापा᳚\-क्रामत्।
तमे॒तेनै॒व प्रत्य॑गृह्णात्।
तेन॒ वै स प॑ञ्च॒ममि॑न्द्रि॒य\-स्या॒\-ऽऽत्मन्नु॒पा\-ध॑त्त।
प॒ञ्च॒ममि॑\-न्द्रि॒य\-स्या॒\-ऽऽत्मन्नु॒पा\-ध॑त्ते।
य ए॒वं वि॒द्वानश्वं॑ प्रति\-गृ॒ह्णाति॑॥१३॥

%2.3.4.4
अथ॒ योऽवि॑द्वान्प्रति\-गृ॒ह्णाति॑।
प॒ञ्च॒मम॑स्येन्द्रि॒य\-स्याप॑\-क्रामति।
तस्य॒ वै प्र॒जा\-प॑तेः॒ पुरु॑षं प्रति\-जग्र॒हुषः॑।
ष॒ष्ठमि॑न्द्रि॒य\-स्यापा᳚क्रामत्।
तमे॒तेनै॒व प्रत्य॑गृह्णात्।
तेन॒ वै स ष॒ष्ठमि॑न्द्रि॒य\-स्या॒\-ऽऽत्मन्नु॒पा\-ध॑त्त।
ष॒ष्ठमि॑न्द्रि॒य\-स्या॒\-ऽऽत्मन्नु॒पा\-ध॑त्ते।
य ए॒वं वि॒द्वान्पुरु॑षं प्रति\-गृ॒ह्णाति॑।
अथ॒ योऽवि॑द्वान्प्रति\-गृ॒ह्णाति॑।
ष॒ष्ठम॑स्येन्द्रि॒य\-स्याप॑\-क्रामति॥१४॥

%2.3.4.5
तस्य॒ वै मनो॒स्तल्पं॑ प्रति\-जग्र॒हुषः॑।
स॒प्त॒ममि॑न्द्रि॒य\-स्यापा᳚\-क्रामत्।
तमे॒तेनै॒व प्रत्य॑\-गृह्णात्।
तेन॒ वै स स॑प्त॒ममि॑न्द्रि॒य\-स्या॒\-ऽऽत्मन्नु॒पा\-ध॑त्त।
स॒प्त॒ममि॑न्द्रि॒य\-स्या॒\-ऽऽत्मन्नु॒पा\-ध॑त्ते।
य ए॒वं वि॒द्वाꣴस्तल्पं॑ प्रति\-गृ॒ह्णाति॑।
अथ॒ यो\-ऽवि॑द्वान्प्रति\-गृ॒ह्णाति॑।
स॒प्त॒मम॑स्येन्द्रि॒य\-स्याप॑\-क्रामति।
तस्य॒ वा उ॑त्ता॒नस्या᳚\-ऽऽङ्गीर॒सस्याप्रा॑णत्प्रति\-जग्र॒हुषः॑।
अ॒ष्ट॒ममि॑न्द्रि॒य\-स्यापा᳚क्रामत्॥१५॥

%2.3.4.6
तदे॒तेनै॒व प्रत्य॑गृह्णात्।
तेन॒ वै सो᳚\-ऽष्ट॒ममि॑न्द्रि॒य\-स्या॒\-ऽऽत्मन्नु॒पा\-ध॑त्त।
अ॒ष्ट॒ममि॑न्द्रि॒य\-स्या॒\-ऽऽत्मन्नु॒पा\-ध॑त्ते।
य ए॒वं वि॒द्वानप्रा॑णत्प्रति\-गृ॒ह्णाति॑।
अथ॒ योऽवि॑द्वान्प्रति\-गृ॒ह्णाति॑।
अ॒ष्ट॒मम॑स्येन्द्रि॒य\-स्याप॑\-क्रामति।
यद्वा इ॒दं किं च॑।
तथ्सर्व॑मुत्ता॒न ए॒वा\-ऽऽङ्गी॑र॒सः प्रत्य॑\-गृह्णात्।
तदे॑नं॒ प्रति॑\-गृहीतं॒ नाहि॑नत्।
यत्किं च॑ प्रति\-गृह्णी॒यात्।
तथ्सर्व॑मुत्ता॒नस्त्वा᳚\-ऽऽङ्गीर॒सः प्रति॑\-गृह्णा॒त्वित्ये॒व प्रति॑\-गृह्णीयात्।
इ॒यं वा उ॑त्ता॒न आ᳚ङ्गीर॒सः।
अ॒नयै॒वैन॒त्प्रति॑\-गृह्णाति।
नैनꣳ॑ हिनस्ति॥१६॥\anuvakamend[तृती॑यमिन्द्रि॒य\-स्यापा᳚क्रामच्चतु॒र्थमि॑न्द्रि॒य\-स्या॒त्मन्नु॒पाध॒त्ताश्वं॑ प्रति\-गृ॒ह्णाति॑ ष॒ष्ठम॑स्येन्द्रि॒य\-स्याप॑क्रामत्यष्ट॒ममि॑\-न्द्रि॒य\-स्यापा᳚क्रामत्प्रति\-गृह्णी॒याच्च॒त्वारि॑ च (तस्य॒ वा अ॒ग्नेर्‌\mbox{}हिर॑ण्य॒ꣳ॒ सोम॑स्य॒ वास॒स्तदे॒तेन॑ रु॒द्रस्य॒ गान्तामे॒तेन॒ वरु॑ण॒स्याश्वं॑ प्र॒जा\-प॑तेः॒ पुरु॑षं॒ मनो॒स्तल्प॒न्तमे॒तेनो᳚त्ता॒नस्य॒ तदे॒तेनाप्रा॑ण॒द्यद्वै।
अ॒र्धं तृती॑यमष्ट॒मं तच्च॑तु॒र्थं तां प॑ञ्च॒मꣳ ष॒ष्ठꣳ स॑प्त॒मन्तम्।
तदे॒तेन॒ द्वे तामे॒तेनैकं॒ तमे॒तेन॒ त्रीणि॒ तदे॒तेनैकम्᳚॥)]

%2.3.5.1
ब्र॒ह्म॒वा॒दिनो॑ वदन्ति।
यद्दश॑होतारः स॒त्रमास॑त।
केन॒ ते गृ॒हप॑तिना\-ऽऽर्ध्नुवन्।
केन॑ प्र॒जा अ॑सृज॒न्तेति॑।
प्र॒जा\-प॑तिना॒ वै ते गृ॒हप॑तिना\-ऽऽर्ध्नुवन्।
तेन॑ प्र॒जा अ॑सृजन्त।
यच्चतु॑र्‌\mbox{}होतारः स॒त्रमास॑त।
केन॒ ते गृ॒हप॑तिना\-ऽऽर्ध्नुवन्।
केनौष॑धीरसृज॒न्तेति॑।
सोमे॑न॒ वै ते गृ॒हप॑तिना\-ऽऽर्ध्नुवन्॥१७॥

%2.3.5.2
तेनौष॑धीरसृजन्त।
यत्पञ्च॑होतारः स॒त्रमास॑त।
केन॒ ते गृ॒हप॑तिना\-ऽऽर्ध्नुवन्।
केनै॒भ्यो लो॒केभ्यो\-ऽसु॑रा॒न्प्राणु॑दन्त।
केनै॑षां प॒शून॑वृञ्ज॒तेति॑।
अ॒ग्निना॒ वै ते गृ॒हप॑तिना\-ऽऽर्ध्नुवन्।
तेनै॒भ्यो लो॒केभ्यो\-ऽसु॑रा॒न्प्राणु॑दन्त।
तेनै॑षां प॒शून॑वृञ्जत।
यथ्षड्ढो॑तारः स॒त्रमास॑त।
केन॒ ते गृ॒हप॑तिना\-ऽऽर्ध्नुवन्॥१८॥

%2.3.5.3
केन॒र्तून॑कल्पय॒न्तेति॑।
धा॒त्रा वै ते गृ॒हप॑तिना\-ऽऽर्ध्नुवन्।
तेन॒र्तून॑कल्पयन्त।
यथ्स॒प्तहो॑तारः स॒त्रमास॑त।
केन॒ ते गृ॒हप॑तिना\-ऽऽर्ध्नुवन्।
केन॒ सुव॑रायन्।
केने॒माँल्लो॒कान्थ्सम॑\-तन्व॒न्निति॑।
अ॒र्य॒म्णा वै ते गृ॒हप॑तिना\-ऽऽर्ध्नुवन्।
तेन॒ सुव॑रायन्।
तेने॒माँल्लो॒कान्थ्सम॑तन्व॒न्निति॑॥१९॥

%2.3.5.4
ए॒ते वै दे॒वा गृ॒हप॑तयः।
तान् य ए॒वं वि॒द्वान्।
अप्य॒न्यस्य॑ गार्‌\mbox{}हप॒ते दीक्ष॑ते।
अ॒वा॒न्त॒रमे॒व स॒त्रिणा॑मृध्नोति।
यो वा अ॑र्य॒मणं॒ वेद॑।
दान॑कामा अस्मै प्र॒जा भ॑वन्ति।
य॒ज्ञो वा अ॑र्य॒मा।
आर्या॑वस॒तिरिति॒ वै तमा॑हु॒र्यं प्र॒शꣳस॑न्ति।
आर्या॑वस॒तिर्भ॑वति।
य ए॒वं वेद॑॥२०॥

%2.3.5.5
यद्वा इ॒दं किं च॑।
तथ्सर्वं॒ चतु॑र्\mbox{}होतारः।
चतु॑र्‌\mbox{}होतृ॒भ्योऽधि॑ य॒ज्ञो निर्मि॑तः।
स य ए॒वं वि॒द्वान्‌ वि॒वदे॑त।
अ॒हमे॒व भूयो॑ वेद।
यश्चतु॑र्‌\mbox{}होतॄ॒न् वेदेति॑।
स ह्ये॑व भूयो॒ वेद॑।
यश्चतु॑र्‌\mbox{}होतॄ॒न् वेद॑।
यो वै चतु॑र्‌\mbox{}होतृणा॒ꣳ॒ होतॄ॒न् वेद॑।
सर्वा॑सु प्र॒जास्वन्न॑मत्ति॥२१॥

%2.3.5.6
सर्वा॒ दिशो॒ऽभि ज॑यति।
प्र॒जा\-प॑ति॒र्वै दश॑होतृणा॒ꣳ॒ होता᳚।
सोम॒श्चतु॑र्‌\mbox{}होतृणा॒ꣳ॒ होता᳚।
अ॒ग्निः पञ्च॑होतृणा॒ꣳ॒ होता᳚।
धा॒ता षड्ढो॑तृणा॒ꣳ॒ होता᳚।
अ॒र्य॒मा स॒प्तहो॑तृणा॒ꣳ॒ होता᳚।
ए॒ते वै चतु॑र्\mbox{}होतृणा॒ꣳ॒ होता॑रः।
तान् य ए॒वं वेद॑।
सर्वा॑सु प्र॒जास्वन्न॑मत्ति।
सर्वा॒ दिशो॒ऽभि ज॑यति॥२२॥\anuvakamend[आ॒र्ध्नु॒व॒न्ना॒र्ध्नु॒व॒न्नित्ये॒वं वेदा᳚त्ति सर्वा॒ दिशो॒ऽभि ज॑यति (वै तेन॑ स॒त्रङ्केन॑॥)]

%2.3.6.1
प्र॒जा\-प॑तिः प्र॒जाः सृ॒ष्ट्वा व्य॑स्रꣳसत।
स हृद॑यं भू॒तो॑\-ऽशयत्।
आत्म॒न्॒ हा (३) इत्यह्व॑यत्।
आपः॒ प्रत्य॑शृण्वन्।
ता अ॑ग्निहो॒त्रेणै॒व य॑ज्ञक्र॒तुनोप॑ प॒र्याव॑र्तन्त।
ताः कुसि॑न्ध॒मुपौ॑हन्।
तस्मा॑दग्निहो॒त्रस्य॑ यज्ञक्र॒तोः।
एक॑ ऋ॒त्विक्।
च॒तु॒ष्कृत्वो\-ऽह्व॑यत्।
अ॒ग्निर्वा॒युरा॑दि॒त्यश्च॒न्द्रमाः᳚॥२३॥

%2.3.6.2
ते प्रत्य॑शृण्वन्।
ते द॑र्‌\mbox{}शपूर्णमा॒साभ्या॑मे॒व य॑ज्ञक्र॒तुनोप॑ प॒र्या\-व॑र्तन्त।
त उपौ॑हꣴश्च॒त्वार्यङ्गा॑नि।
तस्मा᳚द्दर्‌\mbox{}शपूर्ण\-मा॒सयो᳚र्यज्ञक्र॒तोः।
च॒त्वार॑ ऋ॒त्विजः॑।
प॒ञ्च॒कृत्वो\-ऽह्व॑यत्।
प॒शवः॒ प्रत्य॑\-शृण्वन्।
ते चा॑तुर्मा॒स्यैरे॒व य॑ज्ञक्र॒तुनोप॑ प॒र्याव॑र्तन्त।
त उपौ॑हं॒ लोम॑ छ॒वीं मा॒ꣳ॒समस्थि॑ म॒ज्जानम्᳚।
तस्मा᳚च्चातुर्मा॒स्यानां᳚ यज्ञक्र॒तोः॥२४॥

%2.3.6.3
पञ्च॒र्त्विजः॑।
ष॒ट्कृत्वो\-ऽह्व॑यत्।
ऋ॒तवः॒ प्रत्य॑शृण्वन्।
ते प॑शुब॒न्धेनै॒व य॑ज्ञक्र॒तुनोप॑प॒र्याव॑र्तन्त।
त उपौ॑ह॒न्थ्स्तना॑वा॒ण्डौ शि॒श्ञमवा᳚ञ्चं प्रा॒णम्।
तस्मा᳚त्पशुब॒न्धस्य॑ यज्ञक्र॒तोः।
षडृ॒त्विजः॑।
स॒प्त॒कृत्वो\-ऽह्व॑यत्।
होत्राः॒ प्रत्य॑शृण्वन्।
ताः सौ॒म्येनै॒वाध्व॒रेण॑ यज्ञक्र॒तुनोप॑प॒र्याव॑र्तन्त॥२५॥

%2.3.6.4
ता उपौ॑हन्थ्स॒प्त शी॑र्‌\mbox{}ष॒ण्या᳚न्प्रा॒णान्।
तस्मा᳚थ्सौ॒म्यस्या᳚ध्व॒रस्य॑ यज्ञक्र॒तोः।
स॒प्त होत्राः॒ प्राची॒र्वष॑ट्कुर्वन्ति।
द॒श॒कृत्वो\-ऽह्व॑यत्।
तपः॒ प्रत्य॑शृणोत्।
तत्कर्म॑णै॒व सं॑वथ्स॒रेण॒ सर्वै᳚र्यज्ञक्र॒तुभि॒रुप॑ प॒र्याव॑र्तत।
तथ्सर्व॑मा॒त्मान॒मप॑रिवर्ग॒मुपौ॑हत्।
तस्मा᳚थ्संवथ्स॒रे सर्वे॑ यज्ञक्र॒तवो\-ऽव॑रुध्यन्ते।
तस्मा॒द्दश॑होता॒ चतु॑र्‌\mbox{}होता।
पञ्च॑होता॒ षड्ढो॑ता स॒प्तहो॑ता।
एक॑होत्रे ब॒लिꣳ ह॑रन्ति।
हर॑न्त्यस्मै प्र॒जा ब॒लिम्।
ऐन॒मप्र॑तिख्यातं गच्छति।
य ए॒वं वेद॑॥२६॥\anuvakamend[च॒न्द्रमा᳚श्चातुर्मा॒स्यानां᳚ यज्ञक्र॒तोर॑ध्व॒रेण॑ यज्ञक्र॒तुनोप॑ प॒र्याव॑र्तन्त स॒प्तहो॑ता च॒त्वारि॑ च]

%2.3.7.1
प्र॒जा\-प॑तिः॒ पुरु॑षम\-सृजत।
सो᳚ऽग्निर॑ब्रवीत्।
ममा॒यमन्न॑\-म॒स्त्विति॑।
सो॑ऽबिभेत्।
सर्वं॒ वै मा॒ऽयं प्र ध॑क्ष्य॒तीति॑।
स ए॒ताꣴ\-श्चतु॑र्\mbox{}होतॄनात्म॒स्पर॑णानपश्यत्।
तान॑जुहोत्।
तैर्वै स आ॒त्मान॑\-मस्पृणोत्।
यद॑ग्निहो॒त्रं जु॒होति॑।
एक॑होतारमे॒व तद्य॑ज्ञ\-क्र॒तुमा᳚प्नोत्यग्नि\-हो॒त्रम्॥२७॥

%2.3.7.2
कुसि॑न्धं चा॒ऽऽत्मनः॑ स्पृ॒णोति॑।
आ॒दि॒त्यस्य॑ च॒ सायु॑ज्यं गच्छति।
च॒तुरुन्न॑यति।
चतु॑र्‌\mbox{}होतारमे॒व तद्य॑ज्ञक्र॒तुमा᳚प्नोति दर्‌\mbox{}श\-पूर्ण\-मा॒सौ।
च॒त्वारि॑ चा॒ऽऽत्मनो\-ऽङ्गा॑नि स्पृ॒णोति॑।
आ॒दि॒त्यस्य॑ च॒ सायु॑ज्यं गच्छति।
च॒तुरुन्न॑यति।
स॒मित्प॑ञ्च॒मी।
पञ्च॑होतारमे॒व तद्य॑ज्ञ\-क्र॒तुमा᳚प्नोति चातुर्मा॒स्यानि॑।
लोम॑ छ॒वीं मा॒ꣳ॒समस्थि॑ म॒ज्जानम्᳚॥२८॥

%2.3.7.3
तानि॑ चा॒ऽऽत्मनः॑ स्पृ॒णोति॑।
आ॒दि॒त्यस्य॑ च॒ सायु॑ज्यं गच्छति।
च॒तुरुन्न॑यति।
द्विर्जु॑होति।
षड्ढो॑तारमे॒व तद्य॑ज्ञक्र॒तुमा᳚प्नोति पशुब॒न्धम्।
स्तना॑वा॒ण्डौ शि॒श्ञमवा᳚ञ्चं प्रा॒णम्।
तानि॑ चा॒ऽऽत्मनः॑ स्पृ॒णोति॑।
आ॒दि॒त्यस्य॑ च॒ सायु॑ज्यं गच्छति।
च॒तुरुन्न॑यति।
द्विर्जु॑होति॥२९॥

%2.3.7.4
स॒मिथ्स॑प्त॒मी।
स॒प्तहो॑तारमे॒व तद्य॑ज्ञक्र॒तुमा᳚प्नोति सौ॒म्यम॑ध्व॒रम्।
स॒प्त चा॒ऽऽत्मनः॑ शीर्\mbox{}ष॒ण्या᳚न्प्रा॒णान्थ्स्पृ॒णोति॑।
आ॒दि॒त्यस्य॑ च॒ सायु॑ज्यं गच्छति।
च॒तुरुन्न॑यति।
द्विर्जु॒होति॑।
द्विर्निमा᳚र्ष्टि।
द्विः प्राश्ञा॑ति।
दश॑होतारमे॒व तद्य॑ज्ञक्र॒तुमा᳚प्नोति संवथ्स॒रम्।
सर्वं॑ चा॒\-ऽ\-ऽ\-त्मान॒\-मप॑रि\-वर्गꣴ स्पृ॒णोति॑।
आ॒दि॒त्यस्य॑ च॒ सायु॑ज्यं गच्छति॥३०॥\anuvakamend[अ॒ग्नि॒हो॒त्रं म॒ज्जान॒न्द्विर्जु॑हो॒त्यप॑रिवर्गꣴ स्पृ॒णोत्येकं॑ च]

%2.3.8.1
प्र॒जा\-प॑तिरकामयत॒ प्रजा॑ये॒येति॑।
स तपो॑\-ऽतप्यत।
सो᳚ऽन्तर्वा॑नभवत्।
स हरि॑तः श्या॒वो॑\-ऽभवत्।
तस्मा॒थ्स्त्र्य॑न्तर्व॑त्नी।
हरि॑णी स॒ती श्या॒वा भ॑वति।
स वि॒जाय॑मानो॒ गर्भे॑णाताम्यत्।
स ता॒न्तः कृ॒ष्णः श्या॒वो॑\-ऽभवत्।
तस्मा᳚त्ता॒न्तः कृ॒ष्णः श्या॒वो भ॑वति।
तस्यासु॑रे॒वाजी॑वत्॥३१॥

%2.3.8.2
तेनासु॒ना\-ऽसु॑रान\-सृजत।
तदसु॑राणा\-मसुर॒\-त्वम्।
य ए॒वम\-सु॑राणा\-मसुर॒\-त्वं वेद॑।
असु॑माने॒व भ॑वति।
नैन॒\-मसु॑र्जहाति।
सो\-ऽ\-सु॑रान्थ्सृ॒ष्ट्वा पि॒तेवा॑\-मन्यत।
तदनु॑ पि॒तॄन॑\-सृजत।
तत्पि॑तृ॒णां पि॑तृ॒\-त्वम्।
य ए॒वं पि॑तृ॒णां पि॑तृ॒\-त्वं वेद॑।
पि॒तेवै॒व स्वानां᳚ भवति॥३२॥

%2.3.8.3
यन्त्य॑स्य पि॒तरो॒ हवम्᳚।
स पि॒तॄन्थ्सृ॒ष्ट्वा\-ऽऽम॑नस्यत्।
तदनु॑ मनु॒ष्या॑न\-सृजत।
तन्म॑नु॒ष्या॑णां मनुष्य॒त्वम्।
य ए॒वं म॑नु॒ष्या॑णां मनुष्य॒त्वं वेद॑।
म॒न॒स्व्ये॑व भ॑वति।
नैनं॒ मनु॑र्जहाति।
तस्मै॑ मनु॒ष्या᳚न्थ्ससृजा॒नाय॑।
दिवा॑ देव॒त्रा\-ऽभ॑वत्।
तदनु॑ दे॒वान॑\-सृजत।
तद्दे॒वानां᳚ देव॒त्वम्।
य ए॒वं दे॒वानां᳚ देव॒त्वं वेद॑।
दिवा॑ है॒वास्य॑ देव॒त्रा भ॑वति।
तानि॒ वा ए॒तानि॑ च॒त्वार्यम्भाꣳ॑सि।
दे॒वा म॑नु॒ष्याः᳚ पि॒तरो\-ऽसु॑राः।
तेषु॒ सर्वे॒ष्वम्भो॒ नभ॑ इव भवति।
य ए॒वं वेद॑॥३३॥\anuvakamend[अ॒जी॒व॒थ्स्वानां᳚ भवति दे॒वान॑\-सृजत स॒प्त च॑]

%2.3.9.1
ब्र॒ह्म॒वा॒दिनो॑ वदन्ति।
यो वा इ॒मं वि॒द्यात्।
यतो॒ऽयं पव॑ते।
यद॑भि॒ पव॑ते।
यद॑भि स॒म्पव॑ते।
सर्व॒मायु॑रियात्।
न पु॒रा\-ऽऽयु॑षः॒ प्र मी॑येत।
प॒शु॒मान्थ्स्या᳚त्।
वि॒न्देत॑ प्र॒जाम्।
यो वा इ॒मं वेद॑॥३४॥

%2.3.9.2
यतो॒ऽयं पव॑ते।
यद॑भि॒ पव॑ते।
यद॑भि स॒म्पव॑ते।
सर्व॒मायु॑रेति।
न पु॒रा\-ऽऽयु॑षः॒ प्र मी॑यते।
प॒शु॒मान्भ॑वति।
वि॒न्दते᳚ प्र॒जाम्।
अ॒द्भ्यः प॑वते।
अ॒पो॑ऽभि प॑वते।
अ॒पो॑ऽभि सम्प॑वते॥३५॥

%2.3.9.3
अ॒स्याः प॑वते।
इ॒माम॒भि प॑वते।
इ॒माम॒भि सम्प॑वते।
अ॒ग्नेः प॑वते।
अ॒ग्निम॒भि प॑वते।
अ॒ग्निम॒भि सम्प॑वते।
अ॒न्तरि॑क्षात्पवते।
अ॒न्तरि॑क्षम॒भि प॑वते।
अ॒न्तरि॑क्षम॒भि सम्प॑वते।
आ॒दि॒त्यात्प॑वते॥३६॥

%2.3.9.4
आ॒दि॒त्यम॒भि प॑वते।
आ॒दि॒त्यम॒भि सम्प॑वते।
द्योः प॑वते।
दिव॑म॒भि प॑वते।
दिव॑म॒भि सम्प॑वते।
दि॒ग्भ्यः प॑वते।
दिशो॒ऽभि प॑वते।
दिशो॒ऽभि सम्प॑वते।
स यत्पु॒रस्ता॒द्वाति॑।
प्रा॒ण ए॒व भू॒त्वा पु॒रस्ता᳚द्वाति॥३७॥

%2.3.9.5
तस्मा᳚त्पु॒रस्ता॒द्वान्तम्᳚।
सर्वाः᳚ प्र॒जाः प्रति॑ नन्दन्ति।
प्रा॒णो हि प्रि॒यः प्र॒जाना᳚म्।
प्रा॒ण इ॑व प्रि॒यः प्र॒जानां᳚ भवति।
य ए॒वं वेद॑।
स वा ए॒ष प्रा॒ण ए॒व।
अथ॒ यद्द॑क्षिण॒तो वाति॑।
मा॒त॒रिश्वै॒व भू॒त्वा द॑क्षिण॒तो वा॑ति।
तस्मा᳚द्दक्षिण॒तो वान्तं॑ वि॒द्यात्।
सर्वा॒ दिश॒ आ वा॑ति॥३८॥

%2.3.9.6
सर्वा॒ दिशोऽनु॒ वि वा॑ति।
सर्वा॒ दिशोऽनु॒ सं वा॒तीति॑।
स वा ए॒ष मा॑त॒रिश्वै॒व।
अथ॒ यत्प॒श्चाद्वाति॑।
पव॑मान ए॒व भू॒त्वा प॒श्चाद्वा॑ति।
पू॒तम॑स्मा॒ आह॑रन्ति।
पू॒तमुप॑हरन्ति।
पू॒तम॑श्ञाति।
य ए॒वं वेद॑।
स वा ए॒ष पव॑मान ए॒व॥३९॥

%2.3.9.7
अथ॒ यदु॑त्तर॒तो वाति॑।
स॒वि॒तैव भू॒त्त्वोत्त॑र॒तो वा॑ति।
स॒वि॒तेव॒ स्वानां᳚ भवति।
य ए॒वं वेद॑।
स वा ए॒ष स॑वि॒तैव।
ते य ए॑नं पु॒रस्ता॑दा॒यन्त॑मुप॒वद॑न्ति।
य ए॒वास्य॑ पु॒रस्ता᳚त्पा॒प्मानः॑।
ताꣴस्तेऽप॑ घ्नन्ति।
पु॒रस्ता॒दित॑रान्पा॒प्मनः॑ सचन्ते।
अथ॒ य ए॑नं दक्षिण॒त आ॒यन्त॑मुप॒वद॑न्ति॥४०॥

%2.3.9.8
य ए॒वास्य॑ दक्षिण॒तः पा॒प्मानः॑।
ताꣴस्तेऽप॑ घ्नन्ति।
द॒क्षि॒ण॒त इत॑रान्पा॒प्मनः॑ सचन्ते।
अथ॒ य ए॑नं प॒श्चादा॒यन्त॑मुप॒ वद॑न्ति।
य ए॒वास्य॑ प॒श्चात्पा॒प्मानः॑।
ताꣴस्तेऽप॑ घ्नन्ति।
प॒श्चादित॑रान्पा॒प्मनः॑ सचन्ते।
अथ॒ य ए॑नमुत्तर॒त आ॒यन्त॑मुप॒ वद॑न्ति।
य ए॒वास्यो᳚त्तर॒तः पा॒प्मानः॑।
ताꣴस्तेऽप॑ घ्नन्ति॥४१॥

%2.3.9.9
उ॒त्त॒र॒त इत॑रान्पा॒प्मनः॑ सचन्ते।
तस्मा॑दे॒वं वि॒द्वान्।
वीव॑ नृत्येत्।
प्रेव॑ चलेत्।
व्यस्ये॑वा॒क्ष्यौ भा॑षेत।
म॒ण्टये॑दिव।
क्रा॒थये॑दिव।
शृ॒ङ्गा॒येते॑व।
उ॒त मोप॑ वदेयुः।
उ॒त मे॑ पा॒प्मान॒मप॑ हन्यु॒रिति॑।
स यान्दिशꣳ॑ स॒निमे॒ष्यन्थ्स्यात्।
य॒दा तान्दिशं॒ वातो॑ वा॒यात्।
अथ॒ प्रवे॒यात्।
प्र वा॑ धावयेत्।
सा॒तमे॒व र॑दि॒तं व्यू॑ढं ग॒न्धम॒भि प्रच्य॑वते।
आऽस्य॒ तं ज॑नप॒दं पूर्वा॑ की॒र्तिर्ग॑च्छति।
दान॑कामा अस्मै प्र॒जा भ॑वन्ति।
य ए॒वं वेद॑॥४२॥\anuvakamend[वेद॒ सम्प॑वत आदि॒त्यात्प॑वते वा॒त्या वा᳚त्ये॒ष पव॑मान ए॒व द॑क्षिण॒त आ॒यन्त॑मुप॒ वद॑न्त्युत्तर॒तः पा॒प्मान॒स्ताꣴ स्तेप॑ घ्न॒न्तीत्य॒ष्टौ च॑]

%2.3.10.1
प्र॒जा\-प॑तिः॒ सोम॒ꣳ॒ राजा॑नम\-सृजत।
तं त्रयो॒ वेदा॒ अन्व॑सृज्यन्त।
तान् हस्ते॑\-ऽकुरुत।
अथ॒ ह सीता॑ सावि॒त्री।
सोम॒ꣳ॒ राजा॑नं चकमे।
श्र॒द्धामु॒ स च॑कमे।
साऽऽह॑ पि॒तरं॑ प्र॒जा\-प॑ति॒मुप॑ससार।
तꣳ हो॑वाच।
नम॑स्ते अस्तु भगवः।
उप॑ त्वा\-ऽयानि॥४३॥

%2.3.10.2
प्र त्वा॑ पद्ये।
सोमं॒ वै राजा॑नं कामये।
श्र॒द्धामु॒ स का॑मयत॒ इति॑।
तस्या॑ उ॒ ह स्था॑ग॒रम॑लङ्का॒रं क॑ल्पयि॒त्वा।
दश॑होतारं पु॒रस्ता᳚द्व्या॒ख्याय॑।
चतु॑र्\mbox{}होतारं दक्षिण॒तः।
पञ्च॑होतारं प॒श्चात्।
षड्ढो॑तारमुत्तर॒तः।
स॒प्तहो॑तारमु॒परि॑ष्टात्।
स॒म्भा॒रैश्च॒ पत्नि॑भिश्च॒ मुखे॑\-ऽल॒ङ्कृत्य॑॥४४॥

%2.3.10.3
आऽस्यार्धं व॑व्राज।
ताꣳ हो॒दीक्ष्यो॑वाच।
उप॒ मा व॑र्त॒स्वेति॑।
तꣳ हो॑वाच।
भोगं॒ तु म॒ आच॑क्ष्व।
ए॒तन्म॒ आच॑क्ष्व।
यत्ते॑ पा॒णाविति॑।
तस्या॑ उ॒ ह त्रीन् वेदा॒न्प्रद॑दौ।
तस्मा॒दुह॒ स्त्रियो॒ भोग॒मैव हा॑रयन्ते।
स यः का॒मये॑त प्रि॒यः स्या॒मिति॑॥४५॥

%2.3.10.4
यं वा॑ का॒मये॑त प्रि॒यः स्या॒दिति॑।
तस्मा॑ ए॒तꣴ स्था॑ग॒रम॑लङ्का॒रं क॑ल्पयि॒त्वा।
दश॑होतारं पु॒रस्ता᳚द्व्या॒ख्याय॑।
चतु॑र्\mbox{}होतारं दक्षिण॒तः।
पञ्च॑होतारं प॒श्चात्।
षड्ढो॑तारमुत्तर॒तः।
स॒प्तहो॑तारमु॒परि॑ष्टात्।
स॒म्भा॒रैश्च॒ पत्नि॑भिश्च॒ मुखे॑\-ऽल॒ङ्कृत्य॑।
आस्यार्धं व्र॑जेत्।
प्रि॒यो है॒व भ॑वति॥४६॥\anuvakamend[अ॒या॒न्य॒ल॒ङ्कृत्य॑ स्या॒मिति॑ भवति]

%2.3.11.1
ब्रह्मा᳚त्म॒न्वद॑\-सृजत।
तद॑कामयत।
समा॒त्मना॑ पद्ये॒येति॑।
आत्म॒न्नात्म॒न्नित्याम॑न्त्रयत।
तस्मै॑ दश॒मꣳ हू॒तः प्रत्य॑शृणोत्।
स दश॑हूतो\-ऽभवत्।
दश॑हूतो ह॒ वै नामै॒षः।
तं वा ए॒तं दश॑हूत॒ꣳ॒ सन्तम्᳚।
दश॑हो॒तेत्याच॑क्षते प॒रोक्षे॑ण।
प॒रोक्ष॑प्रिया इव॒ हि दे॒वाः॥४७॥

%2.3.11.2
आत्म॒न्नात्म॒न्नित्याम॑न्त्रयत।
तस्मै॑ सप्त॒मꣳ हू॒तः प्रत्य॑शृणोत्।
स स॒प्तहू॑तो\-ऽभवत्।
स॒प्तहू॑तो ह॒ वै नामै॒षः।
तं वा ए॒तꣳ स॒प्तहू॑त॒ꣳ॒ सन्तम्᳚।
स॒प्तहो॒तेत्याच॑क्षते प॒रोक्षे॑ण।
प॒रोक्ष॑प्रिया इव॒ हि दे॒वाः।
आत्म॒न्नात्म॒न्नित्याम॑न्त्रयत।
तस्मै॑ ष॒ष्ठꣳ हू॒तः प्रत्य॑शृणोत्।
स षड्ढू॑तो\-ऽभवत्॥४८॥

%2.3.11.3
षड्ढू॑तो ह॒ वै नामै॒षः।
तं वा ए॒तꣳ षड्ढू॑त॒ꣳ॒ सन्तम्᳚।
षड्ढो॒तेत्याच॑क्षते प॒रोक्षे॑ण।
प॒रोक्ष॑प्रिया इव॒ हि दे॒वाः।
आत्म॒न्नात्म॒न्नित्याम॑न्त्रयत।
तस्मै॑ पञ्च॒मꣳ हू॒तः प्रत्य॑शृणोत्।
स पञ्च॑हूतो\-ऽभवत्।
पञ्च॑हूतो ह॒ वै नामै॒षः।
तं वा ए॒तं पञ्च॑हूत॒ꣳ॒ सन्तम्᳚।
पञ्च॑हो॒तेत्याच॑क्षते प॒रोक्षे॑ण॥४९॥

%2.3.11.4
प॒रोक्ष॑प्रिया इव॒ हि दे॒वाः।
आत्म॒न्नात्म॒न्नित्याम॑न्त्रयत।
तस्मै॑ चतु॒र्थꣳ हू॒तः प्रत्य॑शृणोत्।
स चतु॑र्\mbox{}हूतो\-ऽभवत्।
चतु॑र्‌\mbox{}हूतो ह॒ वै नामै॒षः।
तं वा ए॒तं चतु॑र्‌\mbox{}हूत॒ꣳ॒ सन्तम्᳚।
चतु॑र्हो॒तेत्याच॑क्षते प॒रोक्षे॑ण।
प॒रोक्ष॑प्रिया इव॒ हि दे॒वाः।
तम॑ब्रवीत्।
त्वं वै मे॒ नेदि॑ष्ठꣳ हू॒तः प्रत्य॑श्रौषीः।
त्वयै॑नानाख्या॒तार॒ इति॑।
तस्मा॒न्नु है॑ना॒ꣴ॒श्चतु॑र्‌\mbox{}होतार॒ इत्याच॑क्षते।
तस्मा᳚च्छुश्रू॒षुः पु॒त्राणा॒ꣳ॒ हृद्य॑तमः।
नेदि॑ष्ठो॒ हृद्य॑तमः।
नेदि॑ष्ठो॒ ब्रह्म॑णो भवति।
य ए॒वं वेद॑॥५०॥\anuvakamend[दे॒वाः षड्ढू॑तो\-ऽभव॒त्पञ्च॑हो॒तेत्याच॑क्षते प॒रोक्षे॑णाश्रौषीः॒ षट्च॑]




\prashnaend{ब्र॒ह्म॒वा॒दिनः॒ किं दक्षि॑णां॒ यो वा अवि॑द्वा॒न्तस्य॒ वै ब्र॑ह्मवा॒दिनो॒ यद्दश॑होतारः प्र॒जा\-प॑ति॒र्व्य॑स्रं प्र॒जा\-प॑तिः॒ पुरु॑षं प्र॒जा\-प॑तिरकामयत॒ स तपः॒ सो᳚\-ऽन्तर्वा᳚न्ब्रह्मवा॒दिनो॒ यो वा इ॒मं वि॒द्यात्प्र॒जा\-प॑तिः॒ सोम॒ꣳ॒ राजा॑नं॒ ब्रह्मा᳚त्म॒न्वदेका॑दश॥११॥}{ब्र॒ह्म॒वा॒दिन॒स्तस्य॒ वा अ॒ग्नेर्यद्वा इ॒दं किं च॑ प्र॒जा\-प॑तिरकामयत॒ य ए॒वास्य॑ दक्षिण॒तः प॑ञ्चा॒शत्॥५०॥}{ब्र॒ह्म॒वा॒दिनो॒ य ए॒वं वेद॑॥}{हरिः॑ ओम्॥}{इति श्रीकृष्णयजुर्वेदीयतैत्तिरीयब्राह्मणे द्वितीयाष्टके तृतीयः प्रपाठकः समाप्तः॥}
\clearpage
\sect{चतुर्थः प्रश्नः}
\setcounter{anuvakam}{0}
\dnsub{तैत्तिरीयब्राह्मणे द्वितीयाष्टके चतुर्थः प्रपाठकः}

%2.4.1.1
जुष्टो॒ दमू॑ना॒ अति॑थिर्दुरो॒णे।
इ॒मं नो॑ य॒ज्ञमुप॑ याहि वि॒द्वान्।
विश्वा॑ अग्ने\-ऽभि॒युजो॑ वि॒हत्य॑।
श॒त्रू॒य॒तामा भ॑रा॒ भोज॑नानि।
अग्ने॒ शर्ध॑ मह॒ते सौभ॑गाय।
तव॑ द्यु॒म्नान्यु॑त्त॒मानि॑ सन्तु।
सञ्जा᳚स्प॒त्यꣳ सु॒यम॒मा कृ॑णुष्व।
श॒त्रू॒य॒ताम॒भि ति॑ष्ठा॒ महाꣳ॑सि।
अग्ने॒ यो नो॒\-ऽभितो॒ जनः॑।
वृको॒ वारो॒ जिघाꣳ॑सति॥१॥

%2.4.1.2
ताꣴस्त्वं वृ॑त्रहं जहि।
वस्व॒स्मभ्य॒मा भ॑र।
अग्ने॒ यो नो॑\-ऽभि॒दास॑ति।
स॒मा॒नो यश्च॒ निष्ट्यः॑।
इ॒ध्मस्ये॑व प्र॒क्षाय॑तः।
मा तस्योच्छे॑षि॒ किञ्च॒न।
त्वमि॑न्द्राभि॒भूर॑सि।
दे॒वो विज्ञा॑तवीर्यः।
वृ॒त्र॒हा पु॑रु॒चेत॑नः।
अप॒ प्राच॑ इन्द्र॒ विश्वाꣳ॑ अ॒मित्रान्॑॥२॥

%2.4.1.3
अपापा॑चो अभिभूते नुदस्व।
अपोदी॑चो॒ अप॑शूराध॒रा च॑ ऊ॒रौ।
यथा॒ तव॒ शर्म॒न्मदे॑म।
तमिन्द्रं॑ वाजयामसि।
म॒हे वृ॒त्राय॒ हन्त॑वे।
स वृषा॑ वृष॒भो भु॑वत्।
यु॒जे रथं॑ ग॒वेष॑ण॒ꣳ॒ हरि॑भ्याम्।
उप॒ ब्रह्मा॑णि जुजुषा॒णम॑स्थुः।
विबा॑धिष्टा॒स्य रोद॑सी महि॒त्वा।
इन्द्रो॑ वृ॒त्राण्य॑प्र॒तीज॑घ॒न्वान्॥३॥

%2.4.1.4
ह॒व्य॒वाह॑मभिमाति॒षाहम्᳚।
र॒क्षो॒हणं॒ पृत॑नासु जि॒ष्णुम्।
ज्योति॑ष्मन्तं॒ दीद्य॑तं॒ पुर॑न्धिम्।
अ॒ग्निꣴ स्वि॑ष्ट॒कृत॒मा हु॑वेम।
स्वि॑ष्टमग्ने अ॒भि तत्पृ॑णाहि।
विश्वा॑ देव॒ पृत॑ना अ॒भि ष्य।
उ॒रुं नः॒ पन्थां᳚ प्रदि॒शन्विभा॑हि।
ज्योति॑ष्मद्धेह्य॒जरं॑ न॒ आयुः॑।
त्वाम॑ग्ने ह॒विष्म॑न्तः।
दे॒वं मर्ता॑स ईडते॥४॥

%2.4.1.5
मन्ये᳚ त्वा जा॒तवे॑दसम्।
स ह॒व्या व॑क्ष्यानु॒षक्।
विश्वा॑नि नो दु॒र्गहा॑ जातवेदः।
सिन्धुं॒ न ना॒वा दु॑रि॒ताऽति॑ पर्‌\mbox{}षि।
अग्ने॑ अत्रि॒वन्मन॑सा गृणा॒नः।
अ॒स्माकं॑ बोध्यवि॒ता त॒नूना᳚म्।
पू॒षा गा अन्वे॑तु नः।
पू॒षा र॑क्ष॒त्वर्व॑तः।
पू॒षा वाजꣳ॑ सनोतु नः।
पू॒षेमा आशा॒ अनु॑वेद॒ सर्वाः᳚॥५॥

%2.4.1.6
सो अ॒स्माꣳ अभ॑यतमेन नेषत्।
स्व॒स्ति॒दा अघृ॑णिः॒ सर्व॑वीरः।
अप्र॑युच्छन्पु॒र ए॑तु॒ प्रजा॒नन्।
त्वम॑ग्ने स॒प्रथा॑ असि।
जुष्टो॒ होता॒ वरे᳚ण्यः।
त्वया॑ य॒ज्ञं वित॑न्वते।
अ॒ग्नी रक्षाꣳ॑सि सेधति।
शु॒क्र\-शो॑चि॒रम॑र्त्यः।
शुचिः॑ पाव॒क ईड्यः॑।
अग्ने॒ रक्षा॑ णो॒ अꣳह॑सः॥६॥

%2.4.1.7
प्रति॑ ष्म देव॒ रीष॑तः।
तपि॑ष्ठैर॒जरो॑ दह।
अग्ने॒ हꣳसि॒ न्य॑त्रिणम्᳚।
दीद्य॒न्मर्त्ये॒ष्वा।
स्वे क्षये॑ शुचिव्रत।
आ वा॑त वाहि भेष॒जम्।
वि वा॑त वाहि॒ यद्रपः॑।
त्वꣳ हि वि॒श्वभे॑षजः।
दे॒वानां᳚ दू॒त ईय॑से।
द्वावि॒मौ वातौ॑ वातः॥७॥

%2.4.1.8
आ सिन्धो॒रा प॑रा॒वतः॑।
दक्षं॑ मे अ॒न्य आ॒वातु॑।
परा॒न्यो वा॑तु॒ यद्रपः॑।
यद॒दो वा॑त ते गृ॒हे।
अ॒मृत॑स्य नि॒धिर्\mbox{}हि॒तः।
ततो॑ नो देहि जी॒वसे᳚।
ततो॑ नो धेहि भेष॒जम्।
ततो॑ नो॒ मह॒ आव॑ह।
वात॒ आवा॑तु भेष॒जम्।
श॒म्भूर्म॑यो॒भूर्नो॑ हृ॒दे॥८॥

%2.4.1.9
प्र ण॒ आयूꣳ॑षि तारिषत्।
त्वम॑ग्ने अ॒यासि॑।
अ॒या सन्मन॑सा हि॒तः।
अ॒या सन् ह॒व्यमू॑हिषे।
अ॒या नो॑ धेहि भेष॒जम्।
इ॒ष्टो अ॒ग्निराहु॑तः।
स्वाहा॑कृतः पिपर्तु नः।
स्व॒गा दे॒वेभ्य॑ इ॒दं नमः॑।
कामो॑ भू॒तस्य॒ भव्य॑स्य।
स॒म्राडेको॒ विरा॑जति॥९॥

%2.4.1.10
स इ॒दं प्रति॑ पप्रथे।
ऋ॒तूनुथ्सृ॑जते व॒शी।
काम॒स्तदग्रे॒ सम॑वर्त॒ताधि॑।
मन॑सो॒ रेतः॑ प्रथ॒मं यदासी᳚त्।
स॒तो बन्धु॒मस॑ति॒ निर॑विन्दन्।
हृ॒दि प्र॒तीष्या॑ क॒वयो॑ मनी॒षा।
त्वया॑ मन्यो स॒रथ॑मारु॒जन्तः॑।
हर्‌\mbox{}ष॑माणासो धृष॒ता म॑रुत्वः।
ति॒ग्मेष॑व॒ आयु॑धा स॒ꣳ॒शिशा॑नाः।
उप॒ प्रय॑न्ति॒ नरो॑ अ॒ग्निरू॑पाः॥१०॥

%2.4.1.11
म॒न्युर्भगो॑ म॒न्युरे॒वास॑ दे॒वः।
म॒न्युर्‌\mbox{}होता॒ वरु॑णो वि॒श्ववे॑दाः।
म॒न्युं विश॑ ईडते देव॒यन्तीः᳚।
पा॒हि नो॑ मन्यो॒ तप॑सा॒ श्रमे॑ण।
त्वम॑ग्ने व्रत॒भृच्छुचिः॑।
दे॒वाꣳ आसा॑दया इ॒ह।
अग्ने॑ ह॒व्याय॒ वोढ॑वे।
व्र॒तानुबिभ्र॑द्व्रत॒पा अदा᳚भ्यः।
यजा॑ नो दे॒वाꣳ अ॒जरः॑ सु॒वीरः॑।
दध॒द्रत्ना॑नि सुविदा॒नो अ॑ग्ने।
गो॒पा॒य नो॑ जी॒वसे॑ जातवेदः॥११॥\anuvakamend[जिघाꣳ॑सत्य॒मित्रा᳚ञ्जघ॒न्वानी॑डते॒ सर्वा॒ अꣳह॑सो वातो हृ॒दे रा॑जत्य॒ग्निरू॑पाः सुविदा॒नो अ॑ग्न॒ एकं॑ च]

%2.4.2.1
चक्षु॑षो हेते॒ मन॑सो हेते।
वाचो॑ हेते॒ ब्रह्म॑णो हेते।
यो मा॑\-ऽघा॒युर॑भि॒दास॑ति।
तम॑ग्ने मे॒न्या मे॒निं कृ॑णु।
यो मा॒ चक्षु॑षा॒ यो मन॑सा।
यो वा॒चा ब्रह्म॑णा\-ऽघा॒युर॑भि॒दास॑ति।
तया᳚ऽग्ने॒ त्वं मे॒न्या।
अ॒मुम॑मे॒निं कृ॑णु।
यत्किञ्चा॒सौ मन॑सा॒ यच्च॑ वा॒चा।
य॒ज्ञैर्जु॒होति॒ यजु॑षा ह॒विर्भिः॑॥१२॥

%2.4.2.2
तन्मृ॒त्युर्निर्\mbox{}ऋ॑त्या संविदा॒नः।
पु॒रादि॒ष्टादाहु॑तीरस्य हन्तु।
या॒तु॒धाना॒ निर्\mbox{}ऋ॑ति॒रादु॒रक्षः॑।
ते अ॑स्य घ्न॒न्त्वनृ॑तेन स॒त्यम्।
इन्द्रे॑षिता॒ आज्य॑मस्य मथ्नन्तु।
मा तथ्समृ॑द्धि॒ यद॒सौ क॒रोति॑।
हन्मि॑ ते॒ऽहं कृ॒तꣳ ह॒विः।
यो मे॑ घो॒रमची॑कृतः।
अपा᳚ञ्चौ त उ॒भौ बा॒हू।
अप॑नह्याम्या॒स्यम्᳚॥१३॥

%2.4.2.3
अप॑ नह्यामि ते बा॒हू।
अप॑ नह्याम्या॒स्यम्᳚।
अ॒ग्नेर्दे॒वस्य॒ ब्रह्म॑णा।
सर्वं॑ तेऽवधिषं कृ॒तम्।
पु॒रा\-ऽमुष्य॑ वषट्का॒रात्।
य॒ज्ञं दे॒वेषु॑ नस्कृधि।
स्वि॑ष्टम॒स्माकं॑ भूयात्।
माऽस्मान्प्राप॒न्न\-रा॑तयः।
अन्ति॑ दू॒रे स॒तो अ॑ग्ने।
भ्रातृ॑व्यस्याभि॒दास॑तः॥१४॥

%2.4.2.4
व॒ष॒ट्का॒रेण॒ वज्रे॑ण।
कृ॒त्याꣳ ह॑न्मि कृ॒ताम॒हम्।
यो मा॒ नक्तं॒ दिवा॑ सा॒यम्।
प्रा॒तश्चाह्नो॑ नि॒पीय॑ति।
अ॒द्या तमि॑न्द्र॒ वज्रे॑ण।
भातृ॑व्यं पादयामसि।
इन्द्र॑स्य गृ॒हो॑ऽसि॒ तन्त्वा᳚।
प्रप॑द्ये॒ सगुः॒ साश्वः॑।
स॒ह यन्मे॒ अस्ति॒ तेन॑।
ईडे॑ अ॒ग्निं वि॑प॒श्चितम्᳚॥१५॥

%2.4.2.5
गि॒रा य॒ज्ञस्य॒ साध॑नम्।
श्रु॒ष्टी॒वान॑न्धि॒तावा॑नम्।
अग्ने॑ श॒केम॑ ते व॒यम्।
यमं॑ दे॒वस्य॑ वा॒जिनः॑।
अति॒ द्वेषाꣳ॑सि तरेम।
अव॑तं मा॒ सम॑नसौ॒ समो॑कसौ।
सचे॑तसौ॒ सरे॑तसौ।
उ॒भौ माम॑वतञ्जातवेदसौ।
शि॒वौ भ॑वतम॒द्य नः॑।
स्व॒यं कृ॑ण्वा॒नः सु॒गमप्र॑यावम्॥१६॥

%2.4.2.6
ति॒ग्मशृ॑ङ्गो वृष॒भः शोशु॑चानः।
प्र॒त्नꣳ स॒धस्थ॒मनु॒ पश्य॑मानः।
आ तन्तु॑म॒ग्निर्दि॒व्यं त॑तान।
त्वन्न॒स्तन्तु॑रु॒त सेतु॑रग्ने।
त्वं पन्था॑ भवसि देव॒यानः॑।
त्वया᳚\-ऽग्ने पृ॒ष्ठं व॒यमारु॑हेम।
अथा॑ दे॒वैः स॑ध॒मादं॑ मदेम।
उदु॑त्त॒मं मु॑मुग्धि नः।
वि पाशं॑ मध्य॒मञ्चृ॑त।
अवा॑ध॒मानि॑ जी॒वसे᳚॥१७॥

%2.4.2.7
व॒यꣳ सो॑म व्र॒ते तव॑।
मन॑स्त॒नूषु॒ बिभ्र॑तः।
प्र॒जाव॑न्तो अशीमहि।
इ॒न्द्रा॒णी दे॒वी सु॒भगा॑ सु॒पत्नी᳚।
उदꣳशे॑न पति॒विद्ये॑ जिगाय।
त्रि॒ꣳ॒शद॑स्या ज॒घनं॒ योज॑नानि।
उ॒पस्थ॒ इन्द्र॒ꣴ॒ स्थवि॑रं बिभर्ति।
सेना॑ ह॒ नाम॑ पृथि॒वी ध॑नञ्ज॒या।
वि॒श्वव्य॑चा॒ अदि॑तिः॒ सूर्य॑त्वक्।
इ॒न्द्रा॒णी दे॒वी प्रा॒सहा॒ ददा॑ना॥१८॥

%2.4.2.8
सा नो॑ दे॒वी सु॒हवा॒ शर्म॑ यच्छतु।
आत्वा॑\-ऽहार्‌\mbox{}षम॒न्तर॑भूः।
ध्रु॒वस्ति॒ष्ठावि॑चाचलिः।
विश॑स्त्वा॒ सर्वा॑ वाञ्छन्तु।
मा त्वद्रा॒ष्ट्रमधि॑ भ्रशत्।
ध्रु॒वा द्यौर्ध्रु॒वा पृ॑थि॒वी।
ध्रु॒वं विश्व॑मि॒दं जग॑त्।
ध्रु॒वा ह॒ पर्व॑ता इ॒मे।
ध्रु॒वो राजा॑ वि॒शाम॒यम्।
इ॒हैवैधि॒ मा व्य॑थिष्ठाः॥१९॥

%2.4.2.9
पर्व॑त इ॒वावि॑चाचलिः।
इन्द्र॑ इवे॒ह ध्रु॒वस्ति॑ष्ठ।
इ॒ह रा॒ष्ट्रमु॑ धारय।
अ॒भिति॑ष्ठ पृतन्य॒तः।
अध॑रे सन्तु॒ शत्र॑वः।
इन्द्र॑ इव वृत्र॒हा ति॑ष्ठ।
अ॒पः क्षेत्रा॑णि स॒ञ्जयन्॑।
इन्द्र॑ एणमदीधरत्।
ध्रु॒वं ध्रु॒वेण॑ ह॒विषा᳚।
तस्मै॑ दे॒वा अधि॑ब्रवन्।
अ॒यं च॒ ब्रह्म॑ण॒स्पतिः॑॥२०॥\anuvakamend[ह॒विर्भि॑रा॒स्य॑मभि॒ दास॑तो विप॒श्चित॒मप्र॑यावञ्जी॒वसे॒ ददा॑ना व्यथिष्ठा ब्रव॒न्नेकं॑ च]

%2.4.3.1
जुष्टी॑ नरो॒ ब्रह्म॑णा वः पितृ॒णाम्।
अक्ष॑मव्ययं॒ न किला॑रिषाथ।
यच्छक्व॑रीषु बृह॒ता रवे॑ण।
इन्द्रे॒ शुष्म॒मद॑धाथा वसिष्ठाः।
पा॒व॒का नः॒ सर॑स्वती।
वाजे॑भिर्वा॒जिनी॑वती।
य॒ज्ञं व॑ष्टु धि॒या व॑सुः।
सर॑स्वत्य॒भिनो॑ नेषि॒ वस्यः॑।
मा प॑स्फरीः॒ पय॑सा॒ मा न॒ आध॑क्।
जु॒षस्व॑ नः स॒ख्या॑ वे॒श्या॑ च॥२१॥

%2.4.3.2
मा त्वक्षेत्रा॒ण्यर॑णानि गन्म।
वृ॒ञ्जे ह॒विर्नम॑सा ब॒र्॒हिर॒ग्नौ।
अया॑मि॒ स्रुग्घृ॒तव॑ती सुवृ॒क्तिः।
अम्य॑क्षि॒ सद्म॒ सद॑ने पृथि॒व्याः।
अश्रा॑यि य॒ज्ञः सूर्ये॒ न चक्षुः॑।
इ॒हार्वाञ्च॒मति॑ ह्वये।
इन्द्रं॒ जैत्रा॑य॒ जेत॑वे।
अ॒स्माक॑मस्तु॒ केव॑लः।
अ॒र्वाञ्च॒मिन्द्र॑म॒मुतो॑ हवामहे।
यो गो॒जिद्ध॑न॒जिद॑श्व॒\-जिद्यः॥२२॥

%2.4.3.3
इ॒मं नो॑ य॒ज्ञं वि॑ह॒वे जु॑षस्व।
अ॒स्य कु॑र्मो हरिवो मे॒दिनं॑ त्वा।
अस॑म्मृष्टो जायसे मातृ॒वोः शुचिः॑।
म॒न्द्रः क॒विरुद॑तिष्ठो॒ विव॑स्वतः।
घृ॒तेन॑ त्वा वर्धयन्नग्न आहुत।
धू॒मस्ते॑ के॒तुर॑भवद्दि॒वि श्रि॒तः।
अ॒ग्निरग्रे᳚ प्रथ॒मो दे॒वता॑नाम्।
संया॑तानामुत्त॒मो विष्णु॑रासीत्।
यज॑मानाय परि॒गृह्य॑ दे॒वान्।
दी॒क्षये॒दꣳ ह॒विरा ग॑च्छतन्नः॥२३॥

%2.4.3.4
अ॒ग्निश्च॑ विष्णो॒ तप॑ उत्त॒मं म॒हः।
दी॒क्षा॒पा॒लेभ्यो॒\-ऽवन॑त॒ꣳ॒ हि श॑क्रा।
विश्वै᳚र्दे॒वैर्य॒ज्ञियैः᳚ संविदा॒नौ।
दी॒क्षाम॒स्मै यज॑मानाय धत्तम्।
प्र तद्विष्णुः॑ स्तवते वी॒र्या॑य।
मृ॒गो न भी॒मः कु॑च॒रो गि॑रि॒ष्ठाः।
यस्यो॒रुषु॑ त्रि॒षु वि॒क्रम॑णेषु।
अधि॑ क्षि॒यन्ति॒ भुव॑नानि॒ विश्वा᳚।
नूमर्तो॑ दयते सनि॒ष्यन् यः।
विष्ण॑व उरुगा॒याय॒ दाश॑त्॥२४॥

%2.4.3.5
प्र यः स॒त्राचा॒ मन॑सा॒ यजा॑तै।
ए॒ताव॑न्त॒न्नर्य॑मा॒ विवा॑सात्।
विच॑क्रमे पृथि॒वीमे॒ष ए॒ताम्।
क्षेत्रा॑य॒ विष्णु॒र्मनु॑षे दश॒स्यन्।
ध्रु॒वासो॑ अस्य की॒रयो॒ जना॑सः।
उ॒रु॒क्षि॒तिꣳ सु॒जनि॑मा चकार।
त्रिर्दे॒वः पृ॑थि॒वीमे॒ष ए॒ताम्।
विच॑क्रमे श॒तर्च॑सं महि॒त्वा।
प्र विष्णु॑रस्तु त॒वस॒स्तवी॑यान्।
त्वे॒षꣴ ह्य॑स्य॒ स्थवि॑रस्य॒ नाम॑॥२५॥

%2.4.3.6
होता॑रं चि॒त्रर॑थमध्व॒रस्य॑।
य॒ज्ञस्य॑यज्ञस्य के॒तुꣳ रुश॑न्तम्।
प्रत्य॑र्धिं दे॒वस्य॑देवस्य म॒ह्ना।
श्रि॒या त्व॑ग्निमति॑थिं॒ जना॑नाम्।
आ नो॒ विश्वा॑भिरू॒तिभिः॑ स॒जोषाः᳚।
ब्रह्म॑ जुषा॒णो ह॑र्यश्व याहि।
वरी॑वृज॒थ्स्थवि॑रेभिः सुशिप्र।
अ॒स्मे दध॒द्वृष॑ण॒ꣳ॒ शुष्म॑मिन्द्र।
इन्द्रः॑ सुव॒र्॒षा ज॒नय॒न्नहा॑नि।
जि॒गायो॒शिग्भिः॒ पृत॑ना अभि॒ श्रीः॥२६॥

%2.4.3.7
प्रारो॑चय॒न्मन॑वे के॒तुमह्ना᳚म्।
अवि॑न्द॒ज्ज्योति॑र्बृह॒ते रणा॑य।
अश्वि॑ना॒वव॑से॒ निह्व॑ये वाम्।
आ नू॒नं या॑तꣳ सुकृ॒ताय॑ विप्रा।
प्रा॒त॒र्यु॒क्तेन॑ सु॒वृता॒ रथे॑न।
उ॒पाग॑च्छत॒मव॒साग॑तन्नः।
अ॒वि॒ष्टं धी॒ष्वश्वि॑ना न आ॒सु।
प्र॒जाव॒द्रेतो॒ अह्र॑यं नो अस्तु।
आवां᳚ तो॒के तन॑ये॒ तूतु॑जानाः।
सु॒रत्ना॑सो दे॒ववी॑तिं गमेम॥२७॥

%2.4.3.8
त्वꣳ सो॑म॒ क्रतु॑भिः सु॒क्रतु॑र्भूः।
त्वं दक्षैः᳚ सु॒दक्षो॑ वि॒श्ववे॑दाः।
त्वं वृषा॑ वृष॒त्वेभि॑र्महि॒त्वा।
द्यु॒म्नेभि॑र्द्यु॒म्न्य॑भवो नृ॒चक्षाः᳚।
अषा॑ढं यु॒थ्सु पृत॑नासु॒ पप्रिम्᳚।
सु॒व॒र्॒षाम॒फ्स्वां वृ॒जन॑स्य गो॒पाम्।
भ॒रे॒षु॒जाꣳ सु॑क्षि॒तिꣳ सु॒श्रव॑सम्।
जय॑न्तं॒ त्वामनु॑ मदेम सोम।
भवा॑ मि॒त्रो न शेव्यो॑ घृ॒तासु॑तिः।
विभू॑तद्युम्न एव॒ या उ॑ स॒प्रथाः᳚॥२८॥

%2.4.3.9
अधा॑ ते विष्णो वि॒दुषा॑ चि॒दृध्यः॑।
स्तोमो॑ य॒ज्ञस्य॒ राध्यो॑ ह॒विष्म॑तः।
यः पू॒र्व्याय॑ वे॒धसे॒ नवी॑यसे।
सु॒मज्जा॑नये॒ विष्ण॑वे॒ ददा॑शति।
यो जा॒तम॒स्य म॑ह॒तो म॒हि ब्रवा᳚त्।
सेदु॒ श्रवो॑भिर्यु॒ज्यं॑         चिद॒भ्य॑सत्।
तमु॑ स्तोतारः पू॒र्व्यं यथा॑ वि॒द ऋ॒तस्य॑।
गर्भꣳ॑ ह॒विषा॑ पिपर्तन।
आऽस्य॑ जा॒नन्तो॒ नाम॑ चिद्विवक्तन।
बृ॒हत्ते॑ विष्णो सुम॒तिं भ॑जामहे॥२९॥

%2.4.3.10
इ॒मा धा॒ना घृ॑त॒स्नुवः॑।
हरी॑ इ॒होप॑वक्षतः।
इन्द्रꣳ॑ सु॒खत॑मे॒ रथे᳚।
ए॒ष ब्र॒ह्मा प्रते॑म॒हे।
वि॒दथे॑ शꣳसिष॒ꣳ॒ हरी᳚।
य ऋ॒त्वियः॒ प्रते॑ वन्वे।
व॒नुषो॑ हर्य॒तं मदम्᳚।
इन्द्रो॒ नाम॑ घृ॒तन्नयः।
हरि॑भि॒श्चारु॒ सेच॑ते।
श्रु॒तो ग॒ण आ त्वा॑ विशन्तु॥३०॥

%2.4.3.11
हरि॑वर्पस॒ङ्गिरः॑।
आच॑र्‌\mbox{}षणि॒प्रा वृ॑ष॒भो जना॑नाम्।
राजा॑ कृष्टी॒नां पु॑रुहू॒त इन्द्रः॑।
स्तु॒तश्र॑व॒स्यन्नव॒सोप॑म॒द्रिक्।
यु॒क्त्वा हरी॒ वृष॒णाया᳚ह्य॒र्वाङ्।
प्र यथ्सिन्ध॑वः प्रस॒वं यदायन्॑।
आपः॑ समु॒द्रꣳ र॒थ्ये॑व जग्मुः।
अत॑श्चि॒दिन्द्रः॒ सद॑सो॒ वरी॑यान्।
यदी॒ꣳ॒ सोमः॑ पृ॒णति॑ दु॒ग्धो अ॒ꣳ॒शुः।
ह्वया॑मसि॒ त्वेन्द्र॑ या॒ह्य॑र्वाङ्॥३१॥

%2.4.3.12
अर॑न्ते॒ सोम॑स्त॒नुवे॑ भवाति।
शत॑क्रतो मा॒दय॑स्वा सु॒तेषु॑।
प्रास्माꣳ अ॑व॒ पृत॑नासु॒ प्रयु॒थ्सु।
इन्द्रा॑य॒ सोमाः᳚ प्र॒दिवो॒ विदा॑नाः।
ऋ॒भुर्येभि॒र्वृष॑पर्वा॒ विहा॑याः।
प्र॒य॒म्यमा॑णा॒न्प्रति॒ षू गृ॑भाय।
इन्द्र॒ पिब॒ वृष॑धूतस्य॒ वृष्णः॑।
अहे॑डमान॒ उप॑याहि य॒ज्ञम्।
तुभ्यं॑ पवन्त॒ इन्द॑वः सु॒तासः॑।
गावो॒ न व॑ज्रिन्थ्स्व॒मोको॒ अच्छ॑॥३२॥

%2.4.3.13
इन्द्रा ग॑हि प्रथ॒मो य॒ज्ञिया॑नाम्।
या ते॑ का॒कुथ्सुकृ॑ता॒ या वरि॑ष्ठा।
यया॒ शश्व॒त्पिब॑सि॒ मध्व॑ ऊ॒र्मिम्।
तया॑ पाहि॒ प्र ते॑ अध्व॒र्युर॑स्थात्।
सन्ते॒ वज्रो॑ वर्ततामिन्द्र ग॒व्युः।
प्रा॒त॒र्युजा॒ वि बो॑धय।
अश्वि॑ना॒वेह ग॑च्छतम्।
अ॒स्य सोम॑स्य पी॒तये᳚।
प्रा॒त॒र्यावा॑णा प्रथ॒मा य॑जध्वम्।
पु॒रा गृध्रा॒दर॑रुषः पिबाथः।
प्रा॒तर्\mbox{}हि य॒ज्ञम॒श्विना॒ दधा॑ते।
प्रशꣳ॑सन्ति क॒वयः॑ पूर्व॒भाजः॑।
प्रा॒तर्य॑जध्वम॒श्विना॑ हिनोत।
न सा॒यम॑स्ति देव॒या अजु॑ष्टम्।
उ॒तान्यो अ॒स्मद्य॑जते॒ विचा॑यः।
पूर्वः॑ पूर्वो॒ यज॑मानो॒ वनी॑यान्॥३३॥\anuvakamend[चा॒श्व॒जिद्यो ग॑च्छतं नो॒ दाश॒न्नामा॑भि॒श्रीर्ग॑मेम स॒प्रथा॑ भजामहे विशन्तु या॒ह्य॑र्वाङच्छ॑ पिबाथः॒ षट्च॑]

%2.4.4.1
न॒क्तं॒ जा॒ता\-ऽस्यो॑षधे।
रामे॒ कृष्णे॒ असि॑क्नि च।
इ॒दꣳ र॑जनि रजय।
कि॒लासं॑ पलि॒तं च॒ यत्।
कि॒लासं॑ च पलि॒तं च॑।
निरि॒तो ना॑शया॒ पृष॑त्।
आ नः॒ स्वो अ॑श्ञुतां॒ वर्णः॑।
परा᳚ श्वे॒तानि॑ पातय।
असि॑तं ते नि॒लय॑नम्।
आ॒स्थान॒मसि॑तं॒ तव॑॥३४॥

%2.4.4.2
असि॑क्नियस्योषधे।
निरि॒तो ना॑शया॒ पृष॑त्।
अ॒स्थि॒जस्य॑ कि॒लास॑स्य।
त॒नू॒जस्य॑ च॒ यत्त्व॒चि।
कृ॒त्यया॑ कृ॒तस्य॒ ब्रह्म॑णा।
लक्ष्म॑ श्वे॒तम॑नीनशम्।
सरू॑पा॒ नाम॑ ते मा॒ता।
सरू॑पो॒ नाम॑ ते पि॒ता।
सरू॑पा\-ऽस्योषधे॒ सा।
सरू॑पमि॒दं कृ॑धि॥३५॥

%2.4.4.3
शु॒नꣳ हु॑वेम म॒घवा॑न॒मिन्द्रम्᳚।
अ॒स्मिन्भरे॒ नृत॑मं॒ वाज॑सातौ।
शृ॒ण्वन्त॑मु॒ग्रमू॒तये॑ स॒मथ्सु॑।
घ्नन्तं॑ वृ॒त्राणि॑ स॒ञ्जितं॒ धना॑नाम्।
धू॒नु॒थ द्यां पर्व॑तान्दा॒शुषे॒ वसु॑।
नि वो॒ वना॑ जिहते॒ याम॑ नो भि॒या।
को॒पय॑थ पृथि॒वीं पृ॑श्ञिमातरः।
यु॒धे यदु॑ग्राः॒ पृष॑ती॒रयु॑ग्ध्वम्।
प्रवे॑पयन्ति॒ पर्व॑तान्।
विवि॑ञ्चन्ति॒ वन॒स्पतीन्॑॥३६॥

%2.4.4.4
प्रोवा॑रत मरुतो दु॒र्मदा॑ इव।
देवा॑सः॒ सर्व॑या वि॒शा।
पु॒रु॒त्रा हि स॒दृङ्ङसि॑।
विशो॒ विश्वा॒ अनु॑ प्र॒भु।
स॒मथ्सु॑ त्वा हवामहे।
स॒मथ्स्व॒ग्निमव॑से।
वा॒ज॒यन्तो॑ हवामहे।
वाजे॑षु चि॒त्ररा॑धसम्।
सङ्ग॑च्छध्व॒ꣳ॒ संव॑दध्वम्।
सं वो॒ मनाꣳ॑सि जानताम्॥३७॥

%2.4.4.5
दे॒वा भा॒गं यथा॒ पूर्वे᳚।
स॒ञ्जा॒ना॒ना उ॒पास॑त।
स॒मा॒नो मन्त्रः॒ समि॑तिः समा॒नी।
स॒मा॒नं मनः॑ स॒ह चि॒त्तमे॑षाम्।
स॒मा॒नं केतो॑ अ॒भि सꣳ र॑भध्वम्।
सं॒ज्ञाने॑न वो ह॒विषा॑ यजामः।
स॒मा॒नी व॒ आकू॑तिः।
स॒मा॒ना हृद॑यानि वः।
स॒मा॒नम॑स्तु वो॒ मनः॑।
यथा॑ वः॒ सुस॒हास॑ति॥३८॥

%2.4.4.6
सं॒ज्ञानं॑ नः॒ स्वैः।
सं॒ज्ञान॒मर॑णैः।
सं॒ज्ञान॑मश्विना यु॒वम्।
इ॒हास्मासु॒ निय॑च्छतम्।
सं॒ज्ञानं॑ मे॒ बृह॒स्पतिः॑।
सं॒ज्ञानꣳ॑ सवि॒ता क॑रत्।
सं॒ज्ञान॑मश्विना यु॒वम्।
इ॒ह मह्यं॒ नि य॑च्छतम्।
उप॑ च्छा॒यामि॑व॒ घृणेः᳚।
अग॑न्म॒ शर्म॑ ते व॒यम्॥३९॥

%2.4.4.7
अग्ने॒ हिर॑ण्यसन्दृशः।
अद॑ब्धेभिः सवितः पा॒युभि॒ष्ट्वम्।
शि॒वेभि॑र॒द्य परि॑पाहि नो॒ गयम्᳚।
हिर॑ण्यजिह्वः सुवि॒ताय॒ नव्य॑से।
रक्षा॒ माकि॑र्नो अ॒घशꣳ॑स ईशत।
मदे॑मदे॒ हि नो॑ द॒दुः।
यू॒था गवा॑मृजु॒क्रतुः॑।
सङ्गृ॑भाय पु॒रूश॒ता।
उ॒भ॒या ह॒स्त्या वसु॑।
शि॒शी॒हि रा॒य आ भ॑र॥४०॥

%2.4.4.8
शिप्रि॑न्वाजानां पते।
शची॑व॒स्तव॑ द॒ꣳ॒सना᳚।
आ तू न॑ इन्द्र भाजय।
गोष्वश्वे॑षु शु॒भ्रुषु॑।
स॒हस्रे॑षु तुवीमघ।
यद्दे॑वा देव॒हेड॑नम्।
देवा॑सश्चकृ॒मा व॒यम्।
आदि॑त्या॒स्तस्मा᳚न्मा यू॒यम्।
ऋ॒तस्य॒र्तेन॑ मुञ्चत।
ऋ॒तस्य॒र्तेना॑ऽऽदित्याः॥४१॥

%2.4.4.9
यज॑त्रा मु॒ञ्चते॒ह मा᳚।
य॒ज्ञैर्वो॑ यज्ञवाहसः।
आ॒शिक्ष॑न्तो॒ न शे॑किम।
मेद॑स्वता॒ यज॑मानाः।
स्रु॒चा\-ऽऽज्ये॑न॒ जुह्व॑तः।
अ॒का॒मा वो॑ विश्वेदेवाः।
शिक्ष॑न्तो॒ नोप॑ शेकिम।
यदि॒ दिवा॒ यदि॒ नक्तम्᳚।
एन॑ एन॒स्योक॑रत्।
भू॒तं मा॒ तस्मा॒द्भव्यं॑ च॥४२॥

%2.4.4.10
द्रु॒प॒दादि॑व मुञ्चतु।
द्रु॒प॒दादि॒वेन्मु॑मुचा॒नः।
स्वि॒न्नः स्ना॒त्वी मला॑दिव।
पू॒तं प॒वित्रे॑णे॒वाऽऽज्यम्᳚।
विश्वे॑ मुञ्चन्तु॒ मैन॑सः।
उद्व॒यं तम॑स॒स्परि॑।
पश्य॑न्तो॒ ज्योति॒रुत्त॑रम्।
दे॒वं दे॑व॒त्रा सूर्यम्᳚।
अग॑न्म॒ ज्योति॑रुत्त॒मम्॥४३॥\anuvakamend[तव॑ कृधि॒ वन॒स्पती᳚ञ्जानता॒मस॑ति व॒यं भ॑रादित्याश्च॒ नव॑ च]

%2.4.5.1
वृषा॒ सो अ॒ꣳ॒शुः प॑वते ह॒विष्मा॒न्थ्सोमः॑।
इन्द्र॑स्य भा॒ग ऋ॑त॒युः श॒तायुः॑।
स मा॒ वृषा॑णं वृष॒भं कृ॑णोतु।
प्रि॒यं वि॒शाꣳ सर्व॑वीरꣳ सु॒वीरम्᳚।
कस्य॒ वृषा॑ सु॒ते सचा᳚।
नि॒युत्वा᳚न्वृष॒भो र॑णत्।
वृ॒त्र॒हा सोम॑पीतये।
यस्ते॑ शृङ्ग वृषोनपात्।
प्रण॑पात्कुण्ड॒पाय्यः॑।
न्य॑स्मिन्दध्र॒ आ मनः॑॥४४॥

%2.4.5.2
तꣳ स॒ध्रीची॑रू॒तयो॒ वृष्णि॑यानि।
पौꣴस्या॑नि नि॒युतः॑ सश्चु॒\-रिन्द्रम्᳚।
स॒मु॒द्रं न सिन्ध॑व उ॒क्थशु॑ष्माः।
उ॒रु॒व्यच॑स॒ङ्गिर॒ आ वि॑शन्ति।
इन्द्रा॑य॒ गिरो॒ अनि॑शितसर्गाः।
अ॒पः प्रैर॑य॒न्थ्सग॑रस्य बु॒ध्नात्।
यो अक्षे॑णेव च॒क्रिया॒ शची॑भिः।
विष्व॑क्त॒स्तम्भ॑ पृथि॒वीमु॒त द्याम्।
अक्षो॑दय॒च्छव॑सा॒ क्षाम॑बु॒ध्नम्।
वार्णवा॑त॒स्तवि॑षीभि॒रिन्द्रः॑॥४५॥

%2.4.5.3
दृ॒ढान्यौ᳚घ्नादु॒शमा॑न॒ ओजः॑।
अवा॑भिनत्क॒कुभः॒ पर्व॑तानाम्।
आ नो॑ अग्ने सुके॒तुना᳚।
र॒यिं वि॒श्वायु॑पोषसम्।
मा॒र्डी॒कं धे॑हि जी॒वसे᳚।
त्वꣳ सो॑म म॒हे भगम्᳚।
त्वं यून॑ ऋताय॒ते।
दक्षं॑ दधासि जी॒वसे᳚।
रथं॑ युञ्जते म॒रुतः॑ शु॒भे सु॒गम्।
सूरो॒ न मि॑त्रावरुणा॒ गवि॑ष्टिषु॥४६॥

%2.4.5.4
रजाꣳ॑सि चि॒त्रा विच॑रन्ति त॒न्यवः॑।
दि॒वः स॑म्राजा॒ पय॑सा न उक्षतम्।
वाच॒ꣳ॒ सुमि॑त्रावरुणा॒विरा॑वतीम्।
प॒र्जन्य॑श्चि॒त्रां व॑दति॒ त्विषी॑मतीम्।
अ॒भ्रा व॑सत मरुतः सुमा॒यया᳚।
द्यां व॑र्‌\mbox{}षयतमरु॒णाम॑रे॒पसम्᳚।
अयु॑क्त स॒प्त शु॒न्ध्युवः॑।
सूरो॒ रथ॑स्य न॒प्त्रियः॑।
ताभि॑र्याति॒ स्वयु॑क्तिभिः।
वहि॑ष्ठेभिर्वि॒हर॑न् यासि॒ तन्तुम्᳚॥४७॥

%2.4.5.5
अ॒व॒व्यय॒न्नसि॑तं देव॒ वस्वः॑।
दवि॑ध्वतो र॒श्मयः॒ सूर्य॑स्य।
चर्मे॒वावा॑धु॒स्तमो॑ अ॒फ्स्व॑न्तः।
प॒र्जन्या॑य॒ प्र गा॑यत।
दि॒वस्पु॒त्राय॑ मी॒ढुषे᳚।
स नो॑ य॒वस॑मिच्छतु।
अच्छा॑ वद त॒वसं॑ गी॒र्भिरा॒भिः।
स्तु॒हि प॒र्जन्यं॒ नम॒सा\-ऽऽवि॑वास।
कनि॑क्रदद्वृष॒भो जी॒रदा॑नुः।
रेतो॑ दधा॒त्वोष॑धीषु॒ गर्भम्᳚॥४८॥

%2.4.5.6
यो गर्भ॒मोष॑धीनाम्।
गवां᳚ कृ॒णोत्यर्व॑ताम्।
प॒र्जन्यः॑ पुरु॒षीणा᳚म्।
तस्मा॒ इदा॒स्ये॑ ह॒विः।
जु॒होता॒ मधु॑मत्तमम्।
इडां नः सं॒यतं॑ करत्।
ति॒स्रो यद॑ग्ने श॒रद॒स्त्वामित्।
शुचिं॑ घृ॒तेन॒ शुच॑यः सप॒र्यन्।
नामा॑नि चिद्दधिरे य॒ज्ञिया॑नि।
असू॑दयन्त त॒नुवः॒ सुजा॑ताः॥४९॥

%2.4.5.7
इन्द्र॑श्च नः शुनासीरौ।
इ॒मं य॒ज्ञं मि॑मिक्षतम्।
गर्भं॑ धत्तꣴ स्व॒स्तये᳚।
ययो॑रि॒दं विश्वं॒ भुव॑नमा वि॒वेश॑।
ययो॑रान॒न्दो निहि॑तो॒ मह॑श्च।
शुना॑सीरावृ॒तुभिः॑ संविदा॒नौ।
इन्द्र॑वन्तौ ह॒विरि॒दं जु॑षेथाम्।
आघा॒ये अ॒ग्निमि॑न्ध॒ते।
स्तृ॒णन्ति॑ ब॒र्॒हिरा॑नु॒षक्।
येषा॒मिन्द्रो॒ युवा॒ सखा᳚।
अग्न॒ इन्द्र॑श्च मे॒दिना᳚।
ह॒थो वृ॒त्राण्य॑प्र॒ति।
यु॒वꣳ हि वृ॑त्र॒हन्त॑मा।
याभ्या॒ꣳ॒ सुव॒रज॑य॒न्नग्र॑ ए॒व।
यावा॑तस्थ॒तुर्भुव॑नस्य॒ मध्ये᳚।
प्रच॑र्‌\mbox{}ष॒णी वृ॑षणा॒ वज्र॑बाहू।
अ॒ग्नी इन्द्रा॑वृत्र॒हणा॑ हुवे वाम्॥५०॥\anuvakamend[मन॒ इन्द्रो॒ गवि॑ष्टिषु॒ तन्तुं॒ गर्भ॒ꣳ॒ सुजा॑ताः॒ सखा॑ स॒प्त च॑]

%2.4.6.1
उ॒त नः॑ प्रि॒या प्रि॒यासु॑।
स॒प्तस्वसा॒ सुजु॑ष्टा।
सर॑स्वती॒ स्तोम्या॑\-ऽभूत्।
इ॒मा जुह्वा॑नायु॒ष्मदा नमो॑भिः।
प्रति॒ स्तोमꣳ॑ सरस्वति जुषस्व।
तव॒ शर्म॑न्प्रि॒यत॑मे॒ दधा॑नाः।
उप॑स्थेयाम शर॒णं न वृ॒क्षम्।
त्रीणि॑ प॒दा विच॑क्रमे।
विष्णु॑र्गो॒पा अदा᳚भ्यः।
ततो॒ धर्मा॑णि धा॒रयन्॑॥५१॥

%2.4.6.2
तद॑स्य प्रि॒यम॒भि पाथो॑ अश्याम्।
नरो॒ यत्र॑ देव॒यवो॒ मद॑न्ति।
उ॒रु॒क्र॒मस्य॒ स हि बन्धु॑रि॒त्था।
विष्णोः᳚ प॒दे प॑र॒मे मध्व॒ उथ्सः॑।
क्र॒त्वा॒दा अ॑स्थु॒ श्रेष्ठः॑।
अ॒द्य त्वा॑ व॒न्वन्थ्सु॒रेक्णाः᳚।
मर्त॑ आनाश सुवृ॒क्तिम्।
इ॒मा ब्र॑ह्म ब्रह्मवाह।
प्रि॒या त॒ आ ब॒र्॒हिः सी॑द।
वी॒हि सू॑र पुरो॒डाशम्᳚॥५२॥

%2.4.6.3
उप॑ नः सू॒नवो॒ गिरः॑।
शृ॒ण्वन्त्व॒मृत॑स्य॒ ये।
सु॒मृ॒डी॒का भ॑वन्तु नः।
अ॒द्या नो॑ देव सवितः।
प्र॒जाव॑थ्सावीः॒ सौभ॑गम्।
परा॑ दुः॒ष्वप्नि॑यꣳ सुव।
विश्वा॑नि देव सवितः।
दु॒रि॒तानि॒ परा॑ सुव।
यद्भ॒द्रं तन्म॒ आ सु॑व।
शुचि॑म॒र्कैर्बृह॒स्पतिम्᳚॥५३॥

%2.4.6.4
अ॒ध्व॒रेषु॑ नमस्यत।
अ॒ना॒म्योज॒ आ च॑के।
या धा॒रय॑न्त दे॒वा सु॒दक्षा॒ दक्ष॑पितारा।
अ॒सु॒र्या॑य॒ प्रम॑हसा।
स इत् क्षेति॒ सुधि॑त॒ ओक॑सि॒ स्वे।
तस्मा॒ इडा॑ पिन्वते विश्व॒दानी᳚।
तस्मै॒ विशः॑ स्व॒यमे॒वान॑मन्ति।
यस्मि॑न्ब्र॒ह्मा राज॑नि॒ पूर्व॒ एति॑।
सकू॑तिमिन्द्र॒ सच्यु॑तिम्।
सच्यु॑तिं ज॒घन॑च्युतिम्॥५४॥

%2.4.6.5
क॒नात्का॒भान्न॒ आ भ॑र।
प्र॒य॒फ्स्यन्नि॑व स॒क्थ्यौ᳚।
वि न॑ इन्द्र॒ मृधो॑ जहि।
कनी॑खुनदिव सा॒पयन्॑।
अ॒भि नः॒ सुष्टु॑तिं नय।
प्र॒जा\-प॑तिः स्त्रि॒यां यशः॑।
मु॒ष्कयो॑रदधा॒थ्सपम्᳚।
काम॑स्य॒ तृप्ति॑मान॒न्दम्।
तस्या᳚ग्ने भाजये॒ह मा᳚।
मोदः॑ प्रमो॒द आ॑न॒न्दः॥५५॥

%2.4.6.6
मु॒ष्कयो॒र्निहि॑तः॒ सपः॑।
सृ॒त्वेव॒ काम॑स्य तृप्याणि।
दक्षि॑णानां प्रतिग्र॒हे।
मन॑सश्चि॒त्तमाकू॑तिम्।
वा॒चः स॒त्यम॑शीमहि।
प॒शू॒नाꣳ रू॒पमन्न॑स्य।
यशः॒ श्रीः श्र॑यतां॒ मयि॑।
यथा॒ऽहम॒स्या अतृ॑पꣴ स्त्रि॒यै पुमान्॑।
यथा॒ स्त्री तृप्य॑ति पु॒ꣳ॒सि प्रि॒ये प्रि॒या।
ए॒वं भग॑स्य तृप्याणि॥५६॥

%2.4.6.7
य॒ज्ञस्य॒ काम्यः॑ प्रि॒यः।
ददा॒मीत्य॒ग्निर्व॑दति।
तथेति॑ वा॒युरा॑ह॒ तत्।
हन्तेति॑ स॒त्यं च॒न्द्रमाः᳚।
आ॒दि॒त्यः स॒त्यमोमिति॑।
आप॒स्तथ्स॒त्यमा भ॑रन्।
यशो॑ य॒ज्ञस्य॒ दक्षि॑णाम्।
अ॒सौ मे॒ कामः॒ समृ॑द्ध्यताम्।
न हि स्पश॒मवि॑दन्न॒न्यम॒स्मात्।
वै॒श्वा॒न॒रात्पु॑रए॒तार॑म॒ग्नेः॥५७॥

%2.4.6.8
अथे॑ममन्थन्न॒मृत॒ममू॑राः।
वै॒श्वा॒न॒रं क्षे᳚त्र॒जित्या॑य दे॒वाः।
येषा॑मि॒मे पूर्वे॒ अर्मा॑स॒ आसन्॑।
अ॒यू॒पाः सद्म॒ विभृ॑ता पु॒रूणि॑।
वैश्वा॑नर॒ त्वया॒ ते नु॒त्ताः।
पृ॒थि॒वीम॒न्याम॒भित॑स्थु॒र्जना॑सः।
पृ॒थि॒वीं मा॒तरं॑ म॒हीम्।
अ॒न्तरि॑क्ष॒मुप॑ ब्रुवे।
बृ॒ह॒तीमू॒तये॒ दिवम्᳚।
विश्वं॑ बिभर्ति पृथि॒वी॥५८॥

%2.4.6.9
अ॒न्तरि॑क्षं॒ वि प॑प्रथे।
दु॒हे द्यौर्बृ॑ह॒ती पयः॑।
न ता न॑शन्ति॒ न द॑भाति॒ तस्क॑रः।
नैना॑ अमि॒त्रो व्यथि॒राद॑धर्‌\mbox{}षति।
दे॒वाꣴश्च॒ याभि॒र्यज॑ते॒ ददा॑ति च।
ज्योगित्ताभिः॑ सचते॒ गोप॑तिः स॒ह।
न ता अर्वा॑ रे॒णुक॑काटो अश्ञुते।
न सꣴ॑स्कृत॒त्रमुप॑ यन्ति॒ ता अ॒भि।
उ॒रु॒गा॒यमभ॑यं॒ तस्य॒ ता अनु॑।
गावो॒ मर्त्य॑स्य॒ वि च॑रन्ति॒ यज्व॑नः॥५९॥

%2.4.6.10
रात्री॒ व्य॑ख्यदाय॒ती।
पु॒रु॒त्रा दे॒व्य॑क्षभिः॑।
विश्वा॒ अधि॒ श्रियो॑ऽधित।
उप॑ ते॒ गा इ॒वाक॑रम्।
वृ॒णी॒ष्व दु॑हितर्दिवः।
रात्री॒ स्तोमं॒ न जि॒ग्युषी᳚।
दे॒वीं वाच॑मजनयन्त दे॒वाः।
तां वि॒श्वरू॑पाः प॒शवो॑ वदन्ति।
सा नो॑ म॒न्द्रेष॒मूर्जं॒ दुहा॑ना।
धे॒नुर्वाग॒स्मानुप॒ सुष्टु॒तैतु॑॥६०॥

%2.4.6.11
यद्वाग्वद॑न्त्यविचेत॒नानि॑।
राष्ट्री॑ दे॒वानां निष॒साद॑ म॒न्द्रा।
चत॑स्र॒ ऊर्जं॑ दुदुहे॒ पयाꣳ॑सि।
क्व॑ स्विदस्याः पर॒मं ज॑गाम।
गौ॒री मि॑माय सलि॒लानि॒ तक्ष॑ती।
एक॑पदी द्वि॒पदी॒ सा चतु॑ष्पदी।
अ॒ष्टाप॑दी॒ नव॑पदी बभू॒वुषी᳚।
स॒हस्रा᳚क्षरा पर॒मे व्यो॑मन्।
तस्याꣳ॑ समु॒द्रा अधि॒ विक्ष॑रन्ति।
तेन॑ जीवन्ति प्र॒दिश॒श्चत॑स्रः॥६१॥

%2.4.6.12
ततः॑ क्षरत्य॒क्षरम्᳚।
तद्विश्व॒मुप॑ जीवति।
इन्द्रा॒सूरा॑ ज॒नय॑न्वि॒श्वक॑र्मा।
म॒रुत्वाꣳ॑ अस्तु ग॒णवा᳚न्थ्सजा॒तवान्॑।
अ॒स्य स्नु॒षा श्वशु॑रस्य॒ प्रशि॑ष्टिम्।
स॒पत्ना॒ वाचं॒ मन॑सा॒ उपा॑सताम्।
इन्द्रः॒ सूरो॑ अतर॒द्रजाꣳ॑सि।
स्नु॒षा स॒पत्ना॒ श्वशु॑रो॒\-ऽयम॑स्तु।
अ॒यꣳ शत्रू᳚ञ्जयतु॒ जर्‌\mbox{}हृ॑षाणः।
अ॒यं वाजं॑ जयतु॒ वाज॑सातौ।
अ॒ग्निः क्ष॑त्र॒भृदनि॑भृष्ट॒मोजः॑।
स॒ह॒स्रियो॑ दीप्यता॒मप्र॑युच्छन्।
वि॒भ्राज॑मानः समिधा॒ न उ॒ग्रः।
आऽन्तरि॑क्षमरुह॒दग॒न्द्याम्॥६२॥\anuvakamend[धा॒रय॑न्पुरो॒डाशं॒ बृह॒स्पतिं॑ ज॒घन॑च्युतिमान॒न्दो भग॑स्य तृप्याण्य॒ग्नेः पृ॑थि॒वी यज्व॑न एतु प्र॒दिश॒श्चत॑स्रो॒ वाज॑सातौ च॒त्वारि॑ च]

%2.4.7.1
वृषा᳚\-ऽस्य॒ꣳ॒शुर्वृ॑ष॒भाय॑ गृह्यसे।
वृषा॒\-ऽयमु॒ग्रो नृ॒चक्ष॑से।
दि॒व्यः क॑र्म॒ण्यो॑ हि॒तो बृ॒हन्नाम॑।
वृ॒ष॒भस्य॒ या क॒कुत्।
वि॒षू॒वान् वि॑ष्णो भवतु।
अ॒यं यो मा॑म॒को वृषा᳚।
अथो॒ इन्द्र॑ इव दे॒वेभ्यः॑।
वि ब्र॑वीतु॒ जने᳚भ्यः।
आयु॑ष्मन्तं॒ वर्च॑स्वन्तम्।
अथो॒ अधि॑पतिं वि॒शाम्॥६३॥

%2.4.7.2
अ॒स्याः पृ॑थि॒व्या अध्य॑क्षम्।
इ॒ममि॑न्द्र वृष॒भं कृ॑णु।
यः सु॒शृङ्गः॑ सुवृष॒भः।
क॒ल्याणो॒ द्रोण॒ आहि॑तः।
कार्‌\mbox{}षी॑वल प्रगाणेन।
वृ॒ष॒भेण॑ यजामहे।
वृ॒ष॒भेण॒ यज॑मानाः।
अक्रू॑रेणेव स॒र्पिषा᳚।
मृद्ध॑श्च॒ सर्वा॒ इन्द्रे॑ण।
पृत॑नाश्च जयामसि॥६४॥

%2.4.7.3
यस्या॒यमृ॑ष॒भो ह॒विः।
इन्द्रा॑य परिणी॒यते᳚।
जया॑ति॒ शत्रु॑मा॒यन्तम्᳚।
अथो॑ हन्ति पृतन्य॒तः।
नृ॒णामह॑ प्र॒णीरस॑त्।
अग्र॑ उद्भिन्द॒ताम॑सत्।
इन्द्र॒ शुष्मं॑ त॒नुवा॒ मेर॑यस्व।
नी॒चा विश्वा॑ अ॒भिति॑ष्ठा॒भिमा॑तीः।
नि शृ॑णीह्याबा॒धं यो नो॒ अस्ति॑।
उ॒रुं नो॑ लो॒कं कृ॑णुहि जीरदानो॥६५॥

%2.4.7.4
प्रेह्य॒भि प्रेहि॒ प्र भ॑रा॒ सह॑स्व।
मा विवे॑नो॒ वि शृ॑णुष्वा॒ जने॑षु।
उदी॑डि॒तो वृ॑षभ॒ तिष्ठ॒ शुष्मैः᳚।
इन्द्र॒ शत्रू᳚न्पु॒रो अ॒स्माक॑ युध्य।
अग्ने॒ जेता॒ त्वं ज॑य।
शत्रू᳚न्थ्सहस॒ ओज॑सा।
वि शत्रू॒न्॒ विमृधो॑ नुद।
ए॒तं ते॒ स्तोमं॑ तुविजात॒ विप्रः॑।
रथं॒ न धीरः॒ स्वपा॑ अतक्षम्।
यदीद॑ग्ने॒ प्रति॒त्वं दे॑व॒ हर्याः᳚॥६६॥

%2.4.7.5
सुव॑र्वतीर॒प ए॑ना जयेम।
यो घृ॒तेना॒भिमा॑नितः।
इन्द्र॒ जैत्रा॑य जज्ञिषे।
स नः॒ सङ्का॑सु पारय।
पृ॒त॒ना॒साह्ये॑षु च।
इन्द्रो॑ जिगाय पृथि॒वीम्।
अ॒न्तरि॑क्ष॒ꣳ॒ सुव॑र्म॒हत्।
वृ॒त्र॒हा पु॑रु॒चेत॑नः।
इन्द्रो॑ जिगाय॒ सह॑सा॒ सहाꣳ॑सि।
इन्द्रो॑ जिगाय॒ पृत॑नानि॒ विश्वा᳚॥६७॥

%2.4.7.6
इन्द्रो॑ जा॒तो वि पुरो॑ रुरोज।
स नः॑ पर॒स्पा वरि॑वः कृणोतु।
अ॒यं कृ॒त्नुरगृ॑भीतः।
वि॒श्व॒जिदु॒द्भिदिथ्सोमः॑।
ऋषि॒र्विप्रः॒ काव्ये॑न।
वा॒युर॑ग्रे॒गा य॑ज्ञ॒प्रीः।
सा॒कङ्ग॒न्मन॑सा य॒ज्ञम्।
शि॒वो नि॒युद्भिः॑ शि॒वाभिः॑।
वायो॑ शु॒क्रो अ॑यामि ते।
मध्वो॒ अग्रं॒ दिवि॑ष्टिषु॥६८॥

%2.4.7.7
आ या॑हि॒ सोम॑ पीतये।
स्वा॒रु॒हो दे॑व नि॒युत्व॑ता।
इ॒ममि॑न्द्र वर्धय क्ष॒त्रिया॑णाम्।
अ॒यं वि॒शां वि॒श्पति॑रस्तु॒ राजा᳚।
अ॒स्मा इ॑न्द्र॒ महि॒ वर्चाꣳ॑सि धेहि।
अ॒व॒र्चसं॑ कणुहि॒ शत्रु॑मस्य।
इ॒ममा भ॑ज॒ ग्रामे॒ अश्वे॑षु॒ गोषु॑।
निर॒मुं भ॑ज॒ यो॑\-ऽमित्रो॑ अस्य।
वर्ष्म॑न् क्ष॒त्रस्य॑ क॒कुभि॑ श्रयस्व।
ततो॑ न उ॒ग्रो वि भ॑जा॒ वसू॑नि॥६९॥

%2.4.7.8
अ॒स्मे द्या॑वापृथिवी॒ भूरि॑ वा॒मम्।
सन्दु॑हाथां घर्म॒दुघे॑व धे॒नुः।
अ॒यꣳ राजा᳚ प्रि॒य इन्द्र॑स्य भूयात्।
प्रि॒यो गवा॒मोष॑धीनामु॒तापाम्।
यु॒नज्मि॑ त उत्त॒राव॑न्त॒मिन्द्रम्᳚।
येन॒ जया॑सि॒ न परा॒ जया॑सै।
स त्वा॑\-ऽकरेकवृष॒भꣴ स्वाना᳚म्।
अथो॑ राजन्नुत्त॒मं मा॑न॒वाना᳚म्।
उत्त॑र॒स्त्वमध॑रे ते स॒पत्नाः᳚।
एक॑वृषा॒ इन्द्र॑सखा जिगी॒वान्॥७०॥

%2.4.7.9
विश्वा॒ आशाः॒ पृत॑नाः स॒ञ्जयं॒ जयन्॑।
अ॒भि ति॑ष्ठ शत्रूय॒तः स॑हस्व।
तुभ्यं॑ भरन्ति क्षि॒तयो॑ यविष्ठ।
ब॒लिम॑ग्ने॒ अन्ति॑त॒ ओत दू॒रात्।
आ भन्दि॑ष्ठस्य सुम॒तिं चि॑किद्धि।
बृ॒हत्ते॑ अग्ने॒ महि॒ शर्म॑ भ॒द्रम्।
यो दे॒ह्यो अन॑मयद्वध॒स्नैः।
यो अर्य॑पत्नीरु॒षस॑श्च॒कार॑।
स नि॒रुध्या॒ नहु॑षो य॒ह्वो अ॒ग्निः।
विश॑श्चक्रे बलि॒हृतः॒ सहो॑भिः॥७१॥

%2.4.7.10
प्र स॒द्यो अ॑ग्ने॒ अत्ये᳚ष्य॒न्यान्।
आ॒विर्यस्मै॒ चारु॑तरो ब॒भूथ॑।
ई॒डेन्यो॑ वपु॒ष्यो॑ वि॒भावा᳚।
प्रि॒यो वि॒शामति॑थि॒र्मानु॑षीणाम्।
ब्रह्म॑ज्येष्ठा वी॒र्या॑ सम्भृ॑तानि।
ब्रह्माग्रे॒ ज्येष्ठं॒ दिव॒मा त॑तान।
ऋ॒तस्य॒ ब्रह्म॑ प्रथ॒मोत ज॑ज्ञे।
तेना॑र्‌\mbox{}हति॒ ब्रह्म॑णा॒ स्पर्धि॑तु॒ङ्कः।
ब्रह्म॒ स्रुचो॑ घृ॒तव॑तीः।
ब्रह्म॑णा॒ स्वर॑वो मि॒ताः॥७२॥

%2.4.7.11
ब्रह्म॑ य॒ज्ञस्य॒ तन्त॑वः।
ऋ॒त्विजो॒ ये ह॑वि॒ष्कृतः॑।
शृङ्गा॑णी॒वेच्छृ॒ङ्गिणा॒ꣳ॒ सन्द॑दृश्रिरे।
च॒षाल॑वन्तः॒ स्वर॑वः पृथि॒व्याम्।
ते दे॒वासः॒ स्वर॑वस्तस्थि॒वाꣳसः॑।
नमः॒ सखि॑भ्यः स॒न्नान्मा\-ऽव॑गात।
अ॒भि॒भूर॒ग्निर॑तर॒द्रजाꣳ॑सि।
स्पृधो॑ वि॒हत्य॒ पृत॑ना अभि॒श्रीः।
जु॒षा॒णो म॒ आहु॑तिं मामहिष्ट।
ह॒त्वा स॒पत्ना॒न्॒ वरि॑वस्करन्नः।
ईशा॑नं त्वा॒ भुव॑नानामभि॒श्रियम्᳚।
स्तौम्य॑ग्न उरु॒कृतꣳ॑ सु॒वीरम्᳚।
ह॒विर्जु॑षा॒णः स॒पत्नाꣳ॑ अभि॒भूर॑सि।
ज॒हि शत्रू॒ꣳ॒ रप॒ मृधो॑ नुदस्व॥७३॥\anuvakamend[वि॒शां ज॑यामसि जीरदानो॒ हर्या॒ विश्वा॒ दिवि॑ष्टिषु॒ वसू॑नि जिगी॒वान्थ्सहो॑भिर्मि॒ता न॑श्च॒त्वारि॑ च]

%2.4.8.1
स प्र॑त्न॒वन्नवी॑यसा।
अग्ने᳚ द्यु॒म्नेन॑ सं॒यता᳚।
बृ॒हत्त॑तन्थ भा॒नुना᳚।
नवं॒ नु स्तोम॑म॒ग्नये᳚।
दि॒वः श्ये॒नाय॑ जीजनम्।
वसोः᳚ कु॒विद्व॒नाति॑ नः।
स्वा॒रु॒हा यस्य॒ श्रियो॑ दृ॒शे।
र॒यिर्वी॒रव॑तो यथा।
अग्रे॑ य॒ज्ञस्य॒ चेत॑तः।
अदा᳚भ्यः पुरए॒ता॥७४॥

%2.4.8.2
अ॒ग्निर्वि॒शां मानु॑षीणाम्।
तूर्णी॒ रथः॒ सदा॒ नवः॑।
नव॒ꣳ॒ सोमा॑य वा॒जिने᳚।
आज्यं॒ पय॑सोऽजनि।
जुष्ट॒ꣳ॒ शुचि॑तमं॒ वसु॑।
नवꣳ॑ सोम जुषस्व नः।
पी॒यूष॑स्ये॒ह तृ॑प्णुहि।
यस्ते॑ भा॒ग ऋ॒ता व॒यम्।
नव॑स्य सोम ते व॒यम्।
आ सु॑म॒तिं वृ॑णीमहे॥७५॥

%2.4.8.3
स नो॑ रास्व सह॒स्रिणः॑।
नवꣳ॑ ह॒विर्जु॑षस्व नः।
ऋ॒तुभिः॑ सोम॒ भूत॑मम्।
तद॒ङ्ग प्रति॑\-हर्य नः।
राज᳚न्थ्सोम स्व॒स्तये᳚।
नव॒ꣴ॒स्तोम॒न्नवꣳ॑ ह॒विः।
इ॒न्द्रा॒ग्निभ्यां॒ नि वे॑दय।
तज्जु॑षेता॒ꣳ॒ सचे॑तसा।
शुचिं॒ नु स्तोमं॒ नव॑जातम॒द्य।
इन्द्रा᳚ग्नी वृत्रहणा जु॒षेथा᳚म्॥७६॥

%2.4.8.4
उ॒भा हि वाꣳ॑ सु॒हवा॒ जोह॑वीमि।
ता वाजꣳ॑ स॒द्य उ॑श॒ते धेष्ठा᳚।
अ॒ग्निरिन्द्रो॒ नव॑स्य नः।
अ॒स्य ह॒व्यस्य॑ तृप्यताम्।
इ॒ह दे॒वौ स॑ह॒स्रिणौ᳚।
य॒ज्ञं न॒ आ हि गच्छ॑ताम्।
वसु॑मन्तꣳ सुव॒र्विदम्᳚।
अ॒स्य ह॒व्यस्य॑ तृप्यताम्।
अ॒ग्निरिन्द्रो॒ नव॑स्य नः।
विश्वा᳚न्दे॒वाꣴस्त॑र्पयत॥७७॥

%2.4.8.5
ह॒विषो॒ऽस्य नव॑स्य नः।
सु॒व॒र्विदो॒ हि ज॑ज्ञि॒रे।
एदं ब॒र्॒हिः सु॒ष्टरी॑मा॒ नवे॑न।
अ॒यं य॒ज्ञो यज॑मानस्य भा॒गः।
अ॒यं ब॑भूव॒ भुव॑नस्य॒ गर्भः॑।
विश्वे॑ दे॒वा इ॒दम॒द्याग॑मिष्ठाः।
इ॒मे नु द्यावा॑पृथि॒वी स॒मीची᳚।
त॒न्वा॒ने य॒ज्ञं पु॑रु॒पेश॑सन्धि॒या।
आऽस्मै॑ पृणीतां॒ भुव॑नानि॒ विश्वा᳚।
प्र॒जां पुष्टि॑म॒मृतं॒ नवे॑न॥७८॥

%2.4.8.6
इ॒मे धे॒नू अ॒मृतं॒ ये दु॒हाते᳚।
पय॑स्वत्युत्त॒रामे॑तु॒ पुष्टिः॑।
इ॒मं य॒ज्ञं जु॒षमा॑णे॒ नवे॑न।
स॒मीची॒ द्यावा॑पृथि॒वी घृ॒ताची᳚।
यवि॑ष्ठो हव्य॒वाह॑नः।
चि॒त्रभा॑नुर्घृ॒तासु॑तिः।
नव॑जातो॒ वि रो॑चसे।
अग्ने॒ तत्ते॑ महित्व॒नम्।
त्वम॑ग्ने दे॒वता᳚भ्यः।
भा॒गे दे॑व॒ न मी॑यसे॥७९॥

%2.4.8.7
स ए॑ना वि॒द्वान् य॑क्ष्यसि।
नव॒ꣴ॒ स्तोमं॑ जुषस्व नः।
अ॒ग्निः प्र॑थ॒मः प्राश्ञा॑तु।
स हि वेद॒ यथा॑ ह॒विः।
शि॒वा अ॒स्मभ्य॒मोष॑धीः।
कृ॒णोतु॑ वि॒श्वच॑र्‌\mbox{}षणिः।
भ॒द्रान्नः॒ श्रेयः॒ सम॑नैष्ट देवाः।
त्वया॑\-ऽव॒सेन॒ सम॑शीमहि त्वा।
स नो॑ मयो॒भूः पि॑तो॒ आ वि॑शस्व।
शं तो॒काय॑ त॒नुवे᳚ स्यो॒नः।
ए॒तमु॒ त्यं मधु॑ना॒ संयु॑तं॒ यवम्᳚।
सर॑स्वत्या॒ अधि॑म॒नाव॑चर्कृषुः।
इन्द्र॑ आसी॒थ्सीर॑पतिः श॒तक्र॑तुः।
की॒नाशा॑ आसन्म॒रुतः॑ सु॒दान॑वः॥८०॥\anuvakamend[पु॒र॒ए॒ता वृ॑णीमहे जु॒षेथा᳚न्तर्पयता॒मृत॒न्नवे॑न मीयसे स्यो॒नश्च॒त्वारि॑ च]




\prashnaend{जुष्ट॒श्चक्षु॑षो॒ जुष्टी॑नरो नक्तञ्जा॒ता वृषा॒स उ॒त नो॒ वृषा᳚\-ऽस्य॒ꣳ॒शुः सप्र॑त्न॒वद॒ष्टौ॥८॥}{जुष्टो॑ म॒न्युर्भगो॒ जुष्टी॑ नरो॒ हरि॑वर्पस॒ङ्गिरः॒ शिप्रि॑न्वाजानामु॒त नः॑ प्रि॒या यद्वाग्वद॑न्ती॒ विश्वा॒ आशा॒ अशी॑तिः॥८०॥}{जुष्टः॑ सु॒दान॑वः॥}{हरिः॑ ओम्॥}{इति श्रीकृष्णयजुर्वेदीयतैत्तिरीयब्राह्मणे द्वितीयाष्टके चतुर्थः प्रपाठकः समाप्तः॥}
\clearpage
\sect{पञ्चमः प्रश्नः}
\setcounter{anuvakam}{0}
\dnsub{तैत्तिरीयब्राह्मणे द्वितीयाष्टके पञ्चमः प्रपाठकः}

%2.5.1.1
प्रा॒णो र॑क्षति॒ विश्व॒मेज॑त्।
इर्यो॑ भू॒त्वा ब॑हु॒धा ब॒हूनि॑।
स इथ्सर्वं॒ व्या॑नशे।
यो दे॒वो दे॒वेषु॑ वि॒भूर॒न्तः।
आवृ॑दू॒दात् क्षेत्रिय॑ध्व॒गद्वृषा᳚।
तमित्प्रा॒णं मन॒सोप॑ शिक्षत।
अग्रं॑ दे॒वाना॑मि॒दम॑त्तु नो ह॒विः।
मन॑स॒श्चित्ते॒दम्।
भू॒तं भव्यं॑ च गुप्यते।
तद्धि दे॒वेष्व॑ग्रि॒यम्॥१॥

%2.5.1.2
आ न॑ एतु पुरश्च॒रम्।
स॒ह दे॒वैरि॒मꣳ हवम्᳚।
मनः॒ श्रेय॑सिश्रेयसि।
कर्म॑न् य॒ज्ञप॑तिं॒ दध॑त्।
जु॒षतां᳚ मे॒ वागि॒दꣳ ह॒विः।
वि॒राड्दे॒वी पु॒रोहि॑ता।
ह॒व्य॒वाडन॑पायिनी।
यया॑ रू॒पाणि॑ बहु॒धा वद॑न्ति।
पेशाꣳ॑सि दे॒वाः प॑र॒मे ज॒नित्रे᳚।
सा नो॑ वि॒राडन॑पस्फुरन्ती॥२॥

%2.5.1.3
वाग्दे॒वी जु॑षतामि॒दꣳ ह॒विः।
चक्षु॑र्दे॒वानां॒ ज्योति॑र॒मृते॒ न्य॑क्तम्।
अ॒स्य वि॒ज्ञाना॑य बहु॒धा निधी॑यते।
तस्य॑ सु॒म्नम॑शीमहि।
मा नो॑ हासीद्विचक्ष॒णम्।
आयु॒रिन्नः॒ प्रती᳚र्यताम्।
अन॑न्धा॒श्चक्षु॑षा व॒यम्।
जी॒वा ज्योति॑रशीमहि।
सुव॒र्ज्योति॑रु॒तामृतम्᳚।
श्रोत्रे॑ण भ॒द्रमु॒त शृ॑ण्वन्ति स॒त्यम्।
श्रोत्रे॑ण॒ वाचं॑ बहु॒धोद्यमा॑नाम्।
श्रोत्रे॑ण॒ मोद॑श्च॒ मह॑श्च श्रूयते।
श्रोत्रे॑ण॒ सर्वा॒ दिश॒ आ शृ॑णोमि।
येन॒ प्राच्या॑ उ॒त द॑क्षि॒णा।
प्र॒तीच्यै॑ दि॒शः शृ॒ण्वन्त्यु॑त्त॒रात्।
तदिच्छ्रोत्रं॑ बहु॒धोद्यमा॑नम्।
अ॒रान्न ने॒मिः परि॒ सर्वं॑ बभूव॥३॥\anuvakamend[अ॒ग्रि॒यमन॑पस्फुरन्ती स॒त्यꣳ स॒प्त च॑]

%2.5.2.1
उ॒देहि॑ वाजि॒न्यो अ॑स्य॒फ्स्व॑न्तः।
इ॒दꣳ रा॒ष्ट्रमा वि॑श सू॒नृता॑वत्।
यो रोहि॑तो॒ विश्व॑मि॒दं ज॒जान॑।
स नो॑ रा॒ष्ट्रेषु॒ सुधि॑तान्दधातु।
रोहꣳ॑रोह॒ꣳ॒ रोहि॑त॒ आरु॑रोह।
प्र॒जाभि॒र्वृद्धिं॑ ज॒नुषा॑मु॒पस्थम्᳚।
ताभिः॒ सꣳर॑ब्धो अविद॒थ्षडु॒र्वीः।
गा॒तुं प्र॒पश्य॑न्नि॒ह रा॒ष्ट्रमा\-ऽहाः᳚।
आऽहा॑र्\mbox{}षीद्रा॒ष्ट्रमि॒ह रोहि॑तः।
मृधो॒ व्या᳚स्थ॒दभ॑यं नो अस्तु॥४॥

%2.5.2.2
अ॒स्मभ्यं॑ द्यावापृथिवी॒ शक्व॑रीभिः।
रा॒ष्ट्रं दु॑हाथामि॒ह रे॒वती॑भिः।
विम॑मर्\mbox{}श॒ रोहि॑तो वि॒श्वरू॑पः।
स॒मा॒च॒क्रा॒णः प्र॒रुहो॒ रुह॑श्च।
दिवं॑ ग॒त्वाय॑ मह॒ता म॑हि॒म्ना।
वि नो॑ रा॒ष्ट्रमु॑नत्तु॒ पय॑सा॒ स्वेन॑।
यास्ते॒ विश॒स्तप॑सा सं बभू॒वुः।
गा॒य॒त्रं व॒थ्समनु॒ तास्त॒ आऽगुः॑।
तास्त्वा वि॑शन्तु॒ मह॑सा॒ स्वेन॑।
सं मा॑ता पु॒त्रो अ॒भ्ये॑तु॒ रोहि॑तः॥५॥

%2.5.2.3
यू॒यमु॑ग्रा मरुतः पृश्ञिमातरः।
इन्द्रे॑ण स॒युजा॒ प्रमृ॑णीथ॒ शत्रून्॑।
आ वो॒ रोहि॑तो अशृणोदभिद्यवः।
त्रिस॑प्तासो मरुतः स्वादुसम्मुदः।
रोहि॑तो॒ द्यावा॑पृथि॒वी ज॑जान।
तस्मि॒ꣴ॒स्तन्तुं॑ परमे॒ष्ठी त॑तान।
तस्मि॑ञ्छिश्रिये अ॒ज एक॑पात्।
अदृꣳ॑ह॒द्द्यावा॑पृथि॒वी बले॑न।
रोहि॑तो॒ द्यावा॑पृथि॒वी अ॑दृꣳहत्।
तेन॒ सुवः॑ स्तभि॒तं तेन॒ नाकः॑॥६॥

%2.5.2.4
सो अ॒न्तरि॑क्षे॒ रज॑सो वि॒मानः॑।
तेन॑ दे॒वाः सुव॒रन्व॑विन्दन्।
सु॒शेवं॑ त्वा भा॒नवो॑ दीदि॒वाꣳसम्᳚।
सम॑ग्रासो जु॒ह्वो॑ जातवेदः।
उ॒क्षन्ति॑ त्वा वा॒जिन॒मा घृ॒तेन॑।
सꣳस॑मग्ने युवसे॒ भोज॑नानि।
अग्ने॒ शर्ध॑ मह॒ते सौभ॑गाय।
तव॑ द्यु॒म्नान्यु॑त्त॒मानि॑ सन्तु।
सञ्जा᳚स्प॒त्यꣳ सु॒यम॒मा कृ॑णुष्व।
श॒त्रू॒य॒ताम॒भि ति॑ष्ठा॒ महाꣳ॑सि॥७॥\anuvakamend[अ॒स्त्वे॒तु॒ रोहि॑तो॒ नाको॒ महाꣳ॑सि]

%2.5.3.1
पुन॑र्न॒ इन्द्रो॑ म॒घवा॑ ददातु।
धना॑नि श॒क्रो धन्यः॑ सु॒राधाः᳚।
अ॒र्वा॒चीनं॑ कृणुतां याचि॒तो मनः॑।
श्रु॒ष्टी नो॑ अ॒स्य ह॒विषो॑ जुषा॒णः।
यानि॑ नोऽजि॒नं धना॑नि।
ज॒हर्थ॑ शूर म॒न्युना᳚।
इन्द्रानु॑विन्द न॒स्तानि॑।
अ॒नेन॑ ह॒विषा॒ पुनः॑।
इन्द्र॒ आशा᳚भ्यः॒ परि॑।
सर्वा॒भ्यो\-ऽभ॑यं करत्॥८॥

%2.5.3.2
जेता॒ शत्रू॒न्॒ विच॑र्\mbox{}षणिः।
आकू᳚त्यै त्वा॒ कामा॑य त्वा स॒मृधे᳚ त्वा।
पु॒रो द॑धे अमृत॒त्वाय॑ जी॒वसे᳚।
आकू॑तिम॒स्याव॑से।
काम॑मस्य॒ समृ॑द्ध्यै।
इन्द्र॑स्य युञ्जते॒ धियः॑।
आकू॑तिं दे॒वीं मन॑सः पु॒रो द॑धे।
य॒ज्ञस्य॑ मा॒ता सु॒हवा॑ मे अस्तु।
यदि॒च्छामि॒ मन॑सा॒ सका॑मः।
वि॒देय॑मेन॒द्धृद॑ये॒ निवि॑ष्टम्॥९॥

%2.5.3.3
सेद॒ग्निर॒ग्नीꣳरत्ये᳚त्य॒न्यान्।
यत्र॑ वा॒जी तन॑यो वी॒डुपा॑णिः।
स॒हस्र॑पाथा अ॒क्षरा॑ स॒मेति॑।
आशा॑नां त्वा\-ऽऽशापा॒लेभ्यः॑।
च॒तुर्भ्यो॑ अ॒मृते᳚भ्यः।
इ॒दं भू॒तस्याध्य॑क्षेभ्यः।
वि॒धेम॑ ह॒विषा॑ व॒यम्।
विश्वा॒ आशा॒ मधु॑ना॒ सꣳ सृ॑जामि।
अ॒न॒मी॒वा आप॒ ओष॑धयो भवन्तु।
अ॒यं यज॑मानो॒ मृधो॒ व्य॑स्यताम्॥१०॥

%2.5.3.4
अगृ॑भीताः प॒शवः॑ सन्तु॒ सर्वे᳚।
अ॒ग्निः सोमो॒ वरु॑णो मि॒त्र इन्द्रः॑।
बृह॒स्पतिः॑ सवि॒ता यः स॑ह॒स्री।
पू॒षा नो॒ गोभि॒रव॑सा॒ सर॑स्वती।
त्वष्टा॑ रू॒पाणि॒ सम॑नक्तु य॒ज्ञैः।
त्वष्टा॑ रू॒पाणि॒ दध॑ती॒ सर॑स्वती।
पू॒षा भगꣳ॑ सवि॒ता नो॑ ददातु।
बृह॒स्पति॒र्दद॒दिन्द्रः॑ स॒हस्रम्᳚।
मि॒त्रो दा॒ता वरु॑णः॒ सोमो॑ अ॒ग्निः॥११॥\anuvakamend[क॒र॒न्निवि॑ष्टमस्यता॒न्नव॑ च]

%2.5.4.1
आ नो॑ भर॒ भग॑मिन्द्र द्यु॒मन्तम्᳚।
नि ते॑ दे॒ष्णस्य॑ धीमहि प्ररे॒के।
उ॒र्व इ॑व पप्रथे॒ कामो॑ अ॒स्मे।
तमापृ॑णा वसुपते॒ वसू॑नाम्।
इ॒मं कामं॑ मन्दया॒ गोभि॒रश्वैः᳚।
च॒न्द्रव॑ता॒ राध॑सा प॒प्रथ॑श्च।
सु॒व॒र्यवो॑ म॒तिभि॒स्तुभ्यं॒ विप्राः᳚।
इन्द्रा॑य॒ वाहः॑ कुशि॒कासो॑ अक्रन्।
इन्द्र॑स्य॒ नु वी॒र्या॑णि॒ प्रवो॑चम्।
यानि॑ च॒कार॑ प्रथ॒मानि॑ व॒ज्री॥१२॥

%2.5.4.2
अह॒न्नहि॒मन्व॒पस्त॑तर्द।
प्रव॒क्षणा॑ अभिन॒त्पर्व॑तानाम्।
अह॒न्नहिं॒ पर्व॑ते शिश्रिया॒णम्।
त्वष्टा᳚\-ऽस्मै॒ वज्रꣴ॑ स्व॒र्य॑न्ततक्ष।
वा॒श्रा इ॑व धे॒नवः॒ स्यन्द॑मानाः।
अञ्जः॑ समु॒द्रमव॑ जग्मु॒रापः॑।
वृ॒षा॒यमा॑णो\-ऽवृणीत॒ सोमम्᳚।
त्रिक॑द्रुकेष्वपिबथ्सु॒तस्य॑।
आ साय॑कं म॒घवा॑ दत्त॒ वज्रम्᳚।
अह॑न्नेनं प्रथम॒जा मही॑नाम्॥१३॥

%2.5.4.3
यदिन्द्राह॑न्प्रथम॒जा मही॑नाम्।
आन्मा॒यिना॒ममि॑नाः॒ प्रोत मा॒याः।
आथ्सूर्यं॑ ज॒नय॒न्द्यामु॒षासम्᳚।
ता॒दीक्ना॒ शत्रू॒न्न किला॑विविथ्से।
अह॑न्वृ॒त्रं वृ॑त्र॒तरं॒ व्यꣳसम्᳚।
इन्द्रो॒ वज्रे॑ण मह॒ता व॒धेन॑।
स्कन्धाꣳ॑सीव॒ कुलि॑शेना॒विवृ॑क्णा।
अहिः॑ शयत उप॒पृक्पृ॑थि॒व्याम्।
अ॒यो॒ध्येव दु॒र्मद॒ आ हि जु॒ह्वे।
म॒हा॒वी॒रं तु॑विबा॒धमृ॑जी॒षम्॥१४॥

%2.5.4.4
नाता॑रीरस्य॒ समृ॑तिं व॒धाना᳚म्।
सꣳ रु॒जानाः᳚ पिपिष॒ इन्द्र॑शत्रुः।
विश्वो॒ विहा॑या अर॒तिः।
वसु॑र्दधे॒ हस्ते॒ दक्षि॑णे।
त॒रणि॒र्न शि॑श्रथत्।
श्र॒व॒स्य॑या॒ न शि॑श्रथत्।
विश्व॑स्मा॒ इदि॑षुध्य॒से।
दे॒व॒त्रा ह॒व्यमूहि॑षे।
विश्व॑स्मा॒ इथ्सु॒कृते॒ वार॑मृण्वति।
अ॒ग्निर्द्वारा॒ व्यृ॑ण्वति॥१५॥

%2.5.4.5
उदु॒ज्जिहा॑नो अ॒भि काम॑मी॒रयन्॑।
प्र॒पृ॒ञ्चन्विश्वा॒ भुव॑नानि पू॒र्वथा᳚।
आ के॒तुना॒ सुष॑मिद्धो॒ यजि॑ष्ठः।
कामं॑ नो अग्ने अ॒भिह॑र्य दि॒ग्भ्यः।
जु॒षा॒णो ह॒व्यम॒मृते॑षु दू॒ढ्यः॑।
आ नो॑ र॒यिं ब॑हु॒लां गोम॑ती॒मिषम्᳚।
नि धे॑हि॒ यक्ष॑द॒मृते॑षु॒ भूषन्॑।
अश्वि॑ना य॒ज्ञमाग॑तम्।
दा॒शुषः॒ पुरु॑दꣳससा।
पू॒षा र॑क्षतु नो र॒यिम्॥१६॥

%2.5.4.6
इ॒मं य॒ज्ञम॒श्विना॑ व॒र्धय॑न्ता।
इ॒मौ र॒यिं यज॑मानाय धत्तम्।
इ॒मौ प॒शून्र॑क्षतां वि॒श्वतो॑ नः।
पू॒षा नः॑ पातु॒ सद॒मप्र॑यच्छन्।
प्रते॑ म॒हे स॑रस्वति।
सुभ॑गे॒ वाजि॑नीवति।
स॒त्य॒वाचे॑ भरे म॒तिम्।
इ॒दं ते॑ ह॒व्यं घृ॒तव॑थ्सरस्वति।
स॒त्य॒वाचे॒ प्रभ॑रेमा ह॒वीꣳषि॑।
इ॒मानि॑ ते दुरि॒ता सौभ॑गानि।
तेभि॑र्व॒यꣳ सु॒भगा॑सः स्याम॥१७॥\anuvakamend[व॒ज्र्यही॑नामृजी॒षं व्यृ॑ण्वति रक्षतु नो र॒यिꣳ सौभ॑गा॒न्येकं॑ च]

%2.5.5.1
य॒ज्ञो रा॒यो य॒ज्ञ ई॑शे॒ वसू॑नाम्।
य॒ज्ञः स॒स्याना॑मु॒त सु॑क्षिती॒नाम्।
य॒ज्ञ इ॒ष्टः पू॒र्वचि॑त्तिं दधातु।
य॒ज्ञो ब्र॑ह्म॒ण्वाꣳ अप्ये॑तु दे॒वान्।
अ॒यं य॒ज्ञो व॑र्धतां॒ गोभि॒रश्वैः᳚।
इ॒यं वेदिः॑ स्वप॒त्या सु॒वीरा᳚।
इ॒दं ब॒र्॒हिरति॑ ब॒र्॒हीꣴष्य॒न्या।
इ॒मं य॒ज्ञं विश्वे॑ अवन्तु दे॒वाः।
भग॑ ए॒व भग॑वाꣳ अस्तु देवाः।
तेन॑ व॒यं भग॑वन्तः स्याम॥१८॥

%2.5.5.2
तं त्वा॑ भग॒ सर्व॒ इज्जो॑हवीमि।
स नो॑ भग पुरए॒ता भ॑वे॒ह।
भग॒ प्रणे॑त॒र्भग॒ सत्य॑राधः।
भगे॒मां धिय॒मुद॑व॒ दद॑न्नः।
भग॒ प्र णो॑ जनय॒ गोभि॒रश्वैः᳚।
भग॒ प्र नृभि॑र्नृ॒वन्तः॑ स्याम।
शश्व॑तीः॒ समा॒ उप॑यन्ति लो॒काः।
शश्व॑तीः॒ समा॒ उप॑य॒न्त्यापः॑।
इ॒ष्टं पू॒र्तꣳ शश्व॑तीना॒ꣳ॒ समा॑नाꣳ शाश्व॒तेन॑।
ह॒विषे॒ष्ट्वा\-ऽन॒न्तं लो॒कं पर॒मा रु॑रोह॥१९॥

%2.5.5.3
इ॒यमे॒व सा या प्र॑थ॒मा व्यौच्छ॑त्।
सा रू॒पाणि॑ कुरुते॒ पञ्च॑ दे॒वी।
द्वे स्वसा॑रौ वयत॒स्तन्त्र॑मे॒तत्।
स॒ना॒तनं॒ वित॑त॒ꣳ॒ षण्म॑यूखम्।
अवा॒न्याꣴस्तन्तू᳚न्कि॒रतो॑ ध॒त्तो अ॒न्यान्।
नाव॑पृ॒ज्याते॒ न ग॑माते॒ अन्तम्᳚।
आ वो॑ यन्तूदवा॒हासो॑ अ॒द्य।
वृष्टिं॒ ये विश्वे॑ म॒रुतो॑ जु॒नन्ति॑।
अ॒यं यो अ॒ग्निर्म॑रुतः॒ समि॑द्धः।
ए॒तं जु॑षध्वं कवयो युवानः॥२०॥

%2.5.5.4
धा॒रा॒व॒रा म॒रुतो॑ धृ॒ष्णुवो॑जसः।
मृ॒गा न भी॒मास्त॑वि॒षेभि॑\-रू॒र्मिभिः॑।
अ॒ग्नयो॒ न शु॑शुचा॒ना ऋ॑जी॒षिणः॑।
भ्रुमि॒न्धम॑न्त॒ उप॒ गा अ॑वृण्वत।
वि च॑क्रमे॒ त्रिर्दे॒वः।
आ वे॒धसं॒ नील॑पृष्ठं बृ॒हन्तम्᳚।
बृह॒स्पति॒ꣳ॒ सद॑ने सादयध्वम्।
सा॒दद्यो॑निं॒ दम॒ आ दी॑दि॒वाꣳसम्᳚।
हिर॑ण्यवर्णमरु॒षꣳ स॑पेम।
स हि शुचिः॑ श॒तप॑त्रः॒ स शु॒न्ध्यूः॥२१॥

%2.5.5.5
हिर॑ण्यवाशीरिषि॒रः सु॑व॒र्॒षाः।
बृह॒स्पतिः॒ स स्वा॑वे॒श ऋ॒ष्वाः।
पू॒रू सखि॑भ्य आसु॒तिं क॑रिष्ठः।
पूष॒ꣴ॒ स्तव॑ व्र॒ते व॒यम्।
नरि॑ष्येम क॒दाच॒न।
स्तो॒तार॑स्त इ॒ह स्म॑सि।
यास्ते॑ पूष॒न्ना वो॑ अ॒न्तः स॑मु॒द्रे।
हि॒र॒ण्ययी॑र॒न्तरि॑क्षे॒ चर॑न्ति।
याभि॑र्यासि दू॒त्याꣳ सूर्य॑स्य।
कामे॑न कृ॒तश्रव॑ इ॒च्छमा॑नः॥२२॥

%2.5.5.6
अर॑ण्या॒न्यर॑ण्यान्य॒सौ।
या प्रेव॒ नश्य॑सि।
क॒था ग्रामं॒ न पृ॑च्छसि।
न त्वा॒भीरि॑व विन्दती (३)।
वृ॒षा॒र॒वाय॒ वद॑ते।
यदु॒पाव॑ति चिच्चि॒कः।
आ॒घा॒टीभि॑रिव धा॒वयन्॑।
अ॒र॒ण्या॒निर्म॑हीयते।
उ॒त गाव॑ इवादन्।
उ॒तो वेश्मे॑व दृश्यते॥२३॥

%2.5.5.7
उ॒तो अ॑रण्या॒निः सा॒यम्।
श॒क॒टीरि॑व सर्जति।
गाम॒ङ्गैष॒ आ ह्व॑यति।
दार्व॒ङ्गैष॒ उपा॑वधीत्।
वस॑न्नरण्या॒न्याꣳ सा॒यम्।
अक्रु॑क्ष॒दिति॑ मन्यते।
न वा अ॑रण्या॒निर्\mbox{}ह॑न्ति।
अ॒न्यश्चेन्नाभि॒गच्छ॑ति।
स्वा॒दोः फल॑स्य ज॒ग्ध्वा।
यत्र॒ कामं॒ नि प॑द्यते।
आञ्ज॑नगन्धीꣳ सुर॒भीम्।
ब॒ह्व॒न्नामकृ॑षीवलाम्।
प्राहं मृ॒गाणां᳚ मा॒तरम्᳚।
अ॒र॒ण्या॒नीम॑शꣳसिषम्॥२४॥\anuvakamend[स्या॒म॒ रु॒रो॒ह॒ यु॒वा॒नः॒ शु॒न्ध्यूरि॒च्छमा॑नो दृश्यते॒ निप॑द्यते च॒त्वारि॑ च]

%2.5.6.1
वार्त्र॑हत्याय॒ शव॑से।
पृ॒त॒ना॒साह्या॑य च।
इन्द्र॒ त्वा व॑र्तयामसि।
सु॒ब्रह्मा॑णं वी॒रव॑न्तं बृ॒हन्तम्᳚।
उ॒रुं ग॑भी॒रं पृ॒थुबु॑ध्नमिन्द्र।
श्रु॒तर्\mbox{}षि॑मु॒ग्रम॑भिमाति॒षाहम्᳚।
अ॒स्मभ्यं॑ चि॒त्रं वृष॑णꣳ र॒यिं दाः᳚।
क्षे॒त्रि॒यै त्वा॒ निर्\mbox{}ऋ॑त्यै त्वा।
द्रु॒हो मु॑ञ्चामि॒ वरु॑णस्य॒ पाशा᳚त्।
अ॒ना॒गसं॒ ब्रह्म॑णे त्वा करोमि॥२५॥

%2.5.6.2
शि॒वे ते॒ द्यावा॑पृथि॒वी उ॒भे इ॒मे।
शं ते॑ अ॒ग्निः स॒हाद्भिर॑स्तु।
शं द्यावा॑पृथि॒वी स॒हौष॑धीभिः।
शम॒न्तरि॑क्षꣳ स॒ह वाते॑न ते।
शं ते॒ चत॑स्रः प्र॒दिशो॑ भवन्तु।
या दैवी॒श्चत॑स्रः प्र॒दिशः॑।
वात॑पत्नीर॒भि सूर्यो॑ विच॒ष्टे।
तासां त्वा ज॒रस॒ आ द॑धामि।
प्र यक्ष्म॑ एतु॒ निर्\mbox{}ऋ॑तिं परा॒चैः।
अमो॑चि॒ यक्ष्मा᳚द्दुरि॒तादव॑र्त्यै॥२६॥

%2.5.6.3
द्रु॒हः पाशा॒न्निर्\mbox{}ऋ॑त्यै॒ चोद॑मोचि।
अहा॒ अव॑र्ति॒मवि॑दथ्स्यो॒नम्।
अप्य॑भूद्भ॒द्रे सु॑कृ॒तस्य॑ लो॒के।
सूर्य॑मृ॒तं तम॑सो॒ ग्राह्या॒ यत्।
दे॒वा अमु॑ञ्च॒न्नसृ॑ज॒न्व्ये॑नसः।
ए॒वम॒हमि॒मं क्षे᳚त्रि॒याज्जा॑मिश॒ꣳ॒सात्।
द्रु॒हो मु॑ञ्चामि॒ वरु॑णस्य॒ पाशा᳚त्।
बृह॑स्पते यु॒वमिन्द्र॑श्च॒ वस्वः॑।
दि॒व्यस्ये॑शाथे उ॒त पार्थि॑वस्य।
ध॒त्तꣳ र॒यिꣴ स्तु॑व॒ते की॒रये॑चित्॥२७॥

%2.5.6.4
यू॒यं पा॑त स्व॒स्तिभिः॒ सदा॑ नः।
दे॒वा॒युध॒मिन्द्र॒मा जोहु॑वानाः।
वि॒श्वा॒वृध॑म॒भि ये रक्ष॑माणाः।
येन॑ ह॒ता दी॒र्घमध्वा॑न॒मायन्॑।
अ॒न॒न्तमर्थ॒मनि॑वर्थ्स्यमानाः।
यत्ते॑ सुजाते हि॒मव॑थ्सु भेष॒जम्।
म॒यो॒भूः शन्त॑मा॒ यद्धृ॒दो\-ऽसि॑।
ततो॑ नो देहि सीबले।
अ॒दो गि॒रिभ्यो॒ अधि॒ यत्प्र॒धाव॑सि।
स॒ꣳ॒शोभ॑माना क॒न्ये॑व शुभ्रे॥२८॥

%2.5.6.5
तां त्वा॒ मुद्ग॑ला ह॒विषा॑ वर्धयन्ति।
सा नः॑ सीबले र॒यिमा भा॑जये॒ह।
पूर्वं॑ देवा॒ अप॑रेणानु॒पश्यं॒ जन्म॑भिः।
जन्मा॒न्यव॑रैः॒ परा॑णि।
वेदा॑नि देवा अ॒यम॒स्मीति॒ माम्।
अ॒हꣳ हि॒त्वा शरी॑रं ज॒रसः॑ प॒रस्ता᳚त्।
प्रा॒णा॒पा॒नौ चक्षुः॒ श्रोत्रम्᳚।
वाचं॒ मन॑सि॒ सम्भृ॑ताम्।
हि॒त्वा शरी॑रं ज॒रसः॑ प॒रस्ता᳚त्।
आ भूतिं॒ भूतिं॑ व॒यम॑श्ञवामहै।
इ॒मा ए॒व ता उ॒षसो॒ याः प्र॑थ॒मा व्यौच्छन्॑।
ता दे॒व्यः॑ कुर्वते॒ पञ्च॑रू॒पा।
शश्व॑ती॒र्नाव॑पृज्यन्ति।
न ग॑म॒न्त्यन्तम्᳚॥२९॥\anuvakamend[क॒रो॒म्यव॑र्त्यै चिच्छुभ्रे\-ऽश्ञवामहै च॒त्वारि॑ च]

%2.5.7.1
वसू॑नां॒ त्वा\-ऽधी॑तेन।
रु॒द्राणा॑मू॒र्म्या।
आ॒दि॒त्यानां॒ तेज॑सा।
विश्वे॑षां दे॒वानां॒ क्रतु॑ना।
म॒रुता॒मेम्ना॑ जुहोमि॒ स्वाहा᳚।
अ॒भि\-भू॑ति\-र॒हमाग॑मम्।
इन्द्र॑सखा स्वा॒युधः॑।
आस्वाशा॑सु दु॒ष्षहः॑।
इ॒दं वर्चो॑ अ॒ग्निना॑ द॒त्तमागा᳚त्।
यशो॒ भर्गः॒ सह॒ ओजो॒ बलं॑ च॥३०॥

%2.5.7.2
दी॒र्घा॒यु॒त्वाय॑ श॒तशा॑रदाय।
प्रति॑\-गृभ्णामि मह॒ते वी॒र्या॑य।
आयु॑रसि वि॒श्वायु॑रसि।
स॒र्वायु॑रसि॒ सर्व॒मायु॑रसि।
सर्वं॑ म॒ आयु॑र्भूयात्।
सर्व॒मायु॑र्गेषम्।
भूर्भुवः॒ सुवः॑।
अ॒ग्निर्धर्मे॑णान्ना॒दः।
मृ॒त्युर्धर्मे॒णान्न॑पतिः।
ब्रह्म॑ क्ष॒त्रꣴ स्वाहा᳚॥३१॥

%2.5.7.3
प्र॒जा\-प॑तिः प्रणे॒ता।
बृह॒स्पतिः॑ पुरए॒ता।
य॒मः पन्थाः᳚।
च॒न्द्रमाः᳚ पुनर॒सुः स्वाहा᳚।
अ॒ग्निर॑न्ना॒दो\-ऽन्न॑पतिः।
अ॒न्नाद्य॑म॒स्मिन् य॒ज्ञे यज॑मानाय ददातु॒ स्वाहा᳚।
सोमो॒ राजा॒ राज॑पतिः।
रा॒ज्यम॒स्मिन् य॒ज्ञे यज॑मानाय ददातु॒ स्वाहा᳚।
वरु॑णः स॒म्राट्थ्स॒म्राट्प॑तिः।
साम्रा᳚ज्यम॒स्मिन् य॒ज्ञे यज॑मानाय ददातु॒ स्वाहा᳚॥३२॥

%2.5.7.4
मि॒त्रः क्ष॒त्रं क्ष॒त्रप॑तिः।
क्ष॒त्रम॒स्मिन् य॒ज्ञे यज॑मानाय ददातु॒ स्वाहा᳚।
इन्द्रो॒ बलं॒ बल॑पतिः।
बल॑म॒स्मिन् य॒ज्ञे यज॑मानाय ददातु॒ स्वाहा᳚।
बृह॒स्पति॒र्ब्रह्म॒ ब्रह्म॑पतिः।
ब्रह्मा॒स्मिन् य॒ज्ञे यज॑मानाय ददातु॒ स्वाहा᳚।
स॒वि॒ता रा॒ष्ट्रꣳ रा॒ष्ट्रप॑तिः।
रा॒ष्ट्रम॒स्मिन् य॒ज्ञे यज॑मानाय ददातु॒ स्वाहा᳚।
पू॒षा वि॒शां विट्प॑तिः।
विश॑म॒स्मिन् य॒ज्ञे यज॑मानाय ददातु॒ स्वाहा᳚।
सर॑स्वती॒ पुष्टिः॒ पुष्टि॑पत्नी।
पुष्टि॑म॒स्मिन् य॒ज्ञे यज॑मानाय ददातु॒ स्वाहा᳚।
त्वष्टा॑ पशू॒नां मि॑थु॒नानाꣳ॑ रूप॒कृद्रू॒पप॑तिः।
रु॒पेणा॒स्मिन् य॒ज्ञे यज॑मानाय प॒शून्द॑दातु॒ स्वाहा᳚॥३३॥\anuvakamend[च॒ स्वाहा॒ साम्रा᳚ज्यम॒स्मिन् य॒ज्ञे यज॑मानाय ददातु॒ स्वाहा॒ विश॑म॒स्मिन् य॒ज्ञे यज॑मानाय ददातु॒ स्वाहा॑ च॒त्वारि॑ च (अ॒ग्निः सोमो॒ वरु॑णो मि॒त्र इन्द्रो॒ बृह॒स्पतिः॑ सवि॒ता पू॒षा सर॑स्वती॒ त्वष्टा॒ दश॑॥)]

%2.5.8.1
स ईं᳚ पाहि॒ य ऋ॑जी॒षी तरु॑त्रः।
यः शिप्र॑वान्वृष॒भो यो म॑ती॒नाम्।
यो गो᳚त्र॒भिद्व॑ज्र॒भृद्यो ह॑रि॒ष्ठाः।
स इ॑न्द्र चि॒त्राꣳ अ॒भि तृ॑न्धि॒ वाजान्॑।
आ ते॒ शुष्मो॑ वृष॒भ ए॑तु प॒श्चात्।
ओत्त॒राद॑ध॒रागा पु॒रस्ता᳚त्।
आ वि॒श्वतो॑ अ॒भिसमे᳚त्व॒र्वाङ्।
इन्द्र॑ द्यु॒म्नꣳ सुव॑र्वद्धेह्य॒स्मे।
प्रोष्व॑स्मै पुरोर॒थम्।
इन्द्रा॑य शू॒षम॑र्चत॥३४॥

%2.5.8.2
अ॒भीके॑ चिदु लोक॒कृत्।
स॒ङ्गे स॒मथ्सु॑ वृत्र॒हा।
अ॒स्माकं॑ बोधि चोदि॒ता।
नभ॑न्तामन्य॒केषा᳚म्।
ज्या॒का अधि॒ धन्व॑सु।
इन्द्रं॑ व॒यꣳ शु॑ना॒सीरम्᳚।
अ॒स्मिन् य॒ज्ञे ह॑वामहे।
आ वाजै॒रुप॑ नो गमत्।
इन्द्रा॑य॒ शुना॒सीरा॑य।
स्रु॒चा जु॑हुत नो ह॒विः॥३५॥

%2.5.8.3
जु॒षतां॒ प्रति॒ मेधि॑रः।
प्र ह॒व्यानि॑ घृ॒तव॑न्त्यस्मै।
हर्य॑श्वाय भरता स॒जोषाः᳚।
इन्द्र॒र्तुभि॒र्ब्रह्म॑णा वावृधा॒नः।
शु॒ना॒सी॒री ह॒विरि॒दं जु॑षस्व।
वयः॑ सुप॒र्णा उप॑सेदु॒रिन्द्रम्᳚।
प्रि॒यमे॑धा॒ ऋष॑यो॒ नाध॑मानाः।
अप॑ ध्वा॒न्तमू᳚र्णु॒हि पू॒र्धि चक्षुः॑।
मु॒मु॒ग्ध्य॑स्मान्नि॒धये॑ऽव ब॒द्धान्।
बृ॒हदिन्द्रा॑य गायत॥३६॥

%2.5.8.4
मरु॑तो वृत्र॒हन्त॑मम्।
येन॒ ज्योति॒रज॑नयन्नृता॒वृधः॑।
दे॒वं दे॒वाय॒ जागृ॑वि।
कामि॒हैकाः॒ क इ॒मे प॑त॒ङ्गाः।
मा॒न्था॒लाः कुलि॒परि॑मापतन्ति।
अना॑वृतैना॒न्प्रध॑मन्तु दे॒वाः।
सौप॑र्णं॒ चक्षु॑स्त॒नुवा॑ विदेय।
ए॒वा व॑न्दस्व॒ वरु॑णं बृ॒हन्तम्᳚।
न॒म॒स्याधीर॑म॒मृत॑स्य गो॒पाम्।
स नः॒ शर्म॑ त्रि॒वरू॑थं॒ वियꣳ॑सत्॥३७॥

%2.5.8.5
यू॒यं पा॑त स्व॒स्तिभिः॒ सदा॑ नः।
नाके॑ सुप॒र्णमुप॒ यत्पत॑न्तम्।
हृ॒दा वेन॑न्तो अ॒भ्यच॑क्षत त्वा।
हिर॑ण्यपक्षं॒ वरु॑णस्य दू॒तम्।
य॒मस्य॒ योनौ॑ शकु॒नं भु॑र॒ण्युम्।
शं नो॑ दे॒वीर॒भिष्ट॑ये।
आपो॑ भवन्तु पी॒तये᳚।
शं योर॒भि स्र॑वन्तु नः।
ईशा॑ना॒ वार्या॑णाम्।
क्षय॑न्तीश्चर्\mbox{}षणी॒नाम्॥३८॥

%2.5.8.6
अ॒पो या॑चामि भेष॒जम्।
अ॒फ्सु मे॒ सोमो॑ अब्रवीत्।
अ॒न्तर्विश्वा॑नि भेष॒जा।
अ॒ग्निं च॑ वि॒श्वश॑म्भुवम्।
आप॑श्च वि॒श्वभे॑षजीः।
यद॒फ्सु ते॑ सरस्वति।
गोष्वश्वे॑षु॒ यन्मधु॑।
तेन॑ मे वाजिनीवति।
मुख॑मङ्ग्धि सरस्वति।
या सर॑स्वती वैशम्भ॒ल्या॥३९॥

%2.5.8.7
तस्यां᳚ मे रास्व।
तस्या᳚स्ते भक्षीय।
तस्या᳚स्ते भूयिष्ठ॒भाजो॑ भूयास्म।
अ॒हं त्वद॑स्मि॒ मद॑सि॒ त्वमे॒तत्।
ममा॑सि॒ योनि॒स्तव॒ योनि॑रस्मि।
ममै॒व सन्वह॑ ह॒व्यान्य॑ग्ने।
पु॒त्रः पि॒त्रे लो॑क॒कृज्जा॑तवेदः।
इ॒हैव सन्तत्र॒ सन्तं॑ त्वा\-ऽग्ने।
प्रा॒णेन॑ वा॒चा मन॑सा बिभर्मि।
ति॒रो मा॒ सन्त॒मायु॒र्मा प्रहा॑सीत्॥४०॥

%2.5.8.8
ज्योति॑षा त्वा वैश्वान॒रेणोप॑तिष्ठे।
अ॒यं ते॒ योनि॑र्\mbox{}ऋ॒त्वियः॑।
यतो॑ जा॒तो अरो॑चथाः।
तं जा॒नन्न॑ग्न॒ आरो॑ह।
अथा॑ नो वर्धया र॒यिम्।
या ते॑ अग्ने य॒ज्ञिया॑ त॒नूस्तयेह्यारो॑हा॒त्मा\-ऽऽत्मानम्᳚।
अच्छा॒ वसू॑नि कृ॒ण्वन्न॒स्मे नर्या॑ पु॒रूणि॑।
य॒ज्ञो भू॒त्वा य॒ज्ञमा सी॑द॒ स्वां योनिम्᳚।
जात॑वेदो॒ भुव॒ आ जाय॑मानः॒ सक्ष॑य॒ एहि॑।
उ॒पाव॑रोह जातवेदः॒ पुन॒स्त्वम्॥४१॥

%2.5.8.9
दे॒वेभ्यो॑ ह॒व्यं व॑ह नः प्रजा॒नन्।
आयुः॑ प्र॒जाꣳ र॒यिम॒स्मासु॑ धेहि।
अज॑स्रो दीदिहि नो दुरो॒णे।
तमिन्द्रं॑ जोहवीमि म॒घवा॑नमु॒ग्रम्।
स॒त्रा दधा॑न॒मप्र॑तिष्कुत॒ꣳ॒ शवाꣳ॑सि।
मꣳहि॑ष्ठो गी॒र्भिरा च॑ य॒ज्ञियो॑\-ऽव॒वर्त॑त्।
रा॒ये नो॒ विश्वा॑ सु॒पथा॑ कृणोतु व॒ज्री।
त्रिक॑द्रुकेषु महि॒षो यवा॑शिरं तुवि॒शुष्म॑स्तृ॒पत्।
सोम॑मपिब॒द्विष्णु॑ना सु॒तं यथा\-ऽव॑शत्।
स ईं᳚ ममाद॒ महि॒ कर्म॒ कर्त॑वे म॒हामु॒रुम्॥४२॥

%2.5.8.10
सैनꣳ॑ सश्चद्दे॒वं दे॒वः स॒त्यमिन्दुꣳ॑ स॒त्य इन्द्रः॑।
वि॒दद्यती॑ स॒रमा॑ रु॒ग्णमद्रेः᳚।
महि॒ पाथः॑ पू॒र्व्यꣳ स॒द्ध्रिय॑क्कः।
अग्रं॑ नयथ्सु॒पद्यक्ष॑राणाम्।
अच्छा॒ रवं॑ प्रथ॒मा जा॑न॒तीगा᳚त्।
वि॒दद्गव्यꣳ॑ स॒रमा॑ दृ॒ढमू॒र्वम्।
येना॒नुकं॒ मानु॑षी॒ भोज॑ते॒ विट्।
आ ये विश्वाः᳚ स्वप॒त्यानि॑ च॒क्रुः।
कृ॒ण्वा॒नासो॑ अमृत॒त्वाय॑ गा॒तुम्।
त्वं नृभि॑र्नृपते दे॒वहू॑तौ॥४३॥

%2.5.8.11
भूरी॑णि वृ॒त्वा ह॑र्यश्व हꣳसि।
त्वन्निद॑स्यु॒ञ्चुमु॑रिम्।
धुनिं॒ चा\-स्वा॑\-पयो द॒भीत॑ये सु॒हन्तु॑।
ए॒वा पा॑हि प्र॒त्नथा॒ मन्द॑तु त्वा।
श्रु॒धि ब्रह्म॑ वावृधस्वो॒त गी॒र्भिः।
आ॒विः सूर्यं॑ कृणु॒हि पी॒पिही॒षः।
ज॒हि शत्रूꣳ॑ र॒भि गा इ॑न्द्र तृन्धि।
अग्ने॒ बाध॑स्व॒ वि मृधो॑ नुदस्व।
अपामी॑वा॒ अप॒ रक्षाꣳ॑सि सेध।
अ॒स्माथ्स॑मु॒द्राद्बृ॑ह॒तो दि॒वो नः॑॥४४॥

%2.5.8.12
अ॒पां भू॒मान॒मुप॑ नः सृजे॒ह।
यज्ञ॒ प्रति॑ तिष्ठ सुम॒तौ सु॒शेवा॒ आ त्वा᳚।
वसू॑नि पुरु॒धा वि॑शन्तु।
दी॒र्घमायु॒र्यज॑मानाय कृ॒ण्वन्।
अथा॒मृते॑न जरि॒तार॑मङ्ग्धि।
इन्द्रः॑ शु॒नाव॒द्वित॑नोति॒ सीरम्᳚।
सं॒व॒थ्स॒रस्य॑ प्रति॒माण॑मे॒तत्।
अ॒र्कस्य॒ ज्योति॒स्तदिदा॑स॒ ज्येष्ठम्᳚।
सं॒व॒थ्स॒रꣳ शु॒नव॒थ्सीर॑मे॒तत्।
इन्द्र॑स्य॒ राधः॒ प्रय॑तं पु॒रु त्मना᳚।
तद॑र्करू॒पं वि॒मिमा॑नमेति।
द्वाद॑शारे॒ प्रति॑ तिष्ठ॒तीद्वृषा᳚।
अ॒श्वा॒यन्तो॑ ग॒व्यन्तो॑ वा॒जय॑न्तः।
हवा॑महे॒ त्वोप॑गन्त॒वा उ॑।
आ॒भूष॑न्तस्त्वा सुम॒तौ नवा॑याम्।
व॒यमि॑न्द्र त्वा शु॒नꣳ हु॑वेम॥४५॥\anuvakamend[अ॒र्च॒त॒ ह॒विर्गा॑यत यꣳसच्चर्\mbox{}षणी॒नां वै॑शम्भ॒ल्या हा॑सी॒त्त्वमु॒रुं दे॒वहू॑तौ न॒स्त्मना॒ षट्च॑]




\prashnaend{प्रा॒ण उ॒देहि॒ पुन॒रा नो॑ भर य॒ज्ञो रा॒यो वार्त्र॑हत्याय॒ वसू॑ना॒ꣳ॒ स ईं᳚ पाह्य॒ष्टौ॥८॥}{प्रा॒णो र॑क्ष॒त्यगृ॑भीता धाराव॒रा म॒रुतो॑ दीर्घायु॒त्वाय॒ ज्योति॑षा त्वा॒ पञ्च॑चत्वारिꣳशत्॥४५॥}{प्रा॒णः शु॒नꣳ हु॑वेम॥}{हरिः॑ ओम्॥}{इति श्रीकृष्णयजुर्वेदीयतैत्तिरीयब्राह्मणे द्वितीयाष्टके पञ्चमः प्रपाठकः समाप्तः॥}
\clearpage
\sect{षष्ठमः प्रश्नः}
\setcounter{anuvakam}{0}
\dnsub{तैत्तिरीयब्राह्मणे द्वितीयाष्टके षष्ठः प्रपाठकः}

%2.6.1.1
स्वा॒द्वीं त्वा᳚ स्वा॒दुना᳚।
ती॒व्रां ती॒व्रेण॑।
अ॒मृता॑म॒मृते॑न।
मधु॑मतीं॒ मधु॑मता।
सृ॒जामि॒ सꣳ सोमे॑न।
सोमो᳚\-ऽस्य॒श्विभ्यां᳚ पच्यस्व।
सर॑स्वत्यै पच्यस्व।
इन्द्रा॑य सु॒त्राम्णे॑ पच्यस्व।
परी॒तो षि॑ञ्चता सु॒तम्।
सोमो॒ य उ॑त्त॒मꣳ ह॒विः॥१॥

%2.6.1.2
द॒ध॒न्वा यो नर्यो॑ अ॒फ्स्व॑न्तरा।
सु॒षाव॒ सोम॒मद्रि॑भिः।
पु॒नातु॑ ते परि॒स्रुतम्᳚।
सोम॒ꣳ॒ सूर्य॑स्य दुहि॒ता।
वारे॑ण॒ शश्व॑ता॒ तना᳚।
वा॒युः पू॒तः प॒वित्रे॑ण।
प्राङ्ख्सोमो॒ अति॑द्रुतः।
इन्द्र॑स्य॒ युज्यः॒ सखा᳚।
वा॒युः पू॒तः प॒वित्रे॑ण।
प्र॒त्यङ्ख्सोमो॒ अति॑द्रुतः॥२॥

%2.6.1.3
इन्द्र॑स्य॒ युज्यः॒ सखा᳚।
ब्रह्म॑ क्ष॒त्रं प॑वते॒ तेज॑ इन्द्रि॒यम्।
सुर॑या॒ सोमः॑ सु॒त आसु॑तो॒ मदा॑य।
शु॒क्रेण॑ देव दे॒वताः᳚ पिपृग्धि।
रसे॒नान्नं॒ यज॑मानाय धेहि।
कु॒विद॒ङ्ग यव॑मन्तो॒ यव॑ञ्चित्।
यथा॒ दान्त्य॑नुपू॒र्वं वि॒यूय॑।
इ॒हेहै॑षां कृणुत॒ भोज॑नानि।
ये ब॒र्‌॒\mbox{}हिषो॒ नमो॑वृक्तिं॒ न ज॒ग्मुः।
उ॒प॒या॒मगृ॑हीतो\-ऽस्य॒श्विभ्यां᳚ त्वा॒ जुष्टं॑ गृह्णामि॥३॥

%2.6.1.4
सर॑स्वत्या॒ इन्द्रा॑य सु॒त्राम्णे᳚।
ए॒ष ते॒ योनि॒स्तेज॑से त्वा।
वी॒र्या॑य त्वा॒ बला॑य त्वा।
तेजो॑ऽसि॒ तेजो॒ मयि॑ धेहि।
वी॒र्य॑मसि वी॒र्यं॑ मयि॑ धेहि।
बल॑मसि॒ बलं॒ मयि॑ धेहि।
नाना॒ हि वां᳚ दे॒वहि॑त॒ꣳ॒ सदः॑ कृ॒तम्।
मा सꣳसृ॑क्षाथां पर॒मे व्यो॑मन्।
सुरा॒ त्वमसि॑ शु॒ष्मिणी॒ सोम॑ ए॒षः।
मा मा॑ हिꣳसीः॒ स्वां योनि॑मावि॒शन्॥४॥

%2.6.1.5
उ॒प॒या॒मगृ॑हीतो\-ऽस्याश्वि॒नं तेजः॑।
सा॒र॒स्व॒तं वी॒र्यम्᳚।
ऐ॒न्द्रं बलम्᳚।
ए॒ष ते॒ योनि॒र्मोदा॑य त्वा।
आ॒न॒न्दाय॑ त्वा॒ मह॑से त्वा।
ओजो॒ऽस्योजो॒ मयि॑ धेहि।
म॒न्युर॑सि म॒न्युं मयि॑ धेहि।
महो॑ऽसि॒ महो॒ मयि॑ धेहि।
सहो॑ऽसि॒ सहो॒ मयि॑ धेहि।
या व्या॒घ्रं विषू॑चिका।
उ॒भौ वृकं॑ च॒ रक्ष॑ति।
श्ये॒नं प॑त॒त्रिणꣳ॑ सि॒ꣳ॒हम्।
सेमं पा॒त्वꣳह॑सः।
स॒म्पृचः॑ स्थ॒ सं मा॑ भ॒द्रेण॑ पृङ्क्त।
वि॒पृचः॑ स्थ॒ वि मा॑ पा॒प्मना॑ पृङ्क्त॥५॥\anuvakamend[ह॒विः प्र॒त्यङ्ख्सोमो॒ अति॑द्रुतो गृह्णाम्यावि॒शन्विषू॑चिका॒ पञ्च॑ च]

%2.6.2.1
सोमो॒ राजा॒\-ऽमृतꣳ॑ सु॒तः।
ऋ॒जी॒षेणा॑जहान्मृ॒त्युम्।
ऋ॒तेन॑ स॒त्यमि॑न्द्रि॒यम्।
विपानꣳ॑ शु॒क्रमन्ध॑सः।
इन्द्र॑स्येन्द्रि॒यम्।
इ॒दं पयो॒\-ऽमृतं॒ मधु॑।
सोम॑म॒द्भ्यो व्य॑पिबत्।
छन्द॑सा ह॒ꣳ॒सः शु॑चि॒षत्।
ऋ॒तेन॑ स॒त्यमि॑न्द्रि॒यम्।
अ॒द्भ्यः क्षी॒रं व्य॑पिबत्॥६॥

%2.6.2.2
क्रुङ्ङा᳚ङ्गिर॒सो धि॒या।
ऋ॒तेन॑ स॒त्यमि॑न्द्रि॒यम्।
अन्ना᳚त्परि॒स्रुतो॒ रसम्᳚।
ब्रह्म॑णा॒ व्य॑पिबत् क्ष॒त्रम्।
ऋ॒तेन॑ स॒त्यमि॑न्द्रि॒यम्।
रेतो॒ मूत्रं॒ विज॑हाति।
योनिं॑ प्रवि॒शदि॑न्द्रि॒यम्।
गर्भो॑ ज॒रायु॒णा\-ऽऽवृ॑तः।
उल्बं॑ जहाति॒ जन्म॑ना।
ऋ॒तेन॑ स॒त्यमि॑न्द्रि॒यम्॥७॥

%2.6.2.3
वेदे॑न रू॒पे व्य॑करोत्।
स॒ता॒स॒ती प्र॒जा\-प॑तिः।
ऋ॒तेन॑ स॒त्यमि॑न्द्रि॒यम्।
सोमे॑न॒ सोमौ॒ व्य॑पिबत्।
सु॒ता॒सु॒तौ प्र॒जा\-प॑तिः।
ऋ॒तेन॑ स॒त्यमि॑न्द्रि॒यम्।
दृ॒ष्ट्वा रू॒पे व्याक॑रोत्।
स॒त्या॒नृ॒ते प्र॒जा\-प॑तिः।
अश्र॑द्धा॒मनृ॒ते\-ऽद॑धात्।
श्र॒द्धाꣳ स॒त्ये प्र॒जा\-प॑तिः।
ऋ॒तेन॑ स॒त्यमि॑न्द्रि॒यम्।
दृ॒ष्ट्वा प॑रि॒स्रुतो॒ रसम्᳚।
शु॒क्रेण॑ शु॒क्रं व्य॑पिबत्।
पयः॒ सोमं॑ प्र॒जा\-प॑तिः।
ऋ॒तेन॑ स॒त्यमि॑न्द्रि॒यम्।
विपानꣳ॑ शु॒क्रमन्ध॑सः।
इन्द्र॑स्येन्द्रि॒यम्।
इ॒दं पयो॒\-ऽमृतं॒ मधु॑॥८॥\anuvakamend[अ॒द्भ्यः क्षी॒रं व्य॑पिब॒ज्जन्म॑न॒र्तेन॑ स॒त्यमि॑न्द्रि॒यꣴ श्र॒द्धाꣳ स॒त्ये प्र॒जा\-प॑तिर॒ष्टौ च॑]

%2.6.3.1
सुरा॑वन्तं बर्‌\mbox{}हि॒षदꣳ॑ सु॒वीरम्᳚।
य॒ज्ञꣳ हि॑न्वन्ति महि॒षा नमो॑भिः।
दधा॑नाः॒ सोमं॑ दि॒वि दे॒वता॑सु।
मदे॒मेन्द्रं॒ यज॑मानाः स्व॒र्काः।
यस्ते॒ रसः॒ सम्भृ॑त॒ ओष॑धीषु।
सोम॑स्य॒ शुष्मः॒ सुर॑या सु॒तस्य॑।
तेन॑ जिन्व॒ यज॑मानं॒ मदे॑न।
सर॑स्वतीम॒श्विना॒विन्द्र॑म॒ग्निम्।
यम॒श्विना॒ नमु॑चेरासु॒रादधि॑।
सर॑स्व॒त्यस॑नोदिन्द्रि॒याय॑॥९॥

%2.6.3.2
इ॒मन्तꣳ शु॒क्रं मधु॑मन्त॒मिन्दुम्᳚।
सोम॒ꣳ॒ राजा॑नमि॒ह भ॑क्षयामि।
यदत्र॑ रि॒प्तꣳ र॒सिनः॑ सु॒तस्य॑।
यदिन्द्रो॒ अपि॑ब॒च्छची॑भिः।
अ॒हं तद॑स्य॒ मन॑सा शि॒वेन॑।
सोम॒ꣳ॒ राजा॑नमि॒ह भ॑क्षयामि।
पि॒तृभ्यः॑ स्वधा॒विभ्यः॑ स्व॒धा नमः॑।
पि॒ता॒म॒हेभ्यः॑ स्वधा॒विभ्यः॑ स्व॒धा नमः॑।
प्रपि॑तामहेभ्यः स्वधा॒विभ्यः॑ स्व॒धा नमः॑।
अक्ष॑न्पि॒तरः॑॥१०॥

%2.6.3.3
अमी॑मदन्त पि॒तरः॑।
अती॑तृपन्त पि॒तरः॑।
अमी॑मृजन्त पि॒तरः॑।
पित॑रः॒ शुन्ध॑ध्वम्।
पु॒नन्तु॑ मा पि॒तरः॑ सो॒म्यासः॑।
पु॒नन्तु॑ मा पिताम॒हाः।
पु॒नन्तु॒ प्रपि॑तामहाः।
प॒वित्रे॑ण श॒तायु॑षा।
पु॒नन्तु॑ मा पिताम॒हाः।
पु॒नन्तु॒ प्रपि॑तामहाः॥११॥

%2.6.3.4
प॒वित्रे॑ण श॒तायु॑षा।
विश्व॒मायु॒र्व्य॑श्ञवै।
अग्न॒ आयूꣳ॑षि पव॒से\-ऽग्ने॒ पव॑स्व।
पव॑मानः॒ सुव॒र्जनः॑ पु॒नन्तु॑ मा देवज॒नाः।
जात॑वेदः प॒वित्र॑व॒द्यत्ते॑ प॒वित्र॑म॒र्चिषि॑।
उ॒भाभ्यां᳚ देव सवितर्वैश्वदे॒वी पु॑न॒ती।
ये स॑मा॒नाः सम॑नसः।
पि॒तरो॑ यम॒राज्ये᳚।
तेषां᳚ लो॒कः स्व॒धा नमः॑।
य॒ज्ञो दे॒वेषु॑ कल्पताम्॥१२॥

%2.6.3.5
ये स॑जा॒ताः सम॑नसः।
जी॒वा जी॒वेषु॑ माम॒काः।
तेषा॒ꣴ॒ श्रीर्मयि॑ कल्पताम्।
अ॒स्मिँल्लो॒के श॒तꣳ समाः᳚।
द्वे स्रु॒ती अ॑शृणवं पितृ॒णाम्।
अ॒हं दे॒वाना॑मु॒त मर्त्या॑नाम्।
याभ्या॑मि॒दं विश्व॒मेज॒थ्समे॑ति।
यद॑न्त॒रा पि॒तरं॑ मा॒तरं॑ च।
इ॒दꣳ ह॒विः प्र॒जन॑नं मे अस्तु।
दश॑वीरꣳ स॒र्वग॑णꣴ स्व॒स्तये᳚।
आ॒त्म॒सनि॑ प्रजा॒सनि॑।
प॒शु॒सन्य॑भय॒सनि॑ लोक॒सनि॑।
अ॒ग्निः प्र॒जां ब॑हु॒लां मे॑ करोतु।
अन्नं॒ पयो॒ रेतो॑ अ॒स्मासु॑ धत्त।
रा॒यस्पोष॒मिष॒मूर्ज॑म॒स्मासु॑ दीधर॒थ्स्वाहा᳚॥१३॥\anuvakamend[इ॒न्द्रि॒याय॑ पि॒तरः॑ श॒तायु॑षा पु॒नन्तु॑ मा पिताम॒हाः पु॒नन्तु॒ प्रपि॑तामहाः कल्पताꣴ स्व॒स्तये॒ पञ्च॑ च]

%2.6.4.1
सीसे॑न॒ तन्त्रं॒ मन॑सा मनी॒षिणः॑।
ऊ॒र्णा॒सू॒त्रेण॑ क॒वयो॑ वयन्ति।
अ॒श्विना॑ य॒ज्ञꣳ स॑वि॒ता सर॑स्वती।
इन्द्र॑स्य रू॒पं वरु॑णो भिष॒ज्यन्।
तद॑स्य रू॒पम॒मृत॒ꣳ॒ शची॑भिः।
ति॒स्रो\-ऽद॑धुर्दे॒वताः᳚ सꣳररा॒णाः।
लोमा॑नि॒ शष्पै᳚र्बहु॒धा न तोक्म॑भिः।
त्वग॑स्य मा॒ꣳ॒सम॑भव॒न्न ला॒जाः।
तद॒श्विना॑ भि॒षजा॑ रु॒द्रव॑र्तनी।
सर॑स्वती वयति॒ पेशो॒ अन्त॑रः॥१४॥

%2.6.4.2
अस्थि॑ म॒ज्जानं॒ मास॑रैः।
का॒रो॒त॒रेण॒ दध॑तो॒ गवां᳚ त्व॒चि।
सर॑स्वती॒ मन॑सा पेश॒लं वसु॑।
नास॑त्याभ्यां वयति दर्‌\mbox{}श॒तं वपुः॑।
रसं॑ परि॒स्रुता॒ न रोहि॑तम्।
न॒ग्नहु॒र्धीर॒स्तस॑र॒न्न वेम॑।
पय॑सा शु॒क्रम॒मृतं॑ ज॒नित्रम्᳚।
सुर॑या॒ मूत्रा᳚ज्जनयन्ति॒ रेतः॑।
अपाम॑तिं दुर्म॒तिं बाध॑मानाः।
ऊव॑ध्यं॒ वातꣳ॑ स॒बुव॒न्तदा॒रात्॥१५॥

%2.6.4.3
इन्द्रः॑ सु॒त्रामा॒ हृद॑येन स॒त्यम्।
पु॒रो॒डाशे॑न सवि॒ता ज॑जान।
यकृ॑त्क्लो॒मानं॒ वरु॑णो भिष॒ज्यन्।
मत॑स्ने वाय॒व्यै᳚र्न मि॑नाति पि॒त्तम्।
आ॒न्त्राणि॑ स्था॒ली मधु॒ पिन्व॑माना।
गुदा॒ पात्रा॑णि सु॒दुघा॒ न धे॒नुः।
श्ये॒नस्य॒ पत्रं॒ न प्ली॒हा शची॑भिः।
आ॒स॒न्दी नाभि॑रु॒दरं॒ न मा॒ता।
कु॒म्भो व॑नि॒ष्ठुर्ज॑नि॒ता शची॑भिः।
यस्मि॒न्नग्रे॒ योन्यां॒ गर्भो॑ अ॒न्तः॥१६॥

%2.6.4.4
प्ला॒शीर्व्य॑क्तः श॒तधा॑र॒ उथ्सः॑।
दु॒हे न कु॒म्भीꣴ स्व॒धां पि॒तृभ्यः॑।
मुख॒ꣳ॒ सद॑स्य॒ शिर॒ इथ्सदे॑न।
जि॒ह्वा प॒वित्र॑म॒श्विना॒ सꣳ सर॑स्वती।
चप्प॒न्न पा॒युर्भि॒षग॑स्य॒ वालः॑।
व॒स्तिर्न शेपो॒ हर॑सा तर॒स्वी।
अ॒श्विभ्यां॒ चक्षु॑र॒मृतं॒ ग्रहा᳚भ्याम्।
छागे॑न॒ तेजो॑ ह॒विषा॑ शृ॒तेन॑।
पक्ष्मा॑णि गो॒धूमैः॒ क्व॑लैरु॒तानि॑।
पेशो॒ न शु॒क्लमसि॑तं वसाते॥१७॥

%2.6.4.5
अवि॒र्न मे॒षो न॒सि वी॒र्या॑य।
प्रा॒णस्य॒ पन्था॑ अ॒मृतो॒ ग्रहा᳚भ्याम्।
सर॑स्व॒त्युप॒वाकै᳚र्व्या॒नम्।
नस्या॑नि ब॒र्॒हिर्बद॑रैर्जजान।
इन्द्र॑स्य रू॒पमृ॑ष॒भो बला॑य।
कर्णा᳚भ्या॒ꣴ॒ श्रोत्र॑म॒मृतं॒ ग्रहा᳚भ्याम्।
यवा॒ न ब॒र्॒हिर्भ्रु॒वि केस॑राणि।
क॒र्कन्धु॑ जज्ञे॒ मधु॑ सार॒घं मुखा᳚त्।
आ॒त्मन्नु॒पस्थे॒ न वृक॑स्य॒ लोम॑।
मुखे॒ श्मश्रू॑णि॒ न व्या᳚घ्रलो॒मम्॥१८॥

%2.6.4.6
केशा॒ न शी॒र्॒षन्‌ यश॑से श्रि॒यै शिखा᳚।
सि॒ꣳ॒हस्य॒ लोम॒ त्विषि॑रिन्द्रि॒याणि॑।
अङ्गा᳚न्या॒त्मन्भि॒षजा॒ तद॒श्विना᳚।
आ॒त्मान॒मङ्गैः॒ सम॑धा॒थ्सर॑स्वती।
इन्द्र॑स्य रू॒पꣳ श॒तमा॑न॒मायुः॑।
च॒न्द्रेण॒ ज्योति॑र॒मृतं॒ दधा॑ना।
सर॑स्वती॒ योन्यां॒ गर्भ॑म॒न्तः।
अ॒श्विभ्यां॒ पत्नी॒ सुकृ॑तं बिभर्ति।
अ॒पाꣳ रसे॑न॒ वरु॑णो॒ न साम्ना᳚।
इन्द्रꣴ॑ श्रि॒यै ज॒नय॑न्न॒फ्सु राजा᳚।
तेजः॑ पशू॒नाꣳ ह॒विरि॑न्द्रि॒याव॑त्।
प॒रि॒स्रुता॒ पय॑सा सार॒घं मधु॑।
अ॒श्विभ्यां᳚ दु॒ग्धं भि॒षजा॒ सर॑स्वत्या सुतासु॒ताभ्या᳚म्।
अ॒मृतः॒ सोम॒ इन्दुः॑॥१९॥\anuvakamend[अन्त॑र आ॒राद॒न्तर्व॑साते व्याघ्रलो॒मꣳ राजा॑ च॒त्वारि॑ च]

%2.6.5.1
मि॒त्रो॑ऽसि॒ वरु॑णोऽसि।
सम॒हं विश्वै᳚र्दे॒वैः।
क्ष॒त्रस्य॒ नाभि॑रसि।
क्ष॒त्रस्य॒ योनि॑रसि।
स्यो॒नामा सी॑द।
सु॒षदा॒मा सी॑द।
मा त्वा॑ हिꣳसीत्।
मा मा॑ हिꣳसीत्।
निष॑साद धृ॒तव्र॑तो॒ वरु॑णः।
प॒स्त्या᳚स्वा॥२०॥

%2.6.5.2
साम्रा᳚ज्याय सु॒क्रतुः॑।
दे॒वस्य॑ त्वा सवि॒तुः प्र॑स॒वे।
अ॒श्विनो᳚र्बा॒हुभ्या᳚म्।
पू॒ष्णो हस्ता᳚भ्याम्।
अ॒श्विनो॒र्भैष॑ज्येन।
तेज॑से ब्रह्मवर्च॒साया॒भिषि॑ञ्चामि।
दे॒वस्य॑ त्वा सवि॒तुः प्र॑स॒वे।
अ॒श्विनो᳚र्बा॒हुभ्या᳚म्।
पू॒ष्णो हस्ता᳚भ्याम्।
सर॑स्वत्यै॒ भैष॑ज्येन॥२१॥

%2.6.5.3
वी॒र्या॑या॒न्नाद्या॑या॒भिषि॑ञ्चामि।
दे॒वस्य॑ त्वा सवि॒तुः प्र॑स॒वे।
अ॒श्विनो᳚र्बा॒हुभ्या᳚म्।
पू॒ष्णो हस्ता᳚भ्याम्।
इन्द्र॑स्येन्द्रि॒येण॑।
श्रि॒यै यश॑से॒ बला॑या॒भिषि॑ञ्चामि।
को॑ऽसि कत॒मो॑ऽसि।
कस्मै᳚ त्वा॒ काय॑ त्वा।
सुश्लो॒काँ(४) सुम॑ङ्ग॒लाँ(४) सत्य॑रा॒जा(३)न्।
शिरो॑ मे॒ श्रीः॥२२॥

%2.6.5.4
यशो॒ मुखम्᳚।
त्विषिः॒ केशा᳚श्च॒ श्मश्रू॑णि।
राजा॑ मे प्रा॒णो॑\-ऽमृतम्᳚।
स॒म्राट्चक्षुः॑।
वि॒राट्छ्रोत्रम्᳚।
जि॒ह्वा मे॑ भ॒द्रम्।
वाङ्महः॑।
मनो॑ म॒न्युः।
स्व॒राड्भामः॑।
मोदाः᳚ प्रमो॒दा अ॒ङ्गुली॒रङ्गा॑नि॥२३॥

%2.6.5.5
चि॒त्तं मे॒ सहः॑।
बा॒हू मे॒ बल॑मिन्द्रि॒यम्।
हस्तौ॑ मे॒ कर्म॑ वी॒र्यम्᳚।
आ॒त्मा क्ष॒त्रमुरो॒ मम॑।
पृ॒ष्टीर्मे॑ रा॒ष्ट्रमु॒दर॒मꣳसौ᳚।
ग्री॒वाश्च॒ श्रोण्यौ᳚।
ऊ॒रू अ॑र॒त्नी जानु॑नी।
विशो॒ मेऽङ्गा॑नि स॒र्वतः॑।
नाभि॑र्मे चि॒त्तं वि॒ज्ञानम्᳚।
पा॒युर्मे\-ऽप॑चितिर्भ॒सत्॥२४॥

%2.6.5.6
आ॒न॒न्द॒न॒न्दावा॒ण्डौ मे᳚।
भगः॒ सौभा᳚ग्यं॒ पसः॑।
जङ्घा᳚भ्यां प॒द्भ्यां धर्मो᳚\-ऽस्मि।
वि॒शि राजा॒ प्रति॑\-ष्ठितः।
प्रति॑ क्ष॒त्रे प्रति॑ तिष्ठामि रा॒ष्ट्रे।
प्रत्यश्वे॑षु॒ प्रति॑ तिष्ठामि॒ गोषु॑।
प्रत्यङ्गे॑षु॒ प्रति॑ तिष्ठाम्या॒त्मन्।
प्रति॑ प्रा॒णेषु॒ प्रति॑ तिष्ठामि पु॒ष्टे।
प्रति॒ द्यावा॑पृथि॒व्योः।
प्रति॑ तिष्ठामि य॒ज्ञे॥२५॥

%2.6.5.7
त्र॒या दे॒वा एका॑दश।
त्र॒य॒स्त्रि॒ꣳ॒शाः सु॒राध॑सः।
बृह॒स्पति॑पुरो\-हिताः।
दे॒वस्य॑ सवि॒तुः स॒वे।
दे॒वा दे॒वैर॑वन्तु मा।
प्र॒थ॒मा द्वि॒तीयैः᳚।
द्वि॒तीया᳚स्तृ॒तीयैः᳚।
तृ॒तीयाः᳚ स॒त्येन॑।
स॒त्यं य॒ज्ञेन॑।
य॒ज्ञो यजु॑र्भिः॥२६॥

%2.6.5.8
यजूꣳ॑षि॒ साम॑भिः।
सामा᳚न्यृ॒ग्भिः।
ऋचो॑ या॒ज्या॑भिः।
या॒ज्या॑ वषट्का॒रैः।
व॒ष॒ट्का॒रा आहु॑तिभिः।
आहु॑तयो मे॒ कामा॒न्थ्सम॑र्धयन्तु।
भूः स्वाहा᳚।
लोमा॑नि॒ प्रय॑ति॒र्मम॑।
त्वङ्म॒ आन॑ति॒राग॑तिः।
मा॒ꣳ॒सं म॒ उप॑नतिः।
वस्वस्थि॑।
म॒ज्जा म॒ आन॑तिः॥२७॥\anuvakamend[प॒स्त्या᳚स्वा सर॑स्वत्यै॒ भैष॑ज्येन॒ श्रीरङ्गा॑नि भ॒सद्य॒ज्ञे य॒ज्ञो यजु॑र्भि॒रुप॑नति॒र्द्वे च॑]

%2.6.6.1
यद्दे॑वा देव॒हेड॑नम्।
देवा॑सश्चकृ॒मा व॒यम्।
अ॒ग्निर्मा॒ तस्मा॒देन॑सः।
विश्वा᳚न्मुञ्च॒त्वꣳह॑सः।
यदि॒ दिवा॒ यदि॒ नक्तम्᳚।
एनाꣳ॑सि चकृ॒मा व॒यम्।
वा॒युर्मा॒ तस्मा॒देन॑सः।
विश्वा᳚न्मुञ्च॒त्वꣳह॑सः।
यदि॒ जाग्र॒द्यदि॒ स्वप्ने᳚।
एनाꣳ॑सि चकृ॒मा व॒यम्॥२८॥

%2.6.6.2
सूर्यो॑ मा॒ तस्मा॒देन॑सः।
विश्वा᳚न्मुञ्च॒त्वꣳह॑सः।
यद्ग्रामे॒ यदर॑ण्ये।
यथ्स॒भायां॒ यदि॑न्द्रि॒ये।
यच्छू॒द्रे यद॒र्ये᳚।
एन॑श्चकृ॒मा व॒यम्।
यदेक॒स्याधि॒ धर्म॑णि।
तस्या॑व॒यज॑नमसि।
यदापो॒ अघ्नि॑या॒ वरु॒णेति॒ शपा॑महे।
ततो॑ वरुण नो मुञ्च॥२९॥

%2.6.6.3
अव॑भृथ निचङ्कुण निचे॒रुर॑सि निचङ्कुण।
अव॑ दे॒वैर्दे॒वकृ॑त॒मेनो॑\-ऽयाट्।
अव॒ मर्त्यै॒र्मर्त्य॑कृतम्।
उ॒रोरा नो॑ देव रि॒षस्पा॑हि।
सु॒मि॒त्रा न॒ आप॒ ओष॑धयः सन्तु।
दु॒र्मि॒त्रास्तस्मै॑ भूयासुः।
यो᳚ऽस्मान्द्वेष्टि॑।
यं च॑ व॒यं द्वि॒ष्मः।
द्रु॒प॒दादि॒वेन्मु॑मुचा॒नः।
स्वि॒न्नः स्ना॒त्वी मला॑दिव॥३०॥

%2.6.6.4
पू॒तं प॒वित्रे॑णे॒वाऽऽज्यम्᳚।
आपः॑ शुन्धन्तु॒ मैन॑सः।
उद्व॒यं तम॑स॒स्परि॑।
पश्य॑न्तो॒ ज्योति॒रुत्त॑रम्।
दे॒वं दे॑व॒त्रा सूर्यम्᳚।
अग॑न्म॒ ज्योति॑रुत्त॒मम्।
प्रति॑\-युतो॒ वरु॑णस्य॒ पाशः॑।
प्रत्य॑स्तो॒ वरु॑णस्य॒ पाशः॑।
एधो᳚ऽस्येधिषी॒महि॑।
स॒मिद॑सि॥३१॥

%2.6.6.5
तेजो॑ऽसि॒ तेजो॒ मयि॑ धेहि।
अ॒पो अन्व॑चारिषम्।
रसे॑न॒ सम॑सृक्ष्महि।
पय॑स्वाꣳ अग्न॒ आग॑मम्।
तं मा॒ सꣳसृ॑ज॒ वर्च॑सा।
प्र॒जया॑ च॒ धने॑न च।
स॒माव॑वर्ति पृथि॒वी।
समु॒षाः।
समु॒ सूर्यः॑।
समु॒ विश्व॑मि॒दं जग॑त्।
वै॒श्वा॒न॒रज्यो॑तिर्भूयासम्।
वि॒भुं कामं॒ व्य॑श्ञवै।
भूः स्वाहा᳚॥३२॥\anuvakamend[स्वप्न॒ एनाꣳ॑सि चकृ॒मा व॒यं मु॑ञ्च॒ मला॑दिव स॒मिद॑सि॒ जग॒त्रीणि॑ च]

%2.6.7.1
होता॑ यक्षथ्स॒मिधेन्द्र॑मि॒डस्प॒दे।
नाभा॑ पृथि॒व्या अधि॑।
दि॒वो वर्ष्म॒न्थ्समि॑ध्यते।
ओजि॑ष्ठश्चर्‌\mbox{}षणी॒ सहान्॑।
वेत्वा\-ऽ\-ऽ\-ज्य॑स्य॒ होत॒र्यज॑।
होता॑ यक्ष॒त्तनू॒नपा॑तम्।
ऊ॒तिभि॒र्जेता॑र॒\-मप॑\-राजितम्।
इन्द्रं॑ दे॒वꣳ सु॑व॒र्विदम्᳚।
प॒थिभि॒र्मधु॑मत्तमैः।
नरा॒शꣳसे॑न॒ तेज॑सा॥३३॥

%2.6.7.2
वेत्वा\-ऽ\-ऽ\-ज्य॑स्य॒ होत॒र्यज॑।
होता॑ यक्ष॒दिडा॑भि॒रिन्द्र॑मीडि॒तम्।
आ॒जुह्वा॑न॒मम॑र्त्यम्।
दे॒वो दे॒वैः सवी᳚र्यः।
वज्र॑हस्तः पुरन्द॒रः।
वेत्वा\-ऽ\-ऽ\-ज्य॑स्य॒ होत॒र्यज॑।
होता॑यक्षद्ब॒र्॒हिषीन्द्र॑न्निषद्व॒रम्।
वृ॒ष॒भं नर्या॑पसम्।
वसु॑भीरु॒द्रैरा॑दि॒त्यैः।
स॒युग्भि॑र्ब॒र्॒हिरा\-स॑दत्॥३४॥

%2.6.7.3
वेत्वा\-ऽ\-ऽ\-ज्य॑स्य॒ होत॒र्यज॑।
होता॑ यक्ष॒दोजो॒ न वी॒र्यम्᳚।
सहो॒ द्वार॒ इन्द्र॑मवर्धयन्।
सु॒प्रा॒य॒णा विश्र॑यन्तामृता॒वृधः॑।
द्वार॒ इन्द्रा॑य मी॒ढुषे᳚।
वि॒यन्त्वाज्य॑स्य॒ होत॒र्यज॑।
होता॑ यक्षदु॒षे इन्द्र॑स्य धे॒नू।
सु॒दुघे॑ मा॒तरौ॑ म॒ही।
सवा॒तरौ॒ न तेज॑सी।
व॒थ्समिन्द्र॑मवर्धताम्॥३५॥

%2.6.7.4
वी॒तामाज्य॑स्य॒ होत॒र्यज॑।
होता॑ यक्ष॒द्दैव्या॒ होता॑रा।
भि॒षजा॒ सखा॑या।
ह॒विषेन्द्रं॑ भिषज्यतः।
क॒वी दे॒वौ प्रचे॑तसौ।
इन्द्रा॑य धत्त इन्द्रि॒यम्।
वी॒तामाज्य॑स्य॒ होत॒र्यज॑।
होता॑ यक्षत्ति॒स्रो दे॒वीः।
त्रय॑स्त्रि॒धात॑वो॒पसः॑।
इडा॒ सर॑स्वती॒ भार॑ती॥३६॥

%2.6.7.5
म॒हीन्द्र॑पत्नीर्‌\mbox{}ह॒विष्म॑तीः।
वि॒यन्त्वाज्य॑स्य॒ होत॒र्यज॑।
होता॑ यक्ष॒त्त्वष्टा॑र॒मिन्द्रं॑ दे॒वम्।
भि॒षजꣳ॑ सु॒यजं॑ घृत॒श्रियम्᳚।
पु॒रु॒रूपꣳ॑ सु॒रेत॑सं म॒घोनिम्᳚।
इन्द्रा॑य॒ त्वष्टा॒ दध॑दिन्द्रि॒याणि॑।
वेत्वा\-ऽ\-ऽ\-ज्य॑स्य॒ होत॒र्यज॑।
होता॑ यक्ष॒द्वन॒स्पतिम्᳚।
श॒मि॒तारꣳ॑ श॒तक्र॑तुम्।
धि॒यो जो॒ष्टार॑मिन्द्रि॒यम्॥३७॥

%2.6.7.6
मध्वा॑ सम॒ञ्जन्प॒थिभिः॑ सु॒गेभिः॑।
स्वदा॑ति ह॒व्यं मधु॑ना घृ॒तेन॑।
वेत्वा\-ऽ\-ऽ\-ज्य॑स्य॒ होत॒र्यज॑।
होता॑ यक्ष॒दिन्द्र॒ꣴ॒ स्वाहा\-ऽऽज्य॑स्य।
स्वाहा॒ मेद॑सः।
स्वाहा᳚ स्तो॒काना᳚म्।
स्वाहा॒ स्वाहा॑कृतीनाम्।
स्वाहा॑ ह॒व्यसू᳚क्तीनाम्।
स्वाहा॑ दे॒वाꣳ आ᳚ज्य॒पान्।
स्वाहेन्द्रꣳ॑ हो॒त्राज्जु॑षा॒णाः।
इन्द्र॒ आज्य॑स्य वियन्तु।
होत॒र्यज॑॥३८॥\anuvakamend[तेज॑सा\-ऽऽसददवर्धतां॒ भार॑तीन्द्रि॒यं जु॑षा॒णा द्वे च॑ (स॒मिधेन्द्र॒न्तनू॒नपा॑त॒मिडा॑भिर्ब॒र्॒हिष्योज॑ उ॒षे दैव्या॑ ति॒स्रस्त्वष्टा॑रं॒ वन॒स्पति॒मिन्द्रम्᳚॥ स॒मिधेन्द्रं॑ च॒तुर्वेत्वेको॑ वि॒यन्तु॒ द्विर्वी॒तामेको॑ वि॒यन्तु॒ द्विर्वेत्वेको॑ वि॒यन्तु॒ होत॒र्यज॑॥)]

%2.6.8.1
समि॑द्ध॒ इन्द्र॑ उ॒षसा॒मनी॑के।
पु॒रो॒रुचा॑ पूर्व॒कृद्वा॑वृधा॒नः।
त्रि॒भिर्दे॒वैस्त्रि॒ꣳ॒शता॒ वज्र॑बाहुः।
ज॒घान॑ वृ॒त्रं वि दुरो॑ ववार।
नरा॒शꣳसः॒ प्रति॒शूरो॒ मिमा॑नः।
तनू॒नपा॒त्प्रति॑ य॒ज्ञस्य॒ धाम॑।
गोभि॑र्व॒पावा॒न्मधु॑ना सम॒ञ्जन्।
हिर॑ण्यैश्च॒न्द्री य॑जति॒ प्रचे॑ताः।
ई॒डि॒तो दे॒वैर्‌\mbox{}हरि॑वाꣳ अभि॒ष्टिः।
आ॒जुह्वा॑नो ह॒विषा॒ शर्ध॑मानः॥३९॥

%2.6.8.2
पु॒र॒न्द॒रो म॒घवा॒न्॒ वज्र॑बाहुः।
आया॑तु य॒ज्ञमुप॑नो जुषा॒णः।
जु॒षा॒णो ब॒र्‌॒\mbox{}हिर्\mbox{}हरि॑वान्न॒ इन्द्रः॑।
प्रा॒चीनꣳ॑ सीदत्प्र॒दिशा॑ पृथि॒व्याः।
उ॒रु॒व्यचाः॒ प्रथ॑मानꣴ स्यो॒नम्।
आ॒दि॒त्यैर॒क्तं वसु॑भिः स॒जोषाः᳚।
इन्द्रं॒ दुरः॑ कव॒ष्यो॑ धाव॑मानाः।
वृषा॑णं यन्तु॒ जन॑यः सु॒पत्नीः᳚।
द्वारो॑ दे॒वीर॒भितो॒ विश्र॑यन्ताम्।
सु॒वीरा॑ वी॒रं प्रथ॑माना॒ महो॑भिः॥४०॥

%2.6.8.3
उ॒षासा॒नक्ता॑ बृह॒ती बृ॒हन्तम्᳚।
पय॑स्वती सु॒दुघे॒ शूर॒मिन्द्रम्᳚।
पेश॑स्वती॒ तन्तु॑ना सं॒व्यय॑न्ती।
दे॒वानां᳚ दे॒वं य॑जतः सुरु॒क्मे।
दैव्या॒ मिमा॑ना॒ मन॑सा पुरु॒त्रा।
होता॑रा॒विन्द्रं॑ प्रथ॒मा सु॒वाचा᳚।
मू॒र्धन् य॒ज्ञस्य॒ मधु॑ना॒ दधा॑ना।
प्रा॒चीनं॒ ज्योति॑र्ह॒विषा॑ वृधातः।
ति॒स्रो दे॒वीर्‌\mbox{}ह॒विषा॒ वर्ध॑मानाः।
इन्द्रं॑ जुषा॒णा वृष॑णं॒ न पत्नीः᳚॥४१॥

%2.6.8.4
अच्छि॑न्नं॒ तन्तुं॒ पय॑सा॒ सर॑स्वती।
इडा॑ दे॒वी भार॑ती वि॒श्वतू᳚र्तिः।
त्वष्टा॒ दध॒दिन्द्रा॑य॒ शुष्मम्᳚।
अपा॒कोचि॑ष्टुर्य॒शसे॑ पु॒रूणि॑।
वृषा॒ यज॒न्वृष॑णं॒ भूरि॑रेताः।
मू॒र्धन् य॒ज्ञस्य॒ सम॑नक्तु दे॒वान्।
वन॒स्पति॒रव॑सृष्टो॒ न पाशैः᳚।
त्मन्या॑ सम॒ञ्जञ्छ॑मि॒ता न दे॒वः।
इन्द्र॑स्य ह॒व्यैर्ज॒ठरं॑ पृणा॒नः।
स्वदा॑ति ह॒व्यं मधु॑ना घृ॒तेन॑।
स्तो॒काना॒मिन्दुं॒ प्रति॒ शूर॒ इन्द्रः॑।
वृ॒षा॒यमा॑णो वृष॒भस्तु॑रा॒षाट्।
घृ॒त॒प्रुषा॒ मधु॑ना ह॒व्यमु॒न्दन्।
मू॒र्धन् य॒ज्ञस्य॑ जुषता॒ꣴ॒ स्वाहा᳚॥४२॥\anuvakamend[शर्ध॑मानो॒ महो॑भिः॒ पत्नी᳚र्घृ॒तेन॑ च॒त्वारि॑ च]

%2.6.9.1
आच॑र्‌\mbox{}षणि॒प्रा वि॒वेष॒ यन्मा᳚।
तꣳ स॒ध्रीचीः᳚।
स॒त्यमित्तन्न त्वावाꣳ॑ अ॒न्यो अस्ति॑।
इन्द्र॑ दे॒वो न मर्त्यो॒ ज्यायान्॑।
अह॒न्नहिं॑ परि॒शया॑न॒मर्णः॑।
अवा॑सृजो॒ऽपो अच्छा॑ समु॒द्रम्।
प्रस॑साहिषे पुरुहूत॒ शत्रून्॑।
ज्येष्ठ॑स्ते॒ शुष्म॑ इ॒ह रा॒तिर॑स्तु।
इन्द्रा भ॑र॒ दक्षि॑णेना॒ वसू॑नि।
पतिः॒ सिन्धू॑नामसि रे॒वती॑नाम्।
स शेवृ॑ध॒मधि॑ धाद्द्यु॒म्नम॒स्मे।
महि॑ क्ष॒त्रं ज॑ना॒षाडि॑न्द्र॒ तव्यम्᳚।
रक्षा॑ च नो म॒घोनः॑ पा॒हि सू॒रीन्।
रा॒ये च॑ नः स्वप॒त्या इ॒षे धाः᳚॥४३॥\anuvakamend[रे॒वती॑नां च॒त्वारि॑ च]

%2.6.10.1
दे॒वं ब॒र्॒हिरिन्द्रꣳ॑ सुदे॒वं दे॒वैः।
वी॒रव॑थ्स्ती॒र्णं वेद्या॑मवर्धयत्।
वस्तो᳚र्वृ॒तं प्राक्तो᳚र्भृ॒तम्।
रा॒या ब॒र्॒हिष्म॒तो\-ऽत्य॑गात्।
व॒सु॒वने॑ वसु॒धेय॑स्य वेतु॒ यज॑।
दे॒वीर्द्वार॒ इन्द्रꣳ॑ सङ्घा॒ते।
वि॒ड्वीर्याम॑न्नवर्धयन्।
आ व॒थ्सेन॒ तरु॑णेन कुमा॒रेण॑ चमीवि॒ता अपार्वा॑णम्।
रे॒णुक॑काटं नुदन्ताम्।
व॒सु॒वने॑ वसु॒धेय॑स्य वियन्तु॒ यज॑॥४४॥

%2.6.10.2
दे॒वी उ॒षासा॒नक्ता᳚।
इन्द्रं॑ य॒ज्ञे प्र॑य॒त्य॑ह्वेताम्।
दैवी॒र्विशः॒ प्राया॑सिष्टाम्।
सुप्री॑ते॒ सुधि॑ते अभूताम्।
व॒सु॒वने॑ वसु॒धेय॑स्य वीतां॒ यज॑।
दे॒वी जोष्ट्री॒ वसु॑धिती।
दे॒वमिन्द्र॑मवर्धताम्।
अया᳚व्य॒न्याघा द्वेषाꣳ॑सि।
आन्यावा᳚क्षी॒द्वसु॒ वार्या॑णि।
यज॑मानाय शिक्षि॒ते॥४५॥

%2.6.10.3
व॒सु॒वने॑ वसु॒धेय॑स्य वीतां॒ यज॑।
दे॒वी ऊ॒र्जाहु॑ती॒ दुघे॑ सु॒दुघे᳚।
पय॒सेन्द्र॑मवर्धताम्।
इष॒मूर्ज॑म॒न्या\-ऽवा᳚क्षीत्।
सग्धि॒ꣳ॒ सपी॑तिम॒न्या।
नवे॑न॒ पूर्वं॒ दय॑माने।
पु॒रा॒णेन॒ नवम्᳚।
अधा॑ता॒मूर्ज॑मू॒र्जाहु॑ती॒ वसु॒ वार्या॑णि।
यज॑मानाय शिक्षि॒ते।
व॒सु॒वने॑ वसु॒धेय॑स्य वीतां॒ यज॑॥४६॥

%2.6.10.4
दे॒वा दैव्या॒ होता॑रा।
दे॒वमिन्द्र॑मवर्धताम्।
ह॒ताघ॑शꣳसा॒वा\-भा᳚र्ष्टां॒ वसु॒वार्या॑णि।
यज॑मानाय शिक्षि॒तौ।
व॒सु॒वने॑ वसु॒धेय॑स्य वीतां॒ यज॑।
दे॒वीस्ति॒स्रस्ति॒स्रो दे॒वीः।
पति॒मिन्द्र॑मवर्धयन्।
अस्पृ॑क्ष॒द्भार॑ती॒ दिवम्᳚।
रु॒द्रैर्य॒ज्ञꣳ सर॑स्वती।
इडा॒ वसु॑मती गृ॒हान्॥४७॥

%2.6.10.5
व॒सु॒वने॑ वसु॒धेय॑स्य वियन्तु॒ यज॑।
दे॒व इन्द्रो॒ नरा॒शꣳसः॑।
त्रि॒व॒रू॒थस्त्रि॑वन्धु॒रः।
दे॒वमिन्द्र॑मवर्धयत्।
श॒तेन॑ शिति\-पृ॒ष्ठाना॒माहि॑तः।
स॒हस्रे॑ण॒ प्रव॑र्तते।
मि॒त्रावरु॒णेद॑स्य हो॒त्रमर्‌\mbox{}ह॑तः।
बृह॒स्पतिः॑ स्तो॒त्रम्।
अ॒श्विना\-ऽऽध्व॑र्यवम्।
व॒सु॒वने॑ वसु॒धेय॑स्य वेतु॒ यज॑॥४८॥

%2.6.10.6
दे॒व इन्द्रो॒ वन॒स्पतिः॑।
हिर॑ण्यपर्णो॒ मधु॑शाखः सुपिप्प॒लः।
दे॒वमिन्द्र॑मवर्धयत्।
दिव॒मग्रे॑णाप्रात्।
आऽन्तरि॑क्षं पृथि॒वीम॑दृꣳहीत्।
व॒सु॒वने॑ वसु॒धेय॑स्य वेतु॒ यज॑।
दे॒वं ब॒र्॒हिर्वारि॑तीनाम्।
दे॒वमिन्द्र॑मवर्धयत्।
स्वा॒स॒स्थमिन्द्रे॒णा\-स॑न्नम्।
अ॒न्या ब॒र्॒हीꣴष्य॒भ्य॑\-भूत्।
व॒सु॒वने॑ वसु॒धेयस्य॑ वेतु॒ यज॑।
दे॒वो अ॒ग्निः स्वि॑ष्ट॒\-कृत्।
दे॒वमिन्द्र॑मवर्धयत्।
स्वि॑ष्टं कु॒र्वन्थ्स्वि॑ष्ट॒\-कृत्।
स्वि॑ष्टम॒द्य क॑रोतु नः।
व॒सु॒वने॑ वसु॒धेय॑स्य वेतु॒ यज॑॥४९॥\anuvakamend[वि॒य॒न्तु॒ यज॑ शिक्षि॒ते शि॑क्षि॒ते व॑सु॒वने॑ वसु॒धेय॑स्य वीतां॒ यज॑ गृ॒हान् वे॑तु॒ यजा॑भू॒थ्षट्च॑ (दे॒वं ब॒र्॒हिर्दे॒वीर्द्वारो॑ दे॒वी उ॒षासा॒नक्ता॑ दे॒वी जोष्ट्री॑ दे॒वी ऊ॒र्जाहु॑ती दे॒वा दैव्या॒ होता॑रा शिक्षि॒तौ दे॒वीस्ति॒स्रस्ति॒स्रो दे॒वीर्दे॒व इन्द्रो॒ नरा॒शꣳसो॑ दे॒व इन्द्रो॒ वन॒स्पति॑र्दे॒वं ब॒र्॒हिर्वारि॑तीनान्दे॒वो अ॒ग्निः स्वि॑ष्ट॒कृद्दे॒वम्।
वे॒तु॒ वि॒य॒न्तु॒ च॒तुर्वी॑ता॒मेको॑ वियन्तु च॒तुर्वे᳚त्ववर्धयदवर्धय॒न्त्रिर॑वर्धता॒मेको॑ऽ वर्धयꣴश्च॒तुर॑वर्धयत्।
वस्तो॒रा व॒थ्सेन॒ दैवी॒रया॒वीषꣳ॑ह॒ता\-ऽस्पृ॑क्षच्छ॒तेन॒ दिवꣴ॑ स्वास॒स्थꣴ स्वि॑ष्टꣳ शिक्षि॒ते शि॑क्षि॒ते शि॑क्षि॒तौ॥)]

%2.6.11.1
होता॑ यक्षथ्स॒मिधा॒\-ऽग्निमि॒डस्प॒दे।
अ॒श्विनेन्द्र॒ꣳ॒ सर॑स्वतीम्।
अ॒जो धू॒म्रो न गो॒धूमैः॒ क्व॑लैर्भेष॒जम्।
मधु॒ शष्पै॒र्न तेज॑ इन्द्रि॒यम्।
पयः॒ सोमः॑ परि॒स्रुता॑ घृ॒तं मधु॑।
वि॒यन्त्वाज्य॑स्य॒ होत॒र्यज॑।
होता॑ यक्ष॒त्तनू॒नपा॒थ्सर॑स्वती।
अवि॑र्मे॒षो न भे॑ष॒जम्।
प॒था मधु॑म॒ताभ॑रन्।
अ॒श्विनेन्द्रा॑य वी॒र्यम्᳚॥५०॥

%2.6.11.2
बद॑रैरुप॒वाका॑भिर्भेष॒जं तोक्म॑भिः।
पयः॒ सोमः॑ परि॒स्रुता॑ घृ॒तं मधु॑।
वि॒यन्त्वाज्य॑स्य॒ होत॒र्यज॑।
होता॑ यक्षं॒ नरा॒शꣳसं॒ न न॒ग्नहुम्᳚।
पति॒ꣳ॒ सुरा॑यै भेष॒जम्।
मे॒षः सर॑स्वती भि॒षक्।
रथो॒ न च॒न्द्र्य॑श्विनो᳚र्व॒पा इन्द्र॑स्य वी॒र्यम्᳚।
बद॑रैरुप॒वाका॑भिर्भेष॒जं तोक्म॑भिः।
पयः॒ सोमः॑ परि॒स्रुता॑ घृ॒तं मधु॑।
वि॒यन्त्वाज्य॑स्य॒ होत॒र्यज॑॥५१॥

%2.6.11.3
होता॑ यक्षदि॒डेडि॒त आ॒जुह्वा॑नः॒ सर॑स्वतीम्।
इन्द्रं॒ बले॑न व॒र्धयन्॑।
ऋ॒ष॒भेण॒ गवे᳚न्द्रि॒यम्।
अ॒श्विनेन्द्रा॑य वी॒र्यम्᳚।
यवैः᳚ क॒र्कन्धु॑भिः।
मधु॑ ला॒जैर्न मास॑रम्।
पयः॒ सोमः॑ परि॒स्रुता॑ घृ॒तं मधु॑।
वि॒यन्त्वाज्य॑स्य॒ होत॒र्यज॑।
होता॑ यक्षद्ब॒र्॒हिः सु॒ष्टरी॒मोर्ण॑म्रदाः।
भि॒षङ्नास॑त्या॥५२॥

%2.6.11.4
भि॒षजा॒\-ऽश्विना\-ऽश्वा॒ शिशु॑मती।
भि॒षग्धे॒नुः सर॑स्वती।
भि॒षग्दु॒ह इन्द्रा॑य भेष॒जम्।
पयः॒ सोमः॑ परि॒स्रुता॑ घृ॒तं मधु॑।
वि॒यन्त्वाज्य॑स्य॒ होत॒र्यज॑।
होता॑ यक्ष॒द्दुरो॒ दिशः॑।
क॒व॒ष्यो॑ न व्यच॑स्वतीः।
अ॒श्विभ्यां॒ न दुरो॒ दिशः॑।
इन्द्रो॒ न रोद॑सी॒ दुघे᳚।
दु॒हे कामा॒न्थ्सर॑स्वती॥५३॥

%2.6.11.5
अ॒श्विनेन्द्रा॑य भेष॒जम्।
शु॒क्रं न ज्योति॑रिन्द्रि॒यम्।
पयः॒ सोमः॑ परि॒स्रुता॑ घृ॒तं मधु॑।
वि॒यन्त्वाज्य॑स्य॒ होत॒र्यज॑।
होता॑ यक्षथ्सु॒पेश॑सो॒षे नक्तं॒ दिवा᳚।
अ॒श्विना॑ सञ्जाना॒ने।
समं॑ जाते॒ सर॑स्वत्या।
त्विषि॒मिन्द्रे॒ न भे॑ष॒जम्।
श्ये॒नो न रज॑सा हृ॒दा।
पयः॒ सोमः॑ परि॒स्रुता॑ घृ॒तं मधु॑॥५४॥

%2.6.11.6
वि॒यन्त्वाज्य॑स्य॒ होत॒र्यज॑।
होता॑ यक्ष॒द्दैव्या॒ होता॑रा भि॒षजा॒\-ऽश्विना᳚।
इन्द्रं॒ न जागृ॑वी॒ दिवा॒ नक्तं॒ न भे॑ष॒जैः।
शूष॒ꣳ॒ सर॑स्वती भि॒षक्।
सीसे॑न दुह इन्द्रि॒यम्।
पयः॒ सोमः॑ परि॒स्रुता॑ घृ॒तं मधु॑।
वि॒यन्त्वाज्य॑स्य॒ होत॒र्यज॑।
होता॑ यक्षत्ति॒स्रो दे॒वीर्न भे॑ष॒जम्।
त्रय॑स्त्रि॒धात॑वो॒\-ऽपसः॑।
रू॒पमिन्द्रे॑ हिर॒ण्ययम्᳚॥५५॥

%2.6.11.7
अ॒श्विनेडा॒ न भार॑ती।
वा॒चा सर॑स्वती।
मह॒ इन्द्रा॑य दधुरिन्द्रि॒यम्।
पयः॒ सोमः॑ परि॒स्रुता॑ घृ॒तं मधु॑।
वि॒यन्त्वाज्य॑स्य॒ होत॒र्यज॑।
होता॑ यक्ष॒त्त्वष्टा॑र॒मिन्द्र॑म॒श्विना᳚।
भि॒षजं॒ न सर॑स्वतीम्।
ओजो॒ न जू॒तिरि॑न्द्रि॒यम्।
वृको॒ न र॑भ॒सो भि॒षक्।
यशः॒ सुर॑या भेष॒जम्॥५६॥

%2.6.11.8
श्रि॒या न मास॑रम्।
पयः॒ सोमः॑ परि॒स्रुता॑ घृ॒तं मधु॑।
वि॒यन्त्वाज्य॑स्य॒ होत॒र्यज॑।
होता॑ यक्ष॒द्वन॒स्पतिम्᳚।
श॒मि॒तारꣳ॑ श॒तक्र॑तुम्।
भी॒मं न म॒न्युꣳ राजा॑नं व्या॒घ्रं नम॑सा॒\-ऽश्विना॒ भामम्᳚।
सर॑स्वती भि॒षक्।
इन्द्रा॑य दुह इन्द्रि॒यम्।
पयः॒ सोमः॑ परि॒स्रुता॑ घृ॒तं मधु॑।
वि॒यन्त्वाज्य॑स्य॒ होत॒र्यज॑॥५७॥

%2.6.11.9
होता॑ यक्षद॒ग्निꣴ स्वाहा\-ऽऽज्य॑स्य स्तो॒काना᳚म्।
स्वाहा॒ मेद॑सां॒ पृथ॑क्।
स्वाहा॒ छाग॑म॒श्विभ्या᳚म्।
स्वाहा॑ मे॒षꣳ सर॑स्वत्यै।
स्वाह॑र्‌\mbox{}ष॒भमिन्द्रा॑य सि॒ꣳ॒हाय॒ सह॑सेन्द्रि॒यम्।
स्वाहा॒\-ऽग्निं न भे॑ष॒जम्।
स्वाहा॒ सोम॑मिन्द्रि॒यम्।
स्वाहेन्द्रꣳ॑ सु॒त्रामा॑णꣳ सवि॒तारं॒ वरु॑णं भि॒षजां॒ पतिम्᳚।
स्वाहा॒ वन॒स्पतिं॑ प्रि॒यं पाथो॒ न भे॑ष॒जम्।
स्वाहा॑ दे॒वाꣳ आ᳚ज्य॒पान्॥५८॥

%2.6.11.10
स्वाहा॒\-ऽग्निꣳ हो॒त्राज्जु॑षा॒णो अ॒ग्निर्भे॑ष॒जम्।
पयः॒ सोमः॑ परि॒स्रुता॑ घृ॒तं मधु॑।
वि॒यन्त्वाज्य॑स्य॒ होत॒र्यज॑।
होता॑ यक्षद॒श्विना॒ सर॑स्वती॒मिन्द्रꣳ॑ सु॒त्रामा॑णम्।
इ॒मे सोमाः᳚ सु॒रामा॑णः।
छागै॒र्न मे॒षैर्\mbox{}ऋ॑ष॒भैः सु॒ताः।
शष्पै॒र्न तोक्म॑भिः।
ला॒जैर्मह॑स्वन्तः।
मदा॒ मास॑रेण॒ परि॑ष्कृताः।
शु॒क्राः पय॑स्वन्तो॒\-ऽमृताः᳚।
प्रस्थि॑ता वो मधु॒श्चुतः॑।
तान॒श्विना॒ सर॑स्व॒तीन्द्रः॑ सु॒त्रामा॑ वृत्र॒हा।
जु॒षन्ताꣳ॑ सौ॒म्यं मधु॑।
पिब॑न्तु॒ मद॑न्तु वि॒यन्तु॒ सोमम्᳚।
होत॒र्यज॑॥५९॥\anuvakamend[वी॒र्यं॑ वि॒यन्त्वाज्य॑स्य॒ होत॒र्यज॒ नास॑त्या॒ सर॑स्वती॒ मधु॑ हिर॒ण्ययं॑ भेष॒जं वि॒यन्त्वाज्य॑स्य॒ होत॒र्यजा᳚ज्य॒पान॒मृताः॒ पञ्च॑ च (स॒मिधा॒\-ऽग्निꣳ षट्।
तनू॒नपा᳚थ्स॒प्त।
नरा॒शꣳस॒मृषिः॑।
इ॒डेडि॒तो यवै॑र॒ष्टौ।
ब॒र्‌॒\mbox{}हिः॒ स॒प्त।
दुरो॒\-ऽश्विना॒ नव॑।
सु॒पेश॒सर्‌\mbox{}षिः॑।
दैव्या॒ होता॑रा॒ सीसे॑न॒ रसः॑।
ति॒स्रस्त्वष्टा॑रम॒ष्टाव॑ष्टौ।
वन॒स्पति॒मृषिः॑।
अ॒ग्निन्त्रयो॑दश।
अ॒श्विना॒ द्वाद॑श त्रयोदश।
स॒मिधा॒\-ऽग्निं बद॑रै॒र्बद॑रै॒र्यवै॑र॒श्विना॒ त्विषि॑म॒श्विना॒ न भे॑ष॒जꣳ रू॒पम॒श्विना॑ भी॒मं भामम्᳚॥)]

%2.6.12.1
समि॑द्धो अ॒ग्निर॑श्विना।
त॒प्तो घ॒र्मो वि॒राट्थ्सु॒तः।
दु॒हे धे॒नुः सर॑स्वती।
सोमꣳ॑ शु॒क्रमि॒हेन्द्रि॒यम्।
त॒नू॒पा भि॒षजा॑ सु॒ते।
अ॒श्विनो॒भा सर॑स्वती।
मध्वा॒ रजाꣳ॑सीन्द्रि॒यम्।
इन्द्रा॑य प॒थिभि॑र्वहान्।
इन्द्रा॒येन्दु॒ꣳ॒ सर॑स्वती।
नरा॒शꣳसे॑न न॒ग्नहुः॑॥६०॥

%2.6.12.2
अधा॑ताम॒श्विना॒ मधु॑।
भे॒ष॒जं भि॒षजा॑ सु॒ते।
आ॒जुह्वा॑ना॒ सर॑स्वती।
इन्द्रा॑येन्द्रि॒याणि॑ वी॒र्यम्᳚।
इडा॑भिरश्विना॒विषम्᳚।
समूर्ज॒ꣳ॒ सꣳ र॒यिं द॑धुः।
अश्वि॑ना॒ नमु॑चेः सु॒तम्।
सोमꣳ॑ शु॒क्रं प॑रि॒स्रुता᳚।
सर॑स्वती॒ तमाभ॑रत्।
ब॒र्॒हिषेन्द्रा॑य॒ पात॑वे॥६१॥

%2.6.12.3
क॒व॒ष्यो॑ न व्यच॑स्वतीः।
अ॒श्विभ्यां॒ न दुरो॒ दिशः॑।
इन्द्रो॒ न रोद॑सी॒ दुघे᳚।
दु॒हे कामा॒न्थ्सर॑स्वती।
उ॒षासा॒ नक्त॑मश्विना।
दिवेन्द्रꣳ॑ सा॒यमि॑न्द्रि॒यैः।
स॒ञ्जा॒ना॒ने सु॒पेश॑सा।
समं॑ जाते॒ सर॑स्वत्या।
पा॒तं नो॑ अश्विना॒ दिवा᳚।
पा॒हि नक्तꣳ॑ सरस्वति॥६२॥

%2.6.12.4
दैव्या॑ होतारा भिषजा।
पा॒तमिन्द्र॒ꣳ॒ सचा॑ सु॒ते।
ति॒स्रस्त्रे॒धा सर॑स्वती।
अ॒श्विना॒ भार॒तीडा᳚।
ती॒व्रं प॑रि॒स्रुता॒ सोमम्᳚।
इन्द्रा॑य सुषवु॒र्मदम्᳚।
अश्वि॑ना भेष॒जं मधु॑।
भे॒ष॒जं नः॒ सर॑स्वती।
इन्द्रे॒ त्वष्टा॒ यशः॒ श्रियम्᳚।
रू॒पꣳ रू॑पमधुः सु॒ते।
ऋ॒तु॒थेन्द्रो॒ वन॒स्पतिः॑।
श॒श॒मा॒नः प॑रि॒स्रुता᳚।
की॒लाल॑म॒श्विभ्यां॒ मधु॑।
दु॒हे धे॒नुः सर॑स्वती।
गोभि॒र्न सोम॑मश्विना।
मास॑रेण परि॒ष्कृता᳚।
सम॑धाता॒ꣳ॒ सर॑स्वत्या।
स्वाहेन्द्रे॑ सु॒तं मधु॑॥६३॥\anuvakamend[न॒ग्नहुः॒ पात॑वे सरस्वत्यधुः सु॒ते᳚\-ऽष्टौ च॑]

%2.6.13.1
अ॒श्विना॑ ह॒विरि॑न्द्रि॒यम्।
नमु॑चेर्धि॒या सर॑स्वती।
आ शु॒क्रमा॑सु॒राद्व॒सु।
म॒घमिन्द्रा॑य जभ्रिरे।
यम॒श्विना॒ सर॑स्वती।
ह॒विषेन्द्र॒मव॑र्धयन्।
स बि॑भेद व॒लं म॒घम्।
नमु॑चावासु॒रे सचा᳚।
तमिन्द्रं॑ प॒शवः॒ सचा᳚।
अ॒श्विनो॒भा सर॑स्वती॥६४॥

%2.6.13.2
दधा॑ना अ॒भ्य॑नूषत।
ह॒विषा॑ य॒ज्ञमि॑न्द्रि॒यम्।
य इन्द्र॑ इन्द्रि॒यं द॒धुः।
स॒वि॒ता वरु॑णो॒ भगः॑।
स सु॒त्रामा॑ ह॒विष्प॑तिः।
यज॑मानाय सश्चत।
स॒वि॒ता वरु॑णो॒\-ऽदध॑त्।
यज॑मानाय दा॒शुषे᳚।
आद॑त्त॒ नमु॑चे॒र्वसु॑।
सु॒त्रामा॒ बल॑मिन्द्रि॒यम्॥६५॥

%2.6.13.3
वरु॑णः क्ष॒त्रमि॑न्द्रि॒यम्।
भगे॑न सवि॒ता श्रियम्᳚।
सु॒त्रामा॒ यश॑सा॒ बलम्᳚।
दधा॑ना य॒ज्ञमा॑शत।
अश्वि॑ना॒ गोभि॑रिन्द्रि॒यम्।
अश्वे॑भिर्वी॒र्यं॑ बलम्᳚।
ह॒विषेन्द्र॒ꣳ॒ सर॑स्वती।
यज॑मानमवर्धयन्।
ता नास॑त्या सु॒पेश॑सा।
हिर॑ण्यवर्तनी॒ नरा᳚।
सर॑स्वती ह॒विष्म॑ती।
इन्द्र॒ कर्म॑सु नोऽवत।
ता भि॒षजा॑ सु॒कर्म॑णा।
सा सु॒दुघा॒ सर॑स्वती।
स वृ॑त्र॒हा श॒तक्र॑तुः।
इन्द्रा॑य दधुरिन्द्रि॒यम्॥६६॥\anuvakamend[उ॒भा सर॑स्वती॒ बल॑मिन्द्रि॒यन्नरा॒ षट्च॑]

%2.6.14.1
दे॒वं ब॒र्॒हिः सर॑स्वती।
सु॒दे॒वमिन्द्रे॑ अ॒श्विना᳚।
तेजो॒ न चक्षु॑र॒क्ष्योः।
ब॒र्॒हिषा॑ दधुरिन्द्रि॒यम्।
व॒सु॒वने॑ वसु॒धेय॑स्य वियन्तु॒ यज॑।
दे॒वीर्द्वारो॑ अ॒श्विना᳚।
भि॒षजेन्द्रे॒ सर॑स्वती।
प्रा॒णं न वी॒र्य॑न्न॒सि।
द्वारो॑ दधुरिन्द्रि॒यम्।
व॒सु॒वने॑ वसु॒धेय॑स्य वियन्तु॒ यज॑॥६७॥

%2.6.14.2
दे॒वी उ॒षासा॑व॒श्विना᳚।
भि॒षजेन्द्रे॒ सर॑स्वती।
बलं॒ न वाच॑मा॒स्ये᳚।
उ॒षाभ्यां᳚ दधुरिन्द्रि॒यम्।
व॒सु॒वने॑ वसु॒धेय॑स्य वियन्तु॒ यज॑।
दे॒वी जोष्ट्री॑ अ॒श्विना᳚।
सु॒त्रामेन्द्रे॒ सर॑स्वती।
श्रोत्रं॒ न कर्ण॑यो॒र्यशः॑।
जोष्ट्री᳚भ्यां दधुरिन्द्रि॒यम्।
व॒सु॒वने॑ वसु॒धेय॑स्य वियन्तु॒ यज॑॥६८॥

%2.6.14.3
दे॒वी ऊ॒र्जाहु॑ती॒ दुघे॑ सु॒दुघे᳚।
पय॒सेन्द्र॒ꣳ॒ सर॑स्वत्य॒श्विना॑ भि॒षजा॑ऽवत।
शु॒क्रं न ज्योतिः॒ स्तन॑यो॒राहु॑ती धत्त इन्द्रि॒यम्।
व॒सु॒वने॑ वसु॒धेय॑स्य वियन्तु॒ यज॑।
दे॒वा दे॒वानां᳚ भि॒षजा᳚।
होता॑रा॒विन्द्र॑म॒श्विना᳚।
व॒ष॒ट्का॒रैः सर॑स्वती।
त्विषिं॒ न हृद॑ये म॒तिम्।
होतृ॑भ्यां दधुरिन्द्रि॒यम्।
व॒सु॒वने॑ वसु॒धेय॑स्य वियन्तु॒ यज॑॥६९॥

%2.6.14.4
दे॒वीस्ति॒स्रस्ति॒स्रो दे॒वीः।
सर॑स्वत्य॒श्विना॒ भार॒तीडा᳚।
शूष॒न्न मध्ये॒ नाभ्या᳚म्।
इन्द्रा॑य दधुरिन्द्रि॒यम्।
व॒सु॒वने॑ वसु॒धेय॑स्य वियन्तु॒ यज॑।
दे॒व इन्द्रो॒ नरा॒शꣳसः॑।
त्रि॒व॒रू॒थः सर॑स्वत्या॒\-ऽश्विभ्या॑मीयते॒ रथः॑।
रेतो॒ न रू॒पम॒मृतं॑ ज॒नित्रम्᳚।
इन्द्रा॑य॒ त्वष्टा॒ दध॑दिन्द्रि॒याणि॑।
व॒सु॒वने॑ वसु॒धेय॑स्य वियन्तु॒ यज॑॥७०॥

%2.6.14.5
दे॒व इन्द्रो॒ वन॒स्पतिः॑।
हिर॑ण्यपर्णो अ॒श्विभ्याम्᳚।
सर॑स्वत्याः सुपिप्प॒लः।
इन्द्रा॑य पच्यते॒ मधु॑।
ओजो॒ न जू॒तिमृ॑ष॒भो न भामम्᳚।
वन॒स्पति॑र्नो॒ दध॑दिन्द्रि॒याणि॑।
व॒सु॒वने॑ वसु॒धेय॑स्य वियन्तु॒ यज॑।
दे॒वं ब॒र्॒हिर्वारि॑तीनाम्।
अ॒ध्व॒रे स्ती॒र्णम॒श्विभ्या᳚म्।
ऊर्ण॑म्रदाः॒ सर॑स्वत्याः॥७१॥

%2.6.14.6
स्यो॒नमि॑न्द्र ते॒ सदः॑।
ई॒शायै॑ म॒न्युꣳ राजा॑नं ब॒र्॒हिषा॑ दधुरिन्द्रि॒यम्।
व॒सु॒वने॑ वसु॒धेय॑स्य वियन्तु॒ यज॑।
दे॒वो अ॒ग्निः स्वि॑ष्ट॒कृत्।
दे॒वान् य॑क्षद्यथाय॒थम्।
होता॑रा॒विन्द्र॑म॒श्विना᳚।
वा॒चा वाच॒ꣳ॒ सर॑स्वतीम्।
अ॒ग्निꣳ सोमꣴ॑ स्विष्ट॒कृत्।
स्वि॑ष्ट॒ इन्द्रः॑ सु॒त्रामा॑ सवि॒ता वरु॑णो भि॒षक्।
इ॒ष्टो दे॒वो वन॒स्पतिः॑।
स्वि॑ष्टा दे॒वा आ᳚ज्य॒पाः।
इ॒ष्टो अ॒ग्निर॒ग्निना᳚।
होता॑ हो॒त्रे स्वि॑ष्ट॒कृत्।
यशो॒ न दध॑दिन्द्रि॒यम्।
ऊर्ज॒मप॑चितिꣴ स्व॒धाम्।
व॒सु॒वने॑ वसु॒धेय॑स्य वियन्तु॒ यज॑॥७२॥\anuvakamend[द्वारो॑ दधुरिन्द्रि॒यं व॑सु॒वने॑ वसु॒धेय॑स्य वियन्तु॒ यज॒ जोष्ट्री᳚भ्यां दधुरिन्द्रि॒यं व॑सु॒वने॑ वसु॒धेय॑स्य वियन्तु॒ यज॒ होतृ॑भ्यां दधुरिन्द्रि॒यं व॑सु॒वने॑ वसु॒धेय॑स्य वियन्तु॒ यजे᳚न्द्रि॒याणि॑ वसु॒वने॑ वसु॒धेय॑स्य वियन्तु॒ यज॒ सर॑स्वत्या॒ वन॒स्पतिः॒ षट्च॑ (दे॒वं ब॒र्॒हिर्दे॒वीर्द्वारो॑ दे॒वी उ॒षासा॑व॒श्विना॑ दे॒वी जोष्ट्री॑ दे॒वी ऊ॒र्जाहु॑ती दे॒वा दे॒वानां᳚ भि॒षजा॑ वषट्का॒रैर्दे॒वीस्ति॒स्रस्ति॒स्रो दे॒वीर्दे॒व इन्द्रो॒ नरा॒शꣳसो॑ दे॒व इन्द्रो॒ वन॒स्पति॑र्दे॒वं ब॒र्॒हिर्वारि॑तीनान्दे॒वो अ॒ग्निः स्वि॑ष्ट॒कृद्दे॒वान्।
स॒मिधा॒\-ऽग्निं दे॒वं ब॒र्॒हिः सर॑स्वत्य॒श्विना॒ सर्व॑ वियन्तु।
द्वार॑स्ति॒स्रः सर्व॑वियन्तु।
अ॒ज इन्द्र॒मोजो॒\-ऽग्निं परः॒ सर॑स्वतीम्।
नक्तं॒ पूर्वः॒ सर॑स्वति।
अ॒न्यत्र॒ सर॑स्वती।
भि॒षक्पूर्वं॑ दुह इन्द्रि॒यम्।
अ॒न्यत्र॑ दधुरिन्द्रि॒यम्।
सौ॒त्रा॒म॒ण्याꣳ सु॑तासु॒ती।
अ॒ञ्जन्त्य॒यं यज॑मानः॥)]

%2.6.15.1
अ॒ग्निम॒द्य होता॑रमवृणीत।
अ॒यꣳ सु॑तासु॒ती यज॑मानः।
पच॑न्प॒क्तीः।
पच॑न्पुरो॒डाशान्॑।
गृ॒ह्णन्ग्रहान्॑।
ब॒ध्नन्न॒श्विभ्यां॒ छाग॒ꣳ॒ सर॑स्वत्या॒ इन्द्रा॑य।
ब॒ध्नन्थ्सर॑स्वत्यै मे॒षमिन्द्रा॑या॒श्वि\-भ्या᳚म्।
ब॒ध्नन्निन्द्रा॑यर्‌\mbox{}ष॒भम॒श्विभ्या॒ꣳ॒ सर॑स्वत्यै।
सू॒प॒स्था अ॒द्य दे॒वो वन॒स्पति॑रभवत्।
अ॒श्विभ्यां॒ छागे॑न॒ सर॑स्वत्या॒ इन्द्रा॑य॥७३॥

%2.6.15.2
सर॑स्वत्यै मे॒षेणेन्द्रा॑या॒श्विभ्या᳚म्।
इन्द्रा॑यर्‌\mbox{}ष॒भेणा॒श्विभ्या॒ꣳ॒ सर॑स्वत्यै।
अक्ष॒ꣴ॒स्तान्मे॑द॒स्तः प्रति॑\-पच॒ताग्र॑भीषुः।
अवी॑वृधन्त॒ ग्रहैः᳚।
अपा॑ता\-म॒श्विना॒ सर॑स्व॒तीन्द्रः॑ सु॒त्रामा॑ वृत्र॒हा।
सोमा᳚न्थ्सु॒राम्णः॑।
उपो॑ उक्थाम॒दाः श्रौ॒द्विमदा॑ अदन्।
अवी॑वृधन्ताङ्गू॒षैः।
त्वाम॒द्यर्‌\mbox{}ष॑ आर्‌\mbox{}षेयर्‌\mbox{}षीणां नपादवृणीत।
अ॒यꣳ सु॑तासु॒ती यज॑मानः।
ब॒हुभ्य॒ आ सङ्ग॑तेभ्यः।
ए॒ष मे॑ दे॒वेषु॒ वसु॒ वार्या य॑क्ष्यत॒ इति॑।
ता या दे॒वा दे॑व॒दाना॒न्यदुः॑।
तान्य॑स्मा॒ आ च॒ शास्व॑।
आ च॑ गुरस्व।
इ॒षि॒तश्च॑ होत॒रसि॑ भद्र॒वाच्या॑य॒ प्रेषि॑तो॒ मानु॑षः।
सू॒क्त॒वा॒काय॑ सू॒क्ता ब्रू॑हि॥७४॥\anuvakamend[इन्द्रा॑य॒ यज॑मानः स॒प्त च॑]

%2.6.16.1
उ॒शन्त॑स्त्वा हवामह॒ आ नो॑ अग्ने सुके॒तुना᳚।
त्वꣳ सो॑म म॒हे भगं॒ त्वꣳ सो॑म॒ प्रचि॑कितो मनी॒षा।
त्वया॒ हि नः॑ पि॒तरः॑ सोम॒ पूर्वे॒ त्वꣳ सो॑म पि॒तृभिः॑ संविदा॒नः।
बर्‌\mbox{}हि॑षदः पितर॒ आऽहं पि॒तॄन्।
उप॑हूताः पि॒तरो\-ऽग्नि॑ष्वात्ताः पितरः।
अ॒ग्नि॒ष्वा॒त्तानृ॑तु॒मतो॑ हवामहे।
नरा॒शꣳसे॑ सोमपी॒थं य आ॒शुः।
ते नो॒ अर्व॑न्तः सु॒हवा॑ भवन्तु।
शं नो॑ भवन्तु द्वि॒पदे॒ शं चतु॑ष्पदे।
ये अ॑ग्निष्वा॒त्ता येऽन॑ग्निष्वात्ताः॥७५॥

%2.6.16.2
अ॒ꣳ॒हो॒मुचः॑ पि॒तरः॑ सो॒म्यासः॑।
परेऽव॑रे॒\-ऽमृता॑सो॒ भव॑न्तः।
अधि॑ ब्रुवन्तु॒ ते अ॑वन्त्व॒स्मान्।
वा॒न्या॑यै दु॒ग्धे जु॒षमा॑णाः कर॒म्भम्।
उ॒दीरा॑णा॒ अव॑रे॒ परे॑ च।
अ॒ग्नि॒ष्वा॒त्ता ऋ॒तुभिः॑ संविदा॒नाः।
इन्द्र॑वन्तो ह॒विरि॒दं जु॑षन्ताम्।
यद॑ग्ने कव्यवाहन॒ त्वम॑ग्न ईडि॒तो जा॑तवेदः।
मात॑ली क॒व्यैः।
ये ता॑तृ॒पुर्दे॑व॒त्रा जेह॑मानाः।
हो॒त्रा॒वृधः॒ स्तोम॑तष्टासो अ॒र्कैः।
आऽग्ने॑ याहि सुवि॒दत्रे॑भिर॒र्वाङ्।
स॒त्यैः क॒व्यैः पि॒तृभि॑र्घर्म॒सद्भिः॑।
ह॒व्य॒वाह॑म॒जरं॑ पुरुप्रि॒यम्।
अ॒ग्निं घृ॒तेन॑ ह॒विषा॑ सप॒र्यन्।
उपा॑सदं कव्य॒वाहं॑ पितृ॒णाम्।
स नः॑ प्र॒जां वी॒रव॑ती॒ꣳ॒ समृ॑ण्वतु॥७६॥\anuvakamend[अन॑ग्निष्वात्ता॒ जेह॑मानाः स॒प्त च॑]

%2.6.17.1
होता॑ यक्षदि॒डस्प॒दे।
स॒मि॒धा॒नं म॒हद्यशः॑।
सुष॑मिद्धं॒ वरे᳚ण्यम्।
अ॒ग्निमिन्द्रं॑ वयो॒धसम्᳚।
गा॒य॒त्रीं छन्द॑ इन्द्रि॒यम्।
त्र्यविं॒ गां वयो॒ दध॑त्।
वेत्वा\-ऽ\-ऽ\-ज्य॑स्य॒ होत॒र्यज॑।
होता॑ यक्ष॒च्छुचि॑व्रतम्।
तनू॒नपा॑तमु॒द्भिदम्᳚।
यं गर्भ॒मदि॑तिर्द॒धे॥७७॥

%2.6.17.2
शुचि॒मिन्द्रं॑ वयो॒धसम्᳚।
उ॒ष्णिहं॒ छन्द॑ इन्द्रि॒यम्।
दि॒त्य॒वाहं॒ गां वयो॒ दध॑त्।
वेत्वा\-ऽ\-ऽ\-ज्य॑स्य॒ होत॒र्यज॑।
होता॑ यक्षदी॒डेन्यम्᳚।
ई॒डि॒तं वृ॑त्र॒हन्त॑मम्।
इडा॑भि॒रीड्य॒ꣳ॒ सहः॑।
सोम॒मिन्द्रं॑ वयो॒धसम्᳚।
अ॒नु॒ष्टुभं॒ छन्द॑ इन्द्रि॒यम्।
त्रि॒व॒थ्सं गां वयो॒ दध॑त्॥७८॥

%2.6.17.3
वेत्वा\-ऽ\-ऽ\-ज्य॑स्य॒ होत॒र्यज॑।
होता॑ यक्षथ्सुबर्\mbox{}हि॒षदम्᳚।
पू॒ष॒ण्वन्त॒मम॑र्त्यम्।
सीद॑न्तं ब॒र्॒हिषि॑ प्रि॒ये।
अ॒मृतेन्द्रं॑ वयो॒धसम्᳚।
बृ॒ह॒तीं छन्द॑ इन्द्रि॒यम्।
पञ्चा॑विं॒ गां वयो॒ दध॑त्।
वेत्वा\-ऽ\-ऽ\-ज्य॑स्य॒ होत॒र्यज॑।
होता॑यक्ष॒द्व्यच॑स्वतीः।
सु॒प्रा॒य॒णा ऋ॑ता॒वृधः॑॥७९॥

%2.6.17.4
द्वारो॑ दे॒वीर्‌\mbox{}हि॑र॒ण्ययीः᳚।
ब्र॒ह्माण॒ इन्द्रं॑ वयो॒धसम्᳚।
प॒ङ्क्तिं छन्द॑ इ॒हेन्द्रि॒यम्।
तु॒र्य॒वाहं॒ गां वयो॒ दध॑त्।
वेत्वा\-ऽ\-ऽ\-ज्य॑स्य॒ होत॒र्यज॑।
होता॑ यक्षथ्सु॒पेश॑से।
सु॒शि॒ल्पे बृ॑ह॒ती उ॒भे।
नक्तो॒षासा॒ न द॑र्‌\mbox{}श॒ते।
विश्व॒मिन्द्रं॑ वयो॒धसम्᳚।
त्रि॒ष्टुभं॒ छन्द॑ इन्द्रि॒यम्॥८०॥

%2.6.17.5
प॒ष्ठ॒वाहं॒ गां वयो॒ दध॑त्।
वेत्वा\-ऽ\-ऽ\-ज्य॑स्य॒ होत॒र्यज॑।
होता॑ यक्ष॒त्प्रचे॑तसा।
दे॒वाना॑मुत्त॒मं यशः॑।
होता॑रा॒ दैव्या॑ क॒वी।
स॒युजेन्द्रं॑ वयो॒धसम्᳚।
जग॑तीं॒ छन्द॑ इ॒हेन्द्रि॒यम्।
अ॒न॒ड्वाहं॒ गां वयो॒ दध॑त्।
वेत्वा\-ऽ\-ऽ\-ज्य॑स्य॒ होत॒र्यज॑।
होता॑ यक्ष॒त्पेश॑स्वतीः॥८१॥

%2.6.17.6
ति॒स्रो दे॒वीर्‌\mbox{}हि॑र॒ण्ययीः᳚।
भार॑तीर्बृह॒तीर्म॒हीः।
पति॒मिन्द्रं॑ वयो॒धसम्᳚।
वि॒राजं॒ छन्द॑ इ॒हेन्द्रि॒यम्।
धे॒नुं गां न वयो॒ दध॑त्।
वेत्वा\-ऽ\-ऽ\-ज्य॑स्य॒ होत॒र्यज॑।
होता॑ यक्षथ्सु॒रेत॑सम्।
त्वष्टा॑रं पुष्टि॒वर्ध॑नम्।
रू॒पाणि॒ बिभ्र॑तं॒ पृथ॑क्।
पुष्टि॒मिन्द्रं॑ वयो॒धसम्᳚॥८२॥

%2.6.17.7
द्वि॒पदं॒ छन्द॑ इ॒हेन्द्रि॒यम्।
उ॒क्षाणं॒ गां न वयो॒ दध॑त्।
वेत्वा\-ऽ\-ऽ\-ज्य॑स्य॒ होत॒र्यज॑।
होता॑ यक्षच्छ॒तक्र॑तुम्।
हिर॑ण्य\-पर्ण\-मु॒क्थिनम्᳚।
र॒श॒नां बिभ्र॑तं व॒शिम्।
भग॒मिन्द्रं॑ वयो॒धसम्᳚।
क॒कुभं॒ छन्द॑ इ॒हेन्द्रि॒यम्।
व॒शां वे॒हतं॒ गां न वयो॒ दध॑त्।
वेत्वा\-ऽ\-ऽ\-ज्य॑स्य॒ होत॒र्यज॑।
होता॑ यक्ष॒थ्स्वाहा॑कृतीः।
अ॒ग्निं गृ॒हप॑तिं॒ पृथ॑क्।
वरु॑णं भेष॒जं क॒विम्।
क्ष॒त्रमिन्द्रं॑ वयो॒धसम्᳚।
अति॑च्छन्दसं॒ छन्द॑ इन्द्रि॒यम्।
बृ॒हदृ॑ष॒भं गां वयो॒ दध॑त्।
वेत्वा\-ऽ\-ऽ\-ज्य॑स्य॒ होत॒र्यज॑॥८३॥\anuvakamend[द॒धे दध॑दृता॒वृध॑ इन्द्रि॒यं पेश॑स्वतीर्वयो॒धसं॒ वेत्वा\-ऽ\-ऽ\-ज्य॑स्य॒ होत॒र्यज॑ स॒प्त च॑ (इ॒डस्प॒दे᳚\-ऽग्निङ्गा॑य॒त्रीन्त्र्यविम्᳚।
शुचि॑व्रत॒ꣳ॒ शुचि॑मु॒ष्णिह॑न्दित्य॒वाहम्᳚।
ई॒डेन्य॒ꣳ॒ सोम॑मनु॒ष्टुभं॑ त्रिव॒थ्सम्।
सु॒ब॒र्॒हि॒षद॑म॒मृतेन्द्रं॑ बृह॒तीं पञ्चा॑विम्।
व्यच॑स्वतीः सुप्राय॒णा द्वारो᳚ ब्र॒ह्माणः॑ प॒ङ्क्तिमि॒ह तु॑र्य॒वाहम्᳚।
सु॒पेश॑से॒ विश्व॒मिन्द्रं॑ त्रि॒ष्टुभं॑ पष्ठ॒वाहम्᳚।
प्रचे॑तसा स॒युजेन्द्रं॒ जग॑तीमि॒हान॒ड्वाहम्᳚।
पेश॑स्वतीस्ति॒स्रः पतिं॑ वि॒राज॑मि॒ह धे॒नुन्न।
सु॒रेत॑स॒न्त्वष्टा॑रं॒ पुष्टि॒मिन्द्रं॑ द्वि॒पद॑मि॒होक्षाण॒न्न।
श॒तक्र॑तुं॒ भग॒मिन्द्रं॑ क॒कुभ॑मि॒ह व॒शान्न।
स्वाहा॑कृतीः क्ष॒त्रमति॑च्छन्दसं बृ॒हदृ॑ष॒भं गां वयो॒ दध॑दिन्द्रि॒यमृषि॑ वसु॒ नव॑ द॒शेहे᳚न्द्रि॒यमष्ट॑ नव दश॒ गां न वयो॒ दध॑दि॒डस्प॒दे सर्व॑ वेतु॥)]

%2.6.18.1
समि॑द्धो अ॒ग्निः स॒मिधा᳚।
सुष॑मिद्धो॒ वरे᳚ण्यः।
गा॒य॒त्री छन्द॑ इन्द्रि॒यम्।
त्र्यवि॒र्गौर्वयो॑ दधुः।
तनू॒नपा॒च्छुचि॑व्रतः।
त॒नू॒पाच्च॒ सर॑स्वती।
उ॒ष्णिक्छन्द॑ इन्द्रि॒यम्।
दि॒त्य॒वाड्गौर्वयो॑ दधुः।
इडा॑भिर॒ग्निरीड्यः॑।
सोमो॑ दे॒वो अम॑र्त्यः॥८४॥

%2.6.18.2
अ॒नु॒ष्टुप्छन्द॑ इन्द्रि॒यम्।
त्रि॒व॒थ्सो गौर्वयो॑ दधुः।
सु॒ब॒र्॒हिर॒ग्निः पू॑ष॒ण्वान्।
स्ती॒र्णब॑र्हि॒रम॑र्त्यः।
बृ॒ह॒ती छन्द॑ इन्द्रि॒यम्।
पञ्चा॑वि॒र्गौर्वयो॑ दधुः।
दुरो॑ दे॒वीर्दिशो॑ म॒हीः।
ब्र॒ह्मा दे॒वो बृह॒स्पतिः॑।
प॒ङ्क्तिश्छन्द॑ इ॒हेन्द्रि॒यम्।
तु॒र्य॒वाड्गौर्वयो॑ दधुः॥८५॥

%2.6.18.3
उ॒षे य॒ह्वी सु॒पेश॑सा।
विश्वे॑ दे॒वा अम॑र्त्याः।
त्रि॒ष्टुप्छन्द॑ इन्द्रि॒यम्।
प॒ष्ठ॒वाद्गौर्वयो॑ दधुः।
दैव्या॑ होतारा भिषजा।
इन्द्रे॑ण स॒युजा॑ यु॒जा।
जग॑ती॒ छन्द॑ इ॒हेन्द्रि॒यम्।
अ॒न॒ड्वान्गौर्वयो॑ दधुः।
ति॒स्र इडा॒ सर॑स्वती।
भार॑ती म॒रुतो॒ विशः॑॥८६॥

%2.6.18.4
वि॒राट्छन्द॑ इ॒हेन्द्रि॒यम्।
धे॒नुर्गौर्न वयो॑ दधुः।
त्वष्टा॑ तु॒रीपो॒ अद्भु॑तः।
इ॒न्द्रा॒ग्नी पु॑ष्टि॒वर्ध॑ना।
द्वि॒पाच्छन्द॑ इ॒हेन्द्रि॒यम्।
उ॒क्षा गौर्न वयो॑ दधुः।
श॒मि॒ता नो॒ वन॒स्पतिः॑।
स॒वि॒ता प्र॑सु॒वन्भगम्᳚।
क॒कुच्छन्द॑ इ॒हेन्द्रि॒यम्।
व॒शा वे॒हद्गौर्न वयो॑ दधुः।
स्वाहा॑ य॒ज्ञं वरु॑णः।
सु॒क्ष॒त्रो भे॑ष॒जं क॑रत्।
अति॑च्छन्दा॒श्छन्द॑ इन्द्रि॒यम्।
बृ॒हदृ॑ष॒भो गौर्वयो॑ दधुः॥८७॥\anuvakamend[अम॑र्त्यस्तुर्य॒वाड्गौर्वयो॑ दधु॒र्विशो॑ व॒शा वे॒हद्गौर्न वयो॑ दधुश्च॒त्वारि॑ च]

%2.6.19.1
व॒स॒न्तेन॒र्तुना॑ दे॒वाः।
वस॑वस्त्रि॒वृता᳚ स्तु॒तम्।
र॒थ॒न्त॒रेण॒ तेज॑सा।
ह॒विरिन्द्रे॒ वयो॑ दधुः।
ग्री॒ष्मेण॑ दे॒वा ऋ॒तुना᳚।
रु॒द्राः प॑ञ्चद॒शे स्तु॒तम्।
बृ॒ह॒ता यश॑सा॒ बलम्᳚।
ह॒विरिन्द्रे॒ वयो॑ दधुः।
व॒र्॒षाभि॑र्\mbox{}ऋ॒तुना॑\-ऽऽदि॒त्याः।
स्तोमे॑ सप्तद॒शे स्तु॒तम्॥८८॥

%2.6.19.2
वै॒रू॒पेण॑ वि॒शौज॑सा।
ह॒विरिन्द्रे॒ वयो॑ दधुः।
शा॒र॒देन॒र्तुना॑ दे॒वाः।
ए॒क॒वि॒ꣳ॒श ऋ॒भवः॑ स्तु॒तम्।
वै॒रा॒जेन॑ श्रि॒या श्रियम्᳚।
ह॒विरिन्द्रे॒ वयो॑ दधुः।
हे॒म॒न्तेन॒र्तुना॑ दे॒वाः।
म॒रुत॑स्त्रिण॒वे स्तु॒तम्।
बले॑न॒ शक्व॑रीः॒ सहः॑।
ह॒विरिन्द्रे॒ वयो॑ दधुः।
शै॒शि॒रेण॒र्तुना॑ दे॒वाः।
त्र॒य॒स्त्रि॒ꣳ॒शे॑\-ऽमृतꣴ॑ स्तु॒तम्।
स॒त्येन॑ रे॒वतीः᳚ क्ष॒त्रम्।
ह॒विरिन्द्रे॒ वयो॑ दधुः॥८९॥\anuvakamend[स्तोमे॑ सप्तद॒शे स्तु॒तꣳ सहो॑ ह॒विरिन्द्रे॒ वयो॑ दधुश्च॒त्वारि॑ च (व॒स॒न्तेन॑ ग्री॒ष्मेण॑ व॒र्‌॒\mbox{}षाभिः॑ शार॒देन॑ हेम॒न्तेन॑ शैशि॒रेण॒ षट्॥)]

%2.6.20.1
दे॒वं ब॒र्॒हिरिन्द्रं॑ वयो॒धसम्᳚।
दे॒वं दे॒वम॑वर्धयत्।
गा॒य॒त्रि॒या छन्द॑सेन्द्रि॒यम्।
तेज॒ इन्द्रे॒ वयो॒ दध॑त्।
व॒सु॒वने॑ वसु॒धेय॑स्य वेतु॒ यज॑।
दे॒वीर्द्वारो॑ दे॒वमिन्द्रं॑ वयो॒धसम्᳚।
दे॒वीर्दे॒वम॑वर्धयन्।
उ॒ष्णिहा॒ छन्द॑सेन्द्रि॒यम्।
प्रा॒णमिन्द्रे॒ वयो॒ दध॑त्।
व॒सु॒वने॑ वसु॒धेय॑स्य वियन्तु॒ यज॑॥९०॥

%2.6.20.2
दे॒वी दे॒वं व॑यो॒धसम्᳚।
उ॒षे इन्द्र॑मवर्धताम्।
अ॒नु॒ष्टुभा॒ छन्द॑सेन्द्रि॒यम्।
वाच॒मिन्द्रे॒ वयो॒ दध॑त्।
व॒सु॒वने॑ वसु॒धेय॑स्य वीतां॒ यज॑।
दे॒वी जोष्ट्री॑ दे॒वमिन्द्रं॑ वयो॒धसम्᳚।
दे॒वी दे॒वम॑वर्धताम्।
बृ॒ह॒त्या छन्द॑सेन्द्रि॒यम्।
श्रोत्र॒मिन्द्रे॒ वयो॒ दध॑त्।
व॒सु॒वने॑ वसु॒धेय॑स्य वीतां॒ यज॑॥९१॥

%2.6.20.3
दे॒वी ऊ॒र्जाहु॑ती दे॒वमिन्द्रं॑ वयो॒धसम्᳚।
दे॒वी दे॒वम॑वर्धताम्।
प॒ङ्क्त्या छन्द॑सेन्द्रि॒यम्।
शु॒क्रमिन्द्रे॒ वयो॒ दध॑त्।
व॒सु॒वने॑ वसु॒धेय॑स्य वीतां॒ यज॑।
दे॒वा दैव्या॒ होता॑रा दे॒वमिन्द्रं॑ वयो॒धसम्᳚।
दे॒वा दे॒वम॑वर्धताम्।
त्रि॒ष्टुभा॒ छन्द॑सेन्द्रि॒यम्।
त्विषि॒मिन्द्रे॒ वयो॒ दध॑त्।
व॒सु॒वने॑ वसु॒धेय॑स्य वीतां॒ यज॑॥९२॥

%2.6.20.4
दे॒वीस्ति॒स्रस्ति॒स्रो दे॒वीर्व॑यो॒धसम्᳚।
पति॒मिन्द्र॑मवर्धयन्।
जग॑त्या॒ छन्द॑सेन्द्रि॒यम्।
बल॒मिन्द्रे॒ वयो॒ दध॑त्।
व॒सु॒व॒ने॑ वसु॒धेय॑स्य वियन्तु॒ यज॑।
दे॒वो नरा॒शꣳसो॑ दे॒वमिन्द्रं॑ वयो॒धसम्᳚।
दे॒वो दे॒वम॑वर्धयत्।
वि॒राजा॒ छन्द॑सेन्द्रि॒यम्।
रेत॒ इन्द्रे॒ वयो॒ दध॑त्।
व॒सु॒वने॑ वसु॒धेय॑स्य वेतु॒ यज॑॥९३॥

%2.6.20.5
दे॒वो वन॒स्पति॑र्दे॒वमिन्द्रं॑ वयो॒धसम्᳚।
दे॒वो दे॒वम॑वर्धयत्।
द्वि॒पदा॒ छन्द॑सेन्द्रि॒यम्।
भग॒मिन्द्रे॒ वयो॒ दध॑त्।
व॒सु॒वने॑ वसु॒धेय॑स्य वेतु॒ यज॑।
दे॒वं ब॒र्॒हिर्वारि॑तीनां दे॒वमिन्द्रं॑ वयो॒धसम्᳚।
दे॒वं दे॒वम॑वर्धयत्।
क॒कुभा॒ छन्द॑सेन्द्रि॒यम्।
यश॒ इन्द्रे॒ वयो॒ दध॑त्।
व॒सु॒वने॑ वसु॒धेय॑स्य वेतु॒ यज॑।
दे॒वो अ॒ग्निः स्वि॑ष्ट॒कृद्दे॒वमिन्द्रं॑ वयो॒धसम्᳚।
दे॒वो दे॒वम॑वर्धयत्।
अति॑च्छन्दसा॒ छन्द॑सेन्द्रि॒यम्।
क्ष॒त्रमिन्द्रे॒ वयो॒ दध॑त्।
व॒सु॒वने॑ वसु॒धेय॑स्य वेतु॒ यज॑॥९४॥\anuvakamend[वि॒य॒न्तु॒ यज॑ वीतां॒ यज॑ वीतां॒ यज॑ वेतु॒ यज॑ वेतु॒ यज॒ पञ्च॑ च (दे॒वं ब॒र्॒हिर्गा॑यत्रि॒या तेजः॑।
दे॒वीर्द्वार॑ उ॒ष्णिहा᳚ प्रा॒णम्।
दे॒वी दे॒वमु॒षे अ॑नु॒ष्टुभा॒ वाचम्᳚।
दे॒वी जोष्ट्री॑ बृह॒त्या श्रोत्रम्᳚।
दे॒वी ऊ॒र्जाहु॑ती प॒ङ्क्त्या शु॒क्रम्।
दे॒वा दैव्या॒ होता॑रा त्रि॒ष्टुभा॒ त्विषिम्᳚।
दे॒वीस्ति॒स्रस्ति॒स्रो दे॒वीः पतिं॒ जग॑त्या॒ बलम्᳚।
दे॒वो नरा॒शꣳसो॑ वि॒राजा॒ रेतः॑।
दे॒वो वन॒स्पति॑र्द्वि॒पदा॒ भगम्᳚।
दे॒वं ब॒र्॒हिर्वारि॑तीनां क॒कुभा॒ यशः॑।
दे॒वो अ॒ग्निः स्वि॑ष्ट॒कृदति॑च्छन्दसा क्ष॒त्रम्।
वे॒तु॒ वि॒य॒न्तु॒ च॒तुर्वी॑ता॒मेको॑ वियन्तु च॒तुर्वे᳚त्ववर्धयदवर्धयꣴश्च॒तुर॑वर्धता॒मेको॑\-ऽवर्धयꣴ श्च॒तुर॑वर्धयत्॥)]




\prashnaend{स्वा॒द्वीं त्वा॒ सोमः॒ सुरा॑वन्तꣳ॒ सीसे॑न मि॒त्रो॑ऽसि॒ यद्दे॑वा॒ होता॑ यक्षथ्स॒मिधेन्द्र॒ꣳ॒ समि॑द्ध॒ इन्द्र॒ आच॑र्‌\mbox{}षणि॒प्रा दे॒वं ब॒र्॒हिर्‌\mbox{}होता॑ यक्षथ्स॒मिधा॒\-ऽग्निꣳ समि॑द्धो अ॒ग्निर॑श्विना॒\-ऽश्विना॑ ह॒विरि॑न्द्रि॒यं दे॒वं ब॒र्॒हिः सर॑स्वत्य॒ग्निम॒द्योशन्तो॒ होता॑ यक्षदि॒डस्प॒दे समि॑द्धो अ॒ग्निः स॒मिधा॑ वस॒न्तेन॒र्तुना॑ दे॒वं ब॒र्॒हिरिन्द्रं॑ वयो॒धसं॑ विꣳश॒तिः॥२०॥}{स्वा॒द्वीं त्वा\-ऽमी॑मदन्त पि॒तरः॒ साम्रा᳚ज्याय पू॒तं प॒वित्रे॑णो॒षासा॒नक्ता॒ बद॑रै॒रधा॑तां दे॒व इन्द्रो॒ वन॒स्पतिः॑ पष्ठ॒वाह॒ङ्गां दे॒वी दे॒वं व॑यो॒धसं॒ चतु॑र्नवतिः॥९४॥}{स्वा॒द्वीं त्वा॑ वेतु॒ यज॑॥}{हरिः॑ ओम्॥}{इति श्रीकृष्णयजुर्वेदीयतैत्तिरीयब्राह्मणे द्वितीयाष्टके षष्ठः प्रपाठकः समाप्तः॥}
\clearpage
\sect{सप्तमः प्रश्नः}
\setcounter{anuvakam}{0}
\dnsub{तैत्तिरीयब्राह्मणे द्वितीयाष्टके सप्तमः प्रपाठकः}

%2.7.1.1
त्रि॒वृथ्स्तोमो॑ भवति।
ब्र॒ह्म॒व॒र्च॒सं वै त्रि॒वृत्।
ब्र॒ह्म॒व॒र्च॒समे॒वाव॑ रुन्धे।
अ॒ग्नि॒ष्टो॒मः सोमो॑ भवति।
ब्र॒ह्म॒व॒र्च॒सं वा अ॑ग्निष्टो॒मः।
ब्र॒ह्म॒व॒र्च॒समे॒वाव॑ रुन्धे।
र॒थ॒न्त॒रꣳ साम॑ भवति।
ब्र॒ह्म॒व॒र्च॒सं वै र॑थन्त॒रम्।
ब्र॒ह्म॒व॒र्च॒समे॒वाव॑ रुन्धे।
प॒रि॒स्र॒जी होता॑ भवति॥१॥

%2.7.1.2
अ॒रु॒णो मि॑र्मि॒रस्त्रिशु॑क्रः।
ए॒तद्वै ब्र॑ह्म\-वर्च॒सस्य॑ रू॒पम्।
रू॒पेणै॒व ब्र॑ह्म\-वर्च॒समव॑ रुन्धे। 
बृह॒स्पति॑रकामयत दे॒वानां᳚ पुरो॒धां ग॑च्छेय॒मिति॑।
स ए॒तं बृ॑हस्पतिस॒वम॑पश्यत्।
तमाऽह॑रत्।
तेना॑यजत।
ततो॒ वै स दे॒वानां᳚ पुरो॒धाम॑गच्छत्।
यः पु॑रो॒धाका॑मः॒ स्यात्।
स बृ॑हस्पतिस॒वेन॑ यजेत॥२॥

%2.7.1.3
पु॒रो॒धामे॒व ग॑च्छति।
तस्य॑ प्रातः सव॒ने स॒न्नेषु॑ नाराश॒ꣳ॒सेषु॑।
एका॑दश॒ दक्षि॑णा नीयन्ते।
एका॑दश॒ माध्यं॑ दिने॒ सव॑ने स॒न्नेषु॑ नाराश॒ꣳ॒सेषु॑।
एका॑दश तृतीयसव॒ने स॒न्नेषु॑ नाराश॒ꣳ॒सेषु॑।
त्रय॑स्त्रिꣳश॒थ्सम्प॑द्यन्ते।
त्रय॑स्त्रिꣳश॒द्वै दे॒वताः᳚।
दे॒वता॑ ए॒वाव॑ रुन्धे।
अश्व॑श्चतुस्त्रि॒ꣳ॒शः।
प्रा॒जा॒प॒त्यो वा अश्वः॑॥३॥

%2.7.1.4
प्र॒जा\-प॑तिश्चतुस्त्रि॒ꣳ॒शो दे॒वता॑नाम्।
याव॑तीरे॒व दे॒वताः᳚।
ता ए॒वाव॑ रुन्धे।
कृ॒ष्णा॒जि॒ने॑\-ऽभिषि॑ञ्चति।
ब्रह्म॑णो॒ वा ए॒तद्रू॒पम्।
यत्कृ॑ष्णाजि॒नम्।
ब्र॒ह्म॒व॒र्च॒सेनै॒वैन॒ꣳ॒ सम॑र्धयति।
आज्ये॑ना॒भिषि॑ञ्चति।
तेजो॒ वा आज्यम्᳚।
तेज॑ ए॒वास्मि॑न्दधाति॥४॥\anuvakamend[होता॑ भवति यजेत॒ वा अश्वो॑ दधाति]

%2.7.2.1
यदा᳚ग्ने॒यो भव॑ति।
अ॒ग्निमु॑खा॒ ह्यृद्धिः॑।
अथ॒ यत्पौ॒ष्णः।
पुष्टि॒र्वै पू॒षा।
पुष्टि॒र्वैश्य॑स्य।
पुष्टि॑मे॒वाव॑ रुन्धे।
प्र॒स॒वाय॑ सावि॒त्रः।
अथ॒ यत्त्वा॒ष्ट्रः।
त्वष्टा॒ हि रू॒पाणि॑ विक॒रोति॑।
नि॒र्व॒रु॒ण॒त्वाय॑ वारु॒णः॥५॥

%2.7.2.2
अथो॒ य ए॒व कश्च॒ सन्थ्सू॒यते᳚।
स हि वा॑रु॒णः।
अथ॒ यद्वै᳚श्वदे॒वः।
वै॒श्व॒दे॒वो हि वैश्यः॑।
अथ॒ यन्मा॑रु॒तः।
मा॒रु॒तो हि वैश्यः॑।
स॒प्तैतानि॑ ह॒वीꣳषि॑ भवन्ति।
स॒प्तग॑णा॒ वै म॒रुतः॑।
पृश्ञिः॑ पष्ठौ॒ही मा॑रु॒त्या ल॑भ्यते।
विड्वै म॒रुतः॑।
विश॑ ए॒वैतन्म॑ध्य॒तो॑\-ऽभिषि॑च्यते।
तस्मा॒द्वा ए॒ष वि॒शः प्रि॒यः।
वि॒शो हि म॑ध्य॒तो॑\-ऽभिषि॒च्यते᳚।
ऋ॒ष॒भ॒च॒र्मे\-ऽध्य॒भिषि॑ञ्चति।
स हि प्र॑जनयि॒ता।
द॒ध्ना\-ऽभिषि॑ञ्चति।
ऊर्ग्वा अ॒न्नाद्यं॒ दधि॑।
ऊ॒र्जैवैन॑म॒न्नाद्ये॑न॒ सम॑र्धयति॥६॥\anuvakamend[वा॒रु॒णो विड्वै म॒रुतो॒\-ऽष्टौ च॑]

%2.7.3.1
यदा᳚ग्ने॒यो भव॑ति।
आ॒ग्ने॒यो वै ब्रा᳚ह्म॒णः।
अथ॒ यथ्सौ॒म्यः।
सौ॒म्यो हि ब्रा᳚ह्म॒णः।
प्र॒स॒वायै॒व सा॑वि॒त्रः।
अथ॒ यद्बा॑र्\mbox{}हस्प॒त्यः।
ए॒तद्वै ब्रा᳚ह्म॒णस्य॑ वाक्प॒तीयम्᳚।
अथ॒ यद॑ग्नीषो॒मीयः॑।
आ॒ग्ने॒यो वै ब्रा᳚ह्म॒णः।
तौ य॒दा स॒ङ्गच्छे॑ते॥७॥

%2.7.3.2
अथ॑ वी॒र्या॑वत्तरो भवति।
अथ॒ यथ्सा॑रस्व॒तः।
ए॒तद्धि प्र॒त्यक्षं॑ ब्राह्म॒णस्य॑ वाक्प॒तीयम्᳚।
नि॒र्व॒रु॒ण॒त्वायै॒व वा॑रु॒णः।
अथो॒ य ए॒व कश्च॒ सन्थ्सू॒यते᳚।
स हि वा॑रु॒णः।
अथ॒ यद्द्या॑वापृथि॒व्यः॑।
इन्द्रो॑ वृ॒त्राय॒ वज्र॒मुद॑यच्छत्।
तं द्यावा॑पृथि॒वी नान्व॑मन्येताम्।
तमे॒तेनै॒व भा॑ग॒धेये॒नान्व॑मन्येताम्॥८॥

%2.7.3.3
वज्र॑स्य॒ वा ए॒षो॑\-ऽनुमा॒नाय॑।
अनु॑मतवज्रः सूयाता॒ इति॑।
अ॒ष्टावे॒तानि॑ ह॒वीꣳषि॑ भवन्ति।
अ॒ष्टाक्ष॑रा गाय॒त्री।
गा॒य॒त्री ब्र॑ह्म\-वर्च॒सम्।
गा॒य॒त्रि॒यैव ब॑ह्म\-वर्च॒समव॑ रुन्धे।
हिर॑ण्येन घृ॒तमुत्पु॑नाति।
तेज॑स ए॒व रु॒चे।
कृ॒ष्णा॒जि॒ने॑\-ऽभिषि॑ञ्चति।
ब्रह्म॑णो॒ वा ए॒तदृ॑ख्सा॒मयो॑ रू॒पम्।
यत्कृ॑ष्णाजि॒नम्।
ब्रह्म॑न्ने॒वैन॑मृख्सा॒मयो॒रध्य॒भिषि॑ञ्चति।
घृ॒तेना॒भिषि॑ञ्चति।
तथा॑ वी॒र्या॑वत्तरो भवति॥९॥\anuvakamend[स॒ङ्गच्छे॑ते भाग॒धेये॒नान्व॑मन्येताꣳ रू॒पं च॒त्वारि॑ च]

%2.7.4.1
न वै सोमे॑न॒ सोम॑स्य स॒वो᳚\-ऽस्ति।
ह॒तो ह्ये॑षः।
अ॒भिषु॑तो॒ ह्ये॑षः।
न हि ह॒तः सू॒यते᳚।
सौ॒मीꣳ सू॒तव॑शा॒मा ल॑भते।
सोमो॒ वै रे॑तो॒धाः।
रेत॑ ए॒व तद्द॑धाति।
सौ॒म्यर्चा\-ऽभिषि॑ञ्चति।
रे॒तो॒धा ह्ये॑षा।
रेतः॒ सोमः॑।
रेत॑ ए॒वास्मि॑न्दधाति।
यत्किं च॑ राज॒सूय॑मृ॒ते सोमम्᳚।
तथ्सर्वं॑ भवति।
अषा॑ढं यु॒थ्सु पृत॑नासु॒ पप्रिम्᳚।
सु॒व॒र्॒षाम॒फ्स्वां वृ॒जन॑स्य गो॒पाम्।
भ॒रे॒षु॒जाꣳ सु॑क्षि॒तिꣳ सु॒श्रव॑सम्।
जय॑न्तं॒ त्वामनु॑ मदेम सोम॥१०॥\anuvakamend[रेतः॒ सोमः॑ स॒प्त च॑]

%2.7.5.1
यो वै सोमे॑न सू॒यते᳚।
स दे॑वस॒वः।
यः प॒शुना॑ सू॒यते᳚।
स दे॑वस॒वः।
य इष्ट्या॑ सू॒यते᳚।
स म॑नुष्यस॒वः।
ए॒तं वै पृथ॑ये दे॒वाः प्राय॑च्छन्।
ततो॒ वै सोऽप्या॑र॒ण्यानां᳚ पशू॒नाम॑सूयत।
याव॑तीः॒ किय॑तीश्च प्र॒जा वाचं॒ वद॑न्ति।
तासा॒ꣳ॒ सर्वा॑साꣳ सूयते॥११॥

%2.7.5.2
य ए॒तेन॒ यज॑ते।
य उ॑ चैनमे॒वं वेद॑।
ना॒रा॒श॒ꣴ॒स्यर्चा\-ऽभिषि॑ञ्चति।
म॒नु॒ष्या॑ वै नरा॒शꣳसः॑।
नि॒ह्नुत्य॒ वावैतत्।
अथा॒भिषि॑ञ्चति।
यत्किं च॑ राज॒सूय॑मनुत्तरवे॒दीकम्᳚।
तथ्सर्वं॑ भवति।
ये मे॑ पञ्चा॒शतं॑ द॒दुः।
अश्वा॑नाꣳ स॒धस्तु॑तिः।
द्यु॒मद॑ग्ने॒ महि॒ श्रवः॑।
बृ॒हत्कृ॑धि म॒घोना᳚म्।
नृ॒वद॑मृत नृ॒णाम्॥१२॥\anuvakamend[सू॒य॒ते॒ स॒धस्तु॑ति॒स्त्रीणि॑ च]

%2.7.6.1
ए॒ष गो॑स॒वः।
ष॒ट्त्रि॒ꣳ॒श उ॒क्थ्यो॑ बृ॒हथ्सा॑मा।
पव॑माने कण्वरथन्त॒रं भ॑वति।
यो वै वा॑ज॒पेयः॑।
स स॑म्राट्थ्स॒वः।
यो रा॑ज॒सूयः॑।
स व॑रुणस॒वः।
प्र॒जा\-प॑तिः॒ स्वारा᳚ज्यं परमे॒ष्ठी।
स्वारा᳚ज्यं॒ गौरे॒व।
गौरि॑व भवति॥१३॥

%2.7.6.2
य ए॒तेन॒ यज॑ते।
य उ॑ चैनमे॒वं वेद॑।
उ॒भे बृ॑हद्रथन्त॒रे भ॑वतः।
तद्धि स्वारा᳚ज्यम्।
अ॒युतं॒ दक्षि॑णाः।
तद्धि स्वारा᳚ज्यम्।
प्र॒ति॒धुषा॒\-ऽभिषि॑ञ्चति।
तद्धि स्वारा᳚ज्यम्।
अनु॑द्धते॒ वेद्यै॑ दक्षिण॒त आ॑\-हव॒नी\-य॑स्य बृह॒तः स्तो॒त्रं प्रत्य॒भिषि॑ञ्चति।
इ॒यं वाव र॑थन्त॒रम्॥१४॥

%2.7.6.3
अ॒सौ बृ॒हत्।
अ॒नयो॑रे॒वैन॒मन॑न्तर्\mbox{}हितम॒भिषि॑ञ्चति।
प॒शु॒स्तो॒मो वा ए॒षः।
तेन॑ गोस॒वः।
ष॒ट्त्रि॒ꣳ॒शः सर्वः॑।
रे॒वज्जा॒तः सह॑सा वृ॒द्धः।
क्ष॒त्राणां᳚ क्षत्र॒भृत्त॑मो वयो॒धाः।
म॒हान्म॑हि॒त्वे त॑स्तभा॒नः।
क्ष॒त्रे रा॒ष्ट्रे च॑ जागृहि।
प्र॒जा\-प॑तेस्त्वा परमे॒ष्ठिनः॒ स्वारा᳚ज्येना॒भिषि॑ञ्चा॒मीत्या॑ह।
स्वारा᳚ज्यमे॒वैनं॑ गमयति॥१५॥\anuvakamend[इ॒व॒ भ॒व॒ति॒ र॒थ॒न्त॒रमा॒हैकं॑ च]

%2.7.7.1
सि॒ꣳ॒हे व्या॒घ्र उ॒त या पृदा॑कौ।
त्विषि॑र॒ग्नौ ब्रा᳚ह्म॒णे सूर्ये॒ या।
इन्द्रं॒ या दे॒वी सु॒भगा॑ ज॒जान॑।
सा न॒ आग॒न्वर्च॑सा संविदा॒ना।
या रा॑ज॒न्ये॑ दुन्दु॒भावाय॑तायाम्।
अश्व॑स्य॒ क्रन्द्ये॒ पुरु॑षस्य मा॒यौ।
इन्द्रं॒ या दे॒वी सु॒भगा॑ ज॒जान॑।
सा न॒ आग॒न्वर्च॑सा संविदा॒ना।
या ह॒स्तिनि॑ द्वी॒पिनि॒ या हिर॑ण्ये।
त्विषि॒रश्वे॑षु॒ पुरु॑षेषु॒ गोषु॑॥१६॥

%2.7.7.2
इन्द्रं॒ या दे॒वी सु॒भगा॑ ज॒जान॑।
सा न॒ आग॒न्वर्च॑सा संविदा॒ना।
रथे॑ अ॒क्षेषु॑ वृष॒भस्य॒ वाजे᳚।
वाते॑ प॒र्जन्ये॒ वरु॑णस्य॒ शुष्मे᳚।
इन्द्रं॒ या दे॒वी सु॒भगा॑ ज॒जान॑।
सा न॒ आग॒न्वर्च॑सा संविदा॒ना।
राड॑सि वि॒राड॑सि।
स॒म्राड॑सि स्व॒राड॑सि।
इन्द्रा॑य त्वा॒ तेज॑स्वते॒ तेज॑स्वन्तꣴ श्रीणामि।
इन्द्रा॑य॒ त्वौज॑स्वत॒ ओज॑स्वन्तꣴ श्रीणामि॥१७॥

%2.7.7.3
इन्द्रा॑य त्वा॒ पय॑स्वते॒ पय॑स्वन्तꣴ श्रीणामि।
इन्द्रा॑य॒ त्वा\-ऽऽयु॑ष्मत॒ आयु॑ष्मन्तꣴ श्रीणामि।
तेजो॑ऽसि।
तत्ते॒ प्र य॑च्छामि।
तेज॑स्वदस्तु मे॒ मुखम्᳚।
तेज॑स्व॒च्छिरो॑ अस्तु मे।
तेज॑स्वान् वि॒श्वतः॑ प्र॒त्यङ्ङ्।
तेज॑सा॒ सम्पि॑पृग्धि मा।
ओजो॑ऽसि।
तत्ते॒ प्र य॑च्छामि॥१८॥

%2.7.7.4
ओज॑स्वदस्तु मे॒ मुखम्᳚।
ओज॑स्व॒च्छिरो॑ अस्तु मे।
ओज॑स्वान् वि॒श्वतः॑ प्र॒त्यङ्ङ्।
ओज॑सा॒ सं पि॑पृग्धि मा।
पयो॑ऽसि।
तत्ते॒ प्र य॑च्छामि।
पय॑स्वदस्तु मे॒ मुखम्᳚।
पय॑स्व॒च्छिरो॑ अस्तु मे।
पय॑स्वान् वि॒श्वतः॑ प्र॒त्यङ्ङ्।
पय॑सा॒ सं पि॑पृग्धि मा॥१९॥

%2.7.7.5
आयु॑रसि।
तत्ते॒ प्र य॑च्छामि।
आयु॑ष्मदस्तु मे॒ मुखम्᳚।
आयु॑ष्म॒च्छिरो॑ अस्तु मे।
आयु॑ष्मान् वि॒श्वतः॑ प्र॒त्यङ्ङ्।
आयु॑षा॒ सं पि॑पृग्धि मा।
इ॒मम॑ग्न॒ आयु॑षे॒ वर्च॑से कृधि।
प्रि॒यꣳ रेतो॑ वरुण सोम राजन्।
मा॒तेवा᳚स्मा अदिते॒ शर्म॑ यच्छ।
विश्वे॑ देवा॒ जर॑दष्टि॒र्यथा\-ऽस॑त्॥२०॥

%2.7.7.6
आयु॑रसि वि॒श्वायु॑रसि।
स॒र्वायु॑रसि॒ सर्व॒मायु॑रसि।
यतो॒ वातो॒ मनो॑जवाः।
यतः॒ क्षर॑न्ति॒ सिन्ध॑वः।
तासां᳚ त्वा॒ सर्वा॑साꣳ रु॒चा।
अ॒भिषि॑ञ्चामि॒ वर्च॑सा।
स॒मु॒द्र इ॑वासि ग॒ह्मना᳚।
सोम॑ इवा॒स्यदा᳚भ्यः।
अ॒ग्निरि॑व वि॒श्वतः॑ प्र॒त्यङ्ङ्।
सूर्य॑ इव॒ ज्योति॑षा वि॒भूः॥२१॥

%2.7.7.7
अ॒पां यो द्रव॑णे॒ रसः॑।
तम॒हम॒स्मा आ॑मुष्याय॒णाय॑।
तेज॑से ब्रह्म\-वर्च॒साय॑ गृह्णामि।
अ॒पां य ऊ॒र्मौ रसः॑।
तम॒हम॒स्मा आ॑\-मुष्या\-य॒णाय॑।
ओज॑से वी॒र्या॑य गृह्णामि।
अ॒पां यो म॑ध्य॒तो रसः॑।
तम॒\-ह\-म॒स्मा आ॑मुष्याय॒णाय॑।
पुष्ट्यै᳚ प्र॒जन॑नाय गृह्णामि।
अ॒पां यो य॒ज्ञियो॒ रसः॑।
तम॒हम॒स्मा आ॑मुष्याय॒णाय॑।
आयु॑षे दीर्घायु॒त्वाय॑ गृह्णामि॥२२॥\anuvakamend[गोष्वोज॑स्वन्तꣴ श्रीणा॒म्योजो॑ऽसि॒ तत्ते॒ प्रय॑च्छामि॒ पय॑सा॒ सम्पि॑पृग्धि॒ माऽस॑द्वि॒भूर्य॒ज्ञियो॒ रसो॒ द्वे च॑]

%2.7.8.1
अ॒भिप्रेहि॑ वी॒रय॑स्व।
उ॒ग्रश्चेत्ता॑ सपत्न॒हा।
आति॑ष्ठ मित्र॒वर्ध॑नः।
तुभ्यं॑ दे॒वा अधि॑ब्रवन्।
अ॒ङ्कौ न्य॒ङ्काव॒भित॒ आति॑ष्ठ वृत्रह॒न्रथम्᳚।
आ॒तिष्ठ॑न्तं॒ परि॒ विश्वे॑ अभूषन्।
श्रियं॒ वसा॑नश्चरति॒ स्वरो॑चाः।
म॒हत्तद॒स्यासु॑रस्य॒ नाम॑।
आ वि॒श्वरू॑पो अ॒मृता॑नि तस्थौ।
अनु॒ त्वेन्द्रो॑ मद॒त्वनु॒ बृह॒स्पतिः॑॥२३॥

%2.7.8.2
अनु॒ सोमो॒ अन्व॒ग्निरा॑वीत्।
अनु॑ त्वा॒ विश्वे॑ दे॒वा अ॑वन्तु।
अनु॑ स॒प्त राजा॑नो॒ य उ॒ताभिषि॑क्ताः।
अनु॑ त्वा मि॒त्रावरु॑णावि॒हाव॑तम्।
अनु॒ द्यावा॑पृथि॒वी वि॒श्वश॑म्भू।
सूर्यो॒ अहो॑भि॒रनु॑\- त्वाऽवतु।
च॒न्द्रमा॒ नक्ष॑त्रै॒रनु॑\- त्वाऽवतु।
द्यौश्च॑ त्वा पृथि॒वी च॒ प्रचे॑तसा।
शु॒क्रो बृ॒हद्दक्षि॑णा त्वा पिपर्तु।
अनु॑ स्व॒धा चि॑किता॒ꣳ॒ सोमो॑ अ॒ग्निः।
आऽयं पृ॑णक्तु॒ रज॑सी उ॒पस्थम्᳚॥२४॥\anuvakamend[बृह॒स्पतिः॒ सोमो॑ अ॒ग्निरेकं॑ च]

%2.7.9.1
प्र॒जा\-प॑तिः प्र॒जा अ॑\-सृजत।
ता अ॑स्माथ्सृ॒ष्टाः परा॑चीरायन्।
स ए॒तं प्र॒जा\-प॑तिरोद॒नम॑पश्यत्।
सोऽन्नं॑ भू॒तो॑\-ऽतिष्ठत्।
ता अ॒न्यत्रा॒न्नाद्य॒मवि॑त्वा।
प्र॒जा\-प॑तिं प्र॒जा उ॒पाव॑र्तन्त।
अन्न॑मे॒वैनं॑ भू॒तं पश्य॑न्तीः प्र॒जा उ॒पाव॑र्तन्ते।
य ए॒तेन॒ यज॑ते।
य उ॑ चैनमे॒वं वेद॑।
सर्वा॒ण्यन्ना॑नि भवन्ति॥२५॥

%2.7.9.2
सर्वे॒ पुरु॑षाः।
सर्वा᳚ण्ये॒वान्ना॒न्यव॑ रुन्धे।
सर्वा॒न्पुरु॑षान्।
राड॑सि वि॒राड॒सीत्या॑ह।
स्वारा᳚ज्यमे॒वैनं॑ गमयति।
यद्धिर॑ण्यं॒ ददा॑ति।
तेज॒स्तेनाव॑ रुन्धे।
यत्ति॑सृध॒न्वम्।
वी॒र्यं॑ तेन॑।
यदष्ट्रा᳚म्॥२६॥

%2.7.9.3
पुष्टिं॒ तेन॑।
यत्क॑म॒ण्डलुम्᳚।
आयु॒ष्टेन॑।
यद्धिर॑ण्यमा ब॒ध्नाति॑।
ज्योति॒र्वै हिर॑ण्यम्।
ज्योति॑रे॒वास्मि॑न्दधाति।
अथो॒ तेजो॒ वै हिर॑ण्यम्।
तेज॑ ए॒वाऽऽत्मन्ध॑त्ते।
यदो॑द॒नं प्रा॒श्ञाति॑।
ए॒तदे॒व सर्व॑मव॒रुध्य॑॥२७॥

%2.7.9.4
तद॑स्मिन्नेक॒धा\-ऽधा᳚त्।
रो॒हि॒ण्यां का॒र्यः॑।
यद्ब्रा᳚ह्म॒ण ए॒व रो॑हि॒णी।
तस्मा॑दे॒व।
अथो॒ वर्ष्मै॒वैनꣳ॑ समा॒नानां᳚ करोति।
उ॒द्य॒ता सूर्ये॑ण का॒र्यः॑।
उ॒द्यन्तं॒ वा ए॒तꣳ सर्वाः᳚ प्र॒जाः प्रति॑\-नन्दन्ति।
दि॒दृ॒क्षेण्यो॑ दर्\mbox{}श॒नीयो॑ भवति।
य ए॒वं वेद॑।
ब्र॒ह्म॒वा॒दिनो॑ वदन्ति॥२८॥

%2.7.9.5
अ॒वेत्यो॑\-ऽवभृ॒था (३) ना (३) इति॑।
यद्द॑र्भपुञ्जी॒लैः प॒वय॑ति।
तथ्स्वि॑दे॒वावै॑ति।
तन्नावै॑ति।
त्रि॒भिः प॑वयति।
त्रय॑ इ॒मे लो॒काः।
ए॒भिरे॒वैनं॑ लो॒कैः प॑वयति।
अथो॑ अ॒पां वा ए॒तत्तेजो॒ वर्चः॑।
यद्द॒र्भाः।
यद्द॑र्भपुञ्जी॒लैः प॒वय॑ति।
अ॒पामे॒वैनं॒ तेज॑सा॒ वर्च॑सा॒\-ऽभिषि॑ञ्चति॥२९॥\anuvakamend[भ॒व॒न्त्यष्ट्रा॑मव॒रुध्य॑ वदन्ति द॒र्भा यद्द॑र्भपुञ्जी॒लैः प॒वय॒त्येकं॑ च]

%2.7.10.1
प्र॒जा\-प॑तिरकामयत ब॒होर्भूया᳚न्थ्स्या॒मिति॑।
स ए॒तं प॑ञ्चशार॒दीय॑मपश्यत्।
तमाऽह॑रत्।
तेना॑यजत।
ततो॒ वै स ब॒होर्भूया॑नभवत्।
यः का॒मये॑त ब॒होर्भूया᳚न्थ्स्या॒मिति॑।
स प॑ञ्चशार॒दीये॑न यजेत।
ब॒होरे॒व भूया᳚न्भवति।
म॒रु॒थ्स्तो॒मो वा ए॒षः।
म॒रुतो॒ हि दे॒वानां॒ भूयि॑ष्ठाः॥३०॥

%2.7.10.2
ब॒हुर्भ॑वति।
य ए॒तेन॒ यज॑ते।
य उ॑चैनमे॒वं वेद॑।
प॒ञ्च॒शा॒र॒दीयो॑ भवति।
पञ्च॒ वा ऋ॒तवः॑ संवथ्स॒रः।
ऋ॒तुष्वे॒व सं॑वथ्स॒रे प्रति॑ तिष्ठति।
अथो॒ पञ्चा᳚क्षरा प॒ङ्क्तिः।
पाङ्क्तो॑ य॒ज्ञः।
य॒ज्ञमे॒वाव॑ रुन्धे।
स॒प्त॒द॒शꣴ स्तोमा॒ नाति॑ यन्ति।
स॒प्त॒द॒शः प्र॒जा\-प॑तिः।
प्र॒जा\-प॑ते॒राप्त्यै᳚॥३१॥\anuvakamend[भूयि॑ष्ठा यन्ति॒ द्वे च॑]

%2.7.11.1
अ॒गस्त्यो॑ म॒रुद्भ्य॑ उ॒क्ष्णः प्रौक्ष॑त्।
तानिन्द्र॒ आद॑त्त।
त ए॑नं॒ वज्र॑मु॒द्यत्या॒भ्या॑यन्त।
तान॒गस्त्य॑श्चै॒वेन्द्र॑श्च कयाशु॒भीये॑नाशमयताम्।
ताञ्छा॒न्तानुपा᳚ह्वयत।
यत्क॑याशु॒भीयं॒ भव॑ति॒ शान्त्यै᳚।
तस्मा॑दे॒त ऐ᳚न्द्रामारु॒ता उ॒क्षाणः॑ सव॒नीया॑ भवन्ति।
त्रयः॑ प्रथ॒मे\-ऽह॒न्ना ल॑भ्यन्ते।
ए॒वं द्वि॒तीये᳚।
ए॒वं तृ॒तीये᳚॥३२॥

%2.7.11.2
ए॒वं च॑तु॒र्थे।
पञ्चो᳚त्त॒मे\-ऽह॒न्ना ल॑भ्यन्ते।
वर्\mbox{}षि॑ष्ठमिव॒ ह्ये॑तदहः॑।
वर्\mbox{}षि॑ष्ठः समा॒नानां᳚ भवति।
य ए॒तेन॒ यज॑ते।
य उ॑चैनमे॒वं वेद॑।
स्वारा᳚ज्यं॒ वा ए॒ष य॒ज्ञः।
ए॒तेन॒ वा एक॒या वा॑ कान्द॒मः स्वारा᳚ज्यमगच्छत्।
स्वारा᳚ज्यं गच्छति।
य ए॒तेन॒ यज॑ते॥३३॥

%2.7.11.3
य उ॑ चैनमे॒वं वेद॑।
मा॒रु॒तो वा ए॒षः स्तोमः॑।
ए॒तेन॒ वै म॒रुतो॑ दे॒वानां॒ भूयि॑ष्ठा अभवन्।
भूयि॑ष्ठः समा॒नानां᳚ भवति।
य ए॒तेन॒ यज॑ते।
य उ॑ चैनमे॒वं वेद॑।
प॒ञ्च॒शा॒र॒दीयो॒ वा ए॒ष य॒ज्ञः।
आ प॑ञ्च॒मात्पुरु॑षा॒दन्न॑मत्ति।
य ए॒तेन॒ यज॑ते।
य उ॑ चैनमे॒वं वेद॑।
स॒प्त॒द॒शꣴ स्तोमा॒ नाति॑ यन्ति।
स॒प्त॒द॒शः प्र॒जा\-प॑तिः।
प्र॒जा\-प॑तेरे॒व नैति॑॥३४॥\anuvakamend[तृ॒तीये॑ गच्छति॒ य ए॒तेन॒ यज॑ते\-ऽत्ति॒ य ए॒तेन॒ यज॑ते॒ य उ॑ चैनमे॒वं वेद॒ त्रीणि॑ च (अ॒गस्त्यः॒ स्वारा᳚ज्यं मारु॒तः प॑ञ्चशार॒दीयो॒ वा ए॒ष य॒ज्ञः स॑प्तद॒शं प्र॒जा\-प॑तेरे॒व नैति॑॥)]

%2.7.12.1
अ॒स्या जरा॑सो द॒मा म॒रित्राः᳚।
अ॒र्चद्धू॑मासो अ॒ग्नयः॑ पाव॒काः।
श्वि॒ची॒चयः॑ श्वा॒त्रासो॑ भुर॒ण्यवः॑।
व॒न॒र्॒षदो॑ वा॒यवो॒ न सोमाः᳚।
यजा॑ नो मि॒त्रावरु॑णा।
यजा॑ दे॒वाꣳ ऋ॒तं बृ॒हत्।
अग्ने॒ यक्षि॒ स्वन्दमम्᳚।
अश्वि॑ना॒ पिब॑तꣳ सु॒तम्।
दीद्य॑ग्नी शुचिव्रता।
ऋ॒तुना॑ यज्ञवाहसा॥३५॥

%2.7.12.2
द्वे विरू॑पे चरतः॒ स्वर्थे᳚।
अ॒न्या\-ऽन्या॑ व॒थ्समुप॑ धापयेते।
हरि॑र॒न्यस्यां॒ भव॑ति स्व॒धावान्॑।
शु॒क्रो अ॒न्यस्यां᳚ ददृशे सु॒वर्चाः᳚।
पू॒र्वा॒प॒रं च॑रतो मा॒ययै॒तौ।
शिशू॒ क्रीड॑न्तौ॒ परि॑ यातो अध्व॒रम्।
विश्वा᳚न्य॒न्यो भुव॑नाऽभि॒ चष्टे᳚।
ऋ॒तून॒न्यो वि॒दध॑ज्जायते॒ पुनः॑।
त्रीणि॑ श॒ता त्रीष॒हस्रा᳚ण्य॒ग्निम्।
त्रि॒ꣳ॒शच्च॑ दे॒वा नव॑ चाऽसपर्यन्॥३६॥

%2.7.12.3
औक्षं॑ घृ॒तैरास्तृ॑णन्ब॒र्॒हिर॑स्मै।
आदिद्धोता॑रं॒ न्य॑षादयन्त।
अ॒ग्निना॒\-ऽग्निः समि॑ध्यते।
क॒विर्गृ॒हप॑ति॒र्युवा᳚।
ह॒व्य॒वाड्जु॒ह्वा᳚ऽऽस्यः।
अ॒ग्निर्दे॒वानां᳚ ज॒ठरम्᳚।
पू॒तद॑क्षः क॒विक्र॑तुः।
दे॒वो दे॒वेभि॒रा ग॑मत्।
अ॒ग्नि॒श्रियो॑ म॒रुतो॑ वि॒श्वकृ॑ष्टयः।
आ त्वे॒षमु॒ग्रमव॑ ईमहे व॒यम्॥३७॥

%2.7.12.4
ते स्वा॒निनो॑ रु॒द्रिया॑ व॒र्॒षनि॑र्णिजः।
सि॒ꣳ॒हा न हे॒षक्र॑तवः सु॒दान॑वः।
यदु॑त्त॒मे म॑रुतो मध्य॒मे वा᳚।
यद्वा॑\-ऽव॒मे सु॑भगासो दि॒वि ष्ठ।
ततो॑ नो रुद्रा उ॒त वा॒\-ऽन्वस्य॑।
अग्ने॑ वि॒त्ताद्ध॒विषो॒ यद्यजा॑मः।
ईडे॑ अ॒ग्निꣴ स्वव॑स॒न्नमो॑भिः।
इ॒ह प्र॑स॒प्तो वि च॑ यत्कृ॒तं नः॑।
रथै॑रिव॒ प्रभ॑रे वाज॒यद्भिः॑।
प्र॒द॒क्षि॒णिन्म॒रुता॒ꣴ॒ स्तोम॑मृद्ध्याम्॥३८॥

%2.7.12.5
श्रु॒धि श्रु॑त्कर्ण॒ वह्नि॑भिः।
दे॒वैर॑ग्ने स॒याव॑भिः।
आसी॑दन्तु ब॒र्॒हिषि॑।
मि॒त्रो वरु॑णो अर्य॒मा।
प्रा॒त॒र्यावा॑णो अध्व॒रम्।
विश्वे॑षा॒मदि॑तिर्य॒ज्ञिया॑नाम्।
विश्वे॑षा॒मति॑थि॒र्मानु॑षाणाम्।
अ॒ग्निर्दे॒वाना॒\-मव॑ आवृणा॒नः।
सु॒मृ॒डी॒को भ॑वतु वि॒श्ववे॑दाः।
त्वे अ॑ग्ने सुम॒तिं भिक्ष॑माणाः॥३९॥

%2.7.12.6
दि॒वि श्रवो॑ दधिरे य॒ज्ञिया॑सः।
नक्ता॑ च च॒क्रुरु॒षसा॒ विरू॑पे।
कृ॒ष्णं च॒ वर्ण॑मरु॒णं च॒ सन्धुः॑।
त्वाम॑ग्न आदि॒त्यास॑ आ॒स्यम्᳚।
त्वां जि॒ह्वाꣳ शुच॑यश्चक्रिरे कवे।
त्वाꣳ रा॑ति॒षाचो॑ अध्व॒रेषु॑ सश्चिरे।
त्वे दे॒वा ह॒विर॑द॒न्त्याहु॑तम्।
नि त्वा॑ य॒ज्ञस्य॒ साध॑नम्।
अग्ने॒ होता॑रमृ॒त्विजम्᳚।
व॒नु॒ष्वद्दे॑व धीमहि॒ प्रचे॑तसम्।
जी॒रं दू॒तमम॑र्त्यम्॥४०॥\anuvakamend[य॒ज्ञ॒वा॒ह॒सा॒स॒प॒र्य॒न्व॒यमृ॑द्ध्यां॒ भिक्ष॑माणाः॒ प्रचे॑तस॒मेकं॑ च]

%2.7.13.1
तिष्ठा॒ हरी॒ रथ॒ आ यु॒ज्यमा॑ना या॒हि।
वा॒युर्न नि॒युतो॑ नो॒ अच्छ॑।
पिबा॒स्यन्धो॑ अ॒भिसृ॑ष्टो अ॒स्मे।
इन्द्रः॒ स्वाहा॑ ररि॒मा ते॒ मदा॑य।
कस्य॒ वृषा॑ सु॒ते सचा᳚।
नि॒युत्वा᳚न्वृष॒भो र॑णत्।
वृ॒त्र॒हा सोम॑पीतये।
इन्द्रं॑ व॒यं म॑हाध॒ने।
इन्द्र॒मर्भे॑ हवामहे।
युजं॑ वृ॒त्रेषु॑ व॒ज्रिणम्᳚॥४१॥

%2.7.13.2
द्वि॒ता यो वृ॑त्र॒हन्त॑मः।
वि॒द इन्द्रः॑ श॒तक्र॑तुः।
उप॑ नो॒ हरि॑भिः सु॒तम्।
स सूर॒ आज॒नयं॒ ज्योति॒रिन्द्रम्᳚।
अ॒या धि॒या त॒रणि॒रद्रि॑बर्\mbox{}हाः।
ऋ॒तेन॑ शु॒ष्मी नव॑मानो अ॒र्कैः।
व्यु॑स्रिधो॑ अ॒स्रो अद्रि॑र्बिभेद।
उ॒तत्यदा॒श्वश्वि॑यम्।
यदि॑न्द्र॒ नाहु॑षी॒ष्वा।
अग्रे॑ वि॒क्षु प्रतीद॑यत्॥४२॥

%2.7.13.3
भरे॒ष्विन्द्रꣳ॑ सु॒हवꣳ॑ हवामहे।
अ॒ꣳ॒हो॒मुचꣳ॑ सु॒कृतं॒ दैव्यं॒ जनम्᳚।
अ॒ग्निं मि॒त्रं वरु॑णꣳ सा॒तये॒ भगम्᳚।
द्यावा॑पृथि॒वी म॒रुतः॑ स्व॒स्तये᳚।
म॒हि क्षेत्रं॑ पु॒रुश्च॒न्द्रं वि वि॒द्वान्।
आदिथ्सखि॑भ्यश्च॒रथ॒ꣳ॒ समै॑रत्।
इन्द्रो॒ नृभि॑रजन॒द्दीद्या॑नः सा॒कम्।
सूर्य॑मु॒षसं॑ गा॒तुम॒ग्निम्।
उ॒रुं नो॑ लो॒कमनु॑ नेषि वि॒द्वान्।
सुव॑र्व॒ज्ज्योति॒रभ॑यꣴ स्व॒स्ति॥४३॥

%2.7.13.4
ऋ॒ष्वा त॑ इन्द्र॒ स्थवि॑रस्य बा॒हू।
उप॑स्थेयाम शर॒णा बृ॒हन्ता᳚।
आ नो॒ विश्वा॑भिरू॒तिभिः॑ स॒जोषाः᳚।
ब्रह्म॑ जुषा॒णो ह॑र्यश्व याहि।
वरी॑वृज॒थ्स्थवि॑रेभिः सुशिप्र।
अ॒स्मे दध॒द्वृष॑ण॒ꣳ॒ शुष्म॑मिन्द्र।
इन्द्रा॑य॒ गाव॑ आ॒शिरम्᳚।
दु॒दु॒ह्रे व॒ज्रिणे॒ मधु॑।
यथ्सी॑मुपह्व॒रे वि॒दत्।
तास्ते॑ वज्रिन्धे॒नवो॑ जोजयुर्नः॥४४॥

%2.7.13.5
गभ॑स्तयो नि॒युतो॑ वि॒श्ववा॑राः।
अह॑रह॒र्भूय॒ इज्जोगु॑वानाः।
पू॒र्णा इ॑न्द्र क्षु॒मतो॒ भोज॑नस्य।
इ॒मां ते॒ धियं॒ प्र भ॑रे म॒हो म॒हीम्।
अ॒स्य स्तो॒त्रे धि॒षणा॒ यत्त॑ आन॒जे।
तमु॑थ्स॒वे च॑ प्रस॒वे च॑ सास॒हिम्।
इन्द्रं॑ दे॒वासः॒ शव॑सा मदं॒ ननु॑॥४५॥\anuvakamend[व॒ज्रिण॑मयथ्स्व॒स्ति जो॑जयुर्नः स॒प्त च॑]

%2.7.14.1
प्र॒जा\-प॑तिः प॒शून॑\-सृजत।
ते᳚ऽस्माथ्सृ॒ष्टाः परां᳚ च आयन्।
तान॑ग्निष्टो॒मेन॒ नाऽऽप्नो᳚त्।
तानु॒क्थ्ये॑न॒ नाऽऽप्नो᳚त्।
तान्थ्षो॑ड॒शिना॒ नाऽऽप्नो᳚त्।
तान्रात्रि॑या॒ नाऽऽप्नो᳚त्।
तान्थ्स॒न्धिना॒ नाऽऽप्नो᳚त्।
सो᳚ऽग्निम॑ब्रवीत्।
इ॒मान्म॑ ई॒फ्सेति॑।
तान॒ग्निस्त्रि॒वृता॒ स्तोमे॑न॒ नाऽऽप्नो᳚त्॥४६॥

%2.7.14.2
स इन्द्र॑मब्रवीत्।
इ॒मान्म॑ ई॒फ्सेति॑।
तानिन्द्रः॑ पञ्चद॒शेन॒ स्तोमे॑न॒ नाऽऽप्नो᳚त्।
स विश्वा᳚न्दे॒वान॑ब्रवीत्।
इ॒मान्म॑ ईफ्स॒तेति॑।
तान् विश्वे॑दे॒वाः स॑प्तद॒शेन॒ स्तोमे॑न॒ नाऽऽप्नु॑वन्।
स विष्णु॑मब्रवीत्।
इ॒मान्म॑ ई॒फ्सेति॑।
तान् विष्णु॑रेकवि॒ꣳ॒शेन॒ स्तोमे॑नाऽऽप्नोत्।
वा॒र॒व॒न्तीये॑नावारयत॥४७॥

%2.7.14.3
इ॒दं विष्णु॒र्वि च॑क्रम॒ इति॒ व्य॑क्रमत।
यस्मा᳚त्प॒शवः॒ प्रप्रेव॒ भ्रꣳशे॑रन्।
स ए॒तेन॑ यजेत।
यदाप्नो᳚त्।
तद॒प्तोर्याम॑स्याप्तोर्याम॒\-त्वम्।
ए॒तेन॒ वै दे॒वा जैत्वा॑नि जि॒त्वा।
यं काम॒मका॑मयन्त॒ तमा᳚ऽऽप्नुवन्।
यं कामं॑ का॒मय॑ते।
तमे॒तेना᳚ऽऽप्नोति॥४८॥\anuvakamend[स्तोमे॑न॒ नाऽऽप्नो॑दवारयत॒ नव॑ च]

%2.7.15.1
व्या॒घ्रो॑\-ऽयम॒ग्नौ च॑रति॒ प्रवि॑ष्टः।
ऋषी॑णां पु॒त्रो अ॑भिशस्ति॒पा अ॒यम्।
न॒म॒स्का॒रेण॒ नम॑सा ते जुहोमि।
मा दे॒वानां᳚ मिथु॒याक॑र्म भा॒गम्।
सावी॒र्॒हि दे॑व प्रस॒वाय॑ पित्रे।
व॒र्ष्माण॑मस्मै वरि॒माण॑मस्मै।
अथा॒स्मभ्यꣳ॑ सवितः स॒र्वता॑ता।
दि॒वेदि॑व॒ आ सु॑वा॒ भूरि॑ प॒श्वः।
भू॒तो भू॒तेषु॑ चरति॒ प्रवि॑ष्टः।
स भू॒ताना॒मधि॑पतिर्बभूव॥४९॥

%2.7.15.2
तस्य॑ मृ॒त्यौ च॑रति राज॒सूयम्᳚।
स राजा॑ रा॒ज्यमनु॑ मन्यतामि॒दम्।
येभिः॒ शिल्पैः᳚ पप्रथा॒नामदृꣳ॑हत्।
येभि॒र्द्याम॒भ्यपिꣳ॑\-शत्प्र॒जा\-प॑तिः।
येभि॒र्वाचं॑ वि॒श्वरू॑पाꣳ स॒मव्य॑यत्।
तेने॒मम॑ग्न इ॒ह वर्च॑सा॒ सम॑ङ्ग्धि।
येभि॑रादि॒त्यस्तप॑ति॒ प्र के॒तुभिः॑।
येभिः॒ सूर्यो॑ ददृ॒शे चि॒त्रभा॑नुः।
येभि॒र्वाचं॑ पुष्क॒लेभि॒रव्य॑यत्।
तेने॒मम॑ग्न इ॒ह वर्च॑सा॒ सम॑ङ्ग्धि॥५०॥

%2.7.15.3
आऽयं भा॑तु॒ शव॑सा॒ पञ्च॑ कृ॒ष्टीः।
इन्द्र॑ इव ज्ये॒ष्ठो भ॑वतु प्र॒जावान्॑।
अ॒स्मा अ॑स्तु पुष्क॒लं चि॒त्रभा॑नु।
आऽयं पृ॑णक्तु॒ रज॑सी उ॒पस्थम्᳚।
यत्ते॒ शिल्पं॑ कश्यप रोच॒नाव॑त्।
इ॒न्द्रि॒याव॑त्पुष्क॒लं चि॒त्रभा॑नु।
यस्मि॒न्थ्सूर्या॒ अर्पि॑ताः स॒प्त सा॒कम्।
तस्मि॒न्राजा॑न॒मधि॒ विश्र॑ये॒मम्।
द्यौर॑सि पृथि॒व्य॑सि।
व्या॒घ्रो वैया॒घ्रे\-ऽधि॑॥५१॥

%2.7.15.4
विश्र॑यस्व॒ दिशो॑ म॒हीः।
विश॑स्त्वा॒ सर्वा॑ वाञ्छन्तु।
मा त्वद्रा॒ष्ट्रमधि॑ भ्रशत्।
या दि॒व्या आपः॒ पय॑सा सम्बभू॒वुः।
या अ॒न्तरि॑क्ष उ॒त पार्थि॑वी॒र्याः।
तासां᳚ त्वा॒ सर्वा॑साꣳ रु॒चा।
अ॒भिषि॑ञ्चामि॒ वर्च॑सा।
अ॒भि त्वा॒ वर्च॑सा\-ऽसिचं दि॒व्येन॑।
पय॑सा स॒ह।
यथासा॑ राष्ट्र॒वर्ध॑नः॥५२॥

%2.7.15.5
तथा᳚ त्वा सवि॒ता क॑रत्।
इन्द्रं॒ विश्वा॑ अवीवृधन्।
स॒मु॒द्रव्य॑चस॒ङ्गिरः॑।
र॒थीत॑मꣳ रथी॒नाम्।
वाजा॑ना॒ꣳ॒ सत्प॑तिं॒ पतिम्᳚।
वस॑वस्त्वा पु॒रस्ता॑द॒भिषि॑ञ्चन्तु गाय॒त्रेण॒ छन्द॑सा।
रु॒द्रास्त्वा॑ दक्षिण॒तो॑\-ऽभिषि॑ञ्चन्तु॒ त्रैष्टु॑भेन॒ छन्द॑सा।
आ॒दि॒त्यास्त्वा॑ प॒श्चाद॒भिषि॑ञ्चन्तु॒ जाग॑तेन॒ छन्द॑सा।
विश्वे᳚ त्वा दे॒वा उ॑त्तर॒तो॑\-ऽभिषि॑ञ्चं॒ त्वाऽनु॑ष्टुभेन॒ छन्द॑सा। 
बृह॒स्पति॑स्त्वो॒परि॑ष्टाद॒भिषि॑ञ्चतु॒ पाङ्क्ते॑न॒ छन्द॑सा॥५३॥

%2.7.15.6
अ॒रु॒णं त्वा॒ वृक॑मु॒ग्रङ्ख॑जङ्क॒रम्।
रोच॑मानं म॒रुता॒मग्रे॑ अ॒र्चिषः॑।
सूर्य॑वन्तं म॒घवा॑नं विषास॒हिम्।
इन्द्र॑मु॒क्थेषु॑ नाम॒हूत॑मꣳ हुवेम।
प्र बा॒हवा॑ सिसृतं जी॒वसे॑ नः।
आ नो॒ गव्यू॑तिमुक्षतं घृ॒तेन॑।
आ नो॒ जने᳚ श्रवयतं युवाना।
श्रु॒तं मे॑ मित्रावरुणा॒ हवे॒मा।
इन्द्र॑स्य ते वीर्य॒कृतः॑।
बा॒हू उ॒पाव॑ हरामि॥५४॥\anuvakamend[ब॒भू॒वाव्य॑य॒त्तेने॒मम॑ग्न इ॒ह वर्च॑सा॒ सम॑ङ्ग्धि॒ वैया॒घ्रेऽधि॑ राष्ट्र॒वर्ध॑नः॒ पाङ्क्ते॑न॒ छन्द॑सो॒पाव॑हरामि]

%2.7.16.1
अ॒भि प्रेहि॑ वी॒रय॑स्व।
उ॒ग्रश्चेत्ता॑ सपत्न॒हा।
आति॑ष्ठ वृत्र॒हन्त॑मः।
तुभ्यं॑ दे॒वा अधि॑ब्रवन्।
अ॒ङ्कौ न्य॒ङ्काव॒भितो॒ रथं॒ यौ।
ध्वा॒न्तं वा॑ता॒ग्रमनु॑ स॒ञ्चर॑न्तौ।
दू॒रेहे॑तिरिन्द्रि॒यावा᳚न्पत॒त्री।
ते नो॒\-ऽग्नयः॒ पप्र॑यः पारयन्तु।
नम॑स्त ऋषे गद।
अव्य॑थायै त्वा स्व॒धायै᳚ त्वा॥५५॥

%2.7.16.2
मा न॑ इन्द्रा॒भित॒स्त्वदृ॒ष्वारि॑ष्टासः।
ए॒वा ब्र॑ह्म॒न्तवेद॑स्तु।
तिष्ठा॒ रथे॒ अधि॒ यद्वज्र॑हस्तः।
आ र॒श्मीन्दे॑व युवसे॒ स्वश्वः॑।
आ ति॑ष्ठ वृत्रहन्ना॒तिष्ठ॑न्तं॒ परि॑।
अनु॒ त्वेन्द्रो॑ मद॒त्वनु॑ त्वा मि॒त्रावरु॑णौ।
द्यौश्च॑ त्वा पृथि॒वी च॒ प्रचे॑तसा।
शु॒क्रो बृ॒हद्दक्षि॑णा त्वा पिपर्तु।
अनु॑ स्व॒धा चि॑किता॒ꣳ॒ सोमो॑ अ॒ग्निः।
अनु॑ त्वाऽवतु सवि॒ता स॒वेन॑॥५६॥

%2.7.16.3
इन्द्रं॒ विश्वा॑ अवीवृधन्।
स॒मु॒द्रव्य॑चस॒ङ्गिरः॑।
र॒थीत॑मꣳ रथी॒नाम्।
वाजा॑ना॒ꣳ॒ सत्प॑तिं॒ पतिम्᳚।
परि॑मा से॒न्या घोषाः᳚।
ज्यानां᳚ वृञ्जन्तु गृ॒ध्नवः॑।
मे॒थि॒ष्ठाः पिन्व॑माना इ॒ह।
मां गोप॑तिम॒भि संवि॑शन्तु।
तन्मे\-ऽनु॑मति॒रनु॑\- मन्यताम्।
तन्मा॒ता पृ॑थि॒वी तत्पि॒ता द्यौः॥५७॥

%2.7.16.4
तद्ग्रावा॑णः सोम॒सुतो॑ मयो॒भुवः॑।
तद॑श्विना शृणुतꣳ सौभगा यु॒वम्।
अव॑ ते॒ हेड॒ उदु॑त्त॒मम्।
ए॒ना व्या॒घ्रं प॑रिषस्वजा॒नाः।
सि॒ꣳ॒हꣳ हि॑न्वन्ति मह॒ते सौभ॑गाय।
स॒मु॒द्रं न सु॒हुव॑न्तस्थि॒वाꣳसम्᳚।
म॒र्मृ॒ज्यन्ते᳚ द्वी॒पिन॑म॒फ्स्व॑न्तः।
उद॒सावे॑तु॒ सूर्यः॑।
उदि॒दं मा॑म॒कं वचः॑।
उदि॑हि देव सूर्य।
स॒ह व॒ग्नुना॒ मम॑।
अ॒हं वा॒चो वि॒वाच॑नम्।
मयि॒ वाग॑स्तु धर्ण॒सिः।
यन्तु॑ न॒दयो॒ वर्\mbox{}ष॑न्तु प॒र्जन्याः᳚।
सु॒पि॒प्प॒ला ओष॑धयो भवन्तु।
अन्न॑वतामोद॒नव॑तामा॒मिक्ष॑वताम्।
ए॒षाꣳ राजा॑ भूयासम्॥५८॥\anuvakamend[स्व॒धायै᳚ त्वा स॒वेन॒ द्यौः सू᳚र्य स॒प्त च॑]

%2.7.17.1
ये के॒शिनः॑ प्रथ॒माः स॒त्रमास॑त।
येभि॒राभृ॑तं॒ यदि॒दं वि॒रोच॑ते।
तेभ्यो॑ जुहोमि बहु॒धा घृ॒तेन॑।
रा॒यस्पोषे॑णे॒मं वर्च॑सा॒ सꣳ सृ॑जाथ।
नर्ते ब्रह्म॑ण॒स्तप॑सो विमो॒कः।
द्वि॒नाम्नी॑ दी॒क्षा व॒शिनी॒ ह्यु॑ग्रा।
प्र केशाः᳚ सु॒वते॑ का॒ण्डिनो॑ भवन्ति।
तेषां᳚ ब्र॒ह्मेदीशे॒ वप॑नस्य॒ नान्यः।
आ रो॑ह॒ प्रोष्ठं॒ विष॑हस्व॒ शत्रून्॑।
अवा᳚स्राग्दी॒क्षा व॒शिनी॒ ह्यु॑ग्रा॥५९॥

%2.7.17.2
दे॒हि दक्षि॑णां॒ प्रति॑\-र॒स्वायुः॑।
अथा॑मुच्यस्व॒ वरु॑णस्य॒ पाशा᳚त्।
येनाव॑पथ्सवि॒ता क्षु॒रेण॑।
सोम॑स्य॒ राज्ञो॒ वरु॑णस्य वि॒द्वान्।
तेन॑ ब्रह्माणो वपते॒दम॒स्योर्जेमम्।
र॒य्या वर्च॑सा॒ सꣳ सृ॑जाथ।
मा ते॒ केशा॒ननु॑ गा॒द्वर्च॑ ए॒तत्।
तथा॑ धा॒ता क॑रोतु ते।
तुभ्य॒मिन्द्रो॒ बृह॒स्पतिः॑।
स॒वि॒ता वर्च॒ आद॑धात्॥६०॥

%2.7.17.3
तेभ्यो॑ नि॒धानं॑ बहु॒धा व्यैच्छन्॑।
अ॒न्त॒रा द्यावा॑पृथि॒वी अ॒पः सुवः॑।
द॒र्भ॒स्त॒म्बे वी॒र्य॑कृते नि॒धाय॑।
पौꣴस्ये॑ने॒मं वर्च॑सा॒ सꣳ सृ॑जाथ।
बलं॑ ते बाहु॒वोः स॑वि॒ता द॑धातु।
सोम॑स्त्वा\-ऽनक्तु॒ पय॑सा घृ॒तेन॑।
स्त्री॒षु रू॒पम॑श्विनै॒तन्नि ध॑त्तम्।
पौꣴस्ये॑ने॒मं वर्च॑सा॒ सꣳसृ॑जाथ।
यथ्सी॒मन्तं॒ कङ्क॑तस्ते लि॒लेख॑।
यद्वा᳚ क्षु॒रः प॑रिव॒वर्ज॒ वपꣴ॑स्ते।
स्त्री॒षु रू॒पम॑श्विनै॒तन्नि ध॑त्तम्।
पौꣴस्ये॑ने॒मꣳ सꣳ सृ॑जाथो वी॒र्ये॑ण॥६१॥\anuvakamend[अवा᳚स्राग्दी॒क्षा व॒शिनी॒ ह्यु॑ग्रा\-ऽद॑धाद्व॒वर्ज॒ वपꣴ॑ स्ते॒ द्वे च॑]

%2.7.18.1
इन्द्रं॒ वै स्वाविशो॑ म॒रुतो॒ नापा॑चायन्।
सोऽन॑पचाय्यमान ए॒तं वि॑घ॒नम॑पश्यत्।
तमाऽह॑रत्।
तेना॑यजत।
तेनै॒वासा॒न्तꣳ सꣴ॑ स्त॒म्भं व्य॑हन्।
यद्व्यहन्॑।
तद्वि॑घ॒नस्य॑ विघन॒त्वम्।
वि पा॒प्मानं॒ भ्रातृ॑व्यꣳ हते।
य ए॒तेन॒ यज॑ते।
य उ॑ चैनमे॒वं वेद॑॥६२॥

%2.7.18.2
यꣳ राजा॑नं॒ विशो॒ नाप॒चाये॑युः।
यो वा᳚ ब्राह्म॒णस्तम॑सा पा॒प्मना॒ प्रावृ॑तः॒ स्यात्।
स ए॒तेन॑ यजेत।
वि॒घ॒नेनै॒वैन॑द्वि॒हत्य॑।
वि॒शामाधि॑पत्यं गच्छति।
तस्य॒ द्वे द्वा॑द॒शे स्तो॒त्रे भव॑तः।
द्वे च॑तुर्वि॒ꣳ॒शे।
औद्भि॑द्यमे॒व तत्।
ए॒तद्वै क्ष॒त्रस्यौद्भि॑द्यम्।
यद॑स्मै॒ स्वाविशो॑ ब॒लिꣳ हर॑न्ति॥६३॥

%2.7.18.3
हर॑न्त्यस्मै॒ विशो॑ ब॒लिम्।
ऐन॒मप्र॑तिख्यातं गच्छति।
य ए॒वं वेद॑।
प्र॒बाहु॒ग्वा अग्रे᳚ क्ष॒त्राण्याते॑पुः।
तेषा॒मिन्द्रः॑ क्ष॒त्राण्याद॑त्त।
न वा इ॒मानि॑ क्ष॒त्राण्य॑भूव॒न्निति॑।
तन्नक्ष॑त्राणां नक्षत्र॒त्वम्।
आ श्रेय॑सो॒ भ्रातृ॑व्यस्य॒ तेज॑ इन्द्रि॒यं द॑त्ते।
य ए॒तेन॒ यज॑ते।
य उ॑ चैनमे॒वं वेद॑॥६४॥

%2.7.18.4
तद्यथा॑ ह॒ वै स॑च॒क्रिणौ॒ कप्ल॑कावु॒पाव॑हितौ॒ स्याता᳚म्।
ए॒वमे॒तौ यु॒ग्मन्तौ॒ स्तोमौ᳚।
अ॒युक्षु॒ स्तोमे॑षु क्रियेते।
पा॒प्मनो\-ऽप॑हत्यै।
अप॑ पा॒प्मानं॒ भ्रातृ॑व्यꣳ हते।
य ए॒तेन॒ यज॑ते।
य उ॑ चैनमे॒वं वेद॑।
तद्यथा॑ ह॒ वै सू॑तग्राम॒ण्यः॑।
ए॒वं छन्दाꣳ॑सि।
तेष्व॒सावा॑दि॒त्यो बृ॑ह॒तीर॒भ्यू॑ढः॥६५॥

%2.7.18.5
स॒तोबृ॑हतीषु स्तुवते स॒तो बृ॑हन्।
प्र॒जया॑ प॒शुभि॑रसा॒नीत्ये॒व।
व्यति॑षक्ताभिः स्तुवते।
व्यति॑षक्तं॒ वै क्ष॒त्रं वि॒शा।
वि॒शैवैनं॑ क्ष॒त्रेण॒ व्यति॑षजति।
व्यति॑षक्ताभिः स्तुवते।
व्यति॑षक्तो॒ वै ग्रा॑म॒णीः स॑जा॒तैः।
स॒जा॒तैरे॒वैनं॒ व्यति॑षजति।
व्यति॑षक्ताभिः स्तुवते।
व्यति॑षक्तो॒ वै पुरु॑षः पा॒प्मभिः॑।
व्यति॑षक्ताभिरे॒वास्य॑ पा॒प्मनो॑ नुदते॥६६॥\anuvakamend[वेद॒ हर॑न्त्येनमे॒वं वेदा॒भ्यू॑ढः पा॒प्मभि॒रेकं॑ च]




\prashnaend{त्रि॒वृद्यदा᳚ग्ने॒यो᳚\-ऽग्निमु॑खा॒ ह्यृद्धि॒र्यदा᳚ग्ने॒य आ᳚ग्ने॒यो न वै सोमे॑न॒ यो वै सोमे॑नै॒ष गो॑स॒वः सि॒ꣳ॒हे॑ऽभि प्रेहि॑ मित्र॒वर्ध॑नः प्र॒जा\-प॑ति॒स्ता ओ॑द॒नं प्र॒जा\-प॑तिरकामयत ब॒होर्भूया॑न॒गस्त्यो॒स्या जरा॑स॒स्तिष्ठा॒ हरी᳚ प्र॒जा\-प॑तिः प॒शून्व्या॒घ्रो॑\-ऽयम॒भिप्रेहि॑ वृत्र॒हन्त॑मो॒ ये के॒शिन॒ इन्द्रं॒ वा अ॒ष्टाद॑श॥१८॥}{त्रि॒वृद्यो वै सोमे॒नायु॑रसि ब॒हुर्भ॑वति॒ तिष्ठा॒ हरी॒रथ॒ आयं भा॑तु॒ तेभ्यो॑ नि॒धान॒ꣳ॒ षट्थ्ष॑ष्टिः॥६६॥}{त्रि॒वृत्पा॒प्मनो॑ नुदते॥}{हरिः॑ ओम्॥}{इति श्रीकृष्णयजुर्वेदीयतैत्तिरीयब्राह्मणे द्वितीयाष्टके सप्तमः प्रपाठकः समाप्तः॥}
\clearpage
\sect{अष्टमः प्रश्नः}
\setcounter{anuvakam}{0}
\dnsub{तैत्तिरीयब्राह्मणे द्वितीयाष्टके अष्टमः प्रपाठकः}

%2.8.1.1
पीवो᳚न्नाꣳ रयि॒वृधः॑ सुमे॒धाः।
श्वे॒तः सि॑षक्ति नि॒युता॑\-मभि॒श्रीः।
ते वा॒यवे॒ सम॑नसो॒ वित॑स्थुः।
विश्वेन्नरः॑ स्वप॒त्यानि॑ चक्रुः।
रा॒येऽनु यञ्ज॒ज्ञतू॒ रोद॑सी उ॒भे।
रा॒ये दे॒वी धि॒षणा॑ धाति दे॒वम्।
अधा॑ वा॒युं नि॒युतः॑ सश्चत॒ स्वाः।
उ॒त श्वे॒तं वसु॑धितिन्निरे॒के।
आ वा॑यो॒ प्र याभिः॑।
प्र वा॒युमच्छा॑ बृह॒ती म॑नी॒षा॥१॥

%2.8.1.2
बृ॒हद्र॑यिं वि॒श्ववा॑राꣳ रथ॒प्राम्।
द्यु॒तद्या॑मा नि॒युतः॒ पत्य॑मानः।
क॒विः क॒विमि॑यक्षसि प्रयज्यो।
आ नो॑ नि॒युद्भिः॑ श॒तिनी॑भिरध्व॒रम्।
स॒ह॒स्रिणी॑भि॒रुप॑ याहि य॒ज्ञम्।
वायो॑ अ॒स्मिन् ह॒विषि॑ मादयस्व।
यू॒यं पा॑त स्व॒स्तिभिः॒ सदा॑ नः।
प्रजा॑पते॒ न त्वदे॒तान्य॒न्यः।
विश्वा॑ जा॒तानि॒ परि॒ ता ब॑भूव।
यत्का॑मास्ते जुहु॒मस्तं नो॑ अस्तु॥२॥

%2.8.1.3
व॒यꣴ स्या॑म॒ पत॑यो रयी॒णाम्।
र॒यी॒णां पतिं॑ यज॒तं बृ॒हन्तम्᳚।
अ॒स्मिन्भरे॒ नृत॑मं॒ वाज॑सातौ।
प्र॒जा\-प॑तिं प्रथम॒जामृ॒तस्य॑।
यजा॑म दे॒वमधि॑ नो ब्रवीतु।
प्रजा॑पते॒ त्वन्नि॑धि॒पाः पु॑रा॒णः।
दे॒वानां᳚ पि॒ता ज॑नि॒ता प्र॒जाना᳚म्।
पति॒र्विश्व॑स्य॒ जग॑तः पर॒स्पाः।
ह॒विर्नो॑ देव विह॒वे जु॑षस्व।
तवे॒मे लो॒काः प्र॒दिशो॒ दिश॑श्च॥३॥

%2.8.1.4
प॒रा॒वतो॑ नि॒वत॑ उ॒द्वत॑श्च।
प्रजा॑पते विश्व॒सृज्जी॒वध॑न्य इ॒दं नो॑ देव।
प्रति॑\-हर्य ह॒व्यम्।
प्र॒जा\-प॑तिं प्रथ॒मं य॒ज्ञिया॑नाम्।
दे॒वाना॒मग्रे॑ यज॒तं य॑जध्वम्।
स नो॑ ददातु॒ द्रवि॑णꣳ सु॒वीर्यम्᳚।
रा॒यस्पोषं॒ वि ष्य॑तु॒ नाभि॑म॒स्मे।
यो रा॒य ईशे॑ शतदा॒य उ॒क्थ्यः॑।
यः प॑शू॒नाꣳ र॑क्षि॒ता विष्ठि॑तानाम्।
प्र॒जा\-प॑तिः प्रथम॒जा ऋ॒तस्य॑॥४॥

%2.8.1.5
स॒हस्र॑धामा जुषताꣳ ह॒विर्नः॑।
सोमा॑पूषणे॒मौ दे॒वौ।
सोमा॑पूषणा॒ रज॑सो वि॒मानम्᳚।
स॒प्तच॑क्र॒ꣳ॒ रथ॒मवि॑श्वमिन्वम्।
वि॒षू॒वृतं॒ मन॑सा यु॒ज्यमा॑नम्।
तं जि॑न्वथो वृषणा॒ पञ्च॑रश्मिम्।
दि॒व्य॑न्यः सद॑नं च॒क्र उ॒च्चा।
पृ॒थि॒व्याम॒न्यो अध्य॒न्तरि॑क्षे।
ताव॒स्मभ्यं॑ पुरु॒वारं॑ पुरु॒क्षुम्।
रा॒यस्पोषं॒ विष्य॑ता॒न्नाभि॑म॒स्मे॥५॥

%2.8.1.6
धियं॑ पू॒षा जि॑न्वतु विश्वमि॒न्वः।
र॒यिꣳ सोमो॑ रयि॒पति॑र्दधातु।
अव॑तु दे॒व्यदि॑तिरन॒र्वा।
बृ॒हद्व॑देम वि॒दथे॑ सु॒वीराः᳚।
विश्वा᳚न्य॒न्यो भुव॑ना ज॒जान॑।
विश्व॑म॒न्यो अ॑भि॒चक्षा॑ण एति।
सोमा॑पूषणा॒वव॑तं॒ धियं॑ मे।
यु॒वभ्यां॒ विश्वाः॒ पृत॑ना जयेम।
उदु॑त्त॒मं व॑रु॒णास्त॑भ्ना॒द्द्याम्।
यत्किं चे॒दं कि॑त॒वासः॑।
अव॑ ते॒ हेड॒स्तत्त्वा॑ यामि।
आ॒दि॒त्याना॒मव॑सा॒ न द॑क्षि॒णा।
धा॒रय॑न्त आदि॒त्यास॑स्ति॒स्रो भूमी᳚र्धारयन्।
य॒ज्ञो दे॒वाना॒ꣳ॒ शुचि॑र॒पः॥६॥\anuvakamend[म॒नी॒षा\-ऽस्तु॑ च॒र्तस्या॒स्मे कि॑त॒वास॑श्च॒त्वारि॑ च]

%2.8.2.1
ते शु॒क्रासः॒ शुच॑यो रश्मि॒वन्तः॑।
सीद॑न्नादि॒त्या अधि॑ ब॒र्॒हिषि॑ प्रि॒ये।
कामे॑न दे॒वाः स॒रथं॑ दि॒वो नः॑।
आ या᳚न्तु य॒ज्ञमुप॑ नो जुषा॒णाः।
ते सू॒नवो॒ अदि॑तेः पीव॒सामिषम्᳚।
घृ॒तं पिन्व॒त्प्रति॑\-हर्यन्नृते॒जाः।
प्र य॒ज्ञिया॒ यज॑मानाय येमुरे।
आ॒दि॒त्याः कामं॑ पितु॒मन्त॑म॒स्मे।
आ नः॑ पु॒त्रा अदि॑तेर्यान्तु य॒ज्ञम्।
आ॒दि॒त्यासः॑ प॒थिभि॑र्देव॒यानैः᳚॥७॥

%2.8.2.2
अ॒स्मे कामं॑ दा॒शुषे॑ स॒न्नम॑न्तः।
पुरो॒डाशं॑ घृ॒तव॑न्तं जुषन्ताम्।
स्क॒भा॒यत॒ निर्\mbox{}ऋ॑ति॒ꣳ॒ सेध॒ताम॑तिम्।
प्र र॒श्मिभि॒र्यत॑माना अमृध्राः।
आदि॑त्याः॒ काम॒ प्रय॑तां॒ वष॑ट्कृतिम्।
जु॒षध्वं॑ नो ह॒व्यदा॑तिं यजत्राः।
आ॒दि॒त्यान्काम॒मव॑से हुवेम।
ये भू॒तानि॑ ज॒नय॑न्तो विचि॒ख्युः।
सीद॑न्तु पु॒त्रा अदि॑तेरु॒पस्थम्᳚।
स्ती॒र्णं ब॒र्॒हिर्\mbox{}ह॑वि॒रद्या॑य दे॒वाः॥८॥

%2.8.2.3
स्ती॒र्णं ब॒र्॒हिः सी॑दता य॒ज्ञे अ॒स्मिन्।
ध्रा॒जाः सेध॑न्तो॒ अम॑तिं दु॒रेवा᳚म्।
अ॒स्मभ्यं॑ पुत्रा अदितेः॒ प्र यꣳ॑सत।
आदि॑त्याः॒ काम॑ ह॒विषो॑ जुषा॒णाः।
अग्ने॒ नय॑ सु॒पथा॑ रा॒ये अ॒स्मान्।
विश्वा॑नि देव व॒युना॑नि वि॒द्वान्।
यु॒यो॒ध्य॑स्मज्जु॑हुरा॒णमेनः॑।
भूयि॑ष्ठान्ते॒ नम॑ उक्तिं विधेम।
प्र वः॑ शु॒क्राय॑ भा॒नवे॑ भरध्वम्।
ह॒व्यं म॒तिं चा॒ग्नये॒ सुपू॑तम्॥९॥

%2.8.2.4
यो दैव्या॑नि॒ मानु॑षा ज॒नूꣳषि॑।
अ॒न्तर्विश्वा॑नि वि॒द्मना॒ जिगा॑ति।
अच्छा॒ गिरो॑ म॒तयो॑ देव॒यन्तीः᳚।
अ॒ग्निं य॑न्ति॒ द्रवि॑णं॒ भिक्ष॑माणाः।
सु॒स॒न्दृशꣳ॑ सु॒प्रती॑क॒ꣴ॒ स्वञ्चम्᳚।
ह॒व्य॒वाह॑मर॒तिं मानु॑षाणाम्।
अग्ने॒ त्वम॒स्मद्यु॑यो॒ध्यमी॑वाः।
अन॑ग्नित्रा अ॒भ्य॑मन्त कृ॒ष्टीः।
पुन॑र॒स्मभ्यꣳ॑ सुवि॒ताय॑ देव।
क्षां विश्वे॑भिर॒जरे॑भिर्यजत्र॥१०॥

%2.8.2.5
अग्ने॒ त्वं पा॑रया॒ नव्यो॑ अ॒स्मान्।
स्व॒स्तिभि॒रति॑ दु॒र्गाणि॒ विश्वा᳚।
पूश्च॑ पृ॒थ्वी ब॑हु॒ला न॑ उ॒र्वी।
भवा॑ तो॒काय॒ तन॑याय॒ शं योः।
प्रका॑रवो मन॒ना व॒च्यमा॑नाः।
दे॒व॒द्रीचीं᳚ नयथ देव॒यन्तः॑।
द॒क्षि॒णा॒वाड्वा॒जिनी॒ प्राच्ये॑ति।
ह॒विर्भर॑न्त्य॒ग्नये॑ घृ॒ताची᳚।
इन्द्रं॒ नरो॑ यु॒जे रथम्᳚।
ज॒गृ॒भ्णाते॒ दक्षि॑णमिन्द्र॒ हस्तम्᳚॥११॥

%2.8.2.6
व॒सू॒यवो॑ वसुपते॒ वसू॑नाम्।
वि॒द्मा हि त्वा॒ गोप॑तिꣳ शूर॒ गोना᳚म्।
अ॒स्मभ्यं॑ चि॒त्रं वृष॑णꣳ र॒यिन्दाः᳚।
तवे॒दं विश्व॑म॒भितः॑ पश॒व्यम्᳚।
यत्पश्य॑सि॒ चक्ष॑सा॒ सूर्य॑स्य।
गवा॑मसि॒ गोप॑ति॒रेक॑ इन्द्र।
भ॒क्षी॒महि॑ ते॒ प्रय॑तस्य॒ वस्वः॑।
समि॑न्द्र णो॒ मन॑सा नेषि॒ गोभिः॑।
सꣳ सू॒रिभि॑र्मघव॒न्थ्सꣴ स्व॒स्त्या।
सं ब्रह्म॑णा दे॒वकृ॑तं॒ यदस्ति॑॥१२॥

%2.8.2.7
सं दे॒वानाꣳ॑ सुम॒त्या य॒ज्ञिया॑नाम्।
आ॒राच्छत्रु॒मप॑ बाधस्व दू॒रम्।
उ॒ग्रो यः शम्बः॑ पुरुहूत॒ तेन॑।
अ॒स्मे धे॑हि॒ यव॑म॒द्गोम॑दिन्द्र।
कृ॒धीधियं॑ जरि॒त्रे वाज॑रत्नाम्।
आ वे॒धस॒ꣳ॒ स हि शुचिः॑।
बृह॒स्पतिः॑ प्रथ॒मं जाय॑मानः।
म॒हो ज्योति॑षः पर॒मे व्यो॑मन्।
स॒प्तास्य॑स्तुविजा॒तो रवे॑ण।
वि स॒प्तर॑श्मिरधम॒त्तमाꣳ॑सि॥१३॥

%2.8.2.8
बृह॒स्पतिः॒ सम॑जय॒द्वसू॑नि।
म॒हो व्र॒जान्गोम॑तो दे॒व ए॒षः।
अ॒पः सिषा॑स॒न्थ्सुव॒रप्र॑तीत्तः।
बृह॒स्पति॒र्॒हन्त्य॒मित्र॑म॒र्कैः।
बृह॑स्पते॒ पर्ये॒वा पि॒त्रे।
आ नो॑ दि॒वः पावी॑रवी।
इ॒मा जुह्वा॑ना॒ यस्ते॒ स्तनः॑।
सर॑स्वत्य॒भि नो॑ नेषि।
इ॒यꣳ शुष्मे॑भिर्बिस॒खा इ॑वारुजत्।
सानु॑ गिरी॒णां त॑वि॒षेभि॑रू॒र्मिभिः॑।
पा॒रा॒व॒द॒घ्नीमव॑से सुवृ॒क्तिभिः॑।
सर॑स्वती॒मा वि॑वासेम धी॒तिभिः॑॥१४॥\anuvakamend[दे॒व॒यानै᳚र्दे॒वाः सुपू॑तं यजत्र॒ हस्त॒मस्ति॒ तमाꣴ॑स्यू॒र्मिभि॒र्द्वे च॑]

%2.8.3.1
सोमो॑ धे॒नुꣳ सोमो॒ अर्व॑न्तमा॒शुम्।
सोमो॑ वी॒रं क॑र्म॒ण्यं॑ ददातु।
सा॒द॒न्यं॑ विद॒थ्यꣳ॑ स॒भेयम्᳚।
पि॒तुः॒ श्रव॑णं॒ यो ददा॑शदस्मै।
अषा॑ढं यु॒थ्सु त्वꣳ सो॑म॒ क्रतु॑भिः।
या ते॒ धामा॑नि ह॒विषा॒ यज॑न्ति।
त्वमि॒मा ओष॑धीः सोम॒ विश्वाः᳚।
त्वम॒पो अ॑जनय॒स्त्वङ्गाः।
त्वमात॑तन्थो॒र्व॑न्तरि॑क्षम्।
त्वं ज्योति॑षा॒ वि तमो॑ ववर्थ॥१५॥

%2.8.3.2
या ते॒ धामा॑नि दि॒वि या पृ॑थि॒व्याम्।
या पर्व॑ते॒ष्वोष॑धीष्व॒फ्सु।
तेभि॑र्नो॒ विश्वैः᳚ सु॒मना॒ अहे॑डन्।
राज᳚न्थ्सोम॒ प्रति॑ ह॒व्या गृ॑भाय।
विष्णो॒र्नुकं॒ तद॑स्य प्रि॒यम्।
प्र तद्विष्णुः॑।
प॒रो मात्र॑या त॒नुवा॑ वृधान।
न ते॑ महि॒त्वमन्व॑श्ञुवन्ति।
उ॒भे ते॑ विद्म॒ रज॑सी पृथि॒व्या विष्णो॑ देव॒ त्वम्।
प॒र॒मस्य॑ विथ्से॥१६॥

%2.8.3.3
विच॑क्रमे॒ त्रिर्दे॒वः।
आ ते॑ म॒हो यो जा॒त ए॒व।
अ॒भि गो॒त्राणि॑।
आभिः॒ स्पृधो॑ मिथ॒तीररि॑षण्यन्।
अ॒मित्र॑स्य व्यथया म॒न्युमि॑न्द्र।
आभि॒र्विश्वा॑ अभि॒युजो॒ विषू॑चीः।
आर्या॑य॒ विशोव॑तारी॒र्दासीः᳚।
अ॒यꣳ शृ॑ण्वे॒ अध॒ जय॑न्नु॒त घ्नन्।
अ॒यमु॒त प्र कृ॑णुते यु॒धा गाः।
य॒दा स॒त्यं कृ॑णु॒ते म॒न्युमिन्द्रः॑॥१७॥

%2.8.3.4
विश्वं॑ दृ॒ढं भ॑यत॒ एज॑दस्मात्।
अनु॑ स्व॒धाम॑क्षर॒न्नापो॑ अस्य।
अव॑र्धत॒ मध्य॒ आ ना॒व्या॑नाम्।
स॒ध्री॒चीने॑न॒ मन॑सा॒ तमि॑न्द्र॒ ओजि॑ष्ठेन।
हन्म॑नाहन्न॒भिद्यून्।
म॒रुत्व॑न्तं वृष॒भं वा॑वृधा॒नम्।
अक॑वारिं दि॒व्यꣳ शा॒समिन्द्रम्᳚।
वि॒श्वा॒साह॒मव॑से॒ नूत॑नाय।
उ॒ग्रꣳ स॑हो॒दामि॒ह तꣳ हु॑वेम।
जनि॑ष्ठा उ॒ग्रः सह॑से तु॒राय॑॥१८॥

%2.8.3.5
म॒न्द्र ओजि॑ष्ठो बहु॒लाभि॑मानः।
अव॑र्ध॒न्निन्द्रं॑ म॒रुत॑श्चि॒दत्र॑।
मा॒ता यद्वी॒रं द॒धन॒द्धनि॑ष्ठा।
क्व॑स्यावो॑ मरुतः स्व॒धा\-ऽऽसी᳚त्।
यन्मामेकꣳ॑ स॒मध॑त्ताहि॒हत्ये᳚।
अ॒हꣴ ह्यु॑ग्रस्त॑वि॒षस्तुवि॑ष्मान्।
विश्व॑स्य॒ शत्रो॒रन॑मं वध॒स्नैः।
वृ॒त्रस्य॑ त्वा श्व॒सथा॒ दीष॑माणाः।
विश्वे॑ दे॒वा अ॑जहु॒र्ये सखा॑यः।
म॒रुद्भि॑रिन्द्र स॒ख्यं ते॑ अस्तु॥१९॥

%2.8.3.6
अथे॒मा विश्वाः॒ पृत॑ना जयासि।
वधीं᳚ वृ॒त्रं म॑रुत इन्द्रि॒येण॑।
स्वेन॒ भामे॑न तवि॒षो ब॑भू॒वान्।
अ॒हमे॒ता मन॑वे वि॒श्वश्च॑न्द्राः।
सु॒गा अ॒पश्च॑कर॒ वज्र॑बाहुः।
स यो वृषा॒ वृष्णि॑येभिः॒ समो॑काः।
म॒हो दि॒वः पृ॑थि॒व्याश्च॑ स॒म्राट्।
स॒ती॒नस॑त्वा॒ हव्यो॒ भरे॑षु।
म॒रुत्वां᳚ नो भव॒त्विन्द्र॑ ऊ॒ती।
इन्द्रो॑ वृ॒त्रम॑तरद्वृत्र॒तूर्ये᳚॥२०॥

%2.8.3.7
अ॒ना॒धृ॒ष्यो म॒घवा॒ शूर॒ इन्द्रः॑।
अन्वे॑नं॒ विशो॑ अमदन्त पू॒र्वीः।
अ॒यꣳ राजा॒ जग॑तश्चर्\mbox{}षणी॒नाम्।
स ए॒व वी॒रः स उ॑ वी॒र्या॑वान्।
स ए॑करा॒जो जग॑तः पर॒स्पाः।
य॒दा वृ॒त्रमत॑र॒च्छूर॒ इन्द्रः॑।
अथा॑भवद्दमि॒ताभिक्र॑तूनाम्।
इन्द्रो॑ य॒ज्ञं व॒र्धय॑न्वि॒श्ववे॑दाः।
पु॒रो॒डाश॑स्य जुषताꣳ ह॒विर्नः॑।
वृ॒त्रं ती॒र्त्वा दा॑न॒वं वज्र॑बाहुः॥२१॥

%2.8.3.8
दिशो॑\-ऽदृꣳहद्दृꣳहि॒ता दृꣳह॑णेन।
इ॒मं य॒ज्ञं व॒र्धय॑न्वि॒श्व\-वे॑दाः।
पु॒रो॒डाशं॒ प्रति॑ गृभ्णा॒त्विन्द्रः॑।
य॒दा वृ॒त्रमत॑र॒च्छूर॒ इन्द्रः॑।
अथै॑करा॒जो अ॑भव॒ज्जना॑नाम्।
इन्द्रो॑ दे॒वाञ्छ॑म्बर॒हत्य॑ आवत्।
इन्द्रो॑ दे॒वाना॑मभवत्पुरो॒गाः।
इन्द्रो॑ य॒ज्ञे ह॒विषा॑ वावृधा॒नः।
वृ॒त्र॒तूर्नो॒ अभ॑य॒ꣳ॒ शर्म॑ यꣳसत्।
यः स॒प्त सिन्धू॒ꣳ॒ रद॑धात्पृथि॒व्याम्।
यः स॒प्त लो॒कानकृ॑णो॒द्दिश॑श्च।
इन्द्रो॑ ह॒विष्मा॒न्थ्सग॑णो म॒रुद्भिः॑।
वृ॒त्र॒तूर्नो॑ य॒ज्ञमि॒होप॑ यासत्॥२२॥\anuvakamend[व॒व॒र्थ॒ वि॒थ्स॒ इन्द्र॑स्तु॒राया᳚स्तु वृत्र॒तूर्ये॒ वज्र॑बाहुः पृथि॒व्यान्त्रीणि॑ च]

%2.8.4.1
इन्द्र॒स्तर॑स्वानभिमाति॒होग्रः।
हिर॑ण्यवाशीरिषि॒रः सु॑व॒र्॒षाः।
तस्य॑ व॒यꣳ सु॑म॒तौ य॒ज्ञिय॑स्य।
अपि॑ भ॒द्रे सौ॑मन॒से स्या॑म।
हिर॑ण्यवर्णो॒ अभ॑यं कृणोतु।
अ॒भि॒मा॒ति॒हेन्द्रः॒ पृत॑नासु जि॒ष्णुः।
स नः॒ शर्म॑ त्रि॒वरू॑थं॒ वि यꣳ॑सत्।
यू॒यं पा॑त स्व॒स्तिभिः॒ सदा॑ नः।
इन्द्रꣴ॑ स्तुहि व॒ज्रिण॒ꣴ॒ स्तोम॑पृष्ठम्।
पु॒रो॒डाश॑स्य जुषताꣳ ह॒विर्नः॑॥२३॥

%2.8.4.2
ह॒त्वाऽभिमा॑तीः॒ पृत॑नाः॒ सह॑स्वान्।
अथाभ॑यं कृणुहि वि॒श्वतो॑ नः।
स्तु॒हि शूरं॑ व॒ज्रिण॒मप्र॑तीत्तम्।
अ॒भि॒मा॒ति॒हनं॑ पुरुहू॒तमिन्द्रम्᳚।
य एक॒ इच्छ॒तप॑ति॒र्जने॑षु।
तस्मा॒ इन्द्रा॑य ह॒विरा जु॑होत।
इन्द्रो॑ दे॒वाना॑मधि॒पाः पु॒रोहि॑तः।
दि॒शां पति॑रभवद्वा॒जिनी॑वान्।
अ॒भि॒मा॒ति॒हा त॑वि॒षस्तुवि॑ष्मान्।
अ॒स्मभ्यं॑ चि॒त्रं वृष॑णꣳ र॒यिन्दा᳚त्॥२४॥

%2.8.4.3
य इ॒मे द्यावा॑पृथि॒वी म॑हि॒त्वा।
बले॒नादृꣳ॑हदभिमाति॒हेन्द्रः॑।
स नो॑ ह॒विः प्रति॑ गृभ्णातु रा॒तये᳚।
दे॒वानां᳚ दे॒वो नि॑धि॒पा नो॑ अव्यात्।
अन॑वस्ते॒ रथं॒ वृष्णे॒ यत्ते᳚।
इन्द्र॑स्य॒ नु वी॒र्या᳚ण्यह॒न्नहिम्᳚।
इन्द्रो॑ या॒तो\-ऽव॑सितस्य॒ राजा᳚।
शम॑स्य च शृ॒ङ्गिणो॒ वज्र॑बाहुः।
सेदु॒ राजा᳚ क्षेति चर्\mbox{}षणी॒नाम्।
अ॒रान्न ने॒मिः परि॒ ता ब॑भूव॥२५॥

%2.8.4.4
अ॒भि सि॒ध्मो अ॑जिगादस्य॒ शत्रून्॑।
विति॒ग्मेन॑ वृष॒भेणा॒ पुरो॑भेत्।
सं वज्रे॑णासृजद्वृ॒त्र\-मिन्द्रः॑।
प्र स्वां म॒तिम॑तिर॒च्छाश॑दानः।
विष्णुं॑ दे॒वं वरु॑णमू॒तये॒ भगम्᳚।
मेद॑सा दे॒वा व॒पया॑ यजध्वम्।
ता नो॑ य॒ज्ञमाग॑तं वि॒श्वधे॑ना।
प्र॒जाव॑द॒स्मे द्रवि॑णे॒ह ध॑त्तम्।
मेद॑सा दे॒वा व॒पया॑ यजध्वम्।
विष्णुं॑ च दे॒वं वरु॑णं च रा॒तिम्॥२६॥

%2.8.4.5
ता नो॒ अमी॑वा अप॒ बाध॑मानौ।
इ॒मं य॒ज्ञं जु॒षमा॑णा॒वुपेतम्᳚।
विष्णू॑वरुणा यु॒वम॑ध्व॒राय॑ नः।
वि॒शे जना॑य॒ महि॒ शर्म॑ यच्छतम्।
दी॒र्घप्र॑यज्ज्यू ह॒विषा॑ वृधा॒ना।
ज्योति॒षा\-ऽरा॑तीर्दह\-त॒न्तमाꣳ॑सि।
ययो॒रोज॑सा स्कभि॒ता रजाꣳ॑सि।
वी॒र्ये॑भिर्वी॒रत॑मा॒ शवि॑ष्ठा।
याऽपत्ये॑ ते॒ अप्र॑तीत्ता॒ सहो॑भिः।
विष्णू॑ अग॒न्वरु॑णा पू॒र्वहू॑तौ॥२७॥

%2.8.4.6
विष्णू॑वरुणावभिशस्ति॒पावा᳚म्।
दे॒वा य॑जन्त ह॒विषा॑ घृ॒तेन॑।
अपामी॑वाꣳ सेधतꣳ र॒क्षस॑श्च।
अथा॑धत्तं॒ यज॑मानाय॒ शं योः।
अ॒ꣳ॒हो॒मुचा॑ वृष॒भा सु॒प्रतू᳚र्ती।
दे॒वानां᳚ दे॒वत॑मा॒ शचि॑ष्ठा।
विष्णू॑वरुणा॒ प्रति॑\-हर्यतन्नः।
इ॒दं नरा॒ प्रय॑तमू॒तये॑ ह॒विः।
म॒ही नु द्यावा॑पृथि॒वी इ॒ह ज्येष्ठे᳚।
रु॒चा भ॑वताꣳ शु॒चय॑द्भिर॒र्कैः॥२८॥

%2.8.4.7
यथ्सीं॒ वरि॑ष्ठे बृह॒ती वि॑मि॒न्वन्।
नृ॒वद्भ्यो॒क्षा प॑प्रथा॒नेभि॒रेवैः᳚।
प्रपू᳚र्व॒जे पि॒तरा॒ नव्य॑सीभिः।
गी॒र्भिः कृ॑णुध्व॒ꣳ॒ सद॑ने ऋ॒तस्य॑।
आ नो᳚ द्यावापृथिवी॒ दैव्ये॑न।
जने॑न यातं॒ महि॑ वां॒ वरू॑थम्।
स इथ्स्वपा॒ भुव॑नेष्वास।
य इ॒मे द्यावा॑पृथि॒वी ज॒जान॑।
उ॒र्वी ग॑भी॒रे रज॑सी सु॒मेके᳚।
अ॒व॒ꣳ॒शे धीरः॒ शच्या॒ समै॑रत्॥२९॥

%2.8.4.8
भूरिं॒ द्वे अच॑रन्ती॒ चर॑न्तम्।
प॒द्वन्तं॒ गर्भ॑म॒पदी॑दधाते।
नित्यं॒ न सू॒नुं पि॒त्रोरु॒पस्थे᳚।
तं पि॑पृतꣳ रोदसी सत्य॒वाचम्᳚।
इ॒दं द्या॑वापृथिवी स॒त्यम॑स्तु।
पित॒र्मात॒र्यदि॒होप॑ ब्रु॒वे वा᳚म्।
भू॒तं दे॒वाना॑मव॒मे अवो॑भिः।
विद्यामे॒षं वृ॒जनं॑ जी॒रदा॑नुम्।
उ॒र्वी पृ॒थ्वी ब॑हु॒ले दू॒रे अ॑न्ते।
उप॑ ब्रुवे॒ नम॑सा य॒ज्ञे अ॒स्मिन्।
दधा॑ते॒ ये सु॒भगे॑ सु॒प्रतू᳚र्ती।
द्यावा॒ रक्ष॑तं पृथि॒वी नो॒ अभ्वा᳚त्।
या जा॒ता ओष॑ध॒योऽति॒ विश्वाः᳚ परि॒ष्ठाः।
या ओष॑धयः॒ सोम॑राज्ञीरश्वाव॒तीꣳ सो॑मव॒तीम्।
ओष॑धी॒रिति॑ मातरो॒\-ऽन्या वो॑ अ॒न्याम॑वतु॥३०॥\anuvakamend[ह॒विर्नो॑ दाद्भभूव रा॒तिं पू॒र्वहू॑ताव॒र्कैरै॑रद॒स्मिन्पञ्च॑ च]

%2.8.5.1
शुचिं॒ नु स्तोम॒ꣴ॒ श्ञथ॑द्वृ॒त्रम्।
उ॒भा वा॑मिन्द्राग्नी॒ प्र च॑र्\mbox{}ष॒णिभ्यः॑।
आ वृ॑त्रहणा गी॒र्भिर्विप्रः॑।
ब्रह्म॑णस्पते॒ त्वम॒स्य य॒न्ता।
सू॒क्तस्य॑ बोधि॒ तन॑यं च जिन्व।
विश्वं॒ तद्भ॒द्रं यद॒वन्ति॑ दे॒वाः।
बृ॒हद्व॑देम वि॒दथे॑ सु॒वीराः᳚।
स ईꣳ॑ स॒त्येभिः॒ सखि॑भिः शु॒चद्भिः॑।
गोधा॑यसं॒ विध॑न॒सैर॑तर्दत्।
ब्रह्म॑ण॒स्पति॒र्वृष॑भिर्व॒राहैः᳚॥३१॥

%2.8.5.2
घ॒र्मस्वे॑देभि॒र्द्रवि॑णं॒ व्या॑नट्।
ब्रह्म॑ण॒स्पते॑रभवद्यथाव॒शम्।
स॒त्यो म॒न्युर्महि॒ कर्मा॑ करिष्य॒तः।
यो गा उ॒दाज॒थ्स दि॒वे वि चा॑भजत्।
म॒हीव॑ री॒तिः शव॑सा सर॒त्पृथ॑क्।
इन्धा॑नो अ॒ग्निं व॑नवद्वनुष्य॒तः।
कृ॒तब्र॑ह्मा शूशुवद्रा॒तह॑व्य॒ इत्।
जा॒तेन॑ जा॒तमति॒सृत्प्र सृꣳ॑सते।
यं यं॒ युजं॑ कृणु॒ते ब्रह्म॑ण॒स्पतिः॑।
ब्रह्म॑णस्पते सु॒यम॑स्य वि॒श्वहा᳚॥३२॥

%2.8.5.3
रा॒यः स्या॑म र॒थ्यो॑ विव॑स्वतः।
वी॒रेषु॑ वी॒राꣳ उप॑पृङ्ग्धि न॒स्त्वम्।
यदीशा॑नो॒ ब्रह्म॑णा॒ वेषि॑ मे॒ हवम्᳚।
स इज्जने॑न॒ स वि॒शा स जन्म॑ना।
स पु॒त्रैर्वाजं॑ भरते॒ धना॒ नृभिः॑।
दे॒वानां॒ यः पि॒तर॑मा॒ विवा॑सति।
श्र॒द्धाम॑ना ह॒विषा॒ ब्रह्म॑ण॒स्पतिम्᳚।
यास्ते॑ पूष॒न्नावो॑ अ॒न्तः।
शु॒क्रं ते॑ अ॒न्यत्पू॒षेमा आशाः᳚।
प्रप॑थे प॒थाम॑जनिष्ट पू॒षा॥३३॥

%2.8.5.4
प्रप॑थे दि॒वः प्रप॑थे पृथि॒व्याः।
उ॒भे अ॒भि प्रि॒यत॑मे स॒धस्थे᳚।
आ च॒ परा॑ च चरति प्रजा॒नन्।
पू॒षा सु॒बन्धु॑र्दि॒व आ पृ॑थि॒व्याः।
इ॒डस्पति॑र्म॒घवा॑ द॒स्मव॑र्चाः।
तं दे॒वासो॒ अद॑दुः सू॒र्यायै᳚।
कामे॑न कृ॒तं त॒वस॒ꣴ॒ स्वञ्चम्᳚।
अ॒जाऽश्वः॑ पशु॒पा वाज॑बस्त्यः।
धि॒यं॒ जि॒न्वो विश्वे॒ भुव॑ने॒ अर्पि॑तः।
अष्ट्रां᳚ पू॒षा शि॑थि॒रामु॒द्वरी॑वृजत्॥३४॥

%2.8.5.5
स॒ञ्चक्षा॑णो॒ भुव॑ना दे॒व ई॑यते।
शुची॑ वो ह॒व्या म॑रुतः॒ शुची॑नाम्।
शुचिꣳ॑ हिनोम्यध्व॒रꣳ शुचि॑भ्यः।
ऋ॒तेन॑ स॒त्यमृत॒साप॑ आयन्।
शुचि॑जन्मानः॒ शुच॑यः पाव॒काः।
प्र चि॒त्रम॒र्कं गृ॑ण॒ते तु॒राय॑।
मारु॑ताय॒ स्वत॑वसे भरध्वम्।
ये सहाꣳ॑सि॒ सह॑सा॒ सह॑न्ते।
रेज॑ते अग्ने पृथि॒वी म॒खेभ्यः॑।
अꣳसे॒ष्वा म॑रुतः खा॒दयो॑ वः॥३५॥

%2.8.5.6
वक्षः॑ सुरु॒क्मा उप॑ शिश्रिया॒णाः।
वि वि॒द्युतो॒ न वृ॒ष्टिभी॑ रुचा॒नाः।
अनु॑ स्व॒धामायु॑धै॒र्यच्छ॑मानाः।
या वः॒ शर्म॑ शशमा॒नाय॒ सन्ति॑।
त्रि॒धातू॑नि दा॒शुषे॑ यच्छ॒ताधि॑।
अ॒स्मभ्यं॒ तानि॑ मरुतो॒ विय॑न्त।
र॒यिं नो॑ धत्त वृषणः सु॒वीरम्᳚।
इ॒मे तु॒रं म॒रुतो॑ रामयन्ति।
इ॒मे सहः॒ सह॑स॒ आ न॑मन्ति।
इ॒मे शꣳसं॑वनुष्य॒तो नि पा᳚न्ति॥३६॥

%2.8.5.7
गु॒रुद्वेषो॒ अर॑रुषे दधन्ति।
अ॒रा इ॒वेदच॑रमा॒ अहे॑व।
प्रप्र॑ जायन्ते॒ अक॑वा॒ महो॑भिः।
पृश्ञेः᳚ पु॒त्रा उ॑प॒मासो॒ रभि॑ष्ठाः।
स्वया॑ म॒त्या म॒रुतः॒ सं मि॑मिक्षुः।
अनु॑ ते दायि म॒ह इ॑न्द्रि॒याय॑।
स॒त्रा ते॒ विश्व॒मनु॑ वृत्र॒हत्ये᳚।
अनु॑ क्ष॒त्रमनु॒ सहो॑ यजत्र।
इन्द्र॑ दे॒वेभि॒रनु॑\- ते नृ॒षह्ये᳚।
य इन्द्र॒ शुष्मो॑ मघवन्ते॒ अस्ति॑॥३७॥

%2.8.5.8
शिक्षा॒ सखि॑भ्यः पुरुहूत॒ नृभ्यः॑।
त्वꣳ हि दृ॒ढा म॑घव॒न्विचे॑ताः।
अपा॑वृधि॒ परि॑वृतिं॒ न राधः॑।
इन्द्रो॒ राजा॒ जग॑तश्चर्‌\mbox{}षणी॒नाम्।
अ॒धि॒क्षमि॒ विषु॑रूपं॒ यदस्ति॑।
ततो॑ ददातु दा॒शुषे॒ वसू॑नि।
चोद॒द्राध॒ उप॑स्तुतश्चिद॒र्वाक्।
तमु॑ष्टुहि॒ यो अ॒भिभू᳚त्योजाः।
व॒न्वन्नवा॑तः पुरुहू॒त इन्द्रः॑।
अषा॑ढमु॒ग्रꣳ सह॑मानमा॒भिः॥३८॥

%2.8.5.9
गी॒र्भिर्व॑र्ध वृष॒भं च॑र्\mbox{}षणी॒नाम्।
स्थू॒रस्य॑ रा॒यो बृ॑ह॒तो य ईशे᳚।
तमु॑ ष्टवाम वि॒दथे॒ष्विन्द्रम्᳚।
यो वा॒युना॒ जय॑ति॒ गोम॑तीषु।
प्र धृ॑ष्णु॒या न॑यति॒ वस्यो॒ अच्छ॑।
आ ते॒ शुष्मो॑ वृष॒भ ए॑तु प॒श्चात्।
ओत्त॒राद॑ध॒रागा पु॒रस्ता᳚त्।
आ वि॒श्वतो॑ अ॒भिसमे᳚त्व॒र्वाङ्।
इन्द्र॑ द्यु॒म्नꣳ सुव॑र्वद्धेह्य॒स्मे॥३९॥\anuvakamend[व॒राहै᳚र्वि॒श्वहा॑\-ऽजनिष्ट पू॒षोद्वरी॑वृजत्खा॒दयो॑ वः पा॒न्त्यस्त्या॒भिर्नव॑ च]

%2.8.6.1
आ दे॒वो या॑तु सवि॒ता सु॒रत्नः॑।
अ॒न्त॒रि॒क्ष॒प्रा वह॑मानो॒ अश्वैः᳚।
हस्ते॒ दधा॑नो॒ नर्या॑ पु॒रूणि॑।
नि॒वे॒शयं॑ च प्रसु॒वं च॒ भूम॑।
अ॒भीवृ॑तं॒ कृश॑नैर्वि॒श्वरू॑पम्।
हिर॑ण्यशम्यं यज॒तो बृ॒हन्तम्᳚।
आस्था॒द्रथꣳ॑ सवि॒ता चि॒त्रभा॑नुः।
कृ॒ष्णा रजाꣳ॑सि॒ तवि॑षीं॒ दधा॑नः।
सघा॑ नो दे॒वः स॑वि॒ता स॒वाय॑।
आ सा॑विष॒द्वसु॑पति॒र्वसू॑नि॥४०॥

%2.8.6.2
वि॒श्रय॑माणो॒ अम॑तिमुरू॒चीम्।
म॒र्त॒भोज॑न॒मध॑रासतेन।
विजना᳚ञ्छ्या॒वाः शि॑ति॒पादो॑ अख्यन्।
रथ॒ꣳ॒ हिर॑ण्यप्रउगं॒ वह॑न्तः।
शश्व॒द्दिशः॑ सवि॒तुर्दैव्य॑स्य।
उ॒पस्थे॒ विश्वा॒ भुव॑नानि तस्थुः।
वि सु॑प॒र्णो अ॒न्तरि॑क्षाण्यख्यत्।
ग॒भी॒रवे॑पा॒ असु॑रः सुनी॒थः।
क्वे॑दानी॒ꣳ॒ सूर्यः॒ कश्चि॑केत।
क॒त॒मान्द्याꣳ र॒श्मिर॒स्या त॑तान॥४१॥

%2.8.6.3
भगं॒ धियं॑ वा॒जय॑न्तः॒ पुर॑न्धिम्।
नरा॒शꣳसो॒ ग्नास्पति॑र्नो अव्यात्।
आ ये वा॒मस्य॑ सङ्ग॒थे र॑यी॒णाम्।
प्रि॒या दे॒वस्य॑ सवि॒तुः स्या॑म।
आ नो॒ विश्वे॒ अस्क्रा॑गमन्तु दे॒वाः।
मि॒त्रो अ॑र्य॒मा वरु॑णः स॒जोषाः᳚।
भुव॒न्॒ यथा॑ नो॒ विश्वे॑ वृ॒धासः॑।
कर᳚न्थ्सु॒षाहा॑ विथु॒रं न शवः॑।
शं नो॑ दे॒वा वि॒श्वदे॑वा भवन्तु।
शꣳ सर॑स्वती स॒ह धी॒भिर॑स्तु॥४२॥

%2.8.6.4
शम॑भि॒षाचः॒ शमु॑ राति॒षाचः॑।
शं नो॑ दि॒व्याः पार्थि॑वाः॒ शं नो॒ अप्याः᳚।
ये स॑वि॒तुः स॒त्यस॑वस्य॒ विश्वे᳚।
मि॒त्रस्य॑ व्र॒ते वरु॑णस्य दे॒वाः।
ते सौभ॑गं वी॒रव॒द्गोम॒दप्नः॑।
दधा॑तन॒ द्रवि॑णं चि॒त्रम॒स्मे।
अग्ने॑ या॒हि दू॒त्यं॑ वारि॑षेण्यः।
दे॒वाꣳ अच्छा᳚ ब्रह्म॒कृता॑ ग॒णेन॑।
सर॑स्वतीं म॒रुतो॑ अ॒श्विना॒\-ऽपः।
य॒क्षि॒ दे॒वान्र॑त्न॒धेया॑य॒ विश्वान्॑॥४३॥

%2.8.6.5
द्यौः पि॑तः॒ पृथि॑वि॒ मात॒रध्रु॑क्।
अग्ने᳚ भ्रातर्वसवो मृ॒डता॑ नः।
विश्व॑ आदित्या अदिते स॒जोषाः᳚।
अ॒स्मभ्य॒ꣳ॒ शर्म॑ बहु॒लं वि य॑न्त।
विश्वे॑ देवाः शृणु॒तेमꣳ हवं॑ मे।
ये अ॒न्तरि॑क्षे॒ य उप॒ द्यवि॒ ष्ठ।
ये अ॑ग्निजि॒ह्वा उ॒त वा॒ यज॑त्राः।
आ॒सद्या॒स्मिन्ब॒र्॒हिषि॑ मादयध्वम्।
आ वां᳚ मित्रावरुणा ह॒व्यजु॑ष्टिम्।
नम॑सा देवा॒वव॑साऽऽववृत्याम्॥४४॥

%2.8.6.6
अ॒स्माकं॒ ब्रह्म॒ पृत॑नासु सह्या अ॒स्माकम्᳚।
वृ॒ष्टिर्दि॒व्या सु॑पा॒रा।
यु॒वं वस्त्रा॑णि पीव॒सा व॑साथे।
यु॒वोरच्छि॑द्रा॒ मन्त॑वो ह॒ सर्गाः᳚।
अवा॑तिरत॒मनृ॑तानि॒ विश्वा᳚।
ऋ॒तेन॑ मित्रावरुणा सचेथे।
तथ्सु वां᳚ मित्रावरुणा महि॒त्वम्।
ई॒र्मा त॒स्थुषी॒रह॑भिर्दुदुह्रे।
विश्वाः᳚ पिन्वथ॒ स्वस॑रस्य॒ धेनाः᳚।
अनु॑ वा॒मेकः॑ प॒विरा व॑वर्ति॥४५॥

%2.8.6.7
यद्बꣳहि॑ष्ठ॒न्नाति॒ विदे॑ सुदानू।
अच्छि॑द्र॒ꣳ॒ शर्म॒ भुव॑नस्य गोपा।
ततो॑ नो मित्रावरुणाववीष्टम्।
सिषा॑सन्तो जीगि॒वाꣳसः॑ स्याम।
आ नो॑ मित्रावरुणा ह॒व्यदा॑तिम्।
घृ॒तैर्गव्यू॑तिमुक्षत॒मिडा॑भिः।
प्रति॑ वा॒मत्र॒ वर॒मा जना॑य।
पृ॒णी॒तमु॒द्नो दि॒व्यस्य॒ चारोः᳚।
प्र बा॒हवा॑ सिसृतं जी॒वसे॑ नः।
आ नो॒ गव्यू॑तिमुक्षतं घृ॒तेन॑॥४६॥

%2.8.6.8
आ नो॒ जने᳚ श्रवयतं युवाना।
श्रु॒तं मे॑ मित्रावरुणा॒ हवे॒मा।
इ॒मा रु॒द्राय॑ स्थि॒रध॑न्वने॒ गिरः॑।
क्षि॒प्रेष॑वे दे॒वाय॑ स्व॒धाम्ने᳚।
अषा॑ढाय॒ सह॑मानाय मी॒ढुषे᳚।
ति॒ग्मायु॑धाय भरता शृ॒णोत॑न।
त्वाद॑त्तेभी रुद्र॒ शन्त॑मेभिः।
श॒तꣳ हिमा॑ अशीय भेष॒जेभिः॑।
व्य॑स्मद्द्वेषो॑ वित॒रं व्यꣳहः॑।
व्यमी॑वाꣴश्चातयस्वा॒ विषू॑चीः॥४७॥

%2.8.6.9
अर्\mbox{}ह॑न्बिभर्\mbox{}षि॒ मा न॑स्तो॒के।
आ ते॑ पितर्मरुताꣳ सु॒म्नमे॑तु।
मा नः॒ सूर्य॑स्य स॒न्दृशो॑ युयोथाः।
अ॒भि नो॑ वी॒रो अर्व॑ति क्षमेत।
प्र जा॑येमहि रुद्र प्र॒जाभिः॑।
ए॒वा ब॑भ्रो वृषभ चेकितान।
यथा॑ देव॒ न हृ॑णी॒षे न हꣳसि॑।
हा॒व॒न॒श्रूर्नो॑ रुद्रे॒ह बो॑धि।
बृ॒हद्व॑देम वि॒दथे॑ सु॒वीराः᳚।
परि॑ णो रु॒द्रस्य॑ हे॒तिः स्तु॒हि श्रु॒तम्।
मीढु॑ष्ट॒मार्\mbox{}ह॑न्बिभर्\mbox{}षि।
त्वम॑ग्ने रु॒द्र आ वो॒ राजा॑नम्॥४८॥\anuvakamend[वसू॑नि ततानास्तु॒ विश्वान्॑ ववृत्यां ववर्ति घृ॒तेन॒ विषू॑चीः श्रु॒तन्द्वे च॑]

%2.8.7.1
सूर्यो॑ दे॒वीमु॒षस॒ꣳ॒ रोच॑माना॒मर्यः॑।
न योषा॑म॒भ्ये॑ति प॒श्चात्।
यत्रा॒ नरो॑ देव॒यन्तो॑ यु॒गानि॑।
वि॒त॒न्वते॒ प्रति॑ भ॒द्राय॑ भ॒द्रम्।
भ॒द्रा अश्वा॑ ह॒रितः॒ सूर्य॑स्य।
चि॒त्रा एद॑ग्वा अनु॒माद्या॑सः।
न॒म॒स्यन्तो॑ दि॒व आ पृ॒ष्ठम॑स्थुः।
परि॒ द्यावा॑पृथि॒वी य॑न्ति स॒द्यः।
तथ्सूर्य॑स्य देव॒त्वं तन्म॑हि॒त्वम्।
म॒ध्या कर्तो॒र्वित॑त॒ꣳ॒ सञ्ज॑भार॥४९॥

%2.8.7.2
य॒देदयु॑क्त ह॒रितः॑ स॒धस्था᳚त्।
आद्रात्री॒ वास॑स्तनुते सि॒मस्मै᳚।
तन्मि॒त्रस्य॒ वरु॑णस्याभि॒चक्षे᳚।
सूर्यो॑ रू॒पं कृ॑णुते॒ द्योरु॒पस्थे᳚।
अ॒न॒न्तम॒न्यद्रुश॑दस्य॒ पाजः॑।
कृ॒ष्णम॒न्यद्ध॒रितः॒ सं भ॑रन्ति।
अ॒द्या दे॑वा॒ उदि॑ता॒ सूर्य॑स्य।
निरꣳह॑सः पिपृ॒तान्निर॑व॒द्यात्।
तन्नो॑ मि॒त्रो वरु॑णो मामहन्ताम्।
अदि॑तिः॒ सिन्धुः॑ पृथि॒वी उ॒त द्यौः॥५०॥

%2.8.7.3
दि॒वो रु॒क्म उ॑रु॒चक्षा॒ उदे॑ति।
दू॒रे अ॑र्थस्त॒रणि॒र्भ्राज॑मानः।
नू॒नं जनाः॒ सूर्ये॑ण॒ प्रसू॑ताः।
आयन्नर्था॑नि कृ॒णव॒न्नपाꣳ॑सि।
शं नो॑ भव॒ चक्ष॑सा॒ शं नो॒ अह्ना᳚।
शं भा॒नुना॒ शꣳ हि॒मा शं घृ॒णेन॑।
यथा॒ शम॒स्मै शमस॑द्दुरो॒णे।
तथ्सू᳚र्य॒ द्रवि॑णं धे॒हि चि॒त्रम्।
चि॒त्रं दे॒वाना॒मुद॑गा॒दनी॑कम्।
चक्षु॑र्मि॒त्रस्य॒ वरु॑णस्या॒ग्नेः॥५१॥

%2.8.7.4
आप्रा॒ द्यावा॑पृथि॒वी अ॒न्तरि॑क्षम्।
सूर्य॑ आ॒त्मा जग॑तस्त॒स्थुष॑श्च।
त्वष्टा॒ दध॒त्तन्न॑स्तु॒रीपम्᳚।
त्वष्टा॑ वी॒रं पि॒शङ्ग॑रूपः।
दशे॒मन्त्वष्टु॑र्जनयन्त॒ गर्भम्᳚।
अत॑न्द्रासो युव॒तयो॒ बिभ॑र्त्रम्।
ति॒ग्मानी॑क॒ꣴ॒ स्वय॑शसं॒ जने॑षु।
वि॒रोच॑मानं॒ परि॑षीन्नयन्ति।
आविष्ट्यो॑ वर्धते॒ चारु॑रासु।
जि॒ह्माना॑मू॒र्ध्वस्वय॑शा उ॒पस्थे᳚॥५२॥

%2.8.7.5
उ॒भे त्वष्टु॑र्बिभ्यतु॒र्जाय॑मानात्।
प्र॒तीची॑ सि॒ꣳ॒हं प्रति॑\-जोषयेते।
मि॒त्रो जना॒न्प्र स मि॑त्र।
अ॒यं मि॒त्रो न॑म॒स्यः॑ सु॒शेवः॑।
राजा॑ सुक्ष॒त्रो अ॑जनिष्ट वे॒धाः।
तस्य॑ व॒यꣳ सु॑म॒तौ य॒ज्ञिय॑स्य।
अपि॑ भ॒द्रे सौ॑मन॒से स्या॑म।
अ॒न॒मी॒वास॒ इड॑या॒ मद॑न्तः।
मि॒तज्म॑वो॒ वरि॑म॒न्ना पृ॑थि॒व्याः।
आ॒दि॒त्यस्य॑ व्र॒तमु॑प॒क्ष्यन्तः॑॥५३॥

%2.8.7.6
व॒यं मि॒त्रस्य॑ सुम॒तौ स्या॑म।
मि॒त्रं न ईꣳ शिम्या॒ गोषु॑ ग॒व्यव॑त्।
स्वा॒धियो॑ वि॒दथे॑ अ॒फ्स्वजी॑जनन्।
अरे॑जयता॒ꣳ॒ रोद॑सी॒ पाज॑सा गि॒रा।
प्रति॑ प्रि॒यं य॑ज॒तं ज॒नुषा॒मवः॑।
म॒हाꣳ आ॑दि॒त्यो नम॑सोप॒सद्यः॑।
या॒त॒यज्ज॑नो गृण॒ते सु॒शेवः॑।
तस्मा॑ ए॒तत्पन्य॑तमाय॒ जुष्टम्᳚।
अ॒ग्नौ मि॒त्राय॑ ह॒विरा जु॑होत।
आ वा॒ꣳ॒ रथो॒ रोद॑सी बद्बधा॒नः॥५४॥

%2.8.7.7
हि॒र॒ण्ययो॒ वृष॑भिर्या॒त्वश्वैः᳚।
घृ॒तव॑र्तनिः प॒विभी॑रुचा॒नः।
इ॒षां वो॒ढा नृ॒पति॑र्वा॒जिनी॑वान्।
स प॑प्रथा॒नो अ॒भि पञ्च॒ भूम॑।
त्रि॒व॒न्धु॒रो मन॒साऽऽया॑तु यु॒क्तः।
विशो॒ येन॒ गच्छ॑थो देव॒यन्तीः᳚।
कुत्रा॑ चि॒द्याम॑मश्विना॒ दधा॑ना।
स्वश्वा॑ य॒शसा\-ऽऽया॑तम॒र्वाक्।
दस्रा॑ नि॒धिं मधु॑मन्तं पिबाथः।
वि वा॒ꣳ॒ रथो॑ व॒ध्वा॑ याद॑मानः॥५५॥

%2.8.7.8
अन्ता᳚न्दि॒वो बा॑धते वर्त॒निभ्या᳚म्।
यु॒वोः श्रियं॒ परि॒ योषा॑वृणीत।
सूरो॑ दुहि॒ता परि॑तक्मियायाम्।
यद्दे॑व॒यन्त॒मव॑थः॒ शची॑भिः।
परि॑घ्र॒ꣳ॒ सवां॒ मना॑वां॒ वयो॑गाम्।
यो ह॒स्यवाꣳ॑ रथिरा॒वस्त॑ उ॒स्राः।
रथो॑ युजा॒नः प॑रि॒याति॑ व॒र्तिः।
तेन॑ नः॒ शं योरु॒षसो॒ व्यु॑ष्टौ।
न्य॑श्विना वहतं य॒ज्ञे अ॒स्मिन्।
यु॒वं भु॒ज्युमव॑विद्धꣳ समु॒द्रे॥५६॥

%2.8.7.9
उदू॑हथु॒रर्ण॑सो॒ अस्रि॑धानैः।
प॒त॒त्रिभि॑रश्र॒मैर॑व्य॒थिभिः॑।
द॒ꣳ॒सना॑भिरश्विना पा॒रय॑न्ता।
अग्नी॑षोमा॒ यो अ॒द्य वा᳚म्।
इ॒दं वचः॑ सप॒र्यति॑।
तस्मै॑ धत्तꣳ सु॒वीर्यम्᳚।
गवां॒ पोष॒ꣴ॒ स्वश्वि॑यम्।
यो अ॒ग्नीषोमा॑ ह॒विषा॑ सप॒र्यात्।
दे॒व॒द्रीचा॒ मन॑सा॒ यो घृ॒तेन॑।
तस्य॑ व्र॒तꣳ र॑क्षतं पा॒तमꣳह॑सः॥५७॥

%2.8.7.10
वि॒शे जना॑य॒ महि॒ शर्म॑ यच्छतम्।
अग्नी॑षोमा॒ य आहु॑तिम्।
यो वां॒ दाशा᳚द्ध॒विष्कृ॑तिम्।
स प्र॒जया॑ सु॒वीर्यम्᳚।
विश्व॒मायु॒र्व्य॑श्ञवत्।
अग्नी॑षोमा॒ चेति॒ तद्वी॒र्यं॑ वाम्।
यदमु॑ष्णीतमव॒सं प॒णिङ्गोः।
अवा॑तिरतं॒ प्रथ॑यस्य॒ शेषः॑।
अवि॑न्दतं॒ ज्योति॒रेकं॑ ब॒हुभ्यः॑।
अग्नी॑षोमावि॒मꣳ सु मेऽग्नी॑षोमा ह॒विषः॒ प्रस्थि॑तस्य॥५८॥\anuvakamend[ज॒भा॒र॒ द्यौर॒ग्नेरु॒पस्थ॑ उप॒क्ष्यन्तो॑ बद्बधा॒नो व॒ध्वा॑ याद॑मानः समु॒द्रे\-ऽꣳह॑सः॒ प्रस्थि॑तस्य]

%2.8.8.1
अ॒हम॑स्मि प्रथम॒जा ऋ॒तस्य॑।
पूर्वं॑ दे॒वेभ्यो॑ अ॒मृत॑स्य॒ नाभिः॑।
यो मा॒ ददा॑ति॒ स इदे॒व माऽऽवाः᳚।
अ॒हमन्न॒मन्न॑\-म॒दन्त॑\-मद्मि।
पूर्व॑म॒ग्नेरपि॑ दह॒त्यन्नम्᳚।
य॒त्तौ हा॑साते अहमुत्त॒रेषु॑।
व्यात्त॑मस्य प॒शवः॑ सु॒जम्भम्᳚।
पश्य॑न्ति॒ धीराः॒ प्रच॑रन्ति॒ पाकाः᳚।
जहा᳚म्य॒न्यन्न ज॑हाम्य॒न्यम्।
अ॒हमन्नं॒ वश॒मिच्च॑रामि॥५९॥

%2.8.8.2
स॒मा॒नमर्थं॒ पर्ये॑मि भु॒ञ्जत्।
को मामन्नं॑ मनु॒ष्यो॑ दयेत।
परा॑के॒ अन्नं॒ निहि॑तं लो॒क ए॒तत्।
विश्वै᳚र्दे॒वैः पि॒तृभि॑र्गु॒प्तमन्नम्᳚।
यद॒द्यते॑ लु॒प्यते॒ यत्प॑रो॒प्यते᳚।
श॒त॒त॒मी सा त॒नूर्मे॑ बभूव।
म॒हान्तौ॑ च॒रू स॑कृद्दु॒ग्धेन॑ पप्रौ।
दिवं॑ च॒ पृश्ञि॑ पृथि॒वीं च॑ सा॒कम्।
तथ्स॒म्पिब॑न्तो॒ न मि॑नन्ति वे॒धसः॑।
नैतद्भूयो॒ भव॑ति॒ नो कनी॑यः॥६०॥

%2.8.8.3
अन्नं॑ प्रा॒णमन्न॑मपा॒नमा॑हुः।
अन्नं॑ मृ॒त्युं तमु॑ जी॒वातु॑माहुः।
अन्नं॑ ब्र॒ह्माणो॑ ज॒रसं॑  वदन्ति।
अन्न॑माहुः प्र॒जन॑नं प्र॒जाना᳚म्।
मोघ॒मन्नं॑ विन्दते॒ अप्र॑चेताः।
स॒त्यं ब्र॑वीमि व॒ध इथ्स तस्य॑।
नार्य॒मणं॒ पुष्य॑ति॒ नो सखा॑यम्।
केव॑लाघो भवति केवला॒दी।
अ॒हं मे॒घः स्त॒नय॒न्वर्\mbox{}ष॑न्नस्मि।
माम॑दन्त्य॒हम॑द्म्य॒न्यान्॥६१॥

%2.8.8.4
अ॒हꣳ सद॒मृतो॑ भवामि।
मदा॑दि॒त्या अधि॒ सर्वे॑ तपन्ति।
दे॒वीं वाच॑मजनयन्त॒ यद्वाग्वद॑न्ती।
अ॒न॒न्तामन्ता॒दधि॒ निर्मि॑तां म॒हीम्।
यस्यां᳚ दे॒वा अ॑दधु॒र्भोज॑नानि।
एका᳚क्षरां द्वि॒पदा॒ꣳ॒ षट्प॑दां च।
वाचं॑ दे॒वा उप॑ जीवन्ति॒ विश्वे᳚।
वाचं॑ दे॒वा उप॑ जीवन्ति॒ विश्वे᳚।
वाचं॑ गन्ध॒र्वाः प॒शवो॑ मनु॒ष्याः᳚।
वा॒चीमा विश्वा॒ भुव॑ना॒न्यर्पि॑ता॥६२॥

%2.8.8.5
सा नो॒ हवं॑ जुषता॒मिन्द्र॑पत्नी।
वाग॒क्षरं॑ प्रथम॒जा ऋ॒तस्य॑।
वेदा॑नां मा॒ता\-ऽमृत॑स्य॒ नाभिः॑।
सा नो॑ जुषा॒णोप॑ य॒ज्ञमागा᳚त्।
अव॑न्ती दे॒वी सु॒हवा॑ मे अस्तु।
यामृष॑यो मन्त्र॒कृतो॑ मनी॒षिणः॑।
अ॒न्वैच्छं॑ दे॒वास्तप॑सा॒ श्रमे॑ण।
तान्दे॒वीं वाचꣳ॑ ह॒विषा॑ यजामहे।
सा नो॑ दधातु सुकृ॒तस्य॑ लो॒के।
च॒त्वारि॒ वाक्परि॑मिता प॒दानि॑॥६३॥

%2.8.8.6
तानि॑ विदुर्ब्राह्म॒णा ये म॑नी॒षिणः॑।
गुहा॒ त्रीणि॒ निहि॑ता॒ नेङ्ग॑यन्ति।
तु॒रीयं॑ वा॒चो म॑नु॒ष्या॑ वदन्ति।
श्र॒द्धया॒\-ऽग्निः समि॑ध्यते।
श्र॒द्धया॑ विन्दते ह॒विः।
श्र॒द्धां भग॑स्य मू॒र्धनि॑।
वच॒सा वे॑दयामसि।
प्रि॒यꣴ श्र॑द्धे॒ दद॑तः।
प्रि॒यꣴ श्र॑द्धे॒ दिदा॑सतः।
प्रि॒यं भो॒जेषु॒ यज्व॑सु॥६४॥

%2.8.8.7
इ॒दं म॑ उदि॒तं कृ॑धि।
यथा॑ दे॒वा असु॑रेषु।
श्र॒द्धामु॒ग्रेषु॑ चक्रि॒रे।
ए॒वं भो॒जेषु॒ यज्व॑सु।
अ॒स्माक॑मुदि॒तं कृ॑धि।
श्र॒द्धां दे॑वा॒ यज॑मानाः।
वा॒युगो॑पा॒ उपा॑सते।
श्र॒द्धाꣳ हृ॑द॒य्य॑या\-ऽऽकू᳚त्या।
श्र॒द्धया॑ हूयते ह॒विः।
श्र॒द्धां प्रा॒तर्\mbox{}ह॑वामहे॥६५॥

%2.8.8.8
श्र॒द्धां म॒ध्यन्दि॑नं॒ परि॑।
श्र॒द्धाꣳ सूर्य॑स्य नि॒म्रुचि॑।
श्रद्धे॒ श्रद्धा॑पये॒ह मा᳚।
श्र॒द्धा दे॒वानधि॑ वस्ते।
श्र॒द्धा विश्व॑मि॒दं जग॑त्।
श्र॒द्धां काम॑स्य मा॒तरम्᳚।
ह॒विषा॑ वर्धयामसि।
ब्रह्म॑ जज्ञा॒नं प्र॑थ॒मं पु॒रस्ता᳚त्।
वि सी॑म॒तः सु॒रुचो॑ वे॒न आ॑वः।
स बु॒ध्निया॑ उप॒ मा अ॑स्य वि॒ष्ठाः॥६६॥

%2.8.8.9
स॒तश्च॒ योनि॒मस॑तश्च॒ विवः॑।
पि॒ता वि॒राजा॑मृष॒भो र॑यी॒णाम्।
अ॒न्तरि॑क्षं वि॒श्वरू॑प॒ आवि॑वेश।
तम॒र्कैर॒भ्य॑र्चन्ति व॒थ्सम्।
ब्रह्म॒ सन्तं॒ ब्रह्म॑णा व॒र्धय॑न्तः।
ब्रह्म॑ दे॒वान॑जनयत्।
ब्रह्म॒ विश्व॑मि॒दं जग॑त्।
ब्रह्म॑णः क्ष॒त्रं निर्मि॑तम्।
ब्रह्म॑ ब्राह्म॒ण आ॒त्मना᳚।
अ॒न्तर॑स्मिन्नि॒मे लो॒काः॥६७॥

%2.8.8.10
अ॒न्तर्विश्व॑मि॒दं जग॑त्।
ब्रह्मै॒व भू॒तानां॒ ज्येष्ठम्᳚।
तेन॒ को॑\-ऽर्\mbox{}हति॒ स्पर्धि॑तुम्।
ब्रह्म॑न्दे॒वास्त्रय॑स्त्रिꣳशत्।
ब्रह्म॑न्निन्द्रप्रजाप॒ती।
ब्रह्म॑न् ह॒ विश्वा॑ भू॒तानि॑।
ना॒वीवा॒न्तः स॒माहि॑ता।
चत॑स्र॒ आशाः॒ प्रच॑रन्त्व॒ग्नयः॑।
इ॒मं नो॑ य॒ज्ञं न॑यतु प्रजा॒नन्।
घृ॒तं पिन्व॑न्न॒जरꣳ॑ सु॒वीरम्᳚॥६८॥

%2.8.8.12
ब्रह्म॑ स॒मिद्भ॑व॒त्याहु॑तीनाम्।
आ गावो॑ अग्मन्नु॒त भ॒द्रम॑क्रन्।
सीद॑न्तु गो॒ष्ठे र॒णय॑न्त्व॒स्मे।
प्र॒जाव॑तीः पुरु॒रूपा॑ इ॒ह स्युः।
इन्द्रा॑य पू॒र्वीरु॒षसो॒ दुहा॑नाः।
इन्द्रो॒ यज्व॑ने पृण॒ते च॑ शिक्षति।
उपेद्द॑दाति॒ न स्वं मु॑षायति।
भूयो॑भूयो र॒यिमिद॑स्य व॒र्धयन्॑।
अभि॑न्ने खि॒ल्ले नि द॑धाति देव॒युम्।
न ता न॑शन्ति॒ न ता अर्वा᳚॥६९॥

%2.8.8.13
गावो॒ भगो॒ गाव॒ इन्द्रो॑ मे अच्छात्।
गावः॒ सोम॑स्य प्रथ॒मस्य॑ भ॒क्षः।
इ॒मा या गावः॒ सज॑नास॒ इन्द्रः॑।
इ॒च्छामीद्धृ॒दा मन॑सा चि॒दिन्द्रम्᳚।
यू॒यं गा॑वो मेदयथा कृ॒शं चि॑त्।
अ॒श्ली॒लं चि॑त्कृणुथा सु॒प्रती॑कम्।
भ॒द्रं गृ॒हं कृ॑णुथ भद्रवाचः।
बृ॒हद्वो॒ वय॑ उच्यते स॒भासु॑।
प्र॒जाव॑तीः सू॒यव॑सꣳ रि॒शन्तीः᳚।
शु॒द्धा अ॒पः सु॑प्रपा॒णे पिब॑न्तीः।
मा वः॑ स्ते॒न ई॑शत॒ माऽघशꣳ॑सः।
परि॑ वो हे॒ती रु॒द्रस्य॑ वृञ्ज्यात्।
उपे॒दमु॑प॒पर्च॑नम्।
आ॒सु गोषूप॑पृच्यताम्।
उप॑र्\mbox{}ष॒भस्य॒ रेत॑सि।
उपे᳚न्द्र॒ तव॑ वी॒र्ये᳚॥७०॥\anuvakamend[च॒रा॒मि॒ कनी॑यो॒\-ऽन्यानर्पि॑ता प॒दानि॒ यज्व॑सु हवामहे वि॒ष्ठा लो॒काः सु॒वीर॒मर्वा॒ पिब॑न्तीः॒ षट्च॑]

%2.8.9.1
ता सू᳚र्याचन्द्र॒मसा॑ विश्व॒भृत्त॑मा म॒हत्।
तेजो॒ वसु॑मद्राजतो दि॒वि।
सामा᳚त्माना चरतः सामचा॒रिणा᳚।
ययो᳚र्व्र॒तं न म॒मे जातु॑ दे॒वयोः᳚।
उ॒भावन्तौ॒ परि॑ यात॒ अर्म्या᳚।
दि॒वो न र॒श्मीꣴस्त॑नु॒तो व्य॑र्ण॒वे।
उ॒भा भु॑व॒न्ती भुव॑ना क॒विक्र॑तू।
सूर्या॒ न च॒न्द्रा च॑रतो ह॒ताम॑ती।
पती᳚ द्यु॒मद्वि॑श्व॒विदा॑ उ॒भा दि॒वः।
सूर्या॑ उ॒भा च॒न्द्रम॑सा विचक्ष॒णा॥७१॥

%2.8.9.2
वि॒श्ववा॑रा वरिवो॒भा वरे᳚ण्या।
ता नो॑ऽवतं मति॒मन्ता॒ महि॑व्रता।
वि॒श्व॒वप॑री प्र॒तर॑णा तर॒न्ता।
सु॒व॒र्विदा॑ दृ॒शये॒ भूरि॑रश्मी।
सूर्या॒ हि च॒न्द्रा वसु॑ त्वे॒षद॑र्\mbox{}शता।
म॒न॒स्विनो॒भानु॑चर॒तोनु॒ सन्दिवम्᳚।
अ॒स्य श्रवो॑ न॒द्यः॑ स॒प्त बि॑भ्रति।
द्यावा॒ क्षामा॑ पृथि॒वी द॑र्\mbox{}श॒तं वपुः॑।
अ॒स्मे सू᳚र्याचन्द्र॒मसा॑\-ऽभि॒चक्षे᳚।
श्र॒द्धेकमि॑न्द्र चरतो विचर्तु॒रम्॥७२॥

%2.8.9.3
पू॒र्वा॒प॒रं च॑रतो मा॒ययै॒तौ।
शिशू॒ क्रीड॑न्तौ॒ परि॑ यातो अध्व॒रम्।
विश्वा᳚न्य॒न्यो भुव॑नाऽभि॒ चष्टे᳚।
ऋ॒तून॒न्यो वि॒दध॑ज्जायते॒ पुनः॑।
हिर॑ण्यवर्णाः॒ शुच॑यः पाव॒का यासा॒ꣳ॒ राजा᳚।
यासां᳚ दे॒वाः शि॒वेन॑ मा॒ चक्षु॑षा पश्यत।
आपो॑ भ॒द्रा आदित्प॑श्यामि।
नास॑दासी॒न्नो सदा॑सीत्त॒दानी᳚म्।
नासी॒द्रजो॒ नो व्यो॑मा प॒रो यत्।
किमाव॑रीवः॒ कुह॒ कस्य॒ शर्मन्॑॥७३॥

%2.8.9.4
अम्भः॒ किमा॑सी॒द्गह॑नं गभी॒रम्।
न मृ॒त्युर॒मृतं॒ तर्\mbox{}हि॒ न।
रात्रि॑या॒ अह्न॑ आसीत्प्रके॒तः।
आनी॑दवा॒तꣴ स्व॒धया॒ तदेकम्᳚।
तस्मा᳚द्धा॒न्यं न प॒रः किञ्च॒नास॑।
तम॑ आसी॒त्तम॑सा गू॒ढमग्रे᳚ प्रके॒तम्।
स॒लि॒लꣳ सर्व॑मा इ॒दम्।
तु॒च्छेना॒भ्वपि॑हितं॒ यदासी᳚त्।
तम॑स॒स्तन्म॑हि॒ना जा॑य॒तैकम्᳚।
काम॒स्तदग्रे॒ सम॑वर्त॒ताधि॑॥७४॥

%2.8.9.5
मन॑सो॒ रेतः॑ प्रथ॒मं यदासी᳚त्।
स॒तो बन्धु॒मस॑ति॒ निर॑विन्दन्।
हृ॒दि प्र॒तीष्या॑ क॒वयो॑ मनी॒षा।
ति॒र॒श्चीनो॒ वित॑तो र॒श्मिरे॑षाम्।
अ॒धः स्वि॑दा॒सी(३)दु॒परि॑ स्विदासी(३)त्।
रे॒तो॒धा आ॑सन्महि॒मान॑ आसन्।
स्व॒धा अ॒वस्ता॒त्प्रय॑तिः प॒रस्ता᳚त्।
को अ॒द्धा वे॑द॒ क इ॒ह प्र वो॑चत्।
कुत॒ आजा॑ता॒ कुत॑ इ॒यं विसृ॑ष्टिः।
अ॒र्वाग्दे॒वा अ॒स्य वि॒सर्ज॑नाय॥७५॥

%2.8.9.6
अथा॒ को वे॑द॒ यत॑ आब॒भूव॑।
इ॒यं विसृ॑ष्टि॒र्यत॑ आब॒भूव॑।
यदि॑ वा द॒धे यदि॑ वा॒ न।
यो अ॒स्याध्य॑क्षः पर॒मे व्यो॑मन्।
सो अ॒ङ्ग वे॑द॒ यदि॑ वा॒ न वेद॑।
किꣴस्वि॒द्वन॒ङ्क उ॒ स वृ॒क्ष आ॑सीत्।
यतो॒ द्यावा॑पृथि॒वी नि॑ष्टत॒क्षुः।
मनी॑षिणो॒ मन॑सा पृ॒च्छतेदु॒तत्।
यद॒ध्यति॑ष्ठ॒द्भुव॑नानि धा॒रयन्॑।
ब्रह्म॒ वनं॒ ब्रह्म॒ स वृ॒क्ष आ॑सीत्॥७६॥

%2.8.9.7
यतो॒ द्यावा॑पृथि॒वी नि॑ष्टत॒क्षुः।
मनी॑षिणो॒ मन॑सा॒ विब्र॑वीमि वः।
ब्रह्मा॒ध्यति॑ष्ठ॒द्भुव॑नानि धा॒रयन्॑।
प्रा॒तर॒ग्निं प्रा॒तरिन्द्रꣳ॑ हवामहे।
प्रा॒तर्मि॒त्रावरु॑णा प्रा॒तर॒श्विना᳚।
प्रा॒तर्भगं॑ पू॒षणं॒ ब्रह्म॑ण॒स्पतिम्᳚।
प्रा॒तः सोम॑मु॒त रु॒द्रꣳ हु॑वेम।
प्रा॒त॒र्जितं॒ भग॑मु॒ग्रꣳ हु॑वेम।
व॒यं पु॒त्रमदि॑ते॒र्यो वि॑ध॒र्ता।
आ॒ध्रश्चि॒द्यं मन्य॑मानस्तु॒रश्चि॑त्॥७७॥

%2.8.9.8
राजा॑ चि॒द्यं भगं॑ भ॒क्षीत्याह॑।
भग॒ प्रणे॑त॒र्भग॒ सत्य॑राधः।
भगे॒मां धिय॒मुद॑व॒ दद॑न्नः।
भग॒ प्र णो॑ जनय॒ गोभि॒रश्वैः᳚।
भग॒ प्र नृभि॑र्नृ॒वन्तः॑ स्याम।
उ॒तेदानीं॒ भग॑वन्तः स्याम।
उ॒त प्रपि॒त्व उ॒त मध्ये॒ अह्ना᳚म्।
उ॒तोदि॑ता मघव॒न्थ्सूर्य॑स्य।
व॒यं दे॒वानाꣳ॑ सुम॒तौ स्या॑म।
भग॑ ए॒व भग॑वाꣳ अस्तु देवाः॥७८॥

%2.8.9.9
तेन॑ व॒यं भग॑वन्तः स्याम।
तं त्वा॑ भग॒ सर्व॒ इज्जो॑हवीमि।
स नो॑ भग पुर ए॒ता भ॑वे॒ह।
सम॑ध्व॒रायो॒षसो॑ नमन्त।
द॒धि॒क्रावे॑व॒ शुच॑ये प॒दाय॑।
अ॒र्वा॒ची॒नं व॑सु॒विदं॒ भगं॑ नः।
रथ॑मि॒वाश्वा॑ वा॒जिन॒ आव॑हन्तु।
अश्वा॑वती॒र्गोम॑तीर्न उ॒षासः॑।
वी॒रव॑तीः॒ सद॑मुच्छन्तु भ॒द्राः।
घृ॒तं दुहा॑ना वि॒श्वतः॒ प्रपी॑नाः।
यू॒यं पा॑त स्व॒स्तिभिः॒ सदा॑ नः॥७९॥\anuvakamend[वि॒च॒क्ष॒णा वि॑चर्तु॒रꣳ शर्म॒न्नधि॑ वि॒सर्ज॑नाय॒ ब्रह्म॒ वनं॒ ब्रह्म॒ स वृ॒क्ष आ॑सीत्तु॒रश्चि॑द्देवाः॒ प्रपी॑ना॒ एकं॑ च]
\prashnaend{पीवो᳚न्ना॒न्ते शु॒क्रासः॒ सोमो॑ धे॒नुमिन्द्र॒स्तर॑स्वा॒ञ्छुचि॒मा दे॒वो या॑तु॒ सूर्यो॑ दे॒वीम॒हम॑स्मि॒ ता सू᳚र्याचन्द्र॒मसा॒ नव॑॥९॥}{पीवो᳚न्ना॒मग्ने॒ त्वं पा॑रयानाधृ॒ष्यः शुचिं॒ नु वि॒श्रय॑माणो दि॒वो रु॒क्मो\-ऽन्नं॑ प्रा॒णमन्न॒न्ता सू᳚र्याचन्द्र॒मसा॒ नव॑सप्ततिः॥७९॥}{पीवो᳚न्नां यू॒यं पा॑त स्व॒स्तिभिः॒ सदा॑ नः॥}{हरिः॑ ओम्॥}{इति श्रीकृष्णयजुर्वेदीयतैत्तिरीयब्राह्मणे द्वितीयाष्टके अष्टमः प्रपाठकः समाप्तः॥}
\clearpage
%%% END ASHTAKAM

\chapt{अष्टकम् ३}
\sect{प्रथमः प्रश्नः}
\setcounter{anuvakam}{0}
\dnsub{तैत्तिरीयब्राह्मणे तृतीयाष्टके प्रथमः प्रपाठकः}

%3.1.1.1
अ॒ग्निर्नः॑ पातु॒ कृत्ति॑काः।
नक्ष॑त्रं दे॒वमि॑न्द्रि॒यम्।
इ॒दमा॑सां विचक्ष॒णम्।
ह॒विरा॒सं जु॑होतन।
यस्य॒ भान्ति॑ र॒श्मयो॒ यस्य॑ के॒तवः॑।
यस्ये॒मा विश्वा॒ भुव॑नानि॒ सर्वा᳚।
स कृत्ति॑काभि\-र॒भिसं॒वसा॑नः।
अ॒ग्निर्नो॑ दे॒वः सु॑वि॒ते द॑धातु।
प्र॒जा\-प॑ते रोहि॒णी वे॑तु॒ पत्नी᳚।
वि॒श्वरू॑पा बृह॒ती चि॒त्रभा॑नुः॥१॥

%3.1.1.2
सा नो॑ य॒ज्ञस्य॑ सुवि॒ते द॑धातु।
यथा॒ जीवे॑म श॒रदः॒ सवी॑राः।
रो॒हि॒णी दे॒व्युद॑गात्पु॒रस्ता᳚त्।
विश्वा॑ रू॒पाणि॑ प्रति॒मोद॑माना।
प्र॒जा\-प॑तिꣳ ह॒विषा॑ व॒र्धय॑न्ती।
प्रि॒या दे॒वाना॒मुप॑यातु य॒ज्ञम्।
सोमो॒ राजा॑ मृगशी॒र्॒‌षेण॒ आगन्॑।
शि॒वं नक्ष॑त्रं प्रि॒यम॑स्य॒ धाम॑।
आ॒प्याय॑मानो बहु॒धा जने॑षु।
रेतः॑ प्र॒जां यज॑माने दधातु॥२॥

%3.1.1.3
यत्ते॒ नक्ष॑त्रं मृगशी॒र्॒‌षमस्ति॑।
प्रि॒यꣳ रा॑जन् प्रि॒यत॑मं प्रि॒याणा᳚म्।
तस्मै॑ ते सोम ह॒विषा॑ विधेम।
शं न॑ एधि द्वि॒पदे॒ शं चतु॑ष्पदे।
आ॒र्द्रया॑ रु॒द्रः प्रथ॑मा न एति।
श्रेष्ठो॑ दे॒वानां॒ पति॑रघ्नि॒याना᳚म्।
नक्ष॑त्रमस्य ह॒विषा॑ विधेम।
मा नः॑ प्र॒जाꣳ री॑रिष॒न्मोत वी॒रान्।
हे॒ती रु॒द्रस्य॒ परि॑ णो वृणक्तु।
आ॒र्द्रा नक्ष॑त्रं जुषताꣳ ह॒विर्नः॑॥३॥

%3.1.1.4
प्र॒मु॒ञ्चमा॑नौ दुरि॒तानि॒ विश्वा᳚।
अपा॒घशꣳ॑ सन्नुदता॒मरा॑तिम्।
पुन॑र्नो दे॒व्यदि॑तिः स्पृणोतु।
पुन॑र्वसू नः॒ पुन॒रेतां᳚ य॒ज्ञम्।
पुन॑र्नो दे॒वा अ॒भिय॑न्तु॒ सर्वे᳚।
पुनः॑ पुनर्वो ह॒विषा॑ यजामः।
ए॒वा न दे॒व्यदि॑तिरन॒र्वा।
विश्व॑स्य भ॒र्त्री जग॑तः प्रति॒ष्ठा।
पुन॑र्वसू ह॒विषा॑ व॒र्धय॑न्ती।
प्रि॒यं दे॒वाना॒मप्ये॑तु॒ पाथः॑॥४॥

%3.1.1.5
बृह॒स्पतिः॑ प्रथ॒मं जाय॑मानः।
ति॒ष्यं॑ नक्ष॑त्रम॒भि सम्ब॑भूव।
श्रेष्ठो॑ दे॒वानां॒ पृत॑नासु जि॒ष्णुः।
दि॒शोऽनु॒ सर्वा॒ अभ॑यं नो अस्तु।
ति॒ष्यः॑ पु॒रस्ता॑दु॒त म॑ध्य॒तो नः॑।
बृह॒स्पति॑र्नः॒ परि॑ पातु प॒श्चात्।
बाधे॑तां॒ द्वेषो॒ अभ॑यं कृणुताम्।
सु॒वीर्य॑स्य॒ पत॑यः स्याम।
इ॒दꣳ स॒र्पेभ्यो॑ ह॒विर॑स्तु॒ जुष्टम्᳚।
आ॒श्रे॒षा येषा॑मनु॒यन्ति॒ चेतः॑॥५॥

%3.1.1.6
ये अ॒न्तरि॑क्षं पृथि॒वीं क्षि॒यन्ति॑।
ते नः॑ स॒र्पासो॒ हव॒माग॑मिष्ठाः।
ये रो॑च॒ने सूर्य॒स्यापि॑ स॒र्पाः।
ये दिवं॑ दे॒वीमनु॑ स॒ञ्चर॑न्ति।
येषा॑माश्रे॒षा अ॑नु॒यन्ति॒ कामम्᳚।
तेभ्यः॑ स॒र्पेभ्यो॒ मधु॑मज्जुहोमि।
उप॑हूताः पि॒तरो॒ ये म॒घासु॑।
मनो॑जवसः सु॒कृतः॑ सुकृ॒त्याः।
ते नो॒ नक्ष॑त्रे॒ हव॒माग॑मिष्ठाः।
स्व॒धाभि॑र्य॒ज्ञं प्रय॑तं जुषन्ताम्॥६॥

%3.1.1.7
ये अ॑ग्निद॒ग्धा येऽन॑ग्निदग्धाः।
ये॑ऽमुं लो॒कं पि॒तरः॑ क्षि॒यन्ति॑।
याꣴश्च॑ वि॒द्म याꣳ उ॑ च॒ न प्र॑वि॒द्म।
म॒घासु॑ य॒ज्ञꣳ सुकृ॑तं जुषन्ताम्।
गवां॒ पतिः॒ फल्गु॑नीनामसि॒ त्वम्।
तद॑र्यमन्वरुणमित्र॒ चारु॑।
तं त्वा॑ व॒यꣳ स॑नि॒तारꣳ॑ सनी॒नाम्।
जी॒वा जीव॑न्त॒मुप॒ संवि॑शेम।
येने॒मा विश्वा॒ भुव॑नानि॒ सञ्जि॑ता।
यस्य॑ दे॒वा अ॑नु सं॒ यन्ति॒ चेतः॑॥७॥

%3.1.1.8
अ॒र्य॒मा राजा॒\-ऽजर॒स्तुवि॑ष्मान्।
फल्गु॑नीनामृष॒भो रो॑रवीति।
श्रेष्ठो॑ दे॒वानां᳚ भगवो भगासि।
तत्त्वा॑ विदुः॒ फल्गु॑नी॒स्तस्य॑ वित्तात्।
अ॒स्मभ्यं॑ क्ष॒त्रम॒जरꣳ॑ सु॒वीर्यम्᳚।
गोम॒दश्व॑व॒दुप॒ सन्नु॑\-दे॒ह।
भगो॑ ह दा॒ता भग॒ इत्प्र॑दा॒ता।
भगो॑ दे॒वीः फल्गु॑नी॒रा वि॑वेश।
भग॒स्येत्तं प्र॑स॒वं ग॑मेम।
यत्र॑ दे॒वैः स॑ध॒मादं॑ मदेम॥८॥

%3.1.1.9
आया॑तु दे॒वः स॑वि॒तोप॑यातु।
हि॒र॒ण्यये॑न सु॒वृता॒ रथे॑न।
वह॒न्॒ हस्तꣳ॑ सु॒भगं॑ विद्म॒नाप॑सम्।
प्र॒यच्छ॑न्तं॒ पपु॑रिं॒ पुण्य॒मच्छ॑।
हस्तः॒ प्रय॑च्छत्व॒मृतं॒ वसी॑यः।
दक्षि॑णेन॒ प्रति॑\-गृभ्णीम एनत्।
दा॒तार॑म॒द्य स॑वि॒ता वि॑देय।
यो नो॒ हस्ता॑य प्रसु॒वाति॑ य॒ज्ञम्।
त्वष्टा॒ नक्ष॑त्रम॒भ्ये॑ति चि॒त्राम्।
सु॒भꣳ स॑सं युव॒तिꣳ रोच॑मानाम्॥९॥

%3.1.1.10
नि॒वे॒शय॑न्न॒\-मृता॒न्मर्त्याꣴ॑श्च।
रू॒पाणि॑ पि॒ꣳ॒शन् भुव॑नानि॒ विश्वा᳚।
तन्न॒स्त्वष्टा॒ तदु॑ चि॒त्रा विच॑ष्टाम्।
तन्नक्ष॑त्रं भूरि॒दा अ॑स्तु॒ मह्यम्᳚।
तन्नः॑ प्र॒जां वी॒रव॑तीꣳ सनोतु।
गोभि॑र्नो॒ अश्वैः॒ सम॑नक्तु य॒ज्ञम्।
वा॒युर्नक्ष॑त्रम॒भ्ये॑ति॒ निष्ट्या᳚म्।
ति॒ग्मशृ॑ङ्गो वृष॒भो रोरु॑वाणः।
स॒मी॒रय॒न् भुव॑ना मात॒रिश्वा᳚।
अप॒ द्वेषाꣳ॑सि नुदता॒मरा॑तीः॥१०॥

%3.1.1.11
तन्नो॑ वा॒युस्तदु॒ निष्ट्या॑ शृणोतु।
तन्नक्ष॑त्रं भूरि॒दा अ॑स्तु॒ मह्यम्᳚।
तन्नो॑ दे॒वासो॒ अनु॑जानन्तु॒ कामम्᳚।
यथा॒ तरे॑म दुरि॒तानि॒ विश्वा᳚।
दू॒रम॒स्मच्छत्र॑वो यन्तु भी॒ताः।
तदि॑न्द्रा॒ग्नी कृ॑णुतां॒ तद्विशा॑खे।
तन्नो॑ दे॒वा अनु॑मदन्तु य॒ज्ञम्।
प॒श्चात् पु॒रस्ता॒दभ॑यं नो अस्तु।
नक्ष॑त्राणा॒मधि॑पत्नी॒ विशा॑खे।
श्रेष्ठा॑विन्द्रा॒ग्नी भुव॑नस्य गो॒पौ॥११॥

%3.1.1.12
विषू॑चः॒ शत्रू॑नप॒ बाध॑मानौ।
अप॒ क्षुधं॑ नुदता॒मरा॑तिम्।
पू॒र्णा प॒श्चादु॒त पू॒र्णा पु॒रस्ता᳚त्।
उन्म॑ध्य॒तः पौ᳚र्णमा॒सी जि॑गाय।
तस्यां᳚ दे॒वा अधि॑ सं॒वस॑न्तः।
उ॒त्त॒मे नाक॑ इ॒ह मा॑दयन्ताम्।
पृ॒थ्वी सु॒वर्चा॑ युव॒तिः स॒जोषाः᳚।
पौ॒र्ण॒मा॒स्युद॑गा॒च्छोभ॑माना।
आ॒प्या॒यय॑न्ती दुरि॒तानि॒ विश्वा᳚।
उ॒रुं दुहां॒ यज॑मानाय य॒ज्ञम्॥१२॥\anuvakamend[चि॒त्रभा॑नु॒र्यज॑माने दधातु ह॒विर्नः॒ पाथ॒श्चेतो॑ जुषन्ता॒ञ्चेतो॑ मदेम॒ रोच॑माना॒मरा॑तीर्गो॒पौ य॒ज्ञम्]

%3.1.2.1
ऋ॒द्ध्यास्म॑ ह॒व्यैर्नम॑सोप॒सद्य॑।
मि॒त्रं दे॒वं मि॑त्र॒धेयं॑ नो अस्तु।
अ॒नू॒रा॒धान् ह॒विषा॑ व॒र्धय॑न्तः।
श॒तं जी॑वेम श॒रदः॒ सवी॑राः।
चि॒त्रं नक्ष॑त्र॒मुद॑गात्पु॒रस्ता᳚त्।
अ॒नू॒रा॒धास॒ इति॒ यद्वद॑न्ति।
तन्मि॒त्र ए॑ति प॒थिभि॑र्देव॒यानैः᳚।
हि॒र॒ण्ययै॒र्वित॑तै\-र॒न्तरि॑क्षे।
इन्द्रो᳚ ज्ये॒ष्ठामनु॒ नक्ष॑त्रमेति।
यस्मि॑न्वृ॒त्रं वृ॑त्र॒तूर्ये॑ त॒तार॑॥१३॥

%3.1.2.2
तस्मि॑न्व॒यम॒मृतं॒ दुहा॑नाः।
क्षुधं॑ तरेम॒ दुरि॑तिं॒ दुरि॑ष्टिम्।
पु॒र॒न्द॒राय॑ वृष॒भाय॑ धृ॒ष्णवे᳚।
अषा॑ढाय॒ सह॑मानाय मी॒ढुषे᳚।
इन्द्रा॑य ज्ये॒ष्ठा मधु॑म॒द्दुहा॑ना।
उ॒रुं कृ॑णोतु॒ यज॑मानाय लो॒कम्।
मूलं॑ प्र॒जां वी॒रव॑तीं विदेय।
परा᳚च्येतु॒ निर्\mbox{}ऋ॑तिः परा॒चा।
गोभि॒र्नक्ष॑त्रं प॒शुभिः॒ सम॑क्तम्।
अह॑र्भूया॒द्यज॑मानाय॒ मह्यम्᳚॥१४॥

%3.1.2.3
अह॑र्नो अ॒द्य सु॑वि॒ते द॑धातु।
मूलं॒ नक्ष॑त्र॒मिति॒ यद्वद॑न्ति।
परा॑चीं वा॒चा निर्\mbox{}ऋ॑तिं नुदामि।
शि॒वं प्र॒जायै॑ शि॒वम॑स्तु॒ मह्यम्᳚।
या दि॒व्या आपः॒ पय॑सा सम्बभू॒वुः।
या अ॒न्तरि॑क्ष उ॒त पार्थि॑वी॒र्याः।
यासा॑मषा॒ढा अ॑नु॒यन्ति॒ कामम्᳚।
ता न॒ आपः॒ शꣴ स्यो॒ना भ॑वन्तु।
याश्च॒ कूप्या॒ याश्च॑ ना॒द्याः᳚ समु॒द्रियाः᳚।
याश्च॑ वैश॒न्तीरु॒त प्रा॑स॒चीर्याः॥१५॥

%3.1.2.4
यासा॑मषा॒ढा मधु॑ भ॒क्षय॑न्ति।
ता न॒ आपः॒ शꣴ स्यो॒ना भ॑वन्तु।
तन्नो॒ विश्वे॒ उप॑ शृण्वन्तु दे॒वाः।
तद॑षा॒ढा अ॒भिसंय॑न्तु य॒ज्ञम्।
तन्नक्ष॑त्रं प्रथतां प॒शुभ्यः॑।
कृ॒षिर्वृ॒ष्टिर्यज॑मानाय कल्पताम्।
शु॒भ्राः क॒न्या॑ युव॒तयः॑ सु॒पेश॑सः।
क॒र्म॒कृतः॑ सु॒कृतो॑ वी॒र्या॑वतीः।
विश्वा᳚न् दे॒वान् ह॒विषा॑ व॒र्धय॑न्तीः।
अ॒षा॒ढाः काम॒मुप॑ यान्तु य॒ज्ञम्॥१६॥

%3.1.2.5
यस्मि॒न् ब्रह्मा॒\-ऽभ्यज॑य॒थ्सर्व॑मे॒तत्।
अ॒मुं च॑ लो॒कमि॒दमू॑ च॒ सर्वम्᳚।
तन्नो॒ नक्ष॑त्रमभि॒जिद्वि॒जित्य॑।
श्रियं॑ दधा॒त्वहृ॑णीय\-मानम्।
उ॒भौ लो॒कौ ब्रह्म॑णा॒ सञ्जि॑ते॒मौ।
तन्नो॒ नक्ष॑त्रमभि॒जिद्विच॑ष्टाम्।
तस्मि॑न्व॒यं पृत॑नाः॒ सञ्ज॑येम।
तन्नो॑ दे॒वासो॒ अनु॑जानन्तु॒ कामम्᳚।
शृ॒ण्वन्ति॑ श्रो॒णाम॒मृत॑स्य गो॒पाम्।
पुण्या॑मस्या॒ उप॑शृणोमि॒ वाचम्᳚॥१७॥


%3.1.2.6
म॒हीं दे॒वीं विष्णु॑पत्नीमजू॒र्याम्।
प्र॒तीची॑मेनाꣳ ह॒विषा॑ यजामः।
त्रे॒धा विष्णु॑रुरुगा॒यो विच॑क्रमे।
म॒हीं दिवं॑ पृथि॒वीम॒न्तरि॑क्षम्।
तच्छ्रो॒णैति॒श्रव॑ इ॒च्छमा॑ना।
पुण्य॒ꣴ॒ श्लोकं॒ यज॑मानाय कृण्व॒ती।
अ॒ष्टौ दे॒वा वस॑वः सो॒म्यासः॑।
चत॑स्रो दे॒वीर॒जराः॒ श्रवि॑ष्ठाः।
ते य॒ज्ञं पा᳚न्तु॒ रज॑सः प॒रस्ता᳚त्।
सं॒व॒थ्स॒रीण॑म॒मृतꣴ॑ स्व॒स्ति॥१८॥

%3.1.2.7
य॒ज्ञं नः॑ पान्तु॒ वस॑वः पु॒रस्ता᳚त्।
द॒क्षि॒ण॒तो॑\-ऽभिय॑न्तु॒ श्रवि॑ष्ठाः।
पुण्यं॒ नक्ष॑त्रम॒भि संवि॑शाम।
मा नो॒ अरा॑तिर॒घश॒ꣳ॒सा\-ऽगन्॑।
क्ष॒त्रस्य॒ राजा॒ वरु॑णो\-ऽधिरा॒जः।
नक्ष॑त्राणाꣳ श॒तभि॑ष॒ग्वसि॑ष्ठः।
तौ दे॒वेभ्यः॑ कृणुतो दी॒र्घमायुः॑।
श॒तꣳ स॒हस्रा॑ भेष॒जानि॑ धत्तः।
य॒ज्ञं नो॒ राजा॒ वरु॑ण॒ उप॑यातु।
तन्नो॒ विश्वे॑ अ॒भि संय॑न्तु दे॒वाः॥१९॥

%3.1.2.8
तन्नो॒ नक्ष॑त्रꣳ श॒तभि॑षग्जुषा॒णम्।
दी॒र्घमायुः॒ प्रति॑\-रद्भेष॒जानि॑।
अ॒ज एक॑पा॒दुद॑गात्पु॒र\-स्ता᳚त्।
विश्वा॑ भू॒तानि॑ प्रति॒ मोद॑मानः।
तस्य॑ दे॒वाः प्र॑स॒वं य॑न्ति॒ सर्वे᳚।
प्रो॒ष्ठ॒प॒दासो॑ अ॒मृत॑स्य गो॒पाः।
वि॒भ्राज॑मानः समिधा॒न उ॒ग्रः।
आऽन्तरि॑क्षमरुह॒दग॒न्द्याम्।
तꣳ सूर्यं॑ दे॒वम॒जमेक॑पादम्।
प्रो॒ष्ठ॒प॒दासो॒ अनु॑यन्ति॒ सर्वे᳚॥२०॥

%3.1.2.9
अहि॑र्बु॒ध्नियः॒ प्रथ॑मान एति।
श्रेष्ठो॑ दे॒वाना॑मु॒त मानु॑षाणाम्।
तं ब्रा᳚ह्म॒णाः सो॑म॒पाः सो॒म्यासः॑।
प्रो॒ष्ठ॒प॒दासो॑ अ॒भि र॑क्षन्ति॒ सर्वे᳚।
च॒त्वार॒ एक॑म॒भि कर्म॑ दे॒वाः।
प्रो॒ष्ठ॒प॒दास॒ इति॒ यान् वद॑न्ति।
ते बु॒ध्नियं॑ परि॒षद्यꣴ॑ स्तु॒वन्तः॑।
अहिꣳ॑ रक्षन्ति॒ नम॑सोप॒सद्य॑।
पू॒षा रे॒वत्यन्वे॑ति॒ पन्था᳚म्।
पु॒ष्टि॒पती॑ पशु॒पा वाज॑बस्त्यौ॥२१॥

%3.1.2.10
इ॒मानि॑ ह॒व्या प्रय॑ता जुषा॒णा।
सु॒गैर्नो॒ यानै॒रुप॑यातां य॒ज्ञम्।
क्षु॒द्रान् प॒शून् र॑क्षतु रे॒वती॑ नः।
गावो॑ नो॒ अश्वा॒ꣳ॒ अन्वे॑तु पू॒षा।
अन्न॒ꣳ॒ रक्ष॑न्तौ बहु॒धा विरू॑पम्।
वाजꣳ॑ सनुतां॒ यज॑मानाय य॒ज्ञम्।
तद॒श्विना॑वश्व॒युजोप॑याताम्।
शुभ॒ङ्गमि॑ष्ठौ सु॒यमे॑भि॒रश्वैः᳚।
स्वं नक्ष॑त्रꣳ ह॒विषा॒ यज॑न्तौ।
मध्वा॒ सम्पृ॑क्तौ॒ यजु॑षा॒ सम॑क्तौ॥२२॥

%3.1.2.11
यौ दे॒वानां᳚ भि॒षजौ॑ हव्यवा॒हौ।
विश्व॑स्य दू॒ताव॒मृत॑स्य गो॒पौ।
तौ नक्ष॑त्रं जुजुषा॒णोप॑याताम्।
नमो॒ऽश्विभ्यां᳚ कृणुमो\-ऽश्व॒युग्भ्या᳚म्।
अप॑ पा॒प्मानं॒ भर॑णीर्भरन्तु।
तद्य॒मो राजा॒ भग॑वा॒न्॒ विच॑ष्टाम्।
लो॒कस्य॒ राजा॑ मह॒तो म॒हान् हि।
सु॒गं नः॒ पन्था॒मभ॑यं कृणोतु।
यस्मि॒न्नक्ष॑त्रे य॒म एति॒ राजा᳚।
यस्मि॑न्नेनम॒भ्यषि॑ञ्चन्त दे॒वाः।
तद॑स्य चि॒त्रꣳ ह॒विषा॑ यजाम।
अप॑ पा॒प्मानं॒ भर॑णीर्भरन्तु।
नि॒वेश॑नी॒ यत्ते॑ दे॒वा अद॑धुः॥२३॥\anuvakamend[त॒तार॒ मह्यं॑ प्रास॒चीर्या या᳚न्तु य॒ज्ञं वाचꣴ॑ स्व॒स्ति दे॒वा अनु॑यन्ति॒ सर्वे॒ वाज॑बस्त्यौ॒ सम॑क्तौ दे॒वास्त्रीणि॑ च]

%3.1.3.1
नवो॑नवो भवति॒ जाय॑मानो॒ यमा॑दि॒त्या अ॒ꣳ॒शुमा᳚प्या॒यय॑न्ति।
ये विरू॑पे॒ सम॑नसा सं॒व्यय॑न्ती।
स॒मा॒नं तन्तुं॑ परितात॒ना ते᳚।
वि॒भू प्र॒भू अ॑नु॒भू वि॒श्वतो॑ हुवे।
ते नो॒ नक्ष॑त्रे॒ हव॒माग॑मेतम्।
व॒यं दे॒वी ब्रह्म॑णा संविदा॒नाः।
सु॒रत्ना॑सो दे॒ववी॑तिं॒ दधा॑नाः।
अ॒हो॒रा॒त्रे ह॒विषा॑ व॒र्धय॑न्तः।
अति॑ पा॒प्मान॒मति॑ मुक्त्या गमेम।
प्रत्यु॑वदृश्याय॒ती॥२४॥

%3.1.3.2
व्यु॒च्छन्ती॑ दुहि॒ता दि॒वः।
अ॒पो म॒ही वृ॑णुते॒ चक्षु॑षा।
तमो॒ ज्योति॑ष्कृणोति सू॒नरी᳚।
उदु॒स्रियाः᳚ सचते॒ सूर्यः॑।
सचा॑ उ॒द्यन्नक्ष॑त्रमर्चि॒मत्।
तवेदु॑षो॒ व्युषि॒ सूर्य॑स्य च।
सं भ॒क्तेन॑ गमेमहि।
तन्नो॒ नक्ष॑त्रमर्चि॒मत्।
भा॒नु॒मत्तेज॑ उ॒च्चर॑त्।
उप॑य॒ज्ञमि॒हाग॑मत्॥२५॥

%3.1.3.3
प्र नक्ष॑त्राय दे॒वाय॑।
इन्द्रा॒येन्दुꣳ॑ हवामहे।
स नः॑ सवि॒ता सु॑वथ्स॒निम्।
पु॒ष्टि॒दां वी॒रव॑त्तमम्।
उदु॒त्यं चि॒त्रम्।
अदि॑तिर्न उरुष्यतु म॒हीमू॒षु मा॒तरम्᳚।
इ॒दं विष्णुः॒ प्रतद्विष्णुः॑।
अ॒ग्निर्मू॒र्धा भुवः॑।
अनु॑नो॒\-ऽद्यानु॑मति॒रन्विद॑नुमते॒ त्वम्।
ह॒व्य॒वाह॒ꣴ॒ स्वि॑ष्टम्॥२६॥\anuvakamend[आ॒य॒त्य॑गम॒थ्स्वि॑ष्टम्]

%3.1.4.1
अ॒ग्निर्वा अ॑कामयत।
अ॒न्ना॒दो दे॒वानाꣴ॑स्या॒मिति॑।
स ए॒तम॒ग्नये॒ कृत्ति॑काभ्यः पुरो॒डाश॑म॒ष्टा\-क॑पालं॒ निर॑वपत्।
ततो॒ वै सो᳚\-ऽन्ना॒दो दे॒वाना॑मभवत्।
अ॒ग्निर्वै दे॒वाना॑मन्ना॒दः।
यथा॑ ह॒ वा अ॒ग्निर्दे॒वाना॑मन्ना॒दः।
ए॒वꣳ ह॒ वा ए॒ष म॑नु॒ष्या॑णां भवति।
य ए॒तेन॑ ह॒विषा॒ यज॑ते।
य उ॑ चैनदे॒वं वेद॑।
सोऽत्र॑ जुहोति।
अ॒ग्नये॒ स्वाहा॒ कृत्ति॑काभ्यः॒ स्वाहा᳚।
अ॒म्बायै॒ स्वाहा॑ दु॒लायै॒ स्वाहा᳚।
नि॒त॒त्न्यै स्वाहा॒\-ऽभ्रय॑न्त्यै॒ स्वाहा᳚।
मे॒घय॑न्त्यै॒ स्वाहा॑ व॒र्॒षय॑न्त्यै॒ स्वाहा᳚।
चु॒पु॒णीका॑यै॒ स्वाहेति॑॥२७॥

%3.1.4.2
प्र॒जा\-प॑तिः प्र॒जा अ॑\-सृजत।
ता अ॑स्माथ्सृ॒ष्टाः परा॑चीरायन्।
तासाꣳ॑ रोहि॒णीम॒भ्य॑ध्यायत्।
सो॑ऽकामयत।
उप॒ मा व॑र्तेत।
समे॑नया गच्छे॒येति॑।
स ए॒तं प्र॒जा\-प॑तये रोहि॒ण्यै च॒रुं निर॑वपत्।
ततो॒ वै सा तमु॒पाव॑र्तत।
समे॑नया गच्छत।
उप॑ ह॒ वा ए॑नं प्रि॒यमाव॑र्तते।
सं प्रि॒येण॑ गच्छते।
य ए॒तेन॑ ह॒विषा॒ यज॑ते।
य उ॑चैनदे॒वं वेद॑।
सोऽत्र॑ जुहोति।
प्र॒जा\-प॑तये॒ स्वाहा॑ रोहि॒ण्यै स्वाहा᳚।
रोच॑मानायै॒ स्वाहा᳚ प्र॒जाभ्यः॒ स्वाहेति॑॥२८॥

%3.1.4.3
सोमो॒ वा अ॑कामयत।
ओष॑धीनाꣳ रा॒ज्यम॒भिज॑येय॒मिति॑।
स ए॒तꣳ सोमा॑य मृगशी॒र्॒षाय॑ श्यामा॒कं च॒रुं पय॑सि॒ निर॑वपत्।
ततो॒ वै स ओष॑धीनाꣳ रा॒ज्यम॒भ्य॑जयत्।
स॒मा॒नानाꣳ॑ ह॒ वै रा॒ज्यम॒भि\-ज॑यति।
य ए॒तेन॑ ह॒विषा॒ यज॑ते।
य उ॑ चैनदे॒वं वेद॑।
सोऽत्र॑ जुहोति।
सोमा॑य॒ स्वाहा॑ मृगशी॒र्‌॒\mbox{}षाय॒ स्वाहा᳚।
इ॒न्व॒काभ्यः॒ स्वाहौष॑धीभ्यः॒ स्वाहा᳚।
रा॒ज्याय॒ स्वाहा॒\-ऽभिजि॑त्यै॒ स्वाहेति॑॥२९॥

%3.1.4.4
रु॒द्रो वा अ॑कामयत।
प॒शु॒मान्थ्स्या॒मिति॑।
स ए॒तꣳ रु॒द्राया॒ऽऽर्द्रायै॒ प्रैय्य॑ङ्गवं च॒रुं पय॑सि॒ निर॑वपत्।
ततो॒ वै स प॑शु॒मान॑भवत्।
प॒शु॒मान् ह॒ वै भ॑वति।
य ए॒तेन॑ ह॒विषा॒ यज॑ते।
य उ॑ चैनदे॒वं वेद॑।
सोऽत्र॑ जुहोति।
रु॒द्राय॒ स्वाहा॒\-ऽऽर्द्रायै॒ स्वाहा᳚।
पिन्व॑मानायै॒ स्वाहा॑ प॒शुभ्यः॒ स्वाहेति॑॥३०॥

%3.1.4.5
ऋ॒क्षा वा इ॒यम॑लो॒मका॑\-ऽऽसीत्।
साऽका॑मयत।
ओष॑धीभि॒र्वन॒\-स्पति॑भिः॒ प्रजा॑ये॒येति॑।
सैतमदि॑त्यै॒ पुन॑र्वसुभ्यां च॒रुं निर॑वपत्।
ततो॒ वा इ॒यमोष॑धीभि॒र्वन॒स्पति॑भिः॒ प्राजा॑यत।
प्रजा॑यते ह॒ वै प्र॒जया॑ प॒शुभिः॑।
य ए॒तेन॑ ह॒विषा॒ यज॑ते।
य उ॑ चैनदे॒वं वेद॑।
सोऽत्र॑ जुहोति।
अदि॑त्यै॒ स्वाहा॒ पुन॑र्वसुभ्याम्।
स्वाहा भू᳚त्यै॒ स्वाहा॒ प्रजा᳚त्यै॒ स्वाहेति॑॥३१॥

%3.1.4.6
बृह॒स्पति॒र्वा अ॑कामयत।
ब्र॒ह्म॒व॒र्च॒सी स्या॒मिति॑।
स ए॒तं बृह॒स्पत॑ये ति॒ष्या॑य नैवा॒रं च॒रुं पय॑सि॒ निर॑वपत्।
ततो॒ वै स ब्र॑ह्म\-वर्च॒स्य॑भवत्।
ब्र॒ह्म॒व॒र्च॒सी ह॒ वै भ॑वति।
य ए॒तेन॑ ह॒विषा॒ यज॑ते।
य उ॑ चैनदे॒वं वेद॑।
सोऽत्र॑ जुहोति।
बृह॒स्पत॑ये॒ स्वाहा॑ ति॒ष्या॑य॒ स्वाहा᳚।
ब्र॒ह्म॒व॒र्च॒साय॒ स्वाहेति॑॥३२॥

%3.1.4.7
दे॒वा॒सु॒राः संय॑त्ता आसन्।
ते दे॒वाः स॒र्पेभ्य॑ आश्रे॒षाभ्य॒ आज्ये॑ कर॒म्भं निर॑वपन्।
ताने॒ताभि॑रे॒व दे॒वता॑भि॒रुपा॑नयन्।
ए॒ताभि॑र्ह॒ वै दे॒वता॑भिर्द्वि॒षन्तं॒ भ्रातृ॑व्य॒मुप॑नयति।
य ए॒तेन॑ ह॒विषा॒ यज॑ते।
य उ॑ चैनदे॒वं वेद॑।
सोऽत्र॑ जुहोति।
स॒र्पेभ्यः॒ स्वाहा᳚\-ऽऽश्रे॒षाभ्यः॒ स्वाहा᳚।
द॒न्द॒शूके᳚भ्यः॒ स्वाहेति॑॥३३॥

%3.1.4.8
पि॒तरो॒ वा अ॑कामयन्त।
पि॒तृ॒लो॒क ऋ॑ध्नुया॒मेति॑।
त ए॒तं पि॒तृभ्यो॑ म॒घाभ्यः॑ पुरो॒डाश॒ꣳ॒ षट्क॑पालं॒ निर॑वपन्।
ततो॒ वै ते पि॑तृलो॒क आ᳚र्ध्नुवन्।
पि॒तृ॒लो॒के ह॒ वा ऋ॑ध्नोति।
य ए॒तेन॑ ह॒विषा॒ यज॑ते।
य उ॑ चैनदे॒वं वेद॑।
सोऽत्र॑ जुहोति।
पि॒तृभ्यः॒ स्वाहा॑ म॒घाभ्यः॑।
स्वाहा॑\-ऽन॒घाभ्यः॒ स्वाहा॑ऽग॒दाभ्यः॑।
स्वाहा॑\-ऽरुन्ध॒तीभ्यः॒ स्वाहेति॑॥३४॥

%3.1.4.9
अ॒र्य॒मा वा अ॑कामयत।
प॒शु॒मान्थ्स्या॒मिति॑।
स ए॒तम॑र्य॒म्णे फल्गु॑नीभ्यां च॒रुं निर॑वपत्।
ततो॒ वै स प॑शु॒मान॑भवत्।
प॒शु॒मान् ह॒ वै भ॑वति।
य ए॒तेन॑ ह॒विषा॒ यज॑ते।
य उ॑ चैनदे॒वं वेद॑।
सोऽत्र॑ जुहोति।
अ॒र्य॒म्णे स्वाहा॒ फल्गु॑नीभ्या॒ꣴ॒ स्वाहा᳚।
प॒शुभ्यः॒ स्वाहेति॑॥३५॥

%3.1.4.10
भगो॒ वा अ॑कामयत।
भ॒गी श्रे॒ष्ठी दे॒वानाꣴ॑स्या॒मिति॑।
स ए॒तं भगा॑य॒ फल्गु॑नीभ्यां च॒रुं निर॑वपत्।
ततो॒ वै स भ॒गी श्रे॒ष्ठी दे॒वाना॑मभवत्।
भ॒गी ह॒ वै श्रे॒ष्ठी स॑मा॒नानां᳚ भवति।
य ए॒तेन॑ ह॒विषा॒ यज॑ते।
य उ॑ चैनदे॒वं वेद॑।
सोऽत्र॑ जुहोति।
भगा॑य॒ स्वाहा॒ फल्गु॑नीभ्या॒ꣴ॒ स्वाहा᳚।
श्रैष्ठ्या॑य॒ स्वाहेति॑॥३६॥

%3.1.4.11
स॒वि॒ता वा अ॑कामयत।
श्रन्मे॑ दे॒वा दधी॑रन्।
स॒वि॒ता स्या॒मिति॑।
स ए॒तꣳ स॑वि॒त्रे हस्ता॑य पुरो॒डाशं॒ द्वाद॑शकपालं॒ निर॑वपदाशू॒नां व्री॑ही॒णाम्।
ततो॒ वै तस्मै॒ श्रद्दे॒वा अद॑धत।
स॒वि॒ता\-ऽभ॑वत्।
श्रद्ध॒वा अ॑स्मै मनु॒ष्या॑ दधते।
स॒वि॒ता स॑मा॒नानां᳚ भवति।
य ए॒तेन॑ ह॒विषा॒ यज॑ते।
य उ॑ चैनदे॒वं वेद॑।
सोऽत्र॑ जुहोति।
स॒वि॒त्रे स्वाहा॒ हस्ता॑य।
स्वाहा॑ दद॒ते स्वाहा॑ पृण॒ते।
स्वाहा᳚ प्र॒यच्छ॑ते॒ स्वाहा᳚ प्रति\-गृभ्ण॒ते स्वाहेति॑॥३७॥

%3.1.4.12
त्वष्टा॒ वा अ॑कामयत।
चि॒त्रं प्र॒जां वि॑न्दे॒येति॑।
स ए॒तं त्वष्ट्रे॑ चि॒त्रायै॑ पुरो॒डाश॑म॒ष्टा\-क॑पालं॒ निर॑वपत्।
ततो॒ वै स चि॒त्रं प्र॒जाम॑विन्दत।
चि॒त्रꣳ ह॒ वै प्र॒जां वि॑न्दते।
य ए॒तेन॑ ह॒विषा॒ यज॑ते।
य उ॑ चैनदे॒वं वेद॑।
सोऽत्र॑ जुहोति।
त्वष्ट्रे॒ स्वाहा॑ चि॒त्रायै॒ स्वाहा᳚।
चैत्रा॑य॒ स्वाहा᳚ प्र॒जायै॒ स्वाहेति॑॥३८॥

%3.1.4.13
वा॒युर्वा अ॑कामयत।
का॒म॒चार॑मे॒षु लो॒केष्व॒भिज॑येय॒मिति॑।
स ए॒तद्वा॒यवे॒ निष्ट्या॑यै गृ॒ष्ट्यै दु॒ग्धं पयो॒ निर॑वपत्।
ततो॒ वै स का॑म॒चार॑मे॒षु लो॒केष्व॒भ्य॑जयत्।
का॒म॒चारꣳ॑ ह॒ वा ए॒षु लो॒केष्व॒भि\-ज॑यति।
य ए॒तेन॑ ह॒विषा॒ यज॑ते।
य उ॑ चैनदे॒वं वेद॑।
सोऽत्र॑ जुहोति।
वा॒यवे॒ स्वाहा॒ निष्ट्या॑यै॒ स्वाहा᳚।
का॒म॒चारा॑य॒ स्वाहा॒\-ऽभिजि॑त्यै॒ स्वाहेति॑॥३९॥

%3.1.4.14
इ॒न्द्रा॒ग्नी वा अ॑कामयेताम्।
श्रैष्ठ्यं॑ दे॒वाना॑म॒भिज॑ये॒वेति॑।
तावे॒तमि॑न्द्रा॒ग्निभ्यां॒ विशा॑खाभ्यां पुरो॒डाश॒मेका॑\-दश\-कपालं॒ निर॑वपताम्।
ततो॒ वै तौ श्रैष्ठ्यं॑ दे॒वाना॑म॒भ्य॑जयताम्।
श्रैष्ठ्यꣳ॑ ह॒ वै स॑मा॒नाना॑म॒भि ज॑यति।
य ए॒तेन॑ ह॒विषा॒ यज॑ते।
य उ॑ चैनदे॒वं वेद॑।
सोऽत्र॑ जुहोति।
इ॒न्द्रा॒ग्निभ्या॒ꣴ॒ स्वाहा॒ विशा॑खाभ्या॒ꣴ॒ स्वाहा᳚।
श्रैष्ठ्या॑य॒ स्वाहा॒\-ऽभिजि॑त्यै॒ स्वाहेति॑॥४०॥

%3.1.4.15
अथै॒तत्पौ᳚र्णमा॒स्या आज्यं॒ निर्व॑पति।
कामो॒ वै पौ᳚र्णमा॒सी।
काम॒ आज्यम्᳚।
कामे॑नै॒व काम॒ꣳ॒ सम॑र्धयति।
क्षि॒प्रमे॑न॒ꣳ॒ सकाम॒ उप॑नमति।
येन॒ कामे॑न॒ यज॑ते।
सोऽत्र॑ जुहोति।
पौ॒र्ण॒मा॒स्यै स्वाहा॒ कामा॑य॒ स्वाहा\-ऽऽग॑त्यै॒ स्वाहेति॑॥४१॥\anuvakamend[अ॒ग्निः पञ्च॑दश प्र॒जा\-प॑तिः॒ षोड॑श॒ सोम॒ एका॑दश रु॒द्रो दश॒र्क्षैका॑दश॒ बृह॒स्पति॒र्दश॑ देवासु॒रा नव॑ पि॒तर॒ एका॑दशार्य॒मा भगो॒ दश॑ दश सवि॒ता चतु॑र्दश॒ त्वष्टा॑ वा॒युरि॑न्द्रा॒ग्नी दश॑ द॒शाथै॒तत्पौ᳚र्णमा॒स्या अ॒ष्टौ पञ्च॑दश]

%3.1.5.1
मि॒त्रो वा अ॑कामयत।
मि॒त्र॒धेय॑मे॒षु लो॒केष्व॒भिज॑येय॒मिति॑।
स ए॒तं मि॒त्राया॑नूरा॒धेभ्य॑श्च॒रुं निर॑वपत्।
ततो॒ वै स मि॑त्र॒धेय॑मे॒षु लो॒केष्व॒भ्य॑जयत्।
मि॒त्र॒धेयꣳ॑ ह॒ वा ए॒षु लो॒केष्व॒भि\-ज॑यति।
य ए॒तेन॑ ह॒विषा॒ यज॑ते।
य उ॑ चैनदे॒वं वेद॑।
सोऽत्र॑ जुहोति।
मि॒त्राय॒ स्वाहा॑\-ऽनूरा॒धेभ्यः॒ स्वाहा᳚।
मि॒त्र॒धेया॑य॒ स्वाहा॒\-ऽभिजि॑त्यै॒ स्वाहेति॑॥४२॥

%3.1.5.2
इन्द्रो॒ वा अ॑कामयत।
ज्यैष्ठ्यं॑ दे॒वाना॑म॒भिज॑येय॒मिति॑।
स ए॒तमिन्द्रा॑य ज्ये॒ष्ठायै॑ पुरो॒डाश॒मेका॑\-दश\-कपालं॒ निर॑वपन्म॒हाव्री॑हीणाम्।
ततो॒ वै स ज्यैष्ठ्यं॑ दे॒वाना॑म॒भ्य॑जयत्।
ज्यैष्ठ्यꣳ॑ ह॒ वै स॑मा॒नाना॑म॒भि\-ज॑यति।
य ए॒तेन॑ ह॒विषा॒ यज॑ते।
य उ॑ चैनदे॒वं वेद॑।
सोऽत्र॑ जुहोति।
इन्द्रा॑य॒ स्वाहा᳚ ज्ये॒ष्ठायै॒ स्वाहा᳚।
ज्यैष्ठ्या॑य॒ स्वाहा॒ऽभिजि॑त्यै॒ स्वाहेति॑॥४३॥

%3.1.5.3
प्र॒जा\-प॑ति॒र्वा अ॑कामयत।
मूलं॑ प्र॒जां वि॑न्दे॒येति॑।
स ए॒तं प्र॒जा\-प॑तये॒ मूला॑य च॒रुं निर॑वपत्।
ततो॒ वै स मूलं॑ प्र॒जाम॑विन्दत।
मूलꣳ॑ ह॒ वै प्र॒जां वि॑न्दते।
य ए॒तेन॑ ह॒विषा॒ यज॑ते।
य उ॑ चैनदे॒वं वेद॑।
सोऽत्र॑ जुहोति।
प्र॒जा\-प॑तये॒ स्वाहा॒ मूला॑य॒ स्वाहा᳚।
प्र॒जायै॒ स्वाहेति॑॥४४॥

%3.1.5.4
आपो॒ वा अ॑कामयन्त।
स॒मु॒द्रं काम॑म॒भिज॑ये॒मेति॑।
ता ए॒तम॒द्भ्यो॑\-ऽषा॒ढाभ्य॑श्च॒रुं निर॑वपन्।
ततो॒ वै ताः स॑मु॒द्रं काम॑म॒भ्य॑जयन्।
स॒मु॒द्रꣳ ह॒ वै काम॑म॒भि\-ज॑यति।
य ए॒तेन॑ ह॒विषा॒ यज॑ते।
य उ॑ चैनदे॒वं वेद॑।
सोऽत्र॑ जुहोति।
अ॒द्भ्यः स्वाहा॑\-ऽषा॒ढाभ्यः॒ स्वाहा᳚।
स॒मु॒द्राय॒ स्वाहा॒ कामा॑य॒ स्वाहा᳚।
अ॒भिजि॑त्यै॒ स्वाहेति॑॥४५॥

%3.1.5.5
विश्वे॒ वै दे॒वा अ॑कामयन्त।
अ॒न॒प॒ज॒य्यं ज॑ये॒मेति॑।
त ए॒तं विश्वे᳚भ्यो दे॒वेभ्यो॑\-ऽषा॒ढाभ्य॑श्च॒रुं निर॑वपन्।
ततो॒ वै ते॑\-ऽनपज॒य्यम॑जयन्।
अ॒न॒प॒ज॒य्यꣳ ह॒ वै ज॑यति।
य ए॒तेन॑ ह॒विषा॒ यज॑ते।
य उ॑ चैनदे॒वं वेद॑।
सोऽत्र॑ जुहोति।
विश्वे᳚भ्यो दे॒वेभ्यः॒ स्वाहा॑\-ऽषा॒ढाभ्यः॒ स्वाहा᳚।
अ॒न॒प॒ज॒य्याय॒ स्वाहा॒ जित्यै॒ स्वाहेति॑॥४६॥

%3.1.5.6
ब्रह्म॒ वा अ॑कामयत।
ब्र॒ह्म॒लो॒कम॒भिज॑येय॒मिति॑।
तदे॒तं ब्रह्म॑णे\-ऽभि॒जिते॑ च॒रुं निर॑वपत्।
ततो॒ वै तद्ब्र॑ह्मलो॒कम॒भ्य॑जयत्।
ब्र॒ह्म॒लो॒कꣳ ह॒ वा अ॒भि\-ज॑यति।
य ए॒तेन॑ ह॒विषा॒ यज॑ते।
य उ॑ चैनदे॒वं वेद॑।
सोऽत्र॑ जुहोति।
ब्रह्म॑णे॒ स्वाहा॑\-ऽभि॒जिते॒ स्वाहा᳚।
ब्र॒ह्म॒लो॒काय॒ स्वाहा॒\-ऽभिजि॑त्यै॒ स्वाहेति॑॥४७॥

%3.1.5.7
विष्णु॒र्वा अ॑कामयत।
पुण्य॒ꣴ॒ श्लोकꣳ॑ शृण्वीय।
न मा॑ पा॒पी की॒र्तिराग॑च्छे॒दिति॑।
स ए॒तं विष्ण॑वे श्रो॒णायै॑ पुरो॒डाशं॑ त्रिकपा॒लं निर॑वपत्।
ततो॒ वै स पुण्य॒ꣴ॒ श्लोक॑मशृणुत।
नैनं॑ पा॒पी की॒र्तिराग॑च्छत्।
पुण्यꣳ॑ ह॒ वै श्लोकꣳ॑ शृणुते।
नैनं॑ पा॒पी की॒र्तिराग॑च्छति।
य ए॒तेन॑ ह॒विषा॒ यज॑ते।
य उ॑ चैनदे॒वं वेद॑।
सोऽत्र॑ जुहोति।
विष्ण॑वे॒ स्वाहा᳚ श्रो॒णायै॒ स्वाहा᳚।
श्लोका॑य॒ स्वाहा᳚ श्रु॒ताय॒ स्वाहेति॑॥४८॥

%3.1.5.8
वस॑वो॒ वा अ॑कामयन्त।
अग्रं॑ दे॒वता॑नां॒ परी॑या॒मेति॑।
त ए॒तं वसु॑भ्यः॒ श्रवि॑ष्ठाभ्यः पुरो॒डाश॑म॒ष्टा\-क॑पालं॒ निर॑वपन्।
ततो॒ वै तेऽग्रं॑ दे॒वता॑नां॒ पर्या॑यन्।
अग्रꣳ॑ ह॒ वै स॑मा॒नानां॒ पर्ये॑ति।
य ए॒तेन॑ ह॒विषा॒ यज॑ते।
य उ॑ चैनदे॒वं वेद॑।
सोऽत्र॑ जुहोति।
वसु॑भ्यः॒ स्वाहा॒ श्रवि॑ष्ठाभ्यः॒ स्वाहा᳚।
अग्रा॑य॒ स्वाहा॒ परी᳚त्यै॒ स्वाहेति॑॥४९॥

%3.1.5.9
इन्द्रो॒ वा अ॑कामयत।
दृ॒ढो\-ऽशि॑थिलः स्या॒मिति॑।
स ए॒तं वरु॑णाय श॒तभि॑षजे भेष॒जेभ्यः॑ पुरो॒डाशं॒ दश॑कपालं॒ निर॑वपत्कृ॒ष्णानां᳚ व्रीही॒णाम्।
ततो॒ वै स दृ॒ढो\-ऽशि॑थिलो\-ऽभवत्।
दृ॒ढो ह॒ वा अशि॑थिलो भवति।
य ए॒तेन॑ ह॒विषा॒ यज॑ते।
य उ॑ चैनदे॒वं वेद॑।
सोऽत्र॑ जुहोति।
वरु॑णाय॒ स्वाहा॑ श॒तभि॑षजे॒ स्वाहा᳚।
भे॒ष॒जेभ्यः॒ स्वाहेति॑॥५०॥

%3.1.5.10
अ॒जो वा एक॑पादकामयत।
ते॒ज॒स्वी ब्र॑ह्म\-वर्च॒सी स्या॒मिति॑।
स ए॒तम॒जायैक॑पदे प्रोष्ठप॒देभ्य॑श्च॒रुं निर॑वपत्।
ततो॒ वै स ते॑ज॒स्वी ब्र॑ह्म\-वर्च॒स्य॑भवत्।
ते॒ज॒स्वी ह॒ वै ब्र॑ह्म\-वर्च॒सी भ॑वति।
य ए॒तेन॑ ह॒विषा॒ यजते।
य उ॑ चैनदे॒वं वेद॑।
सोऽत्र॑ जुहोति।
अ॒जायैक॑पदे॒ स्वाहा᳚ प्रोष्ठप॒देभ्यः॒ स्वाहा᳚।
तेज॑से॒ स्वाहा᳚ ब्रह्मवर्च॒साय॒ स्वाहेति॑॥५१॥

%3.1.5.11
अहि॒र्वै बु॒ध्नियो॑\-ऽकामयत।
इ॒मां प्र॑ति॒ष्ठां वि॑न्दे॒येति॑।
स ए॒तमह॑ये बु॒ध्निया॑य प्रोष्ठप॒देभ्यः॑ पुरो॒डाशं॒ भूमि॑कपालं॒ निर॑वपत्।
ततो॒ वै स इ॒मां प्र॑ति॒ष्ठाम॑विन्दत।
इ॒माꣳ ह॒ वै प्र॑ति॒ष्ठां वि॑न्दते।
य ए॒तेन॑ ह॒विषा॒ यज॑ते।
य उ॑ चैनदे॒वं वेद॑।
सोऽत्र॑ जुहोति।
अह॑ये बु॒ध्निया॑य॒ स्वाहा᳚ प्रोष्ठप॒देभ्यः॒ स्वाहा᳚।
प्र॒ति॒ष्ठायै॒ स्वाहेति॑॥५२॥

%3.1.5.12
पू॒षा वा अ॑कामयत।
प॒शु॒मान्थ्स्या॒मिति॑।
स ए॒तं पू॒ष्णे रे॒वत्यै॑ च॒रुं निर॑वपत्।
ततो॒ वै स प॑शु॒मान॑भवत्।
प॒शु॒मान् ह॒ वै भ॑वति।
य ए॒तेन॑ ह॒विषा॒ यज॑ते।
य उ॑ चैनदे॒वं वेद॑।
सोऽत्र॑ जुहोति।
पू॒ष्णे स्वाहा॑ रे॒वत्यै॒ स्वाहा᳚।
प॒शुभ्यः॒ स्वाहेति॑॥५३॥

%3.1.5.13
अ॒श्विनौ॒ वा अ॑कामयेताम्।
श्रो॒त्र॒स्विना॒वब॑धिरौ स्या॒वेति॑।
तावे॒तम॒श्विभ्या॑मश्व॒युग्भ्यां᳚ पुरो॒डाशं॑ द्विकपा॒लं निर॑वपताम्।
ततो॒ वै तौ श्रो᳚त्र॒स्विना॒वब॑धिरावभवताम्।
श्रो॒त्र॒स्वी ह॒ वा अब॑धिरो भवति।
य ए॒तेन॑ ह॒विषा॒ यज॑ते।
य उ॑ चैनदे॒वं वेद॑।
सोऽत्र॑ जुहोति।
अ॒श्विभ्या॒ꣴ॒ स्वाहा᳚\-ऽश्व॒युग्भ्या॒ꣴ॒ स्वाहा᳚।
श्रोत्रा॑य॒ स्वाहा॒ श्रुत्यै॒ स्वाहेति॑॥५४॥

%3.1.5.14
य॒मो वा अ॑कामयत।
पि॒तृ॒णाꣳ रा॒ज्यम॒भिज॑येय॒मिति॑।
स ए॒तं य॒माया॑प॒भर॑णीभ्यश्च॒रुं निर॑पवत्।
ततो॒ वै स पि॑तृ॒णाꣳ रा॒ज्य\-म॒भ्य॑जयत्।
स॒मा॒नानाꣳ॑ ह॒ वै रा॒ज्यम॒भि ज॑यति।
य ए॒तेन॑ ह॒विषा॒ यज॑ते।
य उ॑ चैनदे॒वं वेद॑।
सोऽत्र॑ जुहोति।
य॒माय॒ स्वाहा॑\-ऽप॒भर॑णीभ्यः॒ स्वाहा᳚।
रा॒ज्याय॒ स्वाहा॒ऽभिजि॑त्यै॒ स्वाहेति॑॥५५॥

%3.1.5.15
अथै॒तद॑मावा॒स्या॑या॒ आज्यं॒ निर्व॑पति।
कामो॒ वा अ॑मावा॒स्या᳚।
काम॒ आज्यम्᳚।
कामे॑नै॒व काम॒ꣳ॒ सम॑र्धयति।
क्षि॒प्रमे॑न॒ꣳ॒ सकाम॒ उप॑नमति।
येन॒ कामे॑न॒ यज॑ते।
सोऽत्र॑ जुहोति।
अ॒मा॒वा॒स्या॑यै॒ स्वाहा॒ कामा॑य॒ स्वाहा\-ऽऽग॑त्यै॒ स्वाहेति॑॥५६॥\anuvakamend[मि॒त्र इन्द्रः॑ प्र॒जा\-प॑ति॒र्दश॑ द॒शाप॒ एका॑दश॒ विश्वे॒ ब्रह्म॒ दश॑दश॒ विष्णु॒स्त्रयो॑दश॒ वस॑व॒ इन्द्रो॒\-ऽजो\-ऽहि॒र्वै बु॒ध्नियः॑ पू॒षा\-ऽश्विनौ॑ य॒मो दश॑ द॒शाथै॒तद॑मावा॒स्या॑या अ॒ष्टौ पञ्च॑दश]

%3.1.6.1
च॒न्द्रमा॒ वा अ॑कामयत।
अ॒हो॒रा॒त्रान॑र्धमा॒सान्मासा॑नृ॒तून्थ्सं॑\-वथ्स॒रमा॒प्त्वा।
च॒न्द्रम॑सः॒ सायु॑ज्यꣳ सलो॒कता॑माप्नुया॒मिति॑।
स ए॒तं च॒न्द्रम॑से प्रती॒दृश्या॑यै पुरो॒डाशं॒ पञ्च॑\-दश\-कपालं॒ निर॑वपत्।
ततो॒ वै सो॑\-ऽहोरा॒त्रान॑र्ध\-मा॒सान्मासा॑\-नृ॒तून्थ्सं॑वथ्स॒र\-मा॒प्त्वा।
च॒न्द्रम॑सः॒ सायु॑ज्यꣳ सलो॒कता॑माप्नोत्।
अ॒हो॒रा॒त्रान् ह॒ वा अ॑र्ध\-मा॒सान्मासा॑\-नृ॒तून्थ्सं॑वथ्स॒र\-मा॒प्त्वा।
च॒न्द्रम॑सः॒ सायु॑ज्यꣳ सलो॒कता॑\-माप्नोति।
य ए॒तेन॑ ह॒विषा॒ यज॑ते।
य उ॑ चैनदे॒वं वेद॑।
सोऽत्र॑ जुहोति।
च॒न्द्रम॑से॒ स्वाहा᳚ प्रती॒दृश्या॑यै॒ स्वाहा᳚।
अ॒हो॒रा॒त्रेभ्यः॒ स्वाहा᳚\-ऽर्धमा॒सेभ्यः॒ स्वाहा᳚।
मासे᳚भ्यः॒ स्वाह॒र्तुभ्यः॒ स्वाहा᳚।
सं॒व॒थ्स॒राय॒ स्वाहेति॑॥५७॥

%3.1.6.2
अ॒हो॒रा॒त्रे वा अ॑कामयेताम्।
अत्य॑होरा॒त्रे मु॑च्येवहि।
न ना॑वहोरा॒त्रे आ᳚प्नुयाता॒मिति॑।
ते ए॒तम॑होरा॒त्राभ्यां᳚ च॒रुं निर॑वपताम्।
द्व॒यानां᳚ व्रीही॒णाम्।
शु॒क्लानां᳚ च कृ॒ष्णानां᳚ च।
स॒वा॒त्योर्दु॒ग्धे।
श्वे॒तायै॑ च कृ॒ष्णायै॑ च।
ततो॒ वै ते अत्य॑होरा॒त्रे अ॑मुच्येते।
नैने॑ अहोरा॒त्रे आ᳚प्नुताम्।
अति॑ ह॒ वा अ॑होरा॒त्रे मु॑च्यते।
नैन॑महोरा॒त्रे आ᳚प्नुतः।
य ए॒तेन॑ ह॒विषा॒ यज॑ते।
य उ॑ चैनदे॒वं वेद॑।
सोऽत्र॑ जुहोति।
अह्ने॒ स्वाहा॒ रात्रि॑यै॒ स्वाहा᳚।
अति॑मुक्त्यै॒ स्वाहेति॑॥५८॥

%3.1.6.3
उ॒षा वा अ॑कामयत।
प्रि॒या\-ऽऽदि॒त्यस्य॑ सु॒भगा᳚ स्या॒मिति॑।
सैतमु॒षसे॑ च॒रुं निर॑वपत्।
ततो॒ वै सा प्रि॒या\-ऽऽदि॒त्यस्य॑ सु॒भगा॑\-ऽभवत्।
प्रि॒यो ह॒ वै स॑मा॒नानाꣳ॑ सु॒भगो॑ भवति।
य ए॒तेन॑ ह॒विषा॒ यज॑ते।
य उ॑ चैनदे॒वं वेद॑।
सोऽत्र॑ जुहोति।
उ॒षसे॒ स्वाहा॒ व्यु॑ष्ट्यै॒ स्वाहा᳚।
व्यू॒षुष्यै॒ स्वाहा᳚ व्यु॒च्छन्त्यै॒ स्वाहा᳚।
व्यु॑ष्टायै॒ स्वाहेति॑॥५९॥

%3.1.6.4
अथै॒तस्मै॒ नक्ष॑त्राय च॒रुं निर्व॑पति।
यथा॒ त्वं दे॒वाना॒मसि॑।
ए॒वम॒हं म॑नु॒ष्या॑णां भूयास॒मिति॑।
यथा॑ ह॒ वा ए॒तद्दे॒वाना᳚म्।
ए॒वꣳ ह॒ वा ए॒ष म॑नु॒ष्या॑णां भवति।
य ए॒तेन॑ ह॒विषा॒ यज॑ते।
य उ॑ चैनदे॒वं वेद॑।
सोऽत्र॑ जुहोति।
नक्ष॑त्राय॒ स्वाहो॑देष्य॒ते स्वाहा᳚।
उ॒द्य॒ते स्वाहोदि॑ताय॒ स्वाहा᳚।
हर॑से॒ स्वाहा॒ भर॑से॒ स्वाहा᳚।
भ्राज॑से॒ स्वाहा॒ तेज॑से॒ स्वाहा᳚।
तप॑से॒ स्वाहा᳚ ब्रह्मवर्च॒साय॒ स्वाहेति॑॥६०॥

%3.1.6.5
सूर्यो॒ वा अ॑कामयत।
नक्ष॑त्राणां प्रति॒ष्ठा स्या॒मिति॑।
स ए॒तꣳ सूर्या॑य॒ नक्ष॑त्रेभ्यश्च॒रुं निर॑वपत्।
ततो॒ वै स नक्ष॑त्राणां प्रति॒ष्ठा\-ऽभ॑वत्।
प्र॒ति॒ष्ठा ह॒ वै स॑मा॒नानां᳚ भवति।
य ए॒तेन॑ ह॒विषा॒ यज॑ते।
य उ॑ चैनदे॒वं वेद॑।
सोऽत्र॑ जुहोति।
सूर्या॑य॒ स्वाहा॒ नक्ष॑त्रेभ्यः॒ स्वाहा᳚।
प्र॒ति॒ष्ठायै॒ स्वाहेति॑॥६१॥

%3.1.6.6
अथै॒तमदि॑त्यै च॒रुं निर्व॑पति।
इ॒यं वा अदि॑तिः।
अ॒स्यामे॒व प्रति॑ तिष्ठति।
सोऽत्र॑ जुहोति।
अदि॑त्यै॒ स्वाहा᳚ प्रति॒ष्ठायै॒ स्वाहेति॑॥६२॥

%3.1.6.7
अथै॒तं विष्ण॑वे च॒रुं निर्व॑पति।
य॒ज्ञो वै विष्णुः॑।
य॒ज्ञ ए॒वान्त॒तः प्रति॑ तिष्ठति।
सोऽत्र॑ जुहोति।
विष्ण॑वे॒ स्वाहा॑ य॒ज्ञाय॒ स्वाहा᳚।
प्र॒ति॒ष्ठायै॒ स्वाहेति॑॥६३॥\anuvakamend[च॒न्द्रमाः॒ पञ्च॑दशाहोरा॒त्रे स॒प्तद॑शो॒षा एका॑द॒शाथै॒तस्मै॒ नक्ष॑त्राय॒ त्रयो॑दश॒ सूर्यो॒ दशाथै॒तमदि॑त्यै॒ पञ्चाथै॒तं विष्ण॑वे॒ षट्थ्स॒प्त (स॒वि॒ता\-ऽऽशू॒नां व्री॑ही॒णामिन्द्रो॑ म॒हाव्री॑हीणा॒मिन्द्रः॑ कृ॒ष्णानां᳚ व्रीही॒णाम॑होरा॒त्रे द्व॒यानां᳚ व्रीही॒णाम्।
पि॒तरः॒ षट्क॑पालꣳ सवि॒ता द्वाद॑शकपालमिन्द्रा॒ग्नी एका॑\-दश\-कपाल॒मिन्द्र॒ एका॑\-दश\-कपाल॒मिन्द्रो॒ दश॑कपालं॒ विष्णु॑स्त्रिकपा॒लमहि॒र्भूमि॑कपालम॒श्विनौ᳚ द्विकपा॒लं च॒न्द्रमाः॒ पञ्च॑\-दश\-कपालम॒ग्निस्त्वष्टा॒ वस॑वो॒\-ऽष्टा\-क॑पालम॒न्यत्र॑ च॒रुम्।
रु॒द्रो᳚\-ऽर्य॒मा पू॒षा प॑शु॒मान्थ्स्या॒ꣳ॒ सोमो॑ रु॒द्रो बृह॒स्पतिः॒ पय॑सि वा॒युः पयः॒ सोमो॑ वा॒युरि॑न्द्रा॒ग्नी मि॒त्र इन्द्र॒ आपो॒ ब्रह्म॑ य॒मो॑\-ऽभिजि॑त्यै॒ त्वष्टा᳚ प्र॒जा\-प॑तिः प्र॒जायै॑ पौर्णमा॒स्या अ॑मावा॒स्या॑या॒ अग॑त्यै॒ विश्वे॒ जित्या॑ अ॒श्विनौ॒ श्रुत्यै᳚।
ब्रह्म॒ तदे॒तं विष्णुः॒ स ए॒तं वा॒युः स ए॒तदाप॒स्ताः।
पि॒तरो॒ विश्वे॒ वस॑वो\-ऽकामयन्त॒ मेति॒ त ए॒तन्निर॑वपन्।
आपो॑ऽकामयन्त॒ मेति॒ ता ए॒तन्निर॑वपन्।
इ॒न्द्रा॒ग्नी अ॒श्विना॑वकामयेतां॒ वेति॒ तावे॒तन्निर॑वपताम्।
अ॒हो॒रा॒त्रे वा अ॑कामयेता॒मिति॒ ते ए॒तन्निर॑वपताम्।
अ॒न्यत्रा॑कामय॒तेति॒ स ए॒तन्निर॑वपत्।
इ॒न्द्रा॒ग्नी श्रैष्ठ्य॒मिन्द्रो॒ ज्यैष्ठ्य॒मिन्द्रो॑ दृ॒ढः।
अहिः॒ सूर्यो\-ऽदि॑त्यै॒ विष्ण॑वे प्रति॒ष्ठायै᳚।
सोमो॑ य॒मः स॑मा॒नाना᳚म्।
अ॒ग्निर्नो॑ रीरिषद॒न्यत्र॑ रीरिषः॥)]




\prashnaend{अ॒ग्निर्न॑ ऋ॒ध्यास्म॒ नवो॑नवो॒\-ऽग्निर्मि॒त्रश्च॒न्द्रमाः॒ षट्॥६॥}{अ॒ग्निर्न॒स्तन्नो॑ वा॒युरहि॑र्बु॒ध्निय॑ ऋ॒क्षा वा इ॒यमथै॒तत्पौ᳚र्णमा॒स्या अ॒जो वा एक॑पा॒थ्सूर्य॒स्त्रिष॑ष्टिः॥६३॥}{अ॒ग्निर्नः॑ पातु प्रति॒ष्ठायै॒ स्वाहेति॑॥}{हरिः॑ ओम्॥}{इति श्रीकृष्णयजुर्वेदीयतैत्तिरीयब्राह्मणे तृतीयाष्टके प्रथमः प्रपाठकः समाप्तः॥}
\clearpage
\sect{द्वितीयः प्रश्नः}
\setcounter{anuvakam}{0}
\dnsub{तैत्तिरीयब्राह्मणे तृतीयाष्टके द्वितीयः प्रपाठकः}

%3.2.1.1
तृ॒तीय॑स्यामि॒तो दि॒वि सोम॑ आसीत्।
तं गा॑य॒त्र्या\-ऽह॑रत्।
तस्य॑ प॒र्णम॑च्छिद्यत।
तत्प॒र्णो॑\-ऽभवत्।
तत्प॒र्णस्य॑ पर्ण॒त्वम्।
ब्रह्म॒ वै प॒र्णः।
यत्प॑र्णशा॒खया॑ व॒थ्सान॑पाक॒रोति॑।
ब्रह्म॑णै॒वैना॑न॒पाक॑रोति।
गा॒य॒त्रो वै प॒र्णः।
गा॒य॒त्राः प॒शवः॑॥१॥

%3.2.1.2
तस्मा॒त्त्रीणि॑त्रीणि प॒र्णस्य॑ पला॒शानि॑।
त्रि॒पदा॑ गाय॒त्री।
यत्प॑र्णशा॒खया॒ गाः प्रा॒र्पय॑ति।
स्वयै॒वैना॑ दे॒वत॑या॒ प्रार्प॑यति।
यं का॒मये॑ताप॒शुः स्या॒दिति॑।
अ॒प॒र्णान्तस्मै॒ शुष्का᳚ग्रा॒माह॑रेत्।
अ॒प॒शुरे॒व भ॑वति।
यं का॒मये॑त पशु॒मान्थ्स्या॒दिति॑।
ब॒हु॒प॒र्णान्तस्मै॑ बहुशा॒खामाह॑रेत्।
प॒शु॒मन्त॑मे॒वैनं॑ करोति॥२॥

%3.2.1.3
यत्प्राची॑मा॒ हरे᳚त्।
दे॒व॒लो॒कम॒भि ज॑येत्।
यदुदी॑चीं मनुष्यलो॒कम्।
प्राची॒मुदी॑ची॒मा ह॑रति।
उ॒भयो᳚र्लो॒कयो॑र॒भि\-जि॑त्यै।
इ॒षे त्वो॒र्जे त्वेत्या॑ह।
इष॑मे॒वोर्जं॒ यज॑माने दधाति।
वा॒यवः॒ स्थेत्या॑ह।
वा॒युर्वा अ॒न्तरि॑क्ष॒स्याध्य॑क्षाः।
अ॒न्त॒रि॒क्ष॒दे॒व॒त्याः᳚ खलु॒ वै प॒शवः॑॥३॥

%3.2.1.4
वा॒यव॑ ए॒वैना॒न्परि॑ ददाति।
प्र वा ए॑नाने॒तदा क॑रोति।
यदाह॑।
वा॒यवः॒ स्थेत्यु॑पा॒यवः॒ स्थेत्या॑ह।
यज॑मानायै॒व प॒शूनुप॑ ह्वयते।
दे॒वो वः॑ सवि॒ता प्रार्प॑य॒त्वित्या॑ह॒ प्रसू᳚त्यै।
श्रेष्ठ॑तमाय॒ कर्म॑ण॒ इत्या॑ह।
य॒ज्ञो हि श्रेष्ठ॑तमं॒ कर्म॑।
तस्मा॑दे॒वमा॑ह।
आप्या॑यध्वमघ्निया देवभा॒गमित्या॑ह॥४॥

%3.2.1.5
व॒थ्सेभ्य॑श्च॒ वा ए॒ताः पु॒रा म॑नु॒ष्ये᳚भ्य॒श्चाप्या॑यन्त।
दे॒वेभ्य॑ ए॒वैना॒ इन्द्रा॒याप्या॑ययति।
ऊर्ज॑स्वतीः॒ पय॑स्वती॒रित्या॑ह।
ऊर्ज॒ꣳ॒ हि पयः॑ स॒म्भर॑न्ति।
प्र॒जाव॑तीरनमी॒वा अ॑य॒क्ष्मा इत्या॑ह॒ प्रजा᳚त्यै।
मा वः॑ स्ते॒न ई॑शत॒ माऽघशꣳ॑स॒ इत्या॑ह॒ गुप्त्यै᳚।
रु॒द्रस्य॑ हे॒तिः परि॑ वो वृण॒क्त्वित्या॑ह।
रु॒द्रादे॒वैना᳚स्त्रायते।
ध्रु॒वा अ॒स्मिन्गोप॑तौ स्यात ब॒ह्वीरित्या॑ह।
ध्रु॒वा ए॒वास्मि॑न्ब॒ह्वीः क॑रोति॥५॥

%3.2.1.6
यज॑मानस्य प॒शून्पा॒हीत्या॑ह।
प॒शू॒नां गो॑पी॒थाय॑।
तस्मा᳚थ्सा॒यं प॒शव॒ उप॑स॒माव॑र्तन्ते।
अन॑धः सादयति।
गर्भा॑णां॒ धृत्या॒ अप्र॑पादाय।
तस्मा॒द्गर्भाः᳚ प्र॒जाना॒मप्र॑पादुकाः।
उ॒परी॑व॒ निद॑धाति।
उ॒परी॑व॒ हि सु॑व॒र्गो लो॒कः।
सु॒व॒र्गस्य॑ लो॒कस्य॒ सम॑ष्ट्यै॥६॥\anuvakamend[प॒शवः॑ करोति प॒शवो॑ देवभा॒गमित्या॑ह करोति॒ नव॑ च]

%3.2.2.1
दे॒वस्य॑ त्वा सवि॒तुः प्र॑स॒व इत्य॑श्वप॒र्॒शुमाद॑त्ते॒ प्रसू᳚त्यै।
अ॒श्विनो᳚र्बा॒हुभ्या॒मित्या॑ह।
अ॒श्विनौ॒ हि दे॒वाना॑मध्व॒र्यू आस्ता᳚म्।
पू॒ष्णो हस्ता᳚भ्या॒मित्या॑ह॒ यत्यै᳚।
यो वा ओष॑धीः पर्व॒शो वेद॑।
नैनाः॒ स हि॑नस्ति।
प्र॒जा\-प॑ति॒र्वा ओष॑धीः पर्व॒शो वे॑द।
स ए॑ना॒ न हि॑नस्ति।
अ॒श्व॒प॒र्श्वा ब॒र्॒हिरच्छै॑ति।
प्रा॒जा॒प॒त्यो वा अश्वः॑ सयोनि॒त्वाय॑॥७॥

%3.2.2.2
ओष॑धीना॒महिꣳ॑सायै।
य॒ज्ञस्य॑ घो॒षद॒सीत्या॑ह।
यज॑मान ए॒व र॒यिं द॑धाति।
प्रत्यु॑ष्ट॒ꣳ॒ रक्षः॒ प्रत्यु॑ष्टा॒ अरा॑तय॒ इत्या॑ह।
रक्ष॑सा॒मप॑हत्यै।
प्रेयम॑गाद्धि॒षणा॑ ब॒र्॒हिरच्छेत्या॑ह।
वि॒द्या वै धि॒षणा᳚।
वि॒द्ययै॒वैन॒दच्छै॑ति।
मनु॑ना कृ॒ता स्व॒धया॒ वित॒ष्टेत्या॑ह।
मा॒न॒वी हि पर्\mbox{}शुः॑ स्व॒धाकृ॑ता॥८॥

%3.2.2.3
त आव॑हन्ति क॒वयः॑ पु॒रस्ता॒दित्या॑ह।
शु॒श्रु॒वाꣳसो॒ वै क॒वयः॑।
य॒ज्ञः पु॒रस्ता᳚त्।
मु॒ख॒त ए॒व य॒ज्ञमा र॑भते।
अथो॒ यदे॒तदु॒क्त्वा यतः॒ कुत॑श्चा॒ हर॑ति।
तत्प्राच्या॑ ए॒व दि॒शो भ॑वति।
दे॒वेभ्यो॒ जुष्ट॑मि॒ह ब॒र्॒हिरा॒सद॒ इत्या॑ह।
ब॒र्॒\mbox{}हिषः॒ समृ॑द्ध्यै।
कर्म॒णो\-ऽन॑पराधाय।
दे॒वानां᳚ परिषू॒तम॒सीत्या॑ह॥९॥

%3.2.2.4
यद्वा इ॒दं किं च॑।
तद्दे॒वानां᳚ परिषू॒तम्।
अथो॒ यथा॒ वस्य॑से प्रति॒प्रोच्याहे॒दं क॑रिष्या॒मीति॑।
ए॒वमे॒व तद॑ध्व॒र्युर्दे॒वेभ्यः॑ प्रति॒प्रोच्य॑ ब॒र्॒हिर्दा॑ति।
आ॒त्मनो\-ऽहिꣳ॑सायै।
याव॑तः स्त॒म्बान्प॑रिदि॒शेत्।
यत्तेषा॑मुच्छि॒ꣴ॒ष्यात्।
अति॒ तद्य॒ज्ञस्य॑ रेचयेत्।
एकꣴ॑ स्त॒म्बं परि॑दिशेत्।
तꣳ सर्वं॑ दायात्॥१०॥

%3.2.2.5
य॒ज्ञस्यान॑तिरेकाय।
व॒र्॒षवृ॑द्धम॒सीत्या॑ह।
व॒र्॒षवृ॑द्धा॒ वा ओष॑धयः।
देव॑बर्\mbox{}हि॒रित्या॑ह।
दे॒वेभ्य॑ ए॒वैन॑त्करोति।
मा त्वा॒\-ऽन्वङ्मा ति॒र्यगित्या॒हाहिꣳ॑सायै।
पर्व॑ ते राध्यास॒मित्या॒हर्ध्यै᳚।
आ॒च्छे॒त्ता ते॒ मा रि॑ष॒मित्या॑ह।
नास्या॒ऽ॒ऽ॒त्मनो॑ मीयते।
य ए॒वं वेद॑॥११॥

%3.2.2.6
देव॑बर्\mbox{}हिः श॒तव॑ल्\mbox{}शं॒ विरो॒हेत्या॑ह।
प्र॒जा वै ब॒र्॒हिः।
प्र॒जानां᳚ प्र॒जन॑नाय।
स॒हस्र॑वल्‌शा॒ वि व॒यꣳ रु॑हे॒मेत्या॑ह।
आ॒\-मे॒वैतामा शा᳚स्ते।
पृ॒थि॒व्याः स॒म्पृचः॑ पा॒हीत्या॑ह॒ प्रति॑\-ष्ठित्यै।
अयु॑ङ्गायुङ्गान्मु॒ष्टीं लु॑नोति।
मि॒थु॒न॒त्वाय॒ प्रजा᳚त्यै।
सु॒स॒म्भृता᳚ त्वा॒ सम्भ॑रा॒मीत्या॑ह।
ब्रह्म॑णै॒वैन॒थ्सम्भ॑रति॥१२॥

%3.2.2.7
अदि॑त्यै॒ रास्ना॒\-ऽसीत्या॑ह।
इ॒यं वा अदि॑तिः।
अ॒स्या ए॒वैन॒द्रास्नां᳚ करोति।
इ॒न्द्रा॒ण्यै स॒न्नह॑न॒मित्या॑ह।
इ॒न्द्रा॒णी वा अग्रे॑ दे॒वता॑ना॒ꣳ॒ सम॑नह्यत।
साऽऽर्ध्नो᳚त्।
ऋद्ध्यै॒ सन्न॑ह्यति।
प्र॒जा वै ब॒र्॒हिः।
प्र॒जाना॒मप॑रावापाय।
तस्मा॒थ्स्नाव॑सन्तताः प्र॒जा जा॑यन्ते॥१३॥

%3.2.2.8
पू॒षा ते᳚ ग्र॒न्थिं ग्र॑थ्ना॒त्वित्या॑ह।
पुष्टि॑मे॒व यज॑माने दधाति।
स ते॒ मास्था॒दित्या॒हाहिꣳ॑सायै।
प॒श्चात्प्राञ्च॒मुप॑गूहति।
प॒श्चाद्वै प्रा॒चीन॒ꣳ॒ रेतो॑ धीयते।
प॒श्चादे॒वास्मै᳚ प्रा॒चीन॒ꣳ॒ रेतो॑ दधाति।
इन्द्र॑स्य त्वा बा॒हुभ्या॒मुद्य॑च्छ॒ इत्या॑ह।
इ॒न्द्रि॒यमे॒व यज॑माने दधाति।
बृह॒स्पते᳚र्मू॒र्ध्ना ह॑रा॒मीत्या॑ह।
ब्रह्म॒ वै दे॒वानां॒ बृह॒स्पतिः॑॥१४॥

%3.2.2.9
ब्रह्म॑णै॒वैन॑द्धरति।
उ॒र्व॑न्तरि॑क्ष॒मन्वि॒हीत्या॑ह॒ गत्यै᳚।
दे॒व॒ङ्ग॒मम॒सी\-त्या॑ह।
दे॒वाने॒वैन॑द्गमयति।
अन॑धः सादयति।
गर्भा॑णां॒ धृत्या॒ अप्र॑पादाय।
तस्मा॒द्गर्भाः᳚ प्र॒जाना॒मप्र॑पादुकाः।
उ॒परी॑व॒ नि द॑धाति।
उ॒परी॑व॒ हि सु॑व॒र्गो लो॒कः।
सु॒व॒र्गस्य॑ लो॒कस्य॒ सम॑ष्ट्यै॥१५॥\anuvakamend[स॒यो॒नि॒त्वाय॑ स्व॒धाकृ॑ता॒\-ऽसीत्या॑ह दाया॒द्वेद॑ भरति जायन्ते॒ बृह॒स्पतिः॒ सम॑ष्ट्यै]

%3.2.3.1
पू॒र्वे॒द्युरि॒ध्माब॒र्॒हिः क॑रोति।
य॒ज्ञमे॒वारभ्य॑ गृही॒त्वोप॑वसति।
प्र॒जा\-प॑तिर्य॒ज्ञम॑\-सृजत।
तस्यो॒खे अ॑स्रꣳसेताम्।
य॒ज्ञो वै प्र॒जा\-प॑तिः।
यथ्सा᳚न्नाय्यो॒खे भव॑तः।
य॒ज्ञस्यै॒व तदु॒खे उप॑दधा॒त्यप्र॑स्रꣳसाय।
शुन्ध॑ध्वं॒ दैव्या॑य॒ कर्म॑णे देवय॒ज्याया॒ इत्या॑ह।
दे॒व॒य॒ज्याया॑ ए॒वैना॑नि शुन्धति।
मा॒त॒रिश्व॑नो घ॒र्मो॑\-ऽसीत्या॑ह॥१६॥

%3.2.3.2
अ॒न्तरि॑क्षं॒ वै मा॑त॒रिश्व॑नो घ॒र्मः।
ए॒षां लो॒कानां॒ विधृ॑त्यै।
द्यौर॑सि पृथि॒व्य॑सीत्या॑ह।
दि॒वश्च॒ ह्ये॑षा पृ॑थि॒व्याश्च॒ सम्भृ॑ता।
यदु॒खा।
तस्मा॑दे॒वमा॑ह।
वि॒श्वधा॑या असि पर॒मेण॒ धाम्नेत्या॑ह।
वृष्टि॒र्वै वि॒श्वधा॑याः।
वृष्टि॑मे॒वाव॑ रुन्धे।
दृꣳह॑स्व॒ मा ह्वा॒रित्या॑ह॒ धृत्यै᳚॥१७॥

%3.2.3.3
वसू॑नां प॒वित्र॑म॒सीत्या॑ह।
प्रा॒णा वै वस॑वः।
तेषां॒ वा ए॒तद्भा॑ग॒धेयम्᳚।
यत्प॒वित्रम्᳚।
तेभ्य॑ ए॒वैन॑त्करोति।
श॒तधा॑रꣳ स॒हस्र॑धार॒मित्या॑ह।
प्रा॒णेष्वे॒वायु॑र्दधाति सर्व॒त्वाय॑।
त्रि॒वृत्प॑लाश\-शा॒खायां᳚ दर्भ॒मयं॑ भवति।
त्रि॒वृद्वै प्रा॒णः।
त्रि॒वृत॑मे॒व प्रा॒णं म॑ध्य॒तो यज॑माने दधाति॥१८॥

%3.2.3.4
सौ॒म्यः प॒र्णः स॑योनि॒त्वाय॑।
सा॒क्षात्प॒वित्रं॑ द॒र्भाः।
प्राख्सा॒यमधि॒नि द॑धाति।
तत्प्रा॑णापा॒नयो॑ रू॒पम्।
ति॒र्यक्प्रा॒तः।
तद्दर्‌श॑स्य रू॒पम्।
दा॒र्श्यꣴ ह्ये॑तदहः॑।
अन्नं॒ वै च॒न्द्रमाः᳚।
अन्नं॑ प्रा॒णाः।
उ॒भय॑मे॒वोपै॒त्यजा॑मित्वाय॥१९॥

%3.2.3.5
तस्मा॑द॒यꣳ स॒र्वतः॑ पवते।
हु॒तः स्तो॒को हु॒तो द्र॒फ्स इत्या॑ह॒ प्रति॑\-ष्ठित्यै।
ह॒विषो\-ऽस्क॑न्दाय।
न हि हु॒तꣴ स्वाहा॑कृत॒ꣴ॒ स्कन्द॑ति।
दि॒वि नाको॒ नामा॒ग्निः।
तस्य॑ वि॒प्रुषो॑ भाग॒धेयम्᳚।
अ॒ग्नये॑ बृह॒ते नाका॒येत्या॑ह।
नाक॑मे॒वाग्निं भा॑ग॒धेये॑न॒ सम॑र्धयति।
स्वाहा॒ द्यावा॑पृथि॒वीभ्या॒मित्या॑ह।
द्यावा॑पृथि॒व्योरे॒वैन॒त्प्रति॑\-ष्ठापयति॥२०॥

%3.2.3.6
प॒वित्र॑व॒त्यान॑यति।
अ॒पां चै॒वौष॑धीनां च॒ रस॒ꣳ॒ सꣳसृ॑जति।
अथो॒ ओष॑धीष्वे॒व प॒शून्प्रति॑\-ष्ठापयति।
अ॒न्वा॒रभ्य॒ वाचं॑ यच्छति।
य॒ज्ञस्य॒ धृत्यै᳚।
धा॒रय॑न्नास्ते।
धा॒रय॑न्त इव॒ हि दु॒हन्ति॑।
काम॑धुक्ष॒ इत्या॒हाऽऽतृ॒तीय॑स्यै।
त्रय॑ इ॒मे लो॒काः।
इ॒माने॒व लो॒कान् ‌यज॑मानो दुहे॥२१॥

%3.2.3.7
अ॒मूमिति॒ नाम॑ गृह्णाति।
भ॒द्रमे॒वासां॒ कर्मा॒ विष्क॑रोति।
सा वि॒श्वायुः॒ सा वि॒श्वव्य॑चाः॒ सा वि॒श्वक॒र्मेत्या॑ह।
इ॒यं वै वि॒श्वायुः॑।
अ॒न्तरि॑क्षं वि॒श्वव्य॑चाः।
अ॒सौ वि॒श्वक॑र्मा।
इ॒माने॒वैताभि॑र्लो॒कान्‌ य॑थापू॒र्वं दु॑हे।
अथो॒ यथा᳚ प्रदा॒त्रे पुण्य॑मा॒शास्ते᳚।
ए॒वमे॒वैना॑ ए॒तदुप॑स्तौति।
तस्मा॒त्प्रादा॒दित्यु॒न्नीय॒ वन्द॑माना उपस्तु॒वन्तः॑ प॒शून्दु॑\-हन्ति॥२२॥

%3.2.3.8
ब॒हु दु॒ग्धीन्द्रा॑य दे॒वेभ्यो॑ ह॒विरिति॒ वाचं॒ विसृ॑जते।
य॒था॒दे॒व॒तमे॒व प्रसौ॑ति।
दैव्य॑स्य च मानु॒षस्य॑ च॒ व्यावृ॑त्यै।
त्रिरा॑ह।
त्रिष॑त्या॒ हि दे॒वाः।
अवा॑चं य॒मो\-ऽन॑न्वार॒भ्योत्त॑राः।
अप॑रिमितमे॒वाव॑ रुन्धे।
न दा॑रुपा॒त्रेण॑ दुह्यात्।
अ॒ग्नि॒वद्वै दा॑रुपा॒त्रम्।
यद्दा॑रुपा॒त्रेण॑ दु॒ह्यात्॥२३॥

%3.2.3.9
या॒तया᳚म्ना ह॒विषा॑ यजेत।
अथो॒ खल्वा॑हुः।
पु॒रो॒डाश॑मुखानि॒ वै ह॒वीꣳषि॑।
नेत इ॑तः पुरो॒डाशꣳ॑ ह॒विषो॒ यामो॒\-ऽस्तीति॑।
काम॑मे॒व दा॑रुपा॒त्रेण॑ दुह्यात्।
शू॒द्र ए॒व न दु॑ह्यात्।
अस॑तो॒ वा ए॒ष सम्भू॑तः।
यच्छू॒द्रः।
अह॑विरे॒व तदित्या॑हुः।
यच्छू॒द्रो दोग्धीति॑॥२४॥

%3.2.3.10
अ॒ग्नि॒हो॒त्रमे॒व न दु॑ह्याच्छू॒द्रः।
तद्धि नोत्पु॒नन्ति॑।
य॒दा खलु॒ वै प॒वित्र॑म॒त्येति॑।
अथ॒ तद्ध॒विरिति॑।
सम्पृ॑च्यध्वमृतावरी॒रित्या॑ह।
अ॒पां चै॒वौष॑धीनां च॒ रस॒ꣳ॒ सꣳ सृ॑जति।
तस्मा॑द॒पां चौष॑धीनां च॒ रस॒मुप॑जीवामः।
म॒न्द्रा धन॑स्य सा॒तय॒ इत्या॑ह।
पुष्टि॑मे॒व यज॑माने दधाति।
सोमे॑न॒ त्वात॑न॒च्मीन्द्रा॑य॒ दधीत्या॑ह॥२५॥

%3.2.3.11
सोम॑मे॒वैन॑त्करोति।
यो वै सोमं॑ भक्षयि॒त्वा।
सं॒व॒थ्स॒रꣳ सोमं॒ न पिब॑ति।
पु॒न॒र्भक्ष्यो᳚ऽस्य सोमपी॒थो भ॑वति।
सोमः॒ खलु॒ वै सा᳚न्ना॒य्यम्।
य ए॒वं वि॒द्वान्थ्सा᳚न्ना॒य्यं पिब॑ति।
अ॒पु॒न॒र्भक्ष्यो᳚ऽस्य सोमपी॒थो भ॑वति।
न मृ॒न्मये॒नापि॑ दध्यात्।
यन्मृ॒न्मये॑नापिद॒ध्यात्।
पि॒तृ॒दे॒व॒त्यꣴ॑ स्यात्॥२६॥

%3.2.3.12
अ॒य॒स्पा॒त्रेण॑ वा दारुपा॒त्रेण॒ वाऽपि॑ दधाति।
तद्धि सदे॑वम्।
उ॒द॒न्वद्भ॑वति।
आपो॒ वै र॑क्षो॒घ्नीः।
रक्ष॑सा॒मप॑हत्यै।
अद॑स्तमसि॒ विष्ण॑वे॒ त्वेत्या॑ह।
य॒ज्ञो वै विष्णुः॑।
य॒ज्ञायै॒वैन॒दद॑स्तं करोति।
विष्णो॑ ह॒व्यꣳ र॑क्ष॒स्वेत्या॑ह॒ गुप्त्यै᳚।
अन॑धः सादयति।
गर्भा॑णां॒ धृत्या॒ अप्र॑पादाय।
तस्मा॒द्गर्भाः᳚ प्र॒जाना॒मप्र॑पादुकाः।
उ॒परी॑व॒ निद॑धाति।
उ॒परी॑व॒ हि सु॑व॒र्गो लो॒कः।
सु॒व॒र्गस्य॑ लो॒कस्य॒ सम॑ष्ट्यै॥२७॥\anuvakamend[अ॒सीत्या॑ह॒ धृत्यै॒ यज॑माने दधा॒त्यजा॑मित्वाय स्थापयति दुहे दुहन्ति दु॒ह्याद्दोग्धीति॒ दधीत्या॑ह स्याथ्सादयति॒ पञ्च॑ च]

%3.2.4.1
कर्म॑णे वां दे॒वेभ्यः॑ शकेय॒मित्या॑ह॒ शक्त्यै᳚।
य॒ज्ञस्य॒ वै सन्त॑ति॒मनु॑ प्र॒जाः प॒शवो॒ यज॑मानस्य॒ सन्ता॑यन्ते।
य॒ज्ञस्य॒ विच्छि॑त्ति॒मनु॑ प्र॒जाः प॒शवो॒ यज॑मानस्य॒ विच्छि॑द्यन्ते।
य॒ज्ञस्य॒ सन्त॑तिरसि य॒ज्ञस्य॑ त्वा॒ सन्त॑त्यै स्तृणामि॒ सन्त॑त्यै त्वा य॒ज्ञस्येत्याह॑व॒नीया॒थ्सं त॑नोति।
यज॑\-मानस्य प्र॒जायै॑ पशू॒नाꣳ सन्त॑त्यै।
अ॒पः प्रण॑यति।
श्र॒द्धा वा आपः॑।
श्र॒द्धामे॒वारभ्य॑ प्र॒णीय॒ प्रच॑रति।
अ॒पः प्रण॑यति।
य॒ज्ञो वा आपः॑॥२८॥

%3.2.4.2
य॒ज्ञमे॒वारभ्य॑ प्र॒णीय॒ प्रच॑रति।
अ॒पः प्रण॑यति।
वज्रो॒ वा आपः॑।
वज्र॑मे॒व भ्रातृ॑व्येभ्यः प्र॒हृत्य॑ प्र॒णीय॒ प्रच॑रति।
अ॒पः प्रण॑यति।
आपो॒ वै र॑क्षो॒घ्नीः।
रक्ष॑सा॒मप॑हत्यै।
अ॒पः प्रण॑यति।
आपो॒ वै दे॒वानां᳚ प्रि॒यं धाम॑।
दे॒वाना॑मे॒व प्रि॒यं धाम॑ प्र॒णीय॒ प्रच॑रति॥२९॥

%3.2.4.3
अ॒पः प्रण॑यति।
आपो॒ वै सर्वा॑ दे॒वताः᳚।
दे॒वता॑ ए॒वाऽऽरभ्य॑ प्र॒णीय॒ प्रच॑रति।
वेषा॑य॒ त्वेत्या॑ह।
वेषा॑य॒ ह्ये॑नदाद॒त्ते।
प्रत्यु॑ष्ट॒ꣳ॒ रक्षः॒ प्रत्यु॑ष्टा॒ अरा॑तय॒ इत्या॑ह।
रक्ष॑सा॒मप॑हत्यै।
धूर॒सीत्या॑ह।
ए॒ष वै धुर्यो॒\-ऽग्निः।
तं यदनु॑पस्पृश्याती॒यात्॥३०॥

%3.2.4.4
अ॒ध्व॒र्युं च॒ यज॑मानं च॒ प्रद॑हेत्।
उ॒प॒स्पृश्यात्ये॑ति।
अ॒ध्व॒र्योश्च॒ यज॑मानस्य॒ चाप्र॑दाहाय।
धूर्व॒ तं यो᳚स्मान्धूर्व॑ति॒ तं धू᳚र्व॒ यं व॒यं धूर्वा॑म॒ इत्या॑ह।
द्वौ वाव पुरु॑षौ।
यं चै॒व धूर्व॑ति।
यश्चै॑नं॒ धूर्व॑ति।
तावु॒भौ शु॒चा\-ऽर्प॑यति।
त्वं दे॒वाना॑मसि॒ सस्नि॑तमं॒ पप्रि॑तमं॒ जुष्ट॑तमं॒ वह्नि॑तमं देव॒हूत॑म॒मित्या॑ह।
य॒था॒\-य॒जु\-रे॒वै\-तत्॥३१॥

%3.2.4.5
अह्रु॑तमसि हवि॒र्धान॒मित्या॒हाना᳚र्त्यै।
दृꣳह॑स्व॒ मा ह्वा॒रित्या॑ह॒ धृत्यै᳚।
मि॒त्रस्य॑ त्वा॒ चक्षु॑षा॒ प्रेक्ष॒ इत्या॑ह मित्र॒त्वाय॑।
मा भेर्मा संवि॑क्था॒ मा त्वा॑ हिꣳसिष॒मित्या॒हाहिꣳ॑सायै।
यद्वै किं च॒ वातो॒ नाभि॒ वाति॑।
तथ्सर्वं॑ वरुणदेव॒त्यम्᳚।
उ॒रु वाता॒येत्या॑ह।
अवा॑रुणमे॒वैन॑त्करोति।
दे॒वस्य॑ त्वा सवि॒तुः प्र॑स॒व इत्या॑ह॒ प्रसू᳚त्यै।
अ॒श्विनो᳚र्बा॒हुभ्या॒मित्या॑ह॥३२॥

%3.2.4.6
अ॒श्विनौ॒ हि दे॒वाना॑मध्व॒र्यू आस्ता᳚म्।
पू॒ष्णो हस्ता᳚भ्या॒मित्या॑ह॒ यत्यै᳚।
अ॒ग्नये॒ जुष्टं॒ निर्व॑पा॒मीत्या॑ह।
अ॒ग्नय॑ ए॒वैनां॒ जुष्टं॒ निर्व॑पति।
त्रिर्यजु॑षा।
त्रय॑ इ॒मे लो॒काः।
ए॒षां लो॒काना॒माप्त्यै᳚।
तू॒ष्णीं च॑तु॒र्थम्।
अप॑रिमितमे॒वाव॑ रुन्धे।
स ए॒वमे॒वानु॑पू॒र्वꣳ ह॒वीꣳषि॒ निर्व॑पति॥३३॥

%3.2.4.7
इ॒दं दे॒वाना॑मि॒दमु॑ नः स॒हेत्या॑ह॒ व्यावृ॑त्यै।
स्फा॒त्यै त्वा॒ नारा᳚त्या॒ इत्या॑ह॒ गुप्त्यै᳚।
तम॑सीव॒ वा ए॒षो᳚\-ऽन्तश्च॑रति।
यः प॑री॒णहि॑।
सुव॑र॒भि वि ख्ये॑षं वैश्वान॒रं ज्योति॒रित्या॑ह।
सुव॑रे॒वाभि वि प॑श्यति वैश्वान॒रं ज्योतिः॑।
द्यावा॑पृथि॒वी ह॒विषि॑ गृही॒त उद॑वेपेताम्।
दृꣳह॑न्ता॒न्दुर्या॒ द्यावा॑पृथि॒व्योरित्या॑ह।
गृ॒हाणां॒ द्यावा॑पृथि॒व्योर्धृत्यै᳚।
उ॒र्व॑न्तरि॑क्ष॒मन्वि॒हीत्या॑ह॒ गत्यै᳚।
अदि॑त्यास्त्वो॒पस्थे॑ सादया॒मीत्या॑ह।
इ॒यं वा अदि॑तिः।
अ॒स्या ए॒वैन॑दु॒पस्थे॑ सादयति।
अग्ने॑ ह॒व्यꣳ र॑क्ष॒स्वेत्या॑ह॒ गुप्त्यै᳚॥३४॥\anuvakamend[य॒ज्ञो वा आपो॒ धाम॑ प्र॒णीय॒ प्रच॑रत्यती॒यादे॒तद्बा॒हुभ्या॒मित्या॑ह ह॒वीꣳषि॒ निर्व॑पति॒ गत्यै॑ च॒त्वारि॑ च]

%3.2.5.1
इन्द्रो॑ वृ॒त्रम॑हन्।
सो॑ऽपः।
अ॒भ्य॑म्रियत।
तासां॒ यन्मेध्यं॑ य॒ज्ञिय॒ꣳ॒ सदे॑व॒मासी᳚त्।
तदपोद॑क्रामत्।
ते द॒र्भा अ॑भवन्।
यद्द॒र्भैर॒प उ॑त्पु॒नाति॑।
या ए॒व मेध्या॑ य॒ज्ञियाः॒ सदे॑वा॒ आपः॑।
ताभि॑रे॒वैना॒ उत्पु॑नाति।
द्वाभ्या॒मुत्पु॑नाति॥३५॥

%3.2.5.2
द्वि॒पाद्यज॑मानः॒ प्रति॑\-ष्ठित्यै।
दे॒वो वः॑ सवि॒तोत्पु॑ना॒त्वित्या॑ह।
स॒वि॒तृप्र॑सूत ए॒वैना॒ उत्पु॑नाति।
अच्छि॑द्रेण प॒वित्रे॒णेत्या॑ह।
अ॒सौ वा आ॑दि॒त्यो\-ऽच्छि॑द्रं प॒वित्रम्᳚।
तेनै॒वैना॒ उत्पु॑नाति।
वसोः॒ सूर्य॑स्य र॒श्मिभि॒रित्या॑ह।
प्रा॒णा वा आपः॑।
प्रा॒णा वस॑वः।
प्रा॒णा र॒श्मयः॑॥३६॥

%3.2.5.3
प्रा॒णैरे॒व प्रा॒णान्थ्सं पृ॑णक्ति।
सा॒वि॒त्रि॒यर्चा।
स॒वि॒तृप्र॑सूतं मे॒ कर्मा॑स॒दिति॑।
स॒वि॒तृप्र॑सूतमे॒वास्य॒ कर्म॑ भवति।
प॒च्छो गा॑यत्रि॒या त्रि॑ष्षमृद्ध॒त्वाय॑।
आपो॑ देवीरग्रेपुवो अग्रेगुव॒ इत्या॑ह।
रू॒पमे॒वासा॑मे॒तन्म॑हि॒मानं॒ व्याच॑ष्टे।
अग्र॑ इ॒मं य॒ज्ञं न॑य॒ताग्रे॑ य॒ज्ञप॑ति॒मित्या॑ह।
अग्र॑ ए॒व य॒ज्ञं न॑यन्ति।
अग्रे॑ य॒ज्ञप॑तिम्॥३७॥

%3.2.5.4
यु॒ष्मानिन्द्रो॑\-ऽवृणीत वृत्र॒तूर्ये॑ यू॒यमिन्द्र॑मवृणीध्वं वृत्र॒तूर्य॒ इत्या॑ह।
वृ॒त्रꣳ ह॑ हनि॒ष्यन्निन्द्र॒ आपो॑ वव्रे।
आपो॒ हेन्द्रं॑ वव्रिरे।
सं॒ज्ञामे॒वासा॑मे॒तथ्सामा॑नं॒ व्याच॑ष्टे।
प्रोक्षि॑ताः॒ स्थेत्या॑ह।
तेनाऽऽपः॒ प्रोक्षि॑ताः।
अ॒ग्नये॑ वो॒ जुष्टं॒ प्रोक्षा᳚म्य॒ग्नीषोमा᳚भ्या॒मित्या॑ह।
य॒था॒दे॒व॒तमे॒वैना॒न्प्रोक्ष॑ति।
त्रिः प्रोक्ष॑ति।
त्र्या॑वृ॒द्धि य॒ज्ञः॥३८॥

%3.2.5.5
अथो॒ रक्ष॑सा॒मप॑हत्यै।
शुन्ध॑ध्वं॒ दैव्या॑य॒ कर्म॑णे देवय॒ज्याया॒ इत्या॑ह।
दे॒व॒य॒ज्याया॑ ए॒वैना॑नि शुन्धति।
त्रिः प्रोक्ष॑ति।
त्र्या॑वृ॒द्धि य॒ज्ञः।
अथो॑ मेध्य॒त्वाय॑।
अव॑धूत॒ꣳ॒ रक्षो\-ऽव॑धूता॒ अरा॑तय॒ इत्या॑ह।
रक्ष॑सा॒मप॑हत्यै।
अदि॑त्या॒स्त्वग॒सीत्या॑ह।
इ॒यं वा अदि॑तिः॥३९॥

%3.2.5.6
अ॒स्या ए॒वैन॒त्त्वचं॑ करोति।
प्रति॑ त्वा पृथि॒वी वे॒त्त्वित्या॑ह॒ प्रति॑\-ष्ठित्यै।
पु॒रस्ता᳚त्प्रती॒चीन॑\-ग्रीव॒\-मुत्त॑रलो॒मोप॑स्तृणाति मेध्य॒त्वाय॑।
तस्मा᳚त्पु॒रस्ता᳚त्प्र॒त्यञ्चः॑ प॒शवो॒ मेध॒मुप॑तिष्ठन्ते।
तस्मा᳚त्प्र॒जा मृ॒गं ग्राहु॑काः।
य॒ज्ञो दे॒वेभ्यो॒ निला॑यत।
कृष्णो॑ रू॒पं कृ॒त्वा।
यत्कृ॑ष्णाजि॒ने ह॒विर॑ध्यव॒हन्ति॑।
य॒ज्ञादे॒व तद्य॒ज्ञं प्रयु॑ङ्क्ते।
ह॒विषो\-ऽस्क॑न्दाय॥४०॥

%3.2.5.7
अ॒धि॒षव॑णमसि वानस्प॒त्यमित्या॑ह।
अ॒धि॒षव॑ण\-मे॒वैन॑त्करोति।
प्रति॒ त्वा\-ऽदि॑त्या॒स्त्वग्वे॒त्त्वित्या॑ह सय॒त्वाय॑।
अ॒ग्नेस्त॒नूर॒सी\-त्या॑ह।
अ॒ग्नेर्वा ए॒षा त॒नूः।
यदोष॑धयः।
वा॒चो वि॒सर्ज॑न॒मित्या॑ह।
य॒दा हि प्र॒जा ओष॑धीनाम॒श्ञन्ति॑।
अथ॒ वाचं॒ विसृ॑जन्ते।
दे॒ववी॑तये त्वा गृह्णा॒मीत्या॑ह॥४१॥

%3.2.5.8
दे॒वता॑भिरे॒वैन॒थ्सम॑र्धयति।
अद्रि॑रसि वानस्प॒त्य इत्या॑ह।
ग्रावा॑णमे॒वैन॑त्करोति।
स इ॒दं दे॒वेभ्यो॑ ह॒व्यꣳ सु॒शमि॑ शमि॒ष्वेत्या॑ह॒ शान्त्यै᳚।
हवि॑ष्कृ॒देहीत्या॑ह।
य ए॒व दे॒वानाꣳ॑ हवि॒ष्कृतः॑।
तान्‌ ह्व॑यति।
त्रिर्ह्व॑यति।
त्रिष॑त्या॒ हि दे॒वाः।
इष॒माव॒दोर्ज॒माव॒देत्या॑ह॥४२॥

%3.2.5.9
इष॑मे॒वोर्जं॒ यज॑माने दधाति।
द्यु॒मद्व॑दत व॒यꣳ स॑ङ्घा॒तं जे॒ष्मेत्या॑ह॒ भ्रातृ॑व्याभिभूत्यै।
मनोः᳚ श्र॒द्धादे॑वस्य॒ यज॑मानस्या\-सुर॒घ्नी वाक्।
य॒ज्ञा॒यु॒धेषु॒ प्रवि॑ष्टा\-ऽऽसीत्।
तेऽसु॑रा॒ याव॑न्तो यज्ञायु॒धाना॑मु॒द्वद॑ता\-मु॒पा\-शृ॑ण्वन्।
ते परा॑भवन्।
तस्मा॒थ्स्वानां॒ मध्ये॑\-ऽव॒साय॑ यजेत।
याव॑न्तो\-ऽस्य॒ भ्रातृ॑व्या यज्ञायु॒धाना॑\-मु॒द्वद॑ता\-मुप\-शृ॒ण्वन्ति॑।
ते परा॑ भवन्ति।
उ॒च्चैः स॒माह॑न्त॒ वा आ॑ह॒ विजि॑त्यै॥४३॥

%3.2.5.10
वृ॒ङ्क्त ए॑षामिन्द्रि॒यं वी॒र्यम्᳚।
श्रेष्ठ॑ एषां भवति।
व॒र्॒षवृ॑द्धमसि॒ प्रति॑ त्वा व॒र्॒षवृ॑द्धं वे॒त्त्वित्या॑ह।
व॒र्॒षवृ॑द्धा॒ वा ओष॑धयः।
व॒र्॒षवृ॑द्धा इ॒षीकाः॒ समृ॑द्ध्यै।
य॒ज्ञꣳ रक्षा॒ꣴ॒स्यनु॒ प्रावि॑शन्।
तान्य॒स्ना प॒शुभ्यो॑ नि॒रवा॑दयन्त।
तुषै॒रोष॑धीभ्यः।
परा॑पूत॒ꣳ॒ रक्षः॒ परा॑पूता॒ अरा॑तय॒ इत्या॑ह।
रक्ष॑सा॒मप॑हत्यै॥४४॥

%3.2.5.11
रक्ष॑सां भा॒गो॑\-ऽसीत्या॑ह।
तुषै॑रे॒व रक्षाꣳ॑सि नि॒रव॑दयते।
अ॒प उप॑स्पृशति मेध्य॒त्वाय॑।
वा॒युर्वो॒ विवि॑न॒क्त्वित्या॑ह।
प॒वित्रं॒ वै वा॒युः।
पु॒नात्ये॒वैनान्॑।
अ॒न्तरि॑क्षादिव॒ वा ए॒ते प्रस्क॑न्दन्ति।
ये शूर्पा᳚त्।
दे॒वो वः॑ सवि॒ता हिर॑ण्यपाणिः॒ प्रति॑\-गृह्णा॒त्वित्या॑ह॒ प्रति॑\-ष्ठित्यै।
ह॒विषो\-ऽस्क॑न्दाय।
त्रिष्फ॒लीक॑र्त॒वा आ॑ह।
त्र्या॑वृ॒द्धि य॒ज्ञः।
अथो॑ मेध्य॒त्वाय॑॥४५॥\anuvakamend[द्वाभ्या॒मुत्पु॑नाति र॒श्मयो॑ नय॒न्त्यग्रे॑ य॒ज्ञप॑तिं य॒ज्ञो\-ऽदि॑ति॒रस्क॑न्दाय गृह्णा॒मीत्या॑ह व॒देत्या॑ह॒ विजि॑त्या॒ अप॑हत्या॒ अस्क॑न्दाय॒ त्रीणि॑ च]

%3.2.6.1
अव॑धूत॒ꣳ॒ रक्षो\-ऽव॑धूता॒ अरा॑तय॒ इत्या॑ह।
रक्ष॑सा॒मप॑हत्यै।
अदि॑त्या॒स्त्वग॒सीत्या॑ह।
इ॒यं वा अदि॑तिः।
अ॒स्या ए॒वैन॒त्त्वचं॑ करोति।
प्रति॑ त्वा पृथि॒वी वे॒त्त्वित्या॑ह॒ प्रति॑\-ष्ठित्यै।
पु॒रस्ता᳚त्प्रती॒चीन॑\-ग्रीव॒मुत्त॑र\-लो॒मोप॑स्तृणाति मेध्य॒त्वाय॑।
तस्मा᳚त्पु॒रस्ता᳚त्प्र॒त्यञ्चः॑ प॒शवो॒ मेध॒मुप॑तिष्ठन्ते।
तस्मा᳚त्प्र॒जा मृ॒गं ग्राहु॑काः।
य॒ज्ञो दे॒वेभ्यो॒ निला॑यत॥४६॥

%3.2.6.2
कृष्णो॑ रू॒पं कृ॒त्वा।
यत्कृ॑ष्णाजि॒ने ह॒विर॑धिपि॒नष्टि॑।
य॒ज्ञादे॒व तद्य॒ज्ञं प्रयु॑ङ्क्ते।
ह॒विषो\-ऽस्क॑न्दाय।
द्यावा॑पृथि॒वी स॒हास्ता᳚म्।
ते श॑म्यामा॒त्रमेक॒मह॒र्व्यैताꣳ॑ शम्यामा॒त्रमेक॒महः॑।
दि॒वः स्क॑म्भ॒निर॑सि॒ प्रति॒ त्वा\-ऽदि॑त्या॒स्त्वग्वे॒त्त्वित्या॑ह।
द्यावा॑पृथि॒व्योर्वीत्यै᳚।
धि॒षणा॑ऽसि पर्व॒त्या प्रति॑ त्वा दि॒वः स्क॑म्भ॒निर्वे॒त्त्वित्या॑ह।
द्यावा॑पृथि॒व्योर्विधृ॑त्यै॥४७॥

%3.2.6.3
धि॒षणा॑ऽसि पार्वते॒यी प्रति॑ त्वा पर्व॒तिर्वे॒त्त्वित्या॑ह।
द्यावा॑पृथि॒व्योर्धृत्यै᳚।
दे॒वस्य॑ त्वा सवि॒तुः प्र॑स॒व इत्या॑ह॒ प्रसू᳚त्यै।
अ॒श्विनो᳚र्बा॒हुभ्या॒मित्या॑ह।
अ॒श्विनौ॒ हि दे॒वाना॑मध्व॒र्यू आस्ता᳚म्।
पू॒ष्णो हस्ता᳚भ्या॒मित्या॑ह॒ यत्त्यै᳚।
अधि॑वपा॒मीत्या॑ह।
य॒था॒दे॒व॒तमे॒वैना॒नधि॑ वपति।
धा॒न्य॑मसि धिनु॒हि दे॒वानित्या॑ह।
ए॒तस्य॒ यजु॑षो वी॒र्ये॑ण॥४८॥

%3.2.6.4
याव॒देका॑ दे॒वता॑ का॒मय॑ते॒ याव॒देका᳚।
ताव॒दाहु॑तिः प्रथते।
न हि तदस्ति॑।
यत्ताव॑दे॒व स्यात्।
याव॑ज्जु॒होति॑।
प्रा॒णाय॑ त्वा\-ऽपा॒नाय॒ त्वेत्या॑ह।
प्रा॒णाने॒व यज॑माने दधाति।
दी॒र्घामनु॒ प्रसि॑ति॒मायु॑षे धा॒मित्या॑ह।
आयु॑रे॒वास्मि॑न्दधाति।
अ॒न्तरि॑क्षादिव॒ वा ए॒तानि॒ प्रस्क॑न्दन्ति।
यानि॑ दृ॒षदः॑।
दे॒वो वः॑ सवि॒ता हिर॑ण्यपाणिः॒ प्रति॑\-गृह्णा॒त्वित्या॑ह॒ प्रति॑\-ष्ठित्यै।
ह॒विषो\-ऽस्क॑न्दाय।
असं॑वपन्ती पिꣳषा॒णूनि॑ कुरुता॒दित्या॑ह मेध्य॒त्वाय॑॥४९॥\anuvakamend[निला॑यत॒ विधृ॑त्यै वी॒र्ये॑ण स्कन्दन्ति च॒त्वारि॑ च]

%3.2.7.1
धृष्टि॑रसि॒ ब्रह्म॑ य॒च्छेत्या॑ह॒ धृत्यै᳚।
अपा᳚ग्ने॒\-ऽग्निमा॒मादं॑ जहि॒ निष्क्र॒व्यादꣳ॑ से॒धा दे॑व॒यजं॑ व॒हेत्या॑ह।
य ए॒वाऽऽमात्क्र॒व्यात्।
तम॑प॒हत्य॑।
मेध्ये॒\-ऽग्नौ क॒पाल॒मुप॑दधाति।
निर्द॑ग्ध॒ꣳ॒ रक्षो॒ निर्द॑ग्धा॒ अरा॑तय॒ इत्या॑ह।
रक्षाꣴ॑स्ये॒व निर्द॑हति।
अ॒ग्नि॒वत्युप॑दधाति।
अ॒स्मिन्ने॒व लो॒के ज्योति॑र्धत्ते।
अङ्गा॑र॒मधि॑ वर्तयति॥५०॥

%3.2.7.2
अ॒न्तरि॑क्ष ए॒व ज्योति॑र्धत्ते।
आ॒दि॒त्यमे॒वामुष्मिँ॑ल्लो॒के ज्योति॑र्धत्ते।
ज्योति॑ष्मन्तो\-ऽस्मा इ॒मे लो॒का भ॑वन्ति।
य ए॒वं वेद॑।
ध्रु॒वम॑सि पृथि॒वीं दृ॒ꣳ॒हेत्या॑ह।
पृ॒थि॒वीमे॒वैतेन॑ दृꣳहति।
ध॒र्त्रम॑स्य॒न्तरि॑क्षं दृ॒ꣳ॒हेत्या॑ह।
अ॒न्तरि॑क्षमे॒वैतेन॑ दृꣳहति।
ध॒रुण॑मसि॒ दिवं॑ दृ॒ꣳ॒हेत्या॑ह।
दिव॑मे॒वैतेन॑ दृꣳहति॥५१॥

%3.2.7.3
धर्मा॑सि॒ दिशो॑ दृ॒ꣳ॒हेत्या॑ह।
दिश॑ ए॒वैतेन॑ दृꣳहति।
इ॒माने॒वैतैर्लो॒कान्दृꣳ॑हति।
दृꣳह॑न्ते\-ऽस्मा इ॒मे लो॒काः प्र॒जया॑ प॒शुभिः॑।
य ए॒वं वेद॑।
त्रीण्यग्रे॑ क॒पाला॒न्युप॑दधाति।
त्रय॑ इ॒मे लो॒काः।
ए॒षां लो॒काना॒माप्त्यै᳚।
एक॒मग्रे॑ क॒पाल॒मुप॑ दधाति।
एकं॒ वा अग्रे॑ क॒पालं॒ पुरु॑षस्य स॒म्भव॑ति॥५२॥

%3.2.7.4
अथ॒ द्वे।
अथ॒ त्रीणि॑।
अथ॑ च॒त्वारि॑।
अथा॒ष्टौ।
तस्मा॑द॒ष्टा\-क॑पालं॒ पुरु॑षस्य॒ शिरः॑।
यदे॒वं क॒पाला᳚न्युप॒दधा॑ति।
य॒ज्ञो वै प्र॒जा\-प॑तिः।
य॒ज्ञमे॒व प्र॒जा\-प॑ति॒ꣳ॒ सꣴस्क॑रोति।
आ॒त्मान॑मे॒व तथ्सꣴस्क॑रोति।
तꣳ सꣴस्कृ॑तमा॒त्मानम्᳚॥५३॥

%3.2.7.5
अ॒मुष्मिँ॑ल्लो॒केऽनु॒ परै॑ति।
यद॒ष्टावु॑प॒दधा॑ति।
गा॒य॒त्रि॒या तथ्सम्मि॑तम्।
यन्नव॑।
त्रि॒वृता॒ तत्।
यद्दश॑।
वि॒राजा॒ तत्।
यदेका॑दश।
त्रि॒ष्टुभा॒ तत्।
यद्द्वाद॑श॥५४॥

%3.2.7.6
जग॑त्या॒ तत्।
छन्दः॑ सम्मितानि॒ स उ॑प॒दध॑त्क॒पाला॑नि।
इ॒माँल्लो॒कान॑नुपू॒र्वं दिशो॒ विधृ॑त्यै दृꣳहति।
अथाऽऽयुः॑ प्रा॒णान्प्र॒जां प॒शून् यज॑माने दधाति।
स॒जा॒तान॑स्मा अ॒भितो॑ बहु॒लान्क॑रोति।
चितः॒ स्थेत्या॑ह।
य॒था॒\-य॒जु\-रे॒वै\-तत्।
भृगू॑णा॒मङ्गि॑रसां॒ तप॑सा तप्यध्व॒मित्या॑ह।
दे॒वता॑नामे॒वैना॑नि॒ तप॑सा तपति।
तानि॒ ततः॒ सꣴस्थि॑ते।
यानि॑ घ॒र्मे क॒पाला᳚न्युपचि॒न्वन्ति॑ वे॒धस॒ इति॒ चतु॑ष्पदय॒र्चा वि मु॑ञ्चति।
चतु॑ष्पादः प॒शवः॑।
प॒शुष्वे॒वोपरि॑ष्टा॒त्प्रति॑ तिष्ठति॥५५॥\anuvakamend[व॒र्त॒य॒ति॒ दिव॑मे॒वैतेन॑ दृꣳहति स॒म्भव॑ति॒ तꣳ सꣴस्कृ॑तमा॒त्मानं॒ द्वाद॑श॒ सꣴस्थि॑ते॒ त्रीणि॑ च]

%3.2.8.1
दे॒वस्य॑ त्वा सवि॒तुः प्र॑स॒व इत्या॑ह॒ प्रसू᳚त्यै।
अ॒श्विनो᳚र्बा॒हुभ्या॒मि\-त्या॑ह।
अ॒श्विनौ॒ हि दे॒वाना॑मध्व॒र्यू आस्ता᳚म्।
पू॒ष्णो हस्ता᳚भ्या॒मित्या॑ह॒ यत्यै᳚।
सं व॑पा॒मीत्या॑ह।
य॒था॒दे॒व॒तमे॒वैना॑नि॒ संव॑पति।
समापो॑ अ॒द्भिर॑ग्मत॒ समोष॑धयो॒ रसे॒नेत्या॑ह।
आपो॒ वा ओष॑धीर्जिन्वन्ति।
ओष॑धयो॒ऽपो जि॑न्वन्ति।
अ॒न्या वा ए॒तासा॑म॒न्या जि॑न्वन्ति॥५६॥

%3.2.8.2
तस्मा॑दे॒वमा॑ह।
सꣳ रे॒वती॒र्जग॑तीभि॒र्मधु॑मती॒र्मधु॑मतीभिः सृज्यध्व॒मित्या॑ह।
आपो॒ वै रे॒वतीः᳚।
प॒शवो॒ जग॑तीः।
ओष॑धयो॒ मधु॑मतीः।
आप॒ ओष॑धीः प॒शून्।
ताने॒वास्मा॑ एक॒धा स॒ꣳ॒सृज्य॑।
मधु॑मतः करोति।
अ॒द्भ्यः परि॒ प्रजा॑ताः स्थ॒ सम॒द्भिः पृ॑च्यध्व॒मिति॑ प॒र्याप्ला॑वयति।
यथा॒ सुवृ॑ष्ट इ॒माम॑नुवि॒सृत्य॑॥५७॥

%3.2.8.3
आप॒ ओष॑धीर्म॒हय॑न्ति।
ता॒दृगे॒व तत्।
जन॑यत्यै त्वा॒ संयौ॒मीत्या॑ह।
प्र॒जा ए॒वैतेन॑ दाधार।
अ॒ग्नये᳚ त्वा॒\-ऽग्नीषोमा᳚भ्या॒मित्या॑ह॒ व्यावृ॑त्त्यै।
म॒खस्य॒ शिरो॒\-ऽसीत्या॑ह।
य॒ज्ञो वै म॒खः।
तस्यै॒तच्छिरः॑।
यत्पु॑रो॒डाशः॑।
तस्मा॑दे॒वमा॑ह॥५८॥

%3.2.8.4
घ॒र्मो॑ऽसि वि॒श्वायु॒रित्या॑ह।
विश्व॑मे॒वायु॒र्यज॑माने दधाति।
उ॒रु प्र॑थस्वो॒रु ते॑ य॒ज्ञप॑तिः प्रथता॒मित्या॑ह।
यज॑मानमे॒व प्र॒जया॑ प॒शुभिः॑ प्रथयति।
त्वचं॑ गृह्णी॒ष्वेत्या॑ह।
सर्व॑मे॒वैन॒ꣳ॒ सत॑नुं करोति।
अथा॒ऽऽप आ॒नीय॒ परि॑मार्ष्टि।
मा॒ꣳ॒स ए॒व तत्त्वचं॑ दधाति।
तस्मा᳚त्त्व॒चा मा॒ꣳ॒सं छ॒न्नम्।
घ॒र्मो वा ए॒षो\-ऽशा᳚न्तः॥५९॥

%3.2.8.5
अ॒र्ध॒मा॒से᳚\-ऽर्धमासे॒ प्रवृ॑ज्यते।
यत्पु॑रो॒डाशः॑।
स ई᳚श्व॒रो यज॑मानꣳ शु॒चा प्र॒दहः॑।
पर्य॑ग्नि करोति।
प॒शुमे॒वैन॑मकः।
शान्त्या॒ अप्र॑दाहाय।
त्रिः पर्य॑ग्नि करोति।
त्र्या॑वृ॒द्धि य॒ज्ञः।
अथो॒ रक्ष॑सा॒मप॑हत्यै।
अ॒न्तरि॑त॒ꣳ॒ रक्षो॒\-ऽन्तरि॑ता॒ अरा॑तय॒ इत्या॑ह॥६०॥

%3.2.8.6
रक्ष॑साम॒न्तर्\mbox{}हि॑त्यै।
पु॒रो॒डाशं॒ वा अधि॑श्रित॒ꣳ॒ रक्षाꣴ॑स्य\-जिघाꣳसन्।
दि॒वि नाको॒ नामा॒ग्नी र॑क्षो॒हा।
स ए॒वास्मा॒द्रक्षा॒ꣴ॒\-स्यपा॑\-हन्।
दे॒वस्त्वा॑ सवि॒ता श्र॑पय॒त्वित्या॑ह।
स॒वि॒तृप्र॑सूत ए॒वैनꣴ॑ श्रपयति।
वर्\mbox{}षि॑ष्ठे॒ अधि॒ नाक॒ इत्या॑ह।
रक्ष॑सा॒मप॑हत्यै।
अ॒ग्निस्ते॑ त॒नुवं॒ माऽति॑धा॒गित्या॒हा\-ऽन॑तिदाहाय।
अग्ने॑ ह॒व्यꣳ र॑क्ष॒स्वेत्या॑ह॒ गुप्त्यै᳚॥६१॥

%3.2.8.7
अवि॑दहन्तः श्रपय॒तेति॒ वाचं॒ विसृ॑जते।
य॒ज्ञमे॒व ह॒वीꣴष्य॑भि\-व्या॒हृत्य॒ प्रत॑नुते।
पु॒रो॒रुच॒मवि॑दाहाय॒ शृत्त्यै॑ करोति।
म॒स्तिष्को॒ वै पु॑रो॒डाशः॑।
तं यन्नाभि॑ वा॒सये᳚त्।
आ॒विर्म॒स्तिष्कः॑ स्यात्।
अ॒भिवा॑सयति।
तस्मा॒द्गुहा॑ म॒स्तिष्कः॑।
भस्म॑ना॒\-ऽभिवा॑सयति।
तस्मा᳚न्मा॒ꣳ॒सेनास्थि॑ छ॒न्नम्॥६२॥

%3.2.8.8
वे॒देना॒भिवा॑सयति।
तस्मा॒त्केशैः॒ शिर॑श्छ॒न्नम्।
अख॑लति\-भावुको भवति।
य ए॒वं वेद॑।
प॒शोर्वै प्र॑ति॒मा पु॑रो॒डाशः॑।
स नाय॒जुष्क॑मभि॒वास्यः॑।
वृथे॑व स्यात्।
ई॒श्व॒रा यज॑मानस्य प॒शवः॒ प्रमे॑तोः।
सं ब्रह्म॑णा पृच्य॒स्वेत्या॑ह।
प्रा॒णा वै ब्रह्म॑॥६३॥

%3.2.8.9
प्रा॒णाः प॒शवः॑।
प्रा॒णैरे॒व प॒शून्थ्सम्पृ॑णक्ति।
न प्र॒मायु॑का भवन्ति।
यज॑मानो॒ वै पु॑रो॒डाशः॑।
प्र॒जा प॒शवः॒ पुरी॑षम्।
यदे॒वम॑भिवा॒सय॑ति।
यज॑मानमे॒व प्र॒जया॑ प॒शुभिः॒ सम॑र्धयति।
दे॒वा वै ह॒विर्भृ॒त्वा\-ऽब्रु॑वन्।
कस्मि॑न्नि॒दं म्र॑क्ष्यामह॒ इति॑।
सो᳚ऽग्निर॑ब्रवीत्॥६४॥

%3.2.8.10
मयि॑ त॒नूः सं निध॑ध्वम्।
अ॒हं व॒स्तं ज॑नयिष्यामि।
यस्मि॑न्म्र॒क्ष्यध्व॒ इति॑।
ते दे॒वा अ॒ग्नौ त॒नूः सन्न्य॑दधत।
तस्मा॑दाहुः।
अ॒ग्निः सर्वा॑ दे॒वता॒ इति॑।
सोऽङ्गा॑रेणा॒ऽ॒ऽ॒पः।
अ॒भ्य॑पातयत्।
तत॑ एक॒तो॑\-ऽजायत।
स द्वि॒तीय॑म॒भ्य॑\-पातयत्॥६५॥

%3.2.8.11
ततो᳚ द्वि॒तो॑\-ऽजायत।
स तृ॒तीय॑म॒भ्य॑पातयत्।
तत॑स्त्रि॒तो॑\-ऽजायत।
यद॒द्भ्यो\-ऽजा॑यन्त।
तदा॒प्याना॑माप्य॒त्वम्।
यदा॒त्मभ्यो\-ऽजा॑यन्त।
तदा॒त्म्याना॑मात्म्य॒त्वम्।
ते दे॒वा आ॒प्येष्व॑मृजत।
आ॒प्या अ॑मृजत॒ सूर्या᳚भ्युदिते।
सूर्या᳚भ्युदितः॒ सूर्या॑भिनिम्रुक्ते॥६६॥

%3.2.8.12
सूर्या॑भिनिम्रुक्तः कुन॒खिनि॑।
कु॒न॒खी श्या॒वद॑ति।
श्या॒वद॑न्नग्र\-दिधि॒षौ।
अ॒ग्र॒दि॒धि॒षुः प॑रिवि॒त्ते।
प॒रि॒वि॒त्तो वी॑र॒हणि॑।
वी॒र॒हा ब्र॑ह्म॒हणि॑।
तद्ब्र॑ह्म॒हणं॒ नात्य॑च्यवत।
अ॒न्त॒र्वे॒दि निन॑य॒त्यव॑रुद्ध्यै।
उल्मु॑केना॒भि गृ॑ह्णाति शृत॒त्वाय॑।
शृ॒तका॑मा इव॒ हि दे॒वाः॥६७॥\anuvakamend[अ॒न्या जि॑न्वन्त्यनु वि॒सृत्यै॒वमा॒हाशा᳚न्त आह॒ गुप्त्यै॑ छ॒न्नं ब्रह्मा᳚ब्रवीद्द्वि॒तीय॑म॒भ्य॑पातय॒थ्सूर्या॑भिनिम्रुक्ते दे॒वाः]

%3.2.9.1
दे॒वस्य॑ त्वा सवि॒तुः प्र॑स॒व इति॒ स्फ्यमाद॑त्ते॒ प्रसू᳚त्यै।
अ॒श्विनो᳚र्बा॒हुभ्या॒मित्या॑ह।
अ॒श्विनौ॒ हि दे॒वाना॑मध्व॒र्यू आस्ता᳚म्।
पू॒ष्णो हस्ता᳚भ्या॒मित्या॑ह॒ यत्यै᳚।
आद॑द॒ इन्द्र॑स्य बा॒हुर॑सि॒ दक्षि॑ण॒ इत्या॑ह।
इ॒न्द्रि॒यमे॒व यज॑माने दधाति।
स॒हस्र॑भृष्टिः श॒तते॑जा॒ इत्या॑ह।
रू॒पमे॒वास्यै॒तन्म॑हि॒मानं॒ व्याच॑ष्टे।
वा॒युर॑सि ति॒ग्मते॑जा॒ इत्या॑ह।
तेजो॒ वै वा॒युः॥६८॥

%3.2.9.2
तेज॑ ए॒वास्मि॑न्दधाति।
वि॒षाद्वै नामा॑सु॒र आ॑सीत्।
सो॑ऽबिभेत्।
य॒ज्ञेन॑ मा दे॒वा अ॒भिभ॑विष्य॒न्तीति॑।
स पृ॑थि॒वीम॒भ्य॑वमीत्।
सा मे॒ध्या\-ऽभ॑वत्।
अथो॒ यदिन्द्रो॑ वृ॒त्रमहन्॑।
तस्य॒ लोहि॑तं पृथि॒वीमनु॒ व्य॑धावत्।
सा मे॒ध्या\-ऽभ॑वत्।
पृथि॑वि देवयज॒नीत्या॑ह॥६९॥

%3.2.9.3
मेध्या॑मे॒वैनां᳚ देव॒यज॑नीं करोति।
ओष॑ध्यास्ते॒ मूलं॒ मा हिꣳ॑सिष॒मित्या॑ह।
ओष॑धीना॒महिꣳ॑सायै।
व्र॒जं ग॑च्छ गो॒स्थान॒मित्या॑ह।
छन्दाꣳ॑सि॒ वै व्र॒जो गो॒स्थानः॑।
छन्दाꣴ॑स्ये॒वास्मै᳚ व्र॒जं गो॒स्थानं॑ करोति।
वर्\mbox{}ष॑तु ते॒ द्यौरित्या॑ह।
वृष्टि॒र्वै द्यौः।
वृष्टि॑मे॒वाव॑ रुन्धे।
ब॒धा॒न दे॑व सवितः पर॒मस्यां᳚ परा॒वतीत्या॑ह॥७०॥

%3.2.9.4
द्वौ वाव पुरु॑षौ।
यं चै॒व द्वेष्टि॑।
यश्चै॑नं॒ द्वेष्टि॑।
तावु॒भौ ब॑ध्नाति पर॒मस्यां᳚ परा॒वति॑ श॒तेन॒ पाशैः᳚।
यो᳚ऽस्मान्द्वेष्टि॒ यं च॑ व॒यं द्वि॒ष्मस्तमतो॒ मा मौ॒गित्या॒हानि॑म्रुक्त्यै।
अ॒ररु॒र्वै नामा॑सु॒र आ॑सीत्।
स पृ॑थि॒व्यामुप॑म्लुप्तो\-ऽशयत्।
तं दे॒वा अप॑हतो॒\-ऽररुः॑ पृथि॒व्या इति॑ पृथि॒व्या अपा᳚घ्नन्।
भ्रातृ॑व्यो॒ वा अ॒ररुः॑।
अप॑हतो॒\-ऽररुः॑ पृथि॒व्या इति॒ यदाह॑॥७१॥

%3.2.9.5
भ्रातृ॑व्यमे॒व पृ॑थि॒व्या अप॑हन्ति।
ते॑ऽमन्यन्त।
दिवं॒ वा अ॒यमि॒तः प॑तिष्य॒तीति॑।
तम॒ररु॑स्ते॒ दिवं॒ माऽस्का॒निति॑ दि॒वः पर्य॑बाधन्त।
भ्रातृ॑व्यो॒ वा अ॒ररुः॑।
अ॒ररु॑स्ते॒ दिवं॒ मा स्का॒निति॒ यदाह॑।
भ्रातृ॑व्यमे॒व दि॒वः परि॑बाधते।
स्त॒म्ब॒य॒जुर्‌\mbox{}ह॑रति।
पृ॒थि॒व्या ए॒व भ्रातृ॑व्य॒मप॑हन्ति।
द्वि॒तीयꣳ॑ हरति॥७२॥

%3.2.9.6
अ॒न्तरि॑क्षादे॒वैन॒मप॑हन्ति।
तृ॒तीयꣳ॑ हरति।
दि॒व ए॒वैन॒मप॑हन्ति।
तू॒ष्णीं च॑तु॒र्थꣳ ह॑रति।
अप॑रिमितादे॒वैन॒मप॑\-हन्ति।
असु॑राणां॒ वा इ॒यमग्र॑ आसीत्।
याव॒दासी॑नः परा॒पश्य॑ति।
ताव॑द्दे॒वाना᳚म्।
ते दे॒वा अ॑ब्रुवन्।
अस्त्वे॒व नो॒\-ऽस्यामपीति॑॥७३॥

%3.2.9.7
क्य॑न्नो दास्य॒थेति॑।
याव॑थ्स्व॒यं प॑रिगृह्णी॒थेति॑।
ते वस॑व॒स्त्वेति॑ दक्षिण॒तः पर्य॑गृह्णन्।
रु॒द्रास्त्वेति॑ प॒श्चात्।
आ॒दि॒त्यास्त्वेत्यु॑त्तर॒तः।
ते᳚ऽग्निना॒ प्राञ्चो॑\-ऽजयन्।
वसु॑भिर्दक्षि॒णा।
रु॒द्रैः प्र॒त्यञ्चः॑।
आ॒दि॒त्यैरुद॑ञ्चः।
यस्यै॒वं वि॒दुषो॒ वेदिं॑ परिगृ॒ह्णन्ति॑॥७४॥

%3.2.9.8
भव॑त्या॒त्मना᳚।
परा᳚ऽस्य॒ भ्रातृ॑व्यो भवति।
दे॒वस्य॑ सवि॒तुः स॒व इत्या॑ह॒ प्रसू᳚त्यै।
कर्म॑ कृण्वन्ति वे॒धस॒ इत्या॑ह।
इ॒षि॒तꣳ हि कर्म॑ क्रि॒यते᳚।
पृ॒थि॒व्यै मेध्यं॑ चामे॒ध्यं च॒ व्युद॑क्रामताम्।
प्रा॒चीन॑मुदी॒चीनं॒ मेध्यम्᳚।
प्र॒ती॒चीनं॑ दक्षि॒णा\-ऽमे॒ध्यम्।
प्राची॒मुदी॑चीं प्रव॒णां क॑रोति।
मेध्या॑मे॒वैनां᳚ देव॒यज॑नीं करोति॥७५॥

%3.2.9.9
प्राञ्चौ॑ वेद्य॒ꣳ॒सावुन्न॑यति।
आ॒ह॒व॒नीय॑स्य॒ परि॑गृहीत्यै।
प्र॒तीची॒ श्रोणी᳚।
गार्ह॑पत्यस्य॒ परि॑गृहीत्यै।
अथो॑ मिथुन॒त्वाय॑।
उद्ध॑न्ति।
यदे॒वास्या॑ अमे॒ध्यम्।
तदप॑हन्ति।
उद्ध॑न्ति।
तस्मा॒दोष॑धयः॒ परा॑भवन्ति॥७६॥

%3.2.9.10
मूलं॑ छिनत्ति।
भ्रातृ॑व्यस्यै॒व मूलं॑ छिनत्ति।
मूलं॒ वा अ॑ति॒तिष्ठ॒द्रक्षा॒ꣴ॒स्यनूत्पि॑पते।
यद्धस्ते॑न छि॒न्द्यात्।
कु॒न॒खिनीः᳚ प्र॒जाः स्युः॑।
स्फ्येन॑ छिनत्ति।
वज्रो॒ वै स्फ्यः।
वज्रे॑णै॒व य॒ज्ञाद्रक्षा॒ꣴ॒स्यप॑हन्ति।
पि॒तृ॒दे॒व॒त्या\-ऽति॑खाता।
इय॑तीं खनति॥७७॥

%3.2.9.11
प्र॒जा\-प॑तिना यज्ञमु॒खेन॒ सम्मि॑ताम्।
वेदि॑र्दे॒वेभ्यो॒ निला॑यत।
तां च॑तुरङ्गु॒ले\-ऽन्व॑विन्दन्।
तस्मा᳚च्चतुरङ्गु॒लं खेया᳚।
च॒तु॒र॒ङ्गु॒लं ख॑नति।
च॒तु॒र॒ङ्गु॒ले ह्योष॑धयः प्रति॒तिष्ठ॑न्ति।
आ प्र॑ति॒ष्ठायै॑ खनति।
यज॑मानमे॒व प्र॑ति॒ष्ठां ग॑मयति।
द॒क्षि॒ण॒तो वर्\mbox{}षी॑यसीं करोति।
दे॒व॒यज॑नस्यै॒व रू॒पम॑कः॥७८॥

%3.2.9.12
पुरी॑षवतीं करोति।
प्र॒जा वै प॒शवः॒ पुरी॑षम्।
प्र॒जयै॒वैनं॑ प॒शुभिः॒ पुरी॑षवन्तं करोति।
उत्त॑रं परिग्रा॒हं परि॑गृह्णाति।
ए॒ताव॑ती॒ वै पृ॑थि॒वी।
याव॑ती॒ वेदिः॑।
तस्या॑ ए॒ताव॑त ए॒व भ्रातृ॑व्यं नि॒र्भज्य॑।
आ॒त्मन॒ उत्त॑रं परिग्रा॒हं परि॑गृह्णाति।
ऋ॒तम॑स्यृत॒सद॑नमस्यृत॒श्रीर॒सीत्या॑ह।
य॒था॒\-य॒जु\-रे॒वै\-तत्॥७९॥

%3.2.9.13
क्रू॒रमि॑व॒ वा ए॒तत्क॑रोति।
यद्वेदिं॑ क॒रोति॑।
धा अ॑सि स्व॒धा अ॒सीति॑ योयुप्यते॒ शान्त्यै᳚।
उ॒र्वी चासि॒ वस्वी॑ चा॒सीत्या॑ह।
उ॒र्वीमे॒वैनां॒ वस्वीं᳚ करोति।
पु॒रा क्रू॒रस्य॑ वि॒सृपो॑ विरफ्शि॒न्नित्या॑ह मेध्य॒त्वाय॑।
उ॒दा॒दाय॑ पृथि॒वीं जी॒रदा॑नु॒र्यामैर॑यं च॒न्द्रम॑सि स्व॒धाभि॒रित्या॑ह।
यदे॒वास्या॑ अमे॒ध्यम्।
तद॑प॒हत्य॑।
मेध्यां᳚ देव॒यज॑नीं कृ॒त्वा॥८०॥

%3.2.9.14
यद॒दश्च॒न्द्रम॑सि॒ मेध्यम्᳚।
तद॒स्यामेर॑यति।
तां धीरा॑सो अनु॒दृश्य॑ यजन्त॒ इत्या॒हानु॑ख्यात्यै।
प्रोक्ष॑णी॒रा सा॑दय।
इ॒ध्माब॒र्॒\mbox{}हिरुप॑सादय।
स्रु॒वं च॒ स्रुच॑श्च॒ सम्मृ॑ड्ढि।
पत्नी॒ꣳ॒ सन्न॑ह्य।
आज्ये॑नो॒देहीत्या॑हानुपू॒र्वता॑यै।
प्रोक्ष॑णी॒रा सा॑दयति।
आपो॒ वै र॑क्षो॒घ्नीः॥८१॥

%3.2.9.15
रक्ष॑सा॒मप॑हत्यै।
स्फ्यस्य॒ वर्त्म᳚न्थ्सादयति।
य॒ज्ञस्य॒ सन्त॑त्यै।
उ॒वाच॒ हासि॑तो दैव॒लः।
ए॒ताव॑ती॒र्वा अ॒मुष्मिँ॑ल्लो॒क आप॑ आसन्।
याव॑तीः॒ प्रोक्ष॑णी॒रिति॑।
तस्मा᳚द्ब॒ह्वीरा॒साद्याः᳚।
स्फ्यमु॒दस्यन्॑।
यं द्वि॒ष्यात्तं ध्या॑येत्।
शु॒चैवैन॑मर्पयति॥८२॥\anuvakamend[वै वा॒युरा॑ह परा॒वतीत्या॒हाह॑ द्वि॒तीयꣳ॑ हर॒तीति॑ परिगृ॒ह्णन्ति॑ देव॒यज॑नीं करोति भवन्ति खनत्यकरे॒तत्कृ॒त्वा र॑क्षो॒घ्नीर॑र्पयति]

%3.2.10.1
वज्रो॒ वै स्फ्यः।
यद॒न्वञ्चं॑ धा॒रये᳚त्।
वज्रे᳚\-ऽध्व॒र्युः क्ष॑ण्वीत।
पु॒रस्ता᳚त्ति॒र्यञ्चं॑ धारयति।
वज्रो॒ वै स्फ्यः।
वज्रे॑णै॒व य॒ज्ञस्य॑ दक्षिण॒तो रक्षा॒ꣴ॒स्यप॑हन्ति।
अ॒ग्निभ्यां॒ प्राच॑श्च प्र॒तीच॑श्च।
स्फ्येनोदी॑चश्चाध॒राच॑श्च।
स्फ्येन॒ वा ए॒ष वज्रे॑णा॒स्यै पा॒प्मानं॒ भ्रातृ॑व्यमप॒हत्य॑।
उ॒त्क॒रेऽधि॒ प्रवृ॑श्चति॥८३॥

%3.2.10.2
यथो॑प॒धाय॑ वृ॒श्चन्त्ये॒वम्।
हस्ता॒वव॑ नेनिक्ते।
आ॒त्मान॑मे॒व प॑वयते।
स्फ्यं प्रक्षा॑लयति मेध्य॒त्वाय॑।
अथो॑ पा॒प्मन॑ ए॒व भ्रातृ॑व्यस्य न्य॒ङ्गं छि॑नत्ति।
इ॒ध्माब॒र्॒\mbox{}हिरुप॑सादयति॒ युक्त्यै᳚।
य॒ज्ञस्य॑ मिथुन॒त्वाय॑।
अथो॑ पुरो॒रुच॑मे॒वैतां द॑धाति।
उत्त॑रस्य॒ कर्म॒णो\-ऽनु॑ख्यात्यै।
न पु॒रस्ता᳚त्प्र॒त्यगुप॑सादयेत्॥८४॥

%3.2.10.3
यत्पु॒रस्ता᳚त्प्र॒त्यगु॑पसा॒दये᳚त्।
अ॒न्यत्रा॑ऽऽहुतिप॒थादि॒ध्मं प्रति॑\-पादयेत्।
प्र॒जा वै ब॒र्॒हिः।
अप॑राध्नुयाद्ब॒र्॒हिषा᳚ प्र॒जानां᳚ प्र॒जन॑नम्।
प॒श्चात्प्रागुप॑सादयति।
आ॒हु॒ति॒प॒थेने॒ध्मं प्रति॑\-पादयति।
स॒म्प्र॒त्ये॑व ब॒र्॒हिषा᳚ प्र॒जानां᳚ प्र॒जन॑न॒मुपै॑ति।
दक्षि॑णमि॒ध्मम्।
उत्त॑रं ब॒र्॒हिः।
आ॒त्मा वा इ॒ध्मः।
प्र॒जा ब॒र्॒हिः।
प्र॒जा ह्या᳚त्मन॒ उत्त॑रतरा ती॒र्थे।
ततो॒ मेध॑मुप॒नीय॑।
य॒था॒दे॒व॒तमे॒वैन॒त्प्रति॑\-ष्ठापयति।
प्रति॑ तिष्ठति प्र॒जया॑ प॒शुभि॒र्यज॑मानः॥८५॥\anuvakamend[वृ॒श्च॒ति॒ सा॒द॒ये॒दि॒ध्मः पञ्च॑ च]




\prashnaend{तृ॒तीय॑स्यां दे॒वस्या᳚श्वप॒र्॒\mbox{}शुं यो वै पू᳚र्वे॒द्युः कर्म॑णे वा॒मिन्द्रो॑ वृ॒त्रम॑ह॒न्थ्सो॑\-ऽपो\-ऽव॑धूतं॒ धृष्टि॑र्दे॒वस्येत्या॑ह॒ सं व॑पामि दे॒वस्य॒ स्फ्यमा द॑दे॒ वज्रो॒ वै स्फ्यो दश॑॥१०॥}{तृ॒तीय॑स्यां य॒ज्ञस्यान॑तिरेकाय प॒वित्र॑वत्यध्व॒र्युं चा॑धि॒षव॑णमस्य॒न्तरि॑क्ष ए॒व रक्ष॑साम॒न्तर्\mbox{}हि॑त्यै॒ द्वौ वाव पुरु॑षौ॒ यद॒दश्च॒न्द्रम॑सि॒ मेध्यं॒ पञ्चाशी॑तिः॥८५॥}{तृ॒तीय॑स्यां॒ यज॑मानः॥}{हरिः॑ ओम्॥}{इति श्रीकृष्णयजुर्वेदीयतैत्तिरीयब्राह्मणे तृतीयाष्टके द्वितीयः प्रपाठकः समाप्तः॥}
\clearpage
\sect{तृतीयः प्रश्नः}
\setcounter{anuvakam}{0}
\dnsub{तैत्तिरीयब्राह्मणे तृतीयाष्टके तृतीयः प्रपाठकः}

%3.3.1.1
प्रत्यु॑ष्ट॒ꣳ॒ रक्षः॒ प्रत्यु॑ष्टा॒ अरा॑तय॒ इत्या॑ह।
रक्ष॑सा॒मप॑हत्यै।
अ॒ग्नेर्व॒स्तेजि॑ष्ठेन॒ तेज॑सा॒ निष्ट॑पा॒मीत्या॑ह मेध्य॒त्वाय॑।
स्रुचः॒ सम्मा᳚र्ष्टि।
स्रु॒वमग्रे᳚।
पुमाꣳ॑समे॒वाभ्यः॒ सꣴश्य॑ति मिथुन॒त्वाय॑।
अथ॑ जु॒हूम्।
अथो॑प॒भृतम्᳚।
अथ॑ ध्रु॒वाम्।
अ॒सौ वै जु॒हूः॥१॥

%3.3.1.2
अ॒न्तरि॑क्षमुप॒भृत्।
पृ॒थि॒वी ध्रु॒वा।
इ॒मे वै लो॒काः स्रुचः॑।
वृष्टिः॑ स॒म्मार्ज॑नानि।
वृष्टि॒र्वा इ॒माँल्लो॒कान॑नुपू॒र्वं क॑ल्पयति।
ते ततः॑ कॢ॒प्ताः समे॑धन्ते।
समे॑धन्ते\-ऽस्मा इ॒मे लो॒काः प्र॒जया॑ प॒शुभिः॑।
य ए॒वं वेद॑।
यदि॑ का॒मये॑त॒ वर्‌\mbox{}षु॑कः प॒र्जन्यः॑ स्या॒दिति॑।
अ॒ग्र॒तः सम्मृ॑ज्यात्॥२॥

%3.3.1.3
वृष्टि॑मे॒व नि य॑च्छति।
अ॒वा॒चीना᳚ग्रा॒ हि वृष्टिः॑।
यदि॑ का॒मये॒ताव॑र्‌\mbox{}षुकः स्या॒दिति॑।
मू॒ल॒तः सम्मृ॑ज्यात्।
वृष्टि॑मे॒वोद्य॑च्छति।
तदु॒ वा आ॑हुः।
अ॒ग्र॒त ए॒वोपरि॑ष्टा॒थ्सम्मृ॑\-ज्यात्।
मू॒ल॒तो॑\-ऽधस्ता᳚त्।
तद॑नुपू॒र्वं क॑ल्पते।
वर्‌\mbox{}षु॑को भव॒तीति॑॥३॥

%3.3.1.4
प्राची॑मभ्या॒कारम्᳚।
अग्रै॑रन्तर॒तः।
ए॒वमि॑व॒ ह्यन्न॑म॒द्यते᳚।
अथो॒ अग्रा॒द्वा ओष॑धीना॒मूर्जं॑ प्र॒जा उप॑जीवन्ति।
ऊ॒र्ज ए॒वान्नाद्य॒स्या\-व॑\-रुद्ध्यै।
अ॒धस्ता᳚त्प्र॒तीची᳚म्।
द॒ण्डमु॑त्तम॒तः।
मूले॑न॒ मूलं॒ प्रति॑\-ष्ठित्यै।
तस्मा॑दर॒त्नौ प्राञ्च्यु॒परि॑ष्टा॒ल्लोमा॑नि।
प्र॒त्यञ्च्य॒धस्ता᳚त्॥४॥

%3.3.1.5
स्रुग्घ्ये॑षा।
प्रा॒णो वै स्रु॒वः।
जु॒हूर्दक्षि॑णो॒ हस्तः॑।
उ॒प॒भृथ्स॒व्यः।
आ॒त्मा ध्रु॒वा।
अन्नꣳ॑ स॒म्मार्ज॑नानि।
मु॒ख॒तो वै प्रा॒णो॑\-ऽपा॒नो भू॒त्वा।
आ॒त्मान॒मन्नं॑ प्र॒विश्य॑।
बा॒ह्य॒तस्त॒नुवꣳ॑ शुभयति।
तस्मा᳚थ्स्रु॒वमे॒वाग्रे॒ सम्मा᳚र्ष्टि।
मु॒ख॒तो हि प्रा॒णो॑\-ऽपा॒नो भू॒त्वा।
आ॒त्मान॒मन्न॑मावि॒शति॑।
तौ प्रा॑णापा॒नौ।
अव्य॑र्धुकः प्राणापा॒नाभ्यां᳚ भवति।
य ए॒वं वेद॑॥५॥\anuvakamend[जु॒हूर्मृ॑ज्याद्भव॒तीति॑ प्र॒त्यञ्च्य॒धस्ता᳚न्मार्ष्टि॒ पञ्च॑ च]

%3.3.2.1
दि॒वः शिल्प॒मव॑ततम्।
पृ॒थि॒व्याः क॒कुभि॑ श्रि॒तम्।
तेन॑ व॒यꣳ स॒हस्र॑वल्‌शेन।
स॒पत्नं॑ नाशयामसि॒ स्वाहेति॑ स्रुख्स॒म्मार्ज॑नान्य॒ग्नौ प्र ह॑रति।
आपो॒ वै द॒र्भाः।
रू॒पमे॒वैषा॑मे॒तन्म॑हि॒मानं॒ व्याच॑ष्टे।
अ॒नु॒ष्टुभ॒र्चा।
आनु॑ष्टुभः प्र॒जा\-प॑तिः।
प्रा॒जा॒प॒त्यो वे॒दः।
वे॒दस्याग्रꣴ॑ स्रुख्स॒म्मार्ज॑नानि॥६॥

%3.3.2.2
स्वेनै॒वैना॑नि॒ छन्द॑सा।
स्वया॑ दे॒वत॑या॒ सम॑र्धयति।
अथो॒ ऋग्वाव योषा᳚।
द॒र्भो वृषा᳚।
तन्मि॑थु॒नम्।
मि॒थु॒नमे॒वास्य॒ तद्य॒ज्ञे क॑रोति प्र॒जन॑नाय।
प्रजा॑यते प्र॒जया॑ प॒शुभि॒र्यज॑मानः।
तान्येके॒ वृथै॒वापा᳚स्यन्ति।
तत्तथा॒ न का॒र्यम्᳚।
आर॑ब्धस्य य॒ज्ञिय॑स्य॒ कर्म॑णः॒ सवि॑दो॒हः॥७॥

%3.3.2.3
यद्ये॑नानि प॒शवो॑ऽभि॒ तिष्ठे॑युः।
न तत्प॒शुभ्यः॒ कम्।
अ॒द्भिर्मा᳚र्जयि॒त्वोत्क॒रे न्य॑स्येत्।
यद्वै य॒ज्ञिय॑स्य॒ कर्म॑णो॒\-ऽन्यत्राऽऽहु॑तीभ्यः स॒न्तिष्ठ॑ते।
उ॒त्क॒रो वाव तस्य॑ प्रति॒ष्ठा।
ए॒ताꣳ हि तस्मै᳚ प्रति॒ष्ठां दे॒वाः स॒मभ॑रन्।
यद॒द्भिर्मा॒र्जय॑ति।
तेन॑ शा॒न्तम्।
यदु॑त्क॒रे न्य॒स्यति॑।
प्र॒ति॒ष्ठामे॒वैना॑नि॒ तद्ग॑मयति॥८॥

%3.3.2.4
प्रति॑ तिष्ठति प्र॒जया॑ प॒शुभि॒र्यज॑मानः।
अथो᳚ स्त॒म्बस्य॒ वा ए॒तद्रू॒पम्।
यथ्स्रु॑ख्स॒म्मार्ज॑नानि।
स्त॒म्ब॒शो वा ओष॑धयः।
तासां᳚ जरत्क॒क्षे प॒शवो॒ न र॑मन्ते।
अप्रि॑यो॒ ह्ये॑षां जरत्क॒क्षः।
याव॑दप्रियो ह॒ वै ज॑रत्क॒क्षः प॑शू॒नाम्।
ताव॑दप्रियः पशू॒नां भ॑वति।
यस्यै॒तान्य॒न्यत्रा॒ग्नेर्दध॑ति।
न॒व॒दाव्या॑सु॒ वा ओष॑धीषु प॒शवो॑ रमन्ते॥९॥

%3.3.2.5
न॒व॒दा॒वो ह्ये॑षां प्रि॒यः।
याव॑त्प्रियो ह॒ वै न॑वदा॒वः प॑शू॒नाम्।
ताव॑त्प्रियः पशू॒नां भ॑वति।
यस्यै॒तान्य॒ग्नौ प्र॒हर॑न्ति।
तस्मा॑दे॒तान्य॒ग्नावे॒व प्रह॑रेत्।
य॒त॒रस्मि᳚न्थ्सम्मृ॒ज्यात्।
प॒शू॒नां धृत्यै᳚।
यो भू॒ताना॒मधि॑पतिः।
रु॒द्रस्त॑न्तिच॒रो वृषा᳚।
प॒शून॒स्माकं॒ मा हिꣳ॑सीः।
ए॒तद॑स्तु हु॒तं तव॒ स्वाहेत्य॑ग्निस॒म्मार्ज॑नान्य॒ग्नौ प्रह॑रति।
ए॒षा वा ए॒तेषां॒ योनिः॑।
ए॒षा प्र॑ति॒ष्ठा।
स्वामे॒वैना॑नि॒ योनिम्᳚।
स्वां प्र॑ति॒ष्ठां ग॑मयति।
प्रति॑ तिष्ठति प्र॒जया॑ प॒शुभि॒र्यज॑मानः॥१०॥\anuvakamend[वे॒दस्याग्रꣴ॑ स्रुख्स॒म्मार्ज॑नानि विदो॒हो ग॑मयति प॒शवो॑ रमन्ते हिꣳसीः॒ षट् च॑]

%3.3.3.1
अय॑ज्ञो॒ वा ए॒षः।
यो॑ऽप॒त्नीकः॑।
न प्र॒जाः प्रजा॑येरन्।
पत्न्यन्वा᳚स्ते।
य॒ज्ञमे॒वाकः॑।
प्र॒जानां᳚ प्र॒जन॑नाय।
यत्तिष्ठ॑न्ती स॒न्नह्ये॑त।
प्रि॒यं ज्ञा॒तिꣳ रु॑न्ध्यात्।
आसी॑ना॒ सन्न॑ह्यते।
आसी॑ना॒ ह्ये॑षा वी॒र्यं॑ क॒रोति॑॥११॥

%3.3.3.2
यत्प॒श्चात्प्राच्य॒न्वासी॑त।
अ॒नया॑ स॒मद॑न्दधीत।
दे॒वानां॒ पत्नि॑या स॒मद॑न्दधीत।
देशा᳚द्दक्षिण॒त उदी॒च्यन्वा᳚स्ते।
आ॒त्मनो॑ गोपी॒थाय॑।
आ॒शासा॑ना सौमन॒समित्या॑ह।
मेध्या॑मे॒वैनां॒ केव॑लीं कृ॒त्वा।
आ॒शिषा॒ सम॑र्धयति।
अ॒ग्नेरनु॑\-व्रता भू॒त्वा सन्न॑ह्ये सुकृ॒ताय॒ कमित्या॑ह।
ए॒तद्वै पत्नि॑यै व्रतोप॒नय॑नम्॥१२॥

%3.3.3.3
तेनै॒वैनां᳚ व्र॒तमुप॑नयति।
तस्मा॑दाहुः।
यश्चै॒वं वेद॒ यश्च॒ न।
योक्त्र॑मे॒व यु॑ते।
यम॒न्वास्ते᳚।
तस्या॒मुष्मिँ॑ल्लो॒के भ॑व॒तीति॒ योक्त्रे॑ण।
यद्योक्त्रम्᳚।
स योगः॑।
यदास्ते᳚।
स क्षेमः॑॥१३॥

%3.3.3.4
यो॒ग॒क्षे॒मस्य॒ कॢप्त्यै᳚।
यु॒क्तं क्रि॑याता आ॒शीः कामे॑ युज्याता॒ इति॑।
आ॒शिषः॒ समृ॑द्ध्यै।
ग्र॒न्थिं ग्र॑थ्नाति।
आ॒शिष॑ ए॒वास्यां॒ परि॑ गृह्णाति।
पुमा॒न्॒ वै ग्र॒न्थिः।
स्त्री पत्नी᳚।
तन्मि॑थु॒नम्।
मि॒थु॒नमे॒वास्य॒ तद्य॒ज्ञे क॑रोति प्र॒जन॑नाय।
प्र जा॑यते प्र॒जया॑ प॒शुभि॒र्यज॑मानः॥१४॥

%3.3.3.5
अथो॑ अ॒र्धो वा ए॒ष आ॒त्मनः॑।
यत्पत्नी᳚।
य॒ज्ञस्य॒ धृत्या॒ अशि॑थिलं भावाय।
सु॒प्र॒जस॑स्त्वा व॒यꣳ सु॒पत्नी॒रुप॑ सेदि॒मेत्या॑ह।
य॒ज्ञमे॒व तन्मि॑थु॒नीक॑रोति।
ऊ॒नेऽति॑रिक्तं धीयाता॒ इति॒ प्रजा᳚त्यै।
म॒ही॒नां पयो॒\-ऽस्योष॑धीना॒ꣳ॒ रस॒ इत्या॑ह।
रू॒पमे॒वास्यै॒तन्म॑हि॒मानं॒ व्याच॑ष्टे।
तस्य॒ तेऽक्षी॑यमाणस्य॒ निर्व॑पामि देवय॒ज्याया॒ इत्या॑ह।
आ॒\-मे॒वैतामा शा᳚स्ते॥१५॥\anuvakamend[क॒रोति॑ व्रतोप॒नय॑नं॒ क्षेमो॒ यज॑मानः शास्ते]

%3.3.4.1
घृ॒तं च॒ वै मधु॑ च प्र॒जा\-प॑तिरासीत्।
यतो॒ मध्वा॑सीत्।
ततः॑ प्र॒जा अ॑\-सृजत।
तस्मा॒न्मधु॑षि प्र॒जन॑नमिवास्ति।
तस्मा॒न्मधु॑षा॒ न प्रच॑रन्ति।
या॒तया॑म॒ हि।
आज्ये॑न॒ प्रच॑रन्ति।
य॒ज्ञो वा आज्यम्᳚।
य॒ज्ञेनै॒व य॒ज्ञं प्रच॑र॒न्त्यया॑तयामत्वाय।
पत्न्यवे᳚क्षते॥१६॥

%3.3.4.2
मि॒थु॒न॒त्वाय॒ प्रजा᳚त्यै।
यद्वै पत्नी॑ य॒ज्ञस्य॑ क॒रोति॑।
मि॒थु॒नं तत्।
अथो॒ पत्नि॑या ए॒वैष य॒ज्ञस्या᳚न्वार॒म्भो\-ऽन॑वच्छित्त्यै।
अ॒मे॒ध्यं वा ए॒तत्क॑रोति।
यत्पत्न्य॒वेक्ष॑ते।
गार्‌\mbox{}ह॑प॒त्येऽधि॑ श्रयति मेध्य॒त्वाय॑।
आ॒ह॒व॒नीय॑म॒भ्युद्द्र॑वति।
य॒ज्ञस्य॒ सन्त॑त्यै।
तेजो॑ऽसि॒ तेजोऽनु॒ प्रेहीत्या॑ह॥१७॥

%3.3.4.3
तेजो॒ वा अ॒ग्निः।
तेज॒ आज्यम्᳚।
तेज॑सै॒व तेजः॒ सम॑र्धयति।
अ॒ग्निस्ते॒ तेजो॒ मा विनै॒दित्या॒हाहिꣳ॑सायै।
स्फ्यस्य॒ वर्त्म᳚न्थ्सादयति।
य॒ज्ञस्य॒ सन्त॑त्यै।
अ॒ग्नेर्जि॒ह्वाऽसि॑ सु॒भूर्दे॒वाना॒मित्या॑ह।
य॒था॒\-य॒जु\-रे॒वै\-तत्।
धाम्ने॑धाम्ने दे॒वेभ्यो॒ यजु॑षेयजुषे भ॒वेत्या॑ह।
आ॒\-मे॒वैतामा शा᳚स्ते॥१८॥

%3.3.4.4
तद्वा अतः॑ प॒वित्रा᳚भ्यामे॒वोत्पु॑नाति।
यज॑मानो॒ वा आज्यम्᳚।
प्रा॒णा॒पा॒नौ प॒वित्रे᳚।
यज॑मान ए॒व प्रा॑णापा॒नौ द॑धाति।
पु॒न॒रा॒हारम्᳚।
ए॒वमि॑व॒ हि प्रा॑णापा॒नौ स॒ञ्चर॑तः।
शु॒क्रम॑सि॒ ज्योति॑रसि॒ तेजो॒\-ऽसीत्या॑ह।
रू॒पमे॒वास्यै॒तन्म॑हि॒मानं॒ व्याच॑ष्टे।
त्रिर्यजु॑षा।
त्रय॑ इ॒मे लो॒काः॥१९॥

%3.3.4.5
ए॒षां लो॒काना॒माप्त्यै᳚।
त्रिः।
त्र्या॑वृ॒द्धि य॒ज्ञः।
अथो॑ मेध्य॒त्वाय॑।
अथाऽऽज्य॑वतीभ्याम॒पः।
रू॒पमे॒वासा॑मे॒तद्वर्णं॑ दधाति।
अपि॒ वा उ॒ताऽऽहुः॑।
यथा॑ ह॒ वै योषा॑ सु॒वर्ण॒ꣳ॒ हिर॑ण्यं पेश॒लं बिभ्र॑ती रू॒पाण्यास्ते᳚।
ए॒वमे॒ता ए॒तर्\mbox{}हीति॑।
आपो॒ वै सर्वा॑ दे॒वताः᳚॥२०॥

%3.3.4.6
ए॒षा हि विश्वे॑षां दे॒वानां᳚ त॒नूः।
यदाज्यम्᳚।
तत्रो॒भयो᳚र्मीमा॒ꣳ॒सा।
जा॒मि स्यात्।
यद्यजु॒षा\-ऽऽज्यं॒ यजु॑षा॒ऽप उ॑त्पुनी॒यात्।
छन्द॑सा॒ऽप उत्पु॑ना॒त्यजा॑मित्वाय।
अथो॑ मिथुन॒त्वाय॑।
सा॒वि॒त्रि॒यर्चा।
स॒वि॒तृप्र॑सूतं मे॒ कर्मा॑स॒दिति॑।
स॒वि॒तृप्र॑सूतमे॒वास्य॒ कर्म॑ भवति।
प॒च्छो गा॑यत्रि॒या त्रि॑ष्षमृद्ध॒त्वाय॑।
अ॒द्भिरे॒वौष॑धीः॒ सं न॑यति।
ओष॑धीभिः प॒शून्।
प॒शुभि॒र्यज॑मानम्।
शु॒क्रं त्वा॑ शु॒क्रायां॒ ज्योति॑स्त्वा॒ ज्योति॑ष्य॒र्चिस्त्वा॒\-ऽर्चिषीत्या॑ह सर्व॒त्वाय॑।
पर्या᳚प्त्या॒ अन॑न्तरायाय॥२१॥\anuvakamend[ई॒क्ष॒त॒ आ॒ह॒ शा॒स्ते॒ लो॒का दे॒वता॑ भवति॒ षट् च॑]

%3.3.5.1
दे॒वा॒सु॒राः संय॑त्ता आसन्।
स ए॒तमिन्द्र॒ आज्य॑स्याव\-का॒शम॑\-पश्यत्।
तेनावै᳚क्षत।
ततो॑ दे॒वा अभ॑वन्।
पराऽसु॑राः।
य ए॒वं वि॒द्वानाज्य॑म॒वेक्ष॑ते।
भव॑त्या॒त्मना᳚।
परा᳚ऽस्य॒ भ्रातृ॑व्यो भवति।
ब्र॒ह्म॒वा॒दिनो॑ वदन्ति।
यदाज्ये॑ना॒न्यानि॑ ह॒वीꣴष्य॑भिघा॒रय॑ति॥२२॥

%3.3.5.2
अथ॒ केनाऽऽज्य॒मिति॑।
स॒त्येनेति॑ ब्रूयात्।
चक्षु॒र्वै स॒त्यम्।
स॒त्येनै॒वैन॑द॒भि घा॑रयति।
ई॒श्व॒रो वा ए॒षो᳚\-ऽन्धो भवि॑तोः।
यश्चक्षु॒षा\-ऽऽज्य॑म॒वेक्ष॑ते।
नि॒मील्यावे᳚क्षेत।
दा॒धारा॒त्मञ्चक्षुः॑।
अ॒भ्याज्यं॑ घारयति।
आज्यं॑ गृह्णाति॥२३॥

%3.3.5.3
छन्दाꣳ॑सि॒ वा आज्यम्᳚।
छन्दाꣴ॑स्ये॒व प्री॑णाति।
च॒तुर्जु॒ह्वां गृ॑ह्णाति।
चतु॑ष्पादः प॒शवः॑।
प॒शूने॒वाव॑ रुन्धे।
अ॒ष्टावु॑प॒भृति॑।
अ॒ष्टाक्ष॑रा गाय॒त्री।
गा॒य॒त्रः प्रा॒णः।
प्रा॒णमे॒व प॒शुषु॑ दधाति।
च॒तुर्ध्रु॒वाया᳚म्॥२४॥

%3.3.5.4
चतु॑ष्पादः प॒शवः॑।
प॒शुष्वे॒वोपरि॑ष्टा॒त्प्रति॑ तिष्ठति।
य॒ज॒मा॒न॒दे॒व॒त्या॑ वै जु॒हूः।
भ्रा॒तृ॒व्य॒दे॒व॒त्यो॑प॒भृत्।
च॒तुर्जु॒ह्वां गृ॒ह्णन्भूयो॑ गृह्णीयात्।
अ॒ष्टावु॑प॒भृति॑ गृ॒ह्णन्कनी॑यः।
यज॑मानायै॒व भ्रातृ॑व्य॒मुप॑स्तिं करोति।
गौर्वै स्रुचः॑।
च॒तुर्जु॒ह्वां गृ॑ह्णाति।
तस्मा॒च्चतु॑ष्पदी॥२५॥

%3.3.5.5
अ॒ष्टावु॑प॒भृति॑।
तस्मा॑द॒ष्टाश॑फा।
च॒तुर्ध्रु॒वाया᳚म्।
तस्मा॒च्चतुः॑ स्तना।
गामे॒व तथ्सꣴस्क॑रोति।
साऽस्मै॒ सꣴस्कृ॒तेष॒मूर्जं॑ दुहे।
यज्जु॒ह्वां गृ॒ह्णाति॑।
प्र॒या॒जेभ्य॒स्तत्।
यदु॑प॒भृति॑।
प्र॒या॒जा॒नू॒या॒जेभ्य॒स्तत्।
सर्व॑स्मै॒ वा ए॒तद्य॒ज्ञाय॑ गृह्यते।
यद्ध्रु॒वाया॒माज्यम्᳚॥२६॥\anuvakamend[अ॒भि॒घा॒रय॑ति गृह्णाति ध्रु॒वायां॒ चतु॑ष्पदी प्रयाजानूया॒जेभ्य॒स्तद्द्वे च॑]

%3.3.6.1
आपो॑ देवीरग्रेपुवो अग्रेगुव॒ इत्या॑ह।
रू॒पमे॒वासा॑मे॒तन्म॑हि॒\-मानं॒ व्याच॑ष्टे।
अग्र॑ इ॒मं य॒ज्ञं न॑य॒ताग्रे॑ य॒ज्ञप॑ति॒मित्या॑ह।
अग्र॑ ए॒व य॒ज्ञं न॑यन्ति।
अग्रे॑ य॒ज्ञप॑तिम्।
यु॒ष्मानिन्द्रो॑\-ऽवृणीत वृत्र॒तूर्ये॑ यू॒यमिन्द्र॑मवृणीध्वं वृत्र॒तूर्य॒ इत्या॑ह।
वृ॒त्रꣳ ह॑ हनि॒ष्यन्निन्द्र॒ आपो॑ वव्रे।
आपो॒ हेन्द्रं॑ वव्रिरे।
सं॒ज्ञामे॒वासा॑मे॒तथ्सामा॑नं॒ व्याच॑ष्टे।
प्रोक्षि॑ताः॒ स्थेत्या॑ह॥२७॥

%3.3.6.2
तेनाऽऽपः॒ प्रोक्षि॑ताः।
अ॒ग्निर्दे॒वेभ्यो॒ निला॑यत।
कृष्णो॑ रू॒पं कृ॒त्वा।
स वन॒स्पती॒न्प्रावि॑शत्।
कृष्णो᳚\-ऽस्याखरे॒ष्ठो᳚\-ऽग्नये᳚ त्वा॒ स्वाहे\-त्या॑ह।
अ॒ग्नय॑ ए॒वैनं॒ जुष्टं॑ करोति।
अथो॑ अ॒ग्नेरे॒व मेध॒मव॑ रुन्धे।
वेदि॑\-रसि ब॒र्॒हिषे᳚ त्वा॒ स्वाहेत्या॑ह।
प्र॒जा वै ब॒र्॒हिः।
पृ॒थि॒वी वेदिः॑॥२८॥

%3.3.6.3
प्र॒जा ए॒व पृ॑थि॒व्यां प्रति॑\-ष्ठापयति।
ब॒र्॒हिर॑सि स्रु॒ग्भ्यस्त्वा॒ स्वाहेत्या॑ह।
प्र॒जा वै ब॒र्॒हिः।
यज॑मानः॒ स्रुचः॑।
यज॑मानमे॒व प्र॒जासु॒ प्रति॑\-ष्ठापयति।
दि॒वे त्वा॒\-ऽन्तरि॑क्षाय त्वा पृथि॒व्यै त्वेति॑ ब॒र्॒हिरा॒साद्य॒ प्रोक्ष॑ति।
ए॒भ्य ए॒वैन॑ल्लो॒केभ्यः॒ प्रोक्ष॑ति।
अथ॒ ततः॑ स॒ह स्रु॒चा पु॒रस्ता᳚त्प्र॒त्यञ्चं॑ ग्र॒न्थिं प्रत्यु॑क्षति।
प्र॒जा वै ब॒र्॒हिः।
यथा॒ सूत्यै॑ का॒ल आपः॑ पु॒रस्ता॒द्यन्ति॑॥२९॥

%3.3.6.4
ता॒दृगे॒व तत्।
स्व॒धा पि॒तृभ्य॒ इत्या॑ह।
स्व॒धा॒का॒रो हि पि॑तृ॒णाम्।
ऊर्ग्भ॑व बर्‌\mbox{}हि॒षद्भ्य॒ इति॒ दक्षि॑णायै॒ श्रोणे॒रोत्त॑रस्यै॒ निन॑यति॒ सन्त॑त्यै।
मासा॒ वै पि॒तरो॑ बर्‌\mbox{}हि॒षदः॑।
मासा॑ने॒व प्री॑णाति।
मासा॒ वा ओष॑धीर्व॒र्धय॑न्ति।
मासाः᳚ पचन्ति॒ समृ॑द्ध्यै।
अन॑तिस्कन्दन् ह प॒र्जन्यो॑ वर्‌\mbox{}षति।
यत्रै॒तदे॒वं क्रि॒यते᳚॥३०॥

%3.3.6.5
ऊ॒र्जा पृ॑थि॒वीं ग॑च्छ॒तेत्या॑ह।
पृ॒थि॒व्यामे॒वोर्जं॑ दधाति।
तस्मा᳚त्\-पृथि॒व्या ऊ॒र्जा भु॑ञ्जते।
ग्र॒न्थिं वि स्रꣳ॑सयति।
प्रज॑नयत्ये॒व तत्।
ऊ॒र्ध्वं प्राञ्च॒मुद्गू॑ढं प्र॒त्यञ्च॒मा य॑च्छति।
तस्मा᳚त्प्रा॒चीन॒ꣳ॒ रेतो॑ धीयते।
प्र॒तीचीः᳚ प्र॒जा जा॑यन्ते।
विष्णोः॒ स्तूपो॒\-ऽसीत्या॑ह।
य॒ज्ञो वै विष्णुः॑॥३१॥

%3.3.6.6
य॒ज्ञस्य॒ धृत्यै᳚।
पु॒रस्ता᳚त्प्रस्त॒रं गृ॑ह्णाति।
मुख्य॑मे॒वैनं॑ करोति।
इय॑न्तं गृह्णाति।
प्र॒जा\-प॑तिना यज्ञमु॒खेन॒ सम्मि॑तम्।
इय॑न्तं गृह्णाति।
य॒ज्ञ॒प॒रुषा॒ सम्मि॑तम्।
इय॑न्तं गृह्णाति।
ए॒ताव॒द्वै पुरु॑षे वी॒र्यम्᳚।
वी॒र्य॑सम्मितम्॥३२॥

%3.3.6.7
अप॑रिमितं गृह्णाति।
अप॑रिमित॒स्या\-व॑\-रुद्ध्यै।
तस्मि॑न्प॒वित्रे॒ अपि॑ \-सृजति।
यज॑मानो॒ वै प्र॑स्त॒रः।
प्रा॒णा॒पा॒नौ प॒वित्रे᳚।
यज॑मान ए॒व प्रा॑णापा॒नौ द॑धाति।
ऊर्णा᳚म्रदसं त्वा स्तृणा॒मीत्या॑ह।
य॒था॒\-य॒जु\-रे॒वै\-तत्।
स्वा॒स॒स्थं दे॒वेभ्य॒ इत्या॑ह।
दे॒वेभ्य॑ ए॒वैन॑थ्स्वास॒स्थं क॑रोति॥३३॥

%3.3.6.8
ब॒र्॒हिः स्तृ॑णाति।
प्र॒जा वै ब॒र्॒हिः।
पृ॒थि॒वी वेदिः॑।
प्र॒जा ए॒व पृ॑थि॒व्यां प्रति॑\-ष्ठापयति।
अन॑तिदृश्ञꣴ स्तृणाति।
प्र॒जयै॒वैनं॑ प॒शुभि॒रन॑तिदृश्ञं करोति।
धा॒रय॑न्प्रस्त॒रं प॑रि॒धीन्परि॑ दधाति।
यज॑मानो॒ वै प्र॑स्त॒रः।
यज॑मान ए॒व तथ्स्व॒यं प॑रि॒धीन्परि॑ दधाति।
ग॒न्ध॒र्वो॑ऽसि वि॒श्वाव॑सु॒रित्या॑ह॥३४॥

%3.3.6.9
विश्व॑मे॒वायु॒र्यज॑माने दधाति।
इन्द्र॑स्य बा॒हुर॑सि॒ दक्षि॑ण॒ इत्या॑ह।
इ॒न्द्रि॒यमे॒व यज॑माने दधाति।
मि॒त्रावरु॑णौ त्वोत्तर॒तः परि॑धत्ता॒मित्या॑ह।
प्रा॒णा॒पा॒नौ मि॒त्रावरु॑णौ।
प्रा॒णा॒पा॒ना\-वे॒वास्मि॑न्द\-धाति।
सूर्य॑स्त्वा पु॒रस्ता᳚त् पा॒त्वित्या॑ह।
रक्ष॑सा॒मप॑हत्यै।
कस्या᳚श्चिद॒भिश॑स्त्या॒ इत्या॑ह।
अप॑रिमितादे॒वैनं॑ पाति॥३५॥

%3.3.6.10
वी॒तिहो᳚त्रं त्वा कव॒ इत्या॑ह।
अ॒ग्निमे॒व हो॒त्रेण॒ सम॑र्धयति।
द्यु॒मन्त॒ꣳ॒ समि॑धीम॒हीत्या॑ह॒ समि॑द्ध्यै।
अग्ने॑ बृ॒हन्त॑मध्व॒र इत्या॑ह॒ वृद्ध्यै᳚।
वि॒शो य॒न्त्रे स्थ॒ इत्या॑ह।
वि॒शां यत्यै᳚।
उ॒दी॒चीना᳚ग्रे॒ नि द॑धाति॒ प्रति॑\-ष्ठित्यै।
वसू॑नाꣳ रु॒द्राणा॑मादि॒त्याना॒ꣳ॒ सद॑सि सी॒देत्या॑ह।
दे॒वता॑नामे॒व सद॑ने प्रस्त॒रꣳ सा॑दयति।
जु॒हूर॑सि घृ॒ताची॒ नाम्नेत्या॑ह॥३६॥

%3.3.6.11
अ॒सौ वै जु॒हूः।
अ॒न्तरि॑क्षमुप॒भृत्।
पृ॒थि॒वी ध्रु॒वा।
तासा॑मे॒तदे॒व प्रि॒यं नाम॑।
यद्\mbox{}घृ॒ताचीति॑।
यद्\mbox{}घृ॒ताचीत्याह॑।
प्रि॒येणै॒वैना॒ नाम्ना॑ सादयति।
ए॒ता अ॑सदन्थ्सुकृ॒तस्य॑ लो॒क इत्या॑ह।
स॒त्यं वै सु॑कृ॒तस्य॑ लो॒कः।
स॒त्य ए॒वैनाः᳚ सुकृ॒तस्य॑ लो॒के सा॑दयति।
ता वि॑ष्णो पा॒हीत्या॑ह।
य॒ज्ञो वै विष्णुः॑।
य॒ज्ञस्य॒ धृत्यै᳚।
पा॒हि य॒ज्ञं पा॒हि य॒ज्ञप॑तिं पा॒हि मां य॑ज्ञ॒निय॒मित्या॑ह।
य॒ज्ञाय॒ यज॑मानाया॒ऽऽत्मने᳚।
तेभ्य॑ ए॒वाऽऽशिष॒माशा॒स्ते\-ऽना᳚र्त्यै॥३७॥\anuvakamend[स्थेत्या॑ह पृथि॒वी वेदि॒र्यन्ति॑ क्रि॒यते॒ वीणु॑र्वी॒र्य॑सम्मितं करोत्याह पाति॒ नाम्नेत्या॑ह लो॒के सा॑दयति॒ षट् च॑]

%3.3.7.1
अ॒ग्निना॒ वै होत्रा᳚।
दे॒वा असु॑रान॒भ्य॑भवन्।
अ॒ग्नये॑ समि॒ध्यमा॑ना॒यानु॑ब्रू॒हीत्या॑ह॒ भ्रातृ॑व्याऽभिभूत्यै।
एक॑विꣳशति\-मिध्मदा॒रूणि॑ भवन्ति।
ए॒क॒वि॒ꣳ॒शो वै पुरु॑षः।
पुरु॑ष॒स्याऽऽप्त्यै᳚।
पञ्च॑दशेध्म\-दा॒रूण्य॒भ्या द॑धाति।
पञ्च॑दश॒ वा अ॑र्धमा॒सस्य॒ रात्र॑यः।
अ॒र्ध॒मा॒स॒शः सं॑वथ्स॒र आ᳚प्यते।
त्रीन्प॑रि॒धीन्परि॑ दधाति॥३८॥

%3.3.7.2
ऊ॒र्ध्वे स॒मिधा॒वा द॑धाति।
अ॒नू॒या॒जेभ्यः॑ स॒मिध॒मति॑ शिनष्टि।
षट्थ्सम्प॑द्यन्ते।
षड्वा ऋ॒तवः॑।
ऋ॒तूने॒व प्री॑णाति।
वे॒देनोप॑ वाजयति।
प्रा॒जा॒प॒त्यो वै वे॒दः।
प्रा॒जा॒प॒त्यः प्रा॒णः।
यज॑मान आहव॒नीयः॑।
यज॑मान ए॒व प्रा॒णं द॑धाति॥३९॥

%3.3.7.3
त्रिरुप॑ वाजयति।
त्रयो॒ वै प्रा॒णाः।
प्रा॒णाने॒वास्मि॑न्दधाति।
वे॒देनो॑प॒यत्य॑ स्रु॒वेण॑ प्राजाप॒त्यमा॑घा॒रमा घा॑रयति।
य॒ज्ञो वै प्र॒जा\-प॑तिः।
य॒ज्ञमे॒व प्र॒जा\-प॑तिं मुख॒त आर॑भते।
अथो᳚ प्र॒जा\-प॑तिः॒ सर्वा॑ दे॒वताः᳚।
सर्वा॑ ए॒व दे॒वताः᳚ प्रीणाति।
अ॒ग्निम॑ग्नी॒त्त्रिस्त्रिः॒ सं मृ॒ड्ढीत्या॑ह।
त्र्या॑वृ॒द्धि य॒ज्ञः॥४०॥

%3.3.7.4
अथो॒ रक्ष॑सा॒मप॑हत्यै।
प॒रि॒धीन्थ्सं मा᳚र्ष्टि।
पु॒नात्ये॒वैनान्॑।
त्रिस्त्रिः॒ सं मा᳚र्ष्टि।
त्र्या॑वृ॒द्धि य॒ज्ञः।
अथो॑ मेध्य॒त्वाय॑।
अथो॑ ए॒ते वै दे॑वा॒श्वाः।
दे॒वा॒श्वाने॒व तथ्सं मा᳚र्ष्टि।
सु॒व॒र्गस्य॑ लो॒कस्य॒ सम॑ष्ट्यै।
आसी॑नो॒\-ऽन्यमा॑घा॒रमा घा॑रयति॥४१॥

%3.3.7.5
तिष्ठ॑न्न॒न्यम्।
यथाऽनो॑ वा॒ रथं॑ वा यु॒ञ्ज्यात्।
ए॒वमे॒व तद॑ध्व॒र्युर्य॒ज्ञं यु॑नक्ति।
सु॒व॒र्गस्य॑ लो॒कस्या॒भ्यू᳚ढ्यै।
वह॑न्त्येनं ग्रा॒म्याः प॒शवः॑।
य ए॒वं वेद॑।
भुव॑नमसि॒ वि प्र॑थ॒स्वेत्या॑ह।
य॒ज्ञो वै भुव॑नम्।
य॒ज्ञ ए॒व यज॑मानं प्र॒जया॑ प॒शुभिः॑ प्रथयति।
अग्ने॒ यष्ट॑रि॒दं नम॒ इत्या॑ह॥४२॥

%3.3.7.6
अ॒ग्निर्वै दे॒वानां॒ यष्टा᳚।
य ए॒व दे॒वानां॒ यष्टा᳚।
तस्मा॑ ए॒व नम॑स्करोति।
जुह्वेह्य॒ग्निस्त्वा᳚ ह्वयति देवय॒ज्याया॒ उप॑भृ॒देहि॑ दे॒वस्त्वा॑ सवि॒ता ह्व॑यति देवय॒ज्याया॒ इत्या॑ह।
आ॒ग्ने॒यी वै जु॒हूः।
सा॒वि॒त्र्यु॑प॒भृत्।
ताभ्या॑मे॒वैने॒ प्रसू॑त॒ आद॑त्ते।
अग्ना॑विष्णू॒ मा वा॒मव॑ क्रमिष॒मित्या॑ह।
अ॒ग्निः पु॒रस्ता᳚त्।
विष्णु॑र्य॒ज्ञः प॒श्चात्॥४३॥

%3.3.7.7
ताभ्या॑मे॒व प्र॑ति॒प्रोच्या॒त्या क्रा॑मति।
विजि॑हाथां॒ मा मा॒ सन्ता᳚प्त॒मित्या॒हाहिꣳ॑सायै।
लो॒कं मे॑ लोककृतौ कृणुत॒मित्या॑ह।
आ॒\-मे॒वैतामा शा᳚स्ते।
विष्णोः॒ स्थान॑म॒सीत्या॑ह।
य॒ज्ञो वै विष्णुः॑।
ए॒तत्खलु॒ वै दे॒वाना॒\-मप॑\-रा\-जित\-मा॒यत॑नम्।
यद्य॒ज्ञः।
दे॒वाना॑\-मे॒वा\-प॑रा\-जित आ॒यत॑ने तिष्ठति।
इ॒त इन्द्रो॑ अकृणोद्वी॒र्या॑\-णीत्या॑ह॥४४॥

%3.3.7.8
इ॒न्द्रि॒यमे॒व यज॑माने दधाति।
स॒मा॒रभ्यो॒र्ध्वो अ॑ध्व॒रो दि॑वि॒स्पृश॒मित्या॑ह॒ वृद्ध्यै᳚।
आ॒घा॒रमा॑घा॒र्यमा॑ण॒मनु॑ समा॒रभ्य॑।
ए॒तस्मि॑न्का॒ले दे॒वाः सु॑व॒र्गं लो॒कमा॑यन्।
सा॒क्षादे॒व यज॑मानः सुव॒र्गं लो॒कमे॑ति।
अथो॒ समृ॑द्धेनै॒व य॒ज्ञेन॒ यज॑मानः सुव॒र्गं लो॒कमे॑ति।
अह्रु॑तो य॒ज्ञो य॒ज्ञप॑ते॒रित्या॒हाना᳚र्त्यै।
इन्द्रा॑वा॒न्थ्स्वाहेत्या॑ह।
इ॒न्द्रि॒यमे॒व यज॑माने दधाति।
बृ॒हद्भा इत्या॑ह॥४५॥

%3.3.7.9
सु॒व॒र्गो वै लो॒को बृ॒हद्भाः।
सु॒व॒र्गस्य॑ लो॒कस्य॒ सम॑ष्ट्यै।
य॒ज॒मा॒न॒दे॒व॒त्या॑ वै जु॒हूः।
भ्रा॒तृ॒व्य॒दे॒व॒त्यो॑प॒भृत्।
प्रा॒ण आ॑घा॒रः।
यथ्सꣴ॑स्प॒र्॒शये᳚त्।
भ्रातृ॑व्येऽस्य प्रा॒णं द॑ध्यात्।
असꣴ॑स्पर्‌\mbox{}शयन्न॒त्या क्रा॑मति।
यज॑मान ए॒व प्रा॒णं द॑धाति।
पा॒हि मा᳚\-ऽग्ने॒ दुश्च॑रिता॒दा मा॒ सुच॑रिते भ॒जेत्या॑ह॥४६॥

%3.3.7.10
अ॒ग्निर्वाव प॒वित्रम्᳚।
वृ॒जि॒नमनृ॑तं॒ दुश्च॑रितम्।
ऋ॒जु॒क॒र्मꣳ स॒त्यꣳ सुच॑रितम्।
अ॒ग्निरे॒वैनं॑ वृजि॒नादनृ॑ता॒द्दुश्च॑रितात्पाति।
ऋ॒जु॒क॒र्मे स॒त्ये सुच॑रिते भजति।
तस्मा॑दे॒वमा शा᳚स्ते।
आ॒त्मनो॑ गोपी॒थाय॑।
शिरो॒ वा ए॒तद्य॒ज्ञस्य॑।
यदा॑घा॒रः।
आ॒त्मा ध्रु॒वा॥४७॥

%3.3.7.11
आ॒घा॒रमा॒घार्य॑ ध्रु॒वाꣳ सम॑नक्ति।
आ॒त्मन्ने॒व य॒ज्ञस्य॒ शिरः॒ प्रति॑ दधाति।
द्विः सम॑नक्ति।
द्वौ हि प्रा॑णापा॒नौ।
तदा॑हुः।
त्रिरे॒व सम॑ञ्ज्यात्।
त्रिधा॑तु॒ हि शिर॒ इति॑।
शिर॑ इवै॒तद्य॒ज्ञस्य॑।
अथो॒ त्रयो॒ वै प्रा॒णाः।
प्रा॒णाने॒वास्मि॑न्दधाति।
म॒खस्य॒ शिरो॑ऽसि॒ सञ्ज्योति॑षा॒ ज्योति॑रङ्क्ता॒मित्या॑ह।
ज्योति॑रे॒वास्मा॑ उ॒परि॑ष्टाद्दधाति।
सु॒व॒र्गस्य॑ लो॒कस्यानु॑ख्यात्यै॥४८॥\anuvakamend[परि॑दधाति प्रा॒णं द॑धाति॒ हि य॒ज्ञो घा॑रयति॒ नम॒ इत्या॑ह प॒श्चाद्वी॒र्या॑णीत्या॑ह॒ भा इत्या॑ह भ॒जेत्या॑ह ध्रु॒वैवास्मि॑न्दधाति॒ त्रीणि॑ च]

%3.3.8.1
धिष्णि॑या॒ वा ए॒ते न्यु॑प्यन्ते।
यद्ब्र॒ह्मा।
यद्धोता᳚।
यद॑ध्व॒र्युः।
यद॒ग्नीत्।
यद्यज॑मानः।
तान् यद॑न्तरे॒यात्।
यज॑मानस्य प्रा॒णा\-न्थ्सङ्क॑\-र्‌\mbox{}षेत्।
प्र॒मायु॑कः स्यात्।
पु॒रो॒डाश॑मप॒गृह्य॒ सञ्च॑र\-त्यध्व॒र्युः॥४९॥

%3.3.8.2
यज॑मानायै॒व तल्लो॒कꣳ शिꣳ॑षति।
नास्य॑ प्रा॒णान्थ्सङ्क॑र्‌\mbox{}\-षति।
न प्र॒मायु॑को भवति।
पु॒रस्ता᳚त् प्र॒त्यङ्ङासी॑नः।
इडा॑या॒ इडा॒\-मा द॑धाति।
हस्त्या॒ꣳ॒ होत्रे᳚।
प॒शवो॒ वा इडा᳚।
प॒शवः॒ पुरु॑षः।
प॒शुष्वे॒व प॒शून्प्रति॑\-ष्ठापयति।
इडा॑यै॒ वा ए॒षा प्रजा॑तिः॥५०॥

%3.3.8.3
तां प्रजा॑तिं॒ यज॑मा॒नोऽनु॒ प्र जा॑यते।
द्विर॒ङ्गुला॑वनक्ति॒ पर्व॑णोः।
द्वि॒पाद्यज॑मानः॒ प्रति॑\-ष्ठित्यै।
स॒कृदुप॑ स्तृणाति।
द्विरा द॑धाति।
स॒कृद॒भि घा॑रयति।
च॒तुः सम्प॑द्यते।
च॒त्वारि॒ वै प॒शोः प्र॑ति॒ष्ठाना॑नि।
यावा॑ने॒व प॒शुः।
तमुप॑ह्वयते॥५१॥

%3.3.8.4
मुख॑मिव॒ प्रत्युप॑ह्वयेत।
स॒म्मु॒खाने॒व प॒शूनुप॑ ह्वयते।
प॒शवो॒ वा इडा᳚।
तस्मा॒थ्सा\-ऽन्वा॒रभ्या᳚।
अ॒ध्व॒र्युणा॑ च॒ यज॑मानेन च।
उप॑हूतः पशु॒मान॑सा॒नीत्या॑ह।
उप॒ ह्ये॑नौ॒ ह्वय॑ते॒ होता᳚।
इडा॑यै दे॒वता॑नामुपह॒वे।
उप॑हूतः पशु॒मान्भ॑वति।
य ए॒वं वेद॑॥५२॥

%3.3.8.5
यां वै हस्त्या॒मिडा॑मा॒दधा॑ति।
वा॒चः सा भा॑ग॒धेयम्᳚।
यामु॑प॒ह्वय॑ते।
प्रा॒णाना॒ꣳ॒ सा।
वाचं॑ चै॒व प्रा॒णाꣴश्चाव॑ रुन्धे।
अथ॒ वा ए॒तर्ह्युप॑हूताया॒मिडा॑याम्।
पु॒रो॒डाश॑स्यै॒व ब॑र्हि॒षदो॑ मीमा॒ꣳ॒सा।
यज॑मानं दे॒वा अ॑ब्रुवन्।
ह॒विर्नो॒ निर्व॒पेति॑।
नाहम॑भा॒गो निर्व॑फ्स्या॒मीत्य॑ब्रवीत्॥५३॥

%3.3.8.6
न मया॑\-ऽभा॒गया\-ऽनु॑वक्ष्य॒थेति॒ वाग॑ब्रवीत्।
नाहम॑भा॒गा पु॑रोनुवा॒क्या॑ भविष्या॒मीति॑ पुरोनुवा॒क्या᳚।
नाहम॑भा॒गा या॒ज्या॑ भविष्या॒मीति॑ या॒ज्या᳚।
न मया॑\-ऽभा॒गेन॒ वष॑ट्करिष्य॒थेति॑ वषट्का॒रः।
यद्य॑जमानभा॒गं नि॒धाय॑ पुरो॒डाशं॑ बर्‌\mbox{}हि॒षदं॑ क॒रोति॑।
ताने॒व तद्भा॒गिनः॑ करोति।
च॒तु॒र्धा क॑रोति।
चत॑स्रो॒ दिशः॑।
दि॒क्ष्वे॑व प्रति॑ तिष्ठति।
ब॒र्॒हि॒षदं॑ करोति॥५४॥

%3.3.8.7
यज॑मानो॒ वै पु॑रो॒डाशः॑।
प्र॒जा ब॒र्॒हिः।
यज॑मानमे॒व प्र॒जासु॒ प्रति॑\-ष्ठापयति।
तस्मा॑द॒स्थ्ना\-ऽन्याः प्र॒जाः प्र॑ति॒तिष्ठ॑न्ति।
मा॒ꣳ॒\-सेना॒न्याः।
अथो॒ खल्वा॑हुः।
दक्षि॑णा॒ वा ए॒ता ह॑विर्य॒ज्ञस्या᳚न्तर्वे॒द्यव॑ रुध्यन्ते।
यत्पु॑रो॒डाशं॑ बर्‌\mbox{}हि॒षदं॑ क॒रोतीति॑।
च॒तु॒र्धा क॑रोति।
च॒त्वारो॒ ह्ये॑ते ह॑विर्य॒ज्ञस्य॒र्त्विजः॑॥५५॥

%3.3.8.8
ब्र॒ह्मा होता᳚\-ऽध्व॒र्युर॒ग्नीत्।
तम॒भि मृ॑शेत्।
इ॒दं ब्र॒ह्मणः॑।
इ॒दꣳ होतुः॑।
इ॒दम॑ध्व॒र्योः।
इ॒दम॒ग्नीध॒ इति॑।
यथै॒वादः सौ॒म्ये᳚\-ऽध्व॒रे।
आ॒देश॑मृ॒त्विग्भ्यो॒ दक्षि॑णा नी॒यन्ते᳚।
ता॒दृगे॒व तत्।
अ॒ग्नीधे᳚ प्रथ॒माया द॑धाति॥५६॥

%3.3.8.9
अ॒ग्निमु॑खा॒ ह्यृद्धिः॑।
अ॒ग्निमु॑खामे॒वर्द्धिं॒ यज॑मान ऋध्नोति।
स॒कृदु॑प॒स्तीर्य॒ द्विरा॒दध॑त्।
उ॒प॒स्तीर्य॒ द्विर॒भि घा॑रयति।
षट्थ्सम्प॑द्यन्ते।
षड्वा ऋ॒तवः॑।
ऋ॒तूने॒व प्री॑णाति।
वे॒देन॑ ब्र॒ह्मणे᳚ ब्रह्मभा॒गं परि॑हरति।
प्रा॒जा॒प॒त्यो वै वे॒दः।
प्रा॒जा॒प॒त्यो ब्र॒ह्मा॥५७॥

%3.3.8.10
स॒वि॒ता य॒ज्ञस्य॒ प्रसू᳚त्यै।
अथ॒ काम॑म॒न्येन॑।
ततो॒ होत्रे᳚।
मध्यं॒ वा ए॒तद्य॒ज्ञस्य॑।
यद्धोता᳚।
म॒ध्य॒त ए॒व य॒ज्ञं प्री॑णाति।
अथा᳚ध्व॒र्यवे᳚।
प्र॒ति॒ष्ठा वा ए॒षा य॒ज्ञस्य॑।
यद॑ध्व॒र्युः।
तस्मा᳚द्धविर्य॒ज्ञस्यै॒तामे॒वाऽऽवृत॒मनु॑॥५८॥

%3.3.8.11
अ॒न्या दक्षि॑णा नीयन्ते।
य॒ज्ञस्य॒ प्रति॑\-ष्ठित्यै।
अ॒ग्निम॑ग्नीथ्स॒कृथ्स॑\-कृ॒थ्सं मृ॒ड्ढीत्या॑ह।
परा॑ङिव॒ ह्ये॑तर्‌\mbox{}हि॑ य॒ज्ञः।
इ॒षि॒ता दैव्या॒ होता॑र॒ इत्या॑ह।
इ॒षि॒तꣳ हि कर्म॑ क्रि॒यते᳚।
भ॒द्र॒वाच्या॑य॒ प्रेषि॑तो॒ मानु॑षः सूक्तवा॒काय॑ सू॒क्ता ब्रू॒हीत्या॑ह।
आ॒\-मे॒वैतामा शा᳚स्ते।
स्व॒गा दैव्या॒ होतृ॑भ्य॒ इत्या॑ह।
य॒ज्ञमे॒व तथ्स्व॒गा क॑रोति।
स्व॒स्तिर्मानु॑षेभ्य॒ इत्या॑ह।
आ॒\-मे॒वैतामा शा᳚स्ते।
शं॒ योर्ब्रू॒हीत्या॑ह।
शं॒युमे॒व बा॑र्‌\mbox{}हस्प॒त्यं भा॑ग॒धेये॑न॒ सम॑र्धयति॥५९॥\anuvakamend[च॒र॒त्य॒ध्व॒र्युः प्रजा॑तिर्ह्वयते॒ वेदा᳚ब्रवीद्बर्‌\mbox{}हि॒षदं॑ करोत्यृ॒त्विजो॑ दधाति ब्र॒ह्मा\-ऽनु॑करोति च॒त्वारि॑ च]

%3.3.9.1
अथ॒ स्रुचा॑वनु॒ष्टुग्भ्यां॒ वाज॑वतीभ्यां॒ व्यू॑हति।
प्र॒ति॒ष्ठा वा अ॑नु॒ष्टुक्।
अन्नं॒ वाजः॒ प्रति॑\-ष्ठित्यै।
अ॒न्नाद्य॒स्या\-व॑\-रुद्ध्यै।
प्राचीं᳚ जु॒हूमू॑हति।
जा॒ताने॒व भ्रातृ॑व्या॒न्प्रणु॑दते।
प्र॒तीची॑मुप॒भृतम्᳚।
ज॒नि॒ष्यमा॑णाने॒व प्रति॑\-नुदते।
सविषू॑च ए॒वापोह्य॑ स॒पत्ना॒न्॒ यज॑मानः।
अ॒स्मिँल्लो॒के प्रति॑ तिष्ठति॥६०॥

%3.3.9.2
द्वाभ्या᳚म्।
द्विप्र॑तिष्ठो॒ हि।
वसु॑भ्यस्त्वा रु॒द्रेभ्य॑स्त्वा\-ऽऽदि॒त्येभ्य॒स्त्वेत्या॑ह।
य॒था॒\-य॒जु\-रे॒वै\-तत्।
स्रु॒क्षु प्र॑स्त॒रम॑नक्ति।
इ॒मे वै लो॒काः स्रुचः॑।
यज॑मानः प्रस्त॒रः।
यज॑मानमे॒व तेज॑सा\-ऽनक्ति।
त्रे॒धा\-ऽन॑क्ति।
त्रय॑ इ॒मे लो॒काः॥६१॥

%3.3.9.3
ए॒भ्य ए॒वैनं॑ लो॒केभ्यो॑\-ऽनक्ति।
अ॒भि॒पू॒र्वम॑नक्ति।
अ॒भि॒पू॒र्वमे॒व यज॑मानं॒ तेज॑सा\-ऽनक्ति।
अ॒क्तꣳ रिहा॑णा॒ इत्या॑ह।
तेजो॒ वा आज्यम्᳚।
यज॑मानः प्रस्त॒रः।
यज॑मानमे॒व तेज॑सा\-ऽनक्ति।
वि॒यन्तु॒ वय॒ इत्या॑ह।
वय॑ ए॒वैनं॑ कृ॒त्वा।
सु॒व॒र्गं लो॒कं ग॑मयति॥६२॥

%3.3.9.4
प्र॒जां योनिं॒ मा निर्मृ॑क्ष॒मित्या॑ह।
प्र॒जायै॑ गोपी॒थाय॑।
आप्या॑यन्ता॒माप॒ ओष॑धय॒ इत्या॑ह।
आप॑ ए॒वौष॑धी॒रा प्या॑ययति।
म॒रुतां॒ पृष॑तयः॒ स्थेत्या॑ह।
म॒रुतो॒ वै वृष्ट्या॑ ईशते।
वृष्टि॑मे॒वाव॑ रुन्धे।
दिवं॑ गच्छ॒ ततो॑ नो॒ वृष्टि॒मेर॒येत्या॑ह।
वृष्टि॒र्वै द्यौः।
वृष्टि॑मे॒वाव॑ रुन्धे॥६३॥

%3.3.9.5
याव॒द्वा अ॑ध्व॒र्युः प्र॑स्त॒रं प्र॒हर॑ति।
ताव॑द॒स्यायु॑र्मीयते।
आ॒यु॒ष्पा अ॑ग्ने॒\-ऽस्यायु॑र्मे पा॒हीत्या॑ह।
आयु॑रे॒वाऽऽत्मन्ध॑त्ते।
याव॒द्वा अ॑ध्व॒र्युः प्र॑स्त॒रं प्र॒हर॑ति।
ताव॑दस्य॒ चक्षु॑र्मीयते।
च॒क्षु॒ष्पा अ॑ग्नेऽसि॒ चक्षु॑र्मे पा॒हीत्या॑ह।
चक्षु॑रे॒वाऽऽत्मन्ध॑त्ते।
ध्रु॒वा\-ऽसीत्या॑ह॒ प्रति॑\-ष्ठित्यै।
यं प॑रि॒धिं प॒र्यध॑त्था॒ इत्या॑ह॥६४॥

%3.3.9.6
य॒था॒\-य॒जु\-रे॒वै\-तत्।
अग्ने॑ देव प॒णिभि॑र्वी॒यमा॑ण॒ इत्या॑ह।
अ॒ग्नय॑ ए॒वैनं॒ जुष्टं॑ करोति।
तन्त॑ ए॒तमनु॒ जोषं॑ भरा॒मीत्या॑ह।
स॒जा॒ताने॒वास्मा॒ अनु॑कान्करोति।
नेदे॒ष त्वद॑पचे॒तया॑ता॒ इत्या॒हानु॑ख्यात्यै।
य॒ज्ञस्य॒ पाथ॒ उप॒ समि॑त॒मित्या॑ह।
भू॒मान॑मे॒वोपै॑ति।
प॒रि॒धीन्प्र ह॑रति।
य॒ज्ञस्य॒ समि॑ष्ट्यै॥६५॥

%3.3.9.7
स्रुचौ॒ सं प्रस्रा॑वयति।
यदे॒व तत्र॑ क्रू॒रम्।
तत्तेन॑ शमयति।
जु॒ह्वामु॑प॒भृतम्᳚।
य॒ज॒मा॒न॒दे॒व॒त्या॑ वै जु॒हूः।
भ्रा॒तृ॒व्य॒दे॒व॒त्यो॑प॒भृत्।
यज॑मानायै॒व भ्रातृ॑व्य॒मुप॑स्तिं करोति।
स॒ꣴ॒स्रा॒वभा॑गाः॒ स्थेत्या॑ह।
वस॑वो॒ वै रु॒द्रा आ॑दि॒त्याः सꣴ॑स्रा॒वभा॑गाः।
तेषां॒ तद्भा॑ग॒धेयम्᳚॥६६॥

%3.3.9.8
ताने॒व तेन॑ प्रीणाति।
वै॒श्व॒दे॒व्यर्चा।
ए॒ते हि विश्वे॑ दे॒वाः।
त्रि॒ष्टुग्भ॑वति।
इ॒न्द्रि॒यं वै त्रि॒ष्टुक्।
इ॒न्द्रि॒यमे॒व यज॑माने दधाति।
अ॒ग्नेर्वा॒मप॑न्नगृहस्य॒ सद॑सि सादया॒मीत्या॑ह।
इ॒यं वा अ॒ग्निरप॑न्नगृहः।
अ॒स्या ए॒वैने॒ सद॑ने सादयति।
सु॒म्नाय॑ सुम्निनी सु॒म्ने मा॑ धत्त॒मित्या॑ह॥६७॥

%3.3.9.9
प्र॒जा वै प॒शवः॑ सु॒म्नम्।
प्र॒जामे॒व प॒शूना॒त्मन्ध॑त्ते।
धु॒रि धु॒र्यौ॑ पात॒मित्या॑ह।
जा॒या॒प॒त्योर्गो॑पी॒थाय॑।
अग्ने॑\-ऽदब्धायो\-ऽशीततनो॒ इत्या॑ह।
य॒था॒\-य॒जु\-रे॒वै\-तत्।
पा॒हि मा॒ऽद्य दि॒वः पा॒हि प्रसि॑त्यै पा॒हि दुरि॑ष्ट्यै पा॒हि दु॑रद्म॒न्यै पा॒हि दुश्च॑रिता॒दित्या॑ह।
आ॒\-मे॒वैतामा शा᳚स्ते।
अवि॑षन्नः पि॒तुं कृ॑णु सु॒षदा॒ योनि॒ꣴ॒ स्वाहेती᳚ध्मसं॒वृश्च॑नान्यन्वाहार्य॒पच॑ने\-ऽभ्या॒धाय॑ फलीकरणहो॒मं जु॑होति।
अति॑रिक्तानि॒ वा इ॑ध्मसं॒ वृश्च॑नानि॥६८॥

%3.3.9.10
अति॑रिक्ताः फली॒कर॑णाः।
अति॑रिक्तमाज्योच्छेष॒णम्।
अति॑रिक्त ए॒वाति॑रिक्तं दधाति।
अथो॒ अति॑रिक्तेनै॒वाति॑\-रिक्त\-मा॒प्त्वाऽव॑ रुन्धे।
वेदि॑र्दे॒वेभ्यो॒ निला॑यत।
तां वे॒देनान्व॑विन्दन्।
वे॒देन॒ वेदिं॑ विविदुः पृथि॒वीम्।
सा प॑प्रथे पृथि॒वी पार्थि॑वानि।
गर्भं॑ बिभर्ति॒ भुव॑नेष्व॒न्तः।
ततो॑ य॒ज्ञो जा॑यते विश्व॒दानि॒रिति॑ पु॒रस्ता᳚थ्स्तम्बय॒जुषो॑ वे॒देन॒ वेदि॒ꣳ॒ सम्मा॒र्ष्ट्यनु॑वित्त्यै॥६९॥

%3.3.9.11
अथो॒ यद्वे॒दश्च॒ वेदि॑श्च॒ भव॑तः।
मि॒थु॒न॒त्वाय॒ प्रजा᳚त्यै।
प्र॒जा\-प॑ते॒र्वा ए॒तानि॒ श्मश्रू॑णि।
यद्वे॒दः।
पत्नि॑या उ॒पस्थ॒ आस्य॑ति।
मि॒थु॒नमे॒व क॑रोति।
वि॒न्दते᳚ प्र॒जाम्।
वे॒दꣳ होता\-ऽऽह॑व॒नीया᳚थ्स्तृ॒णन्ने॑ति।
य॒ज्ञमे॒व तथ्सं त॑नो॒त्योत्त॑रस्मादर्ध\-मा॒सात्।
तꣳ सन्त॑त॒मुत्त॑रे\-ऽर्धमा॒स आल॑भते॥७०॥

%3.3.9.12
तं का॒लेका॑ल॒ आग॑ते यजते।
ब्र॒ह्म॒वा॒दिनो॑ वदन्ति।
स त्वा अ॑ध्व॒र्युः स्या᳚त्।
यो यतो॑ य॒ज्ञं प्र॑यु॒ङ्क्ते।
तदे॑नं प्रतिष्ठा॒पय॒तीति॑।
वाता॒द्वा अ॑ध्व॒र्युर्य॒ज्ञं प्रयु॑ङ्क्ते।
देवा॑ गातुविदो गा॒तुं वि॒त्वा गा॒तुमि॒तेत्या॑ह।
यत॑ ए॒व य॒ज्ञं प्र॑यु॒ङ्क्ते।
तदे॑नं॒ प्रति॑\-ष्ठापयति।
प्रति॑ तिष्ठति प्र॒जया॑ प॒शुभि॒र्यज॑मानः॥७१॥\anuvakamend[ति॒ष्ठ॒ती॒मे लो॒का ग॑मयति॒ द्यौर्वृष्टि॑मे॒वाव॑ रुन्धे प॒र्यध॑त्था॒ इत्या॑ह॒ समि॑ष्ट्यै भाग॒धेय॑न्धत्त॒मित्या॑ह॒ वा इ॑ध्मसं॒ वृश्च॑ना॒न्यनु॑वित्त्यै लभते॒ यज॑मानः]

%3.3.10.1
यो वा अय॑थादेवतं य॒ज्ञमु॑प॒चर॑ति।
आ दे॒वता᳚भ्यो वृश्च्यते।
पापी॑यान्भवति।
यो य॑थादेव॒तम्।
न दे॒वता᳚भ्य॒ आवृ॑श्च्यते।
वसी॑यान्भवति।
वा॒रु॒णो वै पाशः॑।
इ॒मं विष्या॑मि॒ वरु॑णस्य॒ पाश॒मित्या॑ह।
व॒रु॒ण॒पा॒शादे॒वैनां᳚ मुञ्चति।
स॒वि॒तृप्र॑सूतो यथादेव॒तम्॥७२॥

%3.3.10.2
न दे॒वता᳚भ्य॒ आवृ॑श्च्यते।
वसी॑यान्भवति।
धा॒तुश्च॒ योनौ॑ सुकृ॒तस्य॑ लो॒क इत्या॑ह।
अ॒ग्निर्वै धा॒ता।
पुण्यं॒ कर्म॑ सुकृ॒तस्य॑ लो॒कः।
अ॒ग्निरे॒वैनां᳚ धा॒ता।
पुण्ये॒ कर्म॑णि सुकृ॒तस्य॑ लो॒के द॑धाति।
स्यो॒नं मे॑ स॒ह पत्या॑ करो॒मीत्या॑ह।
आ॒त्मन॑श्च॒ यज॑मानस्य॒ चाना᳚त्यै स॒न्त्वाय॑।
समायु॑षा॒ सं प्र॒जयेत्या॑ह॥७३॥

%3.3.10.3
आ॒\-मे॒वैतामा शा᳚स्ते पूर्णपा॒त्रे।
अ॒न्त॒तो॑\-ऽनु॒ष्टुभा᳚।
चतु॑ष्प॒द्वा ए॒तच्छन्दः॒ प्रति॑\-ष्ठितं॒ पत्नि॑यै पूर्णपा॒त्रे भ॑वति।
अ॒स्मिँल्लो॒के प्रति॑ तिष्ठा॒नीति॑।
अ॒स्मिन्ने॒व लो॒के प्रति॑ तिष्ठति।
अथो॒ वाग्वा अ॑नु॒ष्टुक्।
वाङ्मि॑थु॒नम्।
आपो॒ रेतः॑ प्र॒जन॑नम्।
ए॒तस्मा॒द्वै मि॑थु॒नाद्वि॒द्योत॑मानः स्त॒नय॑न्वर्‌\mbox{}षति।
रेतः॑ सि॒ञ्चन्॥७४॥

%3.3.10.4
प्र॒जाः प्र॑ज॒नयन्॑।
यद्वै य॒ज्ञस्य॒ ब्रह्म॑णा यु॒ज्यते᳚।
ब्रह्म॑णा॒ वै तस्य॑ विमो॒कः।
अ॒द्भिः शान्तिः॑।
विमु॑क्तं॒ वा ए॒तर्‌\mbox{}हि॒ योक्त्रं॒ ब्रह्म॑णा।
आ॒दायै॑न॒त्पत्नी॑ स॒हाप उप॑गृह्णीते॒ शान्त्यै᳚।
अ॒ञ्ज॒लौ पू᳚र्णपा॒त्रमा न॑यति।
रेत॑ ए॒वास्यां᳚ प्र॒जां द॑धाति।
प्र॒जया॒ हि म॑नु॒ष्यः॑ पू॒र्णः।
मुखं॒ वि मृ॑ष्टे।
अ॒व॒भृ॒थस्यै॒व रू॒पं कृ॒त्वोत्ति॑ष्ठति॥७५॥\anuvakamend[स॒वि॒तृप्र॑सूतो यथादेव॒तं प्र॒जयेत्या॑ह सि॒ञ्चन्मृ॑ष्ट॒ एकं॑ च]

%3.3.11.1
प॒रि॒वे॒षो वा ए॒ष वन॒स्पती॑नाम्।
यदु॑पवे॒षः।
य ए॒वं वेद॑।
वि॒न्दते॑ परिवे॒ष्टारम्᳚।
तमु॑त्क॒रे।
यं दे॒वा म॑नु॒ष्ये॑षु।
उ॒प॒वे॒षमधा॑रयन्।
ये अ॒स्मदप॑ चेतसः।
तान॒स्मभ्य॑मि॒हा कु॑रु।
उप॑वे॒षोप॑ विड्ढि नः॥७६॥

%3.3.11.2
प्र॒जां पुष्टि॒मथो॒ धनम्᳚।
द्वि॒पदो॑ न॒श्चतु॑ष्पदः।
ध्रु॒वानन॑प\-गान्कु॒र्विति॑ पु॒रस्ता᳚त्प्र॒त्यञ्च॒मुप॑ गूहति।
तस्मा᳚त्पु॒र\-स्ता᳚त्प्र॒त्यञ्चः॑ शू॒द्रा अव॑स्यन्ति।
स्थ॒वि॒म॒त उप॑गूहति।
अप्र॑तिवादिन ए॒वैना᳚न्कुरुते।
धृष्टि॒र्वा उ॑पवे॒षः।
शु॒चर्तो वज्रो॒ ब्रह्म॑णा॒ सꣳशि॑तः।
योप॑वे॒षे शुक्।
साऽमुमृ॑च्छतु॒ यं द्वि॒ष्म इति॑॥७७॥

%3.3.11.3
अथा᳚स्मै नाम॒ गृह्य॒ प्रह॑रति।
निर॒मुन्नु॑द॒ ओक॑सः।
स॒पत्नो॒ यः पृ॑त॒न्यति॑।
नि॒र्बा॒ध्ये॑न ह॒विषा᳚।
इन्द्र॑ एणं॒ परा॑शरीत्।
इ॒हि ति॒स्रः प॑रा॒वतः॑।
इ॒हि पञ्च॒ जना॒ꣳ॒ अति॑।
इ॒हि ति॒स्रोऽति॑ रोच॒नायाव॑त्।
सूर्यो॒ अस॑द्दि॒वि।
प॒र॒मान्त्वा॑ परा॒वतम्᳚॥७८॥

%3.3.11.4
इन्द्रो॑ नयतु वृत्र॒हा।
यतो॒ न पुन॒राय॑सि।
श॒श्व॒तीभ्यः॒ समा᳚भ्य॒ इति॑।
त्रि॒वृद्वा ए॒ष वज्रो॒ ब्रह्म॑णा॒ सꣳशि॑तः।
शु॒चैवैनं॑ वि॒ध्वा।
ए॒भ्यो लो॒केभ्यो॑ नि॒र्णुद्य॑।
वज्रे॑ण॒ ब्रह्म॑णा स्तृणुते।
ह॒तो॑\-ऽसावव॑धिष्मा॒मुमित्या॑ह॒ स्तृत्यै᳚।
यं द्वि॒ष्यात्तं ध्या॑येत्।
शु॒चैवैन॑मर्पयति॥७९॥




\prashnaend{प्रत्यु॑ष्टं दि॒वः शिल्प॒मय॑ज्ञो घृ॒तं च॑ देवासु॒राः स ए॒तमिन्द्र आपो॑ देवीर॒ग्निना॒ धिष्णि॑या॒ अथ॒ स्रुचौ॒ यो वा अय॑थादेवतं परिवे॒षो वा एका॑दश॥११॥}{प्रत्यु॑ष्ट॒मय॑ज्ञ ए॒षा हि विश्वे॑षां दे॒वाना॑मू॒र्जा पृ॑थि॒वीमथो॒ रक्ष॑सा॒न्तां प्रजा॑तिं॒ द्वाभ्यां॒ तं का॒लेका॑ले॒ नव॑सप्ततिः॥७९॥}{प्रत्यु॑ष्टमर्पयति॥}{हरिः॑ ओम्॥}{इति श्रीकृष्णयजुर्वेदीयतैत्तिरीयब्राह्मणे तृतीयाष्टके तृतीयः प्रपाठकः समाप्तः॥}
\clearpage
\sect{चतुर्थः प्रश्नः}
\setcounter{anuvakam}{0}
\dnsub{तैत्तिरीयब्राह्मणे तृतीयाष्टके चतुर्थः प्रपाठकः}

%3.4.1.1
ब्रह्म॑णे ब्राह्म॒णमाल॑भते।
क्ष॒त्राय॑ राज॒न्यम्᳚।
म॒रुद्भ्यो॒ वैश्यम्᳚।
तप॑से शू॒द्रम्।
तम॑से॒ तस्क॑रम्।
नार॑काय वीर॒हणम्᳚।
पा॒प्मने᳚ क्ली॒बम्।
आ॒क्र॒याया॑यो॒गूम्।
कामा॑य पुꣴश्च॒लूम्।
अति॑क्रुष्टाय माग॒धम्॥१॥

%3.4.2.1
गी॒ताय॑ सू॒तम्।
नृ॒त्ताय॑ शैलू॒षम्।
धर्मा॑य सभाच॒रम्।
न॒र्माय॑ रे॒भम्।
नरि॑ष्ठायै भीम॒लम्।
हसा॑य॒ कारिम्᳚।
आ॒न॒न्दाय॑ स्त्रीष॒खम्।
प्रमुदे॑ कुमारीपु॒त्रम्।
मे॒धायै॑ रथका॒रम्।
धैर्या॑य॒ तक्षा॑णम्॥२॥

%3.4.3.1
श्रमा॑य कौला॒लम्।
मा॒यायै॑ कार्मा॒रम्।
रू॒पाय॑ मणिका॒रम्।
शुभे॑ व॒पम्।
श॒र॒व्या॑या इषुका॒रम्।
हे॒त्यै ध॑न्वका॒रम्।
कर्म॑णे ज्या\-का॒रम्।
दि॒ष्टाय॑ रज्जुस॒र्गम्।
मृ॒त्यवे॑ मृग॒युम्।
अन्त॑काय श्व॒नितम्᳚॥३॥

%3.4.4.1
स॒न्धये॑ जा॒रम्।
गे॒हायो॑पप॒तिम्।
निर्\mbox{}ऋ॑त्यै परिवि॒त्तम्।
आर्त्यै॑ परिविविदा॒नम्।
अरा᳚ध्यै दिधिषू॒पतिम्᳚।
प॒वित्रा॑य भि॒षजम्᳚।
प्र॒ज्ञाना॑य नक्षत्रद॒र्॒शम्।
निष्कृ॑त्यै पेशस्का॒रीम्।
बला॑योप॒दाम्।
वर्णा॑यानू॒रुधम्᳚॥४॥

%3.4.5.1
न॒दीभ्यः॑ पौञ्जि॒ष्टम्।
ऋ॒क्षीका᳚भ्यो॒ नैषा॑दम्।
पु॒रु॒ष॒व्या॒घ्राय॑ दु॒र्मदम्᳚।
प्र॒युद्भ्य॒ उन्म॑त्तम्।
ग॒न्ध॒र्वा॒फ्स॒राभ्यो॒ व्रात्यम्᳚।
स॒र्प॒दे॒व॒\-ज॒नेभ्यो\-ऽप्र॑तिपदम्।
अवे᳚भ्यः कित॒वम्।
इ॒र्यता॑या॒ अकि॑तवम्।
पि॒शा॒चेभ्यो॑ बिदलका॒रम्।
या॒तु॒धाने᳚भ्यः कण्टकका॒रम्॥५॥

%3.4.6.1
उ॒थ्सा॒देभ्यः॑ कु॒ब्जम्।
प्र॒मुदे॑ वाम॒नम्।
द्वा॒र्भ्यः स्रा॒मम्।
स्वप्ना॑या॒न्धम्।
अध॑र्माय बधि॒रम्।
सं॒ज्ञाना॑य स्मरका॒रीम्।
प्र॒का॒मोद्या॑योप॒सदम्᳚।
आ॒शि॒क्षायै᳚ प्र॒श्ञिनम्᳚।
उ॒प॒शि॒क्षाया॑ अभिप्र॒श्ञिनम्᳚।
म॒र्यादा॑यै प्रश्ञविवा॒कम्॥६॥

%3.4.7.1
ऋत्यै᳚ स्ते॒नहृ॑दयम्।
वैर॑हत्याय॒ पिशु॑नम्।
विवि॑त्त्यै क्ष॒त्तारम्᳚।
औप॑द्रष्टाय सङ्ग्रही॒तारम्᳚।
बला॑यानुच॒रम्।
भू॒म्ने प॑रिष्क॒न्दम्।
प्रि॒याय॑ प्रियवा॒दिनम्᳚।
अरि॑ष्ट्या अश्वसा॒दम्।
मेधा॑य वासः पल्पू॒लीम्।
प्र॒का॒माय॑ रजयि॒त्रीम्॥७॥

%3.4.8.1
भायै॑ दार्वाहा॒रम्।
प्र॒भाया॑ आग्ने॒न्धम्।
नाक॑स्य पृ॒ष्ठाया॑भि\-षे॒क्तारम्᳚।
ब्र॒ध्नस्य॑ वि॒ष्टपा॑य पात्रनिर्णे॒गम्।
दे॒व॒लो॒काय॑ पेशि॒तारम्᳚।
म॒नु॒ष्य॒लो॒काय॑ प्रकरि॒तारम्᳚।
सर्वे᳚भ्यो लो॒केभ्य॑ उपसे॒क्तारम्᳚।
अव॑र्त्यै व॒धायो॑पमन्थि॒तारम्᳚।
सु॒व॒र्गाय॑ लो॒काय॑ भाग॒दुघम्᳚।
वर्\mbox{}षि॑ष्ठाय॒ नाका॑य परिवे॒ष्टारम्᳚॥८॥

%3.4.9.1
अर्मे᳚भ्यो हस्ति॒पम्।
ज॒वाया᳚श्व॒पम्।
पुष्ट्यै॑ गोपा॒लम्।
तेज॑से\-ऽजपा॒लम्।
वी॒र्या॑याविपा॒लम्।
इरा॑यै की॒नाशम्᳚।
की॒लाला॑य सुराका॒रम्।
भ॒द्राय॑ गृह॒पम्।
श्रेय॑से वित्त॒धम्।
अध्य॑क्षायानुक्ष॒त्तारम्᳚॥९॥

%3.4.10.1
म॒न्यवे॑\-ऽयस्ता॒पम्।
क्रोधा॑य निस॒रम्।
शोका॑याभिस॒रम्।
उ॒त्कू॒ल॒वि॒कू॒लाभ्यां᳚ त्रि॒स्थिनम्᳚।
योगा॑य यो॒क्तारम्᳚।
क्षेमा॑य विमो॒क्तारम्᳚।
वपु॑षे मानस्कृ॒तम्।
शीला॑याञ्जनीका॒रम्।
निर्\mbox{}ऋ॑त्यै कोशका॒रीम्।
य॒माया॒सूम्॥१०॥

%3.4.11.1
य॒म्यै॑ यम॒सूम्।
अथ॑र्व॒भ्यो\-ऽव॑तोकाम्।
सं॒व॒थ्स॒राय॑ पर्या॒रिणी᳚म्।
प॒रि॒व॒थ्स॒राया\-वि॑जाताम्।
इ॒दा॒व॒थ्स॒राया॑प॒\-स्कद्व॑रीम्।
इ॒द्व॒थ्स॒राया॒तीत्व॑रीम्।
व॒थ्स॒राय॒ विज॑र्जराम्।
सं॒व॒थ्स॒राय॒ पलि॑क्नीम्।
वना॑य वन॒पम्।
अ॒न्यतो॑रण्याय दाव॒पम्॥११॥

%3.4.12.1
सरो᳚भ्यो धैव॒रम्।
वेश॑न्ताभ्यो॒ दाशम्᳚।
उ॒प॒स्थाव॑रीभ्यो॒ बैन्दम्᳚।
न॒ड्व॒लाभ्यः॑ शौष्क॒लम्।
पा॒र्या॑य कैव॒र्तम्।
अ॒वा॒र्या॑य मार्गा॒रम्।
ती॒र्थेभ्य॑ आ॒न्दम्।
विष॑मेभ्यो मैना॒लम्।
स्वने᳚भ्यः॒ पर्ण॑कम्।
गुहा᳚भ्यः॒ किरा॑तम्।
सानु॑भ्यो॒ जम्भ॑कम्।
पर्व॑तेभ्यः॒ किम्पू॑रुषम्॥१२॥

%3.4.13.1
प्र॒ति॒श्रुत्का॑या ऋतु॒लम्।
घोषा॑य भ॒षम्।
अन्ता॑य बहुवा॒दिनम्᳚।
अ॒न॒न्ताय॒ मूकम्᳚।
मह॑से वीणावा॒दम्।
क्रोशा॑य तूणव॒ध्मम्।
आ॒क्र॒न्दाय॑ दुन्दुभ्याघा॒तम्।
अ॒व॒र॒स्प॒राय॑ शङ्ख॒ध्मम्।
ऋ॒भुभ्यो॑जिनसन्धा॒यम्।
सा॒ध्येभ्य॑श्चर्म॒म्णम्॥१३॥

%3.4.14.1
बी॒भ॒थ्सायै॑ पौल्क॒सम्।
भूत्यै॑ जागर॒णम्।
अभू᳚त्यै स्वप॒नम्।
तु॒लायै॑ वाणि॒जम्।
वर्णा॑य हिरण्यका॒रम्।
विश्वे᳚भ्यो दे॒वेभ्यः॑ सिध्म॒लम्।
प॒श्चा॒द्दो॒षाय॑ ग्ला॒वम्।
ऋत्यै॑ जनवा॒दिनम्᳚।
व्यृ॑द्ध्या अपग॒ल्भम्।
स॒ꣳ॒श॒राय॑ प्र॒च्छिदम्᳚॥१४॥

%3.4.15.1
हसा॑य पुꣴश्च॒लूमा ल॑भते।
वी॒णा॒वा॒दं गण॑कं गी॒ताय॑।
याद॑से शाबु॒ल्याम्।
न॒र्माय॑ भद्रव॒तीम्।
तू॒ण॒व॒ध्मं ग्रा॑म॒ण्यं॑ पाणिसङ्घा॒तं नृ॒त्ताय॑।
मोदा॑यानु॒क्रोश॑कम्।
आ॒न॒न्दाय॑ तल॒वम्॥१५॥

%3.4.16.1
अ॒क्ष॒रा॒जाय॑ कित॒वम्।
कृ॒ताय॑ सभा॒विनम्᳚।
त्रेता॑या आदि\-नव\-द॒र्॒शम्।
द्वा॒प॒राय॑ बहिः॒ सदम्᳚।
कल॑ये सभास्था॒णुम्।
दु॒ष्कृ॒ताय॑ च॒रका॑\-चार्यम्।
अध्व॑ने ब्रह्म\-चा॒रिणम्᳚।
पि॒शा॒चेभ्यः॑ सैल॒गम्।
पि॒पा॒सायै॑ गोव्\-य॒च्छम्।
निर्\mbox{}ऋ॑त्यै गो\-घा॒तम्।
क्षु॒धे गो॑विक॒र्तम्।
क्षु॒त्तृ॒ष्णा\-भ्या॒न्तम्।
यो गां वि॒कृन्त॑न्तं मा॒ꣳ॒सं भिक्ष॑माण उप॒तिष्ठ॑ते॥१६॥

%3.4.17.1
भूम्यै॑ पीठस॒र्पिण॒मा ल॑भते।
अ॒ग्नये\-ऽꣳ॑स॒लम्।
वा॒यवे॑ चाण्डा॒लम्।
अ॒न्तरि॑क्षाय वꣳशन॒र्तिनम्᳚।
दि॒वे ख॑ल॒तिम्।
सूर्या॑य हर्य॒क्षम्।
च॒न्द्रम॑से मिर्मि॒रम्।
नक्ष॑त्रेभ्यः कि॒लासम्᳚।
अह्ने॑ शु॒क्लं पि॑ङ्ग॒लम्।
रात्रि॑यै कृ॒ष्णं पि॑ङ्गा॒क्षम्॥१७॥

%3.4.18.1
वा॒चे पुरु॑ष॒मा ल॑भते।
प्रा॒णम॑पा॒नं व्या॒नमु॑दा॒नꣳ स॑मा॒नं तान् वा॒यवे᳚।
सूर्या॑य॒ चक्षु॒रा ल॑भते।
मन॑श्च॒न्द्रम॑से।
दि॒ग्भ्यः श्रोत्रम्᳚।
प्र॒जा\-प॑तये॒ पुरु॑षम्॥१८॥

%3.4.19.1
अथै॒तानरू॑पेभ्य॒ आल॑भते।
अति॑\-ह्रस्व॒मति॑\-दीर्घम्।
अति॑\-कृश॒मत्यꣳ॑सलम्।
अति॑\-शुक्ल॒मति॑\-कृष्णम्।
अति॑\-श्लक्ष्ण॒\-मति॑\-लोमशम्।
अति॑\-किरिट॒मति॑\-दन्तुरम्।
अति॑\-मिर्मिर॒मति॑\-मेमिषम्।
आ॒शायै॑ जा॒मिम्।
प्र॒ती॒क्षायै॑ कुमा॒रीम्॥१९॥%\anuvakamend[नो॒ द्वि॒ष्म इति॑ परा॒वत॑मर्पयति]




\prashnaend{ब्रह्म॑णे गी॒ताय॒ श्रमा॑य स॒न्धये॑ न॒दीभ्य॑ उथ्सा॒देभ्य॒ ऋत्यै॒ भाया॒ अर्मे᳚भ्यो म॒न्यवे॑ य॒म्यै॑ दश॑दश॒ सरो᳚भ्यो॒ द्वाद॑श प्रति॒श्रुत्का॑यै बीभ॒थ्सायै॒ दश॑दश॒ हसा॑य स॒प्ताक्ष॑रा॒जाय॒ त्रयो॑दश॒ भूम्यै॒ दश॑ वा॒चे षडथ॒ नवैका॒न्नविꣳ॑शतिः॥१९॥}{ब्रह्म॑णे य॒म्यै॑ नव॑दश॥१९॥}{ब्रह्म॑णे कुमा॒रीम्॥}{हरिः॑ ओम्॥}{इति श्रीकृष्णयजुर्वेदीयतैत्तिरीयब्राह्मणे तृतीयाष्टके चतुर्थः प्रपाठकः समाप्तः॥}
\clearpage
\sect{पञ्चमः प्रश्नः}
\setcounter{anuvakam}{0}
\dnsub{तैत्तिरीयब्राह्मणे तृतीयाष्टके पञ्चमः प्रपाठकः}

%3.5.1.1
स॒त्यं प्रप॑द्ये।
ऋ॒तं प्रप॑द्ये।
अ॒मृतं॒ प्रप॑द्ये।
प्र॒जा\-प॑तेः प्रि॒यां त॒नुव॒मना᳚र्तां॒ प्रप॑द्ये।
इ॒दम॒हं प॑ञ्चद॒शेन॒ वज्रे॑ण।
द्वि॒षन्तं॒ भ्रातृ॑व्य॒मव॑ क्रामामि।
यो᳚ऽस्मान्द्वेष्टि॑।
यं च॑ व॒यं द्वि॒ष्मः।
भूर्भुवः॒ सुवः॑।
हिम्॥१॥\anuvakamend[स॒त्यं दश॑]

%3.5.2.1
प्र वो॒ वाजा॑ अ॒भिद्य॑वः।
ह॒विष्म॑न्तो घृ॒ताच्या᳚।
दे॒वाञ्जि॑गाति सुम्न॒युः।
अग्न॒ आया॑हि वी॒तये᳚।
गृ॒णा॒नो ह॒व्यदा॑तये।
नि होता॑ सथ्सि ब॒र्॒हिषि॑।
तं त्वा॑ स॒मिद्भि॑रङ्गिरः।
घृ॒तेन॑ वर्धयामसि।
बृ॒हच्छो॑चा यविष्ठ्य।
स नः॑ पृ॒थुः श्र॒वाय्यम्᳚॥२॥

%3.5.2.2
अच्छा॑ देव विवाससि।
बृ॒हद॑ग्ने सु॒वीर्यम्᳚।
ई॒डेन्यो॑ नम॒स्य॑स्ति॒रः।
तमाꣳ॑सि दर्‌\mbox{}श॒तः।
सम॒ग्निरि॑ध्यते॒ वृषा᳚।
वृषो॑ अ॒ग्निः समि॑ध्यते।
अश्वो॒ न दे॑व॒वाह॑नः।
तꣳ ह॒विष्म॑न्त ईडते।
वृष॑णं त्वा व॒यं वृषन्॑।
वृषा॑णः॒ समि॑धीमहि॥३॥

%3.5.2.3
अग्ने॒ दीद्य॑तं बृ॒हत्।
अ॒ग्निं दू॒तं वृ॑णीमहे।
होता॑रं वि॒श्ववे॑दसम्।
अ॒स्य य॒ज्ञस्य॑ सु॒क्रतुम्᳚।
स॒मि॒ध्यमा॑नो अध्व॒रे।
अ॒ग्निः पा॑व॒क ईड्यः॑।
शो॒चिष्के॑श॒स्तमी॑महे।
समि॑द्धो अग्न आहुत।
दे॒वान् य॑क्षि स्वध्वर।
त्वꣳ हि ह॑व्य॒वाडसि॑।
आ जु॑होत दुव॒स्यत॑।
अ॒ग्निं प्र॑य॒त्य॑ध्व॒रे।
वृ॒णी॒ध्वꣳ ह॑व्य॒वाह॑नम्।
त्वं वरु॑ण उ॒त मि॒त्रो अ॑ग्ने।
त्वां व॑र्धन्ति म॒तिभि॒र्वसि॑ष्ठाः।
त्वे वसु॑ सुषण॒नानि॑ सन्तु।
यू॒यं पा॑त स्व॒स्तिभिः॒ सदा॑ नः॥४॥\anuvakamend[श्र॒वाय्य॑मिधीम॒ह्यसि॑ स॒प्त च॑]

%3.5.3.1
अग्ने॑ म॒हाꣳ अ॑सि ब्राह्मण भारत।
असा॒वसौ᳚।
दे॒वेद्धो॒ मन्वि॑द्धः।
ऋषि॑ष्टुतो॒ विप्रा॑नुमदितः।
क॒वि॒श॒स्तो ब्रह्म॑सꣳशितो घृ॒ताह॑वनः।
प्र॒णीर्य॒ज्ञाना᳚म्।
र॒थीर॑ध्व॒राणा᳚म्।
अ॒तूर्तो॒ होता᳚।
तूर्णि॑र्‌\mbox{}हव्य॒वाट्।
आस्पात्रं॑ जु॒हूर्दे॒वाना᳚म्॥५॥

%3.5.3.2
च॒म॒सो दे॑व॒पानः॑।
अ॒राꣳ इ॑वाग्ने ने॒मिर्दे॒वाꣴस्त्वं प॑रि॒भूर॑सि।
आ व॑ह दे॒वान् यज॑मानाय।
अ॒ग्निम॑ग्न॒ आव॑ह।
सोम॒माव॑ह।
अ॒ग्निमाव॑ह।
प्र॒जा\-प॑ति॒माव॑ह।
अ॒ग्नीषोमा॒वाव॑ह।
इ॒न्द्रा॒ग्नी आव॑ह।
इन्द्र॒माव॑ह।
म॒हे॒न्द्रमाव॑ह।
दे॒वाꣳ आ᳚ज्य॒पाꣳ आव॑ह।
अ॒ग्निꣳ हो॒त्रायाऽऽव॑ह।
स्वं म॑हि॒मान॒माव॑ह।
आ चा᳚ग्ने दे॒वान् वह॑।
सु॒यजा॑ च यज जातवेदः॥६॥\anuvakamend[दे॒वाना॒मिन्द्र॒मा व॑ह॒ षट् च॑]

%3.5.4.1
अ॒ग्निर्\mbox{}होता॒ वेत्व॒ग्निः।
हो॒त्रं वे᳚त्तु प्रावि॒त्रम्।
स्मो व॒यम्।
सा॒धु ते॑ यजमान दे॒वता᳚।
घृ॒तव॑तीमध्वर्यो॒ स्रुच॒मास्य॑स्व।
दे॒वा॒युवं॑ वि॒श्ववा॑राम्।
ईडा॑महै दे॒वाꣳ ई॒डेन्यान्॑।
न॒म॒स्याम॑ नम॒स्यान्॑।
यजा॑म य॒ज्ञियान्॑॥७॥\anuvakamend[अ॒ग्निर्‌\mbox{}होता॒ नव॑]

%3.5.5.1
स॒मिधो॑ अग्न॒ आज्य॑स्य वियन्तु।
तनू॒नपा॑दग्न॒ आज्य॑स्य वेतु।
इ॒डो अ॑ग्न॒ आज्य॑स्य वियन्तु।
ब॒र्॒हिर॑ग्न॒ आज्य॑स्य वेतु।
स्वाहा॒\-ऽग्निम्।
स्वाहा॒ सोमम्᳚।
स्वाहा॒\-ऽग्निम्।
स्वाहा᳚ प्र॒जा\-प॑तिम्।
स्वाहा॒\-ऽग्नीषोमौ᳚।
स्वाहे᳚न्द्रा॒ग्नी।
स्वाहेन्द्रम्᳚।
स्वाहा॑ महे॒न्द्रम्।
स्वाहा॑ दे॒वाꣳ आ᳚ज्य॒पान्।
स्वाहा॒\-ऽग्निꣳ हो॒त्राज्जु॑षा॒णाः।
अग्न॒ आज्य॑स्य वियन्तु॥८॥\anuvakamend[इ॒न्द्रा॒ग्नी पञ्च॑ च]

%3.5.6.1
अ॒ग्निर्वृ॒त्राणि॑ जङ्घनत्।
द्र॒वि॒ण॒स्युर्वि॑प॒न्यया᳚।
समि॑द्धः शु॒क्र आहु॑तः।
जु॒षा॒णो अ॒ग्निराज्य॑स्य वेतु।
त्वꣳ सो॑मासि॒ सत्प॑तिः।
त्वꣳ राजो॒त वृ॑त्र॒हा।
त्वं भ॒द्रो अ॑सि॒ क्रतुः॑।
जु॒षा॒णः सोम॒ आज्य॑स्य ह॒विषो॑ वेतु।
अ॒ग्निः प्र॒त्नेन॒ जन्म॑ना।
शुम्भा॑नस्त॒नुव॒ꣴ॒ स्वाम्।
क॒विर्विप्रे॑ण वावृधे।
जु॒षा॒णो अ॒ग्निराज्य॑स्य वेतु।
सोम॑ गी॒र्भिष्ट्वा॑ व॒यम्।
व॒र्धया॑मो वचो॒विदः॑।
सु॒मृ॒डी॒को न॒ आवि॑श।
जु॒षा॒णः सोम॒ आज्य॑स्य ह॒विषो॑ वेतु॥९॥\anuvakamend[स्वाꣳ षट् च॑]

%3.5.7.1
अ॒ग्निर्मू॒र्धा दि॒वः क॒कुत्।
पतिः॑ पृथि॒व्या अ॒यम्।
अ॒पाꣳ रेताꣳ॑सि जिन्वति।
भुवो॑ य॒ज्ञस्य॒ रज॑सश्च ने॒ता।
यत्रा॑ नि॒युद्भिः॒ सच॑से शि॒वाभिः॑।
दि॒वि मू॒र्धानं॑ दधिषे सुव॒र्॒षाम्।
जि॒ह्वाम॑ग्ने चकृषे हव्य॒वाहम्᳚।
प्रजा॑पते॒ न त्वदे॒तान्य॒न्यः।
विश्वा॑ जा॒तानि॒ परि॒ ता ब॑भूव।
यत्का॑मास्ते जहु॒मस्तं नो॑ अस्तु॥१०॥

%3.5.7.2
व॒यꣴ स्या॑म॒ पत॑यो रयी॒णाम्।
स वे॑द पु॒त्रः पि॒तर॒ꣳ॒ स मा॒तरम्᳚।
स सू॒नुर्भु॑व॒थ्स भु॑व॒त्पुन॑र्मघः।
स द्यामौर्णो॑द॒न्तरि॑क्ष॒ꣳ॒ स सुवः॑।
स विश्वा॒ भुवो॑ अभव॒थ्स आभ॑वत्।
अग्नी॑षोमा॒ सवे॑दसा।
सहू॑ती वनत॒ङ्गिरः॑।
सन्दे॑व॒त्रा ब॑भूवथुः।
यु॒वमे॒तानि॑ दि॒वि रो॑च॒नानि॑।
अ॒ग्निश्च॑ सोम॒ सक्र॑तू अधत्तम्॥११॥

%3.5.7.3
यु॒वꣳ सिन्धूꣳ॑ र॒भिश॑स्तेरव॒द्यात्।
अग्नी॑षोमा॒वमु॑ञ्चतं गृभी॒तान्।
इन्द्रा᳚ग्नी रोच॒ना दि॒वः।
परि॒ वाजे॑षु भूषथः।
तद्वा᳚ञ्चेति॒ प्रवी॒र्यम्᳚।
श्ञथ॑द्वृ॒त्रमु॒त स॑नोति॒ वाजम्᳚।
इन्द्रा॒ यो अ॒ग्नी सहु॑री सप॒र्यात्।
इ॒र॒ज्यन्ता॑ वस॒व्य॑स्य॒ भूरेः᳚।
सह॑स्तमा॒ सह॑सा वाज॒यन्ता᳚।
एन्द्र॑ सान॒सिꣳ र॒यिम्॥१२॥

%3.5.7.4
स॒जित्वा॑नꣳ सदा॒सहम्᳚।
वर्‌\mbox{}षि॑ष्ठमू॒तये॑ भर।
प्रस॑साहिषे पुरुहूत॒ शत्रून्॑।
ज्येष्ठ॑स्ते॒ शुष्म॑ इ॒ह रा॒तिर॑स्तु।
इन्द्रा भ॑र॒ दक्षि॑णेना॒ वसू॑नि।
पतिः॒ सिन्धू॑नामसि रे॒वती॑नाम्।
म॒हाꣳ इन्द्रो॒ य ओज॑सा।
प॒र्जन्यो॑ वृष्टि॒माꣳ इ॑व।
स्तोमै᳚र्व॒थ्सस्य॑ वावृधे।
म॒हाꣳ इन्द्रो॑ नृ॒वदाच॑र्‌\mbox{}षणि॒प्राः॥१३॥

%3.5.7.5
उ॒त द्वि॒बर्‌\mbox{}हा॑ अमि॒नः सहो॑भिः।
अ॒स्म॒द्रिय॑ग्वावृधे वी॒र्या॑य।
उ॒रुः पृ॒थुः सुकृ॑तः क॒र्तृभि॑र्भूत्।
पि॒प्री॒हि दे॒वाꣳ उ॑श॒तो य॑विष्ठ।
वि॒द्वाꣳ ऋ॒तूꣳर्\mbox{}ऋ॑तुपते यजे॒ह।
ये दैव्या॑ ऋ॒त्विज॒स्तेभि॑रग्ने।
त्वꣳ होतॄ॑णाम॒स्याय॑जिष्ठः।
अ॒ग्निꣴ स्वि॑ष्ट॒कृतम्᳚।
अया॑ड॒ग्निर॒ग्नेः प्रि॒या धामा॑नि।
अया॒ट्थ्सोम॑स्य प्रि॒या धामा॑नि॥१४॥

%3.5.7.6
अया॑ड॒ग्नेः प्रि॒या धामा॑नि।
अया᳚ट्प्र॒जा\-प॑तेः प्रि॒या धामा॑नि।
अया॑ड॒ग्नीषोम॑योः प्रि॒या धामा॑नि।
अया॑डिन्द्राग्नि॒योः प्रि॒या धामा॑नि।
अया॒डिन्द्र॑स्य प्रि॒या धामा॑नि।
अया᳚ण्महे॒न्द्रस्य॑ प्रि॒या धामा॑नि।
अया᳚ड्दे॒वाना॑माज्य॒पानां᳚ प्रि॒या धामा॑नि।
यक्ष॑द॒ग्नेर्‌\mbox{}होतुः॑ प्रि॒या धामा॑नि।
यक्ष॒थ्स्वं म॑हि॒मानम्᳚।
आय॑जता॒मेज्या॒ इषः॑।
कृ॒णोतु॒ सो अ॑ध्व॒रा जा॒तवे॑दाः।
जु॒षताꣳ॑ ह॒विः।
अग्ने॒ यद॒द्य वि॒शो अ॑ध्वरस्य होतः।
पाव॑क शोचे॒ वेष्ट्वꣳ हि यज्वा᳚।
ऋ॒ता य॑जासि महि॒ना वियद्भूः।
ह॒व्या व॑ह यविष्ठ॒ या ते॑ अ॒द्य॥१५॥\anuvakamend[अ॒स्त्व॒ध॒त्त॒ꣳ॒ र॒यिं च॑र्‌\mbox{}षणि॒प्राः सोम॑स्य प्रि॒या धामा॒नीषः॒ षट्च॑]

%3.5.8.1
उप॑हूतꣳ रथन्त॒रꣳ स॒ह पृ॑थि॒व्या।
उप॑ मा रथन्त॒रꣳ स॒ह पृ॑थि॒व्या ह्व॑यताम्।
उप॑हूतं वामदे॒व्यꣳ स॒हान्तरि॑क्षेण।
उप॑ मा वामदे॒व्यꣳ स॒हान्तरि॑क्षेण ह्वयताम्।
उप॑हूतं बृ॒हथ्स॒ह दि॒वा।
उप॑ मा बृ॒हथ्स॒ह दि॒वा ह्व॑यताम्।
उप॑हूताः स॒प्त होत्राः᳚।
उप॑ मा स॒प्त होत्रा᳚ ह्वयन्ताम्।
उप॑हूता धे॒नुः स॒हर्‌\mbox{}ष॑भा।
उप॑ मा धे॒नुः स॒हर्‌\mbox{}ष॑भा ह्वयताम्॥१६॥

%3.5.8.2
उप॑हूतो भ॒क्षः सखा᳚।
उप॑ मा भ॒क्षः सखा᳚ ह्वयताम्।
उप॑हू॒ताँ(४)हो।
इडोप॑हूता।
उप॑हू॒तेडा᳚।
उपो॑ अ॒स्माꣳ इडा᳚ ह्वयताम्।
इडोप॑हूता।
उप॑हू॒तेडा᳚।
मा॒न॒वी घृ॒तप॑दी मैत्रावरु॒णी।
ब्रह्म॑ दे॒वकृ॑त॒मुप॑हूतम्॥१७॥

%3.5.8.3
दैव्या॑ अध्व॒र्यव॒ उप॑हूताः।
उप॑हूता मनु॒ष्याः᳚।
य इ॒मं य॒ज्ञमवान्॑।
ये य॒ज्ञप॑तिं॒ वर्धान्॑।
उप॑हूते॒ द्यावा॑पृथि॒वी।
पू॒र्व॒जे ऋ॒ताव॑री।
दे॒वी दे॒वपु॑त्रे।
उप॑हूतो॒ऽयं यज॑मानः।
उत्त॑रस्यान्देवय॒ज्याया॒मुप॑हूतः।
भूय॑सि हवि॒ष्कर॑ण॒ उप॑हूतः।
दि॒व्ये धाम॒न्नुप॑हूतः।
इ॒दं मे॑ दे॒वा ह॒विर्जु॑षन्ता॒मिति॒ तस्मि॒न्नुप॑हूतः।
विश्व॑मस्य प्रि॒यमुप॑हूतम्।
विश्व॑स्य प्रि॒यस्योप॑हूत॒स्योप॑हूतः॥१८॥\anuvakamend[स॒हर्‌\mbox{}ष॑भा ह्वयता॒मुप॑हूतꣳ हवि॒ष्कर॑ण॒ उप॑हूतश्च॒त्वारि॑ च]

%3.5.9.1
दे॒वं ब॒र्‌॒\mbox{}हिः।
व॒सु॒वने॑ वसु॒धेय॑स्य वेतु।
दे॒वो नरा॒शꣳसः॑।
व॒सु॒वने॑ वसु॒धेय॑स्य वेतु।
दे॒वो अ॒ग्निः स्वि॑ष्ट॒कृत्।
सु॒द्रवि॑णा म॒न्द्रः क॒विः।
स॒त्यम॑न्माय॒जी होता᳚।
होतु॑र्‌\mbox{}होतु॒राय॑जीयान्।
अग्ने॒ यान्दे॒वानया᳚ट्।
याꣳ अपि॑प्रेः।
ये ते॑ हो॒त्रे अम॑थ्सत।
ताꣳ स॑स॒नुषी॒ꣳ॒ होत्रा᳚न्देवङ्ग॒माम्।
दि॒वि दे॒वेषु॑ य॒ज्ञमेर॑ये॒मम्।
स्वि॒ष्ट॒कृच्चाग्ने॒ होता\-ऽभूः᳚।
व॒सु॒वने॑ वसु॒धेय॑स्य नमोवा॒के वीहि॑॥१९॥\anuvakamend[अपि॑प्रेः॒ पञ्च॑ च]

%3.5.10.1
इ॒दं द्या॑वापृथिवी भ॒द्रम॑भूत्।
आर्ध्म॑ सूक्तवा॒कम्।
उ॒त न॑मोवा॒कम्।
ऋ॒ध्यास्म॑ सू॒क्तोच्य॑मग्ने।
त्वꣳ सू᳚क्त॒वाग॑सि।
उप॑श्रितो दि॒वः पृ॑थि॒व्योः।
ओम॑न्वती ते॒\-ऽस्मिन् य॒ज्ञे य॑जमान॒ द्यावा॑पृथि॒वी स्ता᳚म्।
श॒ङ्ग॒ये जी॒रदा॑नू।
अत्र॑स्नू॒ अप्र॑वेदे।
उ॒रुग॑व्यूती अभयं॒ कृतौ᳚॥२०॥

%3.5.10.2
वृ॒ष्टिद्या॑वा री॒त्या॑पा।
श॒म्भुवौ॑ मयो॒भुवौ᳚।
ऊर्ज॑स्वती च॒ पय॑स्वती च।
सू॒प॒च॒र॒णा च॑ स्वधिचर॒णा च॑।
तयो॑रा॒विदि॑।
अ॒ग्निरि॒दꣳ ह॒विर॑जुषत।
अवी॑वृधत॒ महो॒ ज्यायो॑ऽकृत।
सोम॑ इ॒दꣳह॒विर॑जुषत।
अवी॑वृधत॒ महो॒ ज्यायो॑ऽकृत।
अ॒ग्निरि॒दꣳ ह॒विर॑जुषत॥२१॥

%3.5.10.3
अवी॑वृधत॒ महो॒ ज्यायो॑ऽकृत।
प्र॒जा\-प॑तिरि॒दꣳ ह॒विर॑जुषत।
अवी॑वृधत॒ महो॒ ज्यायो॑ऽकृत।
अ॒ग्नीषोमा॑वि॒दꣳ ह॒विर॑जुषेताम्।
अवी॑वृधेतां॒ महो॒ ज्यायो᳚\-ऽक्राताम्।
इ॒न्द्रा॒ग्नी इ॒दꣳ ह॒विर॑जुषेताम्।
अवी॑वृधेतां॒ महो॒ ज्यायो᳚\-ऽक्राताम्।
इन्द्र॑ इ॒दꣳ ह॒विर॑जुषत।
अवी॑वृधत॒ महो॒ ज्यायो॑ऽकृत।
म॒हे॒न्द्र इ॒दꣳ ह॒विर॑जुषत॥२२॥

%3.5.10.4
अवी॑वृधत॒ महो॒ ज्यायो॑ऽकृत।
दे॒वा आ᳚ज्य॒पा आज्य॑मजुषन्त।
अवी॑वृधन्त॒ महो॒ ज्यायो᳚\-ऽक्रत।
अ॒ग्निर्‌\mbox{}हो॒त्रेणे॒दꣳ ह॒विर॑जुषत।
अवी॑वृधत॒ महो॒ ज्यायो॑ऽकृत।
अ॒स्यामृध॒द्धोत्रा॑यान्देवङ्ग॒माया᳚म्।
आशा᳚स्ते॒ऽयं यज॑मानो॒ऽसौ।
आयु॒रा शा᳚स्ते।
सु॒प्र॒जा॒स्त्वमा शा᳚स्ते।
स॒जा॒त॒व॒न॒स्यामा शा᳚स्ते॥२३॥

%3.5.10.5
उत्त॑रान्देवय॒ज्यामा शा᳚स्ते।
भूयो॑ हवि॒ष्कर॑ण॒मा शा᳚स्ते।
दि॒व्यं धामा शा᳚स्ते।
विश्वं॑ प्रि॒यमा शा᳚स्ते।
यद॒नेन॑ ह॒विषा\-ऽऽशा᳚स्ते।
तद॑श्या॒त्तदृ॑ध्यात्।
तद॑स्मै दे॒वा रा॑सन्ताम्।
तद॒ग्निर्दे॒वो दे॒वेभ्यो॒ वन॑ते।
व॒यम॒ग्नेर्मानु॑षाः।
इ॒ष्टं च॑ वी॒तं च॑।
उ॒भे च॑ नो॒ द्यावा॑पृथि॒वी अꣳह॑सस्पाताम्।
इ॒ह गति॑र्वा॒मस्ये॒दं च॑।
नमो॑ दे॒वेभ्यः॑॥२४॥\anuvakamend[अ॒भ॒यं॒ कृता॑वकृता॒ग्निरि॒दꣳ ह॒विर॑जुषत महे॒न्द्र इ॒दꣳ ह॒विर॑जुषत सजातवन॒स्यामा शा᳚स्ते वी॒तं च॒ त्रीणि॑ च]

%3.5.11.1
तच्छं॒ योरावृ॑णीमहे।
गा॒तुं य॒ज्ञाय॑।
गा॒तुं य॒ज्ञप॑तये।
दैवी᳚ स्व॒स्तिर॑स्तु नः।
स्व॒स्तिर्मानु॑षेभ्यः।
ऊ॒र्ध्वं जि॑गातु भेष॒जम्।
शं नो॑ अस्तु द्वि॒पदे᳚।
शं चतु॑ष्पदे॥२५॥\anuvakamend[तच्छं॒ योर॒ष्टौ]

%3.5.12.1
आप्या॑यस्व॒ सन्ते᳚।
इ॒ह त्वष्टा॑रमग्रि॒यं तन्न॑स्तु॒रीपम्᳚।
दे॒वानां॒ पत्नी॑रुश॒तीर॑वन्तु नः।
प्राव॑न्तु नस्तु॒जये॒ वाज॑सातये।
याः पार्थि॑वासो॒ या अ॒पामपि॑ व्र॒ते।
ता नो॑ देवीः सुहवाः॒ शर्म॑ यच्छत।
उ॒त ग्ना वि॑यन्तु दे॒वप॑त्नीः।
इ॒न्द्रा॒ण्य॑ग्नाय्य॒श्विनी॒ राट्।
आ रोद॑सी वरुणा॒नी शृ॑णोतु।
वि॒यन्तु॑ दे॒वीर्य ऋ॒तुर्जनी॑नाम्॥२६॥

%3.5.12.2
अ॒ग्निर्‌\mbox{}होता॑ गृ॒हप॑तिः॒ स राजा᳚।
विश्वा॑ वेद॒ जनि॑मा जा॒तवे॑दाः।
दे॒वाना॑मु॒त यो मर्त्या॑नाम्।
यजि॑ष्ठः॒ स प्र य॑जतामृ॒तावा᳚।
व॒यमु॑ त्वा गृहपते॒ जना॑नाम्।
अग्ने॒ अक॑र्म स॒मिधा॑ बृ॒हन्तम्᳚।
अ॒स्थू॒रि णो॒ गार्‌\mbox{}ह॑पत्यानि सन्तु।
ति॒ग्मेन॑ न॒स्तेज॑सा॒ सꣳशि॑शाधि॥२७॥\anuvakamend[जनी॑नाम॒ष्टौ च॑]

%3.5.13.1
उप॑हूतꣳ रथन्त॒रꣳ स॒ह पृ॑थि॒व्या।
उप॑ मा रथन्त॒रꣳ स॒ह पृ॑थि॒व्या ह्व॑यताम्।
उप॑हूतं वामदे॒व्यꣳ स॒हान्तरि॑क्षेण।
उप॑ मा वामदे॒व्यꣳ स॒हान्तरि॑क्षेण ह्वयताम्।
उप॑हूतं बृ॒हथ्स॒ह दि॒वा।
उप॑ मा बृ॒हथ्स॒ह दि॒वा ह्व॑यताम्।
उप॑हूताः स॒प्त होत्राः᳚।
उप॑ मा स॒प्त होत्रा᳚ ह्वयन्ताम्।
उप॑हूता धे॒नुः स॒हर्\mbox{}ष॑भा।
उप॑ मा धे॒नुः स॒हर्‌\mbox{}ष॑भा ह्वयताम्॥२८॥

%3.5.13.2
उप॑हूतो भ॒क्षः सखा᳚।
उप॑ मा भ॒क्षः सखा᳚ ह्वयताम्।
उप॑हू॒ताँ(४)हो।
इडोप॑हूता।
उप॑हू॒तेडा᳚।
उपो॑ अ॒स्माꣳ इडा᳚ ह्वयताम्।
इडोप॑हूता।
उप॑हू॒तेडा᳚।
मा॒न॒वी घृ॒तप॑दी मैत्रावरु॒णी।
ब्रह्म॑ दे॒वकृ॑त॒मुप॑हूतम्॥२९॥

%3.5.13.3
दैव्या॑ अध्व॒र्यव॒ उप॑हूताः।
उप॑हूता मनु॒ष्याः᳚।
य इ॒मं य॒ज्ञमवान्॑।
ये य॒ज्ञप॑त्नीं॒ वर्धान्॑।
उप॑हूते॒ द्यावा॑पृथि॒वी।
पू॒र्व॒जे ऋ॒ताव॑री।
दे॒वी दे॒वपु॑त्रे।
उप॑हूते॒यं यज॑माना।
इ॒न्द्रा॒णीवा॑\-ऽविध॒वा।
अदि॑तिरिव सुपु॒त्रा।
उत्त॑रस्यान्देवय॒ज्याया॒मुप॑हूता।
भूय॑सि हवि॒ष्कर॑ण॒ उप॑हूता।
दि॒व्ये धाम॒न्नुप॑हूता।
इ॒दं मे॑ दे॒वा ह॒विर्जु॑षन्ता॒मिति॒ तस्मि॒न्नुप॑हूता।
विश्व॑मस्याः प्रि॒यमुप॑हूतम्।
विश्व॑स्य प्रि॒यस्योप॑हूत॒स्योप॑हूता॥३०॥\anuvakamend[स॒हर्\mbox{}ष॑भा ह्वयता॒मुप॑हूतꣳ सुपु॒त्रा षट्च॑]




\prashnaend{स॒त्यं प्रवो\-ऽग्ने॑ म॒हान॒ग्निर्\mbox{}होता॑ स॒मिधो॒\-ऽग्निर्वृ॒त्राण्य॒ग्निर्मू॒र्धोप॑हूतं दे॒वं ब॒र्॒हिरि॒दं द्या॑वापृथिवी॒ तच्छं॒ योरा प्या॑य॒स्वोप॑हूत॒न्त्रयो॑दश॥१३॥}{स॒त्यं व॒यꣴ स्या॑म वृ॒ष्टिद्या॑वा त्रि॒ꣳ॒शत्॥३०॥}{स॒त्यमुप॑हूता॥}{हरिः॑ ओम्॥}{इति श्रीकृष्णयजुर्वेदीयतैत्तिरीयब्राह्मणे तृतीयाष्टके पञ्चमः प्रपाठकः समाप्तः॥}
\clearpage
\sect{षष्ठमः प्रश्नः}
\setcounter{anuvakam}{0}
\dnsub{तैत्तिरीयब्राह्मणे तृतीयाष्टके षष्ठः प्रपाठकः}

%3.6.1.1
अ॒ञ्जन्ति॒ त्वाम॑ध्व॒रे दे॑व॒यन्तः॑।
वन॑स्पते॒ मधु॑ना॒ दैव्ये॑न।
यदू॒र्ध्वस्ति॑ष्ठा॒द्द्रवि॑णे॒ह ध॑त्तात्।
यद्वा॒ क्षयो॑ मा॒तुर॒स्या उ॒पस्थे᳚।
उच्छ्र॑यस्व वनस्पते।
वर्ष्म॑न्पृथि॒व्या अधि॑।
सुमि॑ती मी॒यमा॑नः।
वर्चो॑धा य॒ज्ञवा॑हसे।
समि॑द्धस्य॒ श्रय॑माणः पु॒रस्ता᳚त्।
ब्रह्म॑ वन्वा॒नो अ॒जरꣳ॑ सु॒वीरम्᳚॥१॥

%3.6.1.2
आ॒रे अ॒स्मदम॑तिं॒ बाध॑मानः।
उच्छ्र॑यस्व मह॒ते सौभ॑गाय।
ऊ॒र्ध्व ऊ॒ षु ण॑ ऊ॒तये᳚।
तिष्ठा॑ दे॒वो न स॑वि॒ता।
ऊ॒र्ध्वो वाज॑स्य॒ सनि॑ता॒ यद॒ञ्जिभिः॑।
वा॒घद्भि॑र्वि॒ह्वया॑महे।
ऊ॒र्ध्वो नः॑ पा॒ह्यꣳह॑सो॒ नि के॒तुना᳚।
विश्व॒ꣳ॒ सम॒त्रिणं॑ दह।
कृ॒धी न॑ ऊ॒र्ध्वां च॒ रथा॑य जी॒वसे᳚।
वि॒दा दे॒वेषु॑ नो॒ दुवः॑॥२॥

%3.6.1.3
जा॒तो जा॑यते सुदिन॒त्वे अह्ना᳚म्।
सम॒र्य आ वि॒दथे॒ वर्ध॑मानः।
पु॒नन्ति॒ धीरा॑ अ॒पसो॑ मनी॒षा।
दे॒व॒या विप्र॒ उदि॑यर्ति॒ वाचम्᳚।
युवा॑ सु॒वासाः॒ परि॑वीत॒ आगा᳚त्।
स उ॒ श्रेया᳚न्भवति॒ जाय॑मानः।
तं धीरा॑सः क॒वय॒ उन्न॑यन्ति।
स्वा॒धियो॒ मन॑सा देव॒यन्तः॑।
पृ॒थु॒पाजा॒ अम॑र्त्यः।
घृ॒तनि॑र्णि॒ख्स्वा॑हुतः।
अ॒ग्निर्य॒ज्ञस्य॑ हव्य॒वाट्।
तꣳ स॒बाधो॑ य॒तः स्रु॑चः।
इ॒त्था धि॒या य॒ज्ञव॑न्तः।
आच॑क्रुर॒ग्निमू॒तये᳚।
त्वं वरु॑ण उ॒त मि॒त्रो अ॑ग्ने।
त्वां व॑र्धन्ति म॒तिभि॒र्वसि॑ष्ठाः।
त्वे वसु॑ सुषण॒नानि॑ सन्तु।
यू॒यं पा॑त स्व॒स्तिभिः॒ सदा॑ नः॥३॥\anuvakamend[सु॒वीरं॒ दुवः॒ स्वा॑हुतो॒\-ऽष्टौ च॑]

%3.6.2.1
होता॑ यक्षद॒ग्निꣳ स॒मिधा॑ सुष॒मिधा॒ समि॑द्धं॒ नाभा॑ पृथि॒व्याः स॑ङ्ग॒थे वा॒मस्य॑।
वर्ष्म॑न्दि॒व इ॒डस्प॒दे वेत्वा\-ऽ\-ऽ\-ज्य॑स्य॒ होत॒र्यज॑।
होता॑ यक्ष॒त्तनू॒नपा॑त॒मदि॑ते॒र्गर्भं॒ भुव॑नस्य गो॒पाम्।
मध्वा॒द्य दे॒वो दे॒वेभ्यो॑ देव॒याना᳚न्प॒थो अ॑नक्तु॒ वेत्वा\-ऽ\-ऽ\-ज्य॑स्य॒ होत॒र्यज॑।
होता॑ यक्ष॒न्नरा॒शꣳसं॑ नृश॒स्त्रं नॄꣴः प्र॑णेत्रम्।
गोभि॑र्व॒पावा॒न्थ्स्याद्वी॒रैः शक्ती॑वा॒न्रथैः᳚ प्रथम॒या वा॒ हिर॑ण्यैश्च॒न्द्री वेत्वा\-ऽ\-ऽ\-ज्य॑स्य॒ होत॒र्यज॑।
होता॑ यक्षद॒ग्निमि॒ड ई॑डि॒तो दे॒वो दे॒वाꣳ आव॑क्षद्दू॒तो ह॑व्य॒वाडमू॑रः।
उपे॒मं य॒ज्ञमुपे॒मां दे॒वो दे॒वहू॑तिमवतु॒ वेत्वा\-ऽ\-ऽ\-ज्य॑स्य॒ होत॒र्यज॑।
होता॑ यक्षद्ब॒र्‌॒\mbox{}हिः सु॒ष्टरी॒मोर्ण॑म्रदा अ॒स्मिन् य॒ज्ञे वि च॒ प्र च॑ प्रथताꣴ स्वास॒स्थं दे॒वेभ्यः॑।
एमे॑नद॒द्य वस॑वो रु॒द्रा आ॑दि॒त्याः स॑दन्तु प्रि॒यमिन्द्र॑स्यास्तु॒ वेत्वा\-ऽ\-ऽ\-ज्य॑स्य॒ होत॒र्यज॑॥४॥

%3.6.2.2
होता॑ यक्ष॒द्दुर॑ ऋ॒ष्वाः क॑व॒ष्यो को॑षधावनी॒रुदाता॑भी॒र्जिह॑तां॒ विपक्षो॑भिः श्रयन्ताम्।
सु॒प्रा॒य॒णा अ॒स्मिन् य॒ज्ञे विश्र॑यन्तामृता॒वृधो॑ वि॒यन्त्वाज्य॑स्य॒ होत॒र्यज॑।
होता॑ यक्षदु॒षासा॒नक्ता॑ बृह॒ती सु॒पेश॑सा॒ नॄꣴः पति॑भ्यो॒ योनिं॑ कृण्वा॒ने।
स॒ꣴ॒स्मय॑माने॒ इन्द्रे॑ण दे॒वैरेदं ब॒र्॒हिः सी॑दतां वी॒तामाज्य॑स्य॒ होत॒र्यज॑।
होता॑ यक्ष॒द्दैव्या॒ होता॑रा म॒न्द्रा पोता॑रा क॒वी प्रचे॑तसा।
स्वि॑ष्टम॒द्यान्यः क॑रदि॒षा स्व॑भिगूर्तम॒न्य ऊ॒र्जा सत॑वसे॒मं य॒ज्ञं दि॒वि दे॒वेषु॑ धत्तां वी॒तामाज्य॑स्य॒ होत॒र्यज॑।
होता॑ यक्षत्ति॒स्रो दे॒वीर॒पसा॑म॒पस्त॑मा॒ अच्छि॑द्रम॒द्येदमप॑स्तन्वताम्।
दे॒वेभ्यो॑ दे॒वीर्दे॒वमपो॑ वि॒यन्त्वाज्य॑स्य॒ होत॒र्यज॑।
होता॑ यक्ष॒त्त्वष्टा॑र॒मचि॑ष्टु॒मपा॑कꣳ रेतो॒धां विश्र॑वसं यशो॒धाम्।
पु॒रु॒रूप॒मका॑मकर्‌\mbox{}शनꣳ सु॒पोषः॒ पोषैः॒ स्याथ्सु॒वीरो॑ वी॒रैर्वेत्वा\-ऽ\-ऽ\-ज्य॑स्य॒ होत॒र्यज॑।
होता॑ यक्ष॒द्वन॒स्पति॑मु॒पाव॑स्रक्षद्धि॒यो जो॒ष्टारꣳ॑ श॒शम॒न्नरः॑।
स्वदा॒थ्स्वधि॑तिर्\mbox{}ऋतु॒थाद्य दे॒वो दे॒वेभ्यो॑ ह॒व्यावा॒ड्वेत्वा\-ऽ\-ऽ\-ज्य॑स्य॒ होत॒र्यज॑।
होता॑ यक्षद॒ग्निꣴ स्वाहा\-ऽऽज्य॑स्य॒ स्वाहा॒ मेद॑सः॒ स्वाहा᳚ स्तो॒काना॒ꣴ॒ स्वाहा॒ स्वाहा॑कृतीना॒ꣴ॒ स्वाहा॑ ह॒व्यसू᳚क्तीनाम्।
स्वाहा॑ दे॒वाꣳ आ᳚ज्य॒पान्थ्स्वाहा॒\-ऽग्निꣳ हो॒त्राज्जु॑षा॒णा अग्न॒ आज्य॑स्य वियन्तु॒ होत॒र्यज॑॥५॥\anuvakamend[प्रि॒यमिन्द्र॑स्यास्तु॒ वेत्वा\-ऽ\-ऽ\-ज्य॑स्य॒ होत॒र्यज॑ सु॒वीरो॑ वी॒रैर्वेत्वा\-ऽ\-ऽ\-ज्य॑स्य॒ होत॒र्यज॑ च॒त्वारि॑ च (अ॒ग्निन्तनू॒नपा॑त॒न्नरा॒शꣳस॑म॒ग्निमि॒ड ई॑डि॒तो ब॒र्‌॒\mbox{}हिर्दुर॑ उ॒षासा॒नक्ता॒ दैव्या॑ ति॒स्रस्त्वष्टा॑रं॒ वन॒स्पति॑म॒ग्निम्।
पञ्च॒ वेत्वेको॑ वि॒यन्तु॒ द्विर्वी॒तामेको॑ वि॒यन्तु॒ द्विर्वेत्वेको॑ वियन्तु॒ होत॒र्यज॑॥)]

%3.6.3.1
समि॑द्धो अ॒द्य मनु॑षो दुरो॒णे।
दे॒वो दे॒वान् य॑जसि जातवेदः।
आ च॒ वह॑ मित्रमहश्चिकि॒त्वान्।
त्वं दू॒तः क॒विर॑सि॒ प्रचे॑ताः।
तनू॑नपात्प॒थ ऋ॒तस्य॒ यानान्॑।
मध्वा॑ सम॒ञ्जन्थ्स्व॑दया सुजिह्व।
मन्मा॑नि धी॒भिरु॒त य॒ज्ञमृ॒न्धन्।
दे॒व॒त्रा च॑ कृणुह्यध्व॒रं नः॑।
नरा॒शꣳस॑स्य महि॒मान॑मेषाम्।
उप॑ स्तोषाम यज॒तस्य॑ य॒ज्ञैः॥६॥

%3.6.3.2
ते सु॒क्रत॑वः॒ शुच॑यो धिय॒न्धाः।
स्वद॑न्तु दे॒वा उ॒भया॑नि ह॒व्या।
आ॒जुह्वा॑न॒ ईड्यो॒ वन्द्य॑श्च।
आया᳚ह्यग्ने॒ वसु॑भिः स॒जोषाः᳚।
त्वं दे॒वाना॑मसि यह्व॒ होता᳚।
स ए॑नान् यक्षीषि॒तो यजी॑यान्।
प्रा॒चीनं॑ ब॒र्‌॒\mbox{}हिः प्र॒दिशा॑ पृथि॒व्याः।
वस्तो॑र॒स्या वृ॑ज्यते॒ अग्रे॒ अह्ना᳚म्।
व्यु॑ प्रथते वित॒रं वरी॑यः।
दे॒वेभ्यो॒ अदि॑तये स्यो॒नम्॥७॥

%3.6.3.3
व्यच॑स्वतीरुर्वि॒या विश्र॑यन्ताम्।
पति॑भ्यो॒ न जन॑यः॒ शुम्भ॑\-मानाः।
देवी᳚र्द्वारो बृहतीर्विश्वमिन्वाः।
दे॒वेभ्यो॑ भवथ सुप्राय॒णाः।
आसु॒ष्वय॑न्ती यज॒ते उपा॑के।
उ॒षासा॒नक्ता॑ सदतां॒ नि योनौ᳚।
दि॒व्ये योष॑णे बृह॒ती सु॑रु॒क्मे।
अधि॒ श्रियꣳ॑ शुक्र॒पिशं॒ दधा॑ने।
दैव्या॒ होता॑रा प्रथ॒मा सु॒वाचा᳚।
मिमा॑ना य॒ज्ञं मनु॑षो॒ यज॑ध्यै॥८॥

%3.6.3.4
प्र॒चो॒दय॑न्ता वि॒दथे॑षु का॒रू।
प्रा॒चीनं॒ ज्योतिः॑ प्र॒दिशा॑ दि॒शन्ता᳚।
आ नो॑ य॒ज्ञं भार॑ती॒ तूय॑मेतु।
इडा॑ मनु॒ष्वदि॒ह चे॒तय॑न्ती।
ति॒स्रो दे॒वीर्ब॒र्‌॒\mbox{}हिरेदꣴ स्यो॒नम्।
सर॑स्वती॒ स्वप॑सः सदन्तु।
य इ॒मे द्यावा॑पृथि॒वी जनि॑त्री।
रू॒पैरपिꣳ॑श॒द्भुव॑नानि॒ विश्वा᳚।
तम॒द्य हो॑तरिषि॒तो यजी॑यान्।
दे॒वं त्वष्टा॑रमि॒ह य॑क्षि वि॒द्वान्॥९॥

%3.6.3.5
उ॒पाव॑सृज॒त्मन्या॑ सम॒ञ्जन्।
दे॒वानां॒ पाथ॑ ऋतु॒था ह॒वीꣳषि॑।
वन॒स्पतिः॑ शमि॒ता दे॒वो अ॒ग्निः।
स्वद॑न्तु ह॒व्यं मधु॑ना घृ॒तेन॑।
स॒द्यो जा॒तो व्य॑मिमीत य॒ज्ञम्।
अ॒ग्निर्दे॒वाना॑मभवत्पुरो॒गाः।
अ॒स्य होतुः॑ प्र॒दिश्यृ॒तस्य॑ वा॒चि।
स्वाहा॑कृतꣳ ह॒विर॑दन्तु दे॒वाः॥१०॥\anuvakamend[य॒ज्ञैः स्यो॒नं यज॑ध्यै वि॒द्वान॒ष्टौ च॑]

%3.6.4.1
अ॒ग्निर्‌\mbox{}होता॑ नो अध्व॒रे।
वा॒जी सन्परि॑णीयते।
दे॒वो दे॒वेषु॑ य॒ज्ञियः॑।
परि॑त्रिवि॒ष्ट्य॑ध्व॒रम्।
यात्य॒ग्नी र॒थीरि॑व।
आ दे॒वेषु॒ प्रयो॒ दध॑त्।
परि॒ वाज॑पतिः क॒विः।
अ॒ग्निर्‌\mbox{}ह॒व्यान्य॑क्रमीत्।
दध॒द्रत्ना॑नि दा॒शुषे᳚॥११॥\anuvakamend[अ॒ग्निर्‌\mbox{}होता॑ नो॒ नव॑]

%3.6.5.1
अजै॑द॒ग्निः।
अस॑न॒द्वाज॒न्नि।
दे॒वो दे॒वेभ्यो॑ ह॒व्यावा᳚ट्।
प्राञ्जो॑भिर्‌\mbox{}हिन्वा॒नः।
धेना॑भिः॒ कल्प॑मानः।
य॒ज्ञस्यायुः॑ प्रति॒रन्।
उप॒ प्रेष्य॑ होतः।
ह॒व्या दे॒वेभ्यः॑॥१२॥\anuvakamend[अजै॑द॒ष्टौ]

%3.6.6.1
दैव्याः᳚ शमितार उ॒त म॑नुष्या॒ आर॑भध्वम्।
उप॑नयत॒ मेध्या॒ दुरः॑।
आ॒शासा॑ना॒ मेध॑पतिभ्यां॒ मेधम्᳚।
प्रास्मा॑ अ॒ग्निं भ॑रत।
स्तृ॒णी॒त ब॒र्॒हिः।
अन्वे॑नं मा॒ता म॑न्यताम्।
अनु॑ पि॒ता।
अनु॒ भ्राता॒ सग॑र्भ्यः।
अनु॒ सखा॒ सयू᳚थ्यः।
उ॒दी॒चीनाꣳ॑ अस्य प॒दो निध॑त्तात्॥१३॥

%3.6.6.2
सूर्यं॒ चक्षु॑र्गमयतात्।
वातं॑ प्रा॒णम॒न्वव॑\-सृजतात्।
दिशः॒ श्रोत्रम्᳚।
अ॒न्तरि॑क्ष॒मसुम्᳚।
पृ॒थि॒वीꣳ शरी॑रम्।
ए॒क॒धा\-ऽस्य॒ त्वच॒माच्छ्य॑तात्।
पु॒रा नाभ्या॑ अपि॒शसो॑ व॒पामुत्खि॑दतात्।
अ॒न्तरे॒वोष्माणं॑ वारयतात्।
श्ये॒नम॑स्य॒ वक्षः॑ कृणुतात्।
प्र॒शसा॑ बा॒हू॥१४॥

%3.6.6.3
श॒ला दो॒षणी᳚।
क॒श्यपे॒वाꣳसा᳚।
अच्छि॑द्रे॒ श्रोणी᳚।
क॒वषो॒रू स्रे॒कप॑र्णाष्ठी॒वन्ता᳚।
षड्विꣳ॑शतिरस्य॒ वङ्क्र॑यः।
ता अ॑नु॒ष्ठ्योच्च्या॑वयतात्।
गात्रं॑ गात्रम॒स्यानू॑नं कृणुतात्।
ऊ॒व॒ध्य॒गो॒हं पार्थि॑वं खनतात्।
अ॒स्ना रक्षः॒ सꣳसृ॑जतात्।
व॒नि॒ष्ठुम॑स्य॒ मा रा॑विष्ट॥१५॥

%3.6.6.4
उरू॑कं॒ मन्य॑मानाः।
नेद्व॑स्तो॒के तन॑ये।
रवि॑ता॒रव॑च्छमितारः।
अध्रि॑गो शमी॒ध्वम्।
सु॒शमि॑ शमीध्वम्।
श॒मी॒ध्वम॑ध्रिगो।
अध्रि॑गु॒श्चापा॑पश्च।
उ॒भौ दे॒वानाꣳ॑ शमि॒तारौ᳚।
तावि॒मं प॒शूꣴ श्र॑पयतां प्रवि॒द्वाꣳसौ᳚।
यथा॑यथा\-ऽस्य॒ श्रप॑ण॒न्तथा॑तथा॥१६॥\anuvakamend[ध॒त्ता॒द्बा॒हू मा रा॑विष्ट॒ तथा॑तथा]

%3.6.7.1
जु॒षस्व॑ स॒प्रथ॑स्तमम्।
वचो॑ दे॒वफ्स॑रस्तमम्।
ह॒व्या जुह्वा॑न आ॒सनि॑।
इ॒मं नो॑ य॒ज्ञम॒मृते॑षु धेहि।
इ॒मा ह॒व्या जा॑तवेदो जुषस्व।
स्तो॒काना॑मग्ने॒ मेद॑सो घृ॒तस्य॑।
होतः॒ प्राशा॑न प्रथ॒मो नि॒षद्य॑।
घृ॒तव॑न्तः पावक ते।
स्तो॒काः श्चो॑तन्ति॒ मेद॑सः।
स्वध॑र्मं दे॒ववी॑तये॥१७॥

%3.6.7.2
श्रेष्ठं॑ नो धेहि॒ वार्यम्᳚।
तुभ्यꣴ॑ स्तो॒का घृ॑त॒श्चुतः॑।
अग्ने॒ विप्रा॑य सन्त्य।
ऋषिः॒ श्रेष्ठः॒ समि॑ध्यसे।
य॒ज्ञस्य॑ प्रावि॒ता भ॑व।
तुभ्यꣴ॑ श्चोतन्त्यध्रिगो शचीवः।
स्तो॒कासो॑ अग्ने॒ मेद॑सो घृ॒तस्य॑।
क॒वि॒श॒स्तो बृ॑ह॒ता भा॒नुनागाः᳚।
ह॒व्या जु॑षस्व मेधिर।
ओजि॑ष्ठन्ते मध्य॒तो मेद॒ उद्भृ॑तम्।
प्र ते॑ व॒यं द॑दामहे।
श्चोत॑न्ति ते वसो स्तो॒का अधि॑त्व॒चि।
प्रति॒ तान्दे॑व॒शोवि॑हि॥१८॥\anuvakamend[दे॒ववी॑तय॒ उद्भृ॑त॒न्त्रीणि॑ च]

%3.6.8.1
आवृ॑त्रहणा वृत्र॒हभिः॒ शुष्मैः᳚।
इन्द्र॑ या॒तन्नमो॑भिरग्ने अ॒र्वाक्।
यु॒वꣳ राधो॑भि॒रक॑वेभिरिन्द्र।
अग्ने॑ अ॒स्मे भ॑वतमुत्त॒मेभिः॑।
होता॑ यक्षदिन्द्रा॒ग्नी।
छाग॑स्य व॒पाया॒ मेद॑सः।
जु॒षेताꣳ॑ ह॒विः।
होत॒र्यज॑।
विह्यख्य॒न्मन॑सा॒ वस्य॑ इ॒च्छन्।
इन्द्रा᳚ग्नी ज्ञा॒स उ॒त वा॑ सजा॒तान्॥१९॥\phantom{वा॑}

%3.6.8.2
नान्या यु॒वत्प्रम॑तिरस्ति॒ मह्यम्᳚।
स वां॒ धियं॑ वाज॒यन्ती॑मतक्षम्।
होता॑ यक्षदिन्द्रा॒ग्नी।
पु॒रो॒डाश॑स्य जु॒षेताꣳ॑ ह॒विः।
होत॒र्यज॑।
त्वामी॑डते अजि॒रं दू॒त्या॑य।
ह॒विष्म॑न्तः॒ सद॒मिन्मानु॑षासः।
यस्य॑ दे॒वैरास॑दो ब॒र्‌॒\mbox{}हिर॑ग्ने।
अहा᳚न्यस्मै सु॒दिना॑ भवन्तु।
होता॑ यक्षद॒ग्निम्।
पु॒रो॒डाश॑स्य जु॒षताꣳ॑ ह॒विः।
होत॒र्यज॑॥२०॥\anuvakamend[स॒जा॒तान॒ग्निन्द्वे च॑]

%3.6.9.1
गी॒र्भिर्विप्रः॒ प्रम॑तिमि॒च्छमा॑नः।
ईट्टे॑ र॒यिं य॒शसं॑ पूर्व॒भाजम्᳚।
इन्द्रा᳚ग्नी वृत्रहणा सुवज्रा।
प्र णो॒ नव्ये॑भिस्तिरतं दे॒ष्णैः।
माच्छे᳚द्म र॒श्मीꣳरिति॒ नाध॑मानाः।
पि॒तृ॒णाꣳ शक्ती॑रनु॒\-यच्छ॑मानाः।
इ॒न्द्रा॒ग्निभ्यां॒ कं वृष॑णो मदन्ति।
ताह्यद्री॑ धि॒षणा॑या उ॒पस्थे᳚।
अ॒ग्निꣳ सु॑दी॒तिꣳ सु॒दृशं॑ गृ॒णन्तः॑।
न॒म॒स्याम॒स्त्वेड्यं॑ जातवेदः।
त्वां दू॒तम॑र॒तिꣳ ह॑व्य॒वाहम्᳚।
दे॒वा अ॑कृण्वन्न॒मृत॑स्य॒ नाभिम्᳚॥२१॥\anuvakamend[जा॒त॒वे॒दो॒ द्वे च॑]

%3.6.10.1
त्वꣴ ह्य॑ग्ने प्रथ॒मो म॒नोता᳚।
अ॒स्या धि॒यो अभ॑वो दस्म॒होता᳚।
त्वꣳ सीं वृषन्नकृणोर्दु॒ष्टरी॑तु।
सहो॒ विश्व॑स्मै॒ सह॑से॒ सह॑ध्यै।
अधा॒ होता॒ न्य॑सीदो॒ यजी॑यान्।
इ॒डस्प॒द इ॒षय॒न्नीड्यः॒ सन्।
तं त्वा॒ नरः॑ प्रथ॒मं दे॑व॒यन्तः॑।
म॒हो रा॒ये चि॒तय॑न्तो॒ अनु॑ग्मन्।
वृ॒तेव॒ यन्तं॑ ब॒हुभि॑र्वस॒व्यैः᳚।
त्वे र॒यिं जा॑गृ॒वाꣳसो॒ अनु॑ग्मन्॥२२॥

%3.6.10.2
रुश॑न्तम॒ग्निं द॑र्‌\mbox{}श॒तं बृ॒हन्तम्᳚।
व॒पाव॑न्तं वि॒श्वहा॑ दीदि॒वाꣳसम्᳚।
प॒दं दे॒वस्य॒ नम॑सा वि॒यन्तः॑।
श्र॒व॒स्यवः॒ श्रव॑ आप॒न्नमृ॑क्तम्।
नामा॑नि चिद्दधिरे य॒ज्ञिया॑नि।
भ॒द्रायां ते रणयन्त॒ सन्दृ॑ष्टौ।
त्वां व॑र्धन्ति क्षि॒तयः॑ पृथि॒व्याम्।
त्वꣳ राय॑ उ॒भया॑सो॒ जना॑नाम्।
त्वं त्रा॒ता त॑रणे॒ चेत्यो॑ऽभूः।
पि॒ता मा॒ता सद॒मिन्मानु॑षाणाम्॥२३॥

%3.6.10.3
सप॒र्येण्यः॒ स प्रि॒यो वि॒क्ष्व॑ग्निः।
होता॑ म॒न्द्रो निष॑सादा॒ यजी॑यान्।
तं त्वा॑ व॒यं दम॒ आ दी॑दि॒वाꣳसम्᳚।
उप॑ज्ञु॒बाधो॒ नम॑सा सदेम।
तं त्वा॑ व॒यꣳ सु॒धियो॒ नव्य॑मग्ने।
सु॒म्ना॒यव॑ ईमहे देव॒यन्तः॑।
त्वं विशो॑ अनयो॒ दीद्या॑नः।
दि॒वो अ॑ग्ने बृह॒ता रो॑च॒नेन॑।
वि॒शां क॒विं वि॒श्पति॒ꣳ॒ शश्व॑तीनाम्।
नि॒तोश॑नं वृष॒भं च॑र्‌\mbox{}षणी॒नाम्॥२४॥

%3.6.10.4
प्रेती॑षणि मि॒षय॑न्तं पाव॒कम्।
राज॑न्तम॒ग्निं य॑ज॒तꣳ र॑यी॒णाम्।
सो अ॑ग्न ईजे शश॒मे च॒ मर्तः॑।
यस्त॒ आन॑ट्थ्स॒मिधा॑ ह॒व्यदा॑तिम्।
य आहु॑तिं॒ परि॒ वेदा॒ नमो॑भिः।
विश्वेथ्सवा॒मा द॑धते॒ त्वोतः॑।
अ॒स्मा उ॑ ते॒ महि॑ म॒हे वि॑धेम।
नमो॑भिरग्ने स॒मिधो॒त ह॒व्यैः।
वेदी॑सूनो सहसो गी॒र्भिरु॒क्थैः।
आ ते॑ भ॒द्रायाꣳ॑ सुम॒तौ य॑तेम॥२५॥

%3.6.10.5
आ यस्त॒तन्थ॒ रोद॑सी॒ विभा॒सा।
श्रवो॑भिश्च श्रव॒स्य॑स्तरु॑त्रः।
बृ॒हद्भि॒र्वाजैः॒ स्थवि॑रेभिर॒स्मे।
रे॒वद्भि॑रग्ने वित॒रं वि भा॑हि।
नृ॒वद्व॑सो॒ सद॒मिद्धे᳚ह्य॒स्मे।
भूरि॑तो॒काय॒ तन॑याय प॒श्वः।
पू॒र्वीरिषो॑ बृह॒तीरा॒रे अ॑घाः।
अ॒स्मे भ॒द्रा सौ᳚श्रव॒सानि॑ सन्तु।
पु॒रूण्य॑ग्ने पुरु॒धा त्वा॒या।
वसू॑नि राजन्व॒सुता॑ते अश्याम्।
पु॒रूणि॒ हि त्वे पु॑रुवार॒ सन्ति॑।
अग्ने॒ वसु॑ विध॒ते राज॑नि॒त्वे॥२६॥\anuvakamend[जा॒गृ॒वाꣳसो॒ अनु॑ग्म॒न्मानु॑षाणाञ्चर्‌\mbox{}षणी॒नां य॑तेमाश्या॒न्द्वे च॑]

%3.6.11.1
आभ॑रतꣳ शिक्षतं वज्रबाहू।
अ॒स्माꣳ इ॑न्द्राग्नी अवत॒ꣳ॒ शची॑भिः।
इ॒मे नु ते र॒श्मयः॒ सूर्य॑स्य।
येभिः॑ सपि॒त्वं पि॒तरो॑ न॒ आयन्॑।
होता॑ यक्षदिन्द्रा॒ग्नी।
छाग॑स्य ह॒विष॒ आत्ता॑म॒द्य।
म॒ध्य॒तो मेद॒ उद्भृ॑तम्।
पु॒रा द्वेषो᳚भ्यः।
पु॒रा पौरु॑षेय्या गृ॒भः।
घस्ता᳚न्नू॒नम्॥२७॥

%3.6.11.2
घा॒से अ॑ज्राणां॒ यव॑सप्रथमानाम्।
सु॒मत्क्ष॑राणाꣳ श॒तरु॑द्रि\-याणाम्।
अ॒ग्नि॒ष्वा॒त्तानां॒ पीवो॑पवसनानाम्।
पा॒र्श्व॒तः श्रो॑णि॒तः शि॑ताम॒त उ॑थ्साद॒तः।
अङ्गा॑दङ्गा॒दव॑त्तानाम्।
कर॑त ए॒वेन्द्रा॒ग्नी।
जु॒षेताꣳ॑ ह॒विः।
होत॒र्यज॑।
दे॒वेभ्यो॑ वनस्पते ह॒वीꣳषि॑।
हिर॑ण्यपर्ण प्र॒दिव॑स्ते॒ अर्थम्᳚॥२८॥

%3.6.11.3
प्र॒द॒क्षि॒णिद्र॑श॒नया॑ नि॒यूय॑।
ऋ॒तस्य॑ वक्षि प॒थिभी॒ रजि॑ष्ठैः।
होता॑ यक्ष॒द्वन॒स्पति॑म॒भिहि।
पि॒ष्टत॑मया॒ रभि॑ष्ठया रश॒नयाधि॑त।
यत्रे᳚न्द्राग्नि॒योश्छाग॑स्य ह॒विषः॑ प्रि॒या धामा॑नि।
यत्र॒ वन॒स्पतेः᳚ प्रि॒या पाथाꣳ॑सि।
यत्र॑ दे॒वाना॑माज्य॒पानां᳚ प्रि॒या धामा॑नि।
यत्रा॒ग्नेर्‌\mbox{}होतुः॑ प्रि॒या धामा॑नि।
तत्रै॒तं प्र॒स्तुत्ये॑वोप॒स्तुत्ये॑ वो॒पाव॑स्रक्षत्।
रभी॑याꣳसमिव कृ॒त्वी॥२९॥

%3.6.11.4
कर॑दे॒वं दे॒वो वन॒स्पतिः॑।
जु॒षताꣳ॑ ह॒विः।
होत॒र्यज॑।
पि॒प्री॒हि दे॒वाꣳ उ॑श॒तो य॑विष्ठ।
वि॒द्वाꣳ ऋ॒तूꣳर्\mbox{}ऋ॑तुपते यजे॒ह।
ये दैव्या॑ ऋ॒त्विज॒स्तेभि॑रग्ने।
त्वꣳ होतॄ॑णाम॒स्याय॑जिष्ठः।
होता॑ यक्षद॒ग्निꣴ स्वि॑ष्ट॒कृतम्᳚।
अया॑ड॒ग्निरि॑न्द्राग्नि॒योश्छाग॑स्य ह॒विषः॑ प्रि॒या धामा॑नि।
अया॒ड्वन॒स्पतेः᳚ प्रि॒या पाथाꣳ॑सि।
अया᳚ड्दे॒वाना॑माज्य॒पानां᳚ प्रि॒या धामा॑नि।
यक्ष॑द॒ग्नेर्‌\mbox{}होतुः॑ प्रि॒या धामा॑नि।
यक्ष॒थ्स्वं म॑हि॒मानम्᳚।
आय॑जता॒मेज्या॒ इषः॑।
कृ॒णोतु॒ सो अ॑ध्व॒रा जा॒तवे॑दाः।
जु॒षताꣳ॑ ह॒विः।
होत॒र्यज॑॥३०॥\anuvakamend[नू॒नमर्थं॑ कृ॒त्वी पाथाꣳ॑सि स॒प्त च॑]

%3.6.12.1
उपो॑ ह॒ यद्वि॒दथं॑ वा॒जिनो॒ गूः।
गी॒र्भिर्विप्राः॒ प्रम॑तिमि॒च्छमा॑नाः।
अ॒र्वन्तो॒ न काष्ठा॒न्नक्ष॑माणाः।
इ॒न्द्रा॒ग्नी जोहु॑वतो॒ नर॒स्ते।
वन॑स्पते रश॒नया॑\-ऽभि॒धाय॑।
पि॒ष्टत॑मया व॒युना॑नि वि॒द्वान्।
वह॑ देव॒त्रा दि॑धिषो ह॒वीꣳषि॑।
प्र च॑दा॒तार॑म॒मृते॑षु वोचः।
अ॒ग्निꣴ स्वि॑ष्ट॒कृतम्᳚।
अया॑ड॒ग्निरि॑न्द्राग्नि॒योश्छाग॑स्य ह॒विषः॑ प्रि॒या धामा॑नि॥३१॥

%3.6.12.2
अया॒ड्वन॒स्पतेः᳚ प्रि॒या पाथाꣳ॑सि।
अया᳚ड्दे॒वाना॑माज्य॒पानां᳚ प्रि॒या धामा॑नि।
यक्ष॑द॒ग्नेर्‌\mbox{}होतुः॑ प्रि॒या धामा॑नि।
यक्ष॒थ्स्वं म॑हि॒मानम्᳚।
आय॑जता॒मेज्या॒ इषः॑।
कृ॒णोतु॒ सो अ॑ध्व॒रा जा॒तवे॑दाः।
जु॒षताꣳ॑ ह॒विः।
अग्ने॒ यद॒द्य वि॒शो अ॑ध्वरस्य होतः।
पाव॑क शोचे॒ वेष्ट्वꣳ हि यज्वा᳚।
ऋ॒ता य॑जासि महि॒ना वियद्भूः।
ह॒व्या व॑ह यविष्ठ॒ या ते॑ अ॒द्य॥३२॥\anuvakamend[धामा॑नि॒ भूरेकं॑ च]

%3.6.13.1
दे॒वं ब॒र्॒हिः सु॑दे॒वं दे॒वैः स्याथ्सु॒वीरं॑ वी॒रैर्वस्तो᳚र्वृ॒ज्येता॒क्तोः प्रभ्रि॑ये॒तात्य॒न्यान्रा॒या ब॒र्॒हिष्म॑तो मदेम वसु॒वने॑ वसु॒धेय॑स्य वेतु॒ यज॑।
दे॒वीर्द्वारः॑ सङ्घा॒ते वि॒ड्वीर्याम॑ञ्छिथि॒रा ध्रु॒वा दे॒वहू॑तौ व॒थ्स ई॑मेना॒स्तरु॑ण॒ आमि॑मीयात्कुमा॒रो वा॒ नव॑जातो॒ मैना॒ अर्वा॑ रे॒णुक॑काटः॒ पृण॑ग्वसु॒वने॑ वसु॒धेय॑स्य वियन्तु॒ यज॑।
दे॒वी उ॒षासा॒नक्ता\-ऽद्या॒स्मिन्‌ य॒ज्ञे प्र॑य॒त्य॑ह्वेता॒मपि॑ नू॒नं दैवी॒र्विशः॒ प्राया॑सिष्टा॒ꣳ॒ सुप्री॑ते॒ सुधि॑ते वसु॒वने॑ वसु॒धेय॑स्य वीतां॒ यज॑।
दे॒वी जोष्ट्री॒ वसु॑धिती॒ ययो॑र॒न्या\-ऽघाद्द्वेषाꣳ॑सि यु॒यव॒दान्याव॑क्ष॒द्वसु॒ वार्या॑णि॒ यज॑मानाय वसु॒वने॑ वसु॒धेय॑स्य वीतां॒ यज॑।
दे॒वी ऊ॒र्जाहु॑ती॒ इष॒मूर्ज॑म॒न्याव॑क्ष॒थ्सग्धि॒ꣳ॒ सपी॑तिम॒न्या नवे॑न॒ पूर्वं॒ दय॑मानाः॒ स्याम॑ पुरा॒णेन॒ नवं॒ तामूर्ज॑मू॒र्जाहु॑ती ऊ॒र्जय॑माने अधातां वसु॒वने॑ वसु॒धेय॑स्य वीतां॒ यज॑।
दे॒वा दैव्या॒ होता॑रा॒ नेष्टा॑रा॒ पोता॑रा ह॒ताघ॑शꣳसावाभ॒रद्व॑सू वसु॒वने॑ वसु॒धेय॑स्य वीतां॒ यज॑।
दे॒वीस्ति॒स्रस्ति॒स्रो दे॒वीरिडा॒ सर॑स्वती॒ भार॑ती॒ द्यां भार॑त्यादि॒त्यैर॑स्पृक्ष॒थ्सर॑स्वती॒मꣳ रु॒द्रैर्य॒ज्ञमा॑वीदि॒हैवेड॑या॒ वसु॑मत्या सध॒मादं॑ मदेम वसु॒वने॑ वसु॒धेय॑स्य वियन्तु॒ यज॑।
दे॒वो नरा॒शꣳस॑स्त्रिशी॒र्॒षा ष॑ड॒क्षः श॒तमिदे॑नꣳशितिपृ॒ष्ठा आद॑धति स॒हस्र॑मीं॒ प्रव॑हन्ति मि॒त्रावरु॒णेद॑स्य हो॒त्रमर्\mbox{}ह॑तो॒ बृह॒स्पतिः॑ स्तो॒त्रम॒श्विना\-ऽऽध्व॑र्यवं वसु॒वने॑ वसु॒धेयस्य॑ वेतु॒ यज॑।
दे॒वो वन॒स्पति॑र्व॒र्॒षप्रा॑वा घृ॒तनि॑र्णि॒ग्द्यामग्रे॒णास्पृ॑क्ष॒दान्तरि॑क्षं॒ मध्ये॑नाप्राः पृथि॒वीमुप॑रेणादृꣳहीद्वसु॒वने॑ वसु॒धेय॑स्य वेतु॒ यज॑।
दे॒वं ब॒र्॒हिर्वारि॑तीनां नि॒धेधा॑ऽसि॒ प्रच्यु॑तीना॒मप्र॑\-च्युतन्निकाम॒धर॑णं पुरुस्पा॒र्॒हं यश॑स्वदे॒ना ब॒र्॒हिषा॒\-ऽन्या ब॒र्॒हीꣴष्य॒भि ष्या॑म वसु॒वने॑ वसु॒धेय॑स्य वेतु॒ यज॑।
दे॒वो अ॒ग्निः स्वि॑ष्ट॒कृथ्सु॒द्रवि॑णा म॒न्द्रः क॒विः स॒त्यम॑न्मा\-ऽऽय॒जी होता॒ होतु॑र्\mbox{}होतु॒राय॑जीया॒नग्ने॒ यान्दे॒वानया॒ड्याꣳ अपि॑प्रे॒र्ये ते॑ हो॒त्रे अम॑थ्सत॒ ताꣳ स॑स॒नुषी॒ꣳ॒ होत्रां देवङ्ग॒मान्दि॒वि दे॒वेषु॑ य॒ज्ञमेर॑ये॒मꣴ स्वि॑ष्ट॒कृच्चाग्ने॒ होता\-ऽभू᳚र्वसु॒वने॑ वसु॒धेय॑स्य नमोवा॒के वीहि॒ यज॑॥३३॥\anuvakamend[यजैकं॑ च]

%3.6.14.1
दे॒वं ब॒र्॒हिः।
व॒सु॒वने॑ वसु॒धेय॑स्य वेतु।
दे॒वीर्द्वारः॑।
व॒सु॒वने॑ वसु॒धेय॑स्य वियन्तु।
दे॒वी उ॒षासा॒नक्ता᳚।
व॒सु॒वने॑ वसु॒धेय॑स्य वीताम्।
दे॒वी जोष्ट्री᳚।
व॒सु॒वने॑ वसु॒धेय॑स्य वीताम्।
दे॒वी ऊ॒र्जाहु॑ती।
व॒सु॒वने॑ वसु॒धेयस्य॑ वीताम्॥३४॥

%3.6.14.2
दे॒वा दैव्या॒ होता॑रा।
व॒सु॒वने॑ वसु॒धेय॑स्य वीताम्।
दे॒वीस्ति॒स्रस्ति॒स्रो दे॒वीः।
व॒सु॒वने॑ वसु॒धेय॑स्य वियन्तु।
दे॒वो नरा॒शꣳसः॑।
व॒सु॒वने॑ वसु॒धेय॑स्य वेतु।
दे॒वो वन॒स्पतिः॑।
व॒सु॒वने॑ वसु॒धेय॑स्य वेतु।
दे॒वं ब॒र्॒हिर्वारि॑तीनाम्।
व॒सु॒वने॑ वसु॒धेय॑स्य वेतु॥३५॥

%3.6.14.3
दे॒वो अ॒ग्निः स्वि॑ष्ट॒कृत्।
सु॒द्रवि॑णा म॒न्द्रः क॒विः।
स॒त्यम॑न्माय॒जी होता᳚।
होतु॑र्‌\mbox{}होतु॒राय॑जीयान्।
अग्ने॒ यान्दे॒वानया᳚ट्।
याꣳ अपि॑प्रेः।
ये ते॑ हो॒त्रे अम॑थ्सत।
ताꣳ स॑स॒नुषी॒ꣳ॒ होत्रा᳚न्देवङ्ग॒माम्।
दि॒वि दे॒वेषु॑ य॒ज्ञमेर॑ये॒मम्।
स्वि॒ष्ट॒कृच्चाग्ने॒ होता\-ऽभूः᳚।
व॒सु॒वने॑ वसु॒धेय॑स्य नमोवा॒के वीहि॑॥३६॥\anuvakamend[वी॒तां॒ वे॒त्वभू॒रेकं॑ च]

%3.6.15.1
अ॒ग्निम॒द्य होता॑रमवृणीता॒यं यज॑मानः॒ पच॑न्प॒क्तीः पच॑न्पुरो॒डाशं॑ ब॒ध्नन्नि॑न्द्रा॒ग्निभ्यां॒ छागꣳ॑ सूप॒स्था अ॒द्य दे॒वो वन॒स्पति॑रभवदिन्द्रा॒ग्निभ्यां॒ छागे॒नाघ॑स्ता॒न्तं मे॑द॒स्तः प्रति॑\-पच॒ताग्र॑भीष्टा॒मवी॑वृधेतां पुरो॒डाशे॑न॒ त्वाम॒द्यर्\mbox{}ष॑ आर्\mbox{}षेय ऋषीणां नपादवृणीता॒यं यज॑मानो ब॒हुभ्य॒ आ सङ्ग॑तेभ्य ए॒ष मे॑ दे॒वेषु॒ वसु॒ वार्या य॑क्ष्यत॒ इति॒ ता या दे॒वा दे॑व॒दाना॒न्यदु॒स्तान्य॑स्मा॒ आ च॒ शास्वा च॑ गुरस्वेषि॒तश्च॑ होत॒रसि॑ भद्र॒वाच्या॑य॒ प्रेषि॑तो॒ मानु॑षः सूक्तवा॒काय॑ सू॒क्ता ब्रू॑हि॥३७॥\anuvakamend[अ॒ग्निम॒द्यैकम्᳚]






\prashnaend{अ॒ञ्जन्ति॒ होता॑ यक्ष॒थ्समि॑द्धो अ॒द्याग्निरजै॒द्दैव्या॑ जु॒षस्वा वृ॑त्रहणा गी॒र्भिस्त्वꣴ ह्याभ॑रत॒मुपो॑ह॒ यद्दे॒वं ब॒र्॒हिः सु॑दे॒वं दे॒वं ब॒र्॒हिर॒ग्निम॒द्य पञ्च॑दश॥१५॥}{अ॒ञ्जन्त्य॒ग्निर्\mbox{}होता॑ नो गी॒र्भिरुपो॑ ह॒ यद्वि॒दथं॑ वा॒जिनः॑ स॒प्तत्रिꣳ॑शत्॥३७॥}{अ॒ञ्जन्ति॑ सू॒क्ताब्रू॑हि॥}{हरिः॑ ओम्॥}{इति श्रीकृष्णयजुर्वेदीयतैत्तिरीयब्राह्मणे तृतीयाष्टके षष्ठः प्रपाठकः समाप्तः॥}
\clearpage
\sect{सप्तमः प्रश्नः}
\setcounter{anuvakam}{0}
\dnsub{तैत्तिरीयब्राह्मणे तृतीयाष्टके सप्तमः प्रपाठकः}

%3.7.1.1
सर्वा॒न्॒ वा ए॒षो᳚\-ऽग्नौ कामा॒न्प्रवे॑शयति।
यो᳚ऽग्नीन॑न्वा॒धाय॑ व्र॒तमु॒पैति॑।
सयदनि॑ष्ट्वा प्रया॒यात्।
अका॑मप्रीता एनं॒ कामा॒ नानु॒प्रया॑युः।
अ॒ते॒जा अ॑वी॒र्यः॑ स्यात्।
स जु॑हुयात्।
तुभ्यं॒ ता अ॑ङ्गिरस्तम।
विश्वाः᳚ सुक्षि॒तयः॒ पृथ॑क्।
अग्ने॒ कामा॑य येमिर॒ इति॑।
कामा॑ने॒वास्मि॑न्दधाति॥१॥

%3.7.1.2
काम॑प्रीता एनं॒ कामा॒ अनु॒ प्रया᳚न्ति।
ते॒ज॒स्वी वी॒र्या॑वान्भवति।
सन्त॑ति॒र्वा ए॒षा य॒ज्ञस्य॑।
यो᳚ऽग्नीन॑न्वा॒धाय॑ व्र॒तमु॒पैति॑।
स यदु॒द्वाय॑ति।
विच्छि॑त्तिरे॒वास्य॒ सा।
तं प्राञ्च॑मु॒द्धृत्य॑।
मन॒सोप॑तिष्ठेत।
मनो॒ वै प्र॒जा\-प॑तिः।
प्रा॒जा॒प॒त्यो य॒ज्ञः॥२॥

%3.7.1.3
मन॑सै॒व य॒ज्ञꣳ सं त॑नोति।
भूरित्या॑ह।
भू॒तो वै प्र॒जा\-प॑तिः।
भूति॑मे॒वोपै॑ति।
वि वा ए॒ष इ॑न्द्रि॒येण॑ वी॒र्ये॑णर्ध्यते।
यस्याऽऽहि॑ताग्नेर॒ग्निर॑प॒क्षाय॑ति।
याव॒च्छम्य॑या प्र॒विध्ये᳚त्।
यदि॒ ताव॑दप॒क्षाये᳚त्।
तꣳ सम्भ॑रेत्।
इ॒दं त॒ एकं॑ प॒र उ॑ त॒ एकम्᳚॥३॥

%3.7.1.4
तृ॒तीये॑न॒ ज्योति॑षा॒ संवि॑शस्व।
सं॒वेश॑नस्त॒नुवै॒ चारु॑रेधि।
प्रि॒ये दे॒वानां᳚ पर॒मे ज॒नित्र॒ इति॑।
ब्रह्म॑णै॒वैन॒ꣳ॒ सम्भ॑रति।
सैव ततः॒ प्राय॑श्चित्तिः।
यदि॑ परस्त॒राम॑प॒क्षाये᳚त्।
अ॒नु॒प्र॒यायाव॑स्येत्।
सो ए॒व ततः॒ प्राय॑श्चित्तिः।
ओष॑धी॒र्वा ए॒तस्य॑ प॒शून्पयः॒ प्रवि॑शति।
यस्य॑ ह॒विषे॑ व॒थ्सा अ॒पाकृ॑ता॒ धय॑न्ति॥४॥

%3.7.1.5
तान् यद्दु॒ह्यात्।
या॒तया᳚म्ना ह॒विषा॑ यजेत।
यन्न दु॒ह्यात्।
य॒ज्ञ॒प॒रुर॒न्तरि॑यात्।
वा॒य॒व्यां᳚ यवा॒गूं निर्व॑पेत्।
वा॒युर्वै पय॑सः प्रदापयि॒ता।
स ए॒वास्मै॒ पयः॒ प्रदा॑पयति।
पयो॒ वा ओष॑धयः।
पयः॒ पयः॑।
पय॑सै॒वास्मै॒ पयो\-ऽव॑ रुन्धे॥५॥

%3.7.1.6
अथोत्त॑रस्मै ह॒विषे॑ व॒थ्सान॒पाकु॑र्यात्।
सैव ततः॒ प्राय॑श्चित्तिः।
अ॒न्य॒त॒रान् वा ए॒ष दे॒वान्भा॑ग॒धेये॑न॒ व्य॑र्धयति।
ये यज॑मानस्य सा॒यं गृ॒हमा॒ गच्छ॑न्ति।
यस्य॑ सायं दु॒ग्धꣳ ह॒विरार्ति॑मा॒र्च्छति॑।
इन्द्रा॑य व्री॒हीन्नि॒रुप्योप॑ वसेत्।
पयो॒ वा ओष॑धयः।
पय॑ ए॒वाऽऽरभ्य॑ गृही॒त्वोप॑ वसति।
यत्प्रा॒तः स्यात्।
तच्छृ॒तं कु॑र्यात्॥६॥

%3.7.1.7
अथेत॑र ऐ॒न्द्रः पु॑रो॒डाशः॑ स्यात्।
इ॒न्द्रि॒ये ए॒वास्मै॑ स॒मीची॑ दधाति।
पयो॒ वा ओष॑धयः।
पयः॒ पयः॑।
पय॑सै॒वास्मै॒ पयो\-ऽव॑ रुन्धे।
अथोत्त॑रस्मै ह॒विषे॑ व॒थ्सान॒पाकु॑र्यात्।
सैव ततः॒ प्राय॑श्चित्तिः।
उ॒भया॒न्॒ वा ए॒ष दे॒वान्भा॑ग॒धेये॑न॒ व्य॑र्धयति।
ये यज॑मानस्य सा॒यं च॑ प्रा॒तश्च॑ गृ॒हमा॒ गच्छ॑न्ति।
यस्यो॒भयꣳ॑ ह॒विरार्ति॑मा॒र्च्छति॑॥७॥

%3.7.1.8
ऐ॒न्द्रं पञ्च॑शरावमोद॒नं निर्व॑पेत्।
अ॒ग्निं दे॒वता॑नां प्रथ॒मं य॑जेत्।
अ॒ग्निमु॑खा ए॒व दे॒वताः᳚ प्रीणाति।
अ॒ग्निं वा अन्व॒न्या दे॒वताः᳚।
इन्द्र॒मन्व॒न्याः।
ता ए॒वोभयीः᳚ प्रीणाति।
पयो॒ वा ओष॑धयः।
पयः॒ पयः॑।
पय॑सै॒वास्मै॒ पयो\-ऽव॑ रुन्धे।
अथोत्त॑रस्मै ह॒विषे॑ व॒थ्सान॒पाकु॑र्यात्॥८॥

%3.7.1.9
सैव ततः॒ प्राय॑श्चित्तिः।
अ॒र्धो वा ए॒तस्य॑ य॒ज्ञस्य॑ मीयते।
यस्य॒ व्रत्ये\-ऽह॒न्पत्न्य॑नालम्भु॒का भव॑ति।
ताम॑प॒रुध्य॑ यजेत।
सर्वे॑णै॒व य॒ज्ञेन॑ यजते।
तामि॒ष्ट्वोप॑ ह्वयेत।
अमू॒हम॑स्मि।
सा त्वम्।
द्यौर॒हम्।
पृ॒थि॒वी त्वम्।
सामा॒हम्।
ऋक्त्वम्।
तावेहि॒ सम्भ॑वाव।
स॒ह रेतो॑ दधावहै।
पु॒ꣳ॒से पु॒त्राय॒ वेत्त॑वै।
रा॒यस्पोषा॑य सुप्रजा॒स्त्वाय॑ सु॒वीर्या॒येति॑।
अ॒र्ध ए॒वैना॒मुप॑ ह्वयते।
सैव ततः॒ प्राय॑श्चित्तिः॥९॥\anuvakamend[द॒धा॒ति॒ य॒ज्ञ उ॑त॒ एक॒न्धय॑न्ति रुन्धे कुर्यादा॒र्च्छत्य॒पाकु॑र्यात्पृथि॒वी त्वम॒ष्टौ च॑ (सर्वा॒न्॒ वि वै यदि॑ परस्त॒रामोष॑धीरन्यत॒रानु॒भया॑न॒र्धो वै॥)]

%3.7.2.1
यद्विष्ष॑ण्णेन जुहु॒यात्।
अप्र॑जा अप॒शुर्यज॑मानः स्यात्।
यदना॑यतने नि॒नये᳚त्।
अ॒ना॒य॒त॒नः स्या᳚त्।
प्रा॒जा॒प॒त्यय॒र्चा व॑ल्मीकव॒पाया॒मव॑ नयेत्।
प्रा॒जा॒प॒त्यो वै व॒ल्मीकः॑।
य॒ज्ञः प्र॒जा\-प॑तिः।
प्र॒जा\-प॑तावे॒व य॒ज्ञं प्रति॑\-ष्ठापयति।
भूरित्या॑ह।
भू॒तो वै प्र॒जा\-प॑तिः॥१०॥

%3.7.2.2
भूति॑मे॒वोपै॑ति।
तत्कृ॒त्वा।
अ॒न्यां दु॒ग्ध्वा पुन॑र्‌\mbox{}होत॒व्यम्᳚।
सैव ततः॒ प्राय॑श्चित्तिः।
यत्की॒टाव॑पन्नेन जुहु॒यात्।
अप्र॑जा अप॒शुर्यज॑मानः स्यात्।
यदना॑यतने नि॒नये᳚त्।
अ॒ना॒य॒त॒नः स्या᳚त्।
म॒ध्य॒मेन॑ प॒र्णेन॑ द्यावापृथि॒व्य॑य॒र्चा\-ऽन्तः॑ परि॒धि निन॑येत्।
द्यावा॑पृथि॒व्योरे॒वैन॒त्प्रति॑\-ष्ठापयति॥११॥

%3.7.2.3
तत्कृ॒त्वा।
अ॒न्यां दु॒ग्ध्वा पुन॑र्\mbox{}होत॒व्यम्᳚।
सैव ततः॒ प्राय॑श्चित्तिः।
यदव॑वृष्टेन जुहु॒यात्।
अप॑रूपमस्या॒ऽ॒ऽ॒त्मञ्जा॑येत।
कि॒लासो॑ वा॒स्याद॑र्\mbox{}श॒सो वा᳚।
यत्प्रत्ये॒यात्।
य॒ज्ञं वि\-च्छि॑न्द्यात्।
स जु॑हुयात्।
मि॒त्रो जना᳚न्कल्पयति प्रजा॒नन्॥१२॥

%3.7.2.4
मि॒त्रो दा॑धार पृथि॒वीमु॒त द्याम्।
मि॒त्रः कृ॒ष्टीरनि॑मिषा॒ऽभि च॑ष्टे।
स॒त्याय॑ ह॒व्यं घृ॒तव॑ज्जुहो॒तेति॑।
मि॒त्रेणै॒वैन॑त्कल्पयति।
तत्कृ॒त्वा।
अ॒न्यां दु॒ग्ध्वा पुन॑र्\mbox{}होत॒व्यम्᳚।
सैव ततः॒ प्राय॑श्चित्तिः।
यत्पूर्व॑स्या॒माहु॑त्याꣳ हु॒ताया॒मुत्त॒रा\-ऽऽहु॑तिः॒ स्कन्दे᳚त्।
द्वि॒पाद्भिः॑ प॒शुभि॒र्यज॑मानो॒ व्यृ॑ध्येत।
यदुत्त॑रया॒ऽभि जु॑हु॒यात्॥१३॥

%3.7.2.5
चतु॑ष्पाद्भिः प॒शुभि॒र्यज॑मानो॒ व्यृ॑ध्येत।
यत्र॒ वेत्थ॑ वनस्पते दे॒वानां॒ गुह्या॒ नामा॑नि।
तत्र॑ ह॒व्यानि॑ गाम॒येति॑ वानस्प॒त्यय॒र्चा स॒मिध॑मा॒धाय॑।
तू॒ष्णीमे॒व पुन॑र्जुहुयात्।
वन॒स्पति॑नै॒व य॒ज्ञस्यार्तां॒ चाना᳚र्तां॒ चाऽऽहु॑ती॒ वि दा॑धार।
तत्कृ॒त्वा।
अ॒न्यां दु॒ग्ध्वा पुन॑र्\mbox{}होत॒व्यम्᳚।
सैव ततः॒ प्राय॑श्चित्तिः।
यत्पु॒रा प्र॑या॒जेभ्यः॒ प्राङङ्गा॑रः॒ स्कन्दे᳚त्।
अ॒ध्व॒र्यवे॑ च॒ यज॑मानाय॒ चाकꣴ॑ स्यात्॥१४॥

%3.7.2.6
यद्द॑क्षि॒णा।
ब्र॒ह्मणे॑ च॒ यज॑मानाय॒ चाकꣴ॑ स्यात्।
यत्प्र॒त्यक्।
होत्रे॑ च॒ पत्नि॑यै च॒ यज॑मानाय॒ चाकꣴ॑ स्यात्।
यदुदङ्ङ्॑।
अ॒ग्नीधे॑ च प॒शुभ्य॑श्च॒ यज॑मानाय॒ चाकꣴ॑ स्यात्।
यद॑भिजुहु॒यात्।
रु॒द्रो᳚ऽस्य प॒शून्घातु॑कः स्यात्।
यन्नाभि॑जुहु॒यात्।
अशा᳚न्तः॒ प्रह्रि॑येत॥१५॥

%3.7.2.7
स्रु॒वस्य॒ बुध्ने॑नाभि॒निद॑ध्यात्।
मा त॑मो॒ मा य॒ज्ञस्त॑म॒न्मा यज॑मानस्तमत्।
नम॑स्ते अस्त्वाय॒ते।
नमो॑ रुद्र पराय॒ते।
नमो॒ यत्र॑ नि॒षीद॑सि।
अ॒मुं मा हिꣳ॑सीर॒मुं मा हिꣳ॑सी॒रिति॒ येन॒ स्कन्दे᳚त्।
तं प्रह॑रेत्।
स॒हस्र॑शृङ्गो वृष॒भो जा॒तवे॑दाः।
स्तोम॑पृष्ठो घृ॒तवा᳚न्थ्सु॒प्रती॑कः।
मा नो॑ हासीन्मेत्थि॒तो नेत्त्वा॒ जहा॑म।
गो॒पो॒षं नो॑ वीरपो॒षं च॑ य॒च्छेति॑।
ब्रह्म॑णै॒वैनं॒ प्र ह॑रति।
सैव ततः॒ प्राय॑श्चित्तिः॥१६॥\anuvakamend[वै प्र॒जा\-प॑तिः स्थापयति प्रजा॒नन्न॒भि जु॑हु॒याथ्स्या᳚द्ध्रियेत॒ जहा॑म॒ त्रीणि॑ च (यद्विष्ष॑ण्णेन प्राजाप॒त्यया॒ यत्की॒टा म॑ध्य॒मेन॒ यदव॑वृष्टेन॒ यत्पूर्व॑स्यां॒ यत्पु॒रा प्र॑या॒जेभ्यः॒ प्राङङ्गा॑रो॒ यद्द॑क्षि॒णा यत्प्र॒त्यग्यदुदङ्ङ्॑॥)]

%3.7.3.1
वि वा ए॒ष इ॑न्द्रि॒येण॑ वी॒र्ये॑णर्ध्यते।
यस्याऽऽहि॑ताग्ने\-र॒ग्निर्म॒थ्य\-मा॑नो॒ न जाय॑ते।
यत्रा॒न्यं पश्ये᳚त्।
तत॑ आ॒हृत्य॑ होत॒व्यम्᳚।
अ॒ग्नावे॒वास्या᳚ग्निहो॒त्रꣳ हु॒तं भ॑वति।
यद्य॒न्यन्न वि॒न्देत्।
अ॒जायाꣳ॑ होत॒व्यम्᳚।
आ॒ग्ने॒यी वा ए॒षा।
यद॒जा।
अ॒ग्नावे॒वास्या᳚ग्निहो॒त्रꣳ हु॒तं भ॑वति॥१७॥

%3.7.3.2
अ॒जस्य॒ तु नाश्ञी॑यात्।
यद॒जस्या᳚श्ञी॒यात्।
यामे॒वाग्नावाहु॑तिं जुहु॒यात्।
ताम॑द्यात्।
तस्मा॑द॒जस्य॒ नाश्यम्᳚।
यद्य॒जान्न वि॒न्देत्।
ब्रा॒ह्म॒णस्य॒ दक्षि॑णे॒ हस्ते॑ होत॒व्यम्᳚।
ए॒ष वा अ॒ग्निर्वै᳚श्वान॒रः।
यद्ब्रा᳚ह्म॒णः।
अ॒ग्नावे॒वास्या᳚ग्निहो॒त्रꣳ हु॒तं भ॑वति॥१८॥

%3.7.3.3
ब्रा॒ह्म॒णं तु व॑स॒त्यै॑ नाप॑ रुन्ध्यात्।
यद्ब्रा᳚ह्म॒णं व॑स॒त्या अ॑परु॒न्ध्यात्।
यस्मि॑न्ने॒वाग्नावाहु॑तिं जुहु॒यात्।
तं भा॑ग॒धेये॑न॒ व्य॑र्धयेत्।
तस्मा᳚द्ब्राह्म॒णो व॑स॒त्यै॑ नाप॒रुध्यः॑।
यदि॑ ब्राह्म॒णं न वि॒न्देत्।
द॒र्भ॒स्त॒म्बे हो॑त॒व्यम्᳚।
अ॒ग्नि॒वान् वै द॑र्भस्त॒म्बः।
अ॒ग्नावे॒वास्या᳚ग्निहो॒त्रꣳ हु॒तं भ॑वति।
द॒र्भाꣴस्तु नाध्या॑सीत॥१९॥

%3.7.3.4
यद्द॒र्भान॒ध्यासी॑त।
यामे॒वाग्नावाहु॑तिं जुहु॒यात्।
तामध्या॑सीत।
तस्मा᳚द्द॒र्भा नाध्या॑सित॒व्याः᳚।
यदि॑ द॒र्भान्न वि॒न्देत्।
अ॒फ्सु हो॑त॒व्यम्᳚।
आपो॒ वै सर्वा॑ दे॒वताः᳚।
दे॒वता᳚स्वे॒वास्या᳚ग्निहो॒त्रꣳ हु॒तं भ॑वति।
आप॒स्तु न परि॑चक्षीत।
यदापः॑ परि॒चक्षी॑त॥२०॥

%3.7.3.5
यामे॒वाफ्स्वाहु॑तिं जुहु॒यात्।
तां परि॑चक्षीत।
तस्मा॒दापो॒ न प॑रि॒चक्ष्याः᳚।
मेध्या॑ च॒ वा ए॒तस्या॑मे॒ध्या च॑ त॒नुवौ॒ सꣳ सृ॑ज्येते।
यस्याऽऽहि॑ताग्नेर॒न्यैर॒ग्निभि॑र॒ग्नयः॑ सꣳसृ॒ज्यन्ते᳚।
अ॒ग्नये॒ विवि॑चये पुरो॒डाश॑म॒ष्टा\-क॑पालं॒ निर्व॑पेत्।
मेध्यां चै॒वास्या॑मे॒ध्यां च॑ त॒नुवौ॒ व्याव॑र्तयति।
अ॒ग्नये᳚ व्र॒तप॑तये पुरो॒डाश॑म॒ष्टा\-क॑पालं॒ निर्व॑पेत्।
अ॒ग्निमे॒व व्र॒तप॑ति॒ꣴ॒ स्वेन॑ भाग॒धेये॒नोप॑ धावति।
स ए॒वैनं॑ व्र॒तमा ल॑म्भयति॥२१॥

%3.7.3.6
गर्भ॒ꣴ॒ स्रव॑न्तमग॒दम॑कः।
अ॒ग्निरिन्द्र॒स्त्वष्टा॒ बृह॒स्पतिः॑।
पृ॒थि॒व्यामव॑ चुश्चोतै॒तत्।
नाभिप्राप्नो॑ति॒ निर्‌\mbox{}ऋ॑तिं परा॒चैः।
रेतो॒ वा ए॒तद्वाजि॑न॒माहि॑ताग्नेः।
यद॑ग्निहो॒त्रम्।
तद्यथ्स्रवे᳚त्।
रेतो᳚\-ऽस्य॒ वाजि॑नꣴ स्रवेत्।
गर्भ॒ꣴ॒ स्रव॑न्तमग॒दम॑क॒रित्या॑ह।
रेत॑ ए॒वास्मि॒न्वाजि॑नं दधाति॥२२॥

%3.7.3.7
अ॒ग्निरित्या॑ह।
अ॒ग्निर्वै रे॑तो॒धाः।
रेत॑ ए॒व तद्द॑धाति।
इन्द्र॒ इत्या॑ह।
इ॒न्द्रि॒यमे॒वास्मि॑न्दधाति।
त्वष्टेत्या॑ह।
त्वष्टा॒ वै प॑शू॒नां मि॑थु॒नानाꣳ॑ रूप॒कृत्।
रू॒पमे॒व प॒शुषु॑ दधाति।
बृह॒स्पति॒रित्या॑ह।
ब्रह्म॒ वै दे॒वानां॒ बृह॒स्पतिः॑।
ब्रह्म॑णै॒वास्मै᳚ प्र॒जाः प्र ज॑नयति।
पृ॒थि॒व्यामव॑ चुश्चोतै॒तदित्या॑ह।
अ॒स्यामे॒वैन॒त्प्रति॑\-ष्ठापयति।
नाभिप्राप्नो॑ति॒ निर्‌\mbox{}ऋ॑तिं परा॒चैरित्या॑ह।
रक्ष॑सा॒मप॑हत्यै॥२३॥\anuvakamend[अ॒जा\-ऽग्नावे॒वास्या᳚ग्निहो॒त्रꣳ हु॒तं भ॑वति भवत्यासीत परि॒चक्षी॑त लम्भयति दधाति दे॒वानां॒ बृह॒स्पतिः॒ पञ्च॑ च  (वि वै यद्य॒न्यम॒जायां᳚ ब्राह्म॒णस्य॑ दर्भस्त॒म्बे᳚\-ऽफ्सु हो॑त॒व्यम्᳚।)]

%3.7.4.1
याः पु॒रस्ता᳚त्प्र॒स्रव॑न्ति।
उ॒परि॑ष्टाथ्स॒र्वत॑श्च॒ याः।
ताभी॑ र॒श्मिप॑वित्राभिः।
श्र॒द्धां य॒ज्ञमा र॑भे।
देवा॑ गातुविदः।
गा॒तुं य॒ज्ञाय॑ विन्दत।
मन॑स॒स्पति॑ना दे॒वेन॑।
वाता᳚द्य॒ज्ञः प्र यु॑ज्यताम्।
तृ॒तीय॑स्यै दि॒वः।
गा॒य॒त्रि॒या सोम॒ आभृ॑तः॥२४॥

%3.7.4.2
सो॒म॒पी॒थाय॒ सन्न॑यितुम्।
वक॑ल॒मन्त॑र॒मा द॑दे।
आपो॑ देवीः शु॒द्धाः स्थ॑।
इ॒मा पात्रा॑णि शुन्धत।
उ॒पा॒त॒ङ्क्या॑य दे॒वाना᳚म्।
प॒र्ण॒व॒ल्कमु॒त शु॑न्धत।
पयो॑ गृ॒हेषु॒ पयो॑ अघ्नि॒यासु॑।
पयो॑ व॒थ्सेषु॒ पय॒ इन्द्रा॑य ह॒विषे᳚ ध्रियस्व।
गा॒य॒त्री प॑र्णव॒ल्केन॑।
पयः॒ सोमं॑ करोत्वि॒मम्॥२५॥

%3.7.4.3
अ॒ग्निं गृ॑ह्णामि सु॒रथं॒ यो म॑यो॒भूः।
य उ॒द्यन्त॑मा॒रोह॑ति॒ सूर्य॒मह्ने᳚।
आ॒दि॒त्यं ज्योति॑षां॒ ज्योति॑रुत्त॒मम्।
श्वो य॒ज्ञाय॑ रमतां दे॒वता᳚भ्यः।
वसू᳚न्रु॒द्राना॑दि॒त्यान्।
इन्द्रे॑ण स॒ह दे॒वताः᳚।
ताः पूर्वः॒ परि॑ गृह्णामि।
स्व आ॒यत॑ने मनी॒षया᳚।
इ॒मामूर्जं॑ पञ्चद॒शीं ये प्रवि॑ष्टाः।
तान्दे॒वान्परि॑ गृह्णामि॒ पूर्वः॑॥२६॥

%3.7.4.4
अ॒ग्निर्‌\mbox{}ह॑व्य॒वाडि॒ह ताना व॑हतु।
पौ॒र्ण॒मा॒सꣳ ह॒विरि॒दमे॑षां॒ मयि॑।
आ॒मा॒वा॒स्यꣳ॑ ह॒विरि॒दमे॑षां॒ मयि॑।
अ॒न्त॒रा\-ऽग्नी प॒शवः॑।
दे॒व॒स॒ꣳ॒सद॒मा ग॑मन्।
तान्पूर्वः॒ परि॑ गृह्णामि।
स्व आ॒यत॑ने मनी॒षया᳚।
इ॒ह प्र॒जा वि॒श्वरू॑पा रमन्ताम्।
अ॒ग्निं गृ॒हप॑तिम॒भि सं॒वसा॑नाः।
ताः पूर्वः॒ परि॑ गृह्णामि॥२७॥

%3.7.4.5
स्व आ॒यत॑ने मनी॒षया᳚।
इ॒ह प॒शवो॑ वि॒श्वरू॑पा रमन्ताम्।
अ॒ग्निं गृ॒हप॑तिम॒भि सं॒वसा॑नाः।
तान्पूर्वः॒ परि॑ गृह्णामि।
स्व आ॒यत॑ने मनी॒षया᳚।
अ॒यं पि॑तृ॒णाम॒ग्निः।
अवा᳚ड्ढ॒व्या पि॒तृभ्य॒ आ।
तं पूर्वः॒ परि॑ गृह्णामि।
अवि॑षन्नः पि॒तुं क॑रत्।
अज॑स्रं॒ त्वाꣳ स॑भापा॒लाः॥२८॥

%3.7.4.6
वि॒ज॒यभा॑ग॒ꣳ॒ समि॑न्धताम्।
अग्ने॑ दी॒दा॑य मे सभ्य।
विजि॑त्यै श॒रदः॑ श॒तम्।
अन्न॑मावस॒थीयम्᳚।
अ॒भि ह॑राणि श॒रदः॑ श॒तम्।
आ॒व॒स॒थे श्रियं॒ मन्त्रम्᳚।
अहि॑र्बु॒ध्नियो॒ नि य॑च्छतु।
इ॒द\-म॒ह\-म॒ग्नि\-ज्ये᳚ष्ठेभ्यः।
वसु॑भ्यो य॒ज्ञं प्रब्र॑वीमि।
इ॒द\-म॒ह\-मिन्द्र॑\-ज्येष्ठेभ्यः॥२९॥

%3.7.4.7
रु॒द्रेभ्यो॑ य॒ज्ञं प्र ब्र॑वीमि।
इ॒दम॒हं वरु॑णज्येष्ठेभ्यः।
आ॒दि॒त्येभ्यो॑ य॒ज्ञं प्र ब्र॑वीमि।
पय॑स्वती॒रोष॑धयः।
पय॑स्वद्वी॒रुधां॒ पयः॑।
अ॒पां पय॑सो॒ यत्पयः॑।
तेन॒ मामि॑न्द्र॒ सꣳ सृ॑ज।
अग्ने᳚ व्रतपते व्र॒तं च॑रिष्यामि।
तच्छ॑केयं॒ तन्मे॑ राध्यताम्।
वायो᳚ व्रतपत॒ आदि॑त्य व्रतपते॥३०॥

%3.7.4.8
व्र॒तानां᳚ व्रतपते व्र॒तं च॑रिष्यामि।
तच्छ॑केयं॒ तन्मे॑ राध्यताम्।
इ॒मां प्राची॒मुदी॑चीम्।
इष॒मूर्ज॑म॒भि सꣴस्कृ॑ताम्।
ब॒हु॒प॒र्णामशु॑ष्काग्राम्।
हरा॑मि पशु॒पाम॒हम्।
यत्कृष्णो॑ रू॒पं कृ॒त्वा।
प्रावि॑श॒स्त्वं वन॒स्पतीन्॑।
तत॒स्त्वामे॑कविꣳशति॒धा।
सम्भ॑रामि सुस॒म्भृता᳚॥३१॥

%3.7.4.9
त्रीन्प॑रि॒धीꣴस्ति॒स्रः स॒मिधः॑।
य॒ज्ञायु॑रनु\-सञ्च॒रान्।
उ॒प॒वे॒षं मेक्ष॑णं॒ धृष्टिम्᳚।
सं भ॑रामि सुस॒म्भृता᳚।
या जा॒ता ओष॑धयः।
दे॒वेभ्य॑स्त्रियु॒गं पु॒रा।
तासां॒ पर्व॑ राध्यासम्।
प॒रि॒स्त॒रमा॒हरन्॑।
अ॒पां मेध्यं॑ य॒ज्ञियम्᳚।
सदे॑वꣳ शि॒वम॑स्तु मे॥३२॥

%3.7.4.10
आ॒च्छे॒त्ता वो॒ मा रि॑षम्।
जीवा॑नि श॒रदः॑ श॒तम्।
अप॑रिमितानां॒ परि॑मिताः।
सन्न॑ह्ये सुकृ॒ताय॒ कम्।
एनो॒ मा निगा᳚ङ्कत॒मच्च॒नाहम्।
पुन॑रु॒त्थाय॑ बहु॒ला भ॑वन्तु।
स॒कृ॒दा॒च्छि॒न्नं ब॒र्॒हिरूर्णा॑मृदु।
स्यो॒नं पि॒तृभ्य॑स्त्वा भराम्य॒हम्।
अ॒स्मिन्थ्सी॑दन्तु मे पि॒तरः॑ सो॒म्याः।
पि॒ता॒म॒हाः प्रपि॑तामहाश्चानु॒गैः स॒ह॥३३॥

%3.7.4.11
त्रि॒वृत्प॑ला॒शे द॒र्भः।
इया᳚न्प्रादे॒शस॑म्मितः।
य॒ज्ञे प॒वित्रं॒ पोतृ॑तमम्।
पयो॑ ह॒व्यं क॑रोतु मे।
इ॒मौ प्रा॑णापा॒नौ।
य॒ज्ञस्याङ्गा॑नि सर्व॒शः।
आ॒प्या॒यय॑न्तौ॒ सञ्च॑रताम्।
प॒वित्रे॑ हव्य॒शोध॑ने।
प॒वित्रे᳚ स्थो वैष्ण॒वी।
वा॒युर्वां॒ मन॑सा पुनातु॥३४॥

%3.7.4.12
अ॒यं प्रा॒णश्चा॑पा॒नश्च॑।
यज॑मान॒मपि॑ गच्छताम्।
य॒ज्ञे ह्यभू॑तां॒ पोता॑रौ।
प॒वित्रे॑ हव्य॒शोध॑ने।
त्वया॒ वेदिं॑ विविदुः पृथि॒वीम्।
त्वया॑ य॒ज्ञो जा॑यते विश्व॒दानिः॑।
अच्छि॑द्रं य॒ज्ञमन्वे॑षि वि॒द्वान्।
त्वया॒ होता॒ सं त॑नोत्यर्धमा॒सान्।
त्र॒य॒स्त्रि॒ꣳ॒शो॑ऽसि॒ तन्तू॑नाम्।
प॒वित्रे॑ण स॒हाग॑हि॥३५॥

%3.7.4.13
शि॒वेयꣳ रज्जु॑रभि॒धानी᳚।
अ॒घ्नि॒यामुप॑ सेवताम्।
अप्र॑स्रꣳसाय य॒ज्ञस्य॑।
उ॒खे उप॑दधाम्य॒हम्।
प॒शुभिः॒ सन्नी॑तं बिभृताम्।
इन्द्रा॑य शृ॒तं दधि॑।
उ॒प॒वे॒षो॑ऽसि य॒ज्ञाय॑।
त्वां प॑रिवे॒षम॑धारयन्।
इन्द्रा॑य ह॒विः कृ॒ण्वन्तः॑।
शि॒वः श॒ग्मो भ॑वासि नः॥३६॥

%3.7.4.14
अमृ॑न्मयन्देवपा॒त्रम्।
य॒ज्ञस्याऽऽयु॑षि॒ प्र यु॑ज्यताम्।
ति॒रः॒ प॒वि॒त्रमति॑नीताः।
आपो॑ धारय॒ माति॑गुः।
दे॒वेन॑ सवि॒त्रोत्पू॑ताः।
वसोः॒ सूर्य॑स्य र॒श्मिभिः॑।
गां दो॑हपवि॒त्रे रज्जुम्᳚।
सर्वा॒ पात्रा॑णि शुन्धत।
ए॒ता आ च॑रन्ति॒ मधु॑म॒द्दुहा॑नाः।
प्र॒जाव॑तीर्य॒शसो॑ वि॒श्वरू॑पाः॥३७॥

%3.7.4.15
ब॒ह्वीर्भव॑न्ती॒रुप॒जाय॑मानाः।
इ॒ह व॒ इन्द्रो॑ रमयतु गावः।
पू॒षा स्थ॑।
अ॒य॒क्ष्मा वः॑ प्र॒जया॒ सꣳ सृ॑जामि।
रा॒यस्पोषे॑ण बहु॒लाभव॑न्तीः।
ऊर्जं॒ पयः॒ पिन्व॑माना घृ॒तं च॑।
जी॒वो जीव॑न्ती॒रुप॑वः सदेयम्।
द्यौश्चे॒मं य॒ज्ञं पृ॑थि॒वी च॒ सन्दु॑हाताम्।
धा॒ता सोमे॑न स॒ह वाते॑न वा॒युः।
यज॑मानाय॒ द्रवि॑णं दधातु॥३८॥

%3.7.4.16
उथ्सं॑ दुहन्ति क॒लशं॒ चतु॑र्बिलम्।
इडां᳚ दे॒वीं मधु॑मतीꣳ सुव॒र्विदम्᳚।
तदि॑न्द्रा॒ग्नी जि॑न्वतꣳ सू॒नृता॑वत्।
तद्यज॑मान\-ममृत॒त्वे द॑धातु।
काम॑धुक्षः॒ प्र णो᳚ ब्रूहि।
इन्द्रा॑य ह॒विरि॑न्द्रि॒यम्।
अ॒मूं यस्यां᳚ दे॒वाना᳚म्।
म॒नु॒ष्या॑णां॒ पयो॑ हि॒तम्।
ब॒हु दु॒ग्धीन्द्रा॑य दे॒वेभ्यः॑।
ह॒व्यमा प्या॑यतां॒ पुनः॑॥३९॥

%3.7.4.17
व॒थ्सेभ्यो॑ मनु॒ष्ये᳚भ्यः।
पु॒न॒र्दो॒हाय॑ कल्पताम्।
य॒ज्ञस्य॒ सन्त॑\-ति\-रसि।
य॒ज्ञस्य॑ त्वा॒ सन्त॑\-ति॒\-मनु॒ सं त॑नोमि।
अद॑स्तमसि॒ विष्ण॑वे त्वा।
य॒ज्ञायापि॑ दधाम्य॒हम्।
अ॒द्भिररि॑क्तेन॒ पात्रे॑ण।
याः पू॒ताः प॑रि॒शेर॑ते।
अ॒यं पयः॒ सोमं॑ कृ॒त्वा।
स्वां योनि॒मपि॑ गच्छतु॥४०॥

%3.7.4.18
प॒र्ण॒व॒ल्कः प॒वित्रम्᳚।
सौ॒म्यः सोमा॒द्धि निर्मि॑तः।
इ॒मौ प॒र्णं च॑ द॒र्भं च॑।
दे॒वानाꣳ॑ हव्य॒शोध॑नौ।
प्रा॒त॒र्वे॒षाय॑ गोपाय।
विष्णो॑ ह॒व्यꣳ हि रक्ष॑सि।
उ॒भाव॒ग्नी उ॑पस्तृण॒ते।
दे॒वता॒ उप॑वसन्तु मे।
अ॒हं ग्रा॒म्यानुप॑ वसामि।
मह्यं॒ गोप॑तये प॒शून्॥४१॥\anuvakamend[आभृ॑त इ॒मं गृ॑ह्णामि॒ पूर्व॒स्ताः पूर्वः॒ परि॑गृह्णामि सभापा॒ला इन्द्र॑ज्येष्ठेभ्य॒ आदि॑त्य व्रतपते सुस॒म्भृता॑ मे स॒ह पु॑नातु गहि नो वि॒श्वरू॑पा दधातु॒ पुन॑र्गच्छतु प॒शून् (याः पु॒रस्ता॑दि॒मामूर्ज॑मि॒ह प्र॒जा इ॒ह प॒शवो॒ऽयं पि॑तृ॒णाम॒ग्निः।)]

%3.7.5.1
देवा॑ दे॒वेषु॒ परा᳚क्रमध्वम्।
प्रथ॑मा द्वि॒तीये॑षु।
द्विती॑यास्तृ॒तीये॑षु।
त्रिरे॑कादशा इ॒ह मा॑ऽवत।
इ॒दꣳ श॑केयं॒ यदि॒दं क॒रोमि॑।
आ॒त्मा क॑रोत्वा॒त्मने᳚।
इ॒दं क॑रिष्ये भेष॒जम्।
इ॒दं मे॑ विश्वभेषजा।
अश्वि॑ना॒ प्राव॑तं यु॒वम्।
इ॒दम॒हꣳ सेना॑या अ॒भीत्व॑र्यै॥४२॥

%3.7.5.2
मुख॒मपो॑हामि।
सूर्य॑ ज्योति॒र्वि भा॑हि।
म॒ह॒त इ॑न्द्रि॒याय॑।
आ प्या॑यतां घृ॒तयो॑निः।
अ॒ग्निर्‌\mbox{}ह॒व्याऽनु॑ मन्यताम्।
खम॑ङ्क्ष्व॒ त्वच॑मङ्क्ष्व।
सु॒रू॒पं त्वा॑ वसु॒विदम्᳚।
प॒शू॒नां तेज॑सा।
अ॒ग्नये॒ जुष्ट॑म॒भि घा॑रयामि।
स्यो॒नं ते॒ सद॑नं करोमि॥४३॥

%3.7.5.3
घृ॒तस्य॒ धार॑या सु॒शेवं॑ कल्पयामि।
तस्मि᳚न्थ्सीदा॒मृते॒ प्रति॑ तिष्ठ।
व्री॒ही॒णां मे॑ध सुमन॒स्यमा॑नः।
आ॒र्द्रः प्र॑थस्नु॒र्भुव॑नस्य गो॒पाः।
शृ॒त उथ्स्ना॑ति जनि॒ता म॑ती॒नाम्।
यस्त॑ आ॒त्मा प॒शुषु॒ प्रवि॑ष्टः।
दे॒वानां᳚ वि॒ष्ठामनु॒ यो वि॑त॒स्थे।
आ॒त्म॒न्वान्थ्सो॑म घृ॒तवा॒न्॒ हि भू॒त्वा।
दे॒वान्ग॑च्छ॒ सुव॑र्विन्द॒ यज॑मानाय॒ मह्यम्᳚।
इरा॒ भूतिः॑ पृथि॒व्यै रसो॒ मोत्क्र॑मीत्॥४४॥

%3.7.5.4
देवाः᳚ पितरः॒ पित॑रो देवाः।
यो॑ऽहम॑स्मि॒ स सन् य॑जे।
यस्या᳚स्मि॒ न तम॒न्तरे॑मि।
स्वं म॑ इ॒ष्टꣴ स्वं द॒त्तम्।
स्वं पू॒र्तꣴ स्वꣴ श्रा॒न्तम्।
स्वꣳ हु॒तम्।
तस्य॑ मे॒\-ऽग्निरु॑पद्र॒ष्टा।
वा॒युरु॑पश्रो॒ता।
आ॒दि॒त्यो॑\-ऽनुख्या॒ता।
द्यौः पि॒ता॥४५॥

%3.7.5.5
पृ॒थि॒वी मा॒ता।
प्र॒जा\-प॑ति॒र्बन्धुः॑।
य ए॒वास्मि॒ स सन् य॑जे।
मा भेर्मा संवि॑क्था॒ मा त्वा॑ हिꣳसिषम्।
मा ते॒ तेजोऽप॑ क्रमीत्।
भ॒र॒तमुद्ध॑रे॒मनु॑षिञ्च।
अ॒व॒दाना॑नि ते प्र॒त्यव॑दास्यामि।
नम॑स्ते अस्तु॒ मा मा॑ हिꣳसीः।
यद॑व॒दाना॑नि तेऽव॒द्यन्।
विलो॒माका॑र्‌\mbox{}षमा॒त्मनः॑॥४६॥

%3.7.5.6
आज्ये॑न॒ प्रत्य॑नज्म्येनत्।
तत्त॒ आ प्या॑यतां॒ पुनः॑।
अज्या॑यो यवमा॒त्रात्।
आ॒व्या॒धात्कृ॑त्यतामि॒दम्।
मा रू॑रुपाम य॒ज्ञस्य॑।
शु॒द्धꣴ स्वि॑ष्टमि॒दꣳ ह॒विः।
मनु॑ना दृ॒ष्टां घृ॒तप॑दीम्।
मि॒त्रावरु॑णसमीरिताम्।
द॒क्षि॒णा॒र्धादस॑म्भिन्दन्।
अव॑द्याम्ये\-क॒तोमु॑खाम्॥४७॥

%3.7.5.7
इडे॑ भा॒गं जु॑षस्व नः।
जिन्व॒ गा जिन्वार्व॑तः।
तस्या᳚स्ते भक्षि॒वाणः॑ स्याम।
स॒र्वात्मा॑नः स॒र्वग॑णाः।
ब्रध्न॒ पिन्व॑स्व।
दद॑तो मे॒ मा क्षा॑यि।
कु॒र्व॒तो मे॒ मोप॑दसत्।
दि॒शां कॢप्ति॑रसि।
दिशो॑ मे कल्पन्ताम्।
कल्प॑न्तां मे॒ दिशः॑॥४८॥

%3.7.5.8
दैवी᳚श्च॒ मानु॑षीश्च।
अ॒हो॒रा॒त्रे मे॑ कल्पेताम्।
अ॒र्ध॒मा॒सा मे॑ कल्पन्ताम्।
मासा॑ मे कल्पन्ताम्।
ऋ॒तवो॑ मे कल्पन्ताम्।
सं॒व॒थ्स॒रो मे॑ कल्पताम्।
कॢप्ति॑रसि॒ कल्प॑तां मे।
आशा॑नां त्वा\-ऽऽशापा॒लेभ्यः॑।
च॒तुर्भ्यो॑ अ॒मृते᳚भ्यः।
इ॒दं भू॒तस्याध्य॑क्षेभ्यः॥४९॥

%3.7.5.9
वि॒धेम॑ ह॒विषा॑ व॒यम्।
भज॑तां भा॒गी भा॒गम्।
मा भा॒गो\-ऽभ॑क्त।
निर॑भा॒गं भ॑जामः।
अ॒पस्पि॑न्व।
ओष॑धीर्जिन्व।
द्वि॒पात्पा॑हि।
चतु॑ष्पादव।
दि॒वो वृष्टि॒मेर॑य।
ब्रा॒ह्म॒णाना॑मि॒दꣳ ह॒विः॥५०॥

%3.7.5.10
सो॒म्यानाꣳ॑ सोमपी॒थिना᳚म्।
निर्भ॒क्तो ब्रा᳚ह्मणः।
नेहा ब्रा᳚ह्मणस्यास्ति।
सम॑ङ्क्तां ब॒र्॒हिर्‌\mbox{}ह॒विषा॑ घृ॒तेन॑।
समा॑दि॒त्यैर्वसु॑भिः॒ सं म॒रुद्भिः॑।
समिन्द्रे॑ण॒ विश्वे॑भिर्दे॒वेभि॑रङ्क्ताम्।
दि॒व्यं नभो॑ गच्छतु॒ यथ्स्वाहा᳚।
इ॒न्द्रा॒णीवा॑विध॒वा भू॑यासम्।
अदि॑तिरिव सुपु॒त्रा।
अ॒स्थू॒रि त्वा॑ गार्‌\mbox{}हपत्य॥५१॥

%3.7.5.11
उप॒निष॑दे सुप्रजा॒स्त्वाय॑।
सं पत्नी॒ पत्या॑ सुकृ॒तेन॑ गच्छताम्।
य॒ज्ञस्य॑ यु॒क्तौ धुर्या॑वभूताम्।
स॒ञ्जा॒ना॒नौ विज॑हता॒मरा॑तीः।
दि॒वि ज्योति॑र॒जर॒मा र॑भेताम्।
दश॑ते त॒नुवो॑ यज्ञ य॒ज्ञियाः᳚।
ताः प्री॑णातु॒ यज॑मानो घृ॒तेन॑।
ना॒रि॒ष्ठयोः᳚ प्र॒शिष॒मीड॑मानः।
दे॒वानां॒ दैव्येऽपि॒ यज॑मानो॒\-ऽमृतो॑\-ऽभूत्।
यं वां᳚ दे॒वा अ॑कल्पयन्॥५२॥

%3.7.5.12
ऊ॒र्जो भा॒गꣳ श॑तक्रतू।
ए॒तद्वां॒ तेन॑ प्रीणानि।
तेन॑ तृप्यतमꣳहहौ।
अ॒हं दे॒वानाꣳ॑ सु॒कृता॑मस्मि लो॒के।
ममे॒दमि॒ष्टं न मिथु॑र्भवाति।
अ॒हं ना॑रि॒ष्ठावनु॑ यजामि वि॒द्वान्।
यदा᳚भ्या॒मिन्द्रो॒ अद॑धाद्भाग॒धेयम्᳚।
अदा॑रसृद्भवत देवसोम।
अ॒स्मिन् य॒ज्ञे म॑रुतो मृडता नः।
मा नो॑ विदद॒भिभा॒मो अश॑स्तिः॥५३॥

%3.7.5.13
मा नो॑ विदद्वृ॒जना॒ द्वेष्या॒ या।
ऋ॒ष॒भं वा॒जिनं॑ व॒यम्।
पू॒र्णमा॑सं यजामहे।
स नो॑ दोहताꣳ सु॒वीर्यम्᳚।
रा॒यस्पोषꣳ॑ सह॒स्रिणम्᳚।
प्रा॒णाय॑ सु॒राध॑से।
पू॒र्णमा॑साय॒ स्वाहा᳚।
अ॒मा॒वा॒स्या॑ सु॒भगा॑ सु॒शेवा᳚।
धे॒नुरि॑व॒ भूय॑ आ॒प्याय॑माना।
सा नो॑ दोहताꣳ सु॒वीर्यम्᳚।
रा॒यस्पोषꣳ॑ सह॒स्रिणम्᳚।
अ॒पा॒नाय॑ सु॒राध॑से।
अ॒मा॒वा॒स्या॑यै॒ स्वाहा᳚।
अ॒भि स्तृ॑णीहि॒ परि॑ धेहि॒ वेदिम्᳚।
जा॒मिं मा हिꣳ॑सीरमु॒या शया॑ना।
हो॒तृ॒षद॑ना॒ हरि॑ताः सु॒वर्णाः᳚।
नि॒ष्का इ॒मे यज॑मानस्य ब्र॒ध्ने॥५४॥\anuvakamend[अ॒भीत्व॑र्यै करोमि क्रमीत्पि॒ता\-ऽऽत्मन॑ एक॒तो मु॑खां मे॒ दिशो\-ऽध्य॑क्षेभ्यो ह॒विर्गा॑र्‌\mbox{}हपत्या कल्पय॒न्नश॑स्तिः॒ सा नो॑ दोहताꣳ सु॒वीर्यꣳ॑ स॒प्त च॑]

%3.7.6.1
परि॑स्तृणीत॒ परि॑धत्ता॒ग्निम्।
परि॑हितो॒\-ऽग्निर्यज॑मानं भुनक्तु।
अ॒पाꣳ रस॒ ओष॑धीनाꣳ सु॒वर्णः॑।
नि॒ष्का इ॒मे यज॑मानस्य सन्तु काम॒दुघाः᳚।
अ॒मुत्रा॒मुष्मिँ॑ल्लो॒के।
भूप॑ते॒ भुव॑नपते।
म॒ह॒तो भू॒तस्य॑ पते।
ब्र॒ह्माणं॑ त्वा वृणीमहे।
अ॒हं भूप॑तिर॒हं भुव॑नपतिः।
अ॒हं म॑ह॒तो भू॒तस्य॒ पतिः॑॥५५॥

%3.7.6.2
दे॒वेन॑ सवि॒त्रा प्रसू॑त॒ आर्त्वि॑ज्यं करिष्यामि।
देव॑ सवितरे॒तं त्वा॑ वृणते।
बृह॒स्पतिं॒ दैव्यं॑ ब्र॒ह्माणम्᳚।
तद॒हं मन॑से॒ प्र ब्र॑वीमि।
मनो॑ गायत्रि॒यै।
गा॒य॒त्री त्रि॒ष्टुभे᳚।
त्रि॒ष्टुब्जग॑त्यै।
जग॑त्यनु॒ष्टुभे᳚।
अ॒नु॒ष्टुक्प॒ङ्क्त्यै।
प॒ङ्क्तिः प्र॒जा\-प॑तये॥५६॥

%3.7.6.3
प्र॒जा\-प॑ति॒र्विश्वे᳚भ्यो दे॒वेभ्यः॑।
विश्वे॑ दे॒वा बृह॒स्पत॑ये।
बृह॒स्पति॒र्ब्रह्म॑णे।
ब्रह्म॒ भूर्भुवः॒ सुवः॑।
बृह॒स्पति॑र्दे॒वानां᳚ ब्र॒ह्मा।
अ॒हं म॑नु॒ष्या॑णाम्।
बृह॑स्पते य॒ज्ञं गो॑पाय।
इ॒दं तस्मै॑ ह॒र्म्यं क॑रोमि।
यो वो॑ देवा॒श्चर॑ति ब्रह्म॒चर्यम्᳚।
मे॒धा॒वी दि॒क्षु मन॑सा तप॒स्वी॥५७॥

%3.7.6.4
अ॒न्तर्दू॒तश्च॑रति॒ मानु॑षीषु।
चतुः॑ शिखण्डा युव॒तिः सु॒पेशाः᳚।
घृ॒तप्र॑तीका॒ भुव॑नस्य॒ मध्ये᳚।
म॒र्मृ॒ज्यमा॑ना मह॒ते सौभ॑गाय।
मह्यं॑ धुक्ष्व॒ यज॑मानाय॒ कामान्॑।
भूमि॑र्भू॒त्वा म॑हि॒मानं॑ पुपोष।
ततो॑ दे॒वी व॑र्धयते॒ पयाꣳ॑सि।
य॒ज्ञिया॑ य॒ज्ञं वि च॒ यन्ति॒ शं च॑।
ओष॑धी॒राप॑ इ॒ह शक्व॑रीश्च।
यो मा॑ हृ॒दा मन॑सा॒ यश्च॑ वा॒चा॥५८॥

%3.7.6.5
यो ब्रह्म॑णा॒ कर्म॑णा॒ द्वेष्टि॑ देवाः।
यः श्रु॒तेन॒ हृद॑येनेष्ण॒ता च॑।
तस्ये᳚न्द्र॒ वज्रे॑ण॒ शिर॑श्छिनद्मि।
ऊर्णा॑मृदु॒ प्रथ॑मानꣴ स्यो॒नम्।
दे॒वेभ्यो॒ जुष्ट॒ꣳ॒ सद॑नाय ब॒र्॒हिः।
सु॒व॒र्गे लो॒के यज॑मान॒ꣳ॒ हि धे॒हि।
मां नाक॑स्य पृ॒ष्ठे प॑र॒मे व्यो॑मन्।
चतुः॑ शिखण्डा युव॒तिः सु॒पेशाः᳚।
घृ॒तप्र॑तीका व॒युना॑नि वस्ते।
साऽऽस्ती॒र्यमा॑णा मह॒ते सौभ॑गाय॥५९॥

%3.7.6.6
सा मे॑ धुक्ष्व॒ यज॑मानाय॒ कामान्॑।
शि॒वा च॑ मे श॒ग्मा चै॑धि।
स्यो॒ना च॑ मे सु॒षदा॑ चैधि।
ऊर्ज॑स्वती च मे॒ पय॑स्वती चैधि।
इष॒मूर्जं॑ मे पिन्वस्व।
ब्रह्म॒ तेजो॑ मे पिन्वस्व।
क्ष॒त्रमोजो॑ मे पिन्वस्व।
विशं॒ पुष्टिं॑ मे पिन्वस्व।
आयु॑र॒न्नाद्यं॑ मे पिन्वस्व।
प्र॒जां प॒शून्मे॑ पिन्वस्व॥६०॥

%3.7.6.7
अ॒स्मिन् य॒ज्ञ उप॒ भूय॒ इन्नु मे᳚।
अवि॑क्षोभाय परि॒धीं द॑धामि।
ध॒र्ता ध॒रुणो॒ धरी॑यान्।
अ॒ग्निर्द्वेषाꣳ॑सि॒ निरि॒तो नु॑दातै।
विच्छि॑नद्मि॒ विधृ॑तीभ्याꣳ स॒पत्नान्॑।
जा॒तान्भ्रातृ॑व्या॒न्॒ ये च॑ जनि॒ष्यमा॑णाः।
वि॒शो य॒न्त्राभ्यां॒ विध॑माम्येनान्।
अ॒हꣴ स्वाना॑मुत्त॒मो॑\-ऽसानि देवाः।
वि॒शो य॒न्त्रे नु॒दमा॑ने॒ अरा॑तिम्।
विश्वं॑ पा॒प्मान॒मम॑तिं दुर्मरा॒युम्॥६१॥

%3.7.6.8
सीद॑न्ती दे॒वी सु॑कृ॒तस्य॑ लो॒के।
धृती᳚ स्थो॒ विधृ॑ती॒ स्वधृ॑ती।
प्रा॒णान्मयि॑ धारयतम्।
प्र॒जां मयि॑ धारयतम्।
प॒शून्मयि॑ धारयतम्।
अ॒यं प्र॑स्त॒र उ॒भय॑स्य ध॒र्ता।
ध॒र्ता प्र॑या॒जाना॑मु॒तानू॑या॒जाना᳚म्।
स दा॑धार स॒मिधो॑ वि॒श्वरू॑पाः।
तस्मि॒न्थ्स्रुचो॒ अध्या सा॑दयामि।
आ रो॑ह प॒थो जु॑हु देव॒यानान्॑॥६२॥

%3.7.6.9
यत्रर्‌\mbox{}ष॑यः प्रथम॒जा ये पु॑रा॒णाः।
हिर॑ण्यपक्षा\-ऽजि॒रा सम्भृ॑ताङ्गा।
वहा॑सि मा सु॒कृतां॒ यत्र॑ लो॒काः।
अवा॒हं बा॑ध उप॒भृता॑ स॒पत्नान्॑।
जा॒तान्भ्रातृ॑व्या॒न्॒ ये च॑ जनि॒ष्यमा॑णाः।
दोहै॑ य॒ज्ञꣳ सु॒दुघा॑मिव धे॒नुम्।
अ॒हमुत्त॑रो भूयासम्।
अध॑रे॒ मथ्स॒पत्नाः᳚।
यो मा॑ वा॒चा मन॑सा दुर्मरा॒युः।
हृ॒दा\-ऽरा॑ती॒याद॑भि॒दास॑दग्ने॥६३॥

%3.7.6.10
इ॒दम॑स्य चि॒त्तमध॑रं ध्रु॒वायाः᳚।
अ॒हमुत्त॑रो भूयासम्।
अध॑रे॒ मथ्स॒पत्नाः᳚।
ऋ॒ष॒भो॑ऽसि शाक्व॒रः।
घृ॒ताची॑नाꣳ सू॒नुः।
प्रि॒येण॒ नाम्ना᳚ प्रि॒ये सद॑सि सीद।
स्यो॒नो मे॑ सीद सु॒षदः॑ पृथि॒व्याम्।
प्रथ॑यि प्र॒जया॑ प॒शुभिः॑ सुव॒र्गे लो॒के।
दि॒वि सी॑द पृथि॒व्याम॒न्तरि॑क्षे।
अ॒हमुत्त॑रो भूयासम्॥६४॥\phantom{त्त॑}%for spacing!

%3.7.6.11
अध॑रे॒ मथ्स॒पत्नाः᳚।
इ॒यꣴ स्था॒ली घृ॒तस्य॑ पू॒र्णा।
अच्छि॑न्नपयाः श॒तधा॑र॒ उथ्सः॑।
मा॒रु॒तेन॒ शर्म॑णा॒ दैव्ये॑न।
य॒ज्ञो॑ऽसि स॒र्वतः॑ श्रि॒तः।
स॒र्वतो॒ मां भू॒तं भ॑वि॒ष्यच्छ्र॑यताम्।
श॒तं मे॑ सन्त्वा॒शिषः॑।
स॒हस्रं॑ मे सन्तु सू॒नृताः᳚।
इरा॑वतीः पशु॒मतीः᳚।
प्र॒जा\-प॑तिरसि स॒र्वतः॑ श्रि॒तः॥६५॥

%3.7.6.12
स॒र्वतो॒ मां भू॒तं भ॑वि॒ष्यच्छ्र॑यताम्।
श॒तं मे॑ सन्त्वा॒शिषः॑।
स॒हस्रं॑ मे सन्तु सू॒नृताः᳚।
इरा॑वतीः पशु॒मतीः᳚।
इ॒दमि॑न्द्रि॒यम॒मृतं॑ वी॒र्यम्᳚।
अ॒नेनेन्द्रा॑य प॒शवो॑\-ऽचिकिथ्सन्।
तेन॑ देवा अव॒तोप॒ माम्।
इ॒हेष॒मूर्जं॒ यशः॒ सह॒ ओजः॑ सनेयम्।
शृ॒तं मयि॑ श्रयताम्।
यत्पृ॑थि॒वीमच॑र॒त्तत्प्रवि॑ष्टम्॥६६॥

%3.7.6.13
येनासि॑ञ्च॒द्बल॒मिन्द्रे᳚ प्र॒जा\-प॑तिः।
इ॒दं तच्छु॒क्रं मधु॑ वा॒जिनी॑वत्।
येनो॒परि॑ष्टा॒दधि॑नोन्महे॒\-न्द्रम्।
दधि॒ मां धि॑नोतु।
अ॒यं वे॒दः पृ॑थि॒वीमन्व॑विन्दत्।
गुहा॑ स॒तीं गह॑ने॒ गह्व॑रेषु।
स वि॑न्दतु॒ यज॑मानाय लो॒कम्।
अच्छि॑द्रं य॒ज्ञं भूरि॑कर्मा करोतु।
अ॒यं य॒ज्ञः सम॑सदद्ध॒विष्मान्॑।
ऋ॒चा साम्ना॒ यजु॑षा दे॒वता॑भिः॥६७॥

%3.7.6.14
तेन॑ लो॒कान्थ्सूर्य॑वतो जयेम।
इन्द्र॑स्य स॒ख्यम॑मृत॒त्वम॑\-श्याम्।
यो नः॒ कनी॑य इ॒ह का॒मया॑तै।
अ॒स्मिन् य॒ज्ञे यज॑मानाय॒ मह्यम्᳚।
अप॒ तमि॑न्द्रा॒ग्नी भुव॑नान्नुदेताम्।
अ॒हं प्र॒जां वी॒रव॑तीं विदेय।
अग्ने॑ वाजजित्।
वाजं॑ त्वा सरि॒ष्यन्तम्᳚।
वाजं॑ जे॒ष्यन्तम्᳚।
वा॒जिनं॑ वाज॒जितम्᳚॥६८॥

%3.7.6.15
वा॒ज॒जि॒त्यायै॒ सं मा᳚र्ज्मि।
अ॒ग्निम॑न्ना॒दम॒न्नाद्या॑य।
उप॑हूतो॒ द्यौः पि॒ता।
उप॒ मां द्यौः पि॒ता ह्व॑यताम्।
अ॒ग्निराग्नी᳚ध्रात्।
आयु॑षे॒ वर्च॑से।
जी॒वात्वै पुण्या॑य।
उप॑हूता पृथि॒वी मा॒ता।
उप॒ मां मा॒ता पृ॑थि॒वी ह्व॑यताम्।
अ॒ग्निराग्नी᳚ध्रात्॥६९॥

%3.7.6.16
आयु॑षे॒ वर्च॑से।
जी॒वात्वै पुण्या॑य।
मनो॒ ज्योति॑र्जुषता॒मा\-ज्यम्᳚।
विच्छि॑न्नं य॒ज्ञꣳ समि॒मं द॑धातु।
बृह॒स्पति॑स्तनुतामि॒मं नः॑।
विश्वे॑ दे॒वा इ॒ह मा॑दयन्ताम्।
यन्ते॑ अग्न आवृ॒श्चामि॑।
अ॒हं वा᳚ क्षिपि॒तश्चरन्॑।
प्र॒जां च॒ तस्य॒ मूलं॑ च।
नी॒चैर्दे॑वा॒ नि वृ॑श्चत॥७०॥

%3.7.6.17
अग्ने॒ यो नो॑\-ऽभि॒दास॑ति।
स॒मा॒नो यश्च॒ निष्ट्यः॑।
इ॒ध्मस्ये॑व प्र॒क्षाय॑तः।
मा तस्योच्छे॑षि॒ किञ्च॒न।
यो मां द्वेष्टि॑ जातवेदः।
यं चा॒ऽऽहं द्वेष्मि॒ यश्च॒ माम्।
सर्वा॒ꣴ॒स्तान॑ग्ने॒ सन्द॑ह।
याꣴश्चा॒हं द्वेष्मि॒ ये च॒ माम्।
अग्ने॑ वाजजित्।
वाजं॑ त्वा ससृ॒वाꣳसम्᳚॥७१॥

%3.7.6.18
वाजं॑ जिगि॒वाꣳसम्᳚।
वा॒जिनं॑ वाज॒जितम्᳚।
वा॒ज॒जि॒त्यायै॒ सम्मा᳚र्ज्मि।
अ॒ग्निम॑न्ना॒दम॒न्नाद्या॑य।
वेदि॑र्ब॒र्॒हिः शृ॒तꣳ ह॒विः।
इ॒ध्मः प॑रि॒धयः॒ स्रुचः॑।
आज्यं॑ य॒ज्ञ ऋचो॒ यजुः॑।
या॒ज्या᳚श्च वषट्का॒राः।
सं मे॒ सन्न॑तयो नमन्ताम्।
इ॒ध्म॒स॒न्नह॑ने हु॒ते॥७२॥

%3.7.6.19
दि॒वः खीलो\-ऽव॑ततः।
पृ॒थि॒व्या अध्युत्थि॑तः।
तेना॑ स॒हस्र॑काण्डेन।
द्वि॒षन्तꣳ॑ शोचयामसि।
द्वि॒षन्मे॑ ब॒हु शो॑चतु।
ओष॑धे॒ मो अ॒हꣳ शु॑चम्।
यज्ञ॒ नम॑स्ते यज्ञ।
नमो॒ नम॑श्च ते यज्ञ।
शि॒वेन॑ मे॒ सन्ति॑ष्ठस्व।
स्यो॒नेन॑ मे॒ सन्ति॑ष्ठस्व॥७३॥

%3.7.6.20
सु॒भू॒तेन॑ मे॒ सन्ति॑ष्ठस्व।
ब्र॒ह्म॒व॒र्च॒सेन॑ मे॒ सन्ति॑ष्ठस्व।
य॒ज्ञस्यर्द्धि॒मनु॒ सन्ति॑ष्ठस्व।
उप॑ ते यज्ञ॒ नमः॑।
उप॑ ते॒ नमः॑।
उप॑ ते॒ नमः॑।
त्रिष्फ॒लीक्रि॒यमा॑णानाम्।
यो न्य॒ङ्गो अ॑व॒शिष्य॑ते।
रक्ष॑सां भाग॒धेयम्᳚।
आप॒स्तत्प्र व॑हतादि॒तः॥७४॥

%3.7.6.21
उ॒लूख॑ले॒ मुस॑ले॒ यच्च॒ शूर्पे᳚।
आ॒शि॒श्लेष॑ दृ॒षदि॒ यत्क॒पाले᳚।
अ॒व॒प्रुषो॑ वि॒प्रुषः॒ संय॑जामि।
विश्वे॑ दे॒वा ह॒विरि॒दं जु॑षन्ताम्।
य॒ज्ञे या वि॒प्रुषः॒ सन्ति॑ ब॒ह्वीः।
अ॒ग्नौ ताः सर्वाः॒ स्वि॑ष्टाः॒ सुहु॑ता जुहोमि।
उ॒द्यन्न॒द्यमि॑त्र महः।
स॒पत्ना᳚न्मे अनीनशः।
दिवै॑नान् वि॒द्युता॑ जहि।
नि॒म्रोच॒न्नध॑रान्कृधि॥७५॥

%3.7.6.22
उ॒द्यन्न॒द्य वि नो॑ भज।
पि॒ता पु॒त्रेभ्यो॒ यथा᳚।
दी॒र्घा॒यु॒त्वस्य॑ हेशिषे।
तस्य॑ नो देहि सूर्य।

\dnsub{हृद्रोगघ्न-मन्त्राः}

उ॒द्यन्न॒द्य मि॑त्रमहः।
आ॒रोह॒न्नुत्त॑रां॒ दिवम्᳚।
हृ॒द्रो॒गं मम॑ सूर्य।
ह॒रि॒माणं॑ च नाशय।
शुके॑षु मे हरि॒माणम्᳚।
रो॒प॒णाका॑सु दध्मसि॥७६॥

%3.7.6.23
अथो॑ हारिद्र॒वेषु॑ मे।
ह॒रि॒माणं॒ नि द॑ध्मसि।
उद॑गाद॒यमा॑दि॒त्यः।
विश्वे॑न॒ सह॑सा स॒ह।
द्वि॒षन्तं॒ मम॑ र॒न्धयन्॑।
मो अ॒हं द्वि॑ष॒तो र॑धम्।

{\small \closesub{}}

यो नः॒ शपा॒दश॑पतः।
यश्च॑ नः॒ शप॑तः॒ शपा᳚त्।
उ॒षाश्च॒ तस्मै॑ नि॒म्रुक्च॑।
सर्वं॑ पा॒पꣳ समू॑हताम्॥७७॥

%3.7.6.24
यो नः॑ स॒पत्नो॒ यो रणः॑।
मर्तो॑\-ऽभि॒दास॑ति देवाः।
इ॒ध्मस्ये॑व प्र॒क्षाय॑तः।
मा तस्योच्छे॑षि॒ किञ्च॒न।
अव॑सृष्टः॒ परा॑पत।
श॒रो ब्रह्म॑सꣳशितः।
गच्छा॒\-ऽमित्रा॒न्प्र वि॑श।
मैषां॒ कञ्च॒नोच्छि॑षः॥७८॥\anuvakamend[पतिः॑ प्र॒जा\-प॑तये तप॒स्वी वा॒चा सौभ॑गाय प॒शून्मे॑ पिन्वस्व दुर्मरा॒युं दे॑व॒याना॑नग्ने॒\-ऽन्तरि॑क्षे॒\-ऽहमुत्त॑रो भूयासं प्र॒जा\-प॑तिरसि स॒र्वतः॑ श्रि॒तः प्रवि॑ष्टं दे॒वता॑भिर्वाज॒जितं॑ पृथि॒वी ह्व॑यताम॒ग्निराग्नी᳚ध्राद्वृश्चत ससृ॒वाꣳसꣳ॑ हु॒ते स्यो॒नेन॑ मे॒ सन्ति॑ष्ठस्वे॒तः कृ॑धि दध्मस्यूहताम॒ष्टौ च॑]

%3.7.7.1
सक्षे॒दं प॑श्य।
विध॑र्तरि॒दं प॑श्य।
नाके॒दं प॑श्य।
र॒मतिः॒ पनि॑ष्ठा।
ऋ॒तं वर्‌\mbox{}षि॑ष्ठम्।
अ॒मृता॒यान्या॒हुः।
सूर्यो॒ वरि॑ष्ठो अ॒क्षभि॒र्विभा॑ति।
अनु॒ द्यावा॑पृथि॒वी दे॒वपु॑त्रे।
दी॒क्षाऽसि॒ तप॑सो॒ योनिः॑।
तपो॑ऽसि॒ ब्रह्म॑णो॒ योनिः॑॥७९॥

%3.7.7.2
ब्रह्मा॑सि क्ष॒त्रस्य॒ योनिः॑।
क्ष॒त्रम॑स्यृ॒तस्य॒ योनिः॑।
ऋ॒तम॑सि॒ भूरा र॑भे।
श्र॒द्धां मन॑सा।
दी॒क्षां तप॑सा।
विश्व॑स्य॒ भुव॑न॒स्याधि॑पत्नीम्।
सर्वे॒ कामा॒ यज॑मानस्य सन्तु।
वातं॑ प्रा॒णं मन॑सा॒\-ऽन्वा र॑भामहे।
प्र॒जा\-प॑तिं॒ यो भुव॑नस्य गो॒पाः।
स नो॑ मृ॒त्योस्त्रा॑यतां॒ पात्वꣳह॑सः॥८०॥

%3.7.7.3
ज्योग्जी॒वा ज॒राम॑शीमहि।
इन्द्र॑ शाक्वर गाय॒त्रीं प्रप॑द्ये।
तान्ते॑ युनज्मि।
इन्द्र॑ शाक्वर त्रि॒ष्टुभं॒ प्रप॑द्ये।
तान्ते॑ युनज्मि।
इन्द्र॑ शाक्वर॒ जग॑तीं॒ प्रप॑द्ये।
तान्ते॑ युनज्मि।
इन्द्र॑ शाक्वरानु॒ष्टुभं॒ प्रप॑द्ये।
तान्ते॑ युनज्मि।
इन्द्र॑ शाक्वर प॒ङ्क्तिं प्रप॑द्ये॥८१॥

%3.7.7.4
तान्ते॑ युनज्मि।
आऽहं दी॒क्षाम॑रुहमृ॒तस्य॒ पत्नी᳚म्।
गा॒य॒त्रेण॒ छन्द॑सा॒ ब्रह्म॑णा च।
ऋ॒तꣳ स॒त्ये॑\-ऽधायि।
स॒त्यमृ॒ते॑\-ऽधायि।
ऋ॒तं च॑ मे स॒त्यं चा॑भूताम्।
ज्योति॑रभूव॒ꣳ॒ सुव॑रगमम्।
सु॒व॒र्गं लो॒कं नाक॑स्य पृ॒ष्ठम्।
ब्र॒ध्नस्य॑ वि॒ष्टप॑मगमम्।
पृ॒थि॒वी दी॒क्षा॥८२॥

%3.7.7.5
तया॒ऽग्निर्दी॒क्षया॑ दीक्षि॒तः।
यया॒ऽग्निर्दी॒क्षया॑ दीक्षि॒तः।
तया᳚ त्वा दी॒क्षया॑ दीक्षयामि।
अ॒न्तरि॑क्षं दी॒क्षा।
तया॑ वा॒युर्दी॒क्षया॑ दीक्षि॒तः।
यया॑ वा॒युर्दी॒क्षया॑ दीक्षि॒तः।
तया᳚ त्वा दी॒क्षया॑ दीक्षयामि।
द्यौर्दी॒क्षा।
तया॑ऽऽदि॒त्यो दी॒क्षया॑ दीक्षि॒तः।
यया॑ऽऽदि॒त्यो दी॒क्षया॑ दीक्षि॒तः॥८३॥

%3.7.7.6
तया᳚ त्वा दी॒क्षया॑ दीक्षयामि।
दिशो॑ दी॒क्षा।
तया॑ च॒न्द्रमा॑ दी॒क्षया॑ दीक्षि॒तः।
यया॑ च॒न्द्रमा॑ दी॒क्षया॑ दीक्षि॒तः।
तया᳚ त्वा दी॒क्षया॑ दीक्षयामि।
आपो॑ दी॒क्षा।
तया॒ वरु॑णो॒ राजा॑ दी॒क्षया॑ दीक्षि॒तः।
यया॒ वरु॑णो॒ राजा॑ दी॒क्षया॑ दीक्षि॒तः।
तया᳚ त्वा दी॒क्षया॑ दीक्षयामि।
ओष॑धयो दी॒क्षा॥८४॥

%3.7.7.7
तया॒ सोमो॒ राजा॑ दी॒क्षया॑ दीक्षि॒तः।
यया॒ सोमो॒ राजा॑ दी॒क्षया॑ दीक्षि॒तः।
तया᳚ त्वा दी॒क्षया॑ दीक्षयामि।
वाग्दी॒क्षा।
तया᳚ प्रा॒णो दी॒क्षया॑ दीक्षि॒तः।
यया᳚ प्रा॒णो दी॒क्षया॑ दीक्षि॒तः।
तया᳚ त्वा दी॒क्षया॑ दीक्षयामि।
पृ॒थि॒वी त्वा॒ दीक्ष॑\-माण॒\-मनु॑ दीक्षताम्।
अ॒न्तरि॑क्षं त्वा॒ दीक्ष॑\-माण॒\-मनु॑ दीक्षताम्।
द्यौस्त्वा॒ दीक्ष॑\-माण॒\-मनु॑ दीक्षताम्॥८५॥

%3.7.7.8
दिश॑स्त्वा॒ दीक्ष॑\-माण॒\-मनु॑ दीक्षन्ताम्।
आप॑स्त्वा॒ दीक्ष॑\-माण॒\-मनु॑ दीक्षन्ताम्।
ओष॑धयस्त्वा॒ दीक्ष॑\-माण॒\-मनु॑ दीक्षन्ताम्।
वाक्त्वा॒ दीक्ष॑\-माण॒\-मनु॑ दीक्षताम्।
ऋच॑स्त्वा॒ दीक्ष॑\-माण॒\-मनु॑ दीक्षन्ताम्।
सामा॑नि त्वा॒ दीक्ष॑\-माण॒\-मनु॑ दीक्षन्ताम्।
यजूꣳ॑षि त्वा॒ दीक्ष॑\-माण॒\-मनु॑ दीक्षन्ताम्।
अह॑श्च॒ रात्रि॑श्च।
कृ॒षिश्च॒ वृष्टि॑श्च।
त्विषि॒श्चाप॑चितिश्च॥८६॥

%3.7.7.9
आप॒श्चौष॑धयश्च।
ऊर्क्च॑ सू॒नृता॑ च।
तास्त्वा॒ दीक्ष॑\-माण॒\-मनु॑ दीक्षन्ताम्।
स्वे दक्षे॒ दक्ष॑पिते॒ह सी॑द।
दे॒वानाꣳ॑ सु॒म्नो म॑ह॒ते रणा॑य।
स्वा॒स॒स्थस्त॒नुवा॒ संवि॑शस्व।
पि॒तेवै॑धि सू॒नव॒ आ सु॒शेवः॑।
शि॒वो मा॑ शि॒वमा वि॑श।
स॒त्यं म॑ आ॒त्मा।
श्र॒द्धा मेऽक्षि॑तिः॥८७॥

%3.7.7.10
तपो॑ मे प्रति॒ष्ठा।
स॒वि॒तृप्र॑सूता मा॒ दिशो॑ दीक्षयन्तु।
स॒त्यम॑स्मि।
अ॒हं त्वद॑स्मि॒ मद॑सि॒ त्वमे॒तत्।
ममा॑सि॒ योनि॒स्तव॒ योनि॑रस्मि।
ममै॒व सन्वह॑ ह॒व्यान्य॑ग्ने।
पु॒त्रः पि॒त्रे लो॑क॒कृज्जा॑तवेदः।
आ॒जुह्वा॑नः सु॒प्रती॑कः पु॒रस्ता᳚त्।
अग्ने॒ स्वां योनि॒मा सी॑द सा॒ध्या।
अ॒स्मिन्थ्स॒धस्थे॒ अध्युत्त॑रस्मिन्॥८८॥

%3.7.7.11
विश्वे॑ देवा॒ यज॑मानश्च सीदत।
एक॑मि॒षे विष्णु॒स्त्वा\-ऽन्वे॑तु।
द्वे ऊ॒र्जे विष्णु॒स्त्वा\-ऽन्वे॑तु।
त्रीणि॑ व्र॒ताय॒ विष्णु॒स्त्वा\-ऽन्वे॑तु।
च॒त्वारि॒ मायो॑भवाय॒ विष्णु॒स्त्वा\-ऽन्वे॑तु।
पञ्च॑ प॒शुभ्यो॒ विष्णु॒स्त्वा\-ऽन्वे॑तु।
षड्रा॒यस्पोषा॑य॒ विष्णु॒स्त्वा\-ऽन्वे॑तु।
स॒प्त स॒प्तभ्यो॒ होत्रा᳚भ्यो॒ विष्णु॒स्त्वा\-ऽन्वे॑तु।
सखा॑यः स॒प्तप॑दा अभूम।
स॒ख्यं ते॑ गमेयम्॥८९॥

%3.7.7.12
स॒ख्यात्ते॒ मा यो॑षम्।
स॒ख्यान्मे॒ मा यो᳚ष्ठाः।
साऽसि॑ सुब्रह्मण्ये।
तस्या᳚स्ते पृथि॒वी पादः॑।
साऽसि॑ सुब्रह्मण्ये।
तस्या᳚स्ते॒\-ऽन्तरि॑क्षं॒ पादः॑।
साऽसि॑ सुब्रह्मण्ये।
तस्या᳚स्ते॒ द्यौः पादः॑।
साऽसि॑ सुब्रह्मण्ये।
तस्या᳚स्ते॒ दिशः॒ पादः॑॥९०॥

%3.7.7.13
प॒रोर॑जास्ते पञ्च॒मः पादः॑।
सा न॒ इष॒मूर्जं॑ धुक्ष्व।
तेज॑ इन्द्रि॒यम्।
ब्र॒ह्म॒व॒र्च॒सम॒न्नाद्यम्᳚।
वि मि॑मे त्वा॒ पय॑स्वतीम्।
दे॒वानां धे॒नुꣳ सु॒दुघा॒मन॑पस्फुरन्तीम्।
इन्द्रः॒ सोमं॑ पिबतु।
क्षेमो॑ अस्तु नः।
इ॒मान्न॑राः कृणुत॒ वेदि॒मेत्य॑।
वसु॑मतीꣳ रु॒द्रव॑तीमादि॒त्यव॑तीम्॥९१॥

%3.7.7.14
वर्ष्म॑न्दि॒वः।
नाभा॑ पृथि॒व्याः।
यथा॒ऽयं यज॑मानो॒ न रिष्ये᳚त्।
दे॒वस्य॑ सवि॒तुः स॒वे।
चतुः॑ शिखण्डा युव॒तिः सु॒पेशाः᳚।
घृ॒तप्र॑तीका॒ भुव॑नस्य॒ मध्ये᳚।
तस्याꣳ॑ सुप॒र्णावधि॒ यौ निवि॑ष्टौ।
तयो᳚र्दे॒वाना॒मधि॑ भाग॒धेयम्᳚।
अप॒ जन्यं॑ भ॒यं नु॑द।
अप॑ च॒क्राणि॑ वर्तय।
गृ॒हꣳ सोम॑स्य गच्छतम्।
न वा उ॑ वे॒तन्म्रि॑यसे॒ न रि॑ष्यसि।
दे॒वाꣳ इदे॑षि प॒थिभिः॑ सु॒गेभिः॑।
यत्र॒ यन्ति॑ सु॒कृतो॒ नापि॑ दु॒ष्कृतः॑।
तत्र॑ त्वा दे॒वः स॑वि॒ता द॑धातु॥९२॥\anuvakamend[ब्रह्म॑णो॒ योनि॒रꣳह॑सः प॒ङ्क्तिं प्रप॑द्ये दी॒क्षा यया॑\-ऽऽदि॒त्यो दी॒क्षया॑ दीक्षि॒तस्तया᳚ त्वा दी॒क्षया॑ दीक्षया॒म्योष॑धयो दी॒क्षा द्यौस्त्वा॒ दीक्ष॑\-माण॒\-मनु॑ दीक्षता॒मप॑चिति॒श्चाक्षि॑ति॒रुत्त॑रस्मिन्गमेयं॒ दिशः॒ पाद॑ आदि॒त्यव॑तीं वर्तय॒ पञ्च॑ च]

%3.7.8.1
यद॒स्य पा॒रे रज॑सः।
शु॒क्रं  ज्योति॒रजा॑यत।
तन्नः॑ पर्‌\mbox{}ष॒दति॒ द्विषः॑।
अग्ने॑ वैश्वानर॒ स्वाहा᳚।
यस्मा᳚द्भी॒षा\-ऽवा॑शिष्ठाः।
ततो॑ नो॒ अभ॑यं कृधि।
प्र॒जाभ्यः॒ सर्वा᳚भ्यो मृड।
नमो॑ रु॒द्राय॑ मी॒ढुषे᳚।
यस्मा᳚द्भी॒षा न्यष॑दः।
ततो॑ नो॒ अभ॑यं कृधि॥९३॥

%3.7.8.2
प्र॒जाभ्यः॒ सर्वा᳚भ्यो मृड।
नमो॑ रु॒द्राय॑ मी॒ढुषे᳚।
उदु॑स्र तिष्ठ॒ प्रति॑ तिष्ठ॒ मारि॑षः।
मेमं य॒ज्ञं यज॑मानं च रीरिषः।
सु॒व॒र्गे लो॒के यज॑मान॒ꣳ॒ हि धे॒हि।
शन्न॑ एधि द्वि॒पदे॒ शं चतु॑ष्पदे।
यस्मा᳚द्भी॒षा\-ऽवे॑पिष्ठाः प॒लायि॑ष्ठाः स॒मज्ञा᳚स्थाः।
ततो॑ नो॒ अभ॑यं कृधि।
प्र॒जाभ्यः॒ सर्वा᳚भ्यो मृड।
नमो॑ रु॒द्राय॑ मी॒ढुषे᳚॥९४॥

%3.7.8.3
य इ॒दमकः॑।
तस्मै॒ नमः॑।
तस्मै॒ स्वाहा᳚।
न वा उ॑वे॒तन्म्रि॑यसे।
आशा॑नां त्वा॒ विश्वा॒ आशाः᳚।
य॒ज्ञस्य॒ हि स्थ ऋ॒त्वियौ᳚।
इन्द्रा᳚ग्नी॒ चेत॑नस्य च।
हु॒ता॒हु॒तस्य॑ तृप्यतम्।
अहु॑तस्य हु॒तस्य॑ च।
हु॒तस्य॒ चाहु॑तस्य च।
अहु॑तस्य हु॒तस्य॑ च।
इन्द्रा᳚ग्नी अ॒स्य सोम॑स्य।
वी॒तं पि॑बतं जु॒षेथा᳚म्।
मा यज॑मानं॒ तमो॑ विदत्।
मर्त्विजो॒ मो इ॒माः प्र॒जाः।
मा यः सोम॑मि॒मं पिबा᳚त्।
सꣳसृ॑ष्टमु॒भयं॑ कृ॒तम्॥९५॥\anuvakamend[कृ॒धि॒ मी॒ढुषे\-ऽहु॑तस्य च स॒प्त च॑]

%3.7.9.1
अ॒ना॒गस॑स्त्वा व॒यम्।
इन्द्रे॑ण॒ प्रेषि॑ता॒ उप॑।
वा॒युष्टे॑ अस्त्वꣳश॒भूः।
मि॒त्रस्ते॑ अस्त्वꣳश॒भूः।
वरु॑णस्ते अस्त्वꣳश॒भूः।
अपा᳚ङ्क्षया॒ ऋत॑स्य गर्भाः।
भुव॑नस्य गोपाः॒ श्येना॑ अतिथयः।
पर्व॑तानां ककुभः प्र॒युतो॑ नपातारः।
व॒ग्नुनेन्द्रꣴ॑ ह्वयत।
घोषे॒णामी॑वाꣴश्चातयत॥९६॥

%3.7.9.2
यु॒क्ताः स्थ॒ वह॑त।
दे॒वा ग्रावा॑ण॒ इन्दु॒रिन्द्र॒ इत्य॑वादिषुः।
एन्द्र॑मचुच्यवुः पर॒मस्याः᳚ परा॒वतः॑।
आऽस्माथ्स॒धस्था᳚त्।
ओरोर॒न्तरि॑क्षात्।
आ सु॑भू॒तम॑सुषवुः।
ब्र॒ह्म॒व॒र्च॒सं म॒ आसु॑षवुः।
स॒म॒रे रक्षाꣴ॑स्यवधिषुः।
अप॑हतं ब्रह्म॒ज्यस्य॑।
वाक्च॑ त्वा॒ मन॑श्च श्रीणीताम्॥९७॥

%3.7.9.3
प्रा॒णश्च॑ त्वा\-ऽपा॒नश्च॑ श्रीणीताम्।
चक्षु॑श्च त्वा॒ श्रोत्रं॑ च श्रीणीताम्।
दक्ष॑श्च त्वा॒ बलं॑ च श्रीणीताम्।
ओज॑श्च त्वा॒ सह॑श्च श्रीणीताम्।
आयु॑श्च त्वा\-ऽज॒रा च॑ श्रीणीताम्।
आ॒त्मा च॑ त्वा त॒नूश्च॑ श्रीणीताम्।
शृ॒तो॑ऽसि शृ॒तं कृ॑तः।
शृ॒ताय॑ त्वा शृ॒तेभ्य॑स्त्वा।
यमिन्द्र॑मा॒हुर्वरु॑णं॒ यमा॒हुः।
यं मि॒त्रमा॒हुर्यमु॑ स॒त्यमा॒हुः॥९८॥

%3.7.9.4
यो दे॒वानां᳚ दे॒वत॑मस्तपो॒जाः।
तस्मै᳚ त्वा॒ तेभ्य॑स्त्वा।
मयि॒ त्यदि॑न्द्रि॒यं म॒हत्।
मयि॒ दक्षो॒ मयि॒ क्रतुः॑।
मयि॑ धायि सु॒वीर्यम्᳚।
त्रिशु॑ग्घ॒र्मो वि भा॑तु मे।
आकू᳚त्या॒ मन॑सा स॒ह।
वि॒राजा॒ ज्योति॑षा स॒ह।
य॒ज्ञेन॒ पय॑सा स॒ह।
तस्य॒ दोह॑मशीमहि॥९९॥

%3.7.9.5
तस्य॑ सु॒म्नम॑शीमहि।
तस्य॑ भ॒क्षम॑शीमहि।
वाग्जु॑षा॒णा सोम॑स्य तृप्यतु।
मि॒त्रो जना॒न्प्र स मि॑त्र।
यस्मा॒न्न जा॒तः परो॑ अ॒न्यो अस्ति॑।
य आ॑वि॒वेश॒ भुव॑नानि॒ विश्वा᳚।
प्र॒जा\-प॑तिः प्र॒जया॑ संविदा॒नः।
त्रीणि॒ ज्योतीꣳ॑षि सचते॒ स षो॑ड॒शी।
ए॒ष ब्र॒ह्मा य ऋ॒त्वियः॑।
इन्द्रो॒ नाम॑ श्रु॒तो ग॒णे॥१००॥

%3.7.9.6
प्र ते॑ म॒हे वि॒दथे॑ शꣳसिष॒ꣳ॒ हरी᳚।
य ऋ॒त्वियः॒ प्र ते॑ वन्वे।
व॒नुषो॑ हर्य॒तं मदम्᳚।
इन्द्रो॒ नाम॑ घृ॒तं नयः।
हरि॑भि॒श्चारु॒ सेच॑ते।
श्रु॒तो ग॒ण आ त्वा॑ विशन्तु।
हरि॑वर्पस॒ङ्गिरः॑।
इन्द्राधि॑प॒ते\-ऽधि॑पति॒स्त्वं दे॒वाना॑मसि।
अधि॑पतिं॒ माम्।
आयु॑ष्मन्तं॒ वर्च॑स्वन्तं मनु॒ष्ये॑षु कुरु॥१०१॥

%3.7.9.7
इन्द्र॑श्च स॒म्राड्वरु॑णश्च॒ राजा᳚।
तौ ते॑ भ॒क्षं च॑क्रतु॒रग्र॑ ए॒तम्।
तयो॒रनु॑\- भ॒क्षं भ॑क्षयामि।
वाग्जु॑षा॒णा सोम॑स्य तृप्यतु।
प्र॒जा\-प॑तिर्वि॒श्वक॑र्मा।
तस्य॒ मनो॑ दे॒वं य॒ज्ञेन॑ राध्यासम्।
अ॒र्थे॒गा अ॒स्य ज॑हितः।
अ॒व॒सान॑पते\-ऽव॒सानं॑ मे विन्द।
नमो॑ रु॒द्राय॑ वास्तो॒ष्पत॑ये।
आय॑ने वि॒द्रव॑णे॥१०२॥

%3.7.9.8
उ॒द्याने॒ यत्प॒राय॑णे।
आ॒वर्त॑ने वि॒वर्त॑ने।
यो गो॑पा॒यति॒ तꣳ हु॑वे।
यान्य॑पा॒मित्या॒न्य\-प्र॑तीत्ता॒\-न्यस्मि॑।
य॒मस्य॑ ब॒लिना॒ चरा॑मि।
इ॒हैव सन्तः॒ प्रति॒ तद्या॑तयामः।
जी॒वा जी॒वेभ्यो॒ नि ह॑राम एनत्।
अ॒नृ॒णा अ॒स्मिन्न॑नृ॒णाः पर॑स्मिन्।
तृ॒तीये॑ लो॒के अ॑नृ॒णाः स्या॑म।
ये दे॑व॒याना॑ उ॒त पि॑तृ॒याणाः᳚॥१०३॥

%3.7.9.9
सर्वा᳚न्प॒थो अ॑नृ॒णा आक्षी॑येम।
इ॒दमू॒नु श्रेयो॑\-ऽव॒सान॒मा ग॑न्म।
शि॒वे नो॒ द्यावा॑पृथि॒वी उ॒भे इ॒मे।
गोम॒द्धन॑व॒दश्व॑व॒दूर्ज॑स्वत्।
सु॒वीरा॑ वी॒रैरनु॒\- सञ्च॑रेम।
अ॒र्कः प॒वित्र॒ꣳ॒ रज॑सो वि॒मानः॑।
पु॒नाति॑ दे॒वानां॒ भुव॑नानि॒ विश्वा᳚।
द्यावा॑पृथि॒वी पय॑सा संविदा॒ने।
घृ॒तं दु॑हाते अ॒मृतं॒ प्रपी॑ने।
प॒वित्र॑म॒र्को रज॑सो वि॒मानः॑।
पु॒नाति॑ दे॒वानां॒ भुव॑नानि॒ विश्वा᳚।
सुव॒र्ज्योति॒र्यशो॑ म॒हत्।
अ॒शी॒महि॑ गा॒धमु॒त प्र॑ति॒ष्ठाम्॥१०४॥\anuvakamend[चा॒त॒य॒त॒ श्री॒णी॒ता॒ꣳ॒ स॒त्यमा॒हुर॑शीमहि ग॒णे कु॑रु वि॒द्रव॑णे पितृ॒याणा॑ अ॒र्को रज॑सो वि॒मान॒स्त्रीणि॑ च]

%3.7.10.1
उद॑स्ताम्फ्सीथ्सवि॒ता मि॒त्रो अ॑र्य॒मा।
सर्वा॑न॒मित्रा॑न\-वधीद्यु॒गेन॑।
बृ॒हन्तं॒ माम॑करद्वी॒र\-व॑न्तम्।
र॒थ॒न्त॒रे श्र॑यस्व॒ स्वाहा॑ पृथि॒व्याम्।
वा॒म॒दे॒व्ये श्र॑यस्व॒ स्वाहा॒\-ऽन्तरि॑क्षे।
बृ॒ह॒ति श्र॑यस्व॒ स्वाहा॑ दि॒वि।
बृ॒ह॒ता त्वोप॑स्तभ्नोमि।
आ त्वा॑ ददे॒ यश॑से वी॒र्या॑य च।
अ॒स्मास्व॑घ्निया यू॒यं द॑धाथेन्द्रि॒यं पयः॑।
यस्ते᳚ द्र॒फ्सो यस्त॑ उद॒र्॒षः॥१०५॥

%3.7.10.2
दैव्यः॑ के॒तुर्विश्वं॒ भुव॑नमावि॒वेश॑।
स नः॑ पा॒ह्यरि॑ष्ट्यै॒ स्वाहा᳚।
अनु॑ मा॒ सर्वो॑ य॒ज्ञो॑\-ऽयमे॑तु।
विश्वे॑ दे॒वा म॒रुतः॒ सामा॒र्कः।
आ॒प्रिय॒श्छन्दाꣳ॑सि नि॒विदो॒ यजूꣳ॑षि।
अ॒स्यै पृ॑थि॒व्यै यद्य॒ज्ञियम्᳚।
प्र॒जा\-प॑तेर्वर्त॒निमनु॑ वर्तस्व।
अनु॑वी॒रैरनु॑\- राध्याम॒ गोभिः॑।
अन्वश्वै॒रनु॒\- सर्वै॑रु पु॒ष्टैः।
अनु॑ प्र॒जया\-ऽन्वि॑न्द्रि॒येण॑॥१०६॥

%3.7.10.3
दे॒वा नो॑ य॒ज्ञमृ॑जु॒धा न॑यन्तु।
प्रति॑\-क्ष॒त्रे प्रति॑ तिष्ठामि रा॒ष्ट्रे।
प्रत्यश्वे॑षु॒ प्रति॑ तिष्ठामि॒ गोषु॑।
प्रति॑ प्र॒जायां॒ प्रति॑ तिष्ठामि॒ भव्ये᳚।
विश्व॑म॒न्याऽभि॑ वावृ॒धे।
तद॒न्यस्या॒मधि॑श्रि॒तम्।
दि॒वे च॑ वि॒श्वक॑र्मणे।
पृ॒थि॒व्यै चा॑करं॒ नमः॑।
अस्का॒न्द्यौः पृ॑थि॒वीम्।
अस्का॑नृष॒भो युवा॒गाः॥१०७॥

%3.7.10.4
स्क॒न्नेमा विश्वा॒ भुव॑ना।
स्क॒न्नो य॒ज्ञः प्र ज॑नयतु।
अस्का॒नज॑नि॒ प्राज॑नि।
आ स्क॒न्नाज्जा॑यते॒ वृषा᳚।
स्क॒न्नात्प्र ज॑निषीमहि।
ये दे॒वा येषा॑मि॒दं भा॑ग॒धेयं॑ ब॒भूव॑।
येषां᳚ प्रया॒जा उ॒तानू॑या॒जाः।
इन्द्र॑ज्येष्ठेभ्यो॒ वरु॑णराजभ्यः।
अ॒ग्निहो॑तृभ्यो दे॒वेभ्यः॒ स्वाहा᳚।
उ॒त त्या नो॒ दिवा॑ म॒तिः॥१०८॥

%3.7.10.5
अदि॑तिरू॒त्या ग॑मत्।
सा शन्ता॑ची॒ मय॑स्करत्।
अप॒ स्रिधः॑।
उ॒त त्या दैव्या॑ भि॒षजा᳚।
शन्न॑स्करतो अ॒श्विना᳚।
यू॒याता॑म॒स्मद्रपः॑।
अप॒ स्रिधः॑।
शम॒ग्निर॒ग्निभि॑स्करत्।
शन्न॑स्तपतु॒ सूर्यः॑।
शं वातो॑ वात्वर॒पाः॥१०९॥

%3.7.10.6
अप॒ स्रिधः॑।
तदित्प॒दं न विचि॑केत वि॒द्वान्।
यन्मृ॒तः पुन॑र॒प्येति॑ जी॒वान्।
त्रि॒वृद्यद्भुव॑नस्य रथ॒वृत्।
जी॒वो गर्भो॒ न मृ॒तः स जी॑वात्।
प्रत्य॑स्मै॒ पिपी॑षते।
विश्वा॑नि वि॒दुषे॑ भर।
अ॒र॒ङ्ग॒माय॒ जग्म॑वे।
अप॑श्चाद्दघ्वने॒ नरे᳚।
इन्दु॒रिन्दु॒मवा॑गात्।
इन्दो॒रिन्द्रो॑\-ऽपात्।
तस्य॑ त इन्द॒विन्द्र॑पीतस्य॒ मधु॑मतः।
उप॑हूत॒स्योप॑हूतो भक्षयामि॥११०॥\anuvakamend[उ॒द॒र्॒ष इ॑न्द्रि॒येण॒ गा म॒तिर॑र॒पा अ॑गा॒त्रीणि॑ च]

%3.7.11.1
ब्रह्म॑ प्रति॒ष्ठा मन॑सो॒ ब्रह्म॑ वा॒चः।
ब्रह्म॑ य॒ज्ञानाꣳ॑ ह॒विषा॒माज्य॑स्य।
अति॑रिक्तं॒ कर्म॑णो॒ यच्च॑ ही॒नम्।
य॒ज्ञः पर्वा॑णि प्रति॒रन्ने॑ति क॒ल्पयन्॑।
स्वाहा॑कृ॒ता\-ऽऽहु॑तिरेतु दे॒वान्।
आश्रा॑वितम॒त्याश्रा॑वितम्।
वष॑ट्कृतम॒त्यनू᳚क्तं च य॒ज्ञे।
अति॑रिक्तं॒ कर्म॑णो॒ यच्च॑ ही॒नम्।
य॒ज्ञः पर्वा॑णि प्रति॒रन्ने॑ति क॒ल्पयन्॑।
स्वाहा॑कृ॒ता\-ऽऽहु॑तिरेतु दे॒वान्॥१११॥

%3.7.11.2
यद्वो॑ देवा अतिपा॒दया॑नि।
वा॒चा चि॒त्प्रय॑तं देव॒हेड॑नम्।
अ॒रा॒यो अ॒स्माꣳ अ॒भिदु॑च्छुना॒यते᳚।
अ॒न्यत्रा॒स्मन्म॑रुत॒स्तन्निधे॑\-तन।
त॒तं म॒ आप॒स्तदु॑ तायते॒ पुनः॑।
स्वादि॑ष्ठा धी॒तिरु॒चथा॑य शस्यते।
अ॒यꣳ स॑मु॒द्र उ॒त वि॒श्वभे॑षजः।
स्वाहा॑कृतस्य॒ समु॑तृप्णुतर्भुवः।
उद्व॒यं तम॑स॒स्परि॑।
उदु॒त्यं चि॒त्रम्॥११२॥

%3.7.11.3
इ॒मं मे॑ वरुण॒ तत्त्वा॑ यामि।
त्वन्नो॑ अग्ने॒ स त्वन्नो॑ अग्ने।
त्वम॑ग्ने अ॒यासि॒ प्रजा॑पते।
इ॒मं जी॒वेभ्यः॑ परि॒धिं द॑धामि।
मैषान्नु॑गा॒दप॑रो॒ अर्ध॑मे॒तम्।
श॒तं जी॑वन्तु श॒रदः॑ पुरू॒चीः।
ति॒रो मृ॒त्युं द॑धतां॒ पर्व॑तेन।
इ॒ष्टेभ्यः॒ स्वाहा॒ वष॒डनि॑ष्टेभ्यः॒ स्वाहा᳚।
भे॒ष॒जं दुरि॑ष्ट्यै॒ स्वाहा॒ निष्कृ॑त्यै॒ स्वाहा᳚।
दौरा᳚र्ध्यै॒ स्वाहा॒ दैवी᳚भ्यस्त॒नूभ्यः॒ स्वाहा᳚॥११३॥

%3.7.11.4
ऋद्ध्यै॒ स्वाहा॒ समृ॑द्ध्यै॒ स्वाहा᳚।
यत॑ इन्द्र॒ भया॑महे।
ततो॑ नो॒ अभ॑यं कृधि।
मघ॑वञ्छ॒ग्धि तव॒ तन्न॑ ऊ॒तये᳚।
वि द्विषो॒ वि मृधो॑ जहि।
स्व॒स्ति॒दा वि॒शस्पतिः॑।
वृ॒त्र॒हा वि मृधो॑ व॒शी।
वृषेन्द्रः॑ पु॒र ए॑तु नः।
स्व॒स्ति॒दा अ॑भयङ्क॒रः।
आ॒भिर्गी॒र्भिर्यदतो॑ न ऊ॒नम्॥११४॥

%3.7.11.5
आप्या॑यय हरिवो॒ वर्ध॑मानः।
य॒दा स्तो॒तृभ्यो॒ महि॑ गो॒त्रा रु॒जासि॑।
भू॒यि॒ष्ठ॒भाजो॒ अध॑ ते स्याम।
अना᳚ज्ञातं॒ यदाज्ञा॑तम्।
य॒ज्ञस्य॑ क्रि॒यते॒ मिथु॑।
अग्ने॒ तद॑स्य कल्पय।
त्वꣳ हि वेत्थ॑ यथात॒थम्।
पुरु॑षसम्मितो य॒ज्ञः।
य॒ज्ञः पुरु॑षसम्मितः।
अग्ने॒ तद॑स्य कल्पय।
त्वꣳ हि वेत्थ॑ यथात॒थम्।
यत्पा॑क॒त्रा मन॑सा दी॒नद॑क्षा॒ न।
य॒ज्ञस्य॑ म॒न्वते॒ मर्ता॑सः।
अ॒ग्निष्टद्धोता᳚ क्रतु॒विद्वि॑जा॒नन्।
यजि॑ष्ठो दे॒वाꣳ ऋ॑तु॒शो य॑जाति॥११५॥\anuvakamend[दे॒वाꣴश्चि॒त्रं त॒नूभ्यः॒ स्वाहो॒नं पुरु॑षसम्मि॒तो\-ऽग्ने॒ तद॑स्य कल्पय॒ पञ्च॑ च]

%3.7.12.1
यद्दे॑वा देव॒हेड॑नम्।
देवा॑सश्चकृ॒मा व॒यम्।
आदि॑त्या॒\-स्तस्मा᳚न्मा मुञ्चत।
ऋ॒तस्य॒र्तेन॒ मामु॒त।
देवा॑ जीवनका॒म्या यत्।
वा॒चा\-ऽनृ॑तमूदि॒म।
अ॒ग्निर्मा॒ तस्मा॒देन॑सः।
गार्‌\mbox{}ह॑पत्यः॒ प्रमु॑ञ्चतु।
दु॒रि॒ता यानि॑ चकृ॒म।
क॒रोतु॒ माम॑ने॒नसम्᳚॥११६॥

%3.7.12.2
ऋ॒तेन॑ द्यावापृथिवी।
ऋ॒तेन॒ त्वꣳ स॑रस्वति।
ऋ॒तान्मा॑ मुञ्च॒ताꣳह॑सः।
यद॒न्यकृ॑तमारि॒म।
स॒जा॒त॒श॒ꣳ॒सादु॒त वा॑ जामिश॒ꣳ॒सात्।
ज्याय॑सः॒ शꣳसा॑दु॒त वा॒ कनी॑यसः।
अना᳚ज्ञातं दे॒वकृ॑तं॒ यदेनः॑।
तस्मा॒त्त्वम॒स्माञ्जा॑तवेदो मुमुग्धि।
यद्वा॒चा यन्मन॑सा।
बा॒हुभ्या॑मू॒रुभ्या॑मष्ठी॒वद्भ्या᳚म्॥११७॥

%3.7.12.3
शि॒श्ञैर्यदनृ॑तं चकृ॒मा व॒यम्।
अ॒ग्निर्मा॒ तस्मा॒देन॑सः।
यद्धस्ता᳚भ्यां च॒कर॒ किल्बि॑षाणि।
अ॒क्षाणां᳚ व॒ग्नुमु॑प॒जिघ्न॑मानः।
दू॒रे॒प॒श्या च॑ राष्ट्र॒भृच्च॑।
तान्य॑फ्स॒रसा॒वनु॑दत्तामृ॒णानि॑।
अदी᳚व्यन्नृ॒णं यद॒हं च॒कार॑।
यद्वादा᳚स्यन्थ्सञ्ज॒गारा॒ जने᳚भ्यः।
अ॒ग्निर्मा॒ तस्मा॒देन॑सः।
यन्मयि॑ मा॒ता गर्भे॑ स॒ति॥११८॥

%3.7.12.4
एन॑श्च॒कार॒ यत्पि॒ता।
अ॒ग्निर्मा॒ तस्मा॒देन॑सः।
यदा॑ पि॒पेष॑ मा॒तरं॑ पि॒तरम्᳚।
पु॒त्रः प्रमु॑दितो॒ धयन्॑।
अहिꣳ॑सितौ पि॒तरौ॒ मया॒ तत्।
तद॑ग्ने अनृ॒णो भ॑वामि।
यद॒न्तरि॑क्षं पृथि॒वीमु॒त द्याम्।
यन्मा॒तरं॑ पि॒तरं॑ वा जिहिꣳसि॒म।
अ॒ग्निर्मा॒ तस्मा॒देन॑सः।
यदा॒शसा॑ नि॒शसा॒ यत्प॑रा॒शसा᳚॥११९॥

%3.7.12.5
यदेन॑श्चकृ॒मा नूत॑नं॒ यत्पु॑रा॒णम्।
अ॒ग्निर्मा॒ तस्मा॒देन॑सः।
अति॑ क्रामामि दुरि॒तं यदेनः॑।
जहा॑मि रि॒प्रं प॑र॒मे स॒धस्थे᳚।
यत्र॒ यन्ति॑ सु॒कृतो॒ नापि॑ दु॒ष्कृतः॑।
तमा रो॑हामि सु॒कृतां॒ नु लो॒कम्।
त्रि॒ते दे॒वा अ॑मृजतै॒तदेनः॑।
त्रि॒त ए॒तन्म॑नु॒ष्ये॑षु मामृजे।
ततो॑ मा॒ यदि॒ किञ्चि॑दान॒शे।
अ॒ग्निर्मा॒ तस्मा॒देन॑सः॥१२०॥

%3.7.12.6
गार्‌\mbox{}ह॑पत्यः॒ प्रमु॑ञ्चतु।
दु॒रि॒ता यानि॑ चकृ॒म।
क॒रोतु॒ माम॑ने॒नसम्᳚।
दि॒वि जा॒ता अ॒फ्सु जा॒ताः।
या जा॒ता ओष॑धीभ्यः।
अथो॒ या अ॑ग्नि॒जा आपः॑।
ता नः॑ शुन्धन्तु॒ शुन्ध॑नीः।
यदापो॒ नक्तं॑ दुरि॒तं चरा॑म।
यद्वा॒ दिवा॒ नूत॑नं॒ यत्पु॑रा॒णम्।
हिर॑ण्यवर्णा॒स्तत॒ उत्पु॑नीत नः।
इ॒मं मे॑ वरुण॒ तत्त्वा॑ यामि।
त्वन्नो॑ अग्ने॒ स त्वन्नो॑ अग्ने।
त्वम॑ग्ने अ॒यासि॑॥१२१॥\anuvakamend[अ॒ने॒नस॑मष्ठी॒वद्भ्याꣳ॑ स॒ति प॑रा॒शसा॑\-ऽऽन॒शे᳚\-ऽग्निर्मा॒ तस्मा॒देन॑सः पुनीत न॒स्त्रीणि॑ च (यद्दे॑वा॒ देवा॑ ऋ॒तेन॑ सजातश॒ꣳ॒साद्यद्वा॒चा यद्धस्ता᳚भ्या॒मदी᳚व्यं॒ यन्मयि॑ मा॒ता यदा॑ पि॒पेष॒ यद॒न्तरि॑क्षं॒ यदा॒शसाऽति॑ क्रामामि त्रि॒ते दे॒वा दि॒वि जा॒ता अ॒फ्सु जा॒ता यदाप॑ इ॒मं मे॑ वरुण॒ तत्त्वा॑ यामि॒ त्वन्नो॑ अग्ने॒ स त्वन्नो॑ अग्ने॒ त्वम॑ग्ने अ॒यासि॑।
)]

%3.7.13.1
यत्ते॒ ग्राव्ण्णा॑ चिच्छि॒दुः सो॑म राजन्।
प्रि॒याण्यङ्गा॑नि॒ स्वधि॑ता॒ परूꣳ॑षि।
तथ्सन्ध॒थ्स्वाज्ये॑नो॒त व॑र्धयस्व।
अ॒ना॒गसो॒ अध॒मिथ्स॒ङ्क्षये॑म।
यत्ते॒ ग्रावा॑ बा॒हुच्यु॑तो॒ अचु॑च्यवुः।
नरो॒ यत्ते॑ दुदु॒हुर्दक्षि॑णेन।
तत्त॒ आप्या॑यतां॒ तत्ते᳚।
निष्ट्या॑यतां देव सोम।
यत्ते॒ त्वचं॑ बिभि॒दुर्यच्च॒ योनिम्᳚।
यदा॒स्थाना॒त्प्रच्यु॑तो॒ वेन॑सि॒ त्मना᳚॥१२२॥

%3.7.13.2
त्वया॒ तथ्सो॑म गु॒प्तम॑स्तु नः।
सा नः॑ स॒न्धास॑त्पर॒मे व्यो॑मन्।
अहा॒च्छरी॑रं॒ पय॑सा स॒मेत्य॑।
अ॒न्यो᳚न्यो भवति॒ वर्णो॑ अस्य।
तस्मि॑न्व॒यमुप॑हूता॒स्तव॑ स्मः।
आ नो॑ भज॒ सद॑सि वि॒श्वरू॑पे।
नृ॒चक्षाः॒ सोम॑ उ॒त शु॒श्रुग॑स्तु।
मा नो॒ वि हा॑सी॒द्गिर॑ आवृणा॒नः।
अना॑गास्त॒नुवो॑ वावृधा॒नः।
आ नो॑ रू॒पं व॑हतु॒ जाय॑मानः॥१२३॥

%3.7.13.3
उप॑ क्षरन्ति जु॒ह्वो॑ घृ॒तेन॑।
प्रि॒याण्यङ्गा॑नि॒ तव॑ व॒र्धय॑न्तीः।
तस्मै॑ ते सोम॒ नम॒ इद्वष॑ट्च।
उप॑ मा राजन्थ्सुकृ॒ते ह्व॑यस्व।
सं प्रा॑णापा॒नाभ्या॒ꣳ॒ समु॒ चक्षु॑षा॒ त्वम्।
सꣴ श्रोत्रे॑ण गच्छस्व सोम राजन्।
यत्त॒ आस्थि॑त॒ꣳ॒ शमु॒ तत्ते॑ अस्तु।
जा॒नी॒तान्नः॑ स॒ङ्गम॑ने पथी॒नाम्।
ए॒तं जा॑नीतात्पर॒मे व्यो॑मन्।
वृकाः᳚ सधस्था वि॒द रू॒पम॑स्य॥१२४॥

%3.7.13.4
यदा॒गच्छा᳚त्प॒थिभि॑र्देव॒यानैः᳚।
इ॒ष्टा॒पू॒र्ते कृ॑णुतादा॒विर॑स्मै।
अरि॑ष्टो राजन्नग॒दः परे॑हि।
नम॑स्ते अस्तु॒ चक्ष॑से रघूय॒ते।
नाक॒मारो॑ह स॒ह यज॑मानेन।
सूर्यं॑ गच्छतात्पर॒मे व्यो॑मन्।
अभू᳚द्दे॒वः स॑वि॒ता वन्द्यो॒नु नः॑।
इ॒दानी॒मह्न॑ उप॒वाच्यो॒ नृभिः॑।
वि यो रत्ना॒ भज॑ति मान॒वेभ्यः॑।
श्रेष्ठं॑ नो॒ अत्र॒ द्रवि॑णं॒ यथा॒ दध॑त्।
उप॑ नो मित्रावरुणावि॒हाव॑तम्।
अ॒न्वादी᳚ध्याथामि॒ह नः॑ सखाया।
आ॒दि॒त्यानां॒ प्रसि॑तिर्\mbox{}हे॒तिः।
उ॒ग्रा श॒तापा᳚ष्ठा घ॒विषा॒ परि॑ णो वृणक्तु।
आप्या॑यस्व॒ सन्ते᳚॥१२५॥\anuvakamend[त्मना॒ जाय॑मानो\-ऽस्य॒ दध॒त्पञ्च॑ च]

%3.7.14.1
यद्दि॑दी॒क्षे मन॑सा॒ यच्च॑ वा॒चा।
यद्वा᳚ प्रा॒णैश्चक्षु॑षा॒ यच्च॒ श्रोत्रे॑ण।
यद्रेत॑सा मिथु॒नेनाप्या॒त्मना᳚।
अ॒द्भ्यो लो॒का द॑धिरे॒ तेज॑ इन्द्रि॒यम्।
शु॒क्रा दी॒क्षायै॒ तप॑सो वि॒मोच॑नीः।
आपो॑ विमो॒क्त्रीर्मयि॒ तेज॑ इन्द्रि॒यम्।
यदृ॒चा साम्ना॒ यजु॑षा।
प॒शू॒नां चर्म॑न् ह॒विषा॑ दिदी॒क्षे।
यच्छन्दो॑भि॒रोष॑धीभि॒र्वन॒स्पतौ᳚।
अ॒द्भ्यो लो॒का द॑धिरे॒ तेज॑ इन्द्रि॒यम्॥१२६॥

%3.7.14.2
शु॒क्रा दी॒क्षायै॒ तप॑सो वि॒मोच॑नीः।
आपो॑ विमो॒क्त्रीर्मयि॒ तेज॑ इन्द्रि॒यम्।
येन॒ ब्रह्म॒ येन॑ क्ष॒त्रम्।
येने᳚न्द्रा॒ग्नी प्र॒जा\-प॑तिः॒ सोमो॒ वरु॑णो॒ येन॒ राजा᳚।
विश्वे॑ दे॒वा ऋष॑यो॒ येन॑ प्रा॒णाः।
अ॒द्भ्यो लो॒का द॑धिरे॒ तेज॑ इन्द्रि॒यम्।
शु॒क्रा दी॒क्षायै॒ तप॑सो वि॒मोच॑नीः।
आपो॑ विमो॒क्त्रीर्मयि॒ तेज॑ इन्द्रि॒यम्।
अ॒पां पुष्प॑म॒स्योष॑धीना॒ꣳ॒ रसः॑।
सोम॑स्य प्रि॒यं धाम॑॥१२७॥

%3.7.14.3
अ॒ग्नेः प्रि॒यत॑मꣳ ह॒विः स्वाहा᳚।
अ॒पां पुष्प॑म॒स्योष॑धीना॒ꣳ॒ रसः॑।
सोम॑स्य प्रि॒यं धाम॑।
इन्द्र॑स्य प्रि॒यत॑मꣳ ह॒विः स्वाहा᳚।
अ॒पां पुष्प॑म॒स्योष॑धीना॒ꣳ॒ रसः॑।
सोम॑स्य प्रि॒यं धाम॑।
विश्वे॑षां दे॒वानां᳚ प्रि॒यत॑मꣳ ह॒विः स्वाहा᳚।
व॒यꣳ सो॑म व्र॒ते तव॑।
मन॑स्त॒नूषु॒ पिप्र॑तः।
प्र॒जाव॑न्तो अशीमहि॥१२८॥

%3.7.14.4
दे॒वेभ्यः॑ पि॒तृभ्यः॒ स्वाहा᳚।
सो॒म्येभ्यः॑ पि॒तृभ्यः॒ स्वाहा᳚।
क॒व्येभ्यः॑ पि॒तृभ्यः॒ स्वाहा᳚।
देवा॑स इ॒ह मा॑दयध्वम्।
सोम्या॑स इ॒ह मा॑दयध्वम्।
कव्या॑स इ॒ह मा॑दयध्वम्।
अन॑न्तरिताः पि॒तरः॑ सो॒म्याः सो॑मपी॒थात्।
अपै॑तु मृ॒त्युर॒मृतं॑ न॒ आगन्॑।
वै॒व॒स्व॒तो नो॒ अभ॑यं कृणोतु।
प॒र्णं वन॒स्पते॑रिव॥१२९॥

%3.7.14.5
अ॒भि नः॑ शीयताꣳ र॒यिः।
सच॑तां नः॒ शची॒पतिः॑।
परं॑ मृत्यो॒ अनु॒ परे॑हि॒ पन्था᳚म्।
यस्ते॒ स्व इत॑रो देव॒याना᳚त्।
चक्षु॑ष्मते शृण्व॒ते ते᳚ ब्रवीमि।
मा नः॑ प्र॒जाꣳ री॑रिषो॒ मोत वी॒रान्।
इ॒दमू॒नु श्रेयो॑व॒सान॒माग॑न्म।
यद्गो॒जिद्ध॑न॒जिद॑श्व॒जिद्यत्।
प॒र्णं वन॒स्पते॑रिव।
अ॒भि नः॑ शीयताꣳ र॒यिः।
सच॑तां नः॒ शची॒पतिः॑॥१३०॥\anuvakamend[वन॒स्पता॑व॒द्भ्यो लो॒का द॑धिरे॒ तेज॑ इन्द्रि॒यं धामा॑शीमहीवा॒भिनः॑ शीयताꣳ र॒यिरेकं॑ च]




\prashnaend{सर्वा॒न्॒ यद्विष्ष॑ण्णेन॒ वि वै याः पु॒रस्ता॒द्देवा॑ दे॒वेषु॒ परि॑स्तृणीत॒ सक्षे॒दं यद॒स्य पा॒रे॑\-ऽना॒गस॒ उद॑स्ताम्फ्सी॒द्ब्रह्म॑ प्रति॒ष्ठा यद्दे॑वा॒ यत्ते॒ ग्राव्ण्णा॒ यद्दि॑दी॒क्षे चतु॑र्दश॥१४॥}{सर्वा॒न्भूति॑मे॒व यामे॒वाफ्स्वाहु॑तिं व्र॒तानां᳚ पर्णव॒ल्कः सो॒म्याना॑म॒स्मिन्‌ य॒ज्ञे\-ऽग्ने॒ यो नो॒ ज्योग्जी॒वाः प॒रोर॑जाः॒ प्रते॑महे॒ ब्रह्म॑ प्रति॒ष्ठा गार्‌\mbox{}ह॑पत्यस्त्रि॒ꣳ॒शदु॑त्तरश॒तम्॥१३०॥}{सर्वा॒ञ्छची॒पतिः॑॥}{हरिः॑ ओम्॥}{इति श्रीकृष्णयजुर्वेदीयतैत्तिरीयब्राह्मणे तृतीयाष्टके सप्तमः प्रपाठकः समाप्तः॥}
\clearpage
\sect{अष्टमः प्रश्नः}
\setcounter{anuvakam}{0}
\dnsub{तैत्तिरीयब्राह्मणे तृतीयाष्टके अष्टमः प्रपाठकः}

%3.8.1.1
सा॒ङ्ग्र॒ह॒ण्येष्ट्या॑ यजते।
इ॒माञ्ज॒नता॒ꣳ॒ सङ्गृ॑ह्णा॒नीति॑।
द्वाद॑शारत्नी रश॒ना भ॑वति।
द्वाद॑श॒ मासाः᳚ संवथ्स॒रः।
सं॒व॒थ्स॒रमे॒वाव॑ रुन्धे।
मौ॒ञ्जी भ॑वति।
ऊर्ग्वै मुञ्जाः᳚।
ऊर्ज॑\-मे॒वाव॑ रुन्धे।
चि॒त्रा नक्ष॑त्रं भवति।
चि॒त्रं वा ए॒तत्कर्म॑॥१॥

%3.8.1.2
यद॑श्वमे॒धः समृ॑द्ध्यै।
पुण्य॑नाम देव॒यज॑नम॒ध्यव॑स्यति।
पुण्या॑मे॒व तेन॑ की॒र्तिम॒भि ज॑यति।
अप॑दातीनृ॒त्विजः॑ स॒माव॑ह॒न्त्या सु॑ब्रह्म॒ण्यायाः᳚।
सु॒व॒र्गस्य॑ लो॒कस्य॒ सम॑ष्ट्यै।
के॒श॒श्म॒श्रु व॑पते।
न॒खानि॒ नि कृ॑न्तते।
द॒तो धा॑वते।
स्नाति॑।
अह॑तं॒ वासः॒ परि॑धत्ते।
पा॒प्मनो\-ऽप॑हत्यै।
वाचं॑ य॒त्वोप॑ वसति।
सु॒व॒र्गस्य॑ लो॒कस्य॒ गुप्त्यै᳚।
रात्रिं॑ जाग॒रय॑न्त आसते।
सु॒व॒र्गस्य॑ लो॒कस्य॒ सम॑ष्ट्यै॥२॥\anuvakamend[कर्म॑ धत्ते॒ पञ्च॑ च]

%3.8.2.1
चतु॑ष्टय्य॒ आपो॑ भवन्ति।
चतुः॑ शफो॒ वा अश्वः॑ प्राजाप॒त्यः समृ॑द्ध्यै।
ता दि॒ग्भ्यः स॒माभृ॑ता भवन्ति।
दि॒क्षु वा आपः॑।
अन्नं॒ वा आपः॑।
अ॒द्भ्यो वा अन्नं॑ जायते।
यदे॒वाद्भ्यो\-ऽन्नं॒ जाय॑ते।
तदव॑ रुन्धे।
तासु॑ ब्रह्मौद॒नं प॑चति।
रेत॑ ए॒व तद्द॑धाति॥३॥

%3.8.2.2
चतुः॑ शरावो भवति।
दि॒क्ष्वे॑व प्रति॑ तिष्ठति।
उ॒भ॒यतो॑रु॒क्मौ भ॑वतः।
उ॒भ॒यत॑ ए॒वास्मि॒न्रुचं॑ दधाति।
उद्ध॑रति शृत॒त्वाय॑।
स॒र्पिष्वा᳚न्भवति मेध्य॒त्वाय॑।
च॒त्वार॑ आर्\mbox{}षे॒याः प्राश्ञ॑न्ति।
दि॒शामे॒व ज्योति॑षि जुहोति।
च॒त्वारि॒ हिर॑ण्यानि ददाति।
दि॒शामे॒व ज्योती॒ꣴ॒ष्यव॑ रुन्धे॥४॥

%3.8.2.3
यदाज्य॑मु॒च्छिष्य॑ते।
तस्मि॑न्रश॒नान्यु॑नत्ति।
प्र॒जा\-प॑ति॒र्वा ओ॑द॒नः।
रेत॒ आज्यम्᳚।
यदाज्ये॑ रश॒नान्यु॒नत्ति॑।
प्र॒जा\-प॑तिमे॒व रेत॑सा॒ सम॑र्धयति।
द॒र्भ॒मयी॑ रश॒ना भ॑वति।
ब॒हु वा ए॒ष कु॑च॒रो॑ मे॒ध्यमुप॑गच्छति।
यदश्वः॑।
प॒वित्रं॒ वै द॒र्भाः॥५॥

%3.8.2.4
यद्द॑र्भ॒मयी॑ रश॒ना भव॑ति।
पु॒नात्ये॒वैनम्᳚।
पू॒तमे॑नं॒ मेध्य॒मा ल॑भते।
अश्व॑स्य॒ वा आल॑ब्धस्य महि॒मोद॑क्रामत्।
स म॒हर्त्वि॑जः॒ प्रावि॑शत्।
तन्म॒हर्त्वि॑जां महर्त्वि॒क्त्वम्।
यन्म॒हर्त्वि॑जः प्रा॒श्ञन्ति॑।
म॒हि॒मान॑मे॒वास्मि॒न्तद्द॑धति।
अश्व॑स्य॒ वा आल॑ब्धस्य॒ रेत॒ उद॑क्रामत्।
तथ्सु॒वर्ण॒ꣳ॒ हिर॑ण्यमभवत्।
यथ्सु॒वर्ण॒ꣳ॒ हिर॑ण्यं॒ ददा॑ति।
रेत॑ ए॒व तद्द॑धाति।
ओ॒द॒ने द॑दाति।
रेतो॒ वा ओ॑द॒नः।
रेतो॒ हिर॑ण्यम्।
रेत॑सै॒वास्मि॒न्रेतो॑ दधाति॥६॥\anuvakamend[द॒धा॒ति॒ रु॒न्धे॒ द॒र्भा अ॑भव॒थ्षट् च॑]

%3.8.3.1
यो वै ब्रह्म॑णे दे॒वेभ्यः॑ प्र॒जा\-प॑त॒ये\-ऽप्र॑तिप्रो॒च्याश्वं॒ मेध्यं॑ ब॒ध्नाति॑।
आ दे॒वता᳚भ्यो वृश्च्यते।
पापी॑यान्भवति।
यः प्र॑ति॒प्रोच्य॑।
न दे॒वता᳚भ्य॒ आवृ॑श्च्यते।
वसी॑यान्भवति।
यदाह॑।
ब्रह्म॒न्नश्वं॒ मेध्यं॑ भन्थ्स्यामि दे॒वेभ्यः॑ प्र॒जा\-प॑तये॒ तेन॑ राध्यास॒मिति॑।
ब्रह्म॒ वै ब्र॒ह्मा।
ब्रह्म॑ण ए॒व दे॒वेभ्यः॑ प्र॒जा\-प॑तये प्रति॒प्रोच्याश्वं॒ मेध्यं॑ बध्नाति॥७॥

%3.8.3.2
न दे॒वता᳚भ्य॒ आ वृ॑श्च्यते।
वसी॑यान्भवति।
दे॒वस्य॑ त्वा सवि॒तुः प्र॑स॒व इति॑ रश॒नामाद॑त्ते॒ प्रसू᳚त्यै।
अ॒श्विनो᳚र्बा॒हुभ्या॒मित्या॑ह।
अ॒श्विनौ॒ हि दे॒वाना॑मध्व॒र्यू आस्ता᳚म्।
पू॒ष्णो हस्ता᳚भ्या॒मित्या॑ह॒ यत्यै᳚।
व्यृ॑द्धं॒ वा ए॒तद्य॒ज्ञस्य॑।
यद॑य॒जुष्के॑ण क्रि॒यते᳚।
इ॒माम॑गृभ्णन्रश॒नामृ॒तस्ये\-त्यधि॑ वदति॒ यजु॑ष्कृत्यै।
य॒ज्ञस्य॒ समृ॑द्ध्यै॥८॥

%3.8.3.3
तदा॑हुः।
द्वाद॑शारत्नी रश॒ना क॑र्त॒व्या(३) त्रयो॑दशार॒त्नी(३)\-रिति॑।
ऋ॒ष॒भो वा ए॒ष ऋ॑तू॒नाम्।
यथ्सं॑वथ्स॒रः।
तस्य॑ त्रयोद॒शो मासो॑ वि॒ष्टपम्᳚।
ऋ॒ष॒भ ए॒ष य॒ज्ञाना᳚म्।
यद॑श्वमे॒धः।
यथा॒ वा ऋ॑ष॒भस्य॑ वि॒ष्टपम्᳚।
ए॒वमे॒तस्य॑ वि॒ष्टपम्᳚।
त्र॒यो॒द॒शम॑र॒त्निꣳ र॑श॒नाया॑मु॒पा द॑धाति॥९॥

%3.8.3.4
यथ॑र्\mbox{}ष॒भस्य॑ वि॒ष्टपꣳ॑ सꣴस्क॒रोति॑।
ता॒दृगे॒व तत्।
पूर्व॒ आयु॑षि वि॒दथे॑षु क॒व्येत्या॑ह।
आयु॑रे॒वास्मि॑न्दधाति।
तया॑ दे॒वाः सु॒तमा ब॑भूवु॒रित्या॑ह।
भूति॑मे॒वोपाव॑र्तते।
ऋ॒तस्य॒ साम᳚न्थ्स॒रमा॒रप॒न्तीत्या॑ह।
स॒त्यं वा ऋ॒तम्।
स॒त्येनै॒वैन॑मृ॒तेनार॑भते।
अ॒भि॒धा अ॒सीत्या॑ह॥१०॥

%3.8.3.5
तस्मा॑दश्वमेधया॒जी सर्वा॑णि भू॒तान्य॒भि भ॑वति।
भुव॑नम॒सी\-त्या॑ह।
भू॒मान॑मे॒वोपै॑ति।
य॒न्ता\-ऽसीत्या॑ह।
य॒न्तार॑मे॒वैनं॑ करोति।
ध॒र्ता\-ऽसीत्या॑ह।
ध॒र्तार॑मे॒वैनं॑ करोति।
सो᳚ऽग्निं वै᳚श्वान॒रमित्या॑ह।
अ॒ग्नावे॒वैनं॑ वैश्वान॒रे जु॑होति।
सप्र॑थस॒मित्या॑ह॥११॥

%3.8.3.6
प्र॒जयै॒वैनं॑ प॒शुभिः॑ प्रथयति।
स्वाहा॑कृत॒ इत्या॑ह।
होम॑ ए॒वास्यै॒षः।
पृ॒थि॒व्यामित्या॑ह।
अ॒स्यामे॒वैनं॒ प्रति॑\-ष्ठापयति।
य॒न्ता राड्य॒न्ताऽसि॒ यम॑नो ध॒र्ताऽसि॑ ध॒रुण॒ इत्या॑ह।
रू॒पमे॒वास्यै॒तन्म॑हि॒मानं॒ व्याच॑ष्टे।
कृ॒ष्यै त्वा॒ क्षेमा॑य त्वा र॒य्यै त्वा॒ पोषा॑य॒ त्वेत्या॑ह।
आ॒\-मे॒वैतामा शा᳚स्ते।
स्व॒गा त्वा॑ दे॒वेभ्य॒ इत्या॑ह।
दे॒वेभ्य॑ ए॒वैनꣴ॑ स्व॒गा क॑रोति।
स्वाहा᳚ त्वा प्र॒जा\-प॑तय॒ इत्या॑ह।
प्रा॒जा॒प॒त्यो वा अश्वः॑।
यस्या॑ ए॒व दे॒वता॑या आल॒भ्यते᳚।
तयै॒वैन॒ꣳ॒ सम॑र्धयति॥१२॥\anuvakamend[ब॒ध्ना॒ति॒ समृ॑द्ध्या उ॒पाद॑धात्य॒सीत्या॑ह॒ सप्र॑थस॒मित्या॑ह दे॒वेभ्य॒ इत्या॑ह॒ पञ्च॑ च]

%3.8.4.1
यः पि॒तुर॑नु॒जायाः᳚ पु॒त्रः।
स पु॒रस्ता᳚न्नयति।
यो मा॒तुर॑नु॒जायाः᳚ पु॒त्रः।
स प॒श्चान्न॑यति।
विष्व॑ञ्चमे॒वास्मा᳚त्पा॒प्मानं॒ विवृ॑हतः।
यो अर्व॑न्तं॒ जिघाꣳ॑सति॒ तम॒भ्य॑मीति॒ वरु॑ण॒ इति॒ श्वानं॑ चतुर॒क्षं प्रसौ॑ति।
प॒रो मर्तः॑ प॒रः श्वेति॒ शुन॑श्चतुर॒क्षस्य॒ प्रह॑न्ति।
श्वेव॒ वै पा॒प्मा भ्रातृ॑व्यः।
पा॒प्मान॑मे॒वास्य॒ भ्रातृ॑व्यꣳ हन्ति।
सै॒ध्र॒कं मुस॑लं भवति॥१३॥

%3.8.4.2
कर्म॑कर्मै॒वास्मै॑ साधयति।
पौ॒ꣴ॒श्च॒ले॒यो ह॑न्ति।
पु॒ꣴ॒श्च॒ल्वां वै दे॒वाः शुचं॒ न्य॑दधुः।
शु॒चैवास्य॒ शुचꣳ॑ हन्ति।
पा॒प्मा वा ए॒तमी᳚फ्स॒तीत्या॑हुः।
यो᳚ऽश्वमे॒धेन॒ यज॑त॒ इति॑।
अश्व॑स्याधस्प॒दमु\-पा᳚स्यति।
व॒ज्री वा अश्वः॑ प्राजाप॒त्यः।
वज्रे॑णै॒व पा॒प्मानं॒ भ्रातृ॑व्य॒मव॑\-क्रामति।
द॒क्षि॒णाऽप॑ प्लावयति॥१४॥

%3.8.4.3
पा॒प्मान॑मे॒वास्मा॒च्छम॑ल॒मप॑ प्लावयति।
ऐ॒षी॒क उ॑दू॒हो भ॑वति।
आयु॒र्वा इ॒षीकाः᳚।
आयु॑रे॒वास्मि॑न्दधति।
अ॒मृतं॒ वा इ॒षीकाः᳚।
अ॒मृत॑मे॒वास्मि॑न्दधति।
वे॒त॒स॒शा॒खोप॒सम्ब॑द्धा भवति।
अ॒फ्सुयो॑नि॒र्वा अश्वः॑।
अ॒फ्सु॒जो वे॑त॒सः।
स्वादे॒वैनं॒ योने॒र्निर्मि॑मीते।
पु॒रस्ता᳚त्प्र॒त्यञ्च॑म॒भ्युदू॑हति।
पु॒रस्ता॑दे॒वास्मि॑न्प्र॒तीच्य॒मृतं॑ दधाति।
अ॒हं च॒ त्वं च॑ वृत्रह॒न्निति॑ ब्र॒ह्मा यज॑मानस्य॒ हस्तं॑ गृह्णाति।
ब्र॒ह्म॒क्ष॒त्रे ए॒व सन्द॑धाति।
अ॒भिक्रत्वे᳚न्द्र भू॒रध॒ज्मन्नित्य॑ध्व॒र्युर्यज॑मानं वाचयत्य॒भिजि॑त्यै॥१५॥\anuvakamend[भ॒व॒ति॒ प्ला॒व॒य॒ति॒ मि॒मी॒ते॒ पञ्च॑ च]

%3.8.5.1
च॒त्वार॑ ऋ॒त्विजः॒ समु॑क्षन्ति।
आ॒भ्य ए॒वैनं॑ चत॒सृभ्यो॑ दि॒ग्भ्यो॑\-ऽभि समी॑रयन्ति।
श॒तेन॑ राजपु॒त्रैः स॒हाध्व॒र्युः।
पु॒रस्ता᳚त्\-प्र॒त्यङ्तिष्ठ॒न्प्रोक्ष॑ति।
अ॒नेनाश्वे॑न॒ मेध्ये॑ने॒ष्ट्वा।
अ॒यꣳ राजा॑ वृ॒त्रं व॑ध्या॒दिति॑।
रा॒ज्यं वा अ॑ध्व॒र्युः।
क्ष॒त्रꣳ रा॑जपु॒त्रः।
रा॒ज्ये\-नै॒वास्मि॑न्क्ष॒त्रं द॑धाति।
श॒तेना॑रा॒जभि॑रु॒ग्रैः स॒ह ब्र॒ह्मा॥१६॥

%3.8.5.2
द॒क्षि॒ण॒त उद॒ङ्तिष्ठ॒न्प्रोक्ष॑ति।
अ॒नेनाश्वे॑न॒ मेध्ये॑ने॒ष्ट्वा।
अ॒यꣳ राजा᳚\-ऽप्रतिधृ॒ष्यो᳚\-ऽस्त्विति॑।
बलं॒ वै ब्र॒ह्मा।
बल॑मरा॒जोग्रः।
बले॑\-नै॒वा\-स्मि॒न्बलं॑ दधाति।
श॒तेन॑ सूतग्राम॒णिभिः॑ स॒ह होता᳚।
प॒श्चात्प्राङ्तिष्ठ॒न्प्रोक्ष॑ति।
अ॒नेनाश्वे॑न॒ मेध्ये॑ने॒ष्ट्वा।
अ॒यꣳ राजा॒\-ऽस्यै वि॒शः॥१७॥

%3.8.5.3
ब॒हु॒ग्वै ब॑ह्व॒श्वायै॑ बह्वजावि॒कायै᳚।
ब॒हु॒व्री॒हि॒य॒वायै॑ बहुमाष\-ति॒लायै᳚।
ब॒हु॒हि॒र॒ण्यायै॑ बहुह॒स्तिका॑यै।
ब॒हु॒दा॒स॒\-पू॒रु॒षायै॑ रयि॒मत्यै॒ पुष्टि॑मत्यै।
ब॒हु॒रा॒य॒स्पो॒षायै॒ राजा॒स्त्विति॑।
भू॒मा वै होता᳚।
भू॒मा सू॑तग्राम॒ण्यः॑।
भू॒म्नैवास्मि॑न्भू॒मानं॑ दधाति।
श॒तेन॑ क्षत्तसङ्ग्रही॒तृभिः॑ स॒होद्गा॒ता।
उ॒त्त॒र॒तो द॑क्षि॒णा तिष्ठ॒न्प्रोक्ष॑ति॥१८॥

%3.8.5.4
अ॒नेनाश्वे॑न॒ मेध्ये॑ने॒ष्ट्वा।
अ॒यꣳ राजा॒ सर्व॒मायु॑रे॒त्विति॑।
आयु॒र्वा उ॑द्गा॒ता।
आयुः॑ क्षत्तसङ्ग्रही॒तारः॑।
आयु॑षै॒वास्मि॒न्नायु॑र्द\-धाति।
श॒तꣳ\-श॑तं भवन्ति।
श॒तायुः॒ पुरु॑षः श॒तेन्द्रि॑यः।
आयु॑ष्ये॒वेन्द्रि॒ये प्रति॑ तिष्ठति।
च॒तुः॒ श॒ता भ॑वन्ति।
चत॑स्रो॒ दिशः॑।
दि॒क्ष्वे॑व प्रति॑ तिष्ठति॥१९॥\anuvakamend[ब्र॒ह्मा वि॒श उ॑क्षति॒ दिश॒ एकं॑ च]

%3.8.6.1
यथा॒ वै ह॒विषो॑ गृही॒तस्य॒ स्कन्द॑ति।
ए॒वं वा ए॒तदश्व॑स्य स्कन्दति।
यन्नि॒क्तमना॑लब्धमुथ्सृ॒जन्ति॑।
यथ्स्तोक्या॑ अ॒न्वाह॑।
स॒र्व॒हुत॑मे॒वैनं॑ करो॒त्यस्क॑न्दाय।
अस्क॑न्न॒ꣳ॒ हि तत्।
यद्धु॒तस्य॒ स्कन्द॑ति।
स॒हस्र॒मन्वा॑ह।
स॒हस्र॑सम्मितः सुव॒र्गो लो॒कः।
सु॒व॒र्गस्य॑ लो॒कस्या॒भिजि॑त्यै॥२०॥

%3.8.6.2
यत्परि॑मिता अनुब्रू॒यात्।
परि॑मित॒मव॑ रुन्धीत।
अप॑रिमिता॒ अन्वा॑ह।
अप॑रिमितः सुव॒र्गो लो॒कः।
सु॒व॒र्गस्य॑ लो॒कस्य॒ सम॑ष्ट्यै।
स्तोक्या॑ जुहोति।
या ए॒व वर्ष्या॒ आपः॑।
ता अव॑ रुन्धे।
अ॒स्यां जु॑होति।
इ॒यं वा अ॒ग्निर्वै᳚श्वान॒रः॥२१॥

%3.8.6.3
अ॒स्यामे॒वैनाः॒ प्रति॑\-ष्ठापयति।
उ॒वाच॑ ह प्र॒जा\-प॑तिः।
स्तोक्या॑सु॒ वा अ॒हम॑श्वमे॒धꣳ सꣴस्था॑पयामि।
तेन॒ ततः॒ सꣴस्थि॑तेन चरा॒मीति॑।
अ॒ग्नये॒ स्वाहेत्या॑ह।
अ॒ग्नय॑ ए॒वैनं॑ जुहोति।
सोमा॑य॒ स्वाहेत्या॑ह।
सोमा॑यै॒वैनं॑ जुहोति।
स॒वि॒त्रे स्वाहेत्या॑ह।
स॒वि॒त्र ए॒वैनं॑ जुहोति॥२२॥

%3.8.6.4
सर॑स्वत्यै॒ स्वाहेत्या॑ह।
सर॑स्वत्या ए॒वैनं॑ जुहोति।
पू॒ष्णे स्वाहेत्या॑ह।
पू॒ष्ण ए॒वैनं॑ जुहोति।
बृह॒स्पत॑ये॒ स्वाहेत्या॑ह।
बृह॒स्पत॑य ए॒वैनं॑ जुहोति।
अ॒पां मोदा॑य॒ स्वाहेत्या॑ह।
अ॒द्भ्य ए॒वैनं॑ जुहोति।
वा॒यवे॒ स्वाहेत्या॑ह।
वा॒यव॑ ए॒वैनं॑ जुहोति॥२३॥

%3.8.6.5
मि॒त्राय॒ स्वाहेत्या॑ह।
मि॒त्रायै॒वैनं॑ जुहोति।
वरु॑णाय॒ स्वाहेत्या॑ह।
वरु॑णायै॒वैनं॑ जुहोति।
ए॒ताभ्य॑ ए॒वैनं॑ दे॒वता᳚भ्यो जुहोति।
दश॑दश स॒म्पादं॑ जुहोति।
दशा᳚क्षरा वि॒राट्।
अन्नं॑ वि॒राट्।
वि॒राजै॒वान्नाद्य॒मव॑ रुन्धे।
प्र वा ए॒षो᳚\-ऽस्माल्लो॒काच्च्य॑वते।
यः परा॑ची॒राहु॑तीर्जु॒होति॑।
पुनः॑ पुनरभ्या॒वर्तं॑ जुहोति।
अ॒स्मिन्ने॒व लो॒के प्रति॑ तिष्ठति।
ए॒ताꣳ ह॒ वाव सो᳚\-ऽश्वमे॒धस्य॒ सꣴस्थि॑तिमुवा॒चास्क॑न्दाय।
अस्क॑न्न॒ꣳ॒ हि तत्।
यद्य॒ज्ञस्य॒ सꣴस्थि॑तस्य॒ स्कन्द॑ति॥२४॥\anuvakamend[अ॒भिजि॑त्यै वैश्वान॒रः स॑वि॒त्र ए॒वैनं॑ जुहोति वा॒यव॑ ए॒वैनं॑ जुहोति च्यवते॒ षट् च॑]

%3.8.7.1
प्र॒जा\-प॑तये त्वा॒ जुष्टं॒ प्रोक्षा॒मीति॑ पु॒रस्ता᳚त्प्र॒त्यङ्तिष्ठ॒न्प्रोक्ष॑ति।
प्र॒जा\-प॑ति॒र्वै दे॒वाना॑मन्ना॒दो वी॒र्या॑वान्।
अ॒न्नाद्य॑मे॒वास्मि॑न्वी॒र्यं॑ दधाति।
तस्मा॒दश्वः॑ पशू॒नाम॑न्ना॒दो वी॒र्या॑वत्तमः।
इ॒न्द्रा॒ग्निभ्यां॒ त्वेति॑ दक्षिण॒तः।
इ॒न्द्रा॒ग्नी वै दे॒वाना॒मोजि॑ष्ठौ॒ बलि॑ष्ठौ।
ओज॑ ए॒वास्मि॒न्बलं॑ दधाति।
तस्मा॒दश्वः॑ पशू॒नामोजि॑ष्ठो॒ बलि॑ष्ठः।
वा॒यवे॒ त्वेति॑ प॒श्चात्।
वा॒युर्वै दे॒वाना॑मा॒शुः सा॑रसा॒रित॑मः॥२५॥

%3.8.7.2
ज॒वमे॒वास्मि॑न्दधाति।
तस्मा॒दश्वः॑ पशू॒नामा॒शुः सा॑रसा॒रित॑मः।
विश्वे᳚भ्यस्त्वा दे॒वेभ्य॒ इत्यु॑त्तर॒तः।
विश्वे॒ वै दे॒वा दे॒वानां᳚ यश॒स्वित॑माः।
यश॑ ए॒वास्मि॑न्दधाति।
तस्मा॒दश्वः॑ पशू॒नां य॑श॒स्वित॑मः।
दे॒वेभ्य॒स्त्वेत्य॒धस्ता᳚त्।
दे॒वा वै दे॒वाना॒मप॑चिततमाः।
अप॑चितिमे॒वास्मि॑न्दधाति।
तस्मा॒दश्वः॑ पशू॒नामप॑चिततमः॥२६॥

%3.8.7.3
सर्वे᳚भ्यस्त्वा दे॒वेभ्य॒ इत्यु॒परि॑ष्टात्।
सर्वे॒ वै दे॒वास्त्विषि॑मन्तो हर॒स्विनः॑।
त्विषि॑मे॒वास्मि॒न्॒ हरो॑ दधाति।
तस्मा॒दश्वः॑ पशू॒नां त्विषि॑मान्‌ हर॒स्वित॑मः।
दि॒वे त्वा॒\-ऽन्तरि॑क्षाय त्वा पृथि॒व्यै त्वेत्या॑ह।
ए॒भ्य ए॒वैनं॑ लो॒केभ्यः॒ प्रोक्ष॑ति।
स॒ते त्वा\-ऽस॑ते त्वा॒\-ऽद्भ्यस्त्वौष॑धीभ्यस्त्वा॒ विश्वे᳚भ्यस्त्वा भू॒तेभ्य॒ इत्या॑ह।
तस्मा॑दश्वमेधया॒जिन॒ꣳ॒ सर्वा॑णि भू॒तान्युप॑जीवन्ति।
ब्र॒ह्म॒वा॒दिनो॑ वदन्ति।
यत्प्रा॑जाप॒त्यो\-ऽश्वः॑।
अथ॒ कस्मा॑देनम॒न्याभ्यो॑ दे॒वता॒भ्योऽपि॒ प्रोक्ष॒तीति॑।
अश्वे॒ वै सर्वा॑ दे॒वता॑ अ॒न्वाय॑त्ताः।
तं यद्विश्वे᳚भ्यस्त्वा भू॒तेभ्य॒ इति॑ प्रो॒क्षति॑।
दे॒वता॑ ए॒वास्मि॑न्न॒न्वा या॑तयति।
तस्मा॒दश्वे॒ सर्वा॑ दे॒वता॑ अ॒न्वाय॑त्ताः॥२७॥\anuvakamend[सा॒र॒सा॒रित॒मो\-ऽप॑चिततमः प्राजाप॒त्यो\-ऽश्वः॒ पञ्च॑ च]

%3.8.8.1
यथा॒ वै ह॒विषो॑ गृही॒तस्य॒ स्कन्द॑ति।
ए॒वं वा ए॒तदश्व॑स्य स्कन्दति।
यत्प्रोक्षि॑त॒मना॑लब्धमुथ्सृ॒जन्ति॑।
यद॑श्वचरि॒तानि॑ जु॒होति॑।
स॒र्व॒हुत॑मे॒वैनं॑ करो॒त्यस्क॑न्दाय।
अस्क॑न्न॒ꣳ॒ हि तत्।
यद्धु॒तस्य॒ स्कन्द॑ति।
ई॒ङ्का॒राय॒ स्वाहेङ्कृ॑ताय॒ स्वाहेत्या॑ह।
ए॒तानि॒ वा अ॑श्वचरि॒तानि॑।
च॒रि॒तैरे॒वैन॒ꣳ॒ सम॑र्धयति॥२८॥

%3.8.8.2
तदा॑हुः।
अना॑हुतयो॒ वा अ॑श्वचरि॒तानि॑।
नैता हो॑त॒व्या॑ इति॑।
अथो॒ खल्वा॑हुः।
हो॒त॒व्या॑ ए॒व।
अत्र॒ वावैवं वि॒द्वान॑श्वमे॒धꣳ सꣴस्था॑पयति।
यद॑श्वचरि॒तानि॑ जु॒होति॑।
तस्मा᳚द्धोत॒व्या॑ इति॑।
ब॒हि॒र्धा वा ए॑नमे॒तदा॒यत॑नाद्दधाति।
भ्रातृ॑व्यमस्मै जनयति॥२९॥

%3.8.8.3
यस्या॑नायत॒ने᳚\-ऽन्यत्रा॒ग्नेराहु॑तीर्जु॒होति॑।
सा॒वि॒त्रि॒या इष्ट्याः᳚ पु॒रस्ता᳚थ्स्विष्ट॒कृतः॑।
आ॒ह॒व॒नीये᳚\-ऽश्वचरि॒तानि॑ जुहोति।
आ॒यत॑न ए॒वास्याऽऽहु॑तीर्जुहोति।
नास्मै॒ भ्रातृ॑व्यं जनयति।
तदा॑हुः।
य॒ज्ञ॒\-मु॒खेय॑ज्ञमुखे होत॒व्याः᳚।
य॒ज्ञस्य॒ कॢप्त्यै᳚।
सु॒व॒र्गस्य॑ लो॒कस्यानु॑ख्यात्या॒ इति॑।
अथो॒ खल्वा॑हुः॥३०॥

%3.8.8.4
यद्य॑ज्ञमु॒खेय॑ज्ञमुखे जुहु॒यात्।
प॒शुभि॒र्यज॑मानं॒ व्य॑र्धयेत्।
अव॑ सुव॒र्गाल्लो॒कात्प॑द्येत।
पापी॑यान्थ्स्या॒दिति॑।
स॒कृदे॒व हो॑त॒व्याः᳚।
न यज॑मानं प॒शुभि॒र्व्य॑र्धयति।
अ॒भि सु॑व॒र्गं लो॒कं ज॑यति।
न पापी॑यान्भवति।
अ॒ष्टाच॑त्वारिꣳशतमश्वरू॒पाणि॑ जुहोति।
अ॒ष्टाच॑त्वारिꣳशदक्षरा॒ जग॑ती।
जाग॒तो\-ऽश्वः॑ प्राजाप॒त्यः समृ॑द्ध्यै।
एक॒मति॑रिक्तं जुहोति।
तस्मा॒देकः॑ प्र॒जास्वर्धु॑कः॥३१॥\anuvakamend[अ॒र्ध॒य॒ति॒ ज॒न॒य॒ति॒ खल्वा॑हु॒र्जग॑ती॒ त्रीणि॑ च]

%3.8.9.1
वि॒भूर्मा॒त्रा प्र॒भूः पि॒त्रेत्या॑ह।
इ॒यं वै मा॒ता।
अ॒सौ पि॒ता।
आ॒भ्यामे॒वैनं॒ परि॑ददाति।
अश्वो॑ऽसि॒ हयो॒\-ऽसीत्या॑ह।
शास्त्ये॒वैन॑मे॒तत्।
तस्मा᳚च्छि॒ष्टाः प्र॒जा जा॑यन्ते।
अत्यो॒\-ऽसीत्या॑ह।
तस्मा॒दश्वः॒ सर्वा᳚न्प॒शूनत्ये॑ति।
तस्मा॒दश्वः॒ सर्वे॑षां पशू॒नाꣴ श्रैष्ठ्यं॑ गच्छति॥३२॥

%3.8.9.2
प्र यशः॒ श्रैष्ठ्य॑माप्नोति।
य ए॒वं वेद॑।
नरो॒ऽस्यर्वा॑ऽसि॒ सप्ति॑रसि वा॒ज्य॑सीत्या॑ह।
रू॒पमे॒वास्यै॒तन्म॑हि॒मानं॒ व्याच॑ष्टे।
ययु॒र्नामा॒सीत्या॑ह।
ए॒तद्वा अश्व॑स्य प्रि॒यं ना॑म॒धेयम्᳚।
प्रि॒येणै॒वैनं॑ नाम॒धेये॑ना॒भि व॑दति।
तस्मा॒दप्या॑मि॒त्रौ स॒ङ्गत्य॑।
नाम्ना॒ चेद्‌ध्वये॑ते।
मि॒त्रमे॒व भ॑वतः॥३३॥

%3.8.9.3
आ॒दि॒त्यानां॒ पत्वा\-ऽन्वि॒हीत्या॑ह।
आ॒दि॒त्याने॒वैनं॑ गमयति।
अ॒ग्नये॒ स्वाहा॒ स्वाहे᳚न्द्रा॒ग्निभ्या॒मिति॑ पूर्वहो॒मां जु॑होति।
पूर्व॑ ए॒व द्वि॒षन्तं॒ भ्रातृ॑व्य॒मति॑ क्रामति।
भूर॑सि भु॒वे त्वा॒ भव्या॑य त्वा भविष्य॒ते त्वेत्युथ्सृ॑जति सर्व॒त्वाय॑।
देवा॑ आशापाला ए॒तं दे॒वेभ्यो\-ऽश्वं॒ मेधा॑य॒ प्रोक्षि॑तं गोपाय॒तेत्या॑ह।
श॒तं वै तल्प्या॑ राजपु॒त्रा दे॒वा आ॑शापा॒लाः।
तेभ्य॑ ए॒वैनं॒ परि॑ ददाति।
ई॒श्व॒रो वा अश्वः॒ प्रमु॑क्तः॒ परां᳚ परा॒वतं॒ गन्तोः᳚।
इ॒ह धृतिः॒ स्वाहे॒ह विधृ॑तिः॒ स्वाहे॒ह रन्तिः॒ स्वाहे॒ह रम॑तिः॒ स्वाहेति॑ चतृ॒षु प॒थ्सु जु॑होति॥३४॥

%3.8.9.4
ए॒ता वा अश्व॑स्य॒ बन्ध॑नम्।
ताभि॑रे॒वैनं॑ बध्नाति।
तस्मा॒दश्वः॒ प्रमु॑क्तो॒ बन्ध॑न॒मा ग॑च्छति।
तस्मा॒दश्वः॒ प्रमु॑क्तो॒ बन्ध॑नं॒ न ज॑हाति।
रा॒ष्ट्रं वा अ॑श्वमे॒धः।
रा॒ष्ट्रे खलु॒ वा ए॒ते व्याय॑च्छन्ते।
येऽश्वं॒ मेध्य॒ꣳ॒ रक्ष॑न्ति।
तेषां॒ य उ॒दृचं॒ गच्छ॑न्ति।
रा॒ष्ट्रादे॒व ते रा॒ष्ट्रं ग॑च्छन्ति।
अथ॒ य उ॒दृचं॒ न गच्छ॑न्ति॥३५॥

%3.8.9.5
रा॒ष्ट्रादे॒व ते व्यव॑च्छिद्यन्ते।
परा॒ वा ए॒ष सि॑च्यते।
यो॑ऽब॒लो᳚\-ऽश्वमे॒धेन॒ यज॑ते।
यद॒मित्रा॒ अश्वं॑ वि॒न्देरन्॑।
ह॒न्येता᳚स्य य॒ज्ञः।
च॒तुः॒ श॒ता र॑क्षन्ति।
य॒ज्ञस्याघा॑ताय।
अथा॒न्यमा॒नीय॒ प्रोक्षे॑युः।
सैव ततः॒ प्राय॑श्चित्तिः॥३६॥\anuvakamend[ग॒च्छ॒ति॒ भ॒व॒तः॒ प॒थ्सु जु॑होति॒ न गच्छ॑न्ति॒ नव॑ च]

%3.8.10.1
प्र॒जा\-प॑तिरकामयताश्वमे॒धेन॑ यजे॒येति॑।
स तपो॑\-ऽतप्यत।
तस्य॑ तेपा॒नस्य॑।
स॒प्तात्मनो॑ दे॒वता॒ उद॑क्रामन्।
सा दी॒क्षा\-ऽभ॑वत्।
स ए॒तानि॑ वैश्वदे॒वान्य॑पश्यत्।
तान्य॑जुहोत्।
तैर्वै स दी॒क्षामवा॑रुन्ध।
यद्वै᳚श्वदे॒वानि॑ जु॒होति॑।
दी॒क्षामे॒व तैर्यज॑मा॒नोऽव॑ रुन्धे॥३७॥

%3.8.10.2
स॒प्त जु॑होति।
स॒प्त हि ता दे॒वता॑ उ॒दक्रा॑मन्।
अ॒न्व॒हं जु॑होति।
अ॒न्व॒हमे॒व दी॒क्षामव॑ रुन्धे।
त्रीणि॑ वैश्वदे॒वानि॑ जुहोति।
च॒त्वार्यौ᳚द्ग्रह॒णानि॑।
स॒प्त सम्प॑द्यन्ते।
स॒प्त वै शी॑र्‌\mbox{}ष॒ण्याः᳚ प्रा॒णाः।
प्रा॒णा दी॒क्षा।
प्रा॒णैरे॒व प्रा॒णान्दी॒क्षामव॑ रुन्धे॥३८॥

%3.8.10.3
एक॑विꣳशतिं वैश्वदे॒वानि॑ जुहोति।
एक॑विꣳशति॒र्वै दे॑वलो॒काः।
द्वाद॑श॒ मासाः॒ पञ्च॒र्तवः॑।
त्रय॑ इ॒मे लो॒काः।
अ॒सावा॑दि॒त्य ए॑कवि॒ꣳ॒शः।
ए॒ष सु॑व॒र्गो लो॒कः।
तद्दैव्यं॑ क्ष॒त्रम्।
सा श्रीः।
तद्ब्र॒ध्नस्य॑ वि॒ष्टपम्᳚।
तथ्स्वारा᳚ज्यमुच्यते॥३९॥

%3.8.10.4
त्रि॒ꣳ॒शत॑मौद्ग्रह॒णानि॑ जुहोति।
त्रि॒ꣳ॒शद॑क्षरा वि॒राट्।
अन्नं॑ वि॒राट्।
वि॒राजै॒वान्नाद्य॒मव॑ रुन्धे।
त्रे॒धा वि॒भज्य॑ दे॒वतां᳚ जुहोति।
त्र्या॑वृतो॒ वै दे॒वाः।
त्र्या॑वृत इ॒मे लो॒काः।
ए॒षां लो॒काना॒माप्त्यै᳚।
ए॒षां लो॒कानां॒ कॢप्त्यै᳚।
अप॒ वा ए॒तस्मा᳚त्प्रा॒णाः क्रा॑मन्ति॥४०॥

%3.8.10.5
यो दी॒क्षाम॑तिरे॒चय॑ति।
स॒प्ता॒हं प्रच॑रन्ति।
स॒प्त वै शी॑र्\mbox{}ष॒ण्याः᳚ प्रा॒णाः।
प्रा॒णा दी॒क्षा।
प्रा॒णैरे॒व प्रा॒णान्दी॒क्षामव॑ रुन्धे।
पू॒र्णा॒हु॒तिमु॑त्त॒मां जु॑होति।
सर्वं॒ वै पू᳚र्णाहु॒तिः।
सर्व॑\-मे॒वा\-प्नो॑ति।
अथो॑ इ॒यं वै पू᳚र्णाहु॒तिः।
अ॒स्यामे॒व प्रति॑ तिष्ठति॥४१॥\anuvakamend[रु॒न्धे॒ प्रा॒णान्दी॒क्षामव॑ रुन्ध उच्यते क्रामन्ति तिष्ठति]

%3.8.11.1
प्र॒जा\-प॑तिरश्वमे॒धम॑\-सृजत।
तꣳ सृ॒ष्टं न किञ्च॒नोद॑यच्छत्।
तं वै᳚श्वदे॒वान्ये॒वोद॑यच्छन्।
यद्वै᳚श्वदे॒वानि॑ जु॒होति॑।
य॒ज्ञस्योद्य॑त्यै।
स्वाहा॒\-ऽऽधिमाधी॑ताय॒ स्वाहा᳚।
स्वाहा\-ऽधी॑तं॒ मन॑से॒ स्वाहा᳚।
स्वाहा॒ मनः॑ प्र॒जा\-प॑तये॒ स्वाहा᳚।
काय॒ स्वाहा॒ कस्मै॒ स्वाहा॑ कत॒मस्मै॒ स्वाहेति॑ प्राजाप॒त्ये मुख्ये॑ भवतः।
प्र॒जा\-प॑तिमुखाभिरे॒वैनं॑ दे॒वता॑भि॒रुद्य॑च्छते॥४२॥

%3.8.11.2
अदि॑त्यै॒ स्वाहा\-ऽदि॑त्यै म॒ह्यै᳚ स्वाहा\-ऽदि॑त्यै सुमृडी॒कायै॒ स्वाहेत्या॑ह।
इ॒यं वा अदि॑तिः।
अ॒स्या ए॒वैनं॑ प्रति॒ष्ठायोद्य॑च्छते।
सर॑स्वत्यै॒ स्वाहा॒ सर॑स्वत्यै बृह॒त्यै᳚ स्वाहा॒ सर॑स्वत्यै पाव॒कायै॒ स्वाहेत्या॑ह।
वाग्वै सर॑स्वती।
वा॒चैवैन॒मुद्य॑च्छते।
पू॒ष्णे स्वाहा॑ पू॒ष्णे प्र॑प॒थ्या॑य॒ स्वाहा॑ पू॒ष्णे न॒रन्धि॑षाय॒ स्वाहेत्या॑ह।
प॒शवो॒ वै पू॒षा।
प॒शुभि॑रे॒वैन॒मुद्य॑च्छते।
त्वष्ट्रे॒ स्वाहा॒ त्वष्ट्रे॑ तु॒रीपा॑य॒ स्वाहा॒ त्वष्ट्रे॑ पुरु॒रूपा॑य॒ स्वाहेत्या॑ह।
त्वष्टा॒ वै प॑शू॒नां मि॑थु॒नानाꣳ॑ रूप॒कृत्।
रू॒पमे॒व प॒शुषु॑ दधाति।
अथो॑ रू॒पैरे॒वैन॒मुद्य॑च्छते।
विष्ण॑वे॒ स्वाहा॒ विष्ण॑वे निखुर्य॒पाय॒ स्वाहा॒ विष्ण॑वे निभूय॒पाय॒ स्वाहेत्या॑ह।
य॒ज्ञो वै विष्णुः॑।
य॒ज्ञायै॒वैन॒मुद्य॑च्छते।
पू॒र्णा॒हु॒तिमु॑त्त॒मां जु॑होति।
प्रत्युत्त॑ब्ध्यै सय॒त्वाय॑॥४३॥\anuvakamend[य॒च्छ॒ते॒ पु॒रु॒रूपा॑य॒ स्वाहेत्या॑हा॒ष्टौ च॑]

%3.8.12.1
सा॒वि॒त्रम॒ष्टा\-क॑पालं प्रा॒तर्निर्व॑पति।
अ॒ष्टाक्ष॑रा गाय॒त्री।
गा॒य॒त्रं प्रा॑तः सव॒नम्।
प्रा॒तः॒ स॒व॒नादे॒वैनं॑ गायत्रि॒याश्छन्द॒सो\-ऽधि॒ निर्मि॑मीते।
अथो᳚ प्रातः सव॒नमे॒व तेना᳚ऽऽप्नोति।
गा॒य॒त्रीं छन्दः॑।
स॒वि॒त्रे प्र॑सवि॒त्र एका॑\-दश\-कपालं म॒ध्यन्दि॑ने।
एका॑दशाक्षरा त्रि॒ष्टुप्।
त्रैष्टु॑भं॒ माध्यं॑ दिन॒ꣳ॒ सव॑नम्।
माध्यं॑ दिनादे॒वैन॒ꣳ॒ सव॑नात्त्रि॒ष्टुभ॒श्छन्द॒सोऽधि॒ निर्मि॑मीते॥४४॥

%3.8.12.2
अथो॒ माध्यं॑ दिनमे॒व सव॑नं॒ तेना᳚ऽऽप्नोति।
त्रि॒ष्टुभं॒ छन्दः॑।
स॒वि॒त्र आ॑सवि॒त्रे द्वाद॑शकपालमपरा॒ह्णे।
द्वाद॑शाक्षरा॒ जग॑ती।
जाग॑तं तृतीयसव॒नम्।
तृ॒ती॒य॒स॒व॒नादे॒वैनं॒ जग॑त्या॒श्छन्द॒सोऽधि॒ निर्मि॑मीते।
अथो॑ तृतीयसव॒नमे॒व तेना᳚ऽऽप्नोति।
जग॑तीं॒ छन्दः॑।
ई॒श्व॒रो वा अश्वः॒ प्रमु॑क्तः॒ परां᳚ परा॒वतं॒ गन्तोः᳚।
इ॒ह धृतिः॒ स्वाहे॒ह विधृ॑तिः॒ स्वाहे॒ह रन्तिः॒ स्वाहे॒ह रम॑तिः॒ स्वाहेति॒ चत॑स्र॒ आहु॑तीर्जुहोति॥४५॥

%3.8.12.3
चत॑स्रो॒ दिशः॑।
दि॒ग्भिरे॒वैनं॒ परि॑गृह्णाति।
आश्व॑त्थो व्र॒जो भ॑वति।
प्र॒जा\-प॑तिर्दे॒वेभ्यो॒ निला॑यत।
अश्वो॑ रू॒पं कृ॒त्वा।
सो᳚ऽश्व॒त्थे सं॑वथ्स॒रम॑तिष्ठत्।
तद॑श्व॒त्थस्या᳚श्वत्थ॒त्वम्।
यदाश्व॑त्थो व्र॒जो भव॑ति।
स्व ए॒वैनं॒ योनौ॒ प्रति॑\-ष्ठापयति॥४६॥\anuvakamend[त्रि॒ष्टुभ॒श्छन्द॒सोऽधि॒ निर्मि॑मीते जुहोति॒ नव॑ च]

%3.8.13.1
आ ब्रह्म॑न्ब्राह्म॒णो ब्र॑ह्म\-वर्च॒सी जा॑यता॒मित्या॑ह।
ब्रा॒ह्म॒ण ए॒व ब्र॑ह्म\-वर्च॒सं द॑धाति।
तस्मा᳚त्पु॒रा ब्रा᳚ह्म॒णो ब्र॑ह्म\-वर्च॒स्य॑जायत।
आऽस्मिन्रा॒ष्ट्रे रा॑ज॒न्य॑ इष॒व्यः॑ शूरो॑ महार॒थो जा॑यता॒मित्या॑ह।
रा॒ज॒न्य॑ ए॒व शौ॒र्यं म॑हि॒मानं॑ दधाति।
तस्मा᳚त्पु॒रा रा॑ज॒न्य॑ इष॒व्यः॑ शूरो॑ महार॒थो॑\-ऽजायत।
दोग्ध्री॑ धे॒नुरित्या॑ह।
धे॒न्वामे॒व पयो॑ दधाति।
तस्मा᳚त्पु॒रा दोग्ध्री॑ धे॒नुर॑जायत।
वोढा॑\-ऽन॒ड्वानित्या॑ह॥४७॥

%3.8.13.2
अ॒न॒डुह्ये॒व वी॒र्यं॑ दधाति।
तस्मा᳚त्पु॒रा वोढा॑\-ऽन॒ड्वान॑जायत।
आ॒शुः सप्ति॒रित्या॑ह।
अश्व॑ ए॒व ज॒वं द॑धाति।
तस्मा᳚त्पु॒रा\-ऽऽशुरश्वो॑\-ऽजायत।
पुर॑न्धि॒र्योषेत्या॑ह।
यो॒षित्ये॒व रू॒पं द॑धाति।
तस्मा॒थ्स्त्री यु॑व॒तिः प्रि॒या भावु॑का।
जि॒ष्णू र॑थे॒ष्ठा इत्या॑ह।
आ ह॒ वै तत्र॑ जि॒ष्णू र॑थे॒ष्ठा जा॑यते॥४८॥

%3.8.13.3
यत्रै॒तेन॑ य॒ज्ञेन॒ यज॑न्ते।
स॒भेयो॒ युवेत्या॑ह।
यो वै पू᳚र्ववय॒सी।
स स॒भेयो॒ युवा᳚।
तस्मा॒द्युवा॒ पुमा᳚न्प्रि॒यो भावु॑कः।
आऽस्य यज॑मानस्य वी॒रो जा॑यता॒मित्या॑ह।
आ ह॒ वै तत्र॒ यज॑मानस्य वी॒रो जा॑यते।
यत्रै॒तेन॑ य॒ज्ञेन॒ यज॑न्ते।
नि॒का॒मेनि॑कामे नः प॒र्जन्यो॑ वर्\mbox{}ष॒त्वित्या॑ह।
नि॒का॒मेनि॑कामे ह॒ वै तत्र॑ प॒र्जन्यो॑ वर्\mbox{}षति।
यत्रै॒तेन॑ य॒ज्ञेन॒ यज॑न्ते।
फ॒लिन्यो॑ न॒ ओष॑धयः पच्यन्ता॒मित्या॑ह।
फ॒लिन्यो॑ ह॒ वै तत्रौष॑धयः पच्यन्ते।
यत्रै॒तेन॑ य॒ज्ञेन॒ यज॑न्ते।
यो॒ग॒क्षे॒मो नः॑ कल्पता॒मित्या॑ह।
कल्प॑ते ह॒ वै तत्र॑ प्र॒जाभ्यो॑ योगक्षे॒मः।
यत्रै॒तेन॑ य॒ज्ञेन॒ यज॑न्ते॥४९॥\anuvakamend[अ॒न॒ड्वानित्या॑ह जायते वर्‌\mbox{}षति स॒प्त च॑]

%3.8.14.1
प्र॒जा\-प॑तिर्दे॒वेभ्यो॑ य॒ज्ञान्व्यादि॑शत्।
स आ॒त्मन्न॑श्वमे॒धम॑धत्त।
तं दे॒वा अ॑ब्रुवन्।
ए॒ष वाव य॒ज्ञः।
यद॑श्वमे॒धः।
अप्ये॒व नोऽत्रा॒स्त्विति॑।
तेभ्य॑ ए॒तान॑न्नहो॒मान्प्राय॑च्छत्।
तान॑जुहोत्।
तैर्वै स दे॒वान॑प्रीणात्।
यद॑न्नहो॒मां जु॒होति॑॥५०॥

%3.8.14.2
दे॒वाने॒व तैर्यज॑मानः प्रीणाति।
आज्ये॑न जुहोति।
अ॒ग्नेर्वा ए॒तद्रू॒पम्।
यदाज्यम्᳚।
यदाज्ये॑न जु॒होति॑।
अ॒ग्निमे॒व तत्प्री॑णाति।
मधु॑ना जुहोति।
म॒ह॒त्यै वा ए॒तद्दे॒वता॑यै रू॒पम्।
यन्मधु॑।
यन्मधु॑ना जु॒होति॑॥५१॥

%3.8.14.3
म॒ह॒तीमे॒व तद्दे॒वतां᳚ प्रीणाति।
त॒ण्डु॒लैर्जु॑होति।
वसू॑नां॒ वा ए॒तद्रू॒पम्।
यत्त॑ण्डु॒लाः।
यत्त॑ण्डु॒लैर्जु॒होति॑।
वसू॑ने॒व तत्प्री॑णाति।
पृथु॑कैर्जुहोति।
रु॒द्राणां॒ वा ए॒तद्रू॒पम्।
यत्पृथु॑काः।
यत्पृथु॑कैर्जु॒होति॑॥५२॥

%3.8.14.4
रु॒द्राने॒व तत्प्री॑णाति।
ला॒जैर्जु॑होति।
आ॒दि॒त्यानां॒ वा ए॒तद्रू॒पम्।
यल्ला॒जाः।
यल्ला॒जैर्जु॒होति॑।
आ॒दि॒त्याने॒व तत्प्री॑णाति।
क॒रम्बै᳚र्जुहोति।
विश्वे॑षां॒ वा ए॒तद्दे॒वानाꣳ॑ रू॒पम्।
यत्क॒रम्बाः᳚।
यत्क॒रम्बै᳚र्जु॒होति॑॥५३॥

%3.8.14.5
विश्वा॑ने॒व तद्दे॒वान्प्री॑णाति।
धा॒नाभि॑र्जुहोति।
नक्ष॑त्राणां॒ वा ए॒तद्रू॒पम्।
यद्धा॒नाः।
यद्धा॒नाभि॑र्जु॒होति॑।
नक्ष॑त्राण्ये॒व तत्प्री॑णाति।
सक्तु॑भिर्जुहोति।
प्र॒जा\-प॑ते॒र्वा ए॒तद्रू॒पम्।
यथ्सक्त॑वः।
यथ्सक्तु॑भिर्जु॒होति॑॥५४॥

%3.8.14.6
प्र॒जा॑पतिमे॒व तत्प्री॑णाति।
म॒सूस्यै᳚र्जुहोति।
सर्वा॑सां॒ वा ए॒तद्दे॒वता॑नाꣳ रू॒पम्।
यन्म॒सूस्या॑नि।
यन्म॒सूस्यै᳚र्जु॒होति॑।
सर्वा॑ ए॒व तद्दे॒वताः᳚ प्रीणाति।
प्रि॒य॒ङ्गु॒त॒ण्डु॒लैर्जु॑होति।
प्रि॒याङ्गा॑ ह॒ वै नामै॒ते।
ए॒तैर्वै दे॒वा अश्व॒स्याङ्गा॑नि॒ सम॑दधुः।
यत्प्रि॑यङ्गुतण्डु॒लैर्जु॒होति॑।
अश्व॑स्यै॒वाङ्गा॑नि॒ सन्द॑धाति।
दशान्ना॑नि जुहोति।
दशा᳚क्षरा वि॒राट्।
वि॒राट्कृ॒थ्स्नस्या॒न्नाद्य॒स्या\-व॑\-रुद्ध्यै॥५५॥\anuvakamend[जु॒होति॒ मधु॑ना जु॒होति॒ पृथु॑कैर्जु॒होति॑ क॒रम्बै᳚र्जु॒होति॒ सक्तु॑भिर्जु॒होति॑ प्रियङ्गुतण्डु॒लैर्जु॒होति॑ च॒त्वारि॑ च (अ॒न्नहो॒मानाऽऽज्ये॑ना॒ग्नेर्मधु॑ना तण़्डु॒लैः पृथु॑कैर्ला॒जैः क॒रम्बै᳚र्धा॒नाभिः॒ सक्तु॑भिर्म॒सूस्यैः᳚ प्रियङ्गुतण्डु॒लैर्द॒शान्ना॑नि॒ द्वाद॑श।
)]

%3.8.15.1
प्र॒जा\-प॑तिरश्वमे॒धम॑\-सृजत।
तꣳ सृ॒ष्टꣳ रक्षाꣴ॑स्यजिघाꣳसन्।
स ए॒तान्प्र॒जा\-प॑तिर्न॒क्तꣳ हो॒मान॑पश्यत्।
तान॑जुहोत्।
तैर्वै स य॒ज्ञाद्रक्षा॒ꣴ॒स्यपा॑हन्।
यन्न॑क्तꣳ हो॒मां जु॒होति॑।
य॒ज्ञादे॒व तैर्यज॑मानो॒ रक्षा॒ꣴ॒स्यप॑ हन्ति।
आज्ये॑न जुहोति।
वज्रो॒ वा आज्यम्᳚।
वज्रे॑णै॒व य॒ज्ञाद्रक्षा॒ꣴ॒स्यप॑ हन्ति॥५६॥

%3.8.15.2
आज्य॑स्य प्रति॒पदं॑ करोति।
प्रा॒णो वा आज्यम्᳚।
मु॒ख॒त ए॒वास्य॑ प्रा॒णं द॑धाति।
अ॒न्न॒हो॒माञ्जु॑होति।
शरी॑रवदे॒वाव॑ रुन्धे।
व्य॒त्यासं॑ जुहोति।
उ॒भय॒स्या\-व॑\-रुद्ध्यै।
नक्तं॑ जुहोति।
रक्ष॑सा॒मप॑हत्यै।
आज्ये॑नान्त॒तो जु॑होति॥५७॥

%3.8.15.3
प्रा॒णो वा आज्यम्᳚।
उ॒भ॒यत॑ ए॒वास्य॑ प्रा॒णं द॑धाति।
पु॒रस्ता᳚च्चो॒परि॑ष्टाच्च।
एक॑स्मै॒ स्वाहेत्या॑ह।
अ॒स्मिन्ने॒व लो॒के प्रति॑ तिष्ठति।
द्वाभ्या॒ꣴ॒ स्वाहेत्या॑ह।
अ॒मुष्मि॑न्ने॒व लो॒के प्रति॑ तिष्ठति।
उ॒भयो॑रे॒व लो॒कयोः॒ प्रति॑ तिष्ठति।
अ॒स्मिꣴश्चा॒मुष्मिꣴ॑श्च।
श॒ताय॒ स्वाहेत्या॑ह।
श॒तायु॒र्वै पुरु॑षः श॒तवी᳚र्यः।
आयु॑रे॒व वी॒र्य॑मव॑ रुन्धे।
स॒हस्रा॑य॒ स्वाहेत्या॑ह।
आयु॒र्वै स॒हस्रम्᳚।
आयु॑रे॒वाव॑ रुन्धे।
सर्व॑स्मै॒ स्वाहेत्या॑ह।
अप॑रिमितमे॒वाव॑ रुन्धे॥५८॥\anuvakamend[ए॒व य॒ज्ञाद्रक्षा॒ꣴ॒स्यप॑हन्त्यन्त॒तो जु॑होति श॒ताय॒ स्वाहेत्या॑ह स॒प्त च॑]

%3.8.16.1
प्र॒जा\-प॑तिं॒ वा ए॒ष ई᳚फ्स॒तीत्या॑हुः।
यो᳚ऽश्वमे॒धेन॒ यज॑त॒ इति॑।
अथो॑ आहुः।
सर्वा॑णि भू॒तानीति॑।
एक॑स्मै॒ स्वाहेत्या॑ह।
प्र॒जा\-प॑ति॒र्वा एकः॑।
तमे॒वाऽऽप्नो॑ति।
एक॑स्मै॒ स्वाहा॒ द्वाभ्या॒ꣴ॒ स्वाहेत्य॑भिपू॒र्वमाहु॑तीर्जुहोति।
अ॒भि॒पू॒र्वमे॒व सु॑व॒र्गं लो॒कमे॑ति।
ए॒को॒त्त॒रं जु॑होति॥५९॥

%3.8.16.2
ए॒क॒वदे॒व सु॑व॒र्गं लो॒कमे॑ति।
सन्त॑तं जुहोति।
सु॒व॒र्गस्य॑ लो॒कस्य॒ सन्त॑त्यै।
श॒ताय॒ स्वाहेत्या॑ह।
श॒तायु॒र्वै पुरु॑षः श॒तवी᳚र्यः।
आयु॑रे॒व वी॒र्य॑मव॑ रुन्धे।
स॒हस्रा॑य॒ स्वाहेत्या॑ह।
आयु॒र्वै स॒हस्रम्᳚।
आयु॑रे॒वाव॑ रुन्धे।
अ॒युता॑य॒ स्वाहा॑ नि॒युता॑य॒ स्वाहा᳚ प्र॒युता॑य॒ स्वाहेत्या॑ह॥६०॥

%3.8.16.3
त्रय॑ इ॒मे लो॒काः।
इ॒माने॒व लो॒कानव॑ रुन्धे।
अर्बु॑दाय॒ स्वाहेत्या॑ह।
वाग्वा अर्बु॑दम्।
वाच॑मे॒वाव॑ रुन्धे।
न्य॑र्बुदाय॒ स्वाहेत्या॑ह।
यो वै वा॒चो भू॒मा।
तन्न्य॑र्बुदम्।
वा॒च ए॒व भू॒मान॒मव॑ रुन्धे।
स॒मु॒द्राय॒ स्वाहेत्या॑ह॥६१॥

%3.8.16.4
स॒मु॒द्रमे॒वाऽऽप्नो॑ति।
मध्या॑य॒ स्वाहेत्या॑ह।
मध्य॑मे॒वाऽऽप्नो॑ति।
अन्ता॑य॒ स्वाहेत्या॑ह।
अन्त॑मे॒वाऽऽप्नो॑ति।
प॒रा॒र्धाय॒ स्वाहेत्या॑ह।
प॒रा॒र्धमे॒वाऽऽप्नो॑ति।
उ॒षसे॒ स्वाहा॒ व्यु॑ष्ट्यै॒ स्वाहेत्या॑ह।
रात्रि॒र्वा उ॒षाः।
अह॒र्व्यु॑ष्टिः।
अ॒हो॒रा॒त्रे ए॒वाव॑ रुन्धे।
अथो॑ अहोरा॒त्रयो॑रे॒व प्रति॑ तिष्ठति।
ता यदु॒भयी॒र्दिवा॑ वा॒ नक्तं॑ वा जुहु॒यात्।
अ॒हो॒रा॒त्रे मो॑हयेत्।
उ॒षसे॒ स्वाहा॒ व्यु॑ष्ट्यै॒ स्वाहो॑देष्य॒ते स्वाहो᳚द्य॒ते स्वाहेत्यनु॑दिते जुहोति।
उदि॑ताय॒ स्वाहा॑ सुव॒र्गाय॒ स्वाहा॑ लो॒काय॒ स्वाहेत्युदि॑ते जुहोति।
अ॒हो॒रा॒त्रयो॒रव्य॑तिमोहाय॥६२॥\anuvakamend[ए॒को॒त्त॒रं जु॑होति प्र॒युता॑य॒ स्वाहेत्या॑ह समु॒द्राय॒ स्वाहेत्या॒हाह॒र्व्यु॑ष्टिः स॒प्त च॑]

%3.8.17.1
वि॒भूर्मा॒त्रा प्र॒भूः पि॒त्रेत्य॑श्वना॒मानि॑ जुहोति।
उ॒भयो॑रे॒वैनं॑ लो॒कयो᳚र्नाम॒धेयं॑ गमयति।
आय॑नाय॒ स्वाहा॒ प्राय॑णाय॒ स्वाहेत्यु॑द्द्रा॒वाञ्जु॑होति।
सर्व॑मे॒वैन॒मस्क॑न्नꣳ सुव॒र्गं लो॒कं ग॑मयति।
अ॒ग्नये॒ स्वाहा॒ सोमा॑य॒ स्वाहेति॑ पूर्वहो॒माञ्जु॑होति।
पूर्व॑ ए॒व द्वि॒षन्तं॒ भ्रातृ॑व्य॒मति॑ क्रामति।
पृ॒थि॒व्यै स्वाहा॒\-ऽन्तरि॑क्षाय॒ स्वाहेत्या॑ह।
य॒था॒\-य॒जु\-रे॒वै\-तत्।
अ॒ग्नये॒ स्वाहा॒ सोमा॑य॒ स्वाहेति॑ पूर्वदी॒क्षा जु॑होति।
पूर्व॑ ए॒व द्वि॒षन्तं॒ भ्रातृ॑व्य॒मति॑ क्रामति॥६३॥

%3.8.17.2
पृ॒थि॒व्यै स्वाहा॒\-ऽन्तरि॑क्षाय॒ स्वाहेत्ये॑कवि॒ꣳ॒शिनीं᳚ दी॒क्षां जु॑होति।
एक॑विꣳशति॒र्वै दे॑वलो॒काः।
द्वाद॑श॒ मासाः॒ पञ्च॒र्तवः॑।
त्रय॑ इ॒मे लो॒काः।
अ॒सावा॑दि॒त्य ए॑कवि॒ꣳ॒शः।
ए॒ष सु॑व॒र्गो लो॒कः।
सु॒व॒र्गस्य॑ लो॒कस्य॒ सम॑ष्ट्यै।
भुवो॑ दे॒वानां॒ कर्म॒णेत्यृ॑तुदी॒क्षा जु॑होति।
ऋ॒तूने॒वास्मै॑ कल्पयति।
अ॒ग्नये॒ स्वाहा॑ वा॒यवे॒ स्वाहेति॑ जुहो॒त्यन॑न्तरित्यै॥६४॥

%3.8.17.3
अ॒र्वाङ्य॒ज्ञः सङ्क्रा॑म॒त्वित्याप्ती᳚र्जुहोति।
सु॒व॒र्गस्य॑ लो॒कस्याप्त्यै᳚।
भू॒तं भव्यं॑ भवि॒ष्यदिति॒ पर्या᳚प्तीर्जुहोति।
सु॒व॒र्गस्य॑ लो॒कस्य॒ पर्या᳚प्त्यै।
आ मे॑ गृ॒हा भ॑व॒न्त्वित्या॒भूर्जु॑होति।
सु॒व॒र्गस्य॑ लो॒कस्याभू᳚त्यै।
अ॒ग्निना॒ तपो\-ऽन्व॑भव॒दित्य॑नु॒भूर्जु॑होति।
सु॒व॒र्गस्य॑ लो॒कस्यानु॑भूत्यै।
स्वाहा॒\-ऽऽधिमाधी॑ताय॒ स्वाहेति॒ सम॑स्तानि वैश्वदे॒वानि॑ जुहोति।
सम॑स्तमे॒व द्वि॒षन्तं॒ भ्रातृ॑व्य॒मति॑ क्रामति॥६५॥

%3.8.17.4
द॒द्भ्यः स्वाहा॒ हनू᳚भ्या॒ꣴ॒ स्वाहेत्य॑ङ्गहो॒माञ्जु॑होति।
अङ्गे॑अङ्गे॒ वै पुरु॑षस्य पा॒प्मोप॑श्लिष्टः।
अङ्गा॑दङ्गादे॒वैनं॑ पा॒प्मन॒स्तेन॑ मुञ्चति।
अ॒ञ्ज्ये॒ताय॒ स्वाहा॑ कृ॒ष्णाय॒ स्वाहा᳚ श्वे॒ताय॒ स्वाहेत्य॑श्वरू॒पाणि॑ जुहोति।
रू॒पैरे॒वैन॒ꣳ॒ सम॑र्धयति।
ओष॑धीभ्यः॒ स्वाहा॒ मूले᳚भ्यः॒ स्वाहेत्यो॑षधिहो॒माञ्जु॑होति।
द्व॒य्यो वा ओष॑धयः।
पुष्पे᳚भ्यो॒\-ऽन्याः फलं॑ गृ॒ह्णन्ति॑।
मूले᳚भ्यो॒\-ऽन्याः।
ता ए॒वोभयी॒रव॑ रुन्धे॥६६॥

%3.8.17.5
वन॒स्पति॑भ्यः॒ स्वाहेति॑ वनस्पतिहो॒माञ्जु॑होति।
आ॒र॒ण्यस्या॒\-न्नाद्य॒स्या\-व॑\-रुद्ध्यै।
मे॒षस्त्वा॑ पच॒तैर॑व॒त्वित्यपा᳚व्यानि जुहोति।
प्रा॒णा वै दे॒वा अपा᳚व्याः।
प्रा॒णाने॒वाव॑ रुन्धे।
कूप्या᳚भ्यः॒ स्वाहा॒ऽद्भ्यः स्वाहेत्य॒पाꣳ होमा᳚ञ्जुहोति।
अ॒फ्सु वा आपः॑।
अन्नं॒ वा आपः॑।
अ॒द्भ्यो वा अन्नं॑ जायते।
यदे॒वाद्भ्यो\-ऽन्नं॒ जाय॑ते।
तदव॑ रुन्धे॥६७॥\anuvakamend[पू॒र्व॒दी॒क्षा जु॑होति॒ पूर्व॑ ए॒व द्वि॒षन्तं॒ भ्रातृ॑व्य॒मति॑ क्राम॒त्यन॑न्तरित्यै क्रामति रुन्धे॒ जाय॑त॒ एकं॑ च]

%3.8.18.1
अम्भाꣳ॑सि जुहोति।
अ॒यं वै लो॒को\-ऽम्भाꣳ॑सि।
तस्य॒ वस॒वो\-ऽधि॑पतयः।
अ॒ग्निर्ज्योतिः॑।
यदम्भाꣳ॑सि जु॒होति॑।
इ॒ममे॒व लो॒कमव॑ रुन्धे।
वसू॑ना॒ꣳ॒ सायु॑ज्यं गच्छति।
अ॒ग्निं ज्योति॒रव॑ रुन्धे।
नभाꣳ॑सि जुहोति।
अ॒न्तरि॑क्षं॒ वै नभाꣳ॑सि॥६८॥

%3.8.18.2
तस्य॑ रु॒द्रा अधि॑पतयः।
वा॒युर्ज्योतिः॑।
यन्नभाꣳ॑सि जु॒होति॑।
अ॒न्तरि॑क्षमे॒वाव॑ रुन्धे।
रु॒द्राणा॒ꣳ॒ सायु॑ज्यं गच्छति।
वा॒युं ज्योति॒रव॑ रुन्धे।
महाꣳ॑सि जुहोति।
अ॒सौ वै लो॒को महाꣳ॑सि।
तस्या॑दि॒त्या अधि॑पतयः।
सूर्यो॒ ज्योतिः॑॥६९॥

%3.8.18.3
यन्महाꣳ॑सि जु॒होति॑।
अ॒मुमे॒व लो॒कमव॑ रुन्धे।
आ॒दि॒त्याना॒ꣳ॒ सायु॑ज्यं गच्छति।
सूर्यं॒ ज्योति॒रव॑ रुन्धे।
नमो॒ राज्ञे॒ नमो॒ वरु॑णा॒येति॑ य॒व्यानि॑ जुहोति।
अ॒न्नाद्य॒स्या\-व॑\-रुद्ध्यै।
म॒यो॒भूर्वातो॑ अ॒भि वा॑तू॒स्रा इति॑ ग॒व्यानि॑ जुहोति।
प॒शू॒नामव॑रुद्ध्यै।
प्रा॒णाय॒ स्वाहा᳚ व्या॒नाय॒ स्वाहेति॑ सन्ततिहो॒माञ्जु॑होति।
सु॒व॒र्गस्य॑ लो॒कस्य॒ सन्त॑त्यै॥७०॥

%3.8.18.4
सि॒ताय॒ स्वाहा\-ऽसि॑ताय॒ स्वाहेति॒ प्रमु॑क्तीर्जुहोति।
सु॒व॒र्गस्य॑ लो॒कस्य॒ प्रमु॑क्त्यै।
पृ॒थि॒व्यै स्वाहा॒\-ऽन्तरि॑क्षाय॒ स्वाहेत्या॑ह।
य॒था॒\-य॒जु\-रे॒वै\-तत्।
द॒त्वते॒ स्वाहा॑\-ऽद॒न्तका॑य॒ स्वाहेति॑ शरीरहो॒माञ्जु॑होति।
पि॒तृ॒लो॒कमे॒व तैर्यज॑मा॒नोऽव॑ रुन्धे।
कस्त्वा॑ युनक्ति॒ स त्वा॑ युन॒क्त्विति॑ परि॒धीन् यु॑नक्ति।
इ॒मे वै लो॒काः प॑रि॒धयः॑।
इ॒माने॒वास्मै॑ लो॒कान् यु॑नक्ति।
सु॒व॒र्गस्य॑ लो॒कस्य॒ सम॑ष्ट्यै॥७१॥

%3.8.18.5
यः प्रा॑ण॒तो य आ᳚त्म॒दा इति॑ महि॒मानौ॑ जुहोति।
सु॒व॒र्गो वै लो॒को महः॑।
सु॒व॒र्गमे॒व ताभ्यां᳚ लो॒कं यज॑मा॒नोऽव॑ रुन्धे।
आ ब्रह्म॑न्ब्राह्म॒णो ब्र॑ह्म\-वर्च॒सी जा॑यता॒मिति॒ सम॑स्तानि ब्रह्मवर्च॒सानि॑ जुहोति।
ब्र॒ह्म॒व॒र्चसमे॒व तैर्यज॑मा॒नोऽव॑ रुन्धे।
जज्ञि॒ बीज॒मिति॑ जुहो॒त्यन॑न्तरित्यै।
अ॒ग्नये॒ सम॑नमत्पृथि॒व्यै सम॑नम॒दिति॑ सन्नतिहो॒माञ्जु॑होति।
सु॒व॒र्गस्य॑ लो॒कस्य॒ सन्न॑त्यै।
भू॒ताय॒ स्वाहा॑ भविष्य॒ते स्वाहेति॑ भूताभ॒व्यौ होमौ॑ जुहोति।
अ॒यं वै लो॒को भू॒तम्॥७२॥

%3.8.18.6
अ॒सौ भ॑वि॒ष्यत्।
अ॒नयो॑रे॒व लो॒कयोः॒ प्रति॑ तिष्ठति।
सर्व॒स्याऽऽप्त्यै᳚।
सर्व॒स्या\-व॑\-रुद्ध्यै।
यदक्र॑न्दः प्रथ॒मं जाय॑मान॒ इत्य॑श्वस्तो॒मीयं॑ जुहोति।
सर्व॒स्याऽऽप्त्यै᳚।
सर्व॑स्य॒ जित्यै᳚।
सर्व॑मे॒व तेना᳚ऽऽप्नोति।
सर्वं॑ जयति।
यो᳚ऽश्वमे॒धेन॒ यज॑ते॥७३॥

%3.8.18.7
य उ॑ चैनमे॒वं वेद॑।
य॒ज्ञꣳ रक्षाꣴ॑स्यजिघाꣳसन्।
स ए॒तान्प्र॒जा\-प॑तिर्नक्तꣳहो॒मान॑पश्यत्।
तान॑जुहोत्।
तैर्वै स य॒ज्ञाद्रक्षा॒ꣴ॒स्यपा॑हन्।
यन्न॑क्तꣳहो॒माञ्जु॒होति॑।
य॒ज्ञादे॒व तैर्यज॑मानो॒ रक्षा॒ꣴ॒स्यप॑हन्ति।
उ॒षसे॒ स्वाहा॒ व्यु॑ष्ट्यै॒ स्वाहेत्य॑न्त॒तो जु॑होति।
सु॒व॒र्गस्य॑ लो॒कस्य॒ सम॑ष्ट्यै॥७४॥\anuvakamend[वै नभाꣳ॑सि॒ सूर्यो॒ ज्योतिः॒ सन्त॑त्यै॒ सम॑ष्ट्यै भू॒तं यज॑ते॒ नव॑ च]

%3.8.19.1
ए॒क॒यू॒पो वै॑काद॒शिनी॑ वा।
अ॒न्येषां᳚ य॒ज्ञानां॒ यूपा॑ भवन्ति।
ए॒क॒वि॒ꣳ॒शिन्य॑श्वमे॒धस्य॑।
सु॒व॒र्गस्य॑ लो॒कस्या॒भिजि॑त्यै।
बै॒ल्॒वो वा॑ खादि॒रो वा॑ पाला॒शो वा᳚।
अ॒न्येषां᳚ यज्ञक्रतू॒नां यूपा॑ भवन्ति।
राज्जु॑दाल॒ एक॑विꣳशत्यरत्निरश्वमे॒धस्य॑।
सु॒व॒र्गस्य॑ लो॒कस्य॒ सम॑ष्ट्यै।
नान्येषां᳚ पशू॒नां ते॑ज॒न्या अ॑व॒द्यन्ति॑।
अव॑द्य॒न्त्यश्व॑स्य॥७५॥

%3.8.19.2
पा॒प्मा वै ते॑ज॒नी।
पा॒प्मनो\-ऽप॑हत्यै।
प्ल॒क्ष॒शा॒खाया॑म॒न्येषां᳚ पशू॒नाम॑व॒द्यन्ति॑।
वे॒त॒स॒शा॒खाया॒मश्व॑स्य।
अ॒फ्सुयो॑नि॒र्वा अश्वः॑।
अ॒फ्सु॒जो वे॑त॒सः।
स्व ए॒वास्य॒ योना॒वव॑ द्यति।
यूपे॑षु ग्रा॒म्यान्\-प॒शून्नि॑यु॒ञ्जन्ति॑।
आ॒रो॒केष्वा॑र॒ण्यान्धा॑रयन्ति।
प॒शू॒नां व्यावृ॑त्त्यै।
आ ग्रा॒म्यान्प॒शूल्लँभ॑न्ते।
प्रार॒ण्यान्थ्सृ॑जन्ति।
पा॒प्मनो\-ऽप॑हत्यै॥७६॥\anuvakamend[अश्व॑स्य॒ व्यावृ॑त्त्यै॒ त्रीणि॑ च]

%3.8.20.1
राज्जु॑दालमग्नि॒ष्ठं मि॑नोति।
भ्रू॒ण॒ह॒त्याया॒ अप॑हत्यै।
पौतु॑द्रवाव॒भितो॑ भवतः।
पुण्य॑स्य ग॒न्धस्या\-व॑\-रुद्ध्यै।
भ्रू॒ण॒ह॒त्या\-मे॒वा\-स्मा॑दप॒हत्य॑।
पुण्ये॑न ग॒न्धेनो॑भ॒यतः॒ परि॑ गृह्णाति।
षड्बै॒ल्॒वा भ॑वन्ति।
ब्र॒ह्म॒व॒र्च॒सस्या\-व॑\-रुद्ध्यै।
षट्खा॑दि॒राः।
तेज॒सो\-ऽव॑रुद्ध्यै॥७७॥

%3.8.20.2
षट्पा॑ला॒शाः।
सो॒म॒पी॒थस्या\-व॑\-रुद्ध्यै।
एक॑विꣳशतिः॒ सम्प॑द्यन्ते।
एक॑विꣳशति॒र्वै दे॑वलो॒काः।
द्वाद॑श॒ मासाः॒ पञ्च॒र्तवः॑।
त्रय॑ इ॒मे लो॒काः।
अ॒सावा॑दि॒त्य एक॑वि॒ꣳ॒शः।
ए॒ष सु॑व॒र्गो लो॒कः।
सु॒व॒र्गस्य॑ लो॒कस्य॒ सम॑ष्ट्यै।
श॒तं प॒शवो॑ भवन्ति॥७८॥

%3.8.20.3
श॒तायुः॒ पुरु॑षः श॒तेन्द्रि॑यः।
आयु॑ष्ये॒वेन्द्रि॒ये प्रति॑ तिष्ठति।
सर्वं॒ वा अ॑श्वमे॒ध्याप्नो॑ति।
अप॑रिमिता भवन्ति।
अप॑रिमित॒स्या\-व॑\-रुद्ध्यै।
ब्र॒ह्म॒वा॒दिनो॑ वदन्ति।
कस्मा᳚थ्स॒त्यात्।
द॒क्षि॒ण॒तो᳚\-ऽन्येषां᳚ पशू॒ना\-म॑व॒\-द्यन्ति॑।
उ॒त्त॒र॒तो\-ऽश्व॒स्येति॑।
वा॒रु॒णो वा अश्वः॑॥७९॥

%3.8.20.4
ए॒षा वै वरु॑णस्य॒ दिक्।
स्वाया॑मे॒वास्य॑ दि॒श्यव॑द्यति।
यदित॑रेषां पशू॒नाम॑व॒द्यति॑।
श॒त॒दे॒व॒त्यं॑ तेनाव॑ रुन्धे।
चि॒ते᳚\-ऽग्नावधि॑ वैत॒से कटे\-ऽश्वं॑ चिनोति।
अ॒फ्सुयो॑नि॒र्वा अश्वः॑।
अ॒फ्सु॒जो वे॑त॒सः।
स्व ए॒वैनं॒ योनौ॒ प्रति॑\-ष्ठापयति।
पु॒रस्ता᳚त्प्र॒त्यञ्चं॑ तूप॒रं चि॑नोति।
प॒श्चात्प्रा॒चीनं॑ गोमृ॒गम्॥८०॥

%3.8.20.5
प्रा॒णा॒पा॒नावे॒वास्मि᳚न्थ्स॒म्यञ्चौ॑ दधाति।
अश्वं॑ तूप॒रं गो॑मृ॒गमिति॑ सर्व॒हुत॑ ए॒ताञ्जु॑होति।
ए॒षां लो॒काना॑म॒भिजि॑त्यै।
आ॒त्मना॒ऽभि जु॑\-होति।
सात्मा॑नमे॒वैन॒ꣳ॒ सत॑नुं करोति।
सात्मा॒\-ऽमुष्मिँ॑ल्लो॒के भ॑वति।
य ए॒वं वेद॑।
अथो॒ वसो॑रे॒व धारां॒ तेनाव॑ रुन्धे।
इ॒लु॒\-वर्दा॑य॒ स्वाहा॑ बलि॒वर्दा॑य॒ स्वाहेत्या॑ह।
सं॒व॒थ्स॒रो वा इ॑लु॒वर्दः॑।
प॒रि॒\-व॒थ्स॒रो ब॑लि॒वर्दः॑।
सं॒व॒थ्स॒रा\-दे॒व प॑रि\-वथ्स॒रा\-दायु॒रव॑ रुन्धे।
आयु॑\-रे॒वा\-स्मि॑न्दधाति।
तस्मा॑दश्वमेधया॒जी ज॒रसा॑ वि॒स्रसा॒मुं लो॒कमे॑ति॥८१॥\anuvakamend[तेज॒सो\-ऽव॑रुद्ध्यै भव॒न्त्यश्वो॑ गोमृ॒गमि॑लु॒वर्द॑श्च॒त्वारि॑ च]

%3.8.21.1
ए॒क॒वि॒ꣳ॒शो᳚\-ऽग्निर्भ॑वति।
ए॒क॒वि॒ꣳ॒शः स्तोमः॑।
एक॑\-विꣳशति॒र्यूपाः᳚।
यथा॒ वा अश्वा॑ वर्\mbox{}ष॒भा वा॒ वृषा॑णः सꣴस्फु॒रेरन्॑।
ए॒वमे॒व तथ्स्तोमाः॒ सꣴस्फु॑रन्ते।
यदे॑कवि॒ꣳ॒शाः।
ते यथ्स॑मृ॒च्छेरन्॑।
ह॒न्येता᳚स्य य॒ज्ञः।
द्वा॒द॒श ए॒वाग्निः स्या॒दित्या॑हुः।
द्वा॒द॒शः स्तोमः॑॥८२॥

%3.8.21.2
एका॑दश॒ यूपाः᳚।
यद्द्वा॑द॒शो᳚\-ऽग्निर्भव॑ति।
द्वाद॑श॒ मासाः᳚ संवथ्स॒रः।
सं॒व॒थ्स॒रेणै॒वास्मा॒ अन्न॒मव॑ रुन्धे।
यद्दश॒ यूपा॒ भव॑न्ति।
दशा᳚क्षरा वि॒राट्।
अन्नं॑ वि॒राट्।
वि॒राजै॒वान्नाद्य॒मव॑ रुन्धे।
य ए॑काद॒शः।
स्तन॑ ए॒वास्यै॒ सः॥८३॥

%3.8.21.3
दु॒ह ए॒वैनां॒ तेन॑।
तदा॑हुः।
यद्द्वा॑द॒शो᳚\-ऽग्निः स्या᳚द्द्वाद॒शः स्तोम॒ एका॑दश॒ यूपाः᳚।
यथा॒ स्थूरि॑णा या॒यात्।
ता॒दृक्तत्।
ए॒क॒वि॒ꣳ॒श ए॒वाग्निः स्या॒दित्या॑हुः।
ए॒क॒वि॒ꣳ॒शः स्तोमः॑।
एक॑विꣳशति॒र्यूपाः᳚।
यथा॒ प्रष्टि॑भि॒र्याति॑।
ता॒दृगे॒व तत्॥८४॥

%3.8.21.4
यो वा अ॑श्वमे॒धे ति॒स्रः क॒कुभो॒ वेद॑।
क॒कुद्ध॒ राज्ञां᳚ भवति।
ए॒क॒वि॒ꣳ॒शो᳚\-ऽग्निर्भ॑वति।
ए॒क॒वि॒ꣳ॒शः स्तोमः॑।
एक॑विꣳशति॒र्यूपाः᳚।
ए॒ता वा अ॑श्वमे॒धे ति॒स्रः क॒कुभः॑।
य ए॒वं वेद॑।
क॒कुद्ध॒ राज्ञां᳚ भवति।
यो वा अ॑श्वमे॒धे त्रीणि॑ शी॒र्॒षाणि॒ वेद॑।
शिरो॑ ह॒ राज्ञां᳚ भवति।
ए॒क॒वि॒ꣳ॒शो᳚\-ऽग्निर्भ॑वति।
ए॒क॒वि॒ꣳ॒शः स्तोमः॑।
एक॑विꣳशति॒र्यूपाः᳚।
ए॒तानि॒ वा अ॑श्वमे॒धे त्रीणि॑ शी॒र्॒षाणि॑।
य ए॒वं वेद॑।
शिरो॑ ह॒ राज्ञां᳚ भवति॥८५॥\anuvakamend[द्वा॒द॒शः स्तोमः॒ स ए॒व तच्छिरो॑ ह॒ राज्ञां᳚ भवति॒ षट् च॑]

%3.8.22.1
दे॒वा वा अ॑श्वमे॒धे पव॑माने।
सु॒व॒र्गं लो॒कं न प्राजा॑नन्।
तमश्वः॒ प्राजा॑नात्।
यद॑श्वमे॒धे\-ऽश्वे॑न॒ मेध्ये॒नोद॑ञ्चो बहिष्पवमा॒नꣳ सर्प॑न्ति।
सु॒व॒र्गस्य॑ लो॒कस्य॒ प्रज्ञा᳚त्यै।
न वै म॑नु॒ष्यः॑ सुव॒र्गं लो॒कमञ्ज॑सा वेद।
अश्वो॒ वै सु॑व॒र्गं लो॒कमञ्ज॑सा वेद।
यदु॑द्गा॒तोद्गाये᳚त्।
यथा क्षे᳚त्रज्ञो॒\-ऽन्येन॑ प॒था प्र॑तिपा॒दये᳚त्।
ता॒दृक्तत्॥८६॥

%3.8.22.2
उ॒द्गा॒तार॑मप॒रुध्य॑।
अश्व॑मुद्गी॒थाय॑ वृणीते।
यथा᳚ क्षेत्र॒ज्ञो\-ऽञ्ज॑सा॒ नय॑ति।
ए॒वमे॒वैन॒मश्वः॑ सुव॒र्गं लो॒कमञ्ज॑सा नयति।
पुच्छ॑म॒न्वा र॑भन्ते।
सु॒व॒र्गस्य॑ लो॒कस्य॒ सम॑ष्ट्यै।
हिं क॑रोति।
सामै॒वाकः॑।
हिं क॑रोति।
उ॒द्गी॒थ ए॒वास्य॒ सः॥८७॥

%3.8.22.3
वड॑बा॒ उप॑ रुन्धन्ति।
मि॒थु॒न॒त्वाय॒ प्रजा᳚त्यै।
अथो॒ यथो॑पगा॒तार॑ उप॒गाय॑न्ति।
ता॒दृगे॒व तत्।
उद॑गासी॒दश्वो॒ मेध्य॒ इत्या॑ह।
प्रा॒जा॒प॒त्यो वा अश्वः॑।
प्र॒जा\-प॑तिरुद्गी॒थः।
उ॒द्गी॒थमे॒वाव॑ रुन्धे।
अथो॑ ऋख्सा॒मयो॑रे॒व प्रति॑ तिष्ठति।
हिर॑ण्येनो॒पाक॑रोति।
ज्योति॒र्वै हिर॑ण्यम्।
ज्योति॑रे॒व मु॑ख॒तो द॑धाति।
यज॑माने च प्र॒जासु॑ च।
अथो॒ हिर॑ण्यज्योतिरे॒व यज॑मानः सुव॒र्गं लो॒कमे॑ति॥८८॥\anuvakamend[तथ्स उ॒पाक॑रोति च॒त्वारि॑ च]

%3.8.23.1
पुरु॑षो॒ वै य॒ज्ञः।
य॒ज्ञः प्र॒जा\-प॑तिः।
यदश्वे॑ प॒शून्नि॑यु॒ञ्जन्ति॑।
य॒ज्ञादे॒व तद्य॒ज्ञं प्रयु॑ङ्क्ते।
अश्वं॑ तूप॒रं गो॑मृ॒गम्।
तान॑ग्नि॒ष्ठ आल॑भते।
से॒ना॒मु॒खमे॒व तथ्सꣴश्य॑ति।
तस्मा᳚द्राजमु॒खं भी॒ष्मं भावु॑कम्।
आ॒ग्ने॒यं कृ॒ष्णग्री॑वं पु॒रस्ता᳚ल्ल॒लाटे᳚।
पू॒र्वा॒ग्निमे॒व तं कु॑रुते॥८९॥

%3.8.23.2
तस्मा᳚त्पूर्वा॒ग्निं पु॒रस्ता᳚थ्स्थापयन्ति।
पौ॒ष्णम॒न्वञ्चम्᳚।
अन्नं॒ वै पू॒षा।
तस्मा᳚त्पूर्वा॒ग्नावा॑\-हा॒र्य॑मा ह॑रन्ति।
ऐ॒न्द्रा॒पौ॒ष्णमु॒परि॑ष्टात्।
ऐ॒न्द्रो वै रा॑ज॒न्यो\-ऽन्नं॑ पू॒षा।
अ॒न्नाद्ये॑नै॒वैन॑मुभ॒यतः॒ परि॑ गृह्णाति।
तस्मा᳚द्राज॒न्यो᳚\-ऽन्ना॒दो भावु॑कः।
आ॒ग्ने॒यौ कृ॒ष्णग्री॑वौ बाहु॒वोः।
बा॒हु॒वोरे॒व वी॒र्यं॑ धत्ते॥९०॥

%3.8.23.3
तस्मा᳚द्राज॒न्यो॑ बाहुब॒लीभावु॑कः।
त्वा॒ष्ट्रौ लो॑मशस॒क्थौ स॒क्थ्योः।
स॒क्थ्योरे॒व वी॒र्यं॑ धत्ते।
तस्मा᳚द्राज॒न्य॑ ऊरुब॒लीभावु॑कः।
शि॒ति॒पृ॒ष्ठौ बा॑र्\mbox{}हस्प॒त्यौ पृ॒ष्ठे।
ब्र॒ह्म॒व॒र्च॒समे॒वोपरि॑ष्टाद्धत्ते।
अथो॑ क॒वचे॑ ए॒वैते अ॒भितः॒ पर्यू॑हते।
तस्मा᳚द्राज॒न्यः॑ सन्न॑द्धो वी॒र्यं॑ करोति।
धा॒त्रे पृ॑षोद॒रम॒धस्ता᳚त्।
प्र॒ति॒ष्ठामे॒वैतां कु॑रुते।
अथो॑ इ॒यं वै धा॒ता।
अ॒स्यामे॒व प्रति॑ तिष्ठति।
सौ॒र्यं ब॒लक्षं॒ पुच्छे᳚।
उ॒थ्से॒धमे॒व तं कु॑रुते।
तस्मा॑दुथ्से॒धं भ॒ये प्र॒जा अ॒भिसꣴश्र॑यन्ति॥९१॥\anuvakamend[कु॒रु॒ते॒ ध॒त्ते॒ कु॒रु॒ते॒ पञ्च॑ च]




\prashnaend{सा॒ङ्ग्र॒ह॒ण्या चतु॑ष्टय्यो॒ यो वै यः पि॒तुश्च॒त्वारो॒ यथा॑ नि॒क्तं प्र॒जा\-प॑तये त्वा॒ यथा॒ प्रोक्षि॑तं वि॒भूरा॑ह प्र॒जा\-प॑तिरकामयताश्वमे॒धेन॑ प्र॒जा\-प॑ति॒र्न किञ्च॒न सा॑वि॒त्रमा ब्रह्म॑न्प्र॒जा\-प॑तिर्दे॒वैभ्यः॑ प्र॒जा\-प॑ती॒ रक्षाꣳ॑सि प्र॒जा\-प॑तिमीफ्सति वि॒भूर॑श्वना॒मान्यम्भाꣴ॑स्येकयू॒पो राज्जु॑दालमेकवि॒ꣳ॒शो दे॒वाः पुरु॑ष॒स्त्रयो॑विꣳशतिः॥२३॥}{सा॒ङ्ग्रह॒ण्या तस्मा॑दश्वमेधया॒जी यत्परि॑मिता॒ यद्य॑ज्ञमु॒खे यो दी॒क्षां दे॒वाने॒व त्रय॑ इ॒मे सि॒ताय॑ प्राणापा॒नावे॒वास्मि॒न्तस्मा᳚द्राज॒न्य॑ एक॑नवतिः॥९१॥}{सा॒ङ्ग्र॒ह॒ण्या सꣴश्र॑यन्ति॥}{हरिः॑ ओम्॥}{इति श्रीकृष्णयजुर्वेदीयतैत्तिरीयब्राह्मणे तृतीयाष्टके अष्टमः प्रपाठकः समाप्तः॥}
\clearpage
\sect{नवमः प्रश्नः}
\setcounter{anuvakam}{0}
\dnsub{तैत्तिरीयब्राह्मणे तृतीयाष्टके नवमः प्रपाठकः}

%3.9.1.1
प्र॒जा\-प॑तिरश्वमे॒धम॑\-सृजत।
सो᳚ऽस्माथ्सृ॒ष्टो\-ऽपा᳚क्रामत्।
तम॑ष्टा\-द॒शिभि॒\-रनु॒\- प्रायु॑ङ्क्त।
तमा᳚प्नोत्।
तमा॒प्त्वा\-ऽष्टा॑द॒शिभि॒रवा॑\-रुन्ध।
यद॑ष्टा\-द॒शिन॑ आल॒भ्यन्ते᳚।
य॒ज्ञमे॒व तैरा॒प्त्वा यज॑मा॒नो\-ऽव॑ रुन्धे।
सं॒व॒थ्स॒\-र\-स्य॒ वा ए॒षा प्र॑ति॒मा।
यद॑ष्टाद॒शिनः॑।
द्वाद॑श॒ मासाः॒ पञ्च॒र्तवः॑॥१॥

%3.9.1.2
सं॒व॒थ्स॒रो᳚\-ऽष्टाद॒शः।
यद॑ष्टाद॒शिन॑ आल॒भ्यन्ते᳚।
सं॒व॒थ्स॒रमे॒व तैरा॒प्त्वा यज॑मा॒नो\-ऽव॑ रुन्धे।
अ॒ग्नि॒ष्ठे᳚\-ऽन्यान्प॒शूनु॑पाक॒रोति॑।
इत॑रेषु॒ यूपे᳚ष्वष्टाद॒शिनो\-ऽजा॑मित्वाय।
नव॑न॒वाल॑भ्यन्ते सवीर्य॒त्वाय॑।
यदा॑र॒ण्यैः सꣴ॑स्था॒पये᳚त्।
व्यव॑स्येतां पितापु॒त्रौ।
व्यध्वा॑नः क्रामेयुः।
विदू॑रं॒ ग्राम॑योर्ग्रामा॒न्तौ स्या॑ताम्॥२॥

%3.9.1.3
ऋ॒क्षीकाः᳚ पुरुषव्या॒घ्राः प॑रिमो॒षिण॑ आव्या॒धिनी॒स्तस्क॑रा॒ अर॑ण्ये॒ष्वाजा॑येरन्।
तदा॑हुः।
अप॑शवो॒ वा ए॒ते।
यदा॑र॒ण्याः।
यदा॑र॒ण्यैः सꣴ॑स्था॒पये᳚त्।
क्षि॒प्रे यज॑मान॒मर॑ण्यं मृ॒तꣳ ह॑रेयुः।
अर॑ण्यायतना॒ ह्या॑र॒ण्याः प॒शव॒ इति॑।
यत्प॒शून्नालभे॑त।
अन॑वरुद्धा अस्य प॒शवः॑ स्युः।
यत्पर्य॑ग्निकृतानुथ्सृ॒जेत्॥३॥

%3.9.1.4
य॒ज्ञ॒वे॒श॒सं कु॑र्यात्।
यत्प॒शूना॒लभ॑ते।
तेनै॒व प॒शूनव॑ रुन्धे।
यत्पर्य॑ग्निकृतानुथ्सृ॒जत्य\-य॑ज्ञ\-वेशसाय।
अव॑रुद्धा अस्य प॒शवो॒ भव॑न्ति।
न य॑ज्ञवेश॒सं भ॑वति।
न यज॑मान॒मर॑ण्यं मृ॒तꣳ ह॑रन्ति।
ग्रा॒म्यैः सꣴ स्था॑पयति।
ए॒ते वै प॒शवः॒ क्षेमो॒ नाम॑।
सं पि॑तापु॒त्रावव॑स्यतः।
समध्वा॑नः क्रामन्ति।
स॒म॒न्ति॒कं ग्राम॑योर्ग्रामा॒न्तौ भ॑वतः।
नर्क्षीकाः᳚ पुरुषव्या॒घ्राः प॑रिमो॒षिण॑ आव्या॒धिनी॒स्तस्क॑रा॒ अर॑ण्ये॒ष्वाजा॑यन्ते॥४॥\anuvakamend[ऋ॒तवः॑ स्यातामुथ्सृ॒जेथ्स्य॑त॒स्त्रीणि॑ च]

%3.9.2.1
प्र॒जा\-प॑तिरकामयतो॒भौ लो॒कावव॑ रुन्धी॒येति॑।
स ए॒ता\-नु॒भया᳚न्प॒शून॑\-पश्यत्।
ग्रा॒म्याꣴश्चा॑\-र॒ण्याꣴश्च॑।
तानाल॑भत।
तैर्वै स उ॒भौ लो॒का\-ववा॑\-रुन्ध।
ग्रा॒म्यैरे॒व प॒शुभि॑रि॒मं लो॒कमवा॑रुन्ध।
आ॒र॒ण्यै\-र॒मुम्।
यद् ग्रा॒म्यान्प॒शूना॒लभ॑ते।
इ॒ममे॒व तैर्लो॒कमव॑ रुन्धे।
यदा॑र॒ण्यान्॥५॥

%3.9.2.2
अ॒मुं तैः।
अन॑वरुद्धो॒ वा ए॒तस्य॑ संवथ्स॒र इत्या॑हुः।
य इ॒तइ॑तश्चातुर्मा॒स्यानि॑ संवथ्स॒रं प्र॑यु॒ङ्क्त इति॑।
ए॒तावा॒न्॒ वै सं॑वथ्स॒रः।
यच्चा॑तुर्मा॒स्यानि॑।
यदे॒ते चा॑तुर्मा॒स्याः प॒शव॑ आल॒भ्यन्ते᳚।
प्र॒त्यक्ष॑मे॒व तैः सं॑वथ्स॒रं यज॑मा॒नो\-ऽव॑ रुन्धे।
वि वा ए॒ष प्र॒जया॑ प॒शुभि॑र्‌\mbox{}ऋध्यते।
यः सं॑वथ्स॒रं प्र॑यु॒ङ्क्ते।
सं॒व॒थ्स॒रः सु॑व॒र्गो लो॒कः॥६॥

%3.9.2.3
सु॒व॒र्गं तु लो॒कं नाप॑राध्नोति।
प्र॒जा वै प॒शव॑ एकाद॒शिनी᳚।
यदे॒त ऐ॑कादशि॒नाः प॒शव॑ आल॒भ्यन्ते᳚।
सा॒क्षादे॒व प्र॒जां प॒शून् यज॑मा॒नो\-ऽव॑ रुन्धे।
प्र॒जा\-प॑तिर्वि॒राज॑म\-सृजत।
सा सृ॒ष्टा\-ऽश्व॑मे॒धं प्रावि॑शत्।
तान्द॒शिभि॒रनु॒\- प्रायु॑ङ्क्त।
तामा᳚प्नोत्।
तामा॒प्त्वा द॒शिभि॒रवा॑रुन्ध।
यद्द॒शिन॑ आल॒भ्यन्ते᳚॥७॥

%3.9.2.4
वि॒राज॑मे॒व तैरा॒प्त्वा यज॑मा॒नो\-ऽव॑ रुन्धे।
एका॑दश द॒शत॒ आल॑भ्यन्ते।
एका॑दशाक्षरा त्रि॒ष्टुप्।
त्रैष्टु॑भाः प॒शवः॑।
प॒शूने॒वाव॑ रुन्धे।
वै॒श्व॒दे॒वो वा अश्वः॑।
ना॒ना॒दे॒व॒त्याः᳚ प॒शवो॑ भवन्ति।
अश्व॑स्य सर्व॒त्वाय॑।
नाना॑रूपा भवन्ति।
तस्मा॒न्नाना॑रूपाः प॒शवः॑।
ब॒हु॒रू॒पा भ॑वन्ति।
तस्मा᳚द्बहुरू॒पाः प॒शवः॒ समृ॑द्ध्यै॥८॥\anuvakamend[आ॒र॒ण्याँल्लो॒को द॒शिन॑ आल॒भ्यन्ते॒ नाना॑रूपाः प॒शवो॒ द्वे च॑]

%3.9.3.1
अ॒स्मै वै लो॒काय॑ ग्रा॒म्याः प॒शव॒ आल॑भ्यन्ते।
अ॒मुष्मा॑ आर॒ण्याः।
यद्ग्रा॒म्यान्प॒शूना॒लभ॑ते।
इ॒ममे॒व तैर्लो॒कमव॑ रुन्धे।
यदा॑र॒ण्यान्।
अ॒मुं तैः।
उ॒भया᳚न्प॒शूनाल॑भते।
गा॒म्याꣴश्चा॑र॒ण्याꣴश्च॑।
उ॒भयो᳚र्लो॒कयो॒रव॑रुद्ध्यै।
उ॒भया᳚न्प॒शूना\-ल॑भते॥९॥

%3.9.3.2
ग्रा॒म्याꣴश्चा॑\-र॒ण्याꣴश्च॑।
उ॒भय॑स्या॒न्नाद्य॒स्या\-व॑\-रुद्ध्यै।
उ॒भया᳚न्प॒शू\-नाल॑भते।
ग्रा॒म्याꣴश्चा॑\-र॒ण्याꣴश्च॑।
उ॒भये॑षां पशू॒नामव॑रुद्ध्यै।
त्रय॑स्त्रयो भवन्ति।
त्रय॑ इ॒मे लो॒काः।
ए॒षां लो॒काना॒माप्त्यै᳚।
ब्र॒ह्म॒वा॒दिनो॑ वदन्ति।
कस्मा᳚थ्स॒त्यात्॥१०॥

%3.9.3.3
अ॒स्मिँल्लो॒के ब॒हवः॒ कामा॒ इति॑।
यथ्स॑मा॒नीभ्यो॑ दे॒वता᳚भ्यो॒\-ऽन्ये᳚\-ऽन्ये प॒शव॑ आल॒भ्यन्ते᳚।
अ॒स्मिन्ने॒व तल्लो॒के कामा᳚न्दधाति।
तस्मा॑द॒स्मिँल्लो॒के ब॒हवः॒ कामाः᳚।
त्र॒या॒णां त्र॑याणाꣳ स॒ह व॒पा जु॑होति।
त्र्या॑वृतो॒ वै दे॒वाः।
त्र्या॑वृत इ॒मे लो॒काः।
ए॒षां लो॒काना॒माप्त्यै᳚।
ए॒षां लो॒कानां॒ कॢप्त्यै᳚।
पर्य॑ग्निकृतानार॒ण्या\-नुथ्सृ॑ज॒न्त्यहिꣳ॑सायै॥११॥\anuvakamend[अव॑रुद्ध्या उ॒भया᳚न्प॒शूनाल॑भते स॒त्यादहिꣳ॑सायै]

%3.9.4.1
यु॒ञ्जन्ति॑ ब्र॒ध्नमित्या॑ह।
अ॒सौ वा आ॑दि॒त्यो ब्र॒ध्नः।
आ॒दि॒त्यमे॒वास्मै॑ युनक्ति।
अ॒रु॒षमित्या॑ह।
अ॒ग्निर्वा अ॑रु॒षः।
अ॒ग्निमे॒वास्मै॑ युनक्ति।
चर॑न्त॒मित्या॑ह।
वा॒युर्वै चरन्॑।
वा॒युमे॒वास्मै॑ युनक्ति।
परि॑त॒स्थुष॒ इत्या॑ह॥१२॥

%3.9.4.2
इ॒मे वै लो॒काः परि॑त॒स्थुषः॑।
इ॒माने॒वास्मै॑ लो॒कान् यु॑नक्ति।
रोच॑न्ते रोच॒ना दि॒वीत्या॑ह।
नक्ष॑त्राणि॒ वै रो॑च॒ना दि॒वि।
नक्ष॑त्राण्ये॒वास्मै॑ रोचयति।
यु॒ञ्जन्त्य॑स्य॒ काम्येत्या॑ह।
कामा॑ने॒वास्मै॑ युनक्ति।
हरी॒ विप॑क्ष॒सेत्या॑ह।
इ॒मे वै हरी॒ विप॑क्षसा।
इ॒मे ए॒वास्मै॑ युनक्ति॥१३॥

%3.9.4.3
शोणा॑ धृ॒ष्णू नृ॒वाह॒सेत्या॑ह।
अ॒हो॒रा॒त्रे वै नृ॒वाह॑सा।
अ॒हो॒रा॒त्रे ए॒वास्मै॑ युनक्ति।
ए॒ता ए॒वास्मै॑ दे॒वता॑ युनक्ति।
सु॒व॒र्गस्य॑ लो॒कस्य॒ सम॑ष्ट्यै।
के॒तुं कृ॒ण्वन्न॑के॒तव॒ इति॑ ध्व॒जं प्रति॑\-मुञ्चति।
यश॑ ए॒वैन॒ꣳ॒ राज्ञां गमयति।
जी॒मूत॑स्येव भवति॒ प्रती॑क॒मित्या॑ह।
य॒था॒\-य॒जु\-रे॒वै\-तत्।
ये ते॒ पन्था॑नः सवितः पू॒र्व्यास॒ इत्य॑ध्व॒र्युर्यज॑मानं वाचयत्य॒भिजि॑त्यै॥१४॥

%3.9.4.4
परा॒ वा ए॒तस्य॑ य॒ज्ञ ए॑ति।
यस्य॑ प॒शुरु॒पाकृ॑तो॒\-ऽन्यत्र॒ वेद्या॒ एति॑।
ए॒तꣴस्तो॑तरे॒तेन॑ प॒था पुन॒रश्व॒माव॑र्तयासि न॒ इत्या॑ह।
वा॒युर्वै स्तोता᳚।
वा॒युमे॒वास्य॑ प॒रस्ता᳚द्दधा॒त्यावृ॑त्त्यै।
यथा॒ वै ह॒विषो॑ गृही॒तस्य॒ स्कन्द॑ति।
ए॒वं वा ए॒तदश्व॑स्य स्कन्दति।
यद॑स्यो॒पाकृ॑तस्य॒ लोमा॑नि॒ शीय॑न्ते।
यद्वाले॑षु का॒चाना॒वय॑न्ति।
लोमा᳚न्ये॒वास्य॒ तथ्सम्भ॑रन्ति॥१५॥

%3.9.4.5
भूर्भुवः॒ सुव॒रिति॑ प्राजाप॒त्याभि॒राव॑यन्ति।
प्रा॒जा॒प॒त्यो वा अश्वः॑।
स्वयै॒वैनं॑ दे॒वत॑या॒ सम॑र्धयन्ति।
भूरिति॒ महि॑षी।
भुव॒ इति॑ वा॒वाता᳚।
सुव॒रिति॑ परिवृ॒क्ती।
ए॒षां लो॒काना॑म॒भिजि॑त्यै।
हि॒र॒ण्ययाः᳚ का॒चा भ॑वन्ति।
ज्योति॒र्वै हिर॑ण्यम्।
रा॒ष्ट्रम॑श्वमे॒धः॥१६॥

%3.9.4.6
ज्योति॑श्चै॒वास्मै॑ रा॒ष्ट्रं च॑ स॒मीची॑ दधाति।
स॒हस्रं॑ भवन्ति।
स॒हस्र॑सम्मितः सुव॒र्गो लो॒कः।
सु॒व॒र्गस्य॑ लो॒कस्या॒भिजि॑त्यै।
अप॒ वा ए॒तस्मा॒त्तेज॑ इन्द्रि॒यं प॒शवः॒ श्रीः क्रा॑मन्ति।
यो᳚ऽश्वमे॒धेन॒ यज॑ते।
वस॑वस्त्वा\-ऽञ्जन्तु गाय॒त्रेण॒ छन्द॒सेति॒ महि॑ष्य॒भ्य॑नक्ति।
तेजो॒ वा आज्यम्᳚।
तेजो॑ गाय॒त्री।
तेज॑सै॒वास्मै॒ तेजो\-ऽव॑ रुन्धे॥१७॥

%3.9.4.7
रु॒द्रास्त्वा᳚ञ्जन्तु॒ त्रैष्टु॑भेन॒ छन्द॒सेति॑ वा॒वाता᳚।
तेजो॒ वा आज्यम्᳚।
इ॒न्द्रि॒यं त्रि॒ष्टुप्।
तेज॑सै॒वास्मा॑ इन्द्रि॒यमव॑ रुन्धे।
आ॒दि॒त्यास्त्वा᳚\-ऽञ्जन्तु॒ जाग॑तेन॒ छन्द॒सेति॑ परिवृ॒क्ती।
तेजो॒ वा आज्यम्᳚।
प॒शवो॒ जग॑ती।
तेज॑सै॒वास्मे॑ प॒शूनव॑ रुन्धे।
पत्न॑यो॒\-ऽभ्य॑ञ्जन्ति।
श्रि॒या वा ए॒तद्रू॒पम्॥१८॥

%3.9.4.8
यत्पत्न॑यः।
श्रिय॑मे॒वास्मि॒न्तद्द॑धति।
नास्मा॒त्तेज॑ इन्द्रि॒यं प॒शवः॒ श्रीरप॑ क्रामन्ति।
लाजी(३)ञ्छाची(३)न् यशो॑म॒माँ(४) इत्यति॑रिक्त॒मन्न॒मश्वा॑यो॒पाह॑रन्ति।
प्र॒जामे॒वान्ना॒दीं कु॑र्वते।
ए॒तद्दे॑वा॒ अन्न॑मत्तै॒तदन्न॑मद्धि प्रजापत॒ इत्या॑ह।
प्र॒जाया॑मे॒वान्नाद्यं॑ दधते।
यदि॒ नाव॒जिघ्रे᳚त्।
अ॒ग्निः प॒शुरा॑सी॒दित्यव॑घ्रापयेत्।
अव॑ है॒व जि॑घ्रति।
आक्रान्॑ वा॒जी क्रमै॒रत्य॑क्रमीद्वा॒जी द्यौस्ते॑ पृ॒ष्ठं पृ॑थि॒वी स॒धस्थ॒मित्यश्व॒मनु॑मन्त्रयते।
ए॒षां लो॒काना॑म॒भिजि॑त्यै।
समि॑द्धो अ॒ञ्जन्कृद॑रं मती॒नामित्यश्व॑स्या॒प्रियो॑ भवन्ति सरूप॒त्वाय॑॥१९॥\anuvakamend[परि॑त॒स्थुष॒ इत्या॑हे॒मे ए॒वास्मै॑ युनक्त्य॒भिजि॑त्यै भरन्त्यश्वमे॒धो रु॑न्धे रू॒पञ्जि॑घ्रति॒ त्रीणि॑ च]

%3.9.5.1
तेज॑सा॒ वा ए॒ष ब्र॑ह्म\-वर्च॒सेन॒ व्यृ॑द्ध्यते।
यो᳚ऽश्वमे॒धेन॒ यज॑ते।
होता॑ च ब्र॒ह्मा च॑ ब्र॒ह्मोद्यं॑ वदतः।
तेज॑सा चै॒वैनं॑ ब्रह्मवर्च॒सेन॑ च॒ सम॑र्धयतः।
द॒क्षि॒ण॒तो ब्र॒ह्मा भ॑वति।
द॒क्षि॒ण॒त आ॑यतनो॒ वै ब्र॒ह्मा।
बा॒र्॒ह॒स्प॒त्यो वै ब्र॒ह्मा।
ब्र॒ह्म॒व॒र्च॒समे॒वास्य॑ दक्षिण॒तो द॑धाति।
तस्मा॒द्दक्षि॒णो\-ऽर्धो᳚ ब्रह्मवर्च॒सित॑रः।
उ॒त्त॒र॒तो होता॑ भवति॥२०॥

%3.9.5.2
उ॒त्त॒र॒त आ॑यतनो॒ वै होता᳚।
आ॒ग्ने॒यो वै होता᳚।
तेजो॒ वा अ॒ग्निः।
तेज॑ ए॒वास्यो᳚त्तर॒तो द॑धाति।
तस्मा॒दुत्त॒रो\-ऽर्ध॑स्तेज॒स्वित॑रः।
यूप॑म॒भितो॑ वदतः।
य॒ज॒मा॒न॒दे॒व॒त्यो॑ वै यूपः॑।
यज॑मानमे॒व तेज॑सा च ब्रह्मवर्च॒सेन॑ च॒ सम॑र्धयतः।
किꣴ स्वि॑दासीत्पू॒र्वचि॑त्ति॒रित्या॑ह।
द्यौर्वै वृष्टिः॑ पू॒र्वचि॑त्तिः॥२१॥

%3.9.5.3
दिव॑मे॒व वृष्टि॒मव॑ रुन्धे।
किꣴ स्वि॑दासीद्बृ॒हद्वय॒ इत्या॑ह।
अश्वो॒ वै बृ॒हद्वयः॑।
अश्व॑मे॒वाव॑ रुन्धे।
किꣴ स्वि॑दासीत्पिशङ्गि॒लेत्या॑ह।
रात्रि॒र्वै पि॑शङ्गि॒ला।
रात्रि॑मे॒वाव॑ रुन्धे।
किꣴ स्वि॑दासीत्पिलिप्पि॒ले\-त्या॑ह।
श्रीर्वै पि॑लिप्पि॒ला।
अ॒न्नाद्य॑मे॒वाव॑ रुन्धे॥२२॥

%3.9.5.4
कः स्वि॑देका॒की च॑र॒तीत्या॑ह।
अ॒सौ वा आ॑दि॒त्य ए॑का॒की च॑रति।
तेज॑ ए॒वाव॑ रुन्धे।
क उ॑स्विज्जायते॒ पुन॒रित्या॑ह।
च॒न्द्रमा॒ वै जा॑यते॒ पुनः॑।
आयु॑रे॒वाव॑ रुन्धे।
किꣴ स्वि॑द्धि॒मस्य॑ भेष॒जमित्या॑ह।
अ॒ग्निर्वै हि॒मस्य॑ भेष॒जम्।
ब्र॒ह्म॒व॒र्च॒समे॒वाव॑ रुन्धे।
किꣴ स्वि॑दा॒वप॑नं म॒हदित्या॑ह॥२३॥

%3.9.5.5
अ॒यं वै लो॒क आ॒वप॑नं म॒हत्।
अ॒स्मिन्ने॒व लो॒के प्रति॑ तिष्ठति।
पृ॒च्छामि॑ त्वा॒ पर॒मन्तं॑ पृथि॒व्या इत्या॑ह।
वेदि॒र्वै परो\-ऽन्तः॑ पृथि॒व्याः।
वेदि॑मे॒वाव॑ रुन्धे।
पृ॒च्छामि॑ त्वा॒ भुव॑नस्य॒ नाभि॒मित्या॑ह।
य॒ज्ञो वै भुव॑नस्य॒ नाभिः॑।
य॒ज्ञमे॒वाव॑ रुन्धे।
पृ॒च्छामि॑ त्वा॒ वृष्णो॒ अश्व॑स्य॒ रेत॒ इत्या॑ह।
सोमो॒ वै वृष्णो॒ अश्व॑स्य॒ रेतः॑।
सो॒म॒पी॒थमे॒वाव॑ रुन्धे।
पृ॒च्छामि॑ वा॒चः प॑र॒मं व्यो॑मेत्या॑ह।
ब्रह्म॒ वै वा॒चः प॑र॒मं व्यो॑म।
ब्र॒ह्म॒व॒र्च॒समे॒वाव॑ रुन्धे॥२४॥\anuvakamend[होता॑ भवति॒ वै वृष्टिः॑ पू॒र्वचि॑त्तिर॒न्नाद्य॑मे॒वाव॑ रुन्धे म॒हदित्या॑ह॒ सोमो॒ वै वृष्णो॒ अश्व॑स्य॒ रेत॑श्च॒त्वारि॑ च]

%3.9.6.1
अप॒ वा ए॒तस्मा᳚त्प्रा॒णाः क्रा॑मन्ति।
यो᳚ऽश्वमे॒धेन॒ यज॑ते।
प्रा॒णाय॒ स्वाहा᳚ व्या॒नाय॒ स्वाहेति॑ संज्ञ॒प्यमा॑न॒ आहु॑तीर्जुहोति।
प्रा॒णाने॒वास्मि॑न्दधाति।
नास्मा᳚त्प्रा॒णा अप॑क्रामन्ति।
अव॑न्तीः॒ स्थाव॑न्तीस्त्वा\-ऽवन्तु।
प्रि॒यं त्वा᳚ प्रि॒याणा᳚म्।
वर्‌\mbox{}षि॑ष्ठ॒माप्या॑नाम्।
नि॒धी॒नां त्वा॑ निधि॒पतिꣳ॑ हवामहे वसो म॒मेत्या॑ह।
अपै॒वास्मै॒ तद्ध्नु॑वते॥२५॥

%3.9.6.2
अथो॑ धु॒वन्त्ये॒वैनम्᳚।
अथो॒ न्ये॑वास्मै᳚ ह्नुवते।
त्रिः परि॑यन्ति।
त्रय॑ इ॒मे लो॒काः।
ए॒भ्य ए॒वैनं॑ लो॒केभ्यो॑ धुवते।
त्रिः पुनः॒ परि॑यन्ति।
षट्थ्सम्प॑द्यन्ते।
षड्वा ऋ॒तवः॑।
ऋ॒तुभि॑रे॒वैनं॑ धुवते।
अप॒ वा ए॒तेभ्यः॑ प्रा॒णाः क्रा॑मन्ति॥२६॥

%3.9.6.3
ये य॒ज्ञे धुव॑नं त॒न्वते᳚।
न॒व॒कृत्वः॒ परि॑यन्ति।
नव॒ वै पुरु॑षे प्रा॒णाः।
प्रा॒णाने॒वाऽऽत्मन्द॑धते।
नैभ्यः॑ प्रा॒णा अप॑क्रामन्ति।
अम्बे॒ अम्बा॒ल्यम्बि॑क॒ इति॒ पत्नी॑मु॒दान॑यति।
अह्व॑तै॒वैना᳚म्।
सुभ॑गे॒ काम्पी॑लवासि॒नीत्या॑ह।
तप॑ ए॒वैना॒मुप॑नयति।
सु॒व॒र्गे लो॒के सम्प्रोर्ण्वा॑था॒मित्या॑ह॥२७॥

%3.9.6.4
सु॒व॒र्गमे॒वैनां᳚ लो॒कं ग॑मयति।
आऽहम॑जानि गर्भ॒धमा त्वम॑जाऽसि गर्भ॒धमित्या॑ह।
प्र॒जा वै प॒शवो॒ गर्भः॑।
प्र॒जामे॒व प॒शूना॒त्मन्ध॑त्ते।
दे॒वा वा अ॑श्वमे॒धे पव॑माने।
सु॒व॒र्गं लो॒कं न प्राजा॑नन्।
तमश्वः॒ प्राजा॑नात्।
यथ्सू॒चीभि॑रसिप॒थान्क॒ल्पय॑न्ति।
सु॒व॒र्गस्य॑ लो॒कस्य॒ प्रज्ञा᳚त्यै।
गा॒य॒त्री त्रि॒ष्टुब्जग॒तीत्या॑ह॥२८॥

%3.9.6.5
य॒था॒\-य॒जु\-रे॒वै\-तत्।
त्र॒य्यः सू॒च्यो॑ भवन्ति।
अ॒य॒स्मय्यो॑ रज॒ता हरि॑ण्यः।
अ॒स्य वै लो॒कस्य॑ रू॒पम॑य॒स्मय्यः॑।
अ॒न्तरि॑क्षस्य रज॒ताः।
दि॒वो हरि॑ण्यः।
दिशो॒ वा अ॑य॒स्मय्यः॑।
अ॒वा॒न्त॒र॒दि॒शा र॑ज॒ताः।
ऊ॒र्ध्वा हरि॑ण्यः।
दिश॑ ए॒वास्मै॑ कल्पयति।
कस्त्वा᳚ छ्यति॒ कस्त्वा॒ विशा॒स्तीत्या॒हाहिꣳ॑सायै॥२९॥\anuvakamend[ह्नु॒व॒ते॒ क्रा॒म॒न्त्यू॒र्ण्वा॒था॒मित्या॑ह॒ जग॒तीत्या॑ह कल्पय॒त्येकं॑ च]

%3.9.7.1
अप॒ वा ए॒तस्मा॒च्छ्री रा॒ष्ट्रं क्रा॑मति।
यो᳚ऽश्वमे॒धेन॒ यज॑ते।
ऊ॒र्ध्वामे॑ना॒मुच्छ्र॑यता॒दित्या॑ह।
श्रीर्वै रा॒ष्ट्रम॑श्वमे॒धः।
श्रिय॑मे॒वास्मै॑ रा॒ष्ट्रमू॒र्ध्वमुच्छ्र॑यति।
वे॒णु॒भा॒रङ्गि॒रावि॒वेत्या॑ह।
रा॒ष्ट्रं वै भा॒रः।
रा॒ष्ट्रमे॒वास्मै॒ पर्यू॑हति।
अथा᳚स्या॒ मध्य॑मेधता॒मित्या॑ह।
श्रीर्वै रा॒ष्ट्रस्य॒ मध्यम्᳚॥३०॥

%3.9.7.2
श्रिय॑मे॒वाव॑ रुन्धे।
शी॒ते वाते॑ पु॒नन्नि॒वेत्या॑ह।
क्षेमो॒ वै रा॒ष्ट्रस्य॑ शी॒तो वातः॑।
क्षेम॑मे॒वाव॑ रुन्धे।
यद्ध॑रि॒णी यव॒मत्तीत्या॑ह।
विड्वै ह॑रि॒णी।
रा॒ष्ट्रं यवः॑।
विशं॑ चै॒वास्मै॑ रा॒ष्ट्रं च॑ स॒मीची॑ दधाति।
न पु॒ष्टं प॒शु म॑न्यत॒ इत्या॑ह।
तस्मा॒द्राजा॑ प॒शून्न पुष्य॑ति॥३१॥

%3.9.7.3
शू॒द्रा यदर्य॑जारा॒ न पोषा॑य धनाय॒तीत्या॑ह।
तस्मा᳚द्वैशीपु॒त्रं नाभिषि॑ञ्चन्ते।
इ॒यं य॒का श॑कुन्ति॒केत्या॑ह।
विड्वै श॑कुन्ति॒का।
रा॒ष्ट्रम॑श्वमे॒धः।
विशं॑ चै॒वास्मै॑ रा॒ष्ट्रं च॑ स॒मीची॑ दधाति।
आ॒हल॒मिति॒ सर्प॒तीत्या॑ह।
तस्मा᳚द्रा॒ष्ट्राय॒ विशः॑ सर्पन्ति।
आह॑तं ग॒भे पस॒ इत्या॑ह।
विड्वै गभः॑॥३२॥

%3.9.7.4
रा॒ष्ट्रं पसः॑।
रा॒ष्ट्रमे॒व वि॒श्याह॑न्ति।
तस्मा᳚द्रा॒ष्ट्रं विशं॒ घातु॑कम्।
मा॒ता च॑ ते पि॒ता च॑ त॒ इत्या॑ह।
इ॒यं वै मा॒ता।
अ॒सौ पि॒ता।
आ॒भ्यामे॒वैनं॒ परि॑ददाति।
अग्रं॑ वृ॒क्षस्य॑ रोहत॒ इत्या॑ह।
श्रीर्वै वृ॒क्षस्याग्रम्᳚।
श्रि॒य॑मे॒वाव॑ रुन्धे॥३३॥

%3.9.7.5
प्रसु॑ला॒मीति॑ ते पि॒ता ग॒भे मु॒ष्टिम॑तꣳसय॒दित्या॑ह।
विड्वै गभः॑।
रा॒ष्ट्रं मु॒ष्टिः।
रा॒ष्ट्रमे॒व वि॒श्याह॑न्ति।
तस्मा᳚द्रा॒ष्ट्रं विशं॒ घातु॑कम्।
अप॒ वा ए॒तेभ्यः॑ प्रा॒णाः क्रा॑मन्ति।
ये य॒ज्ञे\-ऽपू॑तं॒ वद॑न्ति।
द॒धि॒क्राव्ण्णो॑ अकारिष॒मिति॑ सुरभि॒मती॒मृचं॑ वदन्ति।
प्रा॒णा वै सु॑र॒भयः॑।
प्रा॒णाने॒वाऽऽत्मन्द॑धते।
नैभ्यः॑ प्रा॒णा अप॑क्रामन्ति।
आपो॒ हि ष्ठा म॑यो॒भुव॒ इत्य॒द्भिर्मा᳚र्जयन्ते।
आपो॒ वै सर्वा॑ दे॒वताः᳚।
दे॒वता॑भिरे॒वाऽऽत्मानं॑ पवयन्ते॥३४॥\anuvakamend[रा॒ष्ट्रस्य॒ मध्यं॒ पुष्य॑ति॒ गभो॑ रुन्धे दधते च॒त्वारि॑ च]

%3.9.8.1
प्र॒जा\-प॑तिः प्र॒जाः सृ॒ष्ट्वा प्रे॒णाऽनु॒ प्रावि॑शत्।
ताभ्यः॒ पुनः॒ सम्भ॑वितुं॒ नाश॑क्नोत्।
सो᳚ऽब्रवीत्।
ऋ॒ध्नव॒दिथ्सः।
यो मे॒तः पुनः॑ स॒म्भर॒दिति॑।
तं दे॒वा अ॑श्वमे॒धेनै॒व सम॑भरन्।
ततो॒ वै त आ᳚र्ध्नुवन्।
यो᳚ऽश्वमे॒धेन॒ यज॑ते।
प्र॒जा\-प॑तिमे॒व सम्भ॑रत्यृ॒ध्नोति॑।
पुरु॑ष॒माल॑भते॥३५॥

%3.9.8.2
वै॒रा॒जो वै पुरु॑षः।
वि॒राज॑मे॒वाल॑भते।
अथो॒ अन्नं॒ वै वि॒राट्।
अन्न॑मे॒वाव॑ रुन्धे।
अश्व॒माल॑भते।
प्रा॒जा॒प॒त्यो वा अश्वः॑।
प्र॒जा\-प॑तिमे॒वाल॑भते।
अथो॒ श्रीर्वा एक॑शफम्।
श्रिय॑मे॒वाव॑ रुन्धे।
गामाल॑भते॥३६॥

%3.9.8.3
य॒ज्ञो वै गौः।
य॒ज्ञमे॒वाल॑भते।
अथो॒ अन्नं॒ वै गौः।
अन्न॑मे॒वाव॑ रुन्धे।
अ॒जा॒वी आल॑भते भू॒म्ने।
अथो॒ पुष्टि॒र्वै भू॒मा।
पुष्टि॑मे॒वाव॑ रुन्धे।
पर्य॑ग्निकृतं॒ पुरु॑षं चार॒ण्याꣴश्चोथ्सृ॑ज॒न्त्यहिꣳ॑सायै।
उ॒भौ वा ए॒तौ प॒शू आल॑भ्येते।
यश्चा॑व॒मो यश्च॑ पर॒मः।
ते᳚ऽस्यो॒भये॑ य॒ज्ञे ब॒द्धाः।
अ॒भीष्टा॑ अ॒भिप्री॑ताः।
अ॒भिजि॑ता अ॒भिहु॑ता भवन्ति।
नैनं॑ द॒ङ्क्ष्णवः॑ प॒शवो॑ य॒ज्ञे ब॒द्धाः।
अ॒भीष्टा॑ अ॒भिप्री॑ताः।
अ॒भिजि॑ता अ॒भिहु॑ता हिꣳसन्ति।
यो᳚ऽश्वमे॒धेन॒ यज॑ते।
य उ॑ चैनमे॒वं वेद॑॥३७॥\anuvakamend[ल॒भ॒ते॒ गामाल॑भते पर॒मो᳚\-ऽष्टौ च॑]

%3.9.9.1
प्र॒थ॒मेन॒ वा ए॒ष स्तोमे॑न रा॒ध्वा।
च॒तु॒ष्टो॒मेन॑ कृ॒तेनाया॑ना॒मुत्त॒रे\-हन्॑।
ए॒क॒वि॒ꣳ॒शे प्र॑ति॒ष्ठायां॒ प्रति॑ तिष्ठति।
ए॒क॒वि॒ꣳ॒शात्प्र॑ति॒ष्ठाया॑ ऋ॒तून॒न्वारो॑हति।
ऋ॒तवो॒ वै पृ॒ष्ठानि॑।
ऋ॒तवः॑ संवथ्स॒रः।
ऋ॒तुष्वे॒व सं॑वथ्स॒रे प्र॑ति॒ष्ठाय॑।
दे॒वता॑ अ॒भ्यारो॑हति।
शक्व॑रयः पृ॒ष्ठं भ॑वन्त्य॒न्यद॑न्य॒च्छन्दः॑।
अ॒न्ये᳚\-ऽन्ये॒ वा ए॒ते प॒शव॒ आल॑भ्यन्ते॥३८॥

%3.9.9.2
उ॒तेव॑ ग्रा॒म्याः।
उ॒तेवा॑र॒ण्याः।
अह॑रे॒व रू॒पेण॒ सम॑र्धयति।
अथो॒ अह्न॑ ए॒वैष ब॒लिर्‌\mbox{}ह्रि॑यते।
तदा॑हुः।
अप॑शवो॒ वा ए॒ते।
यद॑जा॒वय॑श्चार॒ण्याश्च॑।
ए॒ते वै सर्वे॑ प॒शवः॑।
यद्ग॒व्या इति॑।
ग॒व्यान्प॒शूनु॑त्त॒मेऽहं॒ नाल॑भते॥३९॥

%3.9.9.3
तेनै॒वोभया᳚न्प॒शूनव॑ रुन्धे।
प्रा॒जा॒प॒त्या भ॑वन्ति।
अन॑भि\-जितस्या॒भि\-जि॑त्यै।
सौ॒रीर्नव॑ श्वे॒ता व॒शा अ॑नूब॒न्ध्या॑ भवन्ति।
अ॒न्त॒त ए॒व ब्र॑ह्म\-वर्च॒समव॑ रुन्धे।
सोमा॑य स्व॒राज्ञे॑\-ऽनोवा॒हाव॑न॒ड्वाहा॒विति॑ द्व॒न्द्विनः॑ प॒शूनाल॑भते।
अ॒हो॒रा॒त्राणा॑म॒भिजि॑त्यै।
प॒शुभि॒र्वा ए॒ष व्यृ॑ध्यते।
यो᳚ऽश्वमे॒धेन॒ यज॑ते।
छ॒ग॒लं क॒ल्माषं॑ किकिदी॒विं वि॑दी॒गय॒मिति॑ त्वा॒ष्ट्रान्प॒शूना ल॑भते।
प॒शुभि॑रे॒वाऽऽत्मान॒ꣳ॒ सम॑र्धयति।
ऋ॒तुभि॒र्वा ए॒ष व्यृ॑ध्यते।
यो᳚ऽश्वमे॒धेन॒ यज॑ते।
पि॒शङ्गा॒स्त्रयो॑ वास॒न्ता इत्यृ॑तुप॒शूनाल॑भते।
ऋ॒तुभि॑रे॒वाऽऽत्मान॒ꣳ॒ सम॑र्धयति।
आ वा ए॒ष प॒शुभ्यो॑ वृश्च्यते।
यो᳚ऽश्वमे॒धेन॒ यज॑ते।
पर्य॑ग्निकृता॒ उथ्सृ॑ज॒न्त्यना᳚व्रस्काय॥४०॥\anuvakamend[ल॒भ्य॒न्ते॒ ल॒भ॒ते॒ त्वा॒ष्ट्रान्प॒शूनाल॑भते॒\-ऽष्टौ च॑]

%3.9.10.1
प्र॒जा\-प॑तिरकामयत म॒हान॑न्ना॒दः स्या॒मिति॑।
स ए॒ताव॑श्वमे॒धे म॑हि॒माना॑वपश्यत्।
ताव॑गृह्णीत।
ततो॒ वै स म॒हान॑न्ना॒दो॑\-ऽभवत्।
यः का॒मये॑त म॒हान॑न्ना॒दः स्या॒मिति॑।
स ए॒ताव॑श्वमे॒धे म॑हि॒मानौ॑ गृह्णीत।
म॒हाने॒वान्ना॒दो भ॑वति।
य॒ज॒मा॒न॒दे॒व॒त्या॑ वै व॒पा।
राजा॑ महि॒मा।
यद्व॒पां म॑हि॒म्नोभ॒यतः॑ परि॒यज॑ति।
यज॑मानमे॒व रा॒ज्येनो॑भ॒यतः॒ परि॑गृह्णाति।
पु॒रस्ता᳚थ्स्वाहाकारा॒ वा अ॒न्ये दे॒वाः।
उ॒परि॑ष्टाथ्स्वाहाकारा अ॒न्ये।
ते वा ए॒ते\-ऽश्व॑ ए॒व मेध्य॑ उ॒भये\-ऽव॑रुध्यन्ते।
यद्व॒पां म॑हि॒म्नोभ॒यतः॑ परि॒यज॑ति।
ताने॒वोभया᳚न्प्रीणाति॥४१॥\anuvakamend[प॒रि॒यज॑ति॒ षट्च॑]

%3.9.11.1
वै॒श्व॒दे॒वो वा अश्वः॑।
तं यत्प्रा॑जाप॒त्यं कु॒र्यात्।
या दे॒वता॒ अपि॑भागाः।
ता भा॑ग॒धेये॑न॒ व्य॑र्धयेत्।
दे॒वता᳚भ्यः स॒मदं॑ दध्यात्।
स्ते॒गान्दꣴष्ट्रा᳚भ्यां म॒ण्डूकां॒ जम्भ्ये॑भि॒रिति॑।
आज्य॑मव॒दानं॑ कृ॒त्वा प्र॑तिस॒ङ्ख्याय॒माहु॑तीर्जुहोति।
या ए॒व दे॒वता॒ अपि॑भागाः।
ता भा॑ग॒धेये॑न॒ सम॑र्धयति।
न दे॒वता᳚भ्यः स॒मदं॑ दधाति॥४२॥

%3.9.11.2
चतु॑र्दशै॒तान॑नुवा॒काञ्जु॑हो॒त्यन॑न्तरित्यै।
प्र॒या॒साय॒ स्वाहेति॑ पञ्चद॒शम्।
पञ्च॑दश॒ वा अ॑र्धमा॒सस्य॒ रात्र॑यः।
अ॒र्ध॒मा॒स॒शः सं॑वथ्स॒र आ᳚प्यते।
दे॒वा॒सु॒राः संय॑त्ता आसन्।
ते᳚ऽब्रुवन्न॒ग्नयः॑ स्विष्ट॒कृतः॑।
अश्व॑स्य॒ मेध्य॑स्य व॒यमु॑द्धा॒रमुद्ध॑रामहै।
अथै॒तान॒भि भ॑वा॒मेति॑।
ते लोहि॑त॒मुद॑हरन्त।
ततो॑ दे॒वा अभ॑वन्॥४३॥

%3.9.11.3
पराऽसु॑राः।
यथ्स्वि॑ष्ट॒कृद्भ्यो॒ लोहि॑तं जु॒होति॒ भ्रातृ॑व्याऽभिभूत्यै।
भव॑त्या॒त्मना᳚।
परा᳚ऽस्य॒ भ्रातृ॑व्यो भवति।
गो॒मृ॒ग॒क॒ण्ठेन॑ प्रथ॒मामाहु॑तिं जुहोति।
प॒शवो॒ वै गो॑मृ॒गः।
रु॒द्रो᳚\-ऽग्निः स्वि॑ष्ट॒कृत्।
रु॒द्रादे॒व प॒शून॒न्तर्द॑धाति।
अथो॒ यत्रै॒षा\-ऽऽहु॑तिर्‌\mbox{}हू॒यते᳚।
न तत्र॑ रु॒द्रः प॒शून॒भिम॑न्यते॥४४॥

%3.9.11.4
अ॒श्व॒श॒फेन॑ द्वि॒तीया॒माहु॑तिं जुहोति।
प॒शवो॒ वा एक॑शफम्।
रु॒द्रो᳚\-ऽग्निः स्वि॑ष्ट॒कृत्।
रु॒द्रादे॒व प॒शून॒न्तर्द॑धाति।
अथो॒ यत्रै॒षा\-ऽऽहु॑तिर्‌\mbox{}हू॒यते᳚।
न तत्र॑ रु॒द्रः प॒शून॒भिम॑न्यते।
अ॒य॒स्मये॑न कम॒ण्डलु॑ना तृ॒तीया᳚म्।
आहु॑तिं जुहोत्याया॒स्यो॑ वै प्र॒जाः।
रु॒द्रो᳚\-ऽग्निः स्वि॑ष्ट॒कृत्।
रु॒द्रादे॒व प्र॒जा अ॒न्तर्द॑धाति।
अथो॒ यत्रै॒षा\-ऽऽहु॑तिर्‌\mbox{}हू॒यते᳚।
न तत्र॑ रु॒द्रः प्र॒जा अ॒भिम॑न्यते॥४५॥\anuvakamend[द॒धा॒त्यभ॑वन्मन्यते प्र॒जा अ॒न्तर्द॑धाति॒ द्वे च॑ ]

%3.9.12.1
अश्व॑स्य॒ वा आल॑ब्धस्य॒ मेध॒ उद॑क्रामत्।
तद॑श्वस्तो॒मीय॑\-मभवत्।
यद॑श्वस्तो॒मीयं॑ जु॒होति॑।
समे॑धमे॒वैन॒माल॑भते।
आज्ये॑न जुहोति।
मेधो॒ वा आज्यम्᳚।
मेधो᳚\-ऽश्वस्तो॒मीयम्᳚।
मेधे॑नै॒वास्मि॒न्मेधं॑ दधाति।
षट्त्रिꣳ॑शतं जुहोति।
षट्त्रिꣳ॑शदक्षरा बृह॒ती॥४६॥

%3.9.12.2
बार्‌\mbox{}ह॑ताः प॒शवः॑।
सा प॑शू॒नां मात्रा᳚।
प॒शूने॒व मात्र॑या॒ सम॑र्धयति।
तायद्भूय॑सीर्वा॒ कनी॑यसीर्वा जुहु॒यात्।
प॒शून्मात्र॑या॒ व्य॑र्धयेत्।
षट्त्रिꣳ॑शतं जुहोति।
षट्त्रिꣳ॑शदक्षरा बृह॒ती।
बार्‌\mbox{}ह॑ताः प॒शवः॑।
सा प॑शू॒नां मात्रा᳚।
प॒शूने॒व मात्र॑या॒ सम॑र्धयति॥४७॥

%3.9.12.3
अ॒श्व॒स्तो॒मीयꣳ॑ हु॒त्वा द्वि॒पदा॑ जुहोति।
द्वि॒पाद्वै पुरु॑षो॒ द्विप्र॑तिष्ठः।
तदे॑नं प्रति॒ष्ठया॒ सम॑र्धयति।
तदा॑हुः।
अ॒श्व॒स्तो॒मीयं॒ पूर्वꣳ॑ होत॒व्याँ(३)न्द्वि॒पदा(३) इति॑।
अश्वो॒ वा अ॑श्वस्तो॒मीयम्᳚।
पुरु॑षो द्वि॒पदाः᳚।
अ॒श्व॒स्तो॒मीयꣳ॑ हु॒त्वा द्वि॒पदा॑ जुहोति।
तस्मा᳚द्\-द्वि॒पाच्चतु॑ष्पादमत्ति।
अथो᳚ द्वि॒पद्ये॒व चतु॑ष्पदः॒ प्रति॑\-ष्ठापयति।
द्वि॒पदा॑ हु॒त्वा।
नान्यामुत्त॑रा॒माहु॑तिं जुहुयात्।
यद॒न्यामुत्त॑रा॒माहु॑तिं जुहु॒यात्।
प्र प्र॑ति॒ष्ठाया᳚श्च्यवेत।
द्वि॒पदा॑ अन्त॒तो जु॑होति॒ प्रति॑\-ष्ठित्यै॥४८॥\anuvakamend[बृ॒ह॒त्य॑र्धयति स्थापयति॒ पञ्च॑ च]

%3.9.13.1
प्र॒जा\-प॑तिरश्वमे॒धम॑\-सृजत।
सो᳚ऽस्माथ्सृ॒ष्टो\-ऽपा᳚क्रामत्।
तं य॑ज्ञ\-क्र॒तुभि॒रन्वै᳚च्छत्।
तं य॑ज्ञ\-क्र॒तुभि॒र्नान्व॑विन्दत्।
तमिष्टि॑भि॒रन्वै᳚च्छत्।
तमिष्टि॑भि॒रन्व॑विन्दत्।
तदिष्टी॑नामिष्टि॒\-त्वम्।
यथ्सं॑वथ्स॒रमिष्टि॑\-भि॒र्यज॑ते।
अश्व॑मे॒व तदन्वि॑च्छति।
सा॒वि॒त्रियो॑ भवन्ति॥४९॥

%3.9.13.2
इ॒यं वै स॑वि॒ता।
यो वा अ॒स्यान्नश्य॑ति॒ यो नि॒लय॑ते।
अ॒स्यां वाव तं वि॑न्दन्ति।
न वा इ॒मां कश्च॒नेत्या॑हुः।
ति॒र्यङ्नोर्ध्वोत्ये॑तुमर्ह॒तीति॑।
यथ्सा॑वि॒त्रियो॒ भव॑न्ति।
स॒वि॒तृ\-प्र॑सूत ए॒वैन॑मिच्छति।
ई॒श्व॒रो वा अश्वः॒ प्रमु॑क्तः॒ परां᳚ परा॒वतं॒ गन्तोः᳚।
यथ्सा॒यं धृती᳚र्जु॒होति॑।
अश्व॑स्य॒ यत्यै॒ धृत्यै᳚॥५०॥

%3.9.13.3
यत्प्रा॒तरिष्टि॑भि॒र्यज॑ते।
अश्व॑मे॒व तदन्वि॑च्छति।
यथ्सा॒यं धृती᳚र्जु॒होति॑।
अश्व॑स्यै॒व यत्यै॒ धृत्यै᳚।
तस्मा᳚थ्सा॒यं प्र॒जाः क्षे॒म्या॑ भवन्ति।
यत्प्रा॒तरिष्टि॑भि॒र्यज॑ते।
अश्व॑मे॒व तदन्वि॑च्छति।
तस्मा॒द्दिवा॑ नष्टै॒ष ए॑ति।
यत्प्रा॒तरिष्टि॑भि॒र्यज॑ते सा॒यं धृती᳚र्जु॒होति॑।
अ॒हो॒रा॒त्राभ्या॑मे॒वैन॒मन्वि॑च्छति।
अथो॑ अहोरा॒त्राभ्या॑मे॒वास्मै॑ योग\-क्षे॒मं क॑ल्पयति॥५१॥\anuvakamend[भ॒व॒न्ति॒ धृत्या॑ एन॒मन्वि॑च्छ॒त्येकं॑ च]

%3.9.14.1
अप॒ वा ए॒तस्मा॒च्छ्री रा॒ष्ट्रं क्रा॑मति।
यो᳚ऽश्वमे॒धेन॒ यज॑ते।
ब्रा॒ह्म॒णौ वी॑णागा॒थिनौ॑ गायतः।
श्रि॒या वा ए॒तद्रू॒पम्।
यद्वीणा᳚।
श्रिय॑मे॒वास्मि॒न्तद्ध॑त्तः।
य॒दा खलु॒ वै पुरु॑षः॒ श्रिय॑मश्ञु॒ते।
वीणा᳚\-ऽस्मै वाद्यते।
तदा॑हुः।
यदु॒भौ ब्रा᳚ह्म॒णौ गाये॑ताम्॥५२॥

%3.9.14.2
प्र॒भ्रꣳशु॑कास्मा॒च्छ्रीः स्या᳚त्।
न वै ब्रा᳚ह्म॒णे श्री र॑मत॒ इति॑।
ब्रा॒ह्म॒णो᳚\-ऽन्यो गाये᳚त्।
रा॒ज॒न्यो᳚\-ऽन्यः।
ब्रह्म॒ वै ब्रा᳚ह्म॒णः।
क्ष॒त्रꣳ रा॑ज॒न्यः॑।
तथा॑ हास्य॒ ब्रह्म॑णा च क्ष॒त्रेण॑ चोभ॒यतः॒ श्रीः परि॑गृहीता भवति।
तदा॑हुः।
यदु॒भौ दिवा॒ गाये॑ताम्।
अपा᳚स्माद्रा॒ष्ट्रं क्रा॑मेत्॥५३॥

%3.9.14.3
न वै ब्रा᳚ह्म॒णे रा॒ष्ट्रꣳ र॑मत॒ इति॑।
य॒दा खलु॒ वै राजा॑ का॒मय॑ते।
अथ॑ ब्राह्म॒णं जि॑नाति।
दिवा᳚ ब्राह्म॒णो गा॑येत्।
नक्तꣳ॑ राज॒न्यः॑।
ब्रह्म॑णो॒ वै रू॒पमहः॑।
क्ष॒त्रस्य॒ रात्रिः॑।
तथा॑ हास्य॒ ब्रह्म॑णा च क्ष॒त्रेण॑ चोभ॒यतो॑ रा॒ष्ट्रं परि॑गृहीतं भवति।
इत्य॑ददा॒ इत्य॑यजथा॒ इत्य॑पच॒ इति॑ ब्राह्म॒णो गाये᳚त्।
इ॒ष्टा॒पू॒र्तं वै ब्रा᳚ह्म॒णस्य॑॥५४॥

%3.9.14.4
इ॒ष्टा॒पू॒र्तेनै॒वैन॒ꣳ॒ स सम॑र्धयति।
इत्य॑जिना॒ इत्य॑युध्यथा॒ इत्य॒मुꣳ स॑ङ्ग्रा॒मम॑ह॒न्निति॑ राज॒न्यः॑।
यु॒द्धं वै रा॑ज॒न्य॑स्य।
यु॒द्धेनै॒वैन॒ꣳ॒ स सम॑र्धयति।
अकॢ॑प्ता॒ वा ए॒तस्य॒र्तव॒ इत्या॑हुः।
यो᳚ऽश्वमे॒धेन॒ यज॑त॒ इति॑।
ति॒स्रो᳚\-ऽन्यो गाय॑ति ति॒स्रो᳚\-ऽन्यः।
षट्थ्सम्प॑द्यन्ते।
षड्वा ऋ॒तवः॑।
ऋ॒तूने॒वास्मै॑ कल्पयतः।
ताभ्याꣳ॑ स॒ꣴ॒स्थाया᳚म्।
अ॒नो॒यु॒क्ते च॑ श॒ते च॑ ददाति।
श॒तायुः॒ पुरु॑षः श॒तेन्द्रि॑यः।
आयु॑ष्ये॒वेन्द्रि॒ये प्रति॑ तिष्ठति॥५५॥\anuvakamend[गाये॑ताङ्क्रामेद्ब्राह्म॒णस्य॑ कल्पयतश्च॒त्वारि॑ च]

%3.9.15.1
सर्वे॑षु॒ वा ए॒षु लो॒केषु॑ मृ॒त्यवो॒\-ऽन्वाय॑त्ताः।
तेभ्यो॒ यदाहु॑ती॒र्न जु॑हु॒यात्।
लो॒केलो॑क एनं मृ॒त्युर्वि॑न्देत्।
मृ॒त्यवे॒ स्वाहा॑ मृ॒त्यवे॒ स्वाहेत्य॑भिपू॒र्वमाहु॑तीर्जुहोति।
लो॒काल्लो॑कादे॒व मृ॒त्युमव॑यजते।
नैनं॑ लो॒केलो॑के मृ॒त्युर्वि॑न्दति।
यद॒मुष्मै॒ स्वाहा॒\-ऽमुष्मै॒ स्वाहेति॒ जुह्व॑थ्स॒ञ्चक्षी॑त।
ब॒हुं मृ॒त्युम॒मित्रं॑ कुर्वीत।
मृ॒त्यवे॒ स्वाहेत्येक॑स्मा ए॒वैकां᳚ जुहुयात्।
एको॒ वा अ॒मुष्मिँ॑ल्लो॒के मृ॒त्युः॥५६॥

%3.9.15.2
अ॒श॒न॒या॒ मृ॒त्युरे॒व।
तमे॒वामुष्मिँ॑ल्लो॒के\-ऽव॑यजते।
भ्रू॒ण॒ह॒त्यायै॒ स्वाहेत्य॑वभृ॒थ आहु॑तिं जुहोति।
भ्रू॒ण॒ह॒त्यामे॒वाव॑ यजते।
तदा॑हुः।
यद्भ्रू॑णह॒त्या पा॒त्र्याऽथ॑।
कस्मा᳚द्य॒ज्ञेऽपि॑ क्रियत॒ इति॑।
अमृ॑त्यु॒र्वा अ॒न्यो भ्रू॑णह॒त्याया॒ इत्या॑हुः।
भ्रू॒ण॒ह॒त्या वाव मृ॒त्युरिति॑।
यद्भ्रू॑णह॒त्यायै॒ स्वाहेत्य॑वभृ॒थ आहु॑तिं जु॒होति॑॥५७॥

%3.9.15.3
मृ॒त्युमे॒वाऽऽहु॑त्या तर्पयि॒त्वा प॑रि॒पाणं॑ कृ॒त्वा।
भ्रू॒ण॒घ्ने भे॑ष॒जं क॑रोति।
ए॒ताꣳ ह॒ वै मु॑ण्डि॒भ औ॑दन्य॒वः।
भ्रू॒ण॒ह॒त्यायै॒ प्राय॑श्चित्तिं वि॒दां च॑कार।
यो हा॒स्यापि॑ प्र॒जायां᳚ ब्राह्म॒णꣳ हन्ति॑।
सर्व॑स्मै॒ तस्मै॑ भेष॒जं क॑रोति।
जु॒म्ब॒काय॒ स्वाहेत्य॑वभृ॒थ उ॑त्त॒मामाहु॑तिं जुहोति।
वरु॑णो॒ वै जु॑म्ब॒कः।
अ॒न्त॒त ए॒व वरु॑ण॒मव॑यजते।
ख॒ल॒तेर्वि॑क्लि॒धस्य॑ शु॒क्लस्य॑ पिङ्गा॒क्षस्य॑ मू॒र्धं जु॑होति।
ए॒तद्वै वरु॑णस्य रू॒पम्।
रू॒पेणै॒व वरु॑ण॒मव॑यजते॥५८॥\anuvakamend[लो॒के मृ॒त्युर्जु॒होति॑ मू॒र्धं जु॑होति॒ द्वे च॑]

%3.9.16.1
वा॒रु॒णो वा अश्वः॑।
तं दे॒वत॑या॒ व्य॑र्धयति।
यत्प्रा॑जाप॒त्यं क॒रोति॑।
नमो॒ राज्ञे॒ नमो॒ वरु॑णा॒येत्या॑ह।
वा॒रु॒णो वा अश्वः॑।
स्वयै॒वैनं॑ दे॒वत॑या॒ सम॑र्धयति।
नमोऽश्वा॑य॒ नमः॑ प्र॒जा\-प॑तय॒ इत्या॑ह।
प्रा॒जा॒प॒त्यो वा अश्वः॑।
स्वयै॒वैनं॑ दे॒वत॑या॒ सम॑र्धयति।
नमोऽधि॑पतय॒ इत्या॑ह॥५९॥

%3.9.16.2
धर्मो॒ वा अधि॑पतिः।
धर्म॑मे॒वाव॑ रुन्धे।
अधि॑पतिर॒स्यधि॑पतिं मा कु॒र्वधि॑पतिर॒हं प्र॒जानां᳚ भूयास॒मित्या॑ह।
अधि॑पतिमे॒वैनꣳ॑ समा॒नानां᳚ करोति।
मां धे॑हि॒ मयि॑ धे॒हीत्या॑ह।
आ॒शिष॑\-मे॒वैतामा शा᳚स्ते।
उ॒पाकृ॑ताय॒ स्वाहेत्यु॒पाकृ॑ते जुहोति।
आल॑ब्धाय॒ स्वाहेति॒ नियु॑क्ते जुहोति।
हु॒ताय॒ स्वाहेति॑ हु॒ते जु॑होति।
ए॒षां लो॒काना॑म॒भिजि॑त्यै॥६०॥

%3.9.16.3
प्र वा ए॒ष ए॒भ्यो लो॒केभ्य॑श्च्यवते।
यो᳚ऽश्वमे॒धेन॒ यज॑ते।
आ॒ग्ने॒यमै᳚न्द्रा॒ग्नमा᳚श्वि॒नम्।
तान्प॒शूनाल॑भते॒ प्रति॑\-ष्ठित्यै।
यदा᳚ग्ने॒यो भव॑ति।
अ॒ग्निः सर्वा॑ दे॒वताः᳚।
दे॒वता॑ ए॒वाव॑ रुन्धे।
ब्रह्म॒ वा अ॒ग्निः।
क्ष॒त्रमिन्द्रः॑।
यदै᳚न्द्रा॒ग्नो भव॑ति॥६१॥

%3.9.16.4
ब्र॒ह्म॒क्ष॒त्रे ए॒वाव॑ रुन्धे।
यदा᳚श्वि॒नो भव॑ति।
आ॒शिषा॒मव॑रुद्ध्यै।
त्रयो॑ भवन्ति।
त्रय॑ इ॒मे लो॒काः।
ए॒ष्वे॑व लो॒केषु॒ प्रति॑ तिष्ठति।
अ॒ग्नये\-ऽꣳ॑हो॒मुचे॒\-ऽष्टा\-क॑पाल॒ इति॒ दश॑हविष॒मिष्टिं॒ निर्व॑पति।
दशा᳚क्षरा वि॒राट्।
अन्नं॑ वि॒राट्।
वि॒राजै॒वान्नाद्य॒मव॑ रुन्धे।
अ॒ग्नेर्म॑न्वे प्रथ॒मस्य॒ प्रचे॑तस॒ इति॑ याज्यानुवा॒क्या॑ भवन्ति सर्व॒त्वाय॑॥६२॥\anuvakamend[अधि॑पतय॒ इत्या॑हा॒भि॑जित्या ऐन्द्रा॒ग्नो भव॑ति रुन्ध॒ एकं॑ च]

%3.9.17.1
यद्यश्व॑मुप॒तप॑द्वि॒न्देत्।
आ॒ग्ने॒यम॒ष्टा\-क॑पालं॒ निर्व॑पेत्।
सौ॒म्यं च॒रुम्।
सा॒वि॒त्रम॒ष्टा\-क॑पालम्।
यदा᳚ग्ने॒यो भव॑ति।
अ॒ग्निः सर्वा॑ दे॒वताः᳚।
दे॒वता॑भिरे॒वैनं॑ भिषज्यति।
यथ्सौ॒म्यो भव॑ति।
सोमो॒ वा ओष॑धीना॒ꣳ॒ राजा᳚।
याभ्य॑ ए॒वैनं॑ वि॒न्दति॑॥६३॥

%3.9.17.2
ताभि॑रे॒वैनं॑ भिषज्यति।
यथ्सा॑वि॒त्रो भव॑ति।
स॒वि॒तृप्र॑सूत ए॒वैनं॑ भिषज्यति।
ए॒ताभि॑रे॒वैनं॑ दे॒वता॑भिर्भिषज्यति।
अ॒ग॒दो है॒व भ॑वति।
पौ॒ष्णं च॒रुं निर्व॑पेत्।
यदि॑ श्लो॒णः स्यात्।
पू॒षा वै श्लौण्य॑स्य भि॒षक्।
स ए॒वैनं॑ भिषज्यति।
अश्लो॑णो है॒व भ॑वति॥६४॥

%3.9.17.3
रौ॒द्रं च॒रुं निर्व॑पेत्।
यदि॑ मह॒ती दे॒वता॑\-ऽभि॒मन्ये॑त।
ए॒त॒द्दे॒व॒त्यो॑ वा अश्वः॑।
स्वयै॒वैनं॑ दे॒वत॑या भिषज्यति।
अ॒ग॒दो है॒व भ॑वति।
वै॒श्वा॒न॒रं द्वाद॑शकपालं॒ निर्व॑पेन्मृगाख॒रे यदि॒ नाऽऽगच्छे᳚त्।
इ॒यं वा अ॒ग्निर्वै᳚श्वान॒रः।
इ॒यमे॒वैन॑म॒र्चिभ्यां᳚ परि॒रोध॒मान॑यति।
आहै॒व सुत्य॒मह॑र्गच्छति।
यद्य॑धी॒यात्॥६५॥

%3.9.17.4
अ॒ग्नये\-ऽꣳ॑हो॒मुचे॒\-ऽष्टा\-क॑पालः।
सौ॒र्यं पयः॑।
वा॒य॒व्य॑ आज्य॑भागः।
यज॑मानो॒ वा अश्वः॑।
अꣳह॑सा॒ वा ए॒ष गृ॑ही॒तः।
यस्याश्वो॒ मेधा॑य॒ प्रोक्षि॑तो॒\-ऽध्येति॑।
यदꣳ॑हो॒मुचे॑ नि॒र्वप॑ति।
अꣳह॑स ए॒व तेन॑ मुच्यते।
यज॑मानो॒ वा अश्वः॑।
रेत॑सा॒ वा ए॒ष व्यृ॑ध्यते॥६६॥

%3.9.17.5
यस्याश्वो॒ मेधा॑य॒ प्रोक्षि॑तो॒\-ऽध्येति॑।
सौ॒र्यꣳ रेतः॑।
यथ्सौ॒र्यं पयो॒ भव॑ति।
रेत॑सै॒वैन॒ꣳ॒ स सम॑र्धयति।
यज॑मानो॒ वा अश्वः॑।
गर्भै॒र्वा ए॒ष व्यृ॑ध्यते।
यस्याश्वो॒ मेधा॑य॒ प्रोक्षि॑तो॒\-ऽध्येति॑।
वा॒य॒व्या॑ गर्भाः᳚।
यद्वा॑य॒व्य॑ आज्य॑भागो॒ भव॑ति।
गर्भै॑रे॒वैन॒ꣳ॒ स सम॑र्धयति।
अथो॒ यस्यै॒षा\-ऽश्व॑मे॒धे प्राय॑श्चित्तिः क्रि॒यते᳚।
इ॒ष्ट्वा वसी॑यान्भवति॥६७॥\anuvakamend[वि॒न्दत्यश्लो॑णो है॒व भ॑वत्यधी॒यादृ॑ध्यते॒ गर्भै॑रे॒वैन॒ꣳ॒ स सम॑र्धयति॒ द्वे च॑]

%3.9.18.1
तदा॑हुः।
द्वाद॑श ब्रह्मौद॒नान्थ्सꣴस्थि॑ते॒ निर्व॑पेत्।
द्वा॒द॒शभि॒र्वेष्टि॑\-भिर्यजे॒तेति॑।
यदिष्टि॑भि॒र्यजे॑त।
उ॒प॒नामु॑क एनं य॒ज्ञः स्या᳚त्।
पापी॑या॒ꣴ॒स्तु स्या᳚त्।
आ॒प्तानि॒ वा ए॒तस्य॒ छन्दाꣳ॑सि।
य ई॑जा॒नः।
तानि॒ क ए॒ताव॑दाशु॒ पुनः॒ प्रयु॑ञ्जी॒तेति॑।
सर्वा॒ वै सꣴस्थि॑ते य॒ज्ञे वागा᳚प्यते॥६८॥

%3.9.18.2
साप्ता भ॑वति या॒तया᳚म्नी।
क्रू॒रीकृ॑तेव॒ हि भव॒त्यरु॑ष्कृता।
सा न पुनः॑ प्र॒युज्येत्या॑हुः।
द्वाद॑शै॒व ब्र॑ह्मौद॒नान्थ्सꣴस्थि॑ते॒ निर्व॑पेत्।
प्र॒जा\-प॑ति॒र्वा ओ॑द॒नः।
य॒ज्ञः प्र॒जा\-प॑तिः।
उ॒प॒नामु॑क एनं य॒ज्ञो भ॑वति।
न पापी॑यान्भवति।
द्वाद॑श भवन्ति।
द्वाद॑श॒मासाः᳚ संवथ्स॒रः।
सं॒व॒थ्स॒र ए॒व प्रति॑ तिष्ठति॥६९॥\anuvakamend[आ॒प्य॒ते॒ सं॒व॒थ्स॒र एकं॑ च]

%3.9.19.1
ए॒ष वै वि॒भूर्नाम॑ य॒ज्ञः।
सर्वꣳ॑ ह॒ वै तत्र॑ वि॒भु भ॑वति।
यत्रै॒\-तेन॑ य॒ज्ञेन॒ यज॑न्ते।
ए॒ष वै प्र॒भूर्नाम॑ य॒ज्ञः।
सर्वꣳ॑ ह॒ वै तत्र॑ प्र॒भु भ॑वति।
यत्रै॒तेन॑ य॒ज्ञेन॒ यज॑न्ते।
ए॒ष वा ऊर्ज॑स्वा॒न्नाम॑ य॒ज्ञः।
सर्वꣳ॑ ह॒ वै तत्रो\-र्ज॑\-स्वद्\-भवति।
यत्रै॒तेन॑ य॒ज्ञेन॒ यज॑न्ते।
ए॒ष वै पय॑स्वा॒न्नाम॑ य॒ज्ञः॥७०॥

%3.9.19.2
सर्वꣳ॑ ह॒ वै तत्र॒ पय॑स्वद्भवति।
यत्रै॒तेन॑ य॒ज्ञेन॒ यज॑न्ते।
ए॒ष वै विधृ॑तो॒ नाम॑ य॒ज्ञः।
सर्वꣳ॑ ह॒ वै तत्र॒ विधृ॑तं भवति।
यत्रै॒तेन॑ य॒ज्ञेन॒ यज॑न्ते।
ए॒ष वै व्यावृ॑त्तो॒ नाम॑ य॒ज्ञः।
सर्वꣳ॑ ह॒ वै तत्र॒ व्यावृ॑त्तं भवति।
यत्रै॒तेन॑ य॒ज्ञेन॒ यज॑न्ते।
ए॒ष वै प्रति॑\-ष्ठितो॒ नाम॑ य॒ज्ञः।
सर्वꣳ॑ ह॒ वै तत्र॒ प्रति॑\-ष्ठितं भवति॥७१॥

%3.9.19.3
यत्रै॒तेन॑ य॒ज्ञेन॒ यज॑न्ते।
ए॒ष वै ते॑ज॒स्वी नाम॑ य॒ज्ञः।
सर्वꣳ॑ ह॒ वै तत्र॑ तेज॒स्वि भ॑वति।
यत्रै॒तेन॑ य॒ज्ञेन॒ यज॑न्ते।
ए॒ष वै ब्र॑ह्म\-वर्च॒सी नाम॑ य॒ज्ञः।
आ ह॒ वै तत्र॑ ब्राह्म॒णो ब्र॑ह्म\-वर्च॒सी जा॑यते।
यत्रै॒तेन॑ य॒ज्ञेन॒ यज॑न्ते।
ए॒ष वा अ॑तिव्या॒धी नाम॑ य॒ज्ञः।
आ ह॒ वै तत्र॑ राज॒न्यो॑\-ऽतिव्या॒धी जा॑यते।
यत्रै॒तेन॑ य॒ज्ञेन॒ यज॑न्ते।
ए॒ष वै दी॒र्घो नाम॑ य॒ज्ञः।
दी॒र्घायु॑षो ह॒ वै तत्र॑ मनु॒ष्या॑ भवन्ति।
यत्रै॒तेन॑ य॒ज्ञेन॒ यज॑न्ते।
ए॒ष वै कॢ॒प्तो नाम॑ य॒ज्ञः।
कल्प॑ते ह॒ वै तत्र॑ प्र॒जाभ्यो॑ योगक्षे॒मः।
यत्रै॒तेन॑ य॒ज्ञेन॒ यज॑न्ते॥७२॥\anuvakamend[पय॑स्वा॒न्नाम॑ य॒ज्ञः प्रति॑\-ष्ठितं भवति॒ यत्रै॒तेन॑ य॒ज्ञेन॒ यज॑न्ते॒ षट्च॑ (ए॒ष वै विभूः प्र॒भूरूर्ज॑स्वा॒न्पय॑स्वा॒न् विधृ॑तो॒ व्यावृ॑त्तः॒ प्रति॑\-ष्ठितस्तेज॒स्वी ब्र॑ह्म\-वर्च॒स्य॑तिव्या॒धी दी॒र्घः कॢ॒प्तो द्वाद॑श॥)]

%3.9.20.1
ता॒र्प्येणाश्व॒ꣳ॒ संज्ञ॑पयन्ति।
य॒ज्ञो वै ता॒र्प्यम्।
य॒ज्ञेनै॒वैन॒ꣳ॒ सम॑र्धयन्ति।
या॒मेन॒ साम्ना᳚ प्रस्तो॒ता\-ऽनूप॑तिष्ठते।
य॒म॒लो॒कमे॒वैनं॑ गमयति।
ता॒र्प्ये च॑ कृत्यधीवा॒से चाश्व॒ꣳ॒ संज्ञ॑पयन्ति।
ए॒तद्वै प॑शू॒नाꣳ रू॒पम्।
रू॒पेणै॒व प॒शूनव॑ रुन्धे।
हि॒र॒ण्य॒क॒शि॒पु भ॑वति।
तेज॒सो\-ऽव॑रुद्ध्यै॥७३॥

%3.9.20.2
रु॒क्मो भ॑वति।
सु॒व॒र्गस्य॑ लो॒कस्यानु॑ख्यात्यै।
अश्वो॑ भवति।
प्र॒जा\-प॑ते॒राप्त्यै᳚।
अ॒स्य वै लो॒कस्य॑ रू॒पं ता॒र्प्यम्।
अ॒न्तरि॑क्षस्य कृत्यधीवा॒सः।
दि॒वो हि॑रण्यकशि॒पु।
आ॒दि॒त्यस्य॑ रु॒क्मः।
प्र॒जा\-प॑ते॒रश्वः॑।
इ॒ममे॒व लो॒कं ता॒र्प्येणा᳚ऽऽप्नोति॥७४॥

%3.9.20.3
अ॒न्तरि॑क्षं कृत्यधीवा॒सेन॑।
दिवꣳ॑ हिरण्यकशि॒पुना᳚।
आ॒दि॒त्यꣳ रु॒क्मेण॑।
अश्वे॑नै॒व मेध्ये॑न प्र॒जा\-प॑तेः॒ सायु॑ज्यꣳ स\-लो॒क\-ता॑\-माप्नोति।
ए॒तासा॑मे॒व दे॒वता॑ना॒ꣳ॒ सायु॑ज्यम्।
सा॒र्ष्टिताꣳ॑ समान\-लो॒क\-ता॑\-माप्नोति।
यो᳚ऽश्वमे॒धेन॒ यज॑ते।
य उ॑ चैनमे॒वं वेद॑॥७५॥\anuvakamend[अव॑रुध्या आप्नोत्य॒ष्टौ च॑]

%3.9.21.1
आ॒दि॒त्याश्चाङ्गि॑रसश्च सुव॒र्गे लो॒के᳚\-ऽस्पर्धन्त।
तेऽङ्गि॑रस आदि॒त्येभ्यः॑।
अ॒मुमा॑दि॒त्यमश्वꣴ॑ श्वे॒तं भू॒तं दक्षि॑णामनयन्।
ते᳚ऽब्रुवन्।
यन्नो ने᳚ष्ट।
स वर्यो॑ भू॒दिति॑।
तस्मा॒दश्व॒ꣳ॒ सव॒र्येत्याह्व॑यन्ति।
तस्मा᳚द्य॒ज्ञे वरो॑ दीयते।
यत्प्र॒जा\-प॑ति॒रा\-ल॒ब्धो\-ऽश्वो\-ऽभ॑वत्।
तस्मा॒दश्वो॒ नाम॑॥७६॥

%3.9.21.2
यच्छ्वय॒दरु॒रासी᳚त्।
तस्मा॒दर्वा॒ नाम॑।
यथ्स॒द्यो वाजा᳚न्थ्स॒म\-ज॑यत्।
तस्मा᳚द्वा॒जी नाम॑।
यदसु॑राणां लो॒कानाद॑त्त।
तस्मा॑दादि॒त्यो नाम॑।
अ॒ग्निर्वा अ॑श्वमे॒धस्य॒ योनि॑रा॒\-यत॑नम्।
सूर्यो॒\-ऽग्नेर्योनि॑रा॒\-यत॑नम्।
यद॑श्वमे॒धे᳚\-ऽग्नौ चित्य॑ उत्तरवे॒दिमु॑प॒वप॑ति।
योनि॑मन्तमे॒वैन॑मा॒यत॑नवन्तं करोति॥७७॥

%3.9.21.3
योनि॑माना॒यत॑नवान्भवति।
य ए॒वं वेद॑।
प्रा॒णा॒पा॒नौ वा ए॒तौ दे॒वाना᳚म्।
यद॑र्काश्वमे॒धौ।
प्रा॒णा॒पा॒नावे॒वाव॑ रुन्धे।
ओजो॒ बलं॒ वा ए॒तौ दे॒वाना᳚म्।
यद॑र्काश्वमे॒धौ।
ओजो॒ बल॑मे॒वाव॑ रुन्धे।
अ॒ग्निर्वा अ॑श्वमे॒धस्य॒ योनि॑रा॒यत॑नम्।
सूर्यो॒\-ऽग्नेर्योनि॑रा॒\-यत॑नम्।
यद॑श्वमे॒धे᳚\-ऽग्नौ चित्य॑ उत्तरवे॒दिं चि॒नोति॑।
ताव॑र्काश्वमे॒धौ।
अ॒र्का॒श्व॒मे॒धावे॒वाव॑ रुन्धे।
अथो॑ अर्काश्वमे॒धयो॑रे॒व प्रति॑ तिष्ठति॥७८॥\anuvakamend[नाम॑ करोति॒ सूर्यो॒\-ऽग्नेर्योनि॑रा॒यत॑नञ्च॒त्वारि॑ च]

%3.9.22.1
प्र॒जा\-प॑तिं॒ वै दे॒वाः पि॒तरम्᳚।
प॒शुं भू॒तं मेधा॒याऽऽऽल॑भन्त।
तमा॒लभ्योपा॑वसन्।
प्रा॒तर्यष्टा᳚स्मह॒ इति॑।
एकं॒ वा ए॒तद्दे॒वाना॒महः॑।
यथ्सं॑वथ्स॒रः।
तस्मा॒दश्वः॑ पु॒रस्ता᳚थ्संवथ्स॒र आल॑भ्यते।
यत्प्र॒जा\-प॑ति॒\-रा\-ल॒ब्धो\-ऽश्वो\-ऽभ॑वत्।
तस्मा॒दश्वः॑।
यथ्स॒द्यो मेधो\-ऽभ॑वत्॥७९॥

%3.9.22.2
तस्मा॑दश्वमे॒धः।
वेदु॒को\-ऽश्व॑मा॒शुं भ॑वति।
य ए॒वं वेद॑।
यद्वै तत्प्र॒जा\-प॑ति॒राल॒ब्धो\-ऽश्वो\-ऽभ॑वत्।
तस्मा॒दश्वः॑ प्र॒जा\-प॑तेः पशू॒नामनु॑रूपतमः।
आऽस्य॑ पु॒त्रः प्रति॑\-रूपो जायते।
य ए॒वं वेद॑।
सर्वा॑णि भू॒तानि॑ स॒म्भृत्याऽऽल॑भते।
समे॑नं दे॒वास्तेज॑से ब्रह्मवर्च॒साय॑ भरन्ति।
यो᳚ऽश्वमे॒धेन॒ यज॑ते॥८०॥

%3.9.22.3
य उ॑ चैनमे॒वं वेद॑।
ए॒तद्वै तद्दे॒वा ए॒तान्दे॒वता᳚म्।
प॒शुं भू॒तं मेधा॒याऽऽऽल॑भन्त।
य॒ज्ञमे॒व।
य॒ज्ञेन॑ य॒ज्ञम॑यजन्त दे॒वाः।
का॒म॒प्रं य॒ज्ञम॑कुर्वत।
ते॑ऽमृत॒त्वम॑कामयन्त।
ते॑ऽमृत॒त्वम॑गच्छन्।
यो᳚ऽश्वमे॒धेन॒ यज॑ते।
दे॒वाना॑मे॒वाय॑नेनैति॥८१॥

%3.9.22.4
प्रा॒जा॒प॒त्येनै॒व य॒ज्ञेन॑ यजते काम॒प्रेण॑।
अपु॑नर्मारमे॒व ग॑च्छति।
ए॒तस्य॒ वै रू॒पेण॑ पु॒रस्ता᳚त्प्राजाप॒त्यमृ॑ष॒भं तू॑प॒रं ब॑हुरू॒पमाल॑भते।
सर्वे᳚भ्यः॒ कामे᳚भ्यः।
सर्व॒स्याऽऽप्त्यै᳚।
सर्व॑स्य॒ जित्यै᳚।
सर्व॑मे॒व तेना᳚\-ऽऽ\-प्नोति।
सर्वं॑ जयति।
यो᳚ऽश्वमे॒धेन॒ यज॑ते।
य उ॑ चैनमे॒वं वेद॑॥८२॥\anuvakamend[मेधो\-ऽभ॑व॒द्यज॑त एति॒ वेद॑]

%3.9.23.1
यो वा अश्व॑स्य॒ मेध्य॑स्य॒ लोम॑नी॒ वेद॑।
अश्व॑स्यै॒व मेध्य॑स्य॒ लोमं॑ लोमं जुहोति।
अ॒हो॒रा॒त्रे वा अश्व॑स्य॒ मेध्य॑स्य॒ लोम॑नी।
यथ्सा॒यं प्रा॑तर्जु॒होति॑।
अश्व॑स्यै॒व मेध्य॑स्य॒ लोमं॑ लोमं जुहोति।
ए॒तद॑नुकृति ह स्म॒ वै पु॒रा।
अश्व॑स्य॒ मेध्य॑स्य॒ लोमं॑ लोमं जुह्वति।
यो वा अश्व॑स्य॒ मेध्य॑स्य प॒दे वेद॑।
अश्व॑स्यै॒व मेध्य॑स्य प॒देप॑दे जुहोति।
द॒र्॒श॒पू॒र्ण॒मा॒सौ वा अश्व॑स्य॒ मेध्य॑स्य प॒दे॥८३॥

%3.9.23.2
यद्द॑र्‌\mbox{}शपूर्णमा॒सौ यज॑ते।
अश्व॑स्यै॒व मेध्य॑स्य प॒देप॑दे जुहोति।
ए॒तद॑नुकृति ह स्म॒ वै पु॒रा।
अश्व॑स्य॒ मेध्य॑स्य प॒देप॑दे जुह्वति।
यो वा अश्व॑स्य॒ मेध्य॑स्य वि॒वर्त॑नं॒ वेद॑।
अश्व॑स्यै॒व मेध्य॑स्य वि॒वर्त॑नेविवर्तने जुहोति।
अ॒सौ वा आ॑दि॒त्यो\-ऽश्वः॑।
स आ॑हव॒नीय॒माग॑च्छति।
तद्विव॑र्तते।
यद॑ग्निहो॒त्रं जु॒होति॑।
अश्व॑स्यै॒व मेध्य॑स्य वि॒वर्त॑नेविवर्तने जुहोति।
ए॒तद॑नुकृति ह स्म॒ वै पु॒रा।
अ॑श्वस्य॒ मेध्य॑स्य वि॒वर्त॑नेविवर्तने जुह्वति॥८४॥\anuvakamend[प॒दे अ॑ग्निहो॒त्रं जु॒होति॒ त्रीणि॑ च]

\prashnaend{प्र॒जा\-प॑ति॒स्तम॑ष्टादशिभिः॑ प्र॒जा\-प॑तिरकामयतो॒भाव॒स्मै यु॒ञ्जन्ति॒ तेज॒सा\-ऽप॑प्राणा अप॒श्रीरू॒र्ध्वां प्र॒जा\-प॑तिः प्रे॒णाऽनु॑ प्रथ॒मेन॑ प्र॒जा\-प॑तिरकामयत म॒हान्वै᳚श्वदे॒वो वा अश्वो\-ऽश्व॑स्य प्र॒जा\-प॑ति॒स्तं य॑ज्ञक्र॒तुभि॒रप॒श्रीर्ब्रा᳚ह्म॒णौ सर्वे॑षु वारु॒णो यद्यश्व॒न्तदा॑हुरे॒ष वै वि॒भूस्ता॒र्प्येणा॑दि॒त्याः प्र॒जा\-प॑तिं पि॒तरं॒ यो वा अश्व॑स्य॒ मेध्य॑स्य॒ लोम॑नी॒ त्रयो॑विꣳशतिः॥२३॥}{प्र॒जा\-प॑तिर॒स्मिँल्लो॒क उ॑त्तर॒तः श्रिय॑मे॒व प्र॒जा\-प॑तिरकामयत म॒हान्यत्प्रा॒तः प्र वा ए॒ष ए॒भ्यो लो॒केभ्यः॒ सर्वꣳ॑ ह॒ वै तत्र॒ पयः॑ स्व॒द्य उ॑ चैनमे॒वं वेद॑ च॒त्वार्यशी॑तिः॥८४॥}{प्र॒जा\-प॑तिरश्वमे॒धं जु॑ह्वति॥}{हरिः॑ ओम्॥}{इति श्रीकृष्णयजुर्वेदीयतैत्तिरीयब्राह्मणे तृतीयाष्टके नवमः प्रपाठकः समाप्तः॥}
\clearpage
%%% END ASHTAKAM
