\chapt{काण्डम् ६}
\sect{पञ्चमः प्रश्नः}\setcounter{anuvakam}{0}
\dnsub{तैत्तिरीयसंहितायां षष्ठमकाण्डे पञ्चमः प्रश्नः}
%6.5.1.1
इन्द्रो॑ वृ॒त्राय॒ वज्र॒मुद॑यच्छ॒थ्स वृ॒त्रो वज्रा॒दुद्य॑तादबिभे॒थ्सो᳚\-ऽब्रवी॒न्मा मे॒ प्र हा॒रस्ति॒ वा इ॒दं मयि॑ वी॒र्यं॑ तत्ते॒ प्र दा᳚स्या॒मीति॒ तस्मा॑ उ॒क्थ्यं॑ प्राय॑च्छ॒त्तस्मै᳚ द्वि॒तीय॒मुद॑यच्छ॒थ्सो᳚\-ऽब्रवी॒न्मा मे॒ प्र हा॒रस्ति॒ वा इ॒दं मयि॑ वी॒र्यं॑ तत्ते॒ प्र दा᳚स्या॒मीति॑~(१)\ip

%6.5.1.2
तस्मा॑ उ॒क्थ्य॑मे॒व प्राय॑च्छ॒त्तस्मै॑ तृ॒तीय॒मुद॑यच्छ॒त्तं विष्णु॒रन्व॑तिष्ठत ज॒हीति॒ सो᳚\-ऽब्रवी॒न्मा मे॒ प्र हा॒रस्ति॒ वा इ॒दं मयि॑ वी॒र्यं॑ तत्ते॒ प्र दा᳚स्या॒मीति॒ तस्मा॑ उ॒क्थ्य॑मे॒व प्राय॑च्छ॒त्तं निर्मा॑यं भू॒तम॑हन् य॒ज्ञो हि तस्य॑ मा॒यासी॒द्यदु॒क्थ्यो॑ गृ॒ह्यत॑ इन्द्रि॒यमे॒व~(२)\ip

%6.5.1.3
तद्वी॒र्यं॑ यज॑मानो॒ भ्रातृ॑व्यस्य वृङ्क्त॒ इन्द्रा॑य त्वा बृ॒हद्व॑ते॒ वय॑स्वत॒ इत्या॒हेन्द्रा॑य॒ हि स तं प्राय॑च्छ॒त्तस्मै᳚ त्वा॒ विष्ण॑वे॒ त्वेत्या॑ह॒ यदे॒व विष्णु॑र॒न्वति॑ष्ठत ज॒हीति॒ तस्मा॒द्विष्णु॑म॒न्वाभ॑जति॒ त्रिर्निर्गृ॑ह्णाति॒ त्रिर्\mbox{}हि स तं तस्मै॒ प्राय॑च्छदे॒ष ते॒ योनिः॒ पुन॑र्\mbox{}हविर॒सीत्या॑ह॒ पुनः॑पुनः~(३)\ip

%6.5.1.4
ह्य॑स्मान्निर्गृ॒ह्णाति॒ चक्षु॒र्वा ए॒तद्य॒ज्ञस्य॒ यदु॒क्थ्य॑स्तस्मा॑दु॒क्थ्यꣳ॑ हु॒तꣳ सोमा॑ अ॒न्वाय॑न्ति॒ तस्मा॑दा॒त्मा चक्षु॒रन्वे॑ति॒ तस्मा॒देकं॒ यन्तं॑ ब॒हवो\-ऽनु॑ यन्ति॒ तस्मा॒देको॑ बहू॒नां भ॒द्रो भ॑वति॒ तस्मा॒देको॑ ब॒ह्वीर्जा॒या वि॑न्दते॒ यदि॑ का॒मये॑ताध्व॒र्युरा॒त्मानं॑ यज्ञयश॒सेना᳚र्पयेय॒मित्य॑न्त॒राह॑व॒नीयं॑ च हवि॒र्धानं॑ च॒ तिष्ठ॒न्नव॑ नयेत्~(४)\ip

%6.5.1.5
आ॒त्मान॑मे॒व य॑ज्ञयश॒सेना᳚र्पयति॒ यदि॑ का॒मये॑त॒ यज॑मानं यज्ञयश॒सेना᳚र्पयेय॒मित्य॑न्त॒रा स॑दोहविर्धा॒ने तिष्ठ॒न्नव॑ नये॒द्यज॑मानमे॒व य॑ज्ञयश॒सेना᳚र्पयति॒ यदि॑ का॒मये॑त सद॒स्यान्॑ यज्ञयश॒सेना᳚र्पयेय॒मिति॒ सद॑ आ॒लभ्याव॑ नयेथ्सद॒स्या॑ने॒व य॑ज्ञयश॒सेना᳚र्पयति॥~(५)\ip

{\anuvakamend[{इती᳚न्द्रि॒यमे॒व पुनः॑पुनर्नये॒त्त्रय॑स्त्रिꣳशच्च}]}

%6.5.2.1
आयु॒र्वा ए॒तद्य॒ज्ञस्य॒ यद्ध्रु॒व उ॑त्त॒मो ग्रहा॑णां गृह्यते॒ तस्मा॒दायुः॑ प्रा॒णाना॑मुत्त॒मं मू॒र्धानं॑ दि॒वो अ॑र॒तिं पृ॑थि॒व्या इत्या॑ह मू॒र्धान॑मे॒वैनꣳ॑ समा॒नानां᳚ करोति वैश्वान॒रमृ॒ताय॑ जा॒तम॒ग्निमित्या॑ह वैश्वान॒रꣳ हि दे॒वत॒यायु॑रुभ॒यतो॑वैश्वानरो गृह्यते॒ तस्मा॑दुभ॒यतः॑ प्रा॒णा अ॒धस्ता᳚च्चो॒परि॑ष्टाच्चा॒र्धिनो॒\-ऽन्ये ग्रहा॑ गृ॒ह्यन्ते॒\-ऽर्धी ध्रु॒वस्तस्मा᳚त्~(६)\ip

%6.5.2.2
अ॒र्ध्यवा᳚ङ्प्रा॒णो᳚\-ऽन्येषां᳚ प्रा॒णाना॒मुपो᳚प्ते॒\-ऽन्ये ग्रहाः᳚ सा॒द्यन्ते\-ऽनु॑पोप्ते ध्रु॒वस्तस्मा॑द॒स्थ्नान्याः प्र॒जाः प्र॑ति॒तिष्ठ॑न्ति मा॒ꣳ॒सेना॒न्या असु॑रा॒ वा उ॑त्तर॒तः पृ॑थि॒वीं प॒र्याचि॑कीर्\mbox{}ष॒न्तां दे॒वा ध्रु॒वेणा॑दृꣳह॒न्तद्ध्रु॒वस्य॑ ध्रुव॒त्वं यद्ध्रु॒व उ॑त्तर॒तः सा॒द्यते॒ धृत्या॒ आयु॒र्वा ए॒तद्य॒ज्ञस्य॒ यद्ध्रु॒व आ॒त्मा होता॒ यद्धो॑तृचम॒से ध्रु॒वम॑व॒नय॑त्या॒त्मन्ने॒व य॒ज्ञस्य॑~(७)\ip

%6.5.2.3
आयु॑र्दधाति पु॒रस्ता॑दु॒क्थस्या॑व॒नीय॒ इत्या॑हुः पु॒रस्ता॒द्ध्यायु॑षो भु॒ङ्क्ते म॑ध्य॒तो॑\-ऽव॒नीय॒ इत्या॑हुर्मध्य॒मेन॒ ह्यायु॑षो भु॒ङ्क्त उ॑त्तरा॒र्धे॑\-ऽव॒नीय॒ इत्या॑हुरुत्त॒मेन॒ ह्यायु॑षो भु॒ङ्क्ते वै᳚श्वदे॒व्यामृ॒चि श॒स्यमा॑नाया॒मव॑ नयति वैश्वदे॒व्यो॑ वै प्र॒जाः प्र॒जास्वे॒वायु॑र्दधाति॥~(८)\ip

{\anuvakamend[{ध्रु॒वस्तस्मा॑दे॒व य॒ज्ञस्यैका॒न्नच॑त्वारि॒ꣳ॒शच्च॑}]}

%6.5.3.1
य॒ज्ञेन॒ वै दे॒वाः सु॑व॒र्गं लो॒कमा॑य॒न्ते॑\-ऽमन्यन्त मनु॒ष्या॑ नो॒\-ऽन्वाभ॑विष्य॒न्तीति॒ ते सं॑वथ्स॒रेण॑ योपयि॒त्वा सु॑व॒र्गं लो॒कमा॑य॒न्तमृष॑य ऋतुग्र॒हैरे॒वानु॒ प्राजा॑न॒न्॒ यदृ॑तुग्र॒हा गृ॒ह्यन्ते॑ सुव॒र्गस्य॑ लो॒कस्य॒ प्रज्ञा᳚त्यै॒ द्वाद॑श गृह्यन्ते॒ द्वाद॑श॒ मासाः᳚ संवथ्स॒रः सं॑वथ्स॒रस्य॒ प्रज्ञा᳚त्यै स॒ह प्र॑थ॒मौ गृ॑ह्येते स॒होत्त॒मौ तस्मा॒द्द्वौद्वा॑वृ॒तू उ॑भ॒यतो॑मुखमृतुपा॒त्रं भ॑वति॒ कः~(९)\ip

%6.5.3.2
हि तद्वेद॒ यत॑ ऋतू॒नां मुख॑मृ॒तुना॒ प्रेष्येति॒ षट्कृत्व॑ आह॒ षड्वा ऋ॒तव॑ ऋ॒तूने॒व प्री॑णात्यृ॒तुभि॒रिति॑ च॒तुश्चतु॑ष्पद ए॒व प॒शून्प्री॑णाति॒ द्विः पुन॑र्\mbox{}ऋ॒तुना॑ह द्वि॒पद॑ ए॒व प्री॑णात्यृ॒तुना॒ प्रेष्येति॒ षट्कृत्व॑ आह॒र्तुभि॒रिति॑ च॒तुस्तस्मा॒च्चतु॑ष्पादः प॒शव॑ ऋ॒तूनुप॑ जीवन्ति॒ द्विः~(१०)\ip

%6.5.3.3
पुन॑र्\mbox{}ऋ॒तुना॑ह॒ तस्मा᳚द्द्वि॒पाद॒श्चतु॑ष्पदः प॒शूनुप॑ जीवन्त्यृ॒तुना॒ प्रेष्येति॒ षट्कृत्व॑ आह॒र्तुभि॒रिति॑ च॒तुर्द्विः पुन॑र्\mbox{}ऋ॒तुना॑हा॒क्रम॑णमे॒व तथ्सेतुं॒ यज॑मानः कुरुते सुव॒र्गस्य॑ लो॒कस्य॒ सम॑ष्ट्यै॒ नान्यो᳚न्यमनु॒ प्र प॑द्येत॒ यद॒न्यो᳚\-ऽन्यम॑नु प्र॒पद्ये॑त॒र्तुर्\mbox{}ऋ॒तुमनु॒ प्र प॑द्येत॒र्तवो॒ मोहु॑काः स्युः~(११)\ip

%6.5.3.4
प्रसि॑द्धमे॒वाध्व॒र्युर्दक्षि॑णेन॒ प्र प॑द्यते॒ प्रसि॑द्धं प्रतिप्रस्था॒तोत्त॑रेण॒ तस्मा॑दादि॒त्यः षण्मा॒सो दक्षि॑णेनैति॒ षडुत्त॑रेणोपया॒मगृ॑हीतो\-ऽसि स॒ꣳ॒सर्पो᳚\-ऽस्यꣳहस्प॒त्याय॒ त्वेत्या॒हास्ति॑ त्रयोद॒शो मास॒ इत्या॑हु॒स्तमे॒व तत्प्री॑णाति॥~(१२)\ip

{\anuvakamend[{को जी॑वन्ति॒ द्विः स्यु॒श्चतु॑स्त्रिꣳशच्च}]}

%6.5.4.1
सु॒व॒र्गाय॒ वा ए॒ते लो॒काय॑ गृह्यन्ते॒ यदृ॑तुग्र॒हा ज्योति॑रिन्द्रा॒ग्नी यदै᳚न्द्रा॒ग्नमृ॑तुपा॒त्रेण॑ गृ॒ह्णाति॒ ज्योति॑रे॒वास्मा॑ उ॒परि॑ष्टाद्दधाति सुव॒र्गस्य॑ लो॒कस्यानु॑ख्यात्या ओजो॒भृतौ॒ वा ए॒तौ दे॒वानां॒ यदि॑न्द्रा॒ग्नी यदै᳚न्द्रा॒ग्नो गृ॒ह्यत॒ ओज॑ ए॒वाव॑ रुन्धे वैश्वदे॒वꣳ शु॑क्रपा॒त्रेण॑ गृह्णाति वैश्वदे॒व्यो॑ वै प्र॒जा अ॒सावा॑दि॒त्यः शु॒क्रो यद्वै᳚श्वदे॒वꣳ शु॑क्रपा॒त्रेण॑ गृ॒ह्णाति॒ तस्मा॑द॒सावा॑दि॒त्यः~(१३)\ip

%6.5.4.2
सर्वाः᳚ प्र॒जाः प्र॒त्यङ्ङुदे॑ति॒ तस्मा॒थ्सर्व॑ ए॒व म॑न्यते॒ मां प्रत्युद॑गा॒दिति॑ वैश्वदे॒वꣳ शु॑क्रपा॒त्रेण॑ गृह्णाति वैश्वदे॒व्यो॑ वै प्र॒जास्तेजः॑ शु॒क्रो यद्वै᳚श्वदे॒वꣳ शु॑क्रपा॒त्रेण॑ गृ॒ह्णाति॑ प्र॒जास्वे॒व तेजो॑ दधाति॥~(१४)\ip

{\anuvakamend[{तस्मा॑द॒सावा॑दि॒त्यस्त्रि॒ꣳ॒शच्च॑}]}

%6.5.5.1
इन्द्रो॑ म॒रुद्भिः॒ सांवि॑द्येन॒ माध्य॑न्दिने॒ सव॑ने वृ॒त्रम॑ह॒न्॒ यन्माध्य॑न्दिने॒ सव॑ने मरुत्व॒तीया॑ गृ॒ह्यन्ते॒ वार्त्र॑घ्ना ए॒व ते यज॑मानस्य गृह्यन्ते॒ तस्य॑ वृ॒त्रं ज॒घ्नुष॑ ऋ॒तवो॑\-ऽमुह्य॒न्थ्स ऋ॑तुपा॒त्रेण॑ मरुत्व॒तीया॑नगृह्णा॒त्ततो॒ वै स ऋ॒तून्प्राजा॑ना॒द्यदृ॑तुपा॒त्रेण॑ मरुत्व॒तीया॑ गृ॒ह्यन्त॑ ऋतू॒नां प्रज्ञा᳚त्यै॒ वज्रं॒ वा ए॒तं यज॑मानो॒ भ्रातृ॑व्याय॒ प्र ह॑रति॒ यन्म॑रुत्व॒तीया॒ उदे॒व प्र॑थ॒मेन॑~(१५)\ip

%6.5.5.2
य॒च्छ॒ति॒ प्र ह॑रति द्वि॒तीये॑न स्तृणु॒ते तृ॒तीये॒नायु॑धं॒ वा ए॒तद्यज॑मानः॒ सꣴस्कु॑रुते॒ यन्म॑रुत्व॒तीया॒ धनु॑रे॒व प्र॑थ॒मो ज्या द्वि॒तीय॒ इषु॑स्तृ॒तीयः॒ प्रत्ये॒व प्र॑थ॒मेन॑ धत्ते॒ वि सृ॑जति द्वि॒तीये॑न॒ विध्य॑ति तृ॒तीये॒नेन्द्रो॑ वृ॒त्रꣳ ह॒त्वा परां᳚ परा॒वत॑मगच्छ॒दपा॑राध॒मिति॒ मन्य॑मानः॒ स हरि॑तो\-ऽभव॒थ्स ए॒तान्म॑रुत्व॒तीया॑नात्म॒स्पर॑णानपश्य॒त्तान॑गृह्णीत~(१६)\ip

%6.5.5.3
प्रा॒णमे॒व प्र॑थ॒मेना᳚स्पृणुतापा॒नं द्वि॒तीये॑ना॒\-ऽऽ\-त्मानं॑ तृ॒तीये॑नात्म॒स्पर॑णा॒ वा ए॒ते यज॑मानस्य गृह्यन्ते॒ यन्म॑रुत्व॒तीयाः᳚ प्रा॒णमे॒व प्र॑थ॒मेन॑ स्पृणुते\-ऽपा॒नं द्वि॒तीये॑ना॒\-ऽऽ\-त्मानं॑ तृ॒तीये॒नेन्द्रो॑ वृ॒त्रम॑ह॒न्तं दे॒वा अ॑ब्रुवन्म॒हान् वा अ॒यम॑भू॒द्यो वृ॒त्रमव॑धी॒दिति॒ तन्म॑हे॒न्द्रस्य॑ महेन्द्र॒त्वꣳ स ए॒तं मा॑हे॒न्द्रमु॑द्धा॒रमुद॑हरत वृ॒त्रꣳ ह॒त्वान्यासु॑ दे॒वता॒स्वधि॒ यन्मा॑हे॒न्द्रो गृ॒ह्यत॑ उद्धा॒रमे॒व तं यज॑मान॒ उद्ध॑रते॒\-ऽन्यासु॑ प्र॒जास्वधि॑ शुक्रपा॒त्रेण॑ गृह्णाति यजमानदेव॒त्यो॑ वै मा॑हे॒न्द्रस्तेजः॑ शु॒क्रो यन्मा॑हे॒न्द्रꣳ शु॑क्रपा॒त्रेण॑ गृ॒ह्णाति॒ यज॑मान ए॒व तेजो॑ दधाति॥~(१७)\ip

{\anuvakamend[{प्र॒थ॒मेना॑गृह्णीत दे॒वता᳚स्व॒ष्टाविꣳ॑शतिश्च}]}

%6.5.6.1
अदि॑तिः पु॒त्रका॑मा सा॒ध्येभ्यो॑ दे॒वेभ्यो᳚ ब्रह्मौद॒नम॑पच॒त्तस्या॑ उ॒च्छेष॑णमददु॒स्तत्प्राश्ञा॒थ्सा रेतो॑\-ऽधत्त॒ तस्यै॑ च॒त्वार॑ आदि॒त्या अ॑जायन्त॒ सा द्वि॒तीय॑मपच॒थ्साम॑न्यतो॒च्छेष॑णान्म इ॒मे᳚\-ऽज्ञत॒ यदग्रे᳚ प्राशि॒ष्यामी॒तो मे॒ वसी॑याꣳसो जनिष्यन्त॒ इति॒ साग्रे॒ प्राश्ञा॒थ्सा रेतो॑\-ऽधत्त॒ तस्यै॒ व्यृ॑द्धमा॒ण्डम॑जायत॒ सादि॒त्येभ्य॑ ए॒व~(१८)\ip

%6.5.6.2
तृ॒तीय॑मपच॒द्भोगा॑य म इ॒दꣴ श्रा॒न्तम॒स्त्विति॒ ते᳚\-ऽब्रुव॒न्वरं॑ वृणामहै॒ यो\-ऽतो॒ जाया॑ता अ॒स्माक॒ꣳ॒ स एको॑\-ऽस॒द्यो᳚\-ऽस्य प्र॒जाया॒मृध्या॑ता अ॒स्माक॒म्भोगा॑य भवा॒दिति॒ ततो॒ विव॑स्वानादि॒त्यो॑\-ऽजायत॒ तस्य॒ वा इ॒यं प्र॒जा यन्म॑नु॒ष्या᳚स्तास्वेक॑ ए॒वर्द्धो यो यज॑ते॒ स दे॒वाना॒म्भोगा॑य भवति दे॒वा वै य॒ज्ञात्~(१९)\ip

%6.5.6.3
रु॒द्रम॒न्तरा॑य॒न्थ्स आ॑दि॒त्यान॒न्वाक्र॑मत॒ ते द्वि॑देव॒त्या᳚न्प्राप॑द्यन्त॒ तान्न प्रति॒ प्राय॑च्छ॒न्तस्मा॒दपि॒ वध्यं॒ प्रप॑न्नं॒ न प्रति॒ प्र य॑च्छन्ति॒ तस्मा᳚द्द्विदेव॒त्ये᳚भ्य आदि॒त्यो निर्गृ॑ह्यते॒ यदु॒च्छेष॑णा॒दजा॑यन्त॒ तस्मा॑दु॒च्छेष॑णाद्गृह्यते ति॒सृभि॑र्\mbox{}ऋ॒ग्भिर्गृ॑ह्णाति मा॒ता पि॒ता पु॒त्रस्तदे॒व तन्मि॑थु॒नमुल्बं॒ गर्भो॑ ज॒रायु॒ तदे॒व तत्~(२०)\ip

%6.5.6.4
मि॒थु॒नं प॒शवो॒ वा ए॒ते यदा॑दि॒त्य ऊर्ग्दधि॑ द॒ध्ना म॑ध्य॒तः श्री॑णा॒त्यूर्ज॑मे॒व प॑शू॒नां म॑ध्य॒तो द॑धाति शृतात॒ङ्क्ये॑न मेध्य॒त्वाय॒ तस्मा॑दा॒मा प॒क्वं दु॑हे प॒शवो॒ वा ए॒ते यदा॑दि॒त्यः प॑रि॒श्रित्य॑ गृह्णाति प्रति॒रुध्यै॒वास्मै॑ प॒शून्गृ॑ह्णाति प॒शवो॒ वा ए॒ते यदा॑दि॒त्य ए॒ष रु॒द्रो यद॒ग्निः प॑रि॒श्रित्य॑ गृह्णाति रु॒द्रादे॒व प॒शून॒न्तर्द॑धाति~(२१)\ip

%6.5.6.5
ए॒ष वै विव॑स्वानादि॒त्यो यदु॑पाꣳशु॒सव॑नः॒ स ए॒तमे॒व सो॑मपी॒थं परि॑ शय॒ आ तृ॑तीयसव॒नाद्विव॑स्व आदित्यै॒ष ते॑ सोमपी॒थ इत्या॑ह॒ विव॑स्वन्तमे॒वा\-ऽऽ\-दि॒त्यꣳ सो॑मपी॒थेन॒ सम॑र्धयति॒ या दि॒व्या वृष्टि॒स्तया᳚ त्वा श्रीणा॒मीति॒ वृष्टि॑कामस्य श्रीणीया॒द्वृष्टि॑मे॒वाव॑ रुन्धे॒ यदि॑ ता॒जक्प्र॒स्कन्दे॒द्वर्\mbox{}षु॑कः प॒र्जन्यः॑ स्या॒द्यदि॑ चि॒रमव॑र्\mbox{}षुको॒ न सा॑दय॒त्यस॑न्ना॒द्धि प्र॒जाः प्र॒जाय॑न्ते॒ नानु॒ वष॑ट्करोति॒ यद॑नुवषट्कु॒र्याद्रु॒द्रं प्र॒जा अ॒न्वव॑सृजे॒न्न हु॒त्वान्वी᳚क्षेत॒ यद॒न्वीक्षे॑त॒ चक्षु॑रस्य प्र॒मायु॑कꣴ स्या॒त्तस्मा॒न्नान्वीक्ष्यः॑॥~(२२)\ip

{\anuvakamend[{ए॒व य॒ज्ञाज्ज॒रायु॒ तदे॒व तद॒न्तर्द॑धाति॒ न स॒प्तविꣳ॑शतिश्च}]}

%6.5.7.1
अ॒न्त॒र्या॒म॒पा॒त्रेण॑ सावि॒त्रमा᳚ग्रय॒णाद्गृ॑ह्णाति प्र॒जा\-प॑ति॒र्वा ए॒ष यदा᳚ग्रय॒णः प्र॒जानां᳚ प्र॒जन॑नाय॒ न सा॑दय॒त्यस॑न्ना॒द्धि प्र॒जाः प्र॒जाय॑न्ते॒ नानु॒ वष॑ट्करोति॒ यद॑नुवषट्कु॒र्याद्रु॒द्रं प्र॒जा अ॒न्वव॑सृजेदे॒ष वै गा॑य॒त्रो दे॒वानां॒ यथ्स॑वि॒तैष गा॑यत्रि॒यै लो॒के गृ॑ह्यते॒ यदा᳚ग्रय॒णो यद॑न्तर्यामपा॒त्रेण॑ सावि॒त्रमा᳚ग्रय॒णाद्गृ॒ह्णाति॒ स्वादे॒वैनं॒ योने॒र्निर्गृ॑ह्णाति॒ विश्वे᳚~(२३)\ip

%6.5.7.2
दे॒वास्तृ॒तीय॒ꣳ॒ सव॑नं॒ नोद॑यच्छ॒न्ते स॑वि॒तारं॑ प्रातःसव॒नभा॑ग॒ꣳ॒ सन्तं॑ तृतीयसव॒नम॒भि पर्य॑णय॒न्ततो॒ वै ते तृ॒तीय॒ꣳ॒ सव॑न॒मुद॑यच्छ॒न्॒ यत्तृ॑तीयसव॒ने सा॑वि॒त्रो गृ॒ह्यते॑ तृ॒तीय॑स्य॒ सव॑न॒स्योद्य॑त्यै सवितृपा॒त्रेण॑ वैश्वदे॒वं क॒लशा᳚द्गृह्णाति वैश्वदे॒व्यो॑ वै प्र॒जा वै᳚श्वदे॒वः क॒लशः॑ सवि॒ता प्र॑स॒वाना॑मीशे॒ यथ्स॑वितृपा॒त्रेण॑ वैश्वदे॒वं क॒लशा᳚द्गृ॒ह्णाति॑ सवि॒तृप्र॑सूत ए॒वास्मै᳚ प्र॒जाः प्र~(२४)\ip

%6.5.7.3
ज॒न॒य॒ति॒ सोमे॒ सोम॑म॒भि गृ॑ह्णाति॒ रेत॑ ए॒व तद्द॑धाति सु॒शर्मा॑सि सुप्रतिष्ठा॒न इत्या॑ह॒ सोमे॒ हि सोम॑मभिगृ॒ह्णाति॒ प्रति॑ष्ठित्या ए॒तस्मि॒न्वा अपि॒ ग्रहे॑ मनु॒ष्ये᳚भ्यो दे॒वेभ्यः॑ पि॒तृभ्यः॑ क्रियते सु॒शर्मा॑सि सुप्रतिष्ठा॒न इत्या॑ह मनु॒ष्ये᳚भ्य ए॒वैतेन॑ करोति बृ॒हदित्या॑ह दे॒वेभ्य॑ ए॒वैतेन॑ करोति॒ नम॒ इत्या॑ह पि॒तृभ्य॑ ए॒वैतेन॑ करोत्ये॒ताव॑ती॒र्वै दे॒वता॒स्ताभ्य॑ ए॒वैन॒ꣳ॒ सर्वा᳚भ्यो गृह्णात्ये॒ष ते॒ योनि॒र्विश्वे᳚भ्यस्त्वा दे॒वेभ्य॒ इत्या॑ह वैश्वदे॒वो ह्ये॑षः॥~(२५)\ip

{\anuvakamend[{विश्वे॒ प्र पि॒तृभ्य॑ ए॒वैतेन॑ करो॒त्येका॒न्नविꣳ॑श॒तिश्च॑}]}

%6.5.8.1
प्रा॒णो वा ए॒ष यदु॑पा॒ꣳ॒शुर्यदु॑पाꣳशुपा॒त्रेण॑ प्रथ॒मश्चो᳚त्त॒मश्च॒ ग्रहौ॑ गृ॒ह्येते᳚ प्रा॒णमे॒वानु॑ प्र॒यन्ति॑ प्रा॒णमनूद्य॑न्ति प्र॒जा\-प॑ति॒र्वा ए॒ष यदा᳚ग्रय॒णः प्रा॒ण उ॑पा॒ꣳ॒शुः पत्नीः᳚ प्र॒जाः प्र ज॑नयन्ति॒ यदु॑पाꣳशुपा॒त्रेण॑ पात्नीव॒तमा᳚ग्रय॒णाद्गृ॒ह्णाति॑ प्र॒जानां᳚ प्र॒जन॑नाय॒ तस्मा᳚त्प्रा॒णं प्र॒जा अनु॒ प्र जा॑यन्ते दे॒वा वा इ॒तइ॑तः॒ पत्नीः᳚ सुव॒र्गम्~(२६)\ip

%6.5.8.2
लो॒कम॑जिगाꣳस॒न्ते सु॑व॒र्गं लो॒कं न प्राजा॑न॒न्त ए॒तं पा᳚त्नीव॒तम॑पश्य॒न्तम॑गृह्णत॒ ततो॒ वै ते सु॑व॒र्गं लो॒कं प्राजा॑न॒न्॒ यत्पा᳚त्नीव॒तो गृ॒ह्यते॑ सुव॒र्गस्य॑ लो॒कस्य॒ प्रज्ञा᳚त्यै॒ स सोमो॒ नाति॑ष्ठत स्त्री॒भ्यो गृ॒ह्यमा॑ण॒स्तं घृ॒तं वज्रं॑ कृ॒त्वाघ्न॒न्तं निरि॑न्द्रियं भू॒तम॑गृह्ण॒न्तस्मा॒थ्स्त्रियो॒ निरि॑न्द्रिया॒ अदा॑यादी॒रपि॑ पा॒पात्पु॒ꣳ॒स उप॑स्तितरम्~(२७)\ip

%6.5.8.3
व॒द॒न्ति॒ यद् घृ॒तेन॑ पात्नीव॒तꣴ श्री॒णाति॒ वज्रे॑णै॒वैनं॒ वशे॑ कृ॒त्वा गृ॑ह्णात्युपया॒मगृ॑हीतो॒\-ऽसीत्या॑हे॒यं वा उ॑पया॒मस्तस्मा॑दि॒मां प्र॒जा अनु॒ प्र जा॑यन्ते॒ बृह॒स्पति॑सुतस्य त॒ इत्या॑ह॒ ब्रह्म॒ वै दे॒वानां॒ बृह॒स्पति॒र्ब्रह्म॑णै॒वास्मै᳚ प्र॒जाः प्र ज॑नयतीन्दो॒ इत्या॑ह॒ रेतो॒ वा इन्दू॒ रेत॑ ए॒व तद्द॑धातीन्द्रियाव॒ इति॑~(२८)\ip

%6.5.8.4
आ॒ह॒ प्र॒जा वा इ॑न्द्रि॒यं प्र॒जा ए॒वास्मै॒ प्र ज॑नय॒त्यग्ना(३) इत्या॑हा॒ग्निर्वै रे॑तो॒धाः पत्नी॑व॒ इत्या॑ह मिथुन॒त्वाय॑ स॒जूर्दे॒वेन॒ त्वष्ट्रा॒ सोमं॑ पि॒बेत्या॑ह॒ त्वष्टा॒ वै प॑शू॒नां मि॑थु॒नानाꣳ॑ रूप॒कृद्रू॒पमे॒व प॒शुषु॑ दधाति दे॒वा वै त्वष्टा॑रमजिघाꣳस॒न्थ्स पत्नीः॒ प्राप॑द्यत॒ तं न प्रति॒ प्राय॑च्छ॒न्तस्मा॒दपि॑~(२९)\ip

%6.5.8.5
वध्यं॒ प्रप॑न्नं॒ न प्रति॒ प्र य॑च्छन्ति॒ तस्मा᳚त्पात्नीव॒ते त्वष्ट्रे\-ऽपि॑ गृह्यते॒ न सा॑दय॒त्यस॑न्ना॒द्धि प्र॒जाः प्र॒जाय॑न्ते॒ नानु॒ वष॑ट्करोति॒ यद॑नुवषट्कु॒र्याद्रु॒द्रं प्र॒जा अ॒न्वव॑सृजे॒द्यन्नानु॑वषट्कु॒र्यादशा᳚न्तम॒ग्नीथ्सोमं॑ भक्षयेदुपा॒ꣳ॒श्वनु॒ वष॑ट्करोति॒ न रु॒द्रं प्र॒जा अ॑न्ववसृ॒जति॑ शा॒न्तम॒ग्नीथ्सोम॑म्भक्षय॒त्यग्नी॒न्नेष्टु॑रु॒पस्थ॒मा सी॑द~(३०)\ip

%6.5.8.6
नेष्टः॒ पत्नी॑मु॒दान॒येत्या॑हा॒ग्नीदे॒व नेष्ट॑रि॒ रेतो॒ दधा॑ति॒ नेष्टा॒ पत्नि॑यामुद्गा॒त्रा सं ख्या॑पयति प्र॒जा\-प॑ति॒र्वा ए॒ष यदु॑द्गा॒ता प्र॒जानां᳚ प्र॒जन॑नाया॒प उप॒ प्र व॑र्तयति॒ रेत॑ ए॒व तथ्सि॑ञ्चत्यू॒रुणोप॒ प्र व॑र्तयत्यू॒रुणा॒ हि रेतः॑ सि॒च्यते॑ नग्नं॒ कृत्यो॒रुमुप॒ प्र व॑र्तयति य॒दा हि न॒ग्न ऊ॒रुर्भव॒त्यथ॑ मिथु॒नी भ॑व॒तो\-ऽथ॒ रेतः॑ सिच्य॒ते\-ऽथ॑ प्र॒जाः प्र जा॑यन्ते॥~(३१)\ip

{\anuvakamend[{पत्नीः᳚ सुव॒र्गमुप॑स्तितरमिन्द्रियाव॒ इत्यपि॑ सीद मिथु॒न्य॑ष्टौ च॑}]}

%6.5.9.1
इन्द्रो॑ वृ॒त्रम॑ह॒न्तस्य॑ शीर्\mbox{}षकपा॒लमुदौ᳚ब्ज॒थ्स द्रो॑णकल॒शो॑\-ऽभव॒त्तस्मा॒थ्सोमः॒ सम॑स्रव॒थ्स हा॑रियोज॒नो॑\-ऽभव॒त्तं व्य॑चिकिथ्सज्जु॒हवा॒नी(३) मा हौ॒षा(३) मिति॒ सो॑\-ऽमन्यत॒ यद्धो॒ष्याम्या॒मꣳ हो᳚ष्यामि॒ यन्न हो॒ष्यामि॑ यज्ञवेश॒सं क॑रिष्या॒मीति॒ तम॑ध्रियत॒ होतु॒ꣳ॒ सो᳚\-ऽग्निर॑ब्रवी॒न्न मय्या॒मꣳ हो᳚ष्य॒सीति॒ तं धा॒नाभि॑रश्रीणात्~(३२)\ip

%6.5.9.2
तꣳ शृ॒तं भू॒तम॑जुहो॒द्यद्धा॒नाभि॑र्\mbox{}हारियोज॒नꣴ श्री॒णाति॑ शृत॒त्वाय॑ शृ॒तमे॒वैनं॑ भू॒तं जु॑होति ब॒ह्वीभिः॑ श्रीणात्ये॒ताव॑ती\-रे॒वास्या॒मुष्मिँ॑ल्लो॒के का॑म॒दुघा॑ भव॒न्त्यथो॒ खल्वा॑हुरे॒ता वा इन्द्र॑स्य॒ पृश्ञ॑यः काम॒दुघा॒ यद्धा॑रियोज॒नीरिति॒ तस्मा᳚द्ब॒ह्वीभिः॑ श्रीणीयादृख्सा॒मे वा इन्द्र॑स्य॒ हरी॑ सोम॒पानौ॒ तयोः᳚ परि॒धय॑ आ॒धानं॒ यदप्र॑हृत्य परि॒धीञ्जु॑हु॒याद॒न्तरा॑धानाभ्याम्~(३३)\ip

%6.5.9.3
घा॒सं प्र य॑च्छेत्प्र॒हृत्य॑ परि॒धीञ्जु॑होति॒ निरा॑धानाभ्यामे॒व घा॒सं प्र य॑च्छत्युन्ने॒ता जु॑होति या॒तया॑मेव॒ ह्ये॑तर्\mbox{}ह्य॑ध्व॒र्युः स्व॒गाकृ॑तो॒ यद॑ध्व॒र्युर्जु॑हु॒याद्यथा॒ विमु॑क्तं॒ पुन॑र्यु॒नक्ति॑ ता॒दृगे॒व तच्छी॒र्॒\mbox{}षन्न॑धिनि॒धाय॑ जुहोति शीर्\mbox{}ष॒तो हि स स॒मभ॑वद्वि॒क्रम्य॑ जुहोति वि॒क्रम्य॒ हीन्द्रो॑ वृ॒त्रमह॒न्थ्समृ॑द्ध्यै प॒शवो॒ वै हा॑रियोज॒नीर्यथ्स॑म्भि॒न्द्यादल्पाः᳚~(३४)\ip

%6.5.9.4
ए॒नं॒ प॒शवो॑ भु॒ञ्जन्त॒ उप॑ तिष्ठेर॒न्॒ यन्न स॑म्भि॒न्द्याद्ब॒हव॑ एनं प॒शवो\-ऽभु॑ञ्जन्त॒ उप॑ तिष्ठेर॒न्मन॑सा॒ सम्बा॑धत उ॒भयं॑ करोति ब॒हव॑ ए॒वैनं॑ प॒शवो॑ भु॒ञ्जन्त॒ उप॑ तिष्ठन्त उन्ने॒तर्यु॑पह॒वमि॑च्छन्ते॒ य ए॒व तत्र॑ सोमपी॒थस्तमे॒वाव॑ रुन्धत उत्तरवे॒द्यां नि व॑पति प॒शवो॒ वा उ॑त्तरवे॒दिः प॒शवो॑ हारियोज॒नीः प॒शुष्वे॒व प॒शून्प्रति॑\-ष्ठापयन्ति॥~(३५)\ip

{\anuvakamend[{अ॒श्री॒णा॒द॒न्तरा॑धानाभ्या॒मल्पाः᳚ स्थापयन्ति}]}

%6.5.10.1
ग्रहा॒न् वा अनु॑ प्र॒जाः प॒शवः॒ प्र जा॑यन्त उपाꣴश्वन्तर्या॒माव॑जा॒वयः॑ शु॒क्राम॒न्थिनौ॒ पुरु॑षा ऋतुग्र॒हानेक॑शफा आदित्यग्र॒हं गाव॑ आदित्यग्र॒हो भूयि॑ष्ठाभिर्\mbox{}ऋ॒ग्भिर्गृ॑ह्यते॒ तस्मा॒द्गावः॑ पशू॒नां भूयि॑ष्ठा॒ यत् त्रिरु॑पा॒ꣳ॒शुꣳ हस्ते॑न विगृ॒ह्णाति॒ तस्मा॒द्द्वौ त्रीन॒जा ज॒नय॒त्यथाव॑यो॒ भूय॑सीः पि॒ता वा ए॒ष यदा᳚ग्रय॒णः पु॒त्रः क॒लशो॒ यदा᳚ग्रय॒ण उ॑प॒दस्ये᳚त्क॒लशा᳚द्गृह्णीया॒द्यथा॑ पि॒ता~(३६)\ip

%6.5.10.2
पु॒त्रं क्षि॒त उ॑प॒धाव॑ति ता॒दृगे॒व तद्यत्क॒लश॑ उप॒दस्ये॑दाग्रय॒णाद्गृ॑ह्णीया॒द्यथा॑ पु॒त्रः पि॒तरं॑ क्षि॒त उ॑प॒धाव॑ति ता॒दृगे॒व तदा॒त्मा वा ए॒ष य॒ज्ञस्य॒ यदा᳚ग्रय॒णो यद्ग्रहो॑ वा क॒लशो॑ वोप॒दस्ये॑दाग्रय॒णाद्गृ॑ह्णीयादा॒त्मन॑ ए॒वाधि॑ य॒ज्ञं निष्क॑रो॒त्यवि॑ज्ञातो॒ वा ए॒ष गृ॑ह्यते॒ यदा᳚ग्रय॒णः स्था॒ल्या गृ॒ह्णाति॑ वाय॒व्ये॑न जुहोति॒ तस्मा᳚त्~(३७)\ip

%6.5.10.3
गर्भे॒णावि॑ज्ञातेन ब्रह्म॒हाव॑भृ॒थमव॑ यन्ति॒ परा᳚ स्था॒लीरस्य॒न्त्युद्वा॑य॒व्या॑नि हरन्ति॒ तस्मा॒थ्स्त्रियं॑ जा॒तां परा᳚स्य॒न्त्यु\-त्पुमाꣳ॑सꣳ हरन्ति॒ यत्पु॑रो॒रुच॒माह॒ यथा॒ वस्य॑स आ॒हर॑ति ता॒दृगे॒व तद्यद्ग्रहं॑ गृ॒ह्णाति॒ यथा॒ वस्य॑स आ॒हृत्य॒ प्राह॑ ता॒दृगे॒व तद्यथ्सा॒दय॑ति॒ यथा॒ वस्य॑स उपनि॒धाया॑प॒क्राम॑ति ता॒दृगे॒व तद्यद्वै य॒ज्ञस्य॒ साम्ना॒ यजु॑षा क्रि॒यते॑ शिथि॒लं तद्यदृ॒चा तद्दृ॒ढं पु॒रस्ता॑दुपयामा॒ यजु॑षा गृह्यन्त उ॒परि॑ष्टादुपयामा ऋ॒चा य॒ज्ञस्य॒ धृत्यै᳚॥~(३८)\ip

{\anuvakamend[{यथा॑ पि॒ता तस्मा॑दप॒क्राम॑ति ता॒दृगे॒व तद्यद॒ष्टाद॑श च}]}%॥10॥

%6.5.11.1
प्रान्यानि॒ पात्रा॑णि यु॒ज्यन्ते॒ नान्यानि॒ यानि॑ परा॒चीना॑नि प्रयु॒ज्यन्ते॒\-ऽमुमे॒व तैर्लो॒कम॒भि ज॑यति॒ परा॑ङिव॒ ह्य॑सौ लो॒को यानि॒ पुनः॑ प्रयु॒ज्यन्त॑ इ॒ममे॒व तैर्लो॒कम॒भि ज॑यति॒ पुनः॑पुनरिव॒ ह्य॑यं लो॒कः प्रान्यानि॒ पात्रा॑णि यु॒ज्यन्ते॒ नान्यानि॒ यानि॑ परा॒चीना॑नि प्रयु॒ज्यन्ते॒ तान्यन्वोष॑धयः॒ परा॑ भवन्ति॒ यानि॒ पुनः॑~(३९)\ip

%6.5.11.2
प्र॒यु॒ज्यन्ते॒ तान्यन्वोष॑धयः॒ पुन॒रा भ॑वन्ति॒ प्रान्यानि॒ पात्रा॑णि यु॒ज्यन्ते॒ नान्यानि॒ यानि॑ परा॒चीना॑नि प्रयु॒ज्यन्ते॒ तान्यन्वा॑र॒ण्याः प॒शवो\-ऽर॑ण्य॒मप॑ यन्ति॒ यानि॒ पुनः॑ प्रयु॒ज्यन्ते॒ तान्यनु॑ ग्रा॒म्याः प॒शवो॒ ग्राम॑मु॒पाव॑यन्ति॒ यो वै ग्रहा॑णां नि॒दानं॒ वेद॑ नि॒दान॑वान्भव॒त्याज्य॒मित्यु॒क्थं तद्वै ग्रहा॑णां नि॒दानं॒ यदु॑पा॒ꣳ॒शु शꣳस॑ति॒ तत्~(४०)\ip

%6.5.11.3
उ॒पा॒ꣳ॒श्व॒न्त॒र्या॒मयो॒र्यदु॒च्चैस्तदित॑रेषां॒ ग्रहा॑णामे॒तद्वै ग्रहा॑णां नि॒दानं॒ य ए॒वं वेद॑ नि॒दान॑वान्भवति॒ यो वै ग्रहा॑णां मिथु॒नं वेद॒ प्र प्र॒जया॑ प॒शुभि॑र्मिथु॒नैर्जा॑यते स्था॒लीभि॑र॒न्ये ग्रहा॑ गृ॒ह्यन्ते॑ वाय॒व्यै॑र॒न्य ए॒तद्वै ग्रहा॑णां मिथु॒नं य ए॒वं वेद॒ प्र प्र॒जया॑ प॒शुभि॑र्मिथु॒नैर्जा॑यत॒ इन्द्र॒स्त्वष्टुः॒ सोम॑मभी॒षहा॑पिब॒थ्स विष्वङ्ङ्॑~(४१)\ip

%6.5.11.4
व्या᳚र्च्छ॒थ्स आ॒त्मन्ना॒रम॑णं॒ नावि॑न्द॒थ्स ए॒तान॑नुसव॒नं पु॑रो॒डाशा॑नपश्य॒त्तां निर॑वप॒त्तैर्वै स आ॒त्मन्ना॒रम॑णमकुरुत॒ तस्मा॑दनुसव॒नं पु॑रो॒डाशा॒ निरु॑प्यन्ते॒ तस्मा॑दनुसव॒नं पु॑रो॒डाशा॑नां॒ प्राश्ञी॑यादा॒त्मन्ने॒वारम॑णं कुरुते॒ नैन॒ꣳ॒ सोमो\-ऽति॑ पवते ब्रह्मवा॒दिनो॑ वदन्ति॒ नर्चा न यजु॑षा प॒ङ्क्तिरा᳚प्य॒ते\-ऽथ॒ किं य॒ज्ञस्य॑ पाङ्क्त॒त्वमिति॑ धा॒नाः क॑र॒म्भः प॑रिवा॒पः पु॑रो॒डाशः॑ पय॒स्या॑ तेन॑ प॒ङ्क्तिरा᳚प्यते॒ तद्य॒ज्ञस्य॑ पाङ्क्त॒त्वम्॥~(४२)\ip

{\anuvakamend[{भ॒व॒न्ति॒ यानि॒ पुनः॒ शꣳस॑ति॒ तद्विष्व॒ङ्किञ्चतु॑र्दश च}]}%॥11॥

\prashnaend{इन्द्रो॑ वृ॒त्राया\-ऽऽ\-यु॒र्वै य॒ज्ञेन॑ सुव॒र्गायेन्द्रो॑ म॒रुद्भि॒रदि॑तिरन्तर्यामपा॒त्रेण॑ प्रा॒ण उ॑पाꣳशुपा॒त्रेणेन्द्रो॑ वृ॒त्रम॑ह॒न्तस्य॒ ग्रहा॒न्॒ वै प्रान्यान्येका॑\-दश॥११॥}{इन्द्रो॑ वृ॒त्राय॒ पुन॑र्\mbox{}ऋ॒तुना॑ह मिथु॒नं प॒शवो॒ नेष्टः॒ पत्नी॑मुपाꣴश्वन्तर्या॒मयो॒र्द्विच॑त्वारिꣳशत्॥४२॥}{इन्द्रो॑ वृ॒त्राय॑ पाङ्क्त॒त्वम्॥}%%६-५
{हरिः॑ ॐ}{॥कृष्ण-यजुर्वेदीय-तैत्तिरीय-संहितायां षष्ठकाण्डे पञ्चमः प्रश्नः समाप्तः॥६-५॥}
%%% END PRASHNA
