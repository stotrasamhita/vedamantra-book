\chapt{काण्डम् २}
\sect{पञ्चमः प्रश्नः}\setcounter{anuvakam}{0}
\dnsub{तैत्तिरीयसंहितायां द्वितीयकाण्डे पञ्चमः प्रश्नः}
%2.5.1.1
वि॒श्वरू॑पो॒ वै त्वा॒ष्ट्रः पु॒रोहि॑तो दे॒वाना॑मासीथ्स्व॒स्रीयो\-ऽसु॑राणां॒ तस्य॒ त्रीणि॑ शी॒र्॒\mbox{}षाण्या॑सन्थ्सोम॒पानꣳ॑ सुरा॒पान॑म॒न्नाद॑न॒ꣳ॒ स प्र॒त्यक्षं॑ दे॒वेभ्यो॑ भा॒गम॑वदत्प॒रोक्ष॒मसु॑रेभ्यः॒ सर्व॑स्मै॒ वै प्र॒त्यक्षं॑ भा॒गं व॑दन्ति॒ यस्मा॑ ए॒व प॒रोक्षं॒ वद॑न्ति॒ तस्य॑ भा॒ग उ॑दि॒तस्तस्मा॒दिन्द्रो॑\-ऽबिभेदी॒दृङ् वै रा॒ष्ट्रं वि प॒र्याव॑र्तय॒तीति॒ तस्य॒ वज्र॑मा॒दाय॑ शी॒र्॒\mbox{}षाण्य॑च्छिन॒द्यथ्सो॑म॒पान॒-~(१)

%2.5.1.2
मासी॒थ्स क॒पिञ्ज॑लो\-ऽभव॒द्यथ्सु॑रा॒पान॒ꣳ॒ स क॑ल॒विङ्को॒ यद॒न्नाद॑न॒ꣳ॒ स ति॑त्ति॒रिस्तस्या᳚ञ्ज॒लिना᳚ ब्रह्मह॒त्यामुपा॑गृह्णा॒त्ताꣳ सं॑वथ्स॒रम॑बिभ॒स्तं भू॒तान्य॒भ्य॑क्रोश॒न्ब्रह्म॑ह॒न्निति॒ स पृ॑थि॒वीमुपा॑सीद\-द॒स्यै ब्र॑ह्मह॒त्यायै॒ तृती॑यं॒ प्रति॑ गृहा॒णेति॒ सा\-ऽब्र॑वी॒द्वरं॑ वृणै खा॒तात्प॑रा\-भवि॒ष्यन्ती॑ मन्ये॒ ततो॒ मा परा॑ भूव॒मिति॑ पु॒रा ते॑~(२)

%2.5.1.3
संवथ्स॒रादपि॑ रोहा॒दित्य॑ब्रवी॒त्तस्मा᳚त्पु॒रा सं॑वथ्स॒रात्पृ॑थि॒व्यै खा॒तमपि॑ रोहति॒ वारे॑वृत॒ꣴ॒ ह्य॑स्यै॒ तृती॑यं ब्रह्मह॒त्यायै॒ प्रत्य॑\-गृह्णा॒त् तथ्स्वकृ॑त॒मिरि॑णमभव॒त् तस्मा॒दाहि॑ताग्निः श्र॒द्धादे॑वः॒ स्वकृ॑त॒ इरि॑णे॒ नाव॑ स्येद्ब्रह्मह॒त्यायै॒ ह्ये॑ष वर्णः॒ स वन॒स्पती॒नुपा॑सीदद॒स्यै ब्र॑ह्मह॒त्यायै॒ तृती॑यं॒ प्रति॑ गृह्णी॒तेति॒ ते᳚\-ऽब्रुव॒न्वरं॑ वृणामहै वृ॒क्णात्~(३)

%2.5.1.4
प॑राभवि॒ष्यन्तो॑ मन्यामहे॒ ततो॒ मा परा॑ भू॒मेत्या॒व्रश्च॑नाद्वो॒ भूयाꣳ॑स॒ उत्ति॑ष्ठा॒नित्य॑ब्रवी॒त् तस्मा॑दा॒व्रश्च॑नाद्वृ॒क्षाणां॒ भूयाꣳ॑स॒ उत्ति॑ष्ठन्ति॒ वारे॑वृत॒ꣴ॒ ह्ये॑षां॒ तृती॑यं ब्रह्मह॒त्यायै॒ प्रत्य॑\-गृह्ण॒न्थ्स नि॑र्या॒सो॑\-ऽभव॒त् तस्मा᳚न्निर्या॒सस्य॒ नाश्यं॑ ब्रह्मह॒त्यायै॒ ह्ये॑ष वर्णो\-ऽथो॒ खलु॒ य ए॒व लोहि॑तो॒ यो वा॒\-ऽऽ\-व्रश्च॑नान्नि॒र्येष॑ति॒ तस्य॒ ना\-ऽऽ\-श्यं॑~(४)

%2.5.1.5
काम॑म॒न्यस्य॒ स स्त्री॑षꣳसा॒दमुपा॑सीदद॒स्यै ब्र॑ह्मह॒त्यायै॒ तृती॑यं॒ प्रति॑ गृह्णी॒तेति॒ ता अ॑ब्रुव॒न्वरं॑ वृणामहा॒ ऋत्वि॑यात्प्र॒जां वि॑न्दामहै॒ काम॒मा विज॑नितोः॒ सम्भ॑वा॒मेति॒ तस्मा॒दृत्वि॑या॒थ्स्त्रियः॑ प्र॒जां वि॑न्दन्ते॒ काम॒मा विज॑नितोः॒ सम्भ॑वन्ति॒ वारे॑वृत॒ꣴ॒ ह्या॑सां॒ तृती॑यं ब्रह्मह॒त्यायै॒ प्रत्य॑\-गृह्ण॒न्थ्सा मल॑वद्वासा अभव॒त् तस्मा॒न्मल॑वद्वाससा॒ न सं व॑देत॒~(५)

%2.5.1.6
न स॒हा\-ऽऽ\-सी॑त॒ नास्या॒ अन्न॑मद्याद्ब्रह्मह॒त्यायै॒ ह्ये॑षा वर्णं॑ प्रति॒मुच्या\-ऽऽ\-स्ते\-ऽथो॒ खल्वा॑हुर॒भ्यञ्ज॑नं॒ वाव स्त्रि॒या अन्न॑म॒भ्यञ्ज॑नमे॒व न प्र॑ति॒गृह्यं॒ काम॑म॒न्यदिति॒ यां मल॑वद्वाससꣳ स॒म्भव॑न्ति॒ यस्ततो॒ जाय॑ते॒ सो॑\-ऽभिश॒स्तो यामर॑ण्ये॒ तस्यै᳚ स्ते॒नो यां परा॑चीं॒ तस्यै᳚ ह्रीतमु॒ख्य॑पग॒ल्भो या स्नाति॒ तस्या॑ अ॒फ्सु मारु॑को॒ या-~(६)

%2.5.1.7
ऽभ्य॒ङ्क्ते तस्यै॑ दु॒श्चर्मा॒ या प्र॑लि॒खते॒ तस्यै॑ खल॒तिर॑पमा॒री या\-ऽऽ\-ङ्क्ते तस्यै॑ का॒णो या द॒तो धाव॑ते॒ तस्यै᳚ श्या॒वद॒न्॒ या न॒खानि॑ निकृ॒न्तते॒ तस्यै॑ कुन॒खी या कृ॒णत्ति॒ तस्यै᳚ क्ली॒बो या रज्जुꣳ॑ सृ॒जति॒ तस्या॑ उ॒द्बन्धु॑को॒ या प॒र्णेन॒ पिब॑ति॒ तस्या॑ उ॒न्मादु॑को॒ या ख॒र्वेण॒ पिब॑ति॒ तस्यै॑ ख॒र्वस्ति॒स्रो रात्री᳚र्व्र॒तं च॑रेदञ्ज॒लिना॑ वा॒ पिबे॒दख॑र्वेण वा॒ पात्रे॑ण प्र॒जायै॑ गोपी॒थाय॑॥~(७)

{\anuvakamend[{यथ्सो॑म॒पान॑न्ते वृ॒क्णात् तस्य॒ नाश्यं॑ वदेत॒ मारु॑को॒ या\-ऽख॑र्वेण वा॒ त्रीणि॑ च}]}%~(१)

%2.5.2.1
त्वष्टा॑ ह॒तपु॑त्रो॒ वीन्द्र॒ꣳ॒ सोम॒माह॑र॒त् तस्मि॒न्निन्द्र॑ उपह॒वमै᳚च्छत॒ तं नोपा᳚ह्वयत पु॒त्रं मे॑\-ऽवधी॒रिति॒ स य॑ज्ञवेश॒सं कृ॒त्वा प्रा॒सहा॒ सोम॑मपिब॒त् तस्य॒ यद॒त्यशि॑ष्यत॒ तत् त्वष्टा॑हव॒नीय॒मुप॒ प्राव॑र्तय॒थ्\-स्वाहेन्द्र॑शत्रुर्वर्ध॒स्वेति॒ यदव॑र्तय॒त् तद्वृ॒त्रस्य॑ वृत्र॒त्वं यदब्र॑वी॒थ्\-स्वाहेन्द्र॑शत्रुर्वर्ध॒स्वेति॒ तस्मा॑द॒स्ये-~(८)

%2.5.2.2
न्द्रः॒ शत्रु॑रभव॒थ्स स॒म्भव॑न्न॒ग्नी\-षोमा॑व॒भि सम॑भव॒थ्स इ॑षुमा॒त्रमि॑षुमात्रं॒ विष्व॑ङ्ङवर्धत॒ स इ॒माँल्लो॒कान॑वृणो॒द् यदि॒माँल्लो॒का\-नवृ॑णो॒त् तद्वृ॒त्रस्य॑ वृत्र॒त्वं तस्मा॒दिन्द्रो॑\-ऽबिभे॒थ्स प्र॒जा\-प॑ति॒मुपा॑\-धाव॒च्छत्रु॑र्मे\-ऽज॒नीति॒ तस्मै॒ वज्रꣳ॑ सि॒क्त्वा प्राय॑च्छदे॒तेन॑ ज॒हीति॒ तेना॒भ्या॑यत॒ ताव॑ब्रूताम॒ग्नी\-षोमौ॒ मा~(९)

%2.5.2.3
प्र हा॑रा॒वम॒न्तः स्व॒ इति॒ मम॒ वै यु॒वꣴ स्थ॒ इत्य॑ब्रवी॒न्माम॒भ्ये\-त॒मिति॒ तौ भा॑ग॒धेय॑मैच्छेतां॒ ताभ्या॑मे॒त\-म॑ग्नीषो॒मीय॒\-मेका॑\-दश\-कपालं पू॒र्णमा॑से॒ प्राय॑च्छ॒त् ताव॑ब्रूताम॒भि सन्द॑ष्टौ॒ वै स्वो॒ न श॑क्नुव॒ ऐतु॒मिति॒ स इन्द्र॑ आ॒त्मनः॑ शीतरू॒राव॑जनय॒त् तच्छी॑तरू॒रयो॒र्जन्म॒ य ए॒वꣳ शी॑तरू॒रयो॒र्जन्म॒ वेद॒~(१०)


%2.5.2.4
नैनꣳ॑ शीतरू॒रौ ह॑त॒स्ताभ्या॑मेनम॒भ्य॑नय॒त् तस्मा᳚ज्जञ्ज॒भ्यमा॑\-नाद॒ग्नी\-षोमौ॒ निर॑क्रामतां प्राणापा॒नौ वा ए॑नं॒ तद॑जहितां प्रा॒णो वै दक्षो॑\-ऽपा॒नः क्रतु॒स्तस्मा᳚ज्जञ्ज॒भ्यमा॑नो ब्रूया॒न्मयि॑ दक्षक्र॒तू इति॑ प्राणापा॒नावे॒वा\-ऽऽ\-त्मन्ध॑त्ते॒ सर्व॒मायु॑रेति॒ स दे॒वता॑ वृ॒त्रान्नि॒र्॒\mbox{}हूय॒ वार्त्र॑घ्नꣳ ह॒विः पू॒र्णमा॑से॒ निर॑वप॒द् घ्नन्ति॒ वा ए॑नं पू॒र्णमा॑स॒ आ-~(११)

%2.5.2.5
ऽमा॑वा॒स्या॑यां प्याययन्ति॒ तस्मा॒द्वार्त्र॑घ्नी पू॒र्णमा॒से\-ऽनू᳚च्येते॒ वृध॑न्वती अमावा॒स्या॑यां॒ तथ्स॒ꣴ॒स्थाप्य॒ वार्त्र॑घ्नꣳ ह॒विर्वज्र॑मा॒दाय॒ पुन॑र॒भ्या॑यत॒ ते अ॑ब्रूतां॒ द्यावा॑\-पृथि॒वी मा प्र हा॑रा॒वयो॒र्वै श्रि॒त इति॒ ते अ॑ब्रूतां॒ वरं॑ वृणावहै॒ नक्ष॑त्रविहिता॒\-ऽहमसा॒नीत्य॒साव॑ब्रवीच्चि॒त्रवि॑हिता॒\-ऽहमिती॒यं तस्मा॒न्नक्ष॑त्रविहिता॒\-ऽसौ चि॒त्रवि॑हिते॒यं य ए॒वं द्यावा॑पृथि॒व्योर्-~(१२)

%2.5.2.6
वरं॒ वेदैनं॒ वरो॑ गच्छति॒ स आ॒भ्यामे॒व प्रसू॑त॒ इन्द्रो॑ वृ॒त्रम॑ह॒न्ते दे॒वा वृ॒त्रꣳ ह॒त्वा\-ऽग्नी\-षोमा॑वब्रुवन् ह॒व्यं नो॑ वहत॒मिति॒ ताव॑ब्रूता॒मप॑तेजसौ॒ वै त्यौ वृ॒त्रे वै त्ययो॒स्तेज॒ इति॒ ते᳚\-ऽब्रुव॒न्क इ॒दमच्छै॒तीति॒ गौरित्य॑ब्रुव॒न्गौर्वाव सर्व॑स्य मि॒त्रमिति॒ सा\-ऽब्र॑वी॒द्~(१३)

%2.5.2.7
वरं॑ वृणै॒ मय्ये॒व स॒तोभये॑न भुनजाध्वा॒ इति॒ तद्गौराह॑र॒त् तस्मा॒द्गवि॑ स॒तोभये॑न भुञ्जत ए॒तद्वा अ॒ग्नेस्तेजो॒ यद् घृ॒तमे॒तथ्सोम॑स्य॒ यत्पयो॒ य ए॒वम॒ग्नी\-षोम॑यो॒स्तेजो॒ वेद॑ तेज॒स्व्ये॑व भ॑वति ब्रह्मवा॒दिनो॑ वदन्ति किं देव॒त्यं॑ पौर्णमा॒समिति॑ प्राजाप॒त्यमिति॑ ब्रूया॒त् तेनेन्द्रं॑ ज्ये॒ष्ठं पु॒त्रं नि॒रवा॑सायय॒दिति॒ तस्मा᳚ज्ज्ये॒ष्ठं पु॒त्रं धने॑न नि॒रव॑साययन्ति॥~(१४)

{\anuvakamend[{अ॒स्य॒ मा वेदा द्यावा॑पृथि॒व्योर॑ब्रवी॒दिति॒ तस्मा᳚च्च॒त्वारि॑ च}]}%~(२)

%2.5.3.1
इन्द्रं॑ वृ॒त्रं ज॑घ्नि॒वाꣳस॒म्मृधो॒\-ऽभि प्रावे॑पन्त॒ स ए॒तं वै॑मृ॒धं पू॒र्णमा॑से\-ऽनुनिर्वा॒प्य॑मपश्य॒त्तं निर॑वप॒त् तेन॒ वै स मृधो\-ऽपा॑हत॒ यद्वै॑मृ॒धः पू॒र्णमा॑से\-ऽनुनिर्वा॒प्यो॑ भव॑ति॒ मृध॑ ए॒व तेन॒ यज॑मा॒नो\-ऽप॑ हत॒ इन्द्रो॑ वृ॒त्रꣳ ह॒त्वा दे॒वता॑भिश्चेन्द्रि॒येण॑ च॒ व्या᳚र्ध्यत॒ स ए॒तमा᳚ग्ने॒यम॒ष्टाक॑पालममावा॒स्या॑यामपश्यदै॒न्द्रं दधि॒~(१५)

%2.5.3.2
तन्निर॑वप॒त्तेन॒ वै स दे॒वता᳚श्चेन्द्रि॒यं चावा॑रुन्ध॒ यदा᳚ग्ने॒यो᳚\-ऽष्टाक॑पालो\-ऽमावा॒स्या॑यां॒ भव॑त्यै॒न्द्रं दधि॑ दे॒वता᳚श्चै॒व तेने᳚न्द्रि॒यं च॒ यज॑मा॒नो\-ऽव॑ रुन्ध॒ इन्द्र॑स्य वृ॒त्रं ज॒घ्नुष॑ इन्द्रि॒यं वी॒र्यं॑ पृथि॒वीमनु॒ व्या᳚र्च्छ॒त् तदोष॑धयो वी॒रुधो॑\-ऽभव॒न्थ्स प्र॒जा\-प॑ति॒मुपा॑धावद्वृ॒त्रं मे॑ ज॒घ्नुष॑ इन्द्रि॒यं वी॒र्यं॑~(१६)

%2.5.3.3
पृथि॒वीमनु॒ व्या॑र॒त् तदोष॑धयो वी॒रुधो॑\-ऽभूव॒न्निति॒ स प्र॒जा\-प॑तिः प॒शून॑ब्रवीदे॒तद॑स्मै॒ सं न॑य॒तेति॒ तत्प॒शव॒ ओष॑धी॒भ्यो\-ऽध्या॒त्मन्थ्सम॑नय॒न्तत्प्रत्य॑दुह॒न्॒ यथ्स॒मन॑य॒न्तथ्सा᳚न्ना॒य्यस्य॑ सान्नाय्य॒त्वं यत्प्र॒त्यदु॑ह॒न्तत्प्र॑ति॒धुषः॑ प्रतिधु॒क्त्वꣳ सम॑नैषुः॒ प्रत्य॑धुक्ष॒न्न तु मयि॑ श्रयत॒ इत्य॑ब्रवीदे॒तद॑स्मै~(१७)

%2.5.3.4
शृ॒तं कु॑रु॒तेत्य॑ब्रवी॒त् तद॑स्मै शृ॒तम॑कुर्वन्निन्द्रि॒यं वावास्मि॑न्वी॒र्यं॑ तद॑श्रय॒न्तच्छृ॒तस्य॑ शृत॒त्वꣳ सम॑नैषुः॒ प्रत्य॑धुक्षञ्छृ॒तम॑क्र॒न्न तु मा॑ धिनो॒तीत्य॑ब्रवीदे॒तद॑स्मै॒ दधि॑ कुरु॒तेत्य॑ब्रवी॒त् तद॑स्मै॒ दध्य॑कुर्व॒न्तदे॑नमधिनो॒त् तद्द॒ध्नो द॑धि॒त्वं ब्र॑ह्मवा॒दिनो॑ वदन्ति द॒ध्नः पूर्व॑स्याव॒देयं॒~(१८)

%2.5.3.5
दधि॒ हि पूर्वं॑ क्रि॒यत॒ इत्यना॑दृत्य॒ तच्छृ॒तस्यै॒व पूर्व॒स्या\-ऽव॑ द्येदिन्द्रि॒यमे॒वास्मि॑न्वी॒र्यꣴ॑ श्रि॒त्वा द॒ध्नोपरि॑ष्टाद्धिनोति यथापू॒र्वमुपै॑ति॒ यत्पू॒तीकै᳚र्वा पर्णव॒ल्कैर्वा॑त॒ञ्च्याथ्सौ॒म्यं तद्यत्क्व॑लै राक्ष॒सं तद्यत् त॑ण्डु॒लैर्वै᳚श्वदे॒वं तद्यदा॒तञ्च॑नेन मानु॒षं तद्यद्द॒ध्ना तथ्सेन्द्रं॑ द॒ध्ना त॑नक्ति~(१९)

%2.5.3.6
सेन्द्र॒त्वाया᳚ग्निहोत्रोच्छेष॒णम॒भ्यात॑नक्ति य॒ज्ञस्य॒ सन्त॑त्या॒ इन्द्रो॑ वृ॒त्रꣳ ह॒त्वा परां᳚ परा॒वत॑मगच्छ॒दपा॑राध॒मिति॒ मन्य॑मान॒स्तं दे॒वताः॒ प्रैष॑मैच्छ॒न्थ्सो᳚\-ऽब्रवीत्प्र॒जा\-प॑ति॒र्यः प्र॑थ॒मो॑\-ऽनुवि॒न्दति॒ तस्य॑ प्रथ॒मं भा॑ग॒धेय॒मिति॒ तं पि॒तरो\-ऽन्व॑विन्द॒न्तस्मा᳚त्पि॒तृभ्यः॑ पूर्वे॒द्युः क्रि॑यते॒ सो॑\-ऽमावा॒स्यां᳚ प्रत्याग॑च्छ॒त् तं दे॒वा अ॒भि सम॑गच्छन्ता॒\-ऽमा वै नो॒~-~(२०)

%2.5.3.7
ऽद्य वसु॑ वस॒तीतीन्द्रो॒ हि दे॒वानां॒ वसु॒ तद॑मावा॒स्या॑या अमावास्य॒त्वं ब्र॑ह्मवा॒दिनो॑ वदन्ति किं देव॒त्यꣳ॑ सान्ना॒य्यमिति॑ वैश्वदे॒वमिति॑ ब्रूया॒द्विश्वे॒ हि तद्दे॒वा भा॑ग॒धेय॑म॒भि स॒मग॑च्छ॒न्तेत्यथो॒ खल्वै॒न्द्रमित्ये॒व ब्रू॑या॒दिन्द्रं॒ वाव ते तद्भि॑ष॒ज्यन्तो॒\-ऽभि सम॑गच्छ॒न्तेति॑॥~(२१)

{\anuvakamend[{दधि॑ मे ज॒घ्नुष॑ इन्द्रि॒यं वी॒र्य॑मित्य॑ब्रवीदे॒तद॑स्मा अव॒देय॑न्तनक्ति नो॒ द्विच॑त्वारिꣳशच्च}]}%~(३)

%2.5.4.1
ब्र॒ह्म॒वा॒दिनो॑ वदन्ति॒ स त्वै द॑र्\mbox{}शपूर्णमा॒सौ य॑जेत॒ य ए॑नौ॒ सेन्द्रौ॒ यजे॒तेति॑ वैमृ॒धः पू॒र्णमा॑से\-ऽनुनिर्वा॒प्यो॑ भवति॒ तेन॑ पू॒र्णमा॑सः॒ सेन्द्र॑ ऐ॒न्द्रं दध्य॑मावा॒स्या॑यां॒ तेना॑मावा॒स्या॑ सेन्द्रा॒ य ए॒वं वि॒द्वान्द॑र्\mbox{}शपूर्णमा॒सौ यज॑ते॒ सेन्द्रा॑वे॒वैनौ॑ यजते॒ श्वःश्वो᳚\-ऽस्मा ईजा॒नाय॒ वसी॑यो भवति दे॒वा वै यद्य॒ज्ञे\-ऽकु॑र्वत॒ तदसु॑रा अकुर्वत॒ ते दे॒वा ए॒ता-~(२२)

%2.5.4.2
मिष्टि॑मपश्यन्नाग्नावैष्ण॒वमेका॑\-दश\-कपाल॒ꣳ॒ सर॑स्वत्यै च॒रुꣳ सर॑स्वते च॒रुं तां पौ᳚र्णमा॒सꣳ स॒ꣴ॒स्थाप्यानु॒ निर॑वप॒न्ततो॑ दे॒वा अभ॑व॒न्परासु॑रा॒ यो भ्रातृ॑व्यवा॒न्थ्स्याथ्स पौ᳚र्णमा॒सꣳ स॒ꣴ॒स्थाप्यै॒तामिष्टि॒मनु॒ निर्व॑पेत्पौर्णमा॒सेनै॒व वज्रं॒ भ्रातृ॑व्याय प्र॒हृत्या᳚\-ऽऽ\-ग्नावैष्ण॒वेन॑ दे॒वता᳚श्च य॒ज्ञं च॒ भ्रातृ॑व्यस्य वृङ्क्ते मिथु॒नान्प॒शून्थ्सा॑\-रस्व॒ताभ्यां॒ याव॑दे॒वास्यास्ति॒ तथ्~(२३)

%2.5.4.3
सर्वं॑ वृङ्क्ते पौर्णमा॒सीमे॒व य॑जेत॒ भ्रातृ॑व्यवा॒न्नामा॑वा॒स्याꣳ॑ ह॒त्वा भ्रातृ॑व्यं॒ ना प्या॑ययति साकं प्रस्था॒यीये॑न यजेत प॒शुका॑मो॒ यस्मै॒ वा अल्पे॑ना॒\-ऽऽ\-हर॑न्ति॒ ना\-ऽऽ\-त्मना॒ तृप्य॑ति॒ नान्यस्मै॑ ददाति॒ यस्मै॑ मह॒ता तृप्य॑त्या॒त्मना॒ ददा᳚त्य॒न्यस्मै॑ मह॒ता पू॒र्णꣳ हो॑त॒व्यं॑ तृ॒प्त ए॒वैन॒मिन्द्रः॑ प्र॒जया॑ प॒शुभि॑स्तर्पयति दारुपा॒त्रेण॑ जुहोति॒ न हि मृ॒न्मय॒माहु॑तिमान॒श औदु॑म्बरं~(२४)

%2.5.4.4
भव॒त्यूर्ग्वा उ॑दु॒म्बर॒ ऊर्क्प॒शव॑ ऊ॒र्जैवास्मा॒ ऊर्जं॑ प॒शूनव॑ रुन्धे॒ नाग॑तश्रीर्महे॒न्द्रं य॑जेत॒ त्रयो॒ वै ग॒तश्रि॑यः शुश्रु॒वान्ग्रा॑म॒णी रा॑ज॒न्य॑स्तेषां᳚ महे॒न्द्रो दे॒वता॒ यो वै स्वां दे॒वता॑मति॒यज॑ते॒ प्र स्वायै॑ दे॒वता॑यै च्यवते॒ न परां॒ प्राप्नो॑ति॒ पापी॑यान्भवति संवथ्स॒रमिन्द्रं॑ यजेत संवथ्स॒रꣳ हि व्र॒तं नाति॒ स्वै-~(२५)

%2.5.4.5
वैनं॑ दे॒वते॒ज्यमा॑ना॒ भूत्या॑ इन्द्धे॒ वसी॑यान्भवति संवथ्स॒रस्य॑ प॒रस्ता॑\-द॒ग्नये᳚ व्र॒तप॑तये पुरो॒डाश॑\-म॒ष्टा\-क॑पालं॒ निर्व॑पेथ्संवथ्स॒रमे॒वैनं॑ वृ॒त्रं ज॑घ्नि॒वाꣳस॑म॒ग्निर्व्र॒तप॑तिर्व्र॒तमा ल॑म्भयति॒ ततो\-ऽधि॒ कामं॑ यजेत॥~(२६)

{\anuvakamend[{ए॒तान्तदौदु॑म्बर॒ꣴ॒ स्वा त्रि॒ꣳ॒शच्च॑}]}%~(४)

%2.5.5.1
नासो॑मयाजी॒ सं न॑ये॒दना॑गतं॒ वा ए॒तस्य॒ पयो॒ यो\-ऽसो॑मयाजी॒ यदसो॑मयाजी स॒न्नये᳚त्परिमो॒ष ए॒व सो\-ऽनृ॑तं करो॒त्यथो॒ परै॒व सि॑च्यते सोमया॒ज्ये॑व सं न॑ये॒त्पयो॒ वै सोमः॒ पयः॑ सान्ना॒य्यं पय॑सै॒व पय॑ आ॒त्मन्ध॑त्ते॒ वि वा ए॒तं प्र॒जया॑ प॒शुभि॑रर्धयति व॒र्धय॑त्यस्य॒ भ्रातृ॑व्यं॒ यस्य॑ ह॒विर्निरु॑प्तं पु॒रस्ता᳚च्च॒न्द्रमा॑~-~(२७)

%2.5.5.2
अ॒भ्यु॑देति॑ त्रे॒धा त॑ण्डु॒लान् वि भ॑जे॒द्ये म॑ध्य॒माः स्युस्तान॒ग्नये॑ दा॒त्रे पु॑रो॒डाश॑म॒ष्टाक॑पालं कुर्या॒द्ये स्थवि॑ष्ठा॒स्तानिन्द्रा॑य प्रदा॒त्रे द॒धꣴश्च॒रुं ये\-ऽणि॑ष्ठा॒स्तान् विष्ण॑वे शिपिवि॒ष्टाय॑ शृ॒ते च॒रुम॒ग्निरे॒वास्मै᳚ प्र॒जां प्र॑ज॒नय॑ति वृ॒द्धामिन्द्रः॒ प्र य॑च्छति य॒ज्ञो वै विष्णुः॑ प॒शवः॒ शिपि॑र्य॒ज्ञ ए॒व प॒शुषु॒ प्रति॑ तिष्ठति॒ न द्वे~(२८)

%2.5.5.3
य॑जेत॒ यत्पूर्व॑या सम्प्र॒ति यजे॒तोत्त॑रया छ॒म्बट्कु॑र्या॒द्यदुत्त॑रया सम्प्र॒ति यजे॑त॒ पूर्व॑या छ॒म्बट्कु॑र्या॒न्नेष्टि॒र्भव॑ति॒ न य॒ज्ञस्तदनु॑ ह्रीतमु॒ख्य॑पग॒ल्भो जा॑यत॒ एका॑मे॒व य॑जेत प्रग॒ल्भो᳚\-ऽस्य जाय॒ते\-ऽना॑दृत्य॒ तद्द्वे ए॒व य॑जेत यज्ञमु॒खमे॒व पूर्व॑या॒\-ऽऽ\-लभ॑ते॒ यज॑त॒ उत्त॑रया दे॒वता॑ ए॒व पूर्व॑या\-ऽवरु॒न्ध इ॑न्द्रि॒यमुत्त॑रया देवलो॒कमे॒व~(२९)

%2.5.5.4
पूर्व॑या\-ऽभि॒जय॑ति मनुष्यलो॒कमुत्त॑रया॒ भूय॑सो यज्ञक्र॒तूनुपै᳚त्ये॒षा वै सु॒मना॒ नामेष्टि॒र्यम॒द्येजा॒नं प॒श्चाच्च॒न्द्रमा॑ अ॒भ्यु॑देत्य॒स्मिन्ने॒वास्मै॑ लो॒के\-ऽर्धु॑कं भवति दाक्षायणय॒ज्ञेन॑ सुव॒र्गका॑मो यजेत पू॒र्णमा॑से॒ सं न॑येन्मैत्रावरु॒ण्या\-ऽऽ\-\-मिक्ष॑या\-ऽमावा॒स्या॑यां यजेत पू॒र्णमा॑से॒ वै दे॒वानाꣳ॑ सु॒तस्तेषा॑मे॒तम॑र्धमा॒सं प्रसु॑त॒स्तेषां᳚ मैत्रावरु॒णी व॒शा\-ऽमा॑वा॒स्या॑यामनूब॒न्ध्या॑ यत्~(३०)

%2.5.5.5
पू᳚र्वे॒द्युर्यज॑ते॒ वेदि॑मे॒व तत्क॑रोति॒ यद्व॒थ्सान॑पाक॒रोति॑ सदोहविर्धा॒ने ए॒व सम्मि॑नोति॒ यद्यज॑ते दे॒वैरे॒व सु॒त्याꣳ सम्पा॑दयति॒ स ए॒तम॑र्धमा॒सꣳ स॑ध॒मादं॑ दे॒वैः सोमं॑ पिबति॒ यन्मै᳚त्रावरु॒ण्या\-ऽऽ\-\-मिक्ष॑या\-ऽमावा॒स्या॑यां यज॑ते॒ यैवासौ दे॒वानां᳚ व॒शा\-ऽनू॑ब॒न्ध्या॑ सो ए॒वैषैतस्य॑ सा॒क्षाद्वा ए॒ष दे॒वान॒भ्यारो॑हति॒ य ए॑षां य॒ज्ञ-~(३१)

%2.5.5.6
म॑भ्या॒रोह॑ति॒ यथा॒ खलु॒ वै श्रेया॑न॒भ्यारू॑ढः का॒मय॑ते॒ तथा॑ करोति॒ यद्य॑व॒विध्य॑ति॒ पापी॑यान्भवति॒ यदि॒ नाव॒विध्य॑ति स॒दृङ् व्या॒वृत्का॑म ए॒तेन॑ य॒ज्ञेन॑ यजेत क्षु॒रप॑वि॒र्॒\mbox{}ह्ये॑ष य॒ज्ञस्ता॒जक्पुण्यो॑ वा॒ भव॑ति॒ प्र वा॑ मीयते॒ तस्यै॒तद्व्र॒तं नानृ॑तं वदे॒न्न मा॒ꣳ॒सम॑श्ञीया॒न्न स्त्रिय॒\-मु\-पे॑या॒न्नास्य॒ पल्पू॑लनेन॒ वासः॑ पल्पूलयेयुरे॒तद्धि दे॒वाः सर्वं॒ न कु॒र्वन्ति॑॥~(३२)

{\anuvakamend[{च॒न्द्रमा॒ द्वे दे॑वलो॒कमे॒व यद्य॒ज्ञं प॑ल्पूलयेयुः॒ षट्च॑}]}%~(५)

%2.5.6.1
ए॒ष वै दे॑वर॒थो यद्द॑र्\mbox{}श\-पूर्ण\-मा॒सौ यो द॑र्\mbox{}शपूर्णमा॒सावि॒ष्ट्वा सोमे॑न॒ यज॑ते॒ रथ॑स्पष्ट ए॒वाव॒साने॒ वरे॑ दे॒वाना॒मव॑ स्यत्ये॒तानि॒ वा अङ्गा॒परूꣳ॑षि संवथ्स॒रस्य॒ यद्द॑र्\mbox{}श\-पूर्ण\-मा॒सौ य ए॒वं वि॒द्वान्द॑र्\mbox{}शपूर्णमा॒सौ यज॒ते\-ऽङ्गा॒परूꣴ॑ष्ये॒व सं॑वथ्स॒रस्य॒ प्रति॑ दधात्ये॒ते वै सं॑वथ्स॒रस्य॒ चक्षु॑षी॒ यद्द॑र्\mbox{}श\-पूर्ण\-मा॒सौ य ए॒वं वि॒द्वान्द॑र्\mbox{}शपूर्णमा॒सौ यज॑ते॒ ताभ्या॑मे॒व सु॑व॒र्गं लो॒कमनु॑ \mbox{पश्य-~(३३)}

%2.5.6.2
त्ये॒षा वै दे॒वानां॒ विक्रा᳚न्ति॒र्यद्द॑र्\mbox{}श\-पूर्ण\-मा॒सौ य ए॒वं वि॒द्वान्द॑र्\mbox{}शपूर्णमा॒सौ यज॑ते दे॒वाना॑मे॒व विक्रा᳚न्ति॒मनु॒ वि क्र॑मत ए॒ष वै दे॑व॒यानः॒ पन्था॒ यद्द॑र्\mbox{}श\-पूर्ण\-मा॒सौ य ए॒वं वि॒द्वान्द॑र्\mbox{}शपूर्णमा॒सौ यज॑ते॒ य ए॒व दे॑व॒यानः॒ पन्था॒स्तꣳ स॒मारो॑हत्ये॒तौ वै दे॒वाना॒ꣳ॒ हरी॒ यद्द॑र्\mbox{}श\-पूर्ण\-मा॒सौ य ए॒वं वि॒द्वान्द॑र्\mbox{}शपूर्णमा॒सौ यज॑ते॒ यावे॒व दे॒वाना॒ꣳ॒ हरी॒ ताभ्या॑-~(३४)

%2.5.6.3
मे॒वैभ्यो॑ ह॒व्यं व॑हत्ये॒तद्वै दे॒वाना॑मा॒स्यं॑ यद्द॑र्\mbox{}श\-पूर्ण\-मा॒सौ य ए॒वं वि॒द्वान्द॑र्\mbox{}शपूर्णमा॒सौ यज॑ते सा॒क्षादे॒व दे॒वाना॑मा॒स्ये॑ जुहोत्ये॒ष वै ह॑विर्धा॒नी यो द॑र्\mbox{}शपूर्णमासया॒जी सा॒यं प्रा॑तरग्निहो॒त्रं जु॑होति॒ यज॑ते दर्\mbox{}श\-पूर्ण\-मा॒सावह॑रहर्\mbox{}हविर्धा॒निनाꣳ॑ सु॒तो य ए॒वं वि॒द्वान्द॑र्\mbox{}शपूर्णमा॒सौ यज॑ते हविर्धा॒न्य॑स्मीति॒ सर्व॑मे॒वास्य॑ बर्\mbox{}हि॒ष्यं॑ द॒त्तं भ॑वति दे॒वा वा अह॑र्-~(३५)

%2.5.6.4
य॒ज्ञियं॒ नावि॑न्द॒न्ते द॑र्\mbox{}शपूर्णमा॒साव॑पुन॒न्तौ वा ए॒तौ पू॒तौ मेध्यौ॒ यद्द॑र्\mbox{}श\-पूर्ण\-मा॒सौ य ए॒वं वि॒द्वान्द॑र्\mbox{}शपूर्णमा॒सौ यज॑ते पू॒तावे॒वैनौ॒ मेध्यौ॑ यजते॒ नामा॑वा॒स्या॑यां च पौर्णमा॒स्यां च॒ स्त्रिय॒मुपे॑या॒द्यदु॑पे॒यान्निरि॑न्द्रियः स्या॒थ्सोम॑स्य॒ वै राज्ञो᳚\-ऽर्धमा॒सस्य॒ रात्र॑यः॒ पत्न॑य आस॒न्तासा॑ममावा॒स्यां᳚ च पौर्णमा॒सीं च॒ नोपै॒त्~(३६)

%2.5.6.5
ते ए॑नम॒भि सम॑नह्येतां॒ तं यक्ष्म॑ आर्च्छ॒द्राजा॑नं॒ यक्ष्म॑ आर॒दिति॒ तद्रा॑जय॒क्ष्मस्य॒ जन्म॒ यत्पापी॑या॒नभ॑व॒त् त\-त्पा॑पय॒क्ष्मस्य॒ यज्जा॒याभ्या॒मवि॑न्द॒त् तज्जा॒येन्य॑स्य॒ य ए॒वमे॒तेषां॒ यक्ष्मा॑णां॒ जन्म॒ वेद॒ नैन॑मे॒ते यक्ष्मा॑ विन्दन्ति॒ स ए॒ते ए॒व न॑म॒स्यन्नुपा॑धाव॒त्ते अ॑ब्रूतां॒ वरं॑ वृणावहा आ॒वं दे॒वानां᳚ भाग॒धे अ॑सावा॒-~(३७)

%2.5.6.6
ऽ\-ऽवदधि॑ दे॒वा इ॑ज्यान्ता॒ इति॒ तस्मा᳚थ्स॒दृशी॑ना॒ꣳ॒ रात्री॑णाममावा॒स्या॑यां च पौर्णमा॒स्यां च॑ दे॒वा इ॑ज्यन्त ए॒ते हि दे॒वानां᳚ भाग॒धे भा॑ग॒धा अ॑स्मै मनु॒ष्या॑ भवन्ति॒ य ए॒वं वेद॑ भू॒तानि॒ क्षुध॑मघ्नन्थ्स॒द्यो म॑नु॒ष्या॑ अर्धमा॒से दे॒वा मा॒सि पि॒तरः॑ संवथ्स॒रे वन॒स्पत॑य॒स्तस्मा॒दह॑रहर्मनु॒ष्या॑ अश॑नमिच्छन्ते\-ऽर्धमा॒से दे॒वा इ॑ज्यन्ते मा॒सि पि॒तृभ्यः॑ क्रियते संवथ्स॒रे वन॒स्पत॑यः॒ फलं॑ गृह्णन्ति॒ य ए॒वं वेद॒ हन्ति॒ क्षुधं॒ भ्रातृ॑व्यम्॥~(३८)

{\anuvakamend[{प॒श्य॒ति॒ ताभ्या॒मह॑रैदसाव॒ फलꣳ॑ स॒प्त च॑}]}%~(६)

%2.5.7.1
दे॒वा वै नर्चि न यजु॑ष्यश्रयन्त॒ ते साम॑न्ने॒वाश्र॑यन्त॒ हिं क॑रोति॒ सामै॒वाक॒र्॒\mbox{}हिं क॑रोति॒ यत्रै॒व दे॒वा अश्र॑यन्त॒ तत॑ ए॒वैना॒न्प्र यु॑ङ्क्ते॒ हिं क॑रोति वा॒च ए॒वैष योगो॒ हिं क॑रोति प्र॒जा ए॒व तद्यज॑मानः सृजते॒ त्रिः प्र॑थ॒मामन्वा॑ह॒ त्रिरु॑त्त॒मां य॒ज्ञस्यै॒व तद्ब॒र्॒\mbox{}सं~(३९)

%2.5.7.2
न॑ह्य॒त्यप्र॑स्रꣳसाय॒ सन्त॑त॒मन्वा॑ह प्रा॒णाना॑म॒न्नाद्य॑स्य॒ सन्त॑त्या॒ अथो॒ रक्ष॑सा॒मप॑हत्यै॒ राथ॑न्तरीं प्रथ॒मामन्वा॑ह॒ राथ॑न्तरो॒ वा अ॒यं लो॒क इ॒ममे॒व लो॒कम॒भि ज॑यति॒ त्रिर्वि गृ॑ह्णाति॒ त्रय॑ इ॒मे लो॒का इ॒माने॒व लो॒कान॒भि ज॑यति॒ बार्\mbox{}ह॑तीमुत्त॒मामन्वा॑ह॒ बार्\mbox{}ह॑तो॒ वा अ॒सौ लो॒को॑\-ऽमुमे॒व लो॒कम॒भि ज॑यति॒ प्र वो॒~-~(४०)

%2.5.7.3
वाजा॒ इत्यनि॑रुक्तां प्राजाप॒त्यामन्वा॑ह य॒ज्ञो वै प्र॒जा\-प॑तिर्य॒ज्ञमे॒व प्र॒जा\-प॑ति॒मा र॑भते॒ प्र वो॒ वाजा॒ इत्यन्वा॒हान्नं॒ वै वाजो\-ऽन्न॑मे॒वाव॑ रुन्धे॒ प्र वो॒ वाजा॒ इत्यन्वा॑ह॒ तस्मा᳚त्प्रा॒चीन॒ꣳ॒ रेतो॑ धीय॒ते\-ऽग्न॒ आ या॑हि वी॒तय॒ इत्या॑ह॒ तस्मा᳚त्प्र॒तीचीः᳚ प्र॒जा जा॑यन्ते॒ प्र वो॒ वाजा॒~-~(४१)

%2.5.7.4
इत्यन्वा॑ह॒ मासा॒ वै वाजा॑ अर्धमा॒सा अ॒भिद्य॑वो दे॒वा ह॒विष्म॑न्तो॒ गौर्घृ॒ताची॑ य॒ज्ञो दे॒वाञ्जि॑गाति॒ यज॑मानः सुम्न॒युरि॒दम॑सी॒दम॒सीत्ये॒व य॒ज्ञस्य॑ प्रि॒यं धामाव॑ रुन्धे॒ यं का॒मये॑त॒ सर्व॒मायु॑रिया॒दिति॒ प्र वो॒ वाजा॒ इति॒ तस्या॒नूच्याग्न॒ आ या॑हि वी॒तय॒ इति॒ सन्त॑त॒मुत्त॑रमर्ध॒र्चमा ल॑भेत~(४२)

%2.5.7.5
प्रा॒णेनै॒वास्या॑पा॒नं दा॑धार॒ सर्व॒मायु॑रेति॒ यो वा अ॑र॒त्निꣳ सा॑मिधे॒नीनां॒ वेदा॑र॒त्नावे॒व भ्रातृ॑व्यं कुरुते\-ऽर्ध॒र्चौ सं द॑धात्ये॒ष वा अ॑र॒त्निः सा॑मिधे॒नीनां॒ य ए॒वं वेदा॑र॒त्नावे॒व भ्रातृ॑व्यं कुरुत॒ ऋषेर्॑\mbox{}ऋषे॒र्वा ए॒ता निर्मि॑ता॒ यथ्सा॑मिधे॒न्य॑स्ता यदसं॑युक्ताः॒ स्युः प्र॒जया॑ प॒शुभि॒र्यज॑मानस्य॒ वि ति॑ष्ठेरन्नर्ध॒र्चौ सन्द॑धाति॒ सं यु॑नक्त्ये॒वैना॒स्ता अ॑स्मै॒ संयु॑क्ता॒ अव॑रुद्धाः॒ सर्वा॑मा॒शिषं॑ दुह्रे॥~(४३)


{\anuvakamend[{ब॒र्॒\mbox{}सं वो॑ जायन्ते॒ प्र वो॒ वाजा॑ लभेत दधाति॒ सन्दश॑ च}]}%~(७)

%2.5.8.1
अय॑ज्ञो॒ वा ए॒ष यो॑\-ऽसा॒मा\-ऽग्न॒ आ या॑हि वी॒तय॒ इत्या॑ह रथन्त॒रस्यै॒ष वर्ण॒स्तं त्वा॑ स॒मिद्भि॑रङ्गिर॒ इत्या॑ह वामदे॒व्यस्यै॒ष वर्णो॑ बृ॒हद॑ग्ने सु॒वीर्य॒मित्या॑ह बृह॒त ए॒ष वर्णो॒ यदे॒तं तृ॒चम॒न्वाह॑ य॒ज्ञमे॒व तथ्साम॑न्वन्तं करोत्य॒ग्निर॒मुष्मिँ॑ल्लो॒क आसी॑दादि॒त्यो᳚\-ऽस्मिन्तावि॒मौ लो॒कावशा᳚न्ता-~(४४)

%2.5.8.2
वास्तां॒ ते दे॒वा अ॑ब्रुव॒न्नेते॒मौ वि पर्यू॑हा॒मेत्यग्न॒ आ या॑हि वी॒तय॒ इत्य॒स्मिँल्लो॒के᳚\-ऽग्निम॑दधुर्बृ॒हद॑ग्ने सु॒वीर्य॒मित्य॒मुष्मिँ॑ल्लो॒क आ॑दि॒त्यं ततो॒ वा इ॒मौ लो॒काव॑शाम्यतां॒ यदे॒वम॒न्वाहा॒नयो᳚र्लो॒कयोः॒ शान्त्यै॒ शाम्य॑तो\-ऽस्मा इ॒मौ लो॒कौ य ए॒वं वेद॒ पञ्च॑दश सामिधे॒नीरन्वा॑ह॒ पञ्च॑दश॒~(४५)

%2.5.8.3
वा अ॑र्धमा॒सस्य॒ रात्र॑यो\-ऽर्धमास॒शः सं॑वथ्स॒र आ᳚प्यते॒ तासां॒ त्रीणि॑ च श॒तानि॑ ष॒ष्टिश्चा॒क्षरा॑णि॒ ताव॑तीः संवथ्स॒रस्य॒ रात्र॑यो\-ऽक्षर॒श ए॒व सं॑वथ्स॒रमा᳚प्नोति नृ॒मेध॑श्च॒ परु॑च्छेपश्च ब्रह्म॒वाद्य॑मवदेताम॒स्मिन्दारा॑वा॒र्द्रे᳚\-ऽग्निं ज॑नयाव यत॒रो नौ॒ ब्रह्मी॑या॒निति॑ नृ॒मेधो॒\-ऽभ्य॑वद॒थ्स धू॒मम॑जनय॒त्परु॑च्छेपो॒\-ऽभ्य॑वद॒थ्सो᳚\-ऽग्निम॑जनय॒दृष॒ इत्य॑ब्रवी॒द्-~(४६)

%2.5.8.4
यथ्स॒माव॑द्वि॒द्व क॒था त्वम॒ग्निमजी॑जनो॒ नाहमिति॑ सामिधे॒नीना॑मे॒वाहं वर्णं॑ वे॒देत्य॑ब्रवी॒द्यद् घृ॒तव॑त्प॒दम॑नू॒च्यते॒ स आ॑सां॒ वर्ण॒स्तं त्वा॑ स॒मिद्भि॑रङ्गिर॒ इत्या॑ह सामिधे॒नीष्वे॒व तज्ज्योति॑र्जनयति॒ स्त्रिय॒स्तेन॒ यदृचः॒ स्त्रिय॒स्तेन॒ यद्गा॑य॒त्रियः॒ स्त्रिय॒स्तेन॒ यथ्सा॑मिधे॒न्यो॑ वृष॑ण्वती॒मन्वा॑ह॒~(४७)

%2.5.8.5
तेन॒ पुꣴस्व॑ती॒स्तेन॒ सेन्द्रा॒स्तेन॑ मिथु॒ना अ॒ग्निर्दे॒वानां᳚ दू॒त आसी॑दु॒शना॑ का॒व्यो\-ऽसु॑राणां॒ तौ प्र॒जा\-प॑तिं प्र॒श्ञमै॑ता॒ꣳ॒ स प्र॒जा\-प॑तिर॒ग्निं दू॒तं वृ॑णीमह॒ इत्य॒भि प॒र्याव॑र्तत॒ ततो॑ दे॒वा अभ॑व॒न्परासु॑रा॒ यस्यै॒वं वि॒दुषो॒\-ऽग्निं दू॒तं वृ॑णीमह॒ इत्य॒न्वाह॒ भव॑त्या॒त्मना॒ परा᳚स्य॒ भ्रातृ॑व्यो भवत्यध्व॒रव॑ती॒मन्वा॑ह॒ भ्रातृ॑व्यमे॒वैतया᳚~(४८)

%2.5.8.6
ध्वरति शो॒चिष्के॑श॒स्तमी॑मह॒ इत्या॑ह प॒वित्र॑मे॒वैतद्यज॑\-मान\-मे॒वै\-तया॑ पवयति॒ समि॑द्धो अग्न आहु॒तेत्या॑ह परि॒धिमे॒वैतं परि॑ दधा॒\-त्य\-स्क॑न्दाय॒ यदत॑ ऊ॒र्ध्वम॑भ्याद॒ध्याद्यथा॑ बहिःपरि॒धि स्कन्द॑ति ता॒दृगे॒व तत् त्रयो॒ वा अ॒ग्नयो॑ हव्य॒वाह॑नो दे॒वानां᳚ कव्य॒वाह॑नः पितृ॒णाꣳ स॒हर॑क्षा॒ असु॑राणां॒ त ए॒तर्\mbox{}ह्याशꣳ॑सन्ते॒ मां व॑रिष्यते॒ मा-~(४९)

%2.5.8.7
मिति॑ वृणी॒ध्वꣳ ह॑व्य॒वाह॑न॒मित्या॑ह॒ य ए॒व दे॒वानां॒ तं वृ॑णीत आर्\mbox{}षे॒यं वृ॑णीते॒ बन्धो॑रे॒व नैत्यथो॒ सन्त॑त्यै प॒रस्ता॑द॒र्वाचो॑ वृणीते॒ तस्मा᳚त्प॒रस्ता॑द॒र्वाञ्चो॑ मनु॒ष्या᳚न्पि॒तरो\-ऽनु॒ प्र पि॑पते॥~(५०)

{\anuvakamend[{अशा᳚न्तावाह॒ पञ्च॑दशाब्रवी॒दन्वा॑है॒तया॑ वरिष्यते॒ मामेका॒न्नत्रि॒ꣳ॒शच्च॑}]}%~(८)

%2.5.9.1
अग्ने॑ म॒हाꣳ अ॒सीत्या॑ह म॒हान् ह्ये॑ष यद॒ग्निर्ब्रा᳚ह्म॒णेत्या॑ह ब्राह्म॒णो ह्ये॑ष भा॑र॒तेत्या॑है॒ष हि दे॒वेभ्यो॑ ह॒व्यं भर॑ति दे॒वेद्ध॒ इत्या॑ह दे॒वा ह्ये॑तमैन्ध॑त॒ मन्वि॑द्ध॒ इत्या॑ह॒ मनु॒र्ह्ये॑तमुत्त॑रो दे॒वेभ्य॒ ऐन्द्धर्\mbox{}षि॑ष्टुत॒ इत्या॒हर्\mbox{}ष॑यो॒ ह्ये॑तमस्तु॑व॒न्विप्रा॑नुमदित॒ इत्या॑ह॒~(५१)

%2.5.9.2
विप्रा॒ ह्ये॑ते यच्छु॑श्रु॒वाꣳसः॑ कविश॒स्त इत्या॑ह क॒वयो॒ ह्ये॑ते यच्छु॑श्रु॒वाꣳसो॒ ब्रह्म॑सꣳशित॒ इत्या॑ह॒ ब्रह्म॑सꣳशितो॒ ह्ये॑ष घृ॒ताह॑वन॒ इत्या॑ह घृताहु॒तिर्\mbox{}ह्य॑स्य प्रि॒यत॑मा प्र॒णीर्य॒ज्ञाना॒मित्या॑ह प्र॒णीर्\mbox{}ह्ये॑ष य॒ज्ञानाꣳ॑ र॒थीर॑ध्व॒राणा॒मित्या॑है॒ष हि दे॑वर॒थो॑\-ऽतूर्तो॒ होतेत्या॑ह॒ न ह्ये॑तं कश्च॒न~(५२)

%2.5.9.3
तर॑ति॒ तूर्णि॑र्\mbox{}हव्य॒वाडित्या॑ह॒ सर्व॒ꣴ॒ ह्ये॑ष तर॒त्यास्पात्रं॑ जु॒हूर्दे॒वा\-ना॒\-मि\-त्या॑ह जु॒हूर्\mbox{}ह्ये॑ष दे॒वानां᳚ चम॒सो दे॑व॒पान॒ इत्या॑ह चम॒सो ह्ये॑ष दे॑व॒पानो॒\-ऽराꣳ इ॑वाग्ने ने॒मिर्दे॒वाꣴस्त्वं प॑रि॒भूर॒सीत्या॑ह दे॒वान् ह्ये॑ष प॑रि॒\-भूर्यद्ब्रू॒\-यादा व॑ह दे॒वान्दे॑वय॒ते यज॑माना॒येति॒ भ्रातृ॑व्यमस्मै~(५३)

%2.5.9.4
जनये॒दा व॑ह दे॒वान् यज॑माना॒येत्या॑ह॒ यज॑मानमे॒वैतेन॑ वर्ध\-य\-त्य॒\-ग्नि\-म॑ग्न॒ आ व॑ह॒ सोम॒मा व॒हेत्या॑ह दे॒वता॑ ए॒व तद्य॑थापू॒र्वमुप॑ ह्वयत॒ आ चा᳚ग्ने दे॒वान् वह॑ सु॒यजा॑ च यज जातवेद॒ इत्या॑हा॒ग्निमे॒व तथ्सꣴ श्य॑ति॒ सो᳚\-ऽस्य॒ सꣳशि॑तो दे॒वेभ्यो॑ ह॒व्यं व॑हत्य॒ग्निर्\mbox{}होते-~(५४)

%2.5.9.5
त्या॑हा॒ग्निर्वै दे॒वाना॒ꣳ॒ होता॒ य ए॒व दे॒वाना॒ꣳ॒ होता॒ तं वृ॑णीते॒ स्मो व॒यमित्या॑हा॒\-ऽ॒\-ऽ॒त्मान॑मे॒व स॒त्त्वं ग॑मयति सा॒धु ते॑ यजमान दे॒वतेत्या॑हा॒\-ऽऽ\-शिष॑मे॒वैतामा शा᳚स्ते॒ यद्ब्रू॒याद्यो᳚\-ऽग्निꣳ होता॑र॒मवृ॑था॒ इत्य॒ग्निनो॑भ॒यतो॒ यज॑मानं॒ परि॑ गृह्णीयात् प्र॒मायु॑कः स्याद्यजमानदेव॒त्या॑ वै जु॒हूर्भ्रा॑तृव्यदेव॒त्यो॑प॒भृद्-~(५५)

%2.5.9.6
यद्द्वे इ॑व ब्रू॒याद्भ्रातृ॑व्यमस्मै जनयेद् घृ॒तव॑तीमध्वर्यो॒ स्रुच॒मास्य॒\-स्वे\-त्या॑ह॒ यज॑मानमे॒वैतेन॑ वर्धयति देवा॒युव॒मित्या॑ह दे॒वान् ह्ये॑षा\-व॑ति वि॒श्व\-वा॑रा॒मित्या॑ह॒ विश्व॒ꣴ॒ ह्ये॑षाव॒तीडा॑महै दे॒वाꣳ ई॒डेन्या᳚न्नम॒स्याम॑ नम॒स्यान्॑ यजा॑म य॒ज्ञिया॒नित्या॑ह मनु॒ष्या॑ वा ई॒डेन्याः᳚ पि॒तरो॑ नम॒स्या॑ दे॒वा य॒ज्ञिया॑ दे॒वता॑ ए॒व तद्य॑थाभा॒गं य॑जति॥~(५६)

{\anuvakamend[{विप्रा॑नुमदित॒ इत्या॑ह च॒नास्मै॒ होतो॑प॒भृद्दे॒वता॑ ए॒व त्रीणि॑ च}]}%~(९)

%2.5.10.1
त्रीꣴ स्तृ॒चाननु॑ ब्रूयाद्राज॒न्य॑स्य॒ त्रयो॒ वा अ॒न्ये रा॑ज॒न्या᳚त्पुरु॑षा ब्राह्म॒णो वैश्यः॑ शू॒द्रस्ताने॒वास्मा॒ अनु॑कान्करोति॒ पञ्च॑द॒शानु॑ ब्रूयाद् राज॒न्य॑स्य पञ्चद॒शो वै रा॑ज॒न्यः॑ स्व ए॒वैन॒ꣴ॒ स्तोमे॒ प्रति॑\-ष्ठापयति त्रि॒ष्टुभा॒ परि॑ दध्यादिन्द्रि॒यं वै त्रि॒ष्टुगि॑न्द्रि॒यका॑मः॒ खलु॒ वै रा॑ज॒न्यो॑ यजते त्रि॒ष्टुभै॒वास्मा॑ इन्द्रि॒यं परि॑ गृह्णाति॒ यदि॑ का॒मये॑त~(५७)

%2.5.10.2
ब्रह्मवर्च॒सम॒स्त्विति॑ गायत्रि॒या परि॑ दध्याद्ब्रह्मवर्च॒सं वै गा॑य॒त्री ब्र॑ह्मवर्च॒समे॒व भ॑वति स॒प्तद॒शानु॑ ब्रूया॒द्वैश्य॑स्य सप्तद॒शो वै वैश्यः॒ स्व ए॒वैन॒ꣴ॒ स्तोमे॒ प्रति॑\-ष्ठापयति॒ जग॑त्या॒ परि॑ दध्या॒ज्जाग॑ता॒ वै प॒शवः॑ प॒शुका॑मः॒ खलु॒ वै वैश्यो॑ यजते॒ जग॑त्यै॒वास्मै॑ प॒शून्परि॑ गृह्णा॒त्येक॑विꣳशति॒मनु॑ ब्रूयात्प्रति॒ष्ठाका॑मस्यैकवि॒ꣳ॒शः स्तोमा॑नां प्रति॒ष्ठा प्रति॑ष्ठित्यै॒~(५८)

%2.5.10.3
चतु॑र्विꣳशति॒मनु॑ ब्रूयाद्ब्रह्मवर्च॒सका॑मस्य॒ चतु॑र्विꣳशत्यक्षरा गाय॒त्री गा॑य॒त्री ब्र॑ह्मवर्च॒सं गा॑यत्रि॒यैवास्मै᳚ ब्रह्मवर्च॒समव॑ रुन्धे त्रि॒ꣳ॒शत॒\-मनु॑ ब्रूया॒दन्न॑कामस्य त्रि॒ꣳ॒शद॑क्षरा वि॒राडन्नं॑ वि॒राड्वि॒रा\-जै॒\-वा\-स्मा॑ अ॒न्नाद्\-य॒\-मव॑ रुन्धे॒ द्वात्रिꣳ॑शत॒\-मनु॑\-ब्रूयात्प्र\-ति॒ष्ठा\-का॑मस्य॒ द्वात्रिꣳ॑शद\-क्षरा\-नु॒ष्टु॑ग\-नु॒ष्टुप्छन्द॑सां प्रति॒ष्ठा प्रति॑ष्ठित्यै॒ षट्त्रिꣳ॑शत॒मनु॑ ब्रूयात्प॒शुका॑मस्य॒ षट्त्रिꣳ॑शदक्षरा बृह॒ती बार्\mbox{}ह॑ताः प॒शवो॑ बृह॒त्यैवास्मै॑ प॒शू-~(५९)

%2.5.10.4
नव॑ रुन्धे॒ चतु॑श्चत्वारिꣳशत॒मनु॑ ब्रूयादिन्द्रि॒यका॑मस्य॒ चतु॑श्चत्वारिꣳशदक्षरा त्रि॒ष्टुगि॑न्द्रि॒यं त्रि॒ष्टुप्त्रि॒ष्टुभै॒वास्मा॑ इन्द्रि॒यमव॑ रुन्धे॒\-ऽष्टाच॑त्वारिꣳशत॒मनु॑ ब्रूयात्प॒शुका॑मस्या॒ष्टाच॑त्वारिꣳशदक्षरा॒ जग॑ती॒ जाग॑ताः प॒शवो॒ जग॑त्यै॒वास्मै॑ प॒शूनव॑ रुन्धे॒ सर्वा॑णि॒ छन्दा॒ꣴ॒स्यनु॑ ब्रूयाद्बहुया॒जिनः॒ सर्वा॑णि॒ वा ए॒तस्य॒ छन्दा॒ꣴ॒स्यव॑रुद्धानि॒ यो ब॑हुया॒ज्यप॑रिमित॒मनु॑ ब्रूया॒दप॑रिमित॒स्याव॑रुद्ध्यै॥~(६०)

{\anuvakamend[{का॒मये॑त॒ प्रति॑ष्ठित्यै प॒शून्थ्स॒प्तच॑त्वारिꣳशच्च}]}%॥10॥

%2.5.11.1
निवी॑तं मनु॒ष्या॑णां प्राचीनावी॒तं पि॑तृ॒णामुप॑वीतं दे॒वाना॒मुप॑ व्ययते देवल॒क्ष्ममे॒व तत्कु॑रुते॒ तिष्ठ॒न्नन्वा॑ह॒ तिष्ठ॒न्॒ ह्याश्रु॑ततरं॒ वद॑ति॒ तिष्ठ॒न्नन्वा॑ह सुव॒र्गस्य॑ लो॒कस्या॒भिजि॑त्या॒ आसी॑नो यजत्य॒स्मिन्ने॒व लो॒के प्रति॑ तिष्ठति॒ यत्क्रौ॒ञ्चम॒न्वाहा॑\-ऽऽ\-सु॒रं तद्यन्म॒न्द्रं मा॑नु॒षं तद्यद॑न्त॒रा तथ्सदे॑वमन्त॒रानूच्यꣳ॑ सदेव॒त्वाय॑ वि॒द्वाꣳसो॒ वै~(६१)

%2.5.11.2
पु॒रा होता॑रो\-ऽभूव॒न्तस्मा॒द्विधृ॑ता॒ अध्वा॒नो\-ऽभू॑व॒न्न पन्था॑नः॒ सम॑रुक्षन्नन्तर्वे॒द्य॑न्यः पादो॒ भव॑ति बहिर्वे॒द्य॑न्यो\-ऽथान्वा॒हाध्व॑नां॒ विधृ॑त्यै प॒थामसꣳ॑रोहा॒याथो॑ भू॒तं चै॒व भ॑वि॒ष्यच्चाव॑ रु॒न्धे\-ऽथो॒ परि॑मितं चै॒वाप॑रिमितं॒ चाव॑ रु॒न्धे\-ऽथो᳚ ग्रा॒म्याꣴश्चै॒व प॒शूना॑र॒ण्याꣴश्चाव॑ रु॒न्धे\-ऽथो॑~(६२)

%2.5.11.3
देवलो॒कं चै॒व म॑नुष्यलो॒कं चा॒भि ज॑यति दे॒वा वै सा॑मिधे॒नीर॒नूच्य॑ य॒ज्ञं नान्व॑पश्य॒न्थ्स प्र॒जा\-प॑तिस्तू॒ष्णीमा॑घा॒र\-माघा॑र\-य॒त् ततो॒ वै दे॒वा य॒ज्ञमन्व॑पश्य॒न्॒ यत् तू॒ष्णीमा॑\-घा॒रमा॑\-घा॒रय॑ति य॒ज्ञस्यानु॑\-ख्यात्या॒ अथो॑ सामिधे॒नीरे॒वाभ्य॑न॒क्त्यलू᳚क्षो भवति॒ य ए॒वं वेदाथो॑ त॒र्पय॑त्ये॒वैना॒स्तृप्य॑ति प्र॒जया॑ प॒शुभि॒र्-~(६३)

%2.5.11.4
य ए॒वं वेद॒ यदेक॑याघा॒रये॒देकां᳚ प्रीणीया॒द्यद्द्वाभ्यां॒ द्वे प्री॑णीया॒द्यत् ति॒सृभि॒रति॒ तद्रे॑चये॒न्मन॒सा घा॑रयति॒ मन॑सा॒ ह्यना᳚प्तमा॒प्यते॑ ति॒र्यञ्च॒मा घा॑रय॒त्यछ॑म्बट्कारं॒ वाक्च॒ मन॑श्चार्तीयेताम॒हं दे॒वेभ्यो॑ ह॒व्यं व॑हा॒मीति॒ वाग॑ब्रवीद॒हं दे॒वेभ्य॒ इति॒ मन॒स्तौ प्र॒जा\-प॑तिं प्र॒श्ञमै॑ता॒ꣳ॒ सो᳚\-ऽब्रवीत्~(६४)

%2.5.11.5
प्र॒जा\-प॑तिर्दू॒तीरे॒व त्वं मन॑सो\-ऽसि॒ यद्धि मन॑सा॒ ध्याय॑ति॒ तद्वा॒चा वद॒तीति॒ तत्खलु॒ तुभ्यं॒ न वा॒चा जु॑हव॒न्नित्य॑ब्रवी॒त् तस्मा॒न्मन॑सा प्र॒जा\-प॑तये जुह्वति॒ मन॑ इव॒ हि प्र॒जा\-प॑तिः प्र॒जा\-प॑ते॒राप्त्यै॑ परि॒धीन्थ्सम्मा᳚र्ष्टि पु॒नात्ये॒वैना॒न्त्रिर्म॑ध्य॒मं त्रयो॒ वै प्रा॒णाः प्रा॒णाने॒वाभि ज॑यति॒ त्रिर्द॑क्षिणा॒र्ध्यं॑ त्रय॑~-~(६५)

%2.5.11.6
इ॒मे लो॒का इ॒माने॒व लो॒कान॒भि ज॑यति॒ त्रिरु॑त्तरा॒र्ध्यं॑ त्रयो॒ वै दे॑व॒यानाः॒ पन्था॑न॒स्ताने॒वाभि ज॑यति॒ त्रिरुप॑ वाजयति॒ त्रयो॒ वै दे॑वलो॒का दे॑वलो॒काने॒वाभि ज॑यति॒ द्वाद॑श॒ सं प॑द्यन्ते॒ द्वाद॑श॒ मासाः᳚ संवथ्स॒रः सं॑वथ्स॒रमे॒व प्री॑णा॒त्यथो॑ संवथ्स॒रमे॒वास्मा॒ उप॑ दधाति सुव॒र्गस्य॑ लो॒कस्य॒ सम॑ष्ट्या आघा॒रमा घा॑रयति ति॒र इ॑व॒~(६६)

%2.5.11.7
वै सु॑व॒र्गो लो॒कः सु॑व॒र्गमे॒वास्मै॑ लो॒कं प्र रो॑चयत्यृ॒जुमा घा॑रयत्यृ॒जुरि॑व॒ हि प्रा॒णः सन्त॑त॒मा घा॑रयति प्रा॒णाना॑म॒न्नाद्य॑स्य॒ सन्त॑त्या॒ अथो॒ रक्ष॑सा॒मप॑हत्यै॒ यं का॒मये॑त प्र॒मायु॑कः स्या॒दिति॑ जि॒ह्मं तस्या घा॑रयेत्प्रा॒णमे॒वास्मा᳚ज्जि॒ह्मं न॑यति ता॒जक्प्र मी॑यते॒ शिरो॒ वा ए॒तद्य॒ज्ञस्य॒ यदा॑घा॒र आ॒त्मा ध्रु॒वा-~(६७)

%2.5.11.8
ऽ\-ऽघा॒रमा॒घार्य॑ ध्रु॒वाꣳ सम॑नक्त्या॒त्मन्ने॒व य॒ज्ञस्य॒ शिरः॒ प्रति॑ दधा\-त्य॒\-ग्निर्दे॒वानां᳚ दू॒त आसी॒द्दैव्यो\-ऽसु॑राणां॒ तौ प्र॒जा\-प॑तिं प्र॒श्ञमै॑ता॒ꣳ॒ स प्र॒जा\-प॑तिर्ब्राह्म॒णम॑ब्रवीदे॒तद्वि ब्रू॒हीत्या श्रा॑व॒येती॒दं दे॑वाः शृणु॒तेति॒ वाव तद॑ब्रवीद॒ग्निर्दे॒वो होतेति॒ य ए॒व दे॒वानां॒ तम॑वृणीत॒ ततो॑ \mbox{दे॒वा-~(६८)}

%2.5.11.9
अभ॑व॒न्परा॑सुरा॒ यस्यै॒वं वि॒दुषः॑ प्रव॒रं प्र॑वृ॒णते॒ भव॑त्या॒त्मना॒ परा᳚स्य॒ भ्रातृ॑व्यो भवति॒ यद्ब्रा᳚ह्म॒णश्चाब्रा᳚ह्मणश्च प्र॒श्ञमे॒यातां᳚ ब्राह्म॒णायाधि॑ ब्रूया॒द्यद्ब्रा᳚ह्म॒णाया॒ध्याहा॒\-ऽऽ\-त्मने\-ऽध्या॑ह॒ यद्ब्रा᳚ह्म॒णं प॒राहा॒\-ऽऽ\-त्मानं॒ परा॑ह॒ तस्मा᳚द्ब्राह्म॒णो न प॒रोच्यः॑॥~(६९)

{\anuvakamend[{वा आ॑र॒ण्याꣴश्चाव॑ रु॒न्धे\-ऽथो॑ प॒शुभिः॒ सो᳚\-ऽब्रवीद्दक्षिणा॒र्ध्य॑न्त्रय॑ इव ध्रु॒वा दे॒वाश्च॑त्वारि॒ꣳ॒शच्च॑}]}%॥11॥

%2.5.12.1
आयु॑ष्ट आयु॒र्दा अ॑ग्न॒ आ प्या॑यस्व॒ सं ते\-ऽव॑ ते॒ हेड॒ उदु॑त्त॒मं प्र णो॑ दे॒व्या नो॑ दि॒वो\-ऽग्ना॑विष्णू॒ अग्ना॑विष्णू इ॒मं मे॑ वरुण॒ तत्त्वा॑ या॒म्युदु॒ त्यं चि॒त्रम्। अ॒पां नपा॒दा ह्यस्था॑दु॒पस्थं॑ जि॒ह्माना॑मू॒र्ध्वो वि॒द्युतं॒ वसा॑नः। तस्य॒ ज्येष्ठं॑ महि॒मानं॒ वह॑न्ती॒र्॒\mbox{}हिर॑ण्यवर्णाः॒ परि॑ यन्ति य॒ह्वीः। स-~(७०)

%2.5.12.2
म॒न्या यन्त्युप॑ यन्त्य॒न्याः स॑मा॒नमू॒र्वं न॒द्यः॑ पृणन्ति। तमू॒ शुचि॒ꣳ॒ शुच॑यो दीदि॒वाꣳस॑म॒पां नपा॑तं॒ परि॑ तस्थु॒रापः॑। तमस्मे॑रा युव॒तयो॒ युवा॑नं मर्मृ॒ज्यमा॑नाः॒ परि॑ य॒न्त्यापः॑। स शु॒क्रेण॒ शिक्व॑ना रे॒वद॒ग्निर्दी॒दाया॑नि॒ध्मो घृ॒तनि॑र्णिग॒फ्सु। इन्द्रा॒वरु॑णयोर॒हꣳ स॒म्राजो॒रव॒ आ वृ॑णे। ता नो॑ मृडात ई॒दृशे᳚। इन्द्रा॑वरुणा यु॒वम॑ध्व॒राय॑ नो~(७१)

%2.5.12.3
वि॒शे जना॑य॒ महि॒ शर्म॑ यच्छतम्। दी॒र्घप्र॑यज्यु॒मति॒ यो व॑नु॒ष्यति॑ व॒यं ज॑येम॒ पृत॑नासु दू॒ढ्यः॑। आ नो॑ मित्रावरुणा॒ प्र बा॒हवा᳚। त्वं नो॑ अग्ने॒ वरु॑णस्य वि॒द्वान् दे॒वस्य॒ हेडो\-ऽव॑ यासिसीष्ठाः। यजि॑ष्ठो॒ वह्नि॑तमः॒ शोशु॑चानो॒ विश्वा॒ द्वेषाꣳ॑सि॒ प्र मु॑मुग्ध्य॒स्मत्। स त्वं नो॑ अग्ने\-ऽव॒मो भ॑वो॒ती नेदि॑ष्ठो अ॒स्या उ॒षसो॒ व्यु॑ष्टौ। अव॑ यक्ष्व नो॒ वरु॑ण॒ꣳ॒~(७२)

%2.5.12.4
ररा॑णो वी॒हि मृ॑डी॒कꣳ सु॒हवो॑ न एधि। प्रप्रा॒यम॒ग्निर्भ॑र॒तस्य॑ शृण्वे॒ वि यथ्सूर्यो॒ न रोच॑ते बृ॒हद्भाः। अ॒भि यः पू॒रुं पृत॑नासु त॒स्थौ दी॒दाय॒ दैव्यो॒ अति॑थिः शि॒वो नः॑। प्र ते॑ यक्षि॒ प्र त॑ इयर्मि॒ मन्म॒ भुवो॒ यथा॒ वन्द्यो॑ नो॒ हवे॑षु। धन्व॑न्निव प्र॒पा अ॑सि॒ त्वम॑ग्न इय॒क्षवे॑ पू॒रवे᳚ प्रत्न राजन्न्।~(७३)

%2.5.12.5
वि पाज॑सा॒ वि ज्योति॑षा। स त्वम॑ग्ने॒ प्रती॑केन॒ प्रत्यो॑ष यातुधा॒न्यः॑। उ॒रु॒क्षये॑षु॒ दीद्य॑त्। तꣳ सु॒प्रती॑कꣳ सु॒दृश॒ꣴ॒ स्वञ्च॒मवि॑द्वाꣳसो वि॒दुष्ट॑रꣳ सपेम। स य॑क्ष॒द्विश्वा॑ व॒युना॑नि वि॒द्वान्प्र ह॒व्यम॒ग्निर॒मृते॑षु वोचत्। अ॒ꣳ॒हो॒मुचे॑ वि॒वेष॒ यन्मा॒ वि न॑ इ॒न्द्रेन्द्र॑ क्ष॒त्रमि॑न्द्रि॒याणि॑ शतक्र॒तो\-ऽनु॑ ते दायि॥~(७४)

{\anuvakamend[{य॒ह्वीः सम॑ध्व॒राय॑ नो॒ वरु॑णꣳ राज॒ꣴ॒ श्चतु॑श्चत्वारिꣳशच्च}]}%॥12॥

\prashnaend{वि॒श्वरू॑प॒स्त्वष्टेन्द्रं॑ वृ॒त्रम्ब्र॑ह्मवा॒दिनः॒ स त्वै नासो॑मयाज्ये॒ष वै दे॑वर॒थो दे॒वा वै नर्चि नाय॒ज्ञो\-ऽग्ने॑ म॒हान्त्रीन्निवी॑त॒मायु॑ष्टे॒ द्वाद॑श॥१२॥}{वि॒श्वरू॑पो॒ नैनꣳ॑ शीतरू॒राव॒द्य वसु॑ पूर्वे॒द्युर्वाजा॒ इत्यग्ने॑ म॒हान्निवी॑तम॒न्या यन्ति॒ चतुः॑सप्ततिः॥७४॥}{वि॒श्वरू॒पो\-ऽनु॑ ते दायि॥}%%२-५
{हरिः॑ ॐ}{॥कृष्ण-यजुर्वेदीय-तैत्तिरीय-संहितायां द्वितीयकाण्डे पञ्चमः प्रश्नः समाप्तः॥२-५॥}
%%% END PRASHNA
