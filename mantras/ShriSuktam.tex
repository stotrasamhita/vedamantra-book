% !TeX program = XeLaTeX
% !TeX root = ../vedamantrabook.tex

\chapt{श्रीसूक्तम्}

हिर॑ण्यवर्णां॒ हरि॑णीं सुव॒र्णर॑जत॒स्रजाम्।
च॒न्द्रां॒ हि॒रण्म॑यीं ल॒क्ष्मीं॒ जात॑वेदो म॒ आव॑ह॥१॥

तां म॒ आव॑ह॒ जात॑वेदो ल॒क्ष्मीमन॑पगा॒मिनी᳚म्।
यस्यां॒ हिर॑ण्यं वि॒न्देयं॒ गामश्वं॒ पुरु॑षान॒हम्॥२॥

अ॒श्व॒पू॒र्वां र॑थम॒ध्यां॒ ह॒स्तिना॑दप्र॒बोधि॑नीम्।
श्रियं॑ दे॒वीमुप॑ह्वये॒ श्रीर्मा॑दे॒वीर्जु॑षताम्॥३॥

कां॒ सो॒ऽ॒स्मि॒तां हिर॑ण्यप्राकारामा॒र्द्रां ज्वल॑न्तीं तृ॒प्तां त॒र्पय॑न्तीम्।
प॒द्मे॒ स्थि॒तां प॒द्मव॑र्णां॒ तामि॒होप॑ह्वये॒ श्रियम्॥४॥

च॒न्द्रां प्र॑भा॒सां य॒शसा॒ ज्वल॑न्तीं॒ श्रियं॑ लो॒के दे॒वजु॑ष्टामुदा॒राम्।
तां प॒द्मिनी॑मीं॒ शर॑णम॒हं प्रप॑द्येऽल॒क्ष्मीर्मे॑ नश्यतां॒ त्वां वृ॑णे॥५॥

आ॒दि॒त्यव॑र्णे॒ तप॒सोऽधि॑जा॒तो वन॒स्पति॒स्तव॑ वृ॒क्षोऽथ बि॒ल्वः।
तस्य॒ फला॑नि॒ तप॒सा नु॑दन्तु मा॒यान्त॑रा॒याश्च॑ बा॒ह्या अ॑ल॒क्ष्मीः॥६॥

उपै॑तु॒ मां दे॑वस॒खः की॒र्तिश्च॒ मणि॑ना स॒ह।
प्रा॒दु॒र्भू॒तोऽस्मि॑ राष्ट्रे॒ऽ॒स्मि॒न्॒ की॒र्तिमृद्धिं॑ ददा॒तु मे॥७॥

क्षुत्पि॑पा॒साम॑लां ज्ये॒ष्ठा॒मल॒क्ष्मीं ना॑शया॒म्यहम्।
अभू॑ति॒\-मस॑मृद्धिं॒ च सर्वां॒ निर्णु॑द मे॒ गृहात्॥८॥

ग॒न्ध॒द्वा॒रां दु॑राध॒र्॒‌षां॒ नि॒त्यपु॑ष्टां करी॒षिणी᳚म्।
ई॒श्वरीं᳚ सर्व॑भूता॒नां॒ तामि॒होप॑ह्वये॒ श्रियम्॥९॥

मन॑सः॒ काम॒माकू॑तिं वा॒चः स॒त्यम॑शीमहि।
प॒शू॒नां रू॒पमन्न॑स्य॒ मयि॒ श्रीः श्र॑यतां॒ यशः॑॥१०॥

क॒र्दमे॑न प्र॑जाभू॒ता॒ म॒यि॒ सम्भ॑व क॒र्दम।
श्रियं॑ वा॒सय॑ मे कु॒ले मा॒तरं॑ पद्ममा॒लिनीम्॥११॥

आपः॑ सृ॒जन्तु॑ स्निग्धा॒नि॒ चिक्ली॒त व॑स मे॒ गृहे।
नि च॑ दे॒वीं मा॒तरं॒ श्रियं॑ वा॒सय॑ मे कु॒ले॥१२॥

आ॒र्द्रां पु॒ष्करि॑णीं पु॒ष्टिं॒ सु॒व॒र्णां हे॑ममा॒लिनीम्।
सू॒र्यां हि॒रण्म॑यीं ल॒क्ष्मीं॒ जात॑वेदो म॒ आव॑ह॥१३॥

आ॒र्द्रां यः॒ करि॑णीं य॒ष्टिं॒ पि॒ङ्ग॒लां प॑द्ममा॒लिनीम्।
च॒न्द्रां हि॒रण्म॑यीं ल॒क्ष्मीं॒ जात॑वेदो म॒ आव॑ह॥१४॥

तां म॒ आव॑ह॒ जात॑वेदो ल॒क्ष्मीमन॑पगा॒मिनी᳚म्।
यस्यां॒ हि॑रण्यं॒ प्रभू॑तं॒ गावो॑ दा॒स्योऽश्वा᳚न् वि॒न्देयं॒ पुरु॑षान॒हम्॥१५॥

म॒हा॒दे॒व्यै च॑ वि॒द्महे॑ विष्णुप॒त्न्यै च॑ धीमहि।
तन्नो॑ लक्ष्मीः प्रचो॒दया᳚त्॥१६॥