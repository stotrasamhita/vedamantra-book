\chapt{काण्डम् ६}
\sect{प्रथमः प्रश्नः}\setcounter{anuvakam}{0}
\dnsub{तैत्तिरीयसंहितायां षष्ठमकाण्डे प्रथमः प्रश्नः}
%6.1.1.0
%6.1.1.1
प्रा॒चीन॑वꣳशं करोति देवमनु॒ष्या दिशो॒ व्य॑भजन्त॒ प्राचीं᳚ दे॒वा द॑क्षि॒णा पि॒तरः॑ प्र॒तीचीं᳚ मनु॒ष्या॑ उदी॑चीꣳ रु॒द्रा यत्प्रा॒चीन॑वꣳशं क॒रोति॑ देवलो॒कमे॒व तद्यज॑मान उ॒पाव॑र्तते॒ परि॑ श्रयत्य॒न्तर्\mbox{}हि॑तो॒ हि दे॑वलो॒को म॑नुष्यलो॒का\-न्नास्माल्लो॒काथ्स्वे॑तव्यमि॒वेत्या॑हुः॒ को हि तद्वेद॒ यद्य॒मुष्मिँ॑ल्लो॒के\-ऽस्ति॑ वा॒ न वेति॑ दि॒क्ष्वती॑का॒शान्क॑रोति~(१)

%6.1.1.2
उ॒भयो᳚र्लो॒कयो॑र॒भिजि॑त्यै केशश्म॒श्रु व॑पते न॒खानि॒ नि कृ॑न्तते मृ॒ता वा ए॒षा त्वग॑मे॒ध्या यत्के॑शश्म॒श्रु मृ॒तामे॒व त्वच॑ममे॒ध्याम॑प॒हत्य॑ य॒ज्ञियो॑ भू॒त्वा मेध॒मुपै॒त्यङ्गि॑रसः सुव॒र्गं लो॒कं यन्तो॒\-ऽफ्सु दी᳚क्षात॒पसी॒ प्रावे॑शयन्न॒फ्सु स्ना॑ति सा॒क्षादे॒व दी᳚क्षात॒पसी॒ अव॑ रुन्धे ती॒र्थे स्ना॑ति ती॒र्थे हि ते तां प्रावे॑शयन्ती॒र्थे स्ना॑ति~(२)

%6.1.1.3
ती॒र्थमे॒व स॑मा॒नानां᳚ भवत्य॒पो᳚\-ऽश्ञात्यन्तर॒त ए॒व मेध्यो॑ भवति॒ वास॑सा दीक्षयति सौ॒म्यं वै क्षौमं॑ दे॒वत॑या॒ सोम॑मे॒ष दे॒वता॒मुपै॑ति॒ यो दीक्ष॑ते॒ सोम॑स्य त॒नूर॑सि त॒नुवं॑ मे पा॒हीत्या॑ह॒ स्वामे॒व दे॒वता॒मुपै॒त्यथो॑ आ॒शिष॑मे॒वैतामा शा᳚स्ते॒\-ऽग्नेस्तू॑षा॒धानं॑ वा॒योर्वा॑त॒पानं॑ पितृ॒णां नी॒विरोष॑धीनां प्रघा॒तः~(३)

%6.1.1.4
आ॒दि॒त्यानां᳚ प्राचीनता॒नो विश्वे॑षां दे॒वाना॒मोतु॒र्नक्ष॑त्राणामतीका॒शास्तद्वा ए॒तथ्स॑र्वदेव॒त्यं॑ यद्वासो॒ यद्वास॑सा दी॒क्षय॑ति॒ सर्वा॑भिरे॒वैनं॑ दे॒वता॑भिर्दीक्षयति ब॒हिःप्रा॑णो॒ वै म॑नु॒ष्य॑स्तस्याश॑नं प्रा॒णो᳚\-ऽश्ञाति॒ सप्रा॑ण ए॒व दी᳚क्षत॒ आशि॑तो भवति॒ यावा॑ने॒वास्य॑ प्रा॒णस्तेन॑ स॒ह मेध॒मुपै॑ति घृ॒तं दे॒वानां॒ मस्तु॑ पितृ॒णान्निष्प॑क्वं मनु॒ष्या॑णा॒न्तद्वै~(४)

%6.1.1.5
ए॒तथ्स॑र्वदेव॒त्यं॑ यन्नव॑नीतं॒ यन्नव॑नीतेनाभ्य॒ङ्क्ते सर्वा॑ ए॒व दे॒वताः᳚ प्रीणाति॒ प्रच्यु॑तो॒ वा ए॒षो᳚\-ऽस्माल्लो॒कादग॑तो देवलो॒कं यो दी᳚क्षि॒तो᳚\-ऽन्त॒रेव॒ नव॑नीत॒न्तस्मा॒न्नव॑नीतेना॒भ्य॑ङ्क्ते\-ऽनुलो॒मं यजु॑षा॒ व्यावृ॑त्त्या॒ इन्द्रो॑ वृ॒त्रम॑ह॒न्तस्य॑ क॒नीनि॑का॒ परा॑पत॒त्तदाञ्ज॑नमभव॒द्यदा॒ङ्क्ते चक्षु॑रे॒व भ्रातृ॑व्यस्य वृङ्क्ते॒ दक्षि॑णं॒ पूर्व॒माङ्क्ते᳚~(५)

%6.1.1.6
स॒व्यꣳ हि पूर्वं॑ मनु॒ष्या॑ आ॒ञ्जते॒ न नि धा॑वते॒ नीव॒ हि म॑नु॒ष्या॑ धाव॑न्ते॒ पञ्च॒ कृत्व॒ आङ्क्ते॒ पञ्चा᳚क्षरा प॒ङ्क्तिः पाङ्क्तो॑ य॒ज्ञो य॒ज्ञमे॒वाव॑ रुन्धे॒ परि॑मित॒माङ्क्ते\-ऽप॑रिमित॒ꣳ॒ हि म॑नु॒ष्या॑ आ॒ञ्जते॒ सतू॑ल॒याङ्क्ते\-ऽप॑तूलया॒ हि म॑नु॒ष्या॑ आ॒ञ्जते॒ व्यावृ॑त्त्यै॒ यदप॑तूलयाञ्जी॒त वज्र॑ इव स्या॒थ्सतू॑ल॒याङ्क्ते॑ मित्र॒त्वाय॑~(६)

%6.1.1.7
इन्द्रो॑ वृ॒त्रम॑ह॒न्थ्सो\-ऽ पो\-ऽ भ्य॑म्रियत॒ तासां॒ यन्मेध्यं॑ य॒ज्ञिय॒ꣳ॒ सदे॑व॒मासी॒त्तदपोद॑क्राम॒त्ते द॒र्भा अ॑भव॒न्॒ यद्द॑र्भपुञ्जी॒लैः प॒वय॑ति॒ या ए॒व मेध्या॑ य॒ज्ञियाः॒ सदे॑वा॒ आप॒स्ताभि॑रे॒वैनं॑ पवयति॒ द्वा\-भ्यां᳚ पवयत्यहोरा॒त्राभ्या॑मे॒वैनं॑ पवयति त्रि॒भिः प॑वयति॒ त्रय॑ इ॒मे लो॒का ए॒भिरे॒वैनं॑ लो॒कैः प॑वयति प॒ञ्चभिः॑~(७)

%6.1.1.8
प॒व॒य॒ति॒ पञ्चा᳚क्षरा प॒ङ्क्तिः पाङ्क्तो॑ य॒ज्ञो य॒ज्ञायै॒वैनं॑ पवयति ष॒ड्भिः प॑वयति॒ षड्वा ऋ॒तव॑ ऋ॒तुभि॑रे॒वैनं॑ पवयति स॒प्तभिः॑ पवयति स॒प्त छन्दाꣳ॑सि॒ छन्दो॑भिरे॒वैनं॑ पवयति न॒वभिः॑ पवयति॒ नव॒ वै पुरु॑षे प्रा॒णाः सप्रा॑णमे॒वैनं॑ पव\-य॒त्येक॑विꣳशत्या पवयति॒ दश॒ हस्त्या॑ अ॒ङ्गुल॑यो॒ दश॒ पद्या॑ आ॒त्मैक॑वि॒ꣳ॒शो यावा॑ने॒व पुरु॑ष॒स्तमप॑रिवर्गम्~(८)

%6.1.1.9
प॒व॒य॒ति॒ चि॒त्पति॑स्त्वा पुना॒त्वित्या॑ह॒ मनो॒ वै चि॒त्पति॒र्मन॑सै॒वैनं॑ पवयति वा॒क्पति॑स्त्वा पुना॒त्वित्या॑ह वा॒चैवैनं॑ पवयति दे॒वस्त्वा॑ सवि॒ता पु॑ना॒त्वित्या॑ह सवि॒तृप्र॑सूत ए॒वैनं॑ पवयति॒ तस्य॑ ते पवित्रपते प॒वित्रे॑ण॒ यस्मै॒ कं पु॒ने तच्छ॑केय॒मित्या॑हा॒ऽऽशिष॑मे॒वैतामा शा᳚स्ते॥~(९)

%6.1.2.0
{\anuvakamend[{अ॒ती॒का॒शान्क॑रो॒त्यवे॑शयन्ती॒र्थे स्ना॑ति प्रघा॒तो म॑नु॒ष्या॑णा॒न्तद्वा आङ्क्ते॑ मित्र॒त्वाय॑ प॒ञ्चभि॒रप॑रिवर्गम॒ष्टाच॑त्वारिꣳशच्च}]}%~(१)

%6.1.2.1
याव॑न्तो॒ वै दे॒वा य॒ज्ञायापु॑नत॒ त ए॒वाभ॑व॒न्॒ य ए॒वं वि॒द्वान् य॒ज्ञाय॑ पुनी॒ते भव॑त्ये॒व ब॒हिः प॑वयि॒त्वान्तः प्र पा॑दयति मनुष्यलो॒क ए॒वैनं॑ पवयि॒त्वा पू॒तन्दे॑वलो॒कं प्र ण॑य॒त्यदी᳚क्षित॒ एक॒याहु॒त्येत्या॑हुः स्रु॒वेण॒ चत॑स्रो जुहोति दीक्षित॒त्वाय॑ स्रु॒चा प॑ञ्च॒मीं पञ्चा᳚क्षरा प॒ङ्क्तिः पाङ्क्तो॑ य॒ज्ञो य॒ज्ञमे॒वाव॑ रुन्ध॒ आकू᳚त्यै प्र॒युजे॒\-ऽग्नये᳚~(१०)

%6.1.2.2
स्वाहेत्या॒हाकू᳚त्या॒ हि पुरु॑षो य॒ज्ञम॒भि प्र॑यु॒ङ्क्ते यजे॒येति॑ मे॒धायै॒ मन॑से॒\-ऽग्नये॒ स्वाहेत्या॑ह मे॒धया॒ हि मन॑सा॒ पुरु॑षो य॒ज्ञम॑भि॒गच्छ॑ति॒ सर॑स्वत्यै पू॒ष्णे᳚\-ऽग्नये॒ स्वाहेत्या॑ह॒ वाग्वै सर॑स्वती पृथि॒वी पू॒षा वा॒चैव पृ॑थि॒व्या य॒ज्ञं प्र यु॑ङ्क्त॒ आपो॑ देवीर्बृहतीर्विश्वशम्भुव॒ इत्या॑ह॒ या वै वर्ष्या॒स्ताः~(११)

%6.1.2.3
आपो॑ दे॒वीर्बृ॑ह॒तीर्वि॒श्वश॑म्भुवो॒ यदे॒तद्यजु॒र्न ब्रू॒याद्दि॒व्या आपो\-ऽशा᳚न्ता इ॒मं लो॒कमा ग॑च्छेयु॒रापो॑ देवीर्बृहतीर्विश्वशम्भुव॒ इत्या॑हा॒स्मा ए॒वैना॑ लो॒काय॑ शमयति॒ तस्मा᳚च्छा॒न्ता इ॒मं लो॒कमा ग॑च्छन्ति॒ द्यावा॑पृथि॒वी इत्या॑ह॒ द्यावा॑पृथि॒व्योर्\mbox{}हि य॒ज्ञ उ॒र्व॑न्तरि॑क्ष॒मित्या॑हा॒न्तरि॑क्षे॒ हि य॒ज्ञो बृह॒स्पति॑र्नो ह॒विषा॑ वृधातु~(१२)

%6.1.2.4
इत्या॑ह॒ ब्रह्म॒ वै दे॒वानां॒ बृह॒स्पति॒र्ब्रह्म॑णै॒वास्मै॑ य॒ज्ञमव॑ रुन्धे॒ यद्ब्रू॒याद्वि॑धे॒रिति॑ यज्ञस्था॒णुमृ॑च्छेद्वृधा॒त्वित्या॑ह यज्ञस्था॒णुमे॒व परि॑ वृणक्ति प्र॒जा\-प॑तिर्य॒ज्ञम॑सृजत॒ सो᳚\-ऽस्माथ्सृ॒ष्टः परा॑ङै॒थ्स प्र यजु॒रव्ली॑ना॒त्प्र साम॒ तमृगुद॑यच्छ॒द्यदृगु॒दय॑च्छ॒त्तदौ᳚द्ग्रह॒णस्यौ᳚द्ग्रहण॒त्वमृ॒चा~(१३)

%6.1.2.5
जु॒हो॒ति॒ य॒ज्ञस्योद्य॑त्या अनु॒ष्टुप्छन्द॑सा॒मुद॑यच्छ॒दित्या॑हु॒स्तस्मा॑दनु॒ष्टुभा॑ जुहोति य॒ज्ञस्योद्य॑त्यै॒ द्वाद॑श वाथ्सब॒न्धान्युद॑यच्छ॒न्नित्या॑हु॒स्तस्मा᳚द्द्वाद॒शभि॑र्वाथ्सबन्ध॒विदो॑ दीक्षयन्ति॒ सा वा ए॒षर्ग॑नु॒ष्टुग्वाग॑नु॒ष्टुग्यदे॒तय॒र्चा दी॒क्षय॑ति वा॒चैवैन॒ꣳ॒ सर्व॑या दीक्षयति॒ विश्वे॑ दे॒वस्य॑ ने॒तुरित्या॑ह सावि॒त्र्ये॑तेन॒ मर्तो॑ वृणीत स॒ख्यम्~(१४)

%6.1.2.6
इत्या॑ह पितृदेव॒त्यै॑तेन॒ विश्वे॑ रा॒य इ॑षुध्य॒सीत्या॑ह वैश्वदे॒व्ये॑तेन॑ द्यु॒म्नं वृ॑णीत पु॒ष्यस॒ इत्या॑ह पौ॒ष्ण्ये॑तेन॒ सा वा ए॒षर्ख्स॑र्वदेव॒त्या॑ यदे॒तय॒र्चा दी॒क्षय॑ति॒ सर्वा॑भिरे॒वैनं॑ दे॒वता॑भिर्दीक्षयति स॒प्ताक्ष॑रं प्रथ॒मं प॒दम॒ष्टाक्ष॑राणि॒ त्रीणि॒ यानि॒ त्रीणि॒ तान्य॒ष्टावुप॑ यन्ति॒ यानि॑ च॒त्वारि॒ तान्य॒ष्टौ यद॒ष्टाक्ष॑रा॒ तेन॑~(१५)

%6.1.2.7
गा॒य॒त्री यदेका॑\-दशाक्षरा॒ तेन॑ त्रि॒ष्टुग्यद्द्वाद॑शाक्षरा॒ तेन॒ जग॑ती॒ सा वा ए॒षर्ख्सर्वा॑णि॒ छन्दाꣳ॑सि॒ यदे॒तय॒र्चा दी॒क्षय॑ति॒ सर्वे॑भिरे॒वैनं॒ छन्दो॑भिर्दीक्षयति स॒प्ताक्ष॑रं प्रथ॒मं प॒दꣳ स॒प्तप॑दा॒ शक्व॑री प॒शवः॒ शक्व॑री प॒शूने॒वाव॑ रुन्ध॒ एक॑स्माद॒क्षरा॒दना᳚प्तं प्रथ॒मं प॒दन्तस्मा॒द्यद्वा॒चो\-ऽना᳚प्त॒न्तन्म॑नु॒ष्या॑ उप॑ जीवन्ति पू॒र्णया॑ जुहोति पू॒र्ण इ॑व॒ हि प्र॒जा\-प॑तिः प्र॒जा\-प॑ते॒राप्त्यै॒ न्यू॑नया जुहोति॒ न्यू॑ना॒द्धि प्र॒जा\-प॑तिः प्र॒जा असृ॑जत प्र॒जाना॒ꣳ॒ सृष्ट्यै᳚॥~(१६)

%6.1.3.0
{\anuvakamend[{अ॒ग्नये॒ ता वृ॑धात्वृ॒चा स॒ख्यन्तेन॑ जुहोति॒ पञ्च॑दश च}]}%~(२)

%6.1.3.1
ऋ॒ख्सा॒मे वै दे॒वेभ्यो॑ य॒ज्ञायाति॑ष्ठमाने॒ कृष्णो॑ रू॒पं कृ॒त्वाप॒क्रम्या॑तिष्ठता॒न्ते॑\-ऽमन्यन्त॒ यं वा इ॒मे उ॑पाव॒र्थ्स्यतः॒ स इ॒दं भ॑विष्य॒तीति॒ ते उपा॑मन्त्रयन्त॒ ते अ॑होरा॒त्रयो᳚र्महि॒मान॑मपनि॒धाय॑ दे॒वानु॒पाव॑र्तेतामे॒ष वा ऋ॒चो वर्णो॒ यच्छु॒क्लं कृ॑ष्णाजि॒नस्यै॒ष साम्नो॒ यत्कृ॒ष्णमृ॑ख्सा॒मयोः॒ शिल्पे᳚ स्थ॒ इत्या॑हर्ख्सा॒मे ए॒वाव॑ रुन्ध ए॒षः~(१७)

%6.1.3.2
वा अह्नो॒ वर्णो॒ यच्छु॒क्लं कृ॑ष्णाजि॒नस्यै॒ष रात्रि॑या॒ यत्कृ॒ष्णं यदे॒वैन॑यो॒स्तत्र॒ न्य॑क्तं॒ तदे॒वाव॑ रुन्धे कृष्णाजि॒नेन॑ दीक्षयति॒ ब्रह्म॑णो॒ वा ए॒तद्रू॒पं यत्कृ॑ष्णाजि॒नं ब्रह्म॑णै॒वैनं॑ दीक्षयती॒मान्धिय॒ꣳ॒ शिक्ष॑माणस्य दे॒वेत्या॑ह यथाय॒जुरे॒वैतद्गर्भो॒ वा ए॒ष यद्दी᳚क्षि॒त उल्बं॒ वासः॒ प्रोर्णु॑ते॒ तस्मा᳚त्~(१८)

%6.1.3.3
गर्भाः॒ प्रावृ॑ता जायन्ते॒ न पु॒रा सोम॑स्य क्र॒यादपो᳚र्ण्वीत॒ यत्पु॒रा सोम॑स्य क्र॒याद॑पोर्ण्वी॒त गर्भाः᳚ प्र॒जानां᳚ परा॒पातु॑काः स्युः क्री॒ते सोमे\-ऽपो᳚र्णुते॒ जाय॑त ए॒व तदथो॒ यथा॒ वसी॑याꣳसं प्रत्यपोर्णु॒ते ता॒दृगे॒व तदङ्गि॑रसः सुव॒र्गं लो॒कं यन्त॒ ऊर्जं॒ व्य॑भजन्त॒ ततो॒ यद॒त्यशि॑ष्यत॒ ते श॒रा अ॑भव॒न्नूर्ग्वै श॒रा यच्छ॑र॒मयी᳚~(१९)

%6.1.3.4
मेख॑ला॒ भव॒त्यूर्ज॑मे॒वाव॑ रुन्धे मध्य॒तः सन्न॑ह्यति मध्य॒त ए॒वास्मा॒ ऊर्जं॑ दधाति॒ तस्मा᳚न्मध्य॒त ऊ॒र्जा भु॑ञ्जत ऊ॒र्ध्वं वै पुरु॑षस्य॒ नाभ्यै॒ मेध्य॑मवा॒चीन॑ममे॒ध्यं यन्म॑ध्य॒तः सं॒नह्य॑ति॒ मेध्यं॑ चै॒वास्या॑मे॒ध्यं च॒ व्याव॑र्तय॒तीन्द्रो॑ वृ॒त्राय॒ वज्रं॒ प्राह॑र॒थ्स त्रे॒धा व्य॑भव॒थ्स्फ्यस्तृती॑य॒ꣳ॒ रथ॒स्तृती॑यं॒ यूप॒स्तृती॑यम्~(२०)

%6.1.3.5
ये᳚\-ऽन्तःश॒रा अशी᳚र्यन्त॒ ते श॒रा अ॑भव॒न्तच्छ॒राणाꣳ॑ शर॒त्वं वज्रो॒ वै श॒राः क्षुत्खलु॒ वै म॑नु॒ष्य॑स्य॒ भ्रातृ॑व्यो॒ यच्छ॑र॒मयी॒ मेख॑ला॒ भव॑ति॒ वज्रे॑णै॒व सा॒क्षात्क्षुध॒म्भ्रातृ॑व्यं मध्य॒तो\-ऽप॑ हते त्रि॒वृद्भ॑वति त्रि॒वृद्वै प्रा॒णस्त्रि॒वृत॑मे॒व प्रा॒णं म॑ध्य॒तो यज॑माने दधाति पृ॒थ्वी भ॑वति॒ रज्जू॑ना॒व्व्याँवृ॑त्त्यै॒ मेख॑लया॒ यज॑मानन्दीक्षयति॒ योक्त्रे॑ण॒ पत्नीं᳚ मिथुन॒त्वाय॑~(२१)

%6.1.3.6
य॒ज्ञो दक्षि॑णाम॒भ्य॑ध्याय॒त्ताꣳ सम॑भव॒त्तदिन्द्रो॑\-ऽचाय॒थ्सो॑\-ऽमन्यत॒ यो वा इ॒तो ज॑नि॒ष्यते॒ स इ॒दं भ॑विष्य॒तीति॒ तां प्रावि॑श॒त्तस्या॒ इन्द्र॑ ए॒वाजा॑यत॒ सो॑\-ऽमन्यत॒ यो वै मदि॒तो\-ऽप॑रो जनि॒ष्यते॒ स इ॒दं भ॑विष्य॒तीति॒ तस्या॑ अनु॒मृश्य॒ योनि॒माच्छि॑न॒थ्सा सू॒तव॑शाभव॒त्तथ्सू॒तव॑शायै॒ जन्म॑~(२२)

%6.1.3.7
ताꣳ हस्ते॒ न्य॑वेष्टयत॒ तां मृ॒गेषु॒ न्य॑दधा॒थ्सा कृ॑ष्णविषा॒णाभ॑व॒दिन्द्र॑स्य॒ योनि॑रसि॒ मा मा॑ हिꣳसी॒रिति॑ कृष्णविषा॒णां प्र य॑च्छति॒ सयो॑निमे॒व य॒ज्ञं क॑रोति॒ सयो॑नि॒न्दक्षि॑णा॒ꣳ॒ सयो॑नि॒मिन्द्रꣳ॑ सयोनि॒त्वाय॑ कृ॒ष्यै त्वा॑ सुस॒स्याया॒ इत्या॑ह॒ तस्मा॑दकृष्टप॒च्या ओष॑धयः पच्यन्ते सुपिप्प॒लाभ्य॒स्त्वौष॑धीभ्य॒ इत्या॑ह॒ तस्मा॒दोष॑धयः॒ फलं॑ गृह्णन्ति॒ यद्धस्ते॑न~(२३)

%6.1.3.8
क॒ण्डू॒येत॑ पामनं॒ भावु॑काः प्र॒जाः स्यु॒र्यथ्स्मये॑त नग्नं॒ भावु॑काः कृष्णविषा॒णया॑ कण्डूयते\-ऽपि॒गृह्य॑ स्मयते प्र॒जानां᳚ गोपी॒थाय॒ न पु॒रा दक्षि॑णाभ्यो॒ नेतोः᳚ कृष्णविषा॒णामव॑ चृते॒द्यत्पु॒रा दक्षि॑णाभ्यो॒ नेतोः᳚ कृष्णविषा॒णामव॑चृ॒तेद्योनिः॑ प्र॒जानां᳚ परा॒पातु॑का स्यान्नी॒तासु॒ दक्षि॑णासु॒ चात्वा॑ले कृष्णविषा॒णां प्रास्य॑ति॒ योनि॒र्वै य॒ज्ञस्य॒ चात्वा॑लं॒ योनिः॑ कृष्णविषा॒णा योना॑वे॒व योनि॑न्दधाति य॒ज्ञस्य॑ सयोनि॒त्वाय॑॥~(२४)

%6.1.4.0
{\anuvakamend[{रु॒न्ध॒ ए॒ष तस्मा᳚च्छर॒मयी॒ यूप॒स्तृती॑यं मिथुन॒त्वाय॒ जन्म॒ हस्ते॑ना॒ष्टाच॑त्वारिꣳशच्च}]}%~(३)

%6.1.4.1
वाग्वै दे॒वेभ्यो\-ऽपा᳚क्रामद्यज्ञा॒याति॑ष्ठमाना॒ सा वन॒स्पती॒न्प्रावि॑श॒थ्सैषा वाग्वन॒स्पति॑षु वदति॒ या दु॑न्दु॒भौ या तूण॑वे॒ या वीणा॑यां॒ यद्दी᳚क्षितद॒ण्डं प्र॒यच्छ॑ति॒ वाच॑मे॒वाव॑ रुन्ध॒ औदु॑म्बरो भव॒त्यूर्ग्वा उ॑दु॒म्बर॒ ऊर्ज॑मे॒वाव॑ रुन्धे॒ मुखे॑न॒ सम्मि॑तो भवति मुख॒त ए॒वास्मा॒ ऊर्जं॑ दधाति॒ तस्मा᳚न्मुख॒त ऊ॒र्जा भु॑ञ्जते~(२५)

%6.1.4.2
क्री॒ते सोमे॑ मैत्रावरु॒णाय॑ द॒ण्डं प्र य॑च्छति मैत्रावरु॒णो हि पु॒रस्ता॑दृ॒त्विग्भ्यो॒ वाचं॑ वि॒भज॑ति॒ तामृ॒त्विजो॒ यज॑माने॒ प्रति॑\-ष्ठापयन्ति॒ स्वाहा॑ य॒ज्ञं मन॒सेत्या॑ह॒ मन॑सा॒ हि पुरु॑षो य॒ज्ञम॑भि॒गच्छ॑ति॒ स्वाहा॒ द्यावा॑पृथि॒वीभ्या॒मित्या॑ह॒ द्यावा॑पृथि॒व्योर्\mbox{}हि य॒ज्ञः स्वाहो॒रोर॒न्तरि॑क्षा॒दित्या॑हा॒न्तरि॑क्षे॒ हि य॒ज्ञः स्वाहा॑ य॒ज्ञं वाता॒दार॑भ॒ इत्या॑हा॒यम्~(२६)

%6.1.4.3
वाव यः पव॑ते॒ स य॒ज्ञस्तमे॒व सा॒क्षादा र॑भते मु॒ष्टी क॑रोति॒ वाचं॑ यच्छति य॒ज्ञस्य॒ धृत्या॒ अदी᳚क्षिष्टा॒यं ब्रा᳚ह्म॒ण इति॒ त्रिरु॑पा॒ꣳ॒श्वा॑ह दे॒वेभ्य॑ ए॒वैनं॒ प्राऽऽह॒ त्रिरु॒च्चैरु॒भये᳚भ्य ए॒वैनं॑ देवमनु॒ष्येभ्यः॒ प्राऽऽह॒ न पु॒रा नक्ष॑त्रेभ्यो॒ वाचं॒ वि सृ॑जे॒द्यत्पु॒रा नक्ष॑त्रेभ्यो॒ वाचं॑ विसृ॒जेद्य॒ज्ञं विच्छि॑न्द्यात्~(२७)

%6.1.4.4
उदि॑तेषु॒ नक्ष॑त्रेषु व्र॒तं कृ॑णु॒तेति॒ वाचं॒ वि सृ॑जति य॒ज्ञव्र॑तो॒ वै दी᳚क्षि॒तो य॒ज्ञमे॒वाभि वाचं॒ वि सृ॑जति॒ यदि॑ विसृ॒जेद्वै᳚ष्ण॒वीमृच॒मनु॑ ब्रूयाद्य॒ज्ञो वै विष्णु॑र्य॒ज्ञेनै॒व य॒ज्ञꣳ सं त॑नोति॒ दैवी॒न्धियं॑ मनामह॒ इत्या॑ह य॒ज्ञमे॒व तन्म्र॑दयति सुपा॒रा नो॑ अस॒द्वश॒ इत्या॑ह॒ व्यु॑ष्टिमे॒वाव॑ रुन्धे~(२८)

%6.1.4.5
ब्र॒ह्म॒वा॒दिनो॑ वदन्ति होत॒व्यं॑ दीक्षि॒तस्य॑ गृ॒हा(३)इ न हो॑त॒व्या(३)मिति॑ ह॒विर्वै दी᳚क्षि॒तो यज्जु॑हु॒याद्यज॑मानस्याव॒दाय॑ जुहुया॒द्यन्न जु॑हु॒याद्य॑ज्ञप॒रुर॒न्तरि॑या॒द्ये दे॒वा मनो॑जाता मनो॒युज॒ इत्या॑ह प्रा॒णा वै दे॒वा मनो॑जाता मनो॒युज॒स्तेष्वे॒व प॒रोक्षं॑ जुहोति॒ तन्नेव॑ हु॒तं नेवाहु॑तꣴ स्व॒पन्तं॒ वै दी᳚क्षि॒तꣳ रक्षाꣳ॑सि जिघाꣳसन्त्य॒ग्निः~(२९)

%6.1.4.6
खलु॒ वै र॑क्षो॒हाग्ने॒ त्वꣳ सु जा॑गृहि व॒यꣳ सु म॑न्दिषीम॒हीत्या॑हा॒ग्निमे॒वाधि॒पां कृ॒त्वा स्व॑पिति॒ रक्ष॑सा॒मप॑हत्या अव्र॒त्यमि॑व॒ वा ए॒ष क॑रोति॒ यो दी᳚क्षि॒तः स्वपि॑ति॒ त्वम॑ग्ने व्रत॒पा अ॒सीत्या॑हा॒ग्निर्वै दे॒वानां᳚ व्र॒तप॑तिः॒ स ए॒वैनं॑ व्र॒तमाल॑म्भयति दे॒व आ मर्त्ये॒ष्वेत्या॑ह दे॒वः~(३०)

%6.1.4.7
ह्ये॑ष सन्मर्त्ये॑षु॒ त्वं य॒ज्ञेष्वीड्य॒ इत्या॑है॒तꣳ हि य॒ज्ञेष्वीड॒ते\-ऽप॒ वै दी᳚क्षि॒ताथ्सु॑षु॒पुष॑ इन्द्रि॒यं दे॒वताः᳚ क्रामन्ति॒ विश्वे॑ दे॒वा अ॒भि मामाव॑वृत्र॒न्नित्या॑हेन्द्रि॒येणै॒वैनं॑ दे॒वता॑भिः॒ सं न॑यति॒ यदे॒तद्यजु॒र्न ब्रू॒याद्याव॑त ए॒व प॒शून॒भि दीक्षे॑त॒ ताव॑न्तो\-ऽस्य प॒शवः॑ स्यू॒ रास्वेय॑त्~(३१)

%6.1.4.8
सो॒मा भूयो॑ भ॒रेत्या॒हाप॑रिमिताने॒व प॒शूनव॑ रुन्धे च॒न्द्रम॑सि॒ मम॒ भोगा॑य भ॒वेत्या॑ह यथादेव॒तमे॒वैनाः॒ प्रति॑ गृह्णाति वा॒यवे᳚ त्वा॒ वरु॑णाय॒ त्वेति॒ यदे॒वमे॒ता नानु॑दि॒शेदय॑थादेवतं॒ दक्षि॑णा गमये॒दा दे॒वता᳚भ्यो वृश्च्येत॒ यदे॒वमे॒ता अ॑नुदि॒शति॑ यथादेव॒तमे॒व दक्षि॑णा गमयति॒ न दे॒वता᳚भ्य॒ आ~(३२)

%6.1.4.9
वृ॒श्च्य॒ते॒ देवी॑रापो अपां नपा॒दित्या॑ह॒ यद्वो॒ मेध्यं॑ य॒ज्ञिय॒ꣳ॒ सदे॑वं॒ तद्वो॒ माव॑ क्रमिष॒मिति॒ वावैतदा॒हाच्छि॑न्नं॒ तन्तुं॑ पृथि॒व्या अनु॑ गेष॒मित्या॑ह॒ सेतु॑मे॒व कृ॒त्वात्ये॑ति॥~(३३)

%6.1.5.0
{\anuvakamend[{भु॒ञ्ज॒ते॒\-ऽयञ्छि॑न्द्याद्रुन्धे॒\-ऽग्निरा॑ह दे॒व इय॑द्दे॒वता᳚भ्य॒ आ त्रय॑स्त्रिꣳशच्च}]}%~(४)

%6.1.5.1
दे॒वा वै दे॑व॒यज॑नमध्यव॒साय॒ दिशो॒ न प्राजा॑न॒न्ते\-ऽ न्यो᳚न्यमुपा॑धाव॒न्त्वया॒ प्र जा॑नाम॒ त्वयेति॒ ते\-ऽदि॑त्या॒ꣳ॒ सम॑ध्रियन्त॒ त्वया॒ प्र जा॑ना॒मेति॒ साब्र॑वी॒द्वरं॑ वृणै॒ मत्प्रा॑यणा ए॒व वो॑ य॒ज्ञा मदु॑दयना अस॒न्निति॒ तस्मा॑दादि॒त्यः प्रा॑य॒णीयो॑ य॒ज्ञाना॑मादि॒त्य उ॑दय॒नीयः॒ पञ्च॑ दे॒वता॑ यजति॒ पञ्च॒ दिशो॑ दि॒शां प्रज्ञा᳚त्यै~(३४)

%6.1.5.2
अथो॒ पञ्चा᳚क्षरा प॒ङ्क्तिः पाङ्क्तो॑ य॒ज्ञो य॒ज्ञमे॒वाव॑ रुन्धे॒ पथ्याꣴ॑ स्व॒स्तिम॑यज॒न्प्राची॑मे॒व तया॒ दिशं॒ प्राजा॑नन्न॒ग्निना॑ दक्षि॒णा सोमे॑न प्र॒तीचीꣳ॑ सवि॒त्रोदी॑ची॒मदि॑त्यो॒र्ध्वां पथ्याꣴ॑ स्व॒स्तिं य॑जति॒ प्राची॑मे॒व तया॒ दिशं॒ प्र जा॑नाति॒ पथ्याꣴ॑ स्व॒स्तिमि॒ष्ट्वाग्नी\-षोमौ॑ यजति॒ चक्षु॑षी॒ वा ए॒ते य॒ज्ञस्य॒ यद॒ग्नी\-षोमौ॒ ताभ्या॑मे॒वानु॑ पश्यति~(३५)

%6.1.5.3
अ॒ग्नी\-षोमा॑वि॒ष्ट्वा स॑वि॒तारं॑ यजति सवि॒तृप्र॑सूत ए॒वानु॑ पश्यति सवि॒तार॑मि॒ष्ट्वादि॑तिं यजती॒यं वा अदि॑तिर॒स्यामे॒व प्र॑ति॒ष्ठायानु॑ पश्य॒त्यदि॑तिमि॒ष्ट्वा मा॑रु॒तीमृच॒मन्वा॑ह म॒रुतो॒ वै दे॒वानां॒ विशो॑ देववि॒शं खलु॒ वै कल्प॑मानं मनुष्यवि॒श\-मनु॑ कल्पते॒ यन्मा॑रु॒तीमृच॑म॒न्वाह॑ वि॒शां कॢप्त्यै᳚ ब्रह्मवा॒दिनो॑ वदन्ति प्रया॒जव॑दननूया॒जं प्रा॑य॒णीयं॑ का॒र्य॑मनूया॒जव॑त्~(३६)

%6.1.5.4
अ॒प्र॒या॒जमु॑दय॒नीय॒मिती॒मे वै प्र॑या॒जा अ॒मी अ॑नूया॒जाः सैव सा य॒ज्ञस्य॒ सन्त॑ति॒स्तत्तथा॒ न का॒र्य॑मा॒त्मा वै प्र॑या॒जाः प्र॒जाऽनू॑या॒जा यत्प्र॑या॒जान॑न्तरि॒यादा॒त्मान॑म॒न्तरि॑या॒द्यद॑नूया॒जान॑न्तरि॒यात्प्र॒जाम॒न्तरि॑या॒द्यतः॒ खलु॒ वै य॒ज्ञस्य॒ वित॑तस्य॒ न क्रि॒यते॒ तदनु॑ य॒ज्ञः परा॑ भवति य॒ज्ञं प॑रा॒भव॑न्तं॒ यज॑मा॒नो\-ऽनु॑~(३७)

%6.1.5.5
परा॑ भवति प्रया॒जव॑दे॒वानू॑या॒जव॑त्प्राय॒णीयं॑ का॒र्यं॑ प्रया॒जव॑दनूया॒जव॑दुदय॒नीयं॒ नात्मान॑मन्त॒रेति॒ न प्र॒जां न य॒ज्ञः प॑रा॒भव॑ति॒ न यज॑मानः प्राय॒णीय॑स्य निष्का॒स उ॑दय॒नीय॑म॒भि निर्व॑पति॒ सैव सा य॒ज्ञस्य॒ सन्त॑ति॒र्याः प्रा॑य॒णीय॑स्य या॒ज्या॑ यत्ता उ॑दय॒नीय॑स्य या॒ज्याः᳚ कु॒र्यात्परा॑ङ॒मुं लो॒कमा रो॑हेत्प्र॒मायु॑कः स्या॒द्याः प्रा॑य॒णीय॑स्य पुरोनुवा॒क्या᳚स्ता उ॑दय॒नीय॑स्य या॒ज्याः᳚ करोत्य॒स्मिन्ने॒व लो॒के प्रति॑ तिष्ठति॥~(३८)

%6.1.6.0
{\anuvakamend[{प्रज्ञा᳚त्यै पश्यत्यनूया॒जव॒द्यज॑मा॒नो\-ऽनु॑ पुरोनुवा॒क्या᳚स्ता अ॒ष्टौ च॑}]}%~(५)

%6.1.6.1
क॒द्रूश्च॒ वै सु॑प॒र्णी चा᳚त्मरू॒पयो॑रस्पर्धेता॒ꣳ॒ सा क॒द्रूः सु॑प॒र्णीम॑जय॒थ्साब्र॑वीत्तृ॒तीय॑स्यामि॒तो दि॒वि सोम॒स्तमा ह॑र॒ तेना॒ऽऽत्मानं॒ निष्क्री॑णी॒ष्वेती॒यं वै क॒द्रूर॒सौ सु॑प॒र्णी छन्दाꣳ॑सि सौपर्णे॒याः साब्र॑वीद॒स्मै वै पि॒तरौ॑ पु॒त्रान्बि॑भृतस्तृ॒तीय॑स्यामि॒तो दि॒वि सोम॒स्तमा ह॑र॒ तेना॒ऽऽत्मानं॒ निष्क्री॑णीष्व~(३९)

%6.1.6.2
इति॑ मा क॒द्रूर॑वोच॒दिति॒ जग॒त्युद॑पत॒च्चतु॑र्दशाक्षरा स॒ती साप्रा᳚प्य॒ न्य॑वर्तत॒ तस्यै॒ द्वे अ॒क्षरे॑ अमीयेता॒ꣳ॒ सा प॒शुभि॑श्च दी॒क्षया॒ चाग॑च्छ॒त्तस्मा॒ज्जग॑ती॒ छन्द॑सां पश॒व्य॑तमा॒ तस्मा᳚त्पशु॒मन्तं॑ दी॒क्षोप॑ नमति त्रि॒ष्टुगुद॑पत॒त्त्रयो॑दशाक्षरा स॒ती साप्रा᳚प्य॒ न्य॑वर्तत॒ तस्यै॒ द्वे अ॒क्षरे॑ अमीयेता॒ꣳ॒ सा दक्षि॑णाभिश्च~(४०)

%6.1.6.3
तप॑सा॒ चाग॑च्छ॒त्तस्मा᳚त्त्रि॒ष्टुभो॑ लो॒के माध्य॑न्दिने॒ सव॑ने॒ दक्षि॑णा नीयन्त ए॒तत्खलु॒ वाव तप॒ इत्या॑हु॒र्यः स्वं ददा॒तीति॑ गाय॒त्र्युद॑पत॒च्चतु॑रक्षरा स॒त्य॑जया॒ ज्योति॑षा॒ तम॑स्या अ॒जाभ्य॑रुन्ध॒ तद॒जाया॑ अज॒त्वꣳ सा सोमं॒ चाह॑रच्च॒त्वारि॑ चा॒क्षरा॑णि साष्टाक्ष॑रा॒ सम॑पद्यत ब्रह्मवा॒दिनो॑ वदन्ति~(४१)

%6.1.6.4
कस्मा᳚थ्स॒त्याद्गा॑य॒त्री कनि॑ष्ठा॒ छन्द॑साꣳ स॒ती य॑ज्ञमु॒खं परी॑या॒येति॒ यदे॒वादः सोम॒माह॑र॒त्तस्मा᳚द्यज्ञमु॒खं पर्यै॒त् तस्मा᳚त्तेज॒स्विनी॑तमा प॒द्भ्यां द्वे सव॑ने स॒मगृ॑ह्णा॒न्मुखे॒नैकं॒ यन्मुखे॑न स॒मगृ॑ह्णा॒त्तद॑धय॒त्तस्मा॒द्द्वे सव॑ने शु॒क्रव॑ती प्रातःसव॒नं च॒ माध्य॑न्दिनं च॒ तस्मा᳚त्तृतीयसव॒न ऋ॑जी॒षम॒भि षु॑ण्वन्ति धी॒तमि॑व॒ हि मन्य॑न्ते~(४२)

%6.1.6.5
आ॒शिर॒मव॑ नयति सशुक्र॒त्वायाथो॒ सम्भ॑रत्ये॒वैन॒त्तꣳ सोम॑माह्रि॒यमा॑णं गन्ध॒र्वो वि॒श्वाव॑सुः॒ पर्य॑मुष्णा॒थ्स ति॒स्रो रात्रीः॒ परि॑मुषितो\-ऽवस॒त्तस्मा᳚त्ति॒स्रो रात्रीः᳚ क्री॒तः सोमो॑ वसति॒ ते दे॒वा अ॑ब्रुव॒न्थ्स्त्रीका॑मा॒ वै ग॑न्ध॒र्वाः स्त्रि॒या निष्क्री॑णा॒मेति॒ ते वाच॒ꣴ॒ स्त्रिय॒मेक॑हायनीं कृ॒त्वा तया॒ निर॑क्रीण॒न्थ्सा रो॒हिद्रू॒पं कृ॒त्वा ग॑न्ध॒र्वेभ्यः॑~(४३)

%6.1.6.6
अ॒प॒क्रम्या॑तिष्ठ॒त्तद्रो॒हितो॒ जन्म॒ ते दे॒वा अ॑ब्रुव॒न्नप॑ यु॒ष्मदक्र॑मी॒न्नास्मानु॒पाव॑र्तते॒ वि ह्व॑यामहा॒ इति॒ ब्रह्म॑ गन्ध॒र्वा अव॑द॒न्नगा॑यं दे॒वाः सा दे॒वान्गाय॑त उ॒पाव॑र्तत॒ तस्मा॒द्गाय॑न्त॒ꣴ॒ स्त्रियः॑ कामयन्ते॒ कामु॑का एन॒ꣴ॒ स्त्रियो॑ भवन्ति॒ य ए॒वं वेदाथो॒ य ए॒वं वि॒द्वानपि॒ जन्ये॑षु॒ भव॑ति॒ तेभ्य॑ ए॒व द॑दत्यु॒त यद्ब॒हुत॑याः~(४४)

%6.1.6.7
भव॒न्त्येक॑हायन्या क्रीणाति वा॒चैवैन॒ꣳ॒ सर्व॑या क्रीणाति॒ तस्मा॒देक॑हायना मनु॒ष्या॑ वाचं॑ वद॒न्त्यकू॑ट॒या\-ऽक॑र्ण॒या\-ऽ का॑ण॒या\-ऽश्लो॑ण॒या\-ऽस॑प्तशफया क्रीणाति॒ सर्व॑यै॒वैनं॑ क्रीणाति॒ यच्छ्वे॒तया᳚ क्रीणी॒याद्दु॒श्चर्मा॒ यज॑मानः स्या॒द्यत्कृ॒ष्णया॑नु॒स्तर॑णी स्यात्प्र॒मायु॑को॒ यज॑मानः स्या॒द्यद्द्वि॑रू॒पया॒ वात्र॑घ्नी स्या॒थ्स वा॒न्यं जि॑नी॒यात्तं वा॒न्यो जि॑नीयादरु॒णया॑ पिङ्गा॒क्ष्या क्री॑णात्ये॒तद्वै सोम॑स्य रू॒पꣴ स्वयै॒वैनं॑ दे॒वत॑या क्रीणाति॥~(४५)

%6.1.7.0
{\anuvakamend[{निष्क्री॑णीष्व॒ दक्षि॑णाभिश्च वदन्ति॒ मन्य॑न्ते गन्ध॒र्वेभ्यो॑ ब॒हुत॑याः पिङ्गा॒क्ष्या दश॑ च}]}%~(६)

%6.1.7.1
तद्धिर॑ण्यमभव॒त्तस्मा॑द॒द्भ्यो हिर॑ण्यं पुनन्ति ब्रह्मवा॒दिनो॑ वदन्ति॒ कस्मा᳚थ्स॒त्याद॑न॒स्थिके॑न प्र॒जाः प्र॒वीय॑न्ते\-ऽ\-स्थ॒न्वती᳚र्जायन्त॒ इति॒ यद्धिर॑ण्यं घृ॒ते॑\-ऽव॒धाय॑ जु॒होति॒ तस्मा॑दन॒स्थिके॑न प्र॒जाः प्र वी॑यन्ते\-ऽस्थ॒न्वती᳚र्जायन्त ए॒तद्वा अ॒ग्नेः प्रि॒यं धाम॒ यद् घृ॒तं तेजो॒ हिर॑ण्यमि॒यं ते॑ शुक्र त॒नूरि॒दं वर्च॒ इत्या॑ह॒ सते॑जसमे॒वैन॒ꣳ॒ सत॑नुम्~(४६)

%6.1.7.2
क॒रो॒त्यथो॒ सम्भ॑रत्ये॒वैनं॒ यदब॑द्धमवद॒ध्याद्गर्भाः᳚ प्र॒जानां᳚ परा॒पातु॑काः स्युर्ब॒द्धमव॑ दधाति॒ गर्भा॑णां॒ धृत्यै॑ निष्ट॒र्क्यं॑ बध्नाति प्र॒जानां᳚ प्र॒जन॑नाय॒ वाग्वा ए॒षा यथ्सो॑म॒क्रय॑णी॒ जूर॒सीत्या॑ह॒ यद्धि मन॑सा॒ जव॑ते॒ तद्वा॒चा वद॑ति धृ॒ता मन॒सेत्या॑ह॒ मन॑सा॒ हि वाग्धृ॒ता जुष्टा॒ विष्ण॑व॒ इत्या॑ह~(४७)

%6.1.7.3
य॒ज्ञो वै विष्णु॑र्य॒ज्ञायै॒वैनां॒ जुष्टां᳚ करोति॒ तस्या᳚स्ते स॒त्यस॑वसः प्रस॒व इत्या॑ह सवि॒तृप्र॑सूतामे॒व वाच॒मव॑ रुन्धे॒ काण्डे॑काण्डे॒ वै क्रि॒यमा॑णे य॒ज्ञꣳ रक्षाꣳ॑सि जिघाꣳसन्त्ये॒ष खलु॒ वा अर॑क्षोहतः॒ पन्था॒ यो᳚\-ऽग्नेश्च॒ सूर्य॑स्य च॒ सूर्य॑स्य॒ चक्षु॒रारु॑हम॒ग्नेर॒क्ष्णः क॒नीनि॑का॒मित्या॑ह॒ य ए॒वार॑क्षोहतः॒ पन्था॒स्तꣳ स॒मारो॑हति~(४८)

%6.1.7.4
वाग्वा ए॒षा यथ्सो॑म॒क्रय॑णी॒ चिद॑सि म॒नासीत्या॑ह॒ शास्त्ये॒वैना॑मे॒तत्तस्मा᳚च्छि॒ष्टाः प्र॒जा जा॑यन्ते॒ चिद॒सीत्या॑ह॒ यद्धि मन॑सा चे॒तय॑ते॒ तद्वा॒चा वद॑ति म॒नासीत्या॑ह॒ यद्धि मन॑साभि॒गच्छ॑ति॒ तत्क॒रोति॒ धीर॒सीत्या॑ह॒ यद्धि मन॑सा॒ ध्याय॑ति॒ तद्वा॒चा~(४९)

%6.1.7.5
वद॑ति॒ दक्षि॑णा॒सीत्या॑ह॒ दक्षि॑णा ह्ये॑षा य॒ज्ञिया॒सीत्या॑ह य॒ज्ञिया॑मे॒वैनां᳚ करोति क्ष॒त्रिया॒सीत्या॑ह क्ष॒त्रिया॒ ह्ये॑षादि॑तिरस्युभ॒यतः॑शी॒र्॒\mbox{}ष्णीत्या॑ह॒ यदे॒वाऽऽदि॒त्यः प्रा॑य॒णीयो॑ य॒ज्ञाना॑मादि॒त्य उ॑दय॒नीय॒स्तस्मा॑दे॒वमा॑ह॒ यदब॑द्धा॒ स्यादय॑ता स्या॒द्यत्प॑दिब॒द्धानु॒स्तर॑णी स्यात्प्र॒मायु॑को॒ यज॑मानः स्यात्~(५०)

%6.1.7.6
यत्क॑र्णगृही॒ता वार्त्र॑घ्नी स्या॒थ्स वा॒न्यं जि॑नी॒यात्तं वा॒न्यो जि॑नीयान्मि॒त्रस्त्वा॑ प॒दि ब॑ध्ना॒त्वित्या॑ह मि॒त्रो वै शि॒वो दे॒वाना॒न्तेनै॒वैनां᳚ प॒दि ब॑ध्नाति पू॒षाध्व॑नः पा॒त्वित्या॑हे॒यं वै पू॒षेमामे॒वास्या॑ अधि॒पाम॑कः॒ सम॑ष्ट्या॒ इन्द्रा॒याध्य॑क्षा॒येत्या॒हेन्द्र॑मे॒वास्या॒ अध्य॑क्षं करोति~(५१)

%6.1.7.7
अनु॑ त्वा मा॒ता म॑न्यता॒मनु॑ पि॒तेत्या॒हानु॑मतयै॒वैन॑या क्रीणाति॒ सा दे॑वि दे॒वमच्छे॒हीत्या॑ह दे॒वी ह्ये॑षा दे॒वः सोम॒ इन्द्रा॑य॒ सोम॒मित्या॒हेन्द्रा॑य॒ हि सोम॑ आह्रि॒यते॒ यदे॒तद्यजु॒र्न ब्रू॒यात्परा᳚च्ये॒व सो॑म॒क्रय॑णीयाद्रु॒द्रस्त्वा व॑र्तय॒त्वित्या॑ह रु॒द्रो वै क्रू॒रः~(५२)

%6.1.7.8
दे॒वाना॒न्तमे॒वास्यै॑ प॒रस्ता᳚द्दधा॒त्यावृ॑त्त्यै क्रू॒रमि॑व॒ वा ए॒तत्क॑रोति॒ यद्रु॒द्रस्य॑ की॒र्तय॑ति मि॒त्रस्य॑ प॒थेत्या॑ह॒ शान्त्यै॑ वा॒चा वा ए॒ष वि क्री॑णीते॒ यः सो॑म॒क्रय॑ण्या स्व॒स्ति सोम॑सखा॒ पुन॒रेहि॑ स॒ह र॒य्येत्या॑ह वा॒चैव वि॒क्रीय॒ पुन॑रा॒त्मन्वाचं॑ ध॒त्ते\-ऽनु॑पदासुकास्य॒ वाग्भ॑वति॒ य ए॒वं वेद॑॥~(५३)

%6.1.8.0
{\anuvakamend[{सत॑नुं॒ विष्ण॑व॒ इत्या॑ह स॒मारो॑हति॒ ध्याय॑ति॒ तद्वा॒चा यज॑मानः स्यात्करोति क्रू॒रो वेद॑}]}%~(७)

%6.1.8.1
षट्प॒दान्यनु॒ नि क्रा॑मति षड॒हं वाङ्नाति॑ वदत्यु॒त सं॑वथ्स॒रस्याय॑ने॒ याव॑त्ये॒व वाक्तामव॑ रुन्धे सप्त॒मे प॒दे जु॑होति स॒प्तप॑दा॒ शक्व॑री प॒शवः॒ शक्व॑री प॒शूने॒वाव॑ रुन्धे स॒प्त ग्रा॒म्याः प॒शवः॑ स॒प्तार॒ण्याः स॒प्त छन्दाꣴ॑स्यु॒भय॒स्याव॑रुद्ध्यै॒ वस्व्य॑सि रु॒द्रासीत्या॑ह रू॒पमे॒वास्या॑ ए॒तन्म॑हि॒मानम्᳚~(५४)

%6.1.8.2
व्याच॑ष्टे॒ बृह॒स्पति॑स्त्वा सु॒म्ने र॑ण्व॒त्वित्या॑ह॒ ब्रह्म॒ वै दे॒वानां॒ बृह॒स्पति॒र्ब्रह्म॑णै॒वास्मै॑ प॒शूनव॑ रुन्धे रु॒द्रो वसु॑भि॒रा चि॑के॒त्वित्या॒हावृ॑त्त्यै पृथि॒व्यास्त्वा॑ मू॒र्धन्ना जि॑घर्मि देव॒यज॑न॒ इत्या॑ह पृथि॒व्या ह्ये॑ष मू॒र्धा यद्दे॑व॒यज॑न॒मिडा॑याः प॒द इत्या॒हेडा॑यै॒ ह्ये॑तत्प॒दं यथ्सो॑म॒क्रय॑ण्यै घृ॒तव॑ति॒ स्वाहा᳚~(५)

%6.1.8.3
इत्या॑ह॒ यदे॒वास्यै॑ प॒दाद् घृ॒तमपी᳚ड्यत॒ तस्मा॑दे॒वमा॑ह॒ यद॑ध्व॒र्युर॑न॒ग्नावाहु॑तिं जुहु॒याद॒न्धो᳚\-ऽध्व॒र्युः स्या॒द्रक्षाꣳ॑सि य॒ज्ञꣳ ह॑न्यु॒र्॒\mbox{}हिर॑ण्यमु॒पास्य॑ जुहोत्यग्नि॒वत्ये॒व जु॑होति॒ नान्धो᳚\-ऽध्व॒र्युर्भव॑ति॒ न य॒ज्ञꣳ रक्षाꣳ॑सि घ्नन्ति॒ काण्डे॑काण्डे॒ वै क्रि॒यमा॑णे य॒ज्ञꣳ रक्षाꣳ॑सि जिघाꣳसन्ति॒ परि॑लिखित॒ꣳ॒ रक्षः॒ परि॑लिखिता॒ अरा॑तय॒ इत्या॑ह॒ रक्ष॑सा॒मप॑हत्यै~(५६)

%6.1.8.4
इ॒दम॒हꣳ रक्ष॑सो ग्री॒वा अपि॑ कृन्तामि॒ यो᳚\-ऽस्मान्द्वेष्टि॒ यं च॑ व॒यं द्वि॒ष्म इत्या॑ह॒ द्वौ वाव पुरु॑षौ॒ यं चै॒व द्वेष्टि॒ यश्चै॑नं॒ द्वेष्टि॒ तयो॑रे॒वान॑न्तरायं ग्री॒वाः कृ॑न्तति प॒शवो॒ वै सो॑म॒क्रय॑ण्यै प॒दं या॑वत्त्मू॒तꣳ सं व॑पति प॒शूने॒वाव॑ रुन्धे॒\-ऽस्मे राय॒ इति॒ सं व॑पत्या॒त्मान॑मे॒वाध्व॒र्युः~(५७)

%6.1.8.5
प॒शुभ्यो॒ नान्तरे॑ति॒ त्वे राय॒ इति॒ यज॑मानाय॒ प्र य॑च्छति॒ यज॑मान ए॒व र॒यिं द॑धाति॒ तोते॒ राय॒ इति॒ पत्नि॑या अ॒र्धो वा ए॒ष आ॒त्मनो॒ यत्पत्नी॒ यथा॑ गृ॒हेषु॑ निध॒त्ते ता॒दृगे॒व तत्त्वष्टी॑मती ते सपे॒येत्या॑ह॒ त्वष्टा॒ वै प॑शू॒नां मि॑थु॒नानाꣳ॑ रूप॒कृद्रू॒पमे॒व प॒शुषु॑ दधात्य॒स्मै वै लो॒काय॒ गार्\mbox{}ह॑पत्य॒ आ धी॑यते॒\-ऽमुष्मा॑ आहव॒नीयो॒ यद्गार्\mbox{}ह॑पत्य उप॒वपे॑द॒स्मिँल्लो॒के प॑शु॒मान्थ्स्या॒द्यदा॑हव॒नीये॒\-ऽमुष्मिँ॑ल्लो॒के प॑शु॒मान्थ्स्या॑दु॒भयो॒रुप॑ वपत्यु॒भयो॑रे॒वैनं॑ लो॒कयोः᳚ पशु॒मन्तं॑ करोति॥~(५८)

%6.1.9.0
{\anuvakamend[{म॒हि॒मान॒ꣴ॒ स्वाहाप॑हत्या अध्व॒र्युर्धी॑यते॒ चतु॑र्विHꣳशतिश्च}]}%~(८)

%6.1.9.1
ब्र॒ह्म॒वा॒दिनो॑ वदन्ति वि॒चित्यः॒ सोमा~(३) न वि॒चित्या~(३) इति॒ सोमो॒ वा ओष॑धीना॒ꣳ॒ राजा॒ तस्मि॒न्॒ यदाप॑न्नं ग्रसि॒तमे॒वास्य॒ तद्यद्वि॑चिनु॒याद्यथा॒स्या᳚द्ग्रसि॒तं नि॑ष्खि॒दति॑ ता॒दृगे॒व तद्यन्न वि॑चिनु॒याद्यथा॒क्षन्नाप॑न्नं वि॒धाव॑ति ता॒दृगे॒व तत्क्षोधु॑को\-ऽध्व॒र्युः स्यात्क्षोधु॑को॒ यज॑मानः॒ सोम॑विक्रयि॒न्थ्सोमꣳ॑ शोध॒येत्ये॒व ब्रू॑या॒द्यदीत॑रम्~(५९)

%6.1.9.2
यदीत॑रमु॒भये॑नै॒व सो॑मविक्र॒यिण॑मर्पयति॒ तस्मा᳚थ्सोमविक्र॒यी क्षोधु॑को\-ऽरु॒णो ह॑ स्मा॒हौप॑वेशिः सोम॒क्रय॑ण ए॒वाहं तृ॑तीयसव॒नमव॑ रुन्ध॒ इति॑ पशू॒नां चर्म॑न्मिमीते प॒शूने॒वाव॑ रुन्धे प॒शवो॒ हि तृ॒तीय॒ꣳ॒ सव॑नं॒ यङ्का॒मये॑ताप॒शुः स्या॒दित्यृ॑क्ष॒तस्तस्य॑ मिमीत॒र्क्षं वा अ॑पश॒व्यम॑प॒शुरे॒व भ॑वति॒ यं का॒मये॑त पशु॒मान्थ्स्या᳚त्~(६०)

%6.1.9.3
इति॑ लोम॒तस्तस्य॑ मिमीतै॒तद्वै प॑शू॒नाꣳ रू॒पꣳ रू॒पेणै॒वास्मै॑ प॒शूनव॑ रुन्धे पशु॒माने॒व भ॑वत्य॒पामन्ते᳚ क्रीणाति॒ सर॑समे॒वैनं॑ क्रीणात्य॒मात्यो॒\-ऽसीत्या॑हा॒मैवैनं॑ कुरुते शु॒क्रस्ते॒ ग्रह॒ इत्या॑ह शु॒क्रो ह्य॑स्य॒ ग्रहो\-ऽन॒साच्छ॑ याति महि॒मान॑मे॒वास्याच्छ॑ या॒त्यन॑सा~(६१)

%6.1.9.4
अच्छ॑ याति॒ तस्मा॑दनोवा॒ह्यꣳ॑ स॒मे जीव॑नं॒ यत्र॒ खलु॒ वा ए॒तꣳ शी॒र्॒\mbox{}ष्णा हर॑न्ति॒ तस्मा᳚च्छीर्\mbox{}षहा॒र्यं॑ गि॒रौ जीव॑नम॒भि त्यं दे॒वꣳ स॑वि॒तार॒मित्यति॑छन्दस॒र्चा मि॑मी॒ते\-ऽति॑च्छन्दा॒ वै सर्वा॑णि॒ छन्दाꣳ॑सि॒ सर्वे॑भिरे॒वैनं॒ छन्दो॑भिर्मिमीते॒ वर्\mbox{}ष्म॒ वा ए॒षा छन्द॑सां॒ यदति॑च्छन्दा॒ यदति॑च्छन्दस॒र्चा मिमी॑ते॒ वर्\mbox{}ष्मै॒वैनꣳ॑ समा॒नानां᳚ करो॒त्येक॑यैकयो॒थ्सर्गम्᳚~(६२)

%6.1.9.5
मि॒मी॒ते\-ऽया॑तयाम्नियायातयाम्नियै॒वैनं॑ मिमीते॒ तस्मा॒न्नाना॑वीर्या अ॒ङ्गुल॑यः॒ सर्वा᳚स्वङ्गु॒ष्ठमुप॒ नि गृ॑ह्णाति॒ तस्मा᳚थ्स॒माव॑द्वीर्यो॒\-ऽन्याभि॑र॒ङ्गुलि॑भि॒स्तस्मा॒थ्सर्वा॒ अनु॒ सं च॑रति॒ यथ्स॒ह सर्वा॑भि॒र्मिमी॑त॒ सꣴश्लि॑ष्टा अ॒ङ्गुल॑यो जायेर॒न्नेक॑यैकयो॒थ्सर्गं॑ मिमीते॒ तस्मा॒द्विभ॑क्ता जायन्ते॒ पञ्च॒ कृत्वो॒ यजु॑षा मिमीते॒ पञ्चा᳚क्षरा प॒ङ्क्तिः पाङ्क्तो॑ य॒ज्ञो य॒ज्ञमे॒वाव॑ रुन्धे॒ पञ्च॒ कृत्व॑स्तू॒ष्णीम्~(६३)

%6.1.9.6
दश॒ सम्प॑द्यन्ते॒ दशा᳚क्षरा वि॒राडन्नं॑ वि॒राड्वि॒राजै॒वान्नाद्य॒मव॑ रुन्धे॒ यद्यजु॑षा॒ मिमी॑ते भू॒तमे॒वाव॑ रुन्धे॒ यत्तू॒ष्णीं भ॑वि॒ष्यद्यद्वै तावा॑ने॒व सोमः॒ स्याद्याव॑न्तं॒ मिमी॑ते॒ यज॑मानस्यै॒व स्या॒न्नापि॑ सद॒स्या॑नां प्र॒जाभ्य॒स्त्वेत्युप॒ समू॑हति सद॒स्या॑ने॒वान्वाभ॑जति॒ वास॒सोप॑ नह्यति सर्वदेव॒त्यं॑ वै~(६४)

%6.1.9.7
वासः॒ सर्वा॑भिरे॒वैनं॑ दे॒वता॑भिः॒ सम॑र्धयति प॒शवो॒ वै सोमः॑ प्रा॒णाय॒ त्वेत्युप॑ नह्यति प्रा॒णमे॒व प॒शुषु॑ दधाति व्या॒नाय॒ त्वेत्यनु॑ शृन्थति व्या॒नमे॒व प॒शुषु॑ दधाति॒ तस्मा᳚थ्स्व॒पन्तं॑ प्रा॒णा न ज॑हति॥~(६५)

%6.1.10.0
{\anuvakamend[{इत॑रं पशु॒मान्थ्स्या᳚द्या॒त्यन॑सो॒थ्सर्ग॑न्तू॒ष्णीꣳ स॑र्वदेव॒त्यं॑ वै त्रय॑स्त्रिꣳशच्च}]}%~(९)

%6.1.10.1
यत्क॒लया॑ ते श॒फेन॑ ते क्रीणा॒नीति॒ पणे॒तागो॑अर्घ॒ꣳ॒ सोमं॑ कु॒र्यादगो॑अर्घं॒ यज॑मान॒मगो॑अर्घमध्व॒र्युङ्गोस्तु म॑हि॒मानं॒ नाव॑ तिरे॒द्गवा॑ ते क्रीणा॒नीत्ये॒व ब्रू॑याद्गोअ॒र्घमे॒व सोमं॑ क॒रोति॑ गोअ॒र्घं यज॑मानं गोअ॒र्घम॑ध्व॒र्युन्न गोर्म॑हि॒मान॒मव॑ तिरत्य॒जया᳚ क्रीणाति॒ सत॑पसमे॒वैनं॑ क्रीणाति॒ हिर॑ण्येन क्रीणाति॒ सशु॑क्रमे॒व~(६६)

%6.1.10.2
ए॒नं॒ क्री॒णा॒ति॒ धे॒न्वा क्री॑णाति॒ साशि॑रमे॒वैनं॑ क्रीणात्यृष॒भेण॑ क्रीणाति॒ सेन्द्र॑मे॒वैनं॑ क्रीणात्यन॒डुहा᳚ क्रीणाति॒ वह्नि॒र्वा अ॑न॒ड्वान् वह्नि॑नै॒व वह्नि॑ य॒ज्ञस्य॑ क्रीणाति मिथु॒ना\-भ्यां᳚ क्रीणाति मिथु॒नस्याव॑रुद्ध्यै॒ वास॑सा क्रीणाति सर्वदेव॒त्यं॑ वै वासः॒ सर्वा᳚भ्य ए॒वैनं॑ दे॒वता᳚भ्यः क्रीणाति॒ दश॒ सम्प॑द्यन्ते॒ दशा᳚क्षरा वि॒राडन्नं॑ वि॒राड्वि॒राजै॒वान्नाद्य॒मव॑ रुन्धे~(६७)

%6.1.10.3
तप॑सस्त॒नूर॑सि प्र॒जा\-प॑ते॒र्वर्ण॒ इत्या॑ह प॒शुभ्य॑ ए॒व तद॑ध्व॒र्युर्नि ह्नु॑त आ॒त्मनो\-ऽना᳚व्रस्काय॒ गच्छ॑ति॒ श्रियं प्र प॒शूना᳚प्नोति॒ य ए॒वं वेद॑ शु॒क्रं ते॑ शु॒क्रेण॑ क्रीणा॒मीत्या॑ह यथाय॒जुरे॒वैतद्दे॒वा वै येन॒ हिर॑ण्येन॒ सोम॒मक्री॑ण॒न्तद॑भी॒षहा॒ पुन॒राद॑दत॒ को हि तेज॑सा विक्रे॒ष्यत॒ इति॒ येन॒ हिर॑ण्येन~(६८)

%6.1.10.4
सोमं॑ क्रीणी॒यात्तद॑भी॒षहा॒ पुन॒रा द॑दीत॒ तेज॑ ए॒वात्मन्ध॑त्ते॒\-ऽस्मे ज्योतिः॑ सोमविक्र॒यिणि॒ तम॒ इत्या॑ह॒ ज्योति॑रे॒व यज॑माने दधाति॒ तम॑सा॒ सोमविक्र॒यिण॑मर्पयति॒ यदनु॑पग्रथ्य ह॒न्याद्द॑न्द॒शूका॒स्ताꣳ समाꣳ॑ स॒र्पाः स्यु॑रि॒दम॒हꣳ स॒र्पाणां᳚ दन्द॒शूका॑नां ग्री॒वा उप॑ ग्रथ्ना॒मीत्या॒हाद॑न्दशूका॒स्ताꣳ समाꣳ॑ स॒र्पा भ॑वन्ति॒ तम॑सा सोमविक्र॒यिणं॑ विध्यति॒ स्वान॑~(६९)

%6.1.10.5
भ्राजेत्या॑है॒ते वा अ॒मुष्मिँ॑ल्लो॒के सोम॑मरक्ष॒न्तेभ्यो\-ऽधि॒ सोम॒माह॑र॒न्॒ यदे॒तेभ्यः॑ सोम॒क्रय॑णा॒न्नानु॑दि॒शेदक्री॑तो\-ऽस्य॒ सोमः॑ स्या॒न्नास्यै॒ते॑\-ऽमुष्मिँ॑ल्लो॒के सोमꣳ॑ रक्षेयु॒र्यदे॒तेभ्यः॑ सोम॒क्रय॑णाननुदि॒शति॑ क्री॒तो᳚\-ऽस्य॒ सोमो॑ भवत्ये॒ते᳚\-ऽस्या॒मुष्मिँ॑ल्लो॒के सोमꣳ॑ रक्षन्ति॥~(७०)

%6.1.11.0
{\anuvakamend[{सशु॑क्रमे॒व रु॑न्ध॒ इति॒ येन॒ हिर॑ण्येन॒ स्वान॒ चतु॑श्चत्वारिꣳशच्च}]}%॥10॥

%6.1.11.1
वा॒रु॒णो वै क्री॒तः सोम॒ उप॑नद्धो मि॒त्रो न॒ एहि॒ सुमि॑त्रधा॒ इत्या॑ह॒ शान्त्या॒ इन्द्र॑स्यो॒रुमा वि॑श॒ दक्षि॑ण॒मित्या॑ह दे॒वा वै यꣳ सोम॒मक्री॑ण॒न्तमिन्द्र॑स्यो॒रौ दक्षि॑ण॒ आसा॑दयन्ने॒ष खलु॒ वा ए॒तर्\mbox{}हीन्द्रो॒ यो यज॑ते॒ तस्मा॑दे॒वमा॒होदायु॑षा स्वा॒युषेत्या॑ह दे॒वता॑ ए॒वान्वा॒रभ्योत्~(७१)

%6.1.11.2
ति॒ष्ठ॒त्यु॒र्व॑न्तरि॑क्ष॒मन्वि॒हीत्या॑हान्तरिक्षदेव॒त्यो  ह्ये॑तर्\mbox{}हि॒ सोमो\-ऽदि॑त्याः॒ सदो॒\-ऽस्यदि॑त्याः॒ सद॒ आ सी॒देत्या॑ह यथाय॒जुरे॒वैतद्वि वा ए॑नमे॒तद॑र्धयति॒ यद्वा॑रु॒णꣳ सन्तं॑ मै॒त्रं क॒रोति॑ वारु॒ण्यर्चा सा॑दयति॒ स्वयै॒वैनं॑ दे॒वत॑या॒ सम॑र्धयति॒ वास॑सा प॒र्यान॑ह्यति सर्वदेव॒त्यं॑ वै वासः॒ सर्वा॑भिरे॒व~(७२)

%6.1.11.3
ए॒नं॒ दे॒वता॑भिः॒ सम॑र्धय॒त्यथो॒ रक्ष॑सा॒मप॑हत्यै॒ वने॑षु॒ व्य॑न्तरि॑क्षं तता॒नेत्या॑ह॒ वने॑षु॒ हि व्य॑न्तरि॑क्षं त॒तान॒ वाज॒मर्व॒थ्स्वित्या॑ह॒ वाज॒ꣴ॒ ह्यर्व॑थ्सु॒ पयो॑ अघ्नि॒यास्वित्या॑ह॒ पयो॒ ह्य॑घ्नि॒यासु॑ हृ॒थ्सु क्रतु॒मित्या॑ह हृ॒थ्सु हि क्रतुं॒ वरु॑णो वि॒क्ष्व॑ग्निमित्या॑ह॒ वरु॑णो॒ हि वि॒क्ष्व॑ग्निन्दि॒वि सूर्यम्᳚~(७३)

%6.1.11.4
इत्या॑ह दि॒वि हि सूर्य॒ꣳ॒ सोम॒मद्रा॒वित्या॑ह॒ ग्रावा॑णो॒ वा अद्र॑य॒स्तेषु॒ वा ए॒ष सोमं॑ दधाति॒ यो यज॑ते॒ तस्मा॑दे॒वमा॒होदु॒ त्यं जा॒तवे॑दस॒मिति॑ सौ॒र्यर्चा कृ॑ष्णाजि॒नं प्र॒त्यान॑ह्यति॒ रक्ष॑सा॒मप॑हत्या॒ उस्रा॒वेतं॑ धूर्\mbox{}षाहा॒वित्या॑ह यथाय॒जुरे॒वैतत्प्र च्य॑वस्व भुवस्पत॒ इत्या॑ह भू॒ताना॒ꣳ॒ हि~(७४)

%6.1.11.5
ए॒ष पति॒र्विश्वा᳚न्य॒भि धामा॒नीत्या॑ह॒ विश्वा॑नि॒ ह्ये  षो॑\-ऽभि धामा॑नि प्र॒च्यव॑ते॒ मा त्वा॑ परिप॒री वि॑द॒दित्या॑ह॒ यदे॒वादः सोम॑माह्रि॒यमा॑णं गन्ध॒र्वो वि॒श्वाव॑सुः प॒र्यमु॑ष्णा॒त्तस्मा॑दे॒वमा॒हाप॑रिमोषाय॒ यज॑मानस्य स्व॒स्त्यय॑न्य॒सीत्या॑ह॒ यज॑मानस्यै॒वैष य॒ज्ञस्या᳚न्वार॒म्भो\-ऽन॑वछित्त्यै॒ वरु॑णो॒ वा ए॒ष यज॑मानम॒भ्यैति॒ यत्~(७५)

%6.1.11.6
क्री॒तः सोम॒ उप॑नद्धो॒ नमो॑ मि॒त्रस्य॒ वरु॑णस्य॒ चक्ष॑स॒ इत्या॑ह॒ शान्त्या॒ आ सोमं॒ वह॑न्त्य॒ग्निना॒ प्रति॑ तिष्ठते॒ तौ स॒म्भव॑न्तौ॒ यज॑मानम॒भि सम्भ॑वतः पु॒रा खलु॒ वावैष मेधा॑या॒त्मान॑मा॒रभ्य॑ चरति॒ यो दी᳚क्षि॒तो यद॑ग्नीषो॒मीयं॑ प॒शुमा॒लभ॑त आत्मनि॒ष्क्रय॑ण ए॒वास्य॒ स तस्मा॒त्तस्य॒ नाश्यं॑ पुरुषनि॒ष्क्रय॑ण इव॒ ह्यथो॒ खल्वा॑हुर॒ग्नी\-षोमा᳚भ्यां॒ वा इन्द्रो॑ वृ॒त्रम॑ह॒न्निति॒ यद॑ग्नीषो॒मीयं॑ प॒शुमा॒लभ॑ते॒ वार्त्र॑घ्न ए॒वास्य॒ स तस्मा᳚द्वा॒श्यं॑ वारु॒ण्यर्चा परि॑ चरति॒ स्वयै॒वैनं॑ दे॒वत॑या॒ परि॑ चरति॥~(७६)

%6.2.0.0
{\anuvakamend[{अ॒न्वा॒रभ्योथ्सर्वा॑भिरे॒व सूर्यं॑ भू॒ताना॒ꣴ॒ ह्ये॑ति॒ यदा॑हुः स॒प्तविꣳ॑शतिश्च}]}%॥11॥

%6.2.0.0

{\anuvakamend[{यदु॒भौ दे॑वासु॒रा मि॒थस्तेषाꣳ॑ सुव॒र्गं यद्वा अनी॑शानः पु॒रोह॑विषि॒ तेभ्यः॒ सोत्त॑रवे॒दिर्ब॒द्धं दे॒वस्याभ्रि॒ꣳ॒ शिरो॒ वा एका॑\-दश}]}%॥11॥
{\prashnaend{यदु॒भावित्या॑ह दे॒वानां᳚ य॒ज्ञो दे॒वेभ्यो॒ न रथा॑य॒ यज॑मानाय प॒रस्ता॑द॒र्वाची॒न्नव॑पञ्चा॒शत्॥59॥ यदु॒भौ दु॒ह ए॒वैना᳚म्॥}}
%%% END PRASHNA

\sect{द्वितीयः प्रश्नः}\setcounter{anuvakam}{0}
\dnsub{तैत्तिरीयसंहितायां षष्ठमकाण्डे द्वितीयः प्रश्नः}
%6.2.1.0
%6.2.1.1
यदु॒भौ वि॒मुच्या॑ति॒थ्यं गृ॑ह्णी॒याद्य॒ज्ञं विच्छि॑न्द्या॒द्यदु॒भाववि॑मुच्य॒ यथाना॑गतायाति॒थ्यं क्रि॒यते॑ ता॒दृगे॒व तद्विमु॑क्तो॒\-ऽन्यो॑\-ऽन॒ड्वान्भव॒त्यवि॑मुक्तो॒\-ऽन्यो\-ऽथा॑ति॒थ्यं गृ॑ह्णाति य॒ज्ञस्य॒ सन्त॑त्यै॒ पत्न्य॒न्वार॑भते॒ पत्नी॒ हि पारी॑णह्य॒स्येशे॒ पत्नि॑यै॒वानु॑मतं॒ निर्व॑पति॒ यद्वै पत्नी॑ य॒ज्ञस्य॑ क॒रोति॑ मिथु॒नं तदथो॒ पत्नि॑या ए॒व~(१)

%6.2.1.2
ए॒ष य॒ज्ञस्या᳚न्वार॒म्भो\-ऽन॑वच्छित्त्यै॒ याव॑द्भि॒र्वै राजा॑नुच॒रैरा॒गच्छ॑ति॒ सर्वे᳚भ्यो॒ वै तेभ्य॑ आति॒थ्यं क्रि॑यते॒ छन्दाꣳ॑सि॒ खलु॒ वै सोम॑स्य॒ राज्ञो॑\-ऽनुच॒राण्य॒ग्नेरा॑ति॒थ्यम॑सि॒ विष्ण॑वे॒ त्वेत्या॑ह गायत्रि॒या ए॒वैतेन॑ करोति॒ सोम॑स्याति॒थ्यम॑सि॒ विष्ण॑वे॒ त्वेत्या॑ह त्रि॒ष्टुभ॑ ए॒वैतेन॑ करो॒त्यति॑थेराति॒थ्यम॑सि॒ विष्ण॑वे॒ त्वेत्या॑ह॒ जग॑त्यै~(२)

%6.2.1.3
ए॒वैतेन॑ करोत्य॒ग्नये᳚ त्वा रायस्पोष॒दाव्ने॒ विष्ण॑वे॒ त्वेत्या॑हानु॒ष्टुभ॑ ए॒वैतेन॑ करोति श्ये॒नाय॑ त्वा सोम॒भृते॒ विष्ण॑वे॒ त्वेत्या॑ह गायत्रि॒या ए॒वैतेन॑ करोति॒ पञ्च॒ कृत्वो॑ गृह्णाति॒ पञ्चा᳚क्षरा प॒ङ्क्तिः पाङ्क्तो॑ य॒ज्ञो य॒ज्ञमे॒वाव॑ रुन्धे ब्रह्मवा॒दिनो॑ वदन्ति॒ कस्मा᳚थ्स॒त्याद्गा॑यत्रि॒या उ॑भ॒यत॑ आति॒थ्यस्य॑ क्रियत॒ इति॒ यदे॒वादः सोम॒मा~(३)

%6.2.1.4
अह॑र॒त्तस्मा᳚द्गायत्रि॒या उ॑भ॒यत॑ आति॒थ्यस्य॑ क्रियते पु॒रस्ता᳚च्चो॒परि॑ष्टाच्च॒ शिरो॒ वा ए॒तद्य॒ज्ञस्य॒ यदा॑ति॒थ्यं नव॑कपालः पुरो॒डाशो॑ भवति॒ तस्मा᳚न्नव॒धा शिरो॒ विष्यू॑त॒न्नव॑कपालः पुरो॒डाशो॑ भवति॒ ते त्रय॑स्त्रिकपा॒लास्त्रि॒वृता॒ स्तोमे॑न॒ सम्मि॑ता॒स्तेज॑स्त्रि॒वृत्तेज॑ ए॒व य॒ज्ञस्य॑ शी॒र्॒\mbox{}षं द॑धाति॒ नव॑कपालः पुरो॒डाशो॑ भवति॒ ते त्रय॑स्त्रिकपा॒लास्त्रि॒वृता᳚ प्रा॒णेन॒ सम्मि॑तास्त्रि॒वृद्वै~(४)

%6.2.1.5
प्रा॒णस्त्रि॒वृत॑मे॒व प्रा॒णम॑भिपू॒र्वं य॒ज्ञस्य॑ शी॒र्॒\mbox{}षं द॑धाति प्र॒जा\-प॑ते॒र्वा ए॒तानि॒ पक्ष्मा॑णि॒ यद॑श्ववा॒ला ऐ᳚क्ष॒वी ति॒रश्ची॒ यदाश्व॑वालः प्रस्त॒रो भव॑त्यैक्ष॒वी ति॒रश्ची᳚ प्र॒जा\-प॑तेरे॒व तच्चक्षुः॒ सम्भ॑रति दे॒वा वै या आहु॑ती॒रजु॑हवु॒स्ता असु॑रा नि॒ष्काव॑माद॒न्ते दे॒वाः का᳚र्\mbox{}ष्म॒र्य॑मपश्यन्कर्म॒ण्यो॑ वै कर्मै॑नेन कुर्वी॒तेति॒ ते का᳚र्ष्मर्य॒मया᳚न्परि॒धीन्~(५)

%6.2.1.6
अ॒कु॒र्व॒त॒ तैर्वै ते रक्षा॒ꣴ॒स्यपा᳚घ्नत॒ यत्का᳚र्ष्मर्य॒मयाः᳚ परि॒धयो॒ भव॑न्ति॒ रक्ष॑सा॒मप॑हत्यै॒ सꣴस्प॑र्\mbox{}शयति॒ रक्ष॑सा॒मन॑न्ववचाराय॒ न पु॒रस्ता॒त्परि॑ दधात्यादि॒त्यो ह्ये॑वोद्यन्पु॒रस्ता॒द्रक्षाꣴ॑स्यप॒हन्त्यू॒र्ध्वे स॒मिधा॒वा द॑धात्यु॒परि॑ष्टादे॒व रक्षा॒ꣴ॒स्यप॑हन्ति॒ यजु॑षा॒न्यां तू॒ष्णीम॒न्यां मि॑थुन॒त्वाय॒ द्वे आ द॑धाति द्वि॒पाद्यज॑मानः॒ प्रति॑ष्ठित्यै ब्रह्मवा॒दिनो॑ वदन्ति~(६)

%6.2.1.7
अ॒ग्निश्च॒ वा ए॒तौ सोम॑श्च क॒था सोमा॑याति॒थ्यं क्रि॒यते॒ नाग्नय॒ इति॒ यद॒ग्नाव॒ग्निं म॑थि॒त्वा प्र॒हर॑ति॒ तेनै॒वाग्नय॑ आति॒थ्यं क्रि॑य॒ते\-ऽथो॒ खल्वा॑हुर॒ग्निः सर्वा॑ दे॒वता॒ इति॒ यद्ध॒विरा॒साद्या॒ग्निं मन्थ॑ति ह॒व्यायै॒वास॑न्नाय॒ सर्वा॑ दे॒वता॑ जनयति॥~(७)

%6.2.2.0
{\anuvakamend[{पत्नि॑या ए॒व जग॑त्या॒ आ त्रि॒वृद्वै प॑रि॒धीन् व॑द॒न्त्येक॑चत्वारिꣳशच्च}]}%~(१)

%6.2.2.1
दे॒वा॒सु॒राः संय॑त्ता आस॒न्ते दे॒वा मि॒थो विप्रि॑या आस॒न्ते\-ऽ  न्यो᳚न्यस्मै॒ ज्यैष्ठ्या॒याति॑ष्ठमानाः पञ्च॒धा व्य॑क्रामन्न॒ग्नि\-र्वसु॑भिः॒ सोमो॑ रु॒द्रैरिन्द्रो॑ म॒रुद्भि॒र्वरु॑ण आदि॒त्यैर्बृह॒स्पति॒र्विश्वै᳚र्दे॒वैस्ते॑\-ऽमन्य॒न्तासु॑रेभ्यो॒ वा इ॒दं भ्रातृ॑व्येभ्यो रध्यामो॒ यन्मि॒थो विप्रि॑याः॒ स्मो या न॑ इ॒माः प्रि॒यास्त॒नुव॒स्ताः स॒मव॑द्यामहै॒ ताभ्यः॒ स निर्\mbox{}ऋ॑च्छा॒द्यः~(८)

%6.2.2.2
नः॒ प्र॒थ॒मो\-ऽ  न्यो᳚न्यस्मै॒ द्रुह्या॒दिति॒ तस्मा॒द्यः सता॑नूनप्त्रिणां प्रथ॒मो द्रुह्य॑ति॒ स आर्ति॒मार्च्छ॑ति॒ यत्ता॑नून॒प्त्रꣳ स॑मव॒द्यति॒ भ्रातृ॑व्याभिभूत्यै॒ भव॑त्या॒त्मना॒ परा᳚स्य॒ भ्रातृ॑व्यो भवति॒ पञ्च॒ कृत्वो\-ऽव॑ द्यति पञ्च॒धा हि ते तथ्स॑म॒वाद्य॒न्ताथो॒ पञ्चा᳚क्षरा प॒ङ्क्तिः पाङ्क्तो॑ य॒ज्ञो य॒ज्ञमे॒वाव॑ रुन्ध॒ आप॑तये त्वा गृह्णा॒मीत्या॑ह प्रा॒णो वै~(९)

%6.2.2.3
आप॑तिः प्रा॒णमे॒व प्री॑णाति॒ परि॑पतय॒ इत्या॑ह॒ मनो॒ वै परि॑पति॒र्मन॑ ए॒व प्री॑णाति॒ तनू॒नप्त्र॒ इत्या॑ह त॒नुवो॒ हि ते ताः स॑म॒वाद्य॑न्त शाक्व॒रायेत्या॑ह॒ शक्त्यै॒ हि ते ताः स॑म॒वाद्य॑न्त॒ शक्म॒न्नोजि॑ष्ठा॒येत्या॒हौजि॑ष्ठ॒ꣳ॒ हि ते तदा॒त्मनः॑ सम॒वाद्य॒न्ताना॑धृष्टमस्यनाधृ॒ष्यमित्या॒हाना॑धृष्ट॒ꣴ॒ ह्ये॑तद॑नाधृ॒ष्यं दे॒वाना॒मोजः॑~(१०)

%6.2.2.4
इत्या॑ह दे॒वाना॒ꣴ॒ ह्ये॑तदोजो॑\-ऽभिशस्ति॒पा अ॑नभिशस्ते॒न्यमित्या॑हाभिशस्ति॒पा ह्ये॑तद॑नभिशस्ते॒न्यमनु॑ मे दी॒क्षां दी॒क्षाप॑तिर्मन्यता॒मित्या॑ह यथाय॒जुरे॒वैतद् घृ॒तं वै दे॒वा वज्रं॑ कृ॒त्वा सोम॑मघ्नन्नन्ति॒कमि॑व॒ खलु॒ वा अ॑स्यै॒तच्च॑रन्ति॒ यत्ता॑नून॒प्त्रेण॑ प्र॒चर॑न्त्य॒ꣳ॒शुरꣳ॑शुस्ते देव सो॒मा प्या॑यता॒मित्या॑ह॒ यत्~(११)

%6.2.2.5
ए॒वास्या॑पुवा॒यते॒ यन्मीय॑ते॒ तदे॒वास्यै॒तेना प्या॑यय॒त्या तुभ्य॒मिन्द्रः॑ प्यायता॒मा त्वमिन्द्रा॑य प्याय॒स्वेत्या॑हो॒भावे॒वेन्द्रं॑ च॒ सोमं॒ चा प्या॑यय॒त्या प्या॑यय॒ सखी᳚न्थ्स॒न्या मे॒धयेत्या॑ह॒र्त्विजो॒ वा अ॑स्य॒ सखा॑य॒स्ताने॒वा प्या॑ययति स्व॒स्ति ते॑ देव सोम सु॒त्याम॑शीय~(१२)

%6.2.2.6
इत्या॑हा॒ऽऽशिष॑मे॒वैतामा शा᳚स्ते॒ प्र वा ए॒ते᳚\-ऽस्माल्लो॒काच्च्य॑वन्ते॒ ये सोम॑माप्या॒यय॑न्त्यन्तरिक्षदेव॒त्यो॑ हि सोम॒ आप्या॑यित॒ एष्टा॒ रायः॒ प्रेषे भगा॒येत्या॑ह॒ द्यावा॑पृथि॒वीभ्या॑मे॒व न॑म॒स्कृत्या॒स्मिँल्लो॒के प्रति॑ तिष्ठन्ति देवासु॒राः संय॑त्ता आस॒न्ते दे॒वा बिभ्य॑तो॒\-ऽग्निं प्रावि॑श॒न्तस्मा॑दाहुर॒ग्निः सर्वा॑ दे॒वता॒ इति॒ ते~(१३)

%6.2.2.7
अ॒ग्निमे॒व वरू॑थं कृ॒त्वासु॑रान॒भ्य॑भवन्न॒ग्निमि॑व॒ खलु॒ वा ए॒ष प्र वि॑शति॒ यो॑\-ऽवान्तरदी॒क्षामु॒पैति॒ भ्रातृ॑व्याभिभूत्यै॒ भव॑त्या॒त्मना॒ परा᳚स्य॒ भ्रातृ॑व्यो भवत्या॒त्मान॑मे॒व दी॒क्षया॑ पाति प्र॒जाम॑वान्तरदी॒क्षया॑ सन्त॒रां मेख॑लाꣳ स॒माय॑च्छते प्र॒जा ह्या᳚त्मनो\-ऽन्त॑रतरा त॒प्तव्र॑तो भवति॒ मद॑न्तीभिर्मार्जयते॒ निर्\mbox{}ह्य॑ग्निः शी॒तेन॒ वाय॑ति॒ समि॑द्ध्यै॒ या ते॑ अग्ने॒ रुद्रि॑या त॒नूरित्या॑ह॒ स्वयै॒वैन॑द्दे॒वत॑या व्रतयति सयोनि॒त्वाय॒ शान्त्यै᳚॥~(१४)

%6.2.3.0
{\anuvakamend[{यो वा ओज॑ आह॒ यद॑शी॒येति॒ ते᳚\-ऽग्न॒ एका॑\-दश च}]}%~(२)

%6.2.3.1
तेषा॒मसु॑राणान्ति॒स्रः पुर॑ आसन्नय॒स्मय्य॑व॒मा\-ऽथ॑ रज॒ता\-ऽथ॒ हरि॑णी॒ ता दे॒वा जेतु॒न्नाश॑क्नुव॒न्ता उ॑प॒सदै॒वाजि॑गीष॒न्तस्मा॑दाहु॒र्यश्चै॒वं वेद॒ यश्च॒ नोप॒सदा॒ वै म॑हापु॒रं ज॑य॒न्तीति॒ त इषु॒ꣳ॒ सम॑स्कुर्वता॒ग्निमनी॑क॒ꣳ॒ सोमꣳ॑ श॒ल्यं विष्णु॒न्तेज॑न॒न्ते᳚\-ऽब्रुव॒न्क इ॒माम॑सिष्य॒तीति॑~(१५)

%6.2.3.2
रु॒द्र इत्य॑ब्रुवन्रु॒द्रो वै क्रू॒रः सो᳚\-ऽस्य॒त्विति॒ सो᳚\-ऽब्रवी॒द्वरं॑ वृणा अ॒हमे॒व प॑शू॒नामधि॑पतिरसा॒नीति॒ तस्मा᳚द्रु॒द्रः प॑शू॒नामधि॑पति॒स्ताꣳ रु॒द्रो\-ऽवा॑सृज॒थ्स ति॒स्रः पुरो॑ भि॒त्त्वैभ्यो लो॒केभ्यो\-ऽसु॑रा॒न्प्राणु॑दत॒ यदु॑प॒सद॑ उपस॒द्यन्ते॒ भ्रातृ॑व्यपराणुत्त्यै॒ नान्यामाहु॑तिं पु॒रस्ता᳚ज्जुहुया॒द्यद॒न्यामाहु॑तिं पु॒रस्ता᳚ज्जुहु॒यात्~(१६)

%6.2.3.3
अ॒न्यन्मुखं॑ कुर्याथ्स्रु॒वेणा॑घा॒रमा घा॑रयति य॒ज्ञस्य॒ प्रज्ञा᳚त्यै॒ परा॑ङति॒क्रम्य॑ जुहोति॒ परा॑च ए॒वैभ्यो लो॒केभ्यो॒ यज॑मानो॒ भ्रातृ॑व्या॒न्प्र णु॑दते॒ पुन॑रत्या॒क्रम्यो॑प॒सदं॑ जुहोति प्र॒णुद्यै॒वैभ्यो लो॒केभ्यो॒ भ्रातृ॑व्याञ्जि॒त्वा भ्रा॑तृव्यलो॒कम॒भ्यारो॑हति दे॒वा वै याः प्रा॒तरु॑प॒सद॑ उ॒पासी॑द॒न्नह्न॒स्ताभि॒रसु॑रा॒न्प्राणु॑दन्त॒ याः सा॒यꣳ रात्रि॑यै॒ ताभि॒र्यथ्सा॒यं प्रा॑तरुप॒सदः॑~(१७)

%6.2.3.4
उ॒प॒स॒द्यन्ते॑\-ऽहोरा॒त्राभ्या॑मे॒व तद्यज॑मानो॒ भ्रातृ॑व्या॒न्प्र णु॑दते॒ याः प्रा॒तर्या॒ज्याः᳚ स्युस्ताः सा॒यं पु॑रोनुवा॒क्याः᳚ कुर्या॒दया॑तयामत्वाय ति॒स्र उ॑प॒सद॒ उपै॑ति॒ त्रय॑ इ॒मे लो॒का इ॒माने॒व लो॒कान्प्री॑णाति॒ षट्थ्सम्प॑द्यन्ते॒ षड्वा ऋ॒तव॑ ऋ॒तूने॒व प्री॑णाति॒ द्वाद॑शा॒हीने॒ सोम॒ उपै॑ति॒ द्वाद॑श॒ मासाः᳚ संवथ्स॒रः सं॑वथ्स॒रमे॒व प्री॑णाति॒ चतु॑र्विꣳशतिः॒ सम्~(१८)

%6.2.3.5
प॒द्य॒न्ते॒ चतु॑र्विꣳशतिरर्धमा॒सा अ॑र्धमा॒साने॒व प्री॑णा॒त्यारा᳚ग्रामवान्तरदी॒क्षामुपे॑या॒द्यः का॒मये॑ता॒स्मिन्मे॑ लो॒के\-ऽर्धु॑कꣴ स्या॒दित्येक॒मग्रे\-ऽथे॒ द्वावथ॒ त्रीनथ॑ च॒तुर॑ ए॒षा वा आरा᳚ग्रावान्तरदी॒क्षास्मिन्ने॒वास्मै॑ लो॒के\-ऽर्धु॑कं भवति प॒रोव॑रीयसीमवान्तरदी॒क्षामुपे॑या॒द्यः का॒मये॑ता॒मुष्मि॑न्मे लो॒के\-ऽर्धु॑कꣴ स्या॒दिति॑ च॒तुरो\-ऽग्रे\-ऽथ॒ त्रीनथ॒ द्वावथैक॑मे॒षा वै प॒रोव॑रीयस्यवान्तरदी॒क्षामुष्मि॑न्ने॒वास्मै॑ लो॒के\-ऽर्धु॑कं भवति॥~(१९)

%6.2.4.0
{\anuvakamend[{अ॒सि॒ष्य॒तीति॑ जुहु॒याथ्सा॒यं प्रा॑तरुप॒सद॒श्चतु॑र्विꣳशतिः॒ सञ्च॒तुरो\-ऽग्रे॒ षोड॑श च}]}%~(३)

%6.2.4.1
सु॒व॒र्गं वा ए॒ते लो॒कं य॑न्ति॒ य उ॑प॒सद॑ उप॒यन्ति॒ तेषां॒ य उ॒न्नय॑ते॒ हीय॑त ए॒व स नोद॑ने॒षीति॒ सू᳚न्नीयमिव॒ यो वै स्वा॒र्थेतां᳚ य॒ताꣴ श्रा॒न्तो हीय॑त उ॒त स नि॒ष्ट्याय॑ स॒ह व॑सति॒ तस्मा᳚थ्स॒कृदु॒न्नीय॒ नाप॑र॒मुन्न॑येत द॒ध्नोन्न॑येतै॒तद्वै प॑शू॒नाꣳ रू॒पꣳ रू॒पेणै॒व प॒शूनव॑ रुन्धे~(२०)

%6.2.4.2
य॒ज्ञो दे॒वेभ्यो॒ निला॑यत॒ विष्णू॑ रू॒पं कृ॒त्वा स पृ॑थि॒वीं प्रावि॑श॒त्तं दे॒वा हस्ता᳚न्थ्स॒ꣳ॒रभ्यै᳚च्छ॒न्तमिन्द्र॑ उ॒पर्यु॑प॒र्यत्य॑क्राम॒थ्सो᳚\-ऽब्रवी॒त्को मा॒यमु॒पर्यु॑प॒र्यत्य॑क्रमी॒दित्य॒हं दु॒र्गे हन्तेत्यथ॒ कस्त्वमित्य॒हं दु॒र्गादाह॒र्तेति॒ सो᳚\-ऽब्रवीद्दु॒र्गे वै हन्ता॑वोचथा वरा॒हो॑\-ऽयं वा॑ममो॒षः~(२१)

%6.2.4.3
स॒प्ता॒नां गि॑री॒णां प॒रस्ता᳚द्वि॒त्तं वेद्य॒मसु॑राणां बिभर्ति॒ तं ज॑हि॒ यदि॑ दु॒र्गे हन्तासीति॒ स द॑र्भपुञ्जी॒लमु॒द्वृह्य॑ स॒प्त गि॒रीन्भि॒त्त्वा तम॑ह॒न्थ्सो᳚\-ऽब्रवीद्दु॒र्गाद्वा आह॑र्तावोचथा ए॒तमा ह॒रेति॒ तमे᳚भ्यो य॒ज्ञ ए॒व य॒ज्ञमाह॑र॒द्यत्तद्वि॒त्तं वेद्य॒मसु॑राणा॒मवि॑न्दन्त॒ तदेकं॒ वेद्यै॑ वेदि॒त्वमसु॑राणाम्~(२२)

%6.2.4.4
वा इ॒यमग्र॑ आसी॒द्याव॒दासी॑नः परा॒पश्य॑ति॒ ताव॑द्दे॒वाना॒न्ते दे॒वा अ॑ब्रुव॒न्नस्त्वे॒व नो॒\-ऽस्यामपीति॒ किय॑द्वो दास्याम॒ इति॒ याव॑दि॒यꣳ स॑लावृ॒की त्रिः प॑रि॒क्राम॑ति॒ ताव॑न्नो द॒त्तेति॒ स इन्द्रः॑ सलावृ॒की रू॒पं कृ॒त्वेमां त्रिः स॒र्वतः॒ पर्य॑क्राम॒त्तदि॒माम॑विन्दन्त॒ यदि॒मामवि॑न्दन्त॒ तद्वेद्यै॑ वेदि॒त्वम्~(२३)

%6.2.4.5
सा वा इ॒यꣳ सर्वै॒व वेदि॒रिय॑ति शक्ष्या॒मीति॒ त्वा अ॑व॒माय॑ यजन्ते त्रि॒ꣳ॒शत्प॒दानि॑ प॒श्चात्ति॒रश्ची॑ भवति॒ षट्त्रिꣳ॑श॒त्प्राची॒ चतु॑र्विꣳशतिः पु॒रस्ता᳚त्ति॒रश्ची॒ दश॑दश॒ सम्प॑द्यन्ते॒ दशा᳚क्षरा वि॒राडन्नं॑ वि॒राड्वि॒राजै॒वान्नाद्य॒मव॑ रुन्ध॒ उद्ध॑न्ति॒ यदे॒वास्या॑ अमे॒ध्यं तदप॑ ह॒न्त्युद्ध॑न्ति॒ तस्मा॒दोष॑धयः॒ परा॑ भवन्ति ब॒र्॒\mbox{}हिः स्तृ॑णाति॒ तस्मा॒दोष॑धयः॒ पुन॒रा भ॑व॒न्त्युत्त॑रं ब॒र्॒\mbox{}हिष॑ उत्तरब॒र्॒\mbox{}हिः स्तृ॑णाति प्र॒जा वै ब॒र्॒\mbox{}हिर्यज॑मान उत्तरब॒र्॒\mbox{}हिर्यज॑मानमे॒वाय॑जमाना॒दुत्त॑रं करोति॒ तस्मा॒द्यज॑मा॒नो\-ऽय॑जमाना॒दुत्त॑रः॥~(२४)

%6.2.5.0
{\anuvakamend[{रु॒न्धे॒ वा॒म॒मो॒षो वे॑दि॒त्वमसु॑राणां वेदि॒त्वं भ॑वन्ति॒ पञ्च॑विꣳशतिश्च}]}%~(४)

%6.2.5.1
यद्वा अनी॑शानो भा॒रमा॑द॒त्ते वि वै स लि॑शते॒ यद्द्वाद॑श सा॒ह्नस्यो॑प॒सदः॒ स्युस्ति॒स्रो॑\-ऽहीन॑स्य य॒ज्ञस्य॒ विलो॑म क्रियेत ति॒स्र ए॒व सा॒ह्नस्यो॑प॒सदो॒ द्वाद॑शा॒हीन॑स्य य॒ज्ञस्य॑ सवीर्य॒त्वायाथो॒ सलो॑म क्रियते व॒थ्सस्यैकः॒ स्तनो॑ भा॒गी हि सो\-ऽथैक॒ꣴ॒ स्तनं॑ व्र॒तमुपै॒त्यथ॒ द्वावथ॒ त्रीनथ॑ च॒तुर॑ ए॒तद्वै~(२५)

%6.2.5.2
क्षु॒रप॑वि॒ नाम॑ व्र॒तं येन॒ प्र जा॒तान्भ्रातृ॑व्यान्नु॒दते॒ प्रति॑ जनि॒ष्यमा॑णा॒नथो॒ कनी॑यसै॒व भूय॒ उपै॑ति च॒तुरो\-ऽग्रे॒ स्तना᳚न्व्र॒तमुपै॒त्यथ॒ त्रीनथ॒ द्वावथैक॑मे॒तद्वै सु॑जघ॒नं नाम॑ व्र॒तं त॑प॒स्यꣳ॑ सुव॒र्ग्य॑मथो॒ प्रैव जा॑यते प्र॒जया॑ प॒शुभि॑र्यवा॒गू रा॑ज॒न्य॑स्य व्र॒तं क्रू॒रेव॒ वै य॑वा॒गूः क्रू॒र इ॑व~(२६)

%6.2.5.3
रा॒ज॒न्यो॑ वज्र॑स्य रू॒पꣳ समृ॑द्ध्या आ॒मिक्षा॒ वैश्य॑स्य पाकय॒ज्ञस्य॑ रू॒पं पुष्ट्यै॒ पयो᳚ ब्राह्म॒णस्य॒ तेजो॒ वै ब्रा᳚ह्म॒णस्तेजः॒ पय॒स्तेज॑सै॒व तेजः॒ पय॑ आ॒त्मन्ध॒त्ते\-ऽथो॒ पय॑सा॒ वै गर्भा॑ वर्धन्ते॒ गर्भ॑ इव॒ खलु॒ वा ए॒ष यद्दी᳚क्षि॒तो यद॑स्य॒ पयो᳚ व्र॒तं भव॑त्या॒त्मान॑मे॒व तद्व॑र्धयति॒ त्रिव्र॑तो॒ वै मनु॑रासी॒द्द्विव्र॑ता॒ असु॑रा॒ एक॑व्रताः~(२७)

%6.2.5.4
दे॒वाः प्रा॒तर्म॒ध्यन्दि॑ने सा॒यं तन्मनो᳚र्व्र॒तमा॑सीत्पाकय॒ज्ञस्य॑ रू॒पं पुष्ट्यै᳚ प्रा॒तश्च॑ सा॒यं चासु॑राणां निर्म॒ध्यं क्षु॒धो रू॒पं तत॒स्ते परा॑भवन्म॒ध्यन्दि॑ने मध्यरा॒त्रे दे॒वानां॒ तत॒स्ते॑\-ऽभवन्थ्सुव॒र्गं लो॒कमा॑य॒न्॒ यद॑स्य म॒ध्यन्दि॑ने मध्यरा॒त्रे व्र॒तं भव॑ति मध्य॒तो वा अन्ने॑न भुञ्जते मध्य॒त ए॒व तदूर्जं॑ धत्ते॒ भ्रातृ॑व्याभिभूत्यै॒ भव॑त्या॒त्मना᳚~(२८)




%6.2.5.5
परा᳚\-ऽस्य॒ भ्रातृ॑व्यो भवति॒ गर्भो॒ वा ए॒ष यद्दी᳚क्षि॒तो योनि॑र्दीक्षितविमि॒तं यद्दी᳚क्षि॒तो दी᳚क्षितविमि॒तात्प्र॒वसे॒द्यथा॒ योने॒र्गर्भः॒ स्कन्द॑ति ता॒दृगे॒व तन्न प्र॑वस्त॒व्य॑मा॒त्मनो॑ गोपी॒थायै॒ष वै व्या॒घ्रः कु॑लगो॒पो यद॒ग्निस्तस्मा॒द्यद्दी᳚क्षि॒तः प्र॒वसे॒थ्स ए॑नमीश्व॒रो॑\-ऽनू॒त्थाय॒ हन्तो॒र्न प्र॑वस्त॒व्य॑मा॒त्मनो॒ गुप्त्यै॑ दक्षिण॒तः श॑य ए॒तद्वै यज॑मानस्या॒यत॑न॒ꣴ॒ स्व ए॒वायत॑ने शये॒\-ऽग्निम॑भ्या॒वृत्य॑ शये दे॒वता॑ ए॒व य॒ज्ञम॑भ्या॒वृत्य॑ शये॥~(२९)

%6.2.6.0
{\anuvakamend[{ए॒तद्वै क्रू॒र इ॒वैक॑व्रता आ॒त्मना॒ यज॑मानस्य॒ त्रयो॑दश च}]}%~(५)

%6.2.6.1
पु॒रोह॑विषि देव॒यज॑ने याजये॒द्यं का॒मये॒तोपै॑न॒मुत्त॑रो य॒ज्ञो न॑मेद॒भि सु॑व॒र्गं लो॒कं ज॑ये॒दित्ये॒तद्वै पु॒रोह॑विर्देव॒यज॑नं॒ यस्य॒ होता᳚ प्रातरनुवा॒कम॑नुब्रु॒वन्न॒ग्निम॒प आ॑दि॒त्यम॒भि वि॒पश्य॒त्युपै॑न॒मुत्त॑रो य॒ज्ञो न॑मत्य॒भि सु॑व॒र्गं लो॒कं ज॑यत्या॒प्ते दे॑व॒यज॑ने याजये॒द्भ्रातृ॑व्यवन्तं॒ पन्थां᳚ वाधिस्प॒र्॒\mbox{}शये॑त्क॒र्तं वा॒ याव॒न्नान॑से॒ यात॒वै~(३०)

%6.2.6.2
न रथा॑यै॒तद्वा आ॒प्तं दे॑व॒यज॑नमा॒प्नोत्ये॒व भ्रातृ॑व्यं॒ नैन॒म्भ्रातृ॑व्य आप्नो॒त्येको᳚न्नते देव॒यज॑ने याजयेत्प॒शुका॑म॒मेको᳚न्नता॒द्वै दे॑व॒यज॑ना॒दङ्गि॑रसः प॒शून॑सृजन्तान्त॒रा स॑दोहविर्धा॒ने उ॑न्न॒तꣴ स्या॑दे॒तद्वा एको᳚न्नतं देव॒यज॑नं पशु॒माने॒व भ॑वति॒ त्र्यु॑न्नते देव॒यज॑ने याजयेथ्सुव॒र्गका॑म॒न्त्र्यु॑न्नता॒द्वै दे॑व॒यज॑ना॒दङ्गि॑रसः सुव॒र्गं लो॒कमा॑यन्नन्त॒राह॑व॒नीयं॑ च हवि॒र्धानं॑ च~(३१)

%6.2.6.3
उ॒न्न॒तꣴ स्या॑दन्त॒रा ह॑वि॒र्धानं॑ च॒ सद॑श्चान्त॒रा सद॑श्च॒ गार्\mbox{}ह॑पत्यं चै॒तद्वै त्र्यु॑न्नतं देव॒यज॑नꣳ सुव॒र्गमे॒व लो॒कमे॑ति॒ प्रति॑ष्ठिते देव॒यज॑ने याजयेत्प्रति॒ष्ठाका॑ममे॒तद्वै प्रति॑ष्ठितं देव॒यज॑नं॒ यथ्स॒र्वतः॑ स॒मं प्रत्ये॒व ति॑ष्ठति॒ यत्रा॒न्याअ॑न्या॒ ओष॑धयो॒ व्यति॑षक्ताः॒ स्युस्तद्या॑जयेत्प॒शुका॑ममे॒तद्वै प॑शू॒नाꣳ रू॒पꣳ रू॒पेणै॒वास्मै॑ प॒शून्~(३२)

%6.2.6.4
अव॑ रुन्धे पशु॒माने॒व भ॑वति॒ निर्\mbox{}ऋ॑तिगृहीते देव॒यज॑ने याजये॒द्यं का॒मये॑त॒ निर्\mbox{}ऋ॑त्यास्य य॒ज्ञं ग्रा॑हयेय॒मित्ये॒तद्वै निर्\mbox{}ऋ॑तिगृहीतं देव॒यज॑नं॒ यथ्स॒दृश्यै॑ स॒त्या॑ ऋ॒क्षन्निर्\mbox{}ऋ॑त्यै॒वास्य॑ य॒ज्ञं ग्रा॑हयति॒ व्यावृ॑त्ते देव॒यज॑ने याजयेद्व्या॒वृत्का॑मं॒ यं पात्रे॑ वा॒ तल्पे॑ वा॒ मीमाꣳ॑सेरन्प्रा॒चीन॑माहव॒नीया᳚त्प्रव॒णꣴ स्या᳚त्प्रती॒चीनं॒ गार्\mbox{}ह॑पत्यादे॒तद्वै व्यावृ॑त्तं देव॒यज॑नं॒ वि पा॒प्मना॒ भ्रातृ॑व्ये॒णा व॑र्तते॒ नैनं॒ पात्रे॒ न तल्पे॑ मीमाꣳसन्ते का॒र्ये॑ देव॒यज॑ने याजये॒द्भूति॑कामं का॒र्यो॑ वै पुरु॑षो॒ भव॑त्ये॒व॥~(३३)

%6.2.7.0
{\anuvakamend[{यात॒वै ह॑वि॒र्धान॑ञ्च प॒शून्पा॒प्मना॒\-ऽष्टाद॑श च}]}%~(६)

%6.2.7.1
तेभ्य॑ उत्तरवे॒दिः सि॒ꣳ॒ही रू॒पं कृ॒त्वोभया॑नन्त॒राप॒क्रम्या॑तिष्ठ॒त्ते दे॒वा अ॑मन्यन्त यत॒रान् वा इ॒यमु॑पाव॒र्थ्स्यति॒ त इ॒दं भं॑विष्य॒न्तीति॒ तामुपा॑मन्त्रयन्त॒ साब्र॑वी॒द्वरं॑ वृणै॒ सर्वा॒न्मया॒ कामा॒न्व्य॑श्ञवथ॒ पूर्वां तु मा॒\-ऽग्नेराहु॑तिरश्ञवता॒ इति॒ तस्मा॑दुत्तरवे॒दिं पूर्वा॑म॒ग्नेर्व्याघा॑रयन्ति॒ वारे॑वृत॒ꣴ॒ ह्य॑स्यै॒ शम्य॑या॒ परि॑ मिमीते~(३४)

%6.2.7.2
मात्रै॒वास्यै॒ सा\-ऽथो॑ यु॒क्तेनै॒व यु॒क्तमव॑ रुन्धे वि॒त्ताय॑नी मे॒\-ऽसीत्या॑ह वि॒त्ता ह्ये॑ना॒नाव॑त्ति॒क्ताय॑नी मे॒\-ऽसीत्या॑ह ति॒क्तान् ह्ये॑ना॒नाव॒दव॑तान्मा नाथि॒तमित्या॑ह नाथि॒तान् ह्ये॑ना॒नाव॒दव॑तान्मा व्यथि॒तमित्या॑ह व्यथि॒तान् ह्ये॑ना॒नाव॑द्वि॒देर॒ग्निर्नभो॒ नाम॑~(३५)

%6.2.7.3
अग्ने॑ अङ्गिर॒ इति॒ त्रिर्\mbox{}ह॑रति॒ य ए॒वैषु लो॒केष्व॒ग्नय॒स्ताने॒वाव॑ रुन्धे तू॒ष्णीं च॑तु॒र्थꣳ ह॑र॒त्यनि॑रुक्तमे॒वाव॑ रुन्धे सि॒ꣳ॒हीर॑सि महि॒षीर॒सीत्या॑ह सि॒ꣳ॒हीर्\mbox{}ह्ये॑षा रू॒पं कृ॒त्वोभया॑नन्त॒राप॒क्रम्याति॑ष्ठदु॒रु प्र॑थस्वो॒रु ते॑ य॒ज्ञप॑तिः प्रथता॒मित्या॑ह॒ यज॑मानमे॒व प्र॒जया॑ प॒शुभिः॑ प्रथयति ध्रु॒वा~(३६)

%6.2.7.4
अ॒सीति॒ सꣳ ह॑न्ति॒ धृत्यै॑ दे॒वेभ्यः॑ शुन्धस्व दे॒वेभ्यः॑ शुम्भ॒स्वेत्यव॑ चो॒क्षति॒ प्र च॑ किरति॒ शुद्ध्या॑ इन्द्रघो॒षस्त्वा॒ वसु॑भिः पु॒रस्ता᳚त्पा॒त्वित्या॑ह दि॒ग्भ्य ए॒वैनां॒ प्रोक्ष॑ति दे॒वाꣴश्चेदु॑त्तरवे॒दिरु॒पाव॑वर्ती॒हैव वि ज॑यामहा॒ इत्यसु॑रा॒ वज्र॑मु॒द्यत्य॑ दे॒वान॒भ्या॑यन्त॒ तानि॑न्द्रघो॒षो वसु॑भिः पु॒रस्ता॒दप॑~(३७)

%6.2.7.5
अ॒नु॒द॒त॒ मनो॑जवाः पि॒तृभि॑र्दक्षिण॒तः प्रचे॑ता रु॒द्रैः प॒श्चाद्वि॒श्वक॑र्मादि॒त्यैरु॑त्तर॒तो यदे॒वमु॑त्तरवे॒दिं प्रो॒क्षति॑ दि॒ग्भ्य ए॒व तद्यज॑मानो॒ भ्रातृ॑व्या॒न्प्रणु॑दत॒ इन्द्रो॒ यती᳚न्थ्सालावृ॒केभ्यः॒ प्राय॑च्छ॒त्तान्द॑क्षिण॒त उ॑त्तरवे॒द्या आ॑द॒न्॒ यत्प्रोक्ष॑णीनामु॒च्छिष्ये॑त॒ तद्द॑क्षिण॒त उ॑त्तरवे॒द्यै नि न॑ये॒द्यदे॒व तत्र॑ क्रू॒रं तत्तेन॑ शमयति॒ यं द्वि॒ष्यात्तं ध्या॑येच्छु॒चैवैन॑मर्पयति॥~(३८)

%6.2.8.0
{\anuvakamend[{मि॒मी॒ते॒ नाम॑ ध्रु॒वा\-ऽप॑ शु॒चा त्रीणि॑ च}]}%~(७)

%6.2.8.1
सोत्त॑रवे॒दिर॑ब्रवी॒थ्सर्वा॒न्मया॒ कामा॒न्व्य॑श्ञव॒थेति॒ ते दे॒वा अ॑कामय॒न्तासु॑रा॒न्भ्रातृ॑व्यान॒भि भ॑वे॒मेति॒ ते॑\-ऽजुहवुः सि॒ꣳ॒हीर॑सि सपत्नसा॒ही स्वाहेति॒ ते\-ऽसु॑रा॒न्भ्रातृ॑व्यान॒भ्य॑भव॒न्ते\-ऽसु॑रा॒न्भ्रातृ॑व्यानभि॒भूया॑कामयन्त प्र॒जां वि॑न्देम॒हीति॒ ते॑\-ऽजुहवुः सि॒ꣳ॒हीर॑सि सुप्रजा॒वनिः॒ स्वाहेति॒ ते प्र॒जाम॑विन्दन्त॒ ते प्र॒जां वि॒त्त्वा~(३९)

%6.2.8.2
अ॒का॒म॒य॒न्त॒ प॒शून् वि॑न्देम॒हीति॒ ते॑\-ऽजुहवुः सि॒ꣳ॒हीर॑सि रायस्पोष॒वनिः॒ स्वाहेति॒ ते प॒शून॑विन्दन्त॒ ते प॒शून् वि॒त्त्वा\-ऽ का॑मयन्त प्रति॒ष्ठां वि॑न्देम॒हीति॒ ते॑\-ऽजुहवुः सि॒ꣳ॒हीर॑स्यादित्य॒वनिः॒ स्वाहेति॒ त इ॒मां प्र॑ति॒ष्ठाम॑विन्दन्त॒ त इ॒मां प्र॑ति॒ष्ठां वि॒त्त्वाका॑मयन्त दे॒वता॑ आ॒शिष॒ उपे॑या॒मेति॒ ते॑\-ऽजुहवुः सि॒ꣳ॒हीर॒स्या व॑ह दे॒वान्दे॑वय॒ते~(४०)

%6.2.8.3
यज॑मानाय॒ स्वाहेति॒ ते दे॒वता॑ आ॒शिष॒ उपा॑य॒न्पञ्च॒ कृत्वो॒ व्याघा॑रयति॒ पञ्चा᳚क्षरा प॒ङ्क्तिः पाङ्क्तो॑ य॒ज्ञो य॒ज्ञमे॒वाव॑ रुन्धे\-ऽक्ष्ण॒या व्याघा॑रयति॒ तस्मा॑दक्ष्ण॒या प॒शवो\-ऽङ्गा॑नि॒ प्र ह॑रन्ति॒ प्रति॑ष्ठित्यै भू॒तेभ्य॒स्त्वेति॒ स्रुच॒मुद्गृ॑ह्णाति॒ य ए॒व दे॒वा भू॒तास्तेषा॒न्तद्भा॑ग॒धेय॒न्ताने॒व तेन॑ प्रीणाति॒ पौतु॑द्रवान्परि॒धीन्परि॑ दधात्ये॒षाम्~(४१)

%6.2.8.4
लो॒कानां॒ विधृ॑त्या अ॒ग्नेस्त्रयो॒ ज्यायाꣳ॑सो॒ भ्रात॑र आस॒न्ते दे॒वेभ्यो॑ ह॒व्यं वह॑न्तः॒ प्रामी॑यन्त॒ सो᳚\-ऽग्निर॑बिभेदि॒त्थं वाव स्य आर्ति॒मारि॑ष्य॒तीति॒ स निला॑यत॒ स यां वन॒स्पति॒ष्वव॑स॒त्तां पूतु॑द्रौ॒ यामोष॑धीषु॒ ताꣳ सु॑गन्धि॒तेज॑ने॒ यां प॒शुषु॒ तां पेत्व॑स्यान्त॒रा शृङ्गे॒ तं दे॒वताः॒ प्रैष॑मैच्छ॒न्तमन्व॑विन्द॒न्तम॑ब्रुवन्न्~(४२)

%6.2.8.5
उप॑ न॒ आ व॑र्तस्व ह॒व्यं नो॑ व॒हेति॒ सो᳚\-ऽब्रवी॒द्वरं॑ वृणै॒ यदे॒व गृ॑ही॒तस्याहु॑तस्य बहिःपरि॒धि स्कन्दा॒त्तन्मे॒ भ्रातृ॑णां भाग॒धेय॑मस॒दिति॒ तस्मा॒द्यद् गृ॑ही॒तस्याहु॑तस्य बहिःपरि॒धि स्कन्द॑ति॒ तेषा॒न्तद्भा॑ग॒धेयं॒ ताने॒व तेन॑ प्रीणाति॒ सो॑\-ऽमन्यतास्थ॒न्वन्तो॑ मे॒ पूर्वे॒ भ्रात॑रः॒ प्रामे॑षता॒स्थानि॑ शातया॒ इति॒ स यानि॑~(४३)

%6.2.8.6
अ॒स्थान्यशा॑तयत॒ तत्पूतु॑द्र्वभव॒द्यन्मा॒ꣳ॒समुप॑मृतं॒ तद्गुल्गु॑लु॒ यदे॒तान्थ्स॑म्भा॒रान्थ्स॒म्भर॑त्य॒ग्निमे॒व तथ्सम्भ॑रत्य॒ग्नेः पुरी॑षम॒सीत्या॑हा॒ग्नेर्\mbox{}ह्ये॑तत्पुरी॑षं॒ यथ्सं॑भा॒रा अथो॒ खल्वा॑हुरे॒ते वावैनं॒ ते भ्रात॑रः॒ परि॑ शेरे॒ यत्पौतु॑द्रवाः परि॒धय॒ इति॑॥~(४४)

%6.2.9.0
{\anuvakamend[{वि॒त्त्वा दे॑वय॒त ए॒षाम॑ब्रुव॒न्॒ यानि॒ चतु॑श्चत्वारिꣳशच्च}]}%~(८)

%6.2.9.1
ब॒द्धमव॑ स्यति वरुणपा॒शादे॒वैने॑ मुञ्चति॒ प्र णे॑नेक्ति॒ मेध्ये॑ ए॒वैने॑ करोति सावित्रि॒यर्चा हु॒त्वा ह॑वि॒र्धाने॒ प्र व॑र्तयति सवि॒तृप्र॑सूत ए॒वैने॒ प्र व॑र्तयति॒ वरु॑णो॒ वा ए॒ष दु॒र्वागु॑भ॒यतो॑ ब॒द्धो यदक्षः॒ स यदु॒थ्सर्जे॒द्यज॑मानस्य गृ॒हान॒भ्युथ्स॑र्जेथ्सु॒वाग्दे॑व॒ दुर्या॒ꣳ॒ आ व॒देत्या॑ह गृ॒हा वै दुर्याः॒ शान्त्यै॒ पत्नी᳚~(४५)

%6.2.9.2
उपा॑नक्ति॒ पत्नी॒ हि सर्व॑स्य मि॒त्रं मि॑त्र॒त्वाय॒ यद्वै पत्नी॑ य॒ज्ञस्य॑ क॒रोति॑ मिथु॒नं तदथो॒ पत्नि॑या ए॒वैष य॒ज्ञस्या᳚न्वार॒म्भो\-ऽन॑वच्छित्त्यै॒ वर्त्म॑ना॒ वा अ॒न्वित्य॑ य॒ज्ञꣳ रक्षाꣳ॑सि जिघाꣳसन्ति वैष्ण॒वीभ्या॑मृ॒ग्भ्यां वर्त्म॑नोर्जुहोति य॒ज्ञो वै विष्णु॑र्य॒ज्ञादे॒व रक्षा॒ꣴ॒स्यप॑ हन्ति॒ यद॑ध्व॒र्युर॑न॒ग्नावाहु॑तिञ्जुहु॒याद॒न्धो᳚\-ऽध्व॒र्युः स्या॒द्रक्षाꣳ॑सि य॒ज्ञꣳ ह॑न्युः~(४६)

%6.2.9.3
हिर॑ण्यमु॒पास्य॑ जुहोत्यग्नि॒वत्ये॒व जु॑होति॒ नान्धो᳚\-ऽध्व॒र्युर्भव॑ति॒ न य॒ज्ञꣳ रक्षाꣳ॑सि घ्नन्ति॒ प्राची॒ प्रेत॑मध्व॒रं क॒ल्पय॑न्ती॒ इत्या॑ह सुव॒र्गमे॒वैने॑ लो॒कं ग॑मय॒त्यत्र॑ रमेथां॒ वर्ष्म॑न्पृथि॒व्या इत्या॑ह॒ वर्ष्म॒ ह्ये॑तत्पृ॑थि॒व्या यद्दे॑व॒यज॑न॒ꣳ॒ शिरो॒ वा ए॒तद्य॒ज्ञस्य॒ यद्ध॑वि॒र्धान॑न्दि॒वो वा॑ विष्णवु॒त वा॑ पृथि॒व्याः~(४७)

%6.2.9.4
इत्या॒शीर्प॑दय॒र्चा दक्षि॑णस्य हवि॒र्धान॑स्य मे॒थीं नि ह॑न्ति शीर्\mbox{}ष॒त ए॒व य॒ज्ञस्य॒ यज॑मान आ॒शिषो\-ऽव॑ रुन्धे द॒ण्डो वा औ॑प॒रस्तृ॒तीय॑स्य हवि॒र्धान॑स्य वषट्का॒रेणाक्ष॑मच्छिन॒द्यत्तृ॒तीयं॑ छ॒दिर्\mbox{}ह॑वि॒र्धान॑योरुदाह्रि॒यते॑ तृ॒तीय॑स्य हवि॒र्धान॒स्याव॑रुद्ध्यै॒ शिरो॒ वा ए॒तद्य॒ज्ञस्य॒ यद्ध॑वि॒र्धानं॒ विष्णो॑ र॒राट॑मसि॒ विष्णोः᳚ पृ॒ष्ठम॒सीत्या॑ह॒ तस्मा॑देताव॒द्धा शिरो॒ विष्यू॑तं॒ विष्णोः॒ स्यूर॑सि॒ विष्णो᳚र्ध्रु॒वम॒सीत्या॑ह वैष्ण॒वꣳ हि दे॒वत॑या हवि॒र्धानं॒ यं प्र॑थ॒मं ग्र॒न्थिं ग्र॑थ्नी॒याद्यत्तं न वि॑स्र॒ꣳ॒सये॒दमे॑हेनाध्व॒र्युः प्र मी॑येत॒ तस्मा॒थ्स वि॒स्रस्यः॑॥~(४८)

%6.2.10.0
{\anuvakamend[{पत्नी॑ हन्युर्वा पृथि॒व्या विष्यू॑तं॒ विष्णोः॒ षड्विꣳ॑शतिश्च}]}%~(९)

%6.2.10.1
दे॒वस्य॑ त्वा सवि॒तुः प्र॑स॒व इत्यभ्रि॒मा द॑त्ते॒ प्रसू᳚त्या अ॒श्विनो᳚र्बा॒हुभ्या॒मित्या॑हा॒श्विनौ॒ हि दे॒वाना॑मध्व॒र्यू आस्तां᳚ पू॒ष्णो हस्ता᳚भ्या॒मित्या॑ह॒ यत्यै॒ वज्र॑ इव॒ वा ए॒षा यदभ्रि॒रभ्रि॑रसि॒ नारि॑र॒सीत्या॑ह॒ शान्त्यै॒ काण्डे॑काण्डे॒ वै क्रि॒यमा॑णे य॒ज्ञꣳ रक्षाꣳ॑सि जिघाꣳसन्ति॒ परि॑लिखित॒ꣳ॒ रक्षः॒ परि॑लिखिता॒ अरा॑तय॒ इत्या॑ह॒ रक्ष॑सा॒मप॑हत्यै~(४९)

%6.2.10.2
इ॒दम॒हꣳ रक्ष॑सो ग्री॒वा अपि॑ कृन्तामि॒ यो᳚\-ऽस्मान्द्वेष्टि॒ यं च॑ व॒यं द्वि॒ष्म इत्या॑ह॒ द्वौ वाव पुरु॑षौ॒ यं चै॒व द्वेष्टि॒ यश्चै॑नं॒ द्वेष्टि॒ तयो॑रे॒वान॑न्तरायं ग्री॒वाः कृ॑न्तति दि॒वे त्वा॒न्तरि॑क्षाय त्वा पृथि॒व्यै त्वेत्या॑है॒भ्य ए॒वैनाँ᳚ल्लो॒केभ्यः॒ प्रोक्ष॑ति प॒रस्ता॑द॒र्वाचीं॒ प्रोक्ष॑ति॒ तस्मा᳚त्~(५०)

%6.2.10.3
प॒रस्ता॑द॒र्वाचीं᳚ मनु॒ष्या॑ ऊर्ज॒मुप॑ जीवन्ति क्रू॒रमि॑व॒ वा ए॒तत्क॑रोति॒ यत्खन॑त्य॒पो\-ऽव॑ नयति॒ शान्त्यै॒ यव॑मती॒रव॑ नय॒त्यूर्ग्वै यव॒ ऊर्गु॑दु॒म्बर॑ ऊ॒र्जैवोर्ज॒ꣳ॒ सम॑र्धयति॒ यज॑मानेन॒ सम्मि॒तौदु॑म्बरी भवति॒ यावा॑ने॒व यज॑मान॒स्ताव॑तीमे॒वास्मि॒न्नूर्जं॑ दधाति पितृ॒णाꣳ सद॑नम॒सीति॑ ब॒र्॒\mbox{}हिरव॑ स्तृणाति पितृदेव॒त्यम्᳚~(५१)

%6.2.10.4
ह्ये॑तद्यन्निखा॑तं॒ यद्ब॒र्॒\mbox{}हिरन॑वस्तीर्य मिनु॒यात्पि॑तृदेव॒त्या॑ निखा॑ता स्याद्ब॒र्॒\mbox{}हिर॑व॒स्तीर्य॑ मिनोत्य॒स्यामे॒वैनां᳚ मिनो॒त्यथो᳚ स्वा॒रुह॑मे॒वैना᳚ङ्करो॒त्युद्दिवꣴ॑ स्तभा॒नान्तरि॑क्षं पृ॒णेत्या॑है॒षां लो॒कानां॒ विधृ॑त्यै द्युता॒नस्त्वा॑ मारु॒तो मि॑नो॒त्वित्या॑ह द्युता॒नो ह॑ स्म॒ वै मा॑रु॒तो दे॒वाना॒मौदु॑म्बरीं मिनोति॒ तेनै॒व~(५२)

%6.2.10.5
ए॒नां॒ मि॒नो॒ति॒ ब्र॒ह्म॒वनिं᳚ त्वा क्षत्र॒वनि॒मित्या॑ह यथाय॒जुरे॒वैतद् घृ॒तेन॑ द्यावा\-पृथिवी॒ आ पृ॑णेथा॒मित्यौदु॑म्बर्यां जुहोति॒ द्यावा॑पृथि॒वी ए॒व रसे॑नानक्त्या॒न्तम॒न्वव॑स्रावयत्या॒न्तमे॒व यज॑मानं॒ तेज॑सा\-ऽनक्त्यै॒न्द्रम॒सीति॑ छ॒दिरधि॒ नि द॑धात्यै॒न्द्रꣳ हि दे॒वत॑या॒ सदो॑ विश्वज॒नस्य॑ छा॒येत्या॑ह विश्वज॒नस्य॒ ह्ये॑षा छा॒या यथ्सदो॒ नव॑छदि~(५३)

%6.2.10.6
तेज॑स्कामस्य मिनुयात्त्रि॒वृता॒ स्तोमे॑न॒ सम्मि॑त॒न्तेज॑स्त्रि॒वृत्ते॑ज॒स्व्ये॑व भ॑व॒त्येका॑\-दशछदीन्द्रि॒यका॑म॒स्यैका॑\-दशाक्षरा त्रि॒ष्टुगि॑न्द्रि॒यं त्रि॒ष्टुगि॑न्द्रिया॒व्ये॑व भ॑वति॒ पञ्च॑दशछदि॒ भ्रातृ॑व्यवतः पञ्चद॒शो वज्रो॒ भ्रातृ॑व्याभिभूत्यै स॒प्तद॑शछदि प्र॒जाका॑मस्य सप्तद॒शः प्र॒जा\-प॑तिः प्र॒जा\-प॑ते॒राप्त्या॒ एक॑विꣳशतिछदि प्रति॒ष्ठाका॑मस्यैकवि॒ꣳ॒शः स्तोमा॑नां प्रति॒ष्ठा प्रति॑ष्ठित्या उ॒दरं॒ वै सद॒ ऊर्गु॑दु॒म्बरो॑ मध्य॒त औदु॑म्बरीं मिनोति मध्य॒त ए॒व प्र॒जाना॒मूर्जं॑ दधाति॒ तस्मा᳚त्~(५४)

%6.2.10.7
म॒ध्य॒त ऊ॒र्जा भु॑ञ्जते यजमानलो॒के वै दक्षि॑णानि छ॒दीꣳषि॑ भ्रातृव्यलो॒क उत्त॑राणि॒ दक्षि॑णा॒न्युत्त॑राणि करोति॒ यज॑मानमे॒वाय॑जमाना॒दुत्त॑रं करोति॒ तस्मा॒द्यज॑मा॒नो\-ऽय॑जमाना॒दुत्त॑रो\-ऽन्तर्व॒र्तान्क॑रोति॒ व्यावृ॑त्त्यै॒ तस्मा॒दर॑ण्यं प्र॒जा उप॑ जीवन्ति॒ परि॑ त्वा गिर्वणो॒ गिर॒ इत्या॑ह यथाय॒जुरे॒वैतदिन्द्र॑स्य॒ स्यूर॒सीन्द्र॑स्य ध्रु॒वम॒सीत्या॑है॒न्द्रꣳ हि दे॒वत॑या॒ सदो॒ यं प्र॑थ॒मं ग्र॒न्थिं ग्र॑थ्नी॒याद्यत्तं न वि॑स्र॒ꣳ॒सये॒दमे॑हेनाध्व॒र्युः प्र मी॑येत॒ तस्मा॒थ्स वि॒स्रस्यः॑॥~(५)

%6.2.11.0
{\anuvakamend[{अप॑हत्यै॒ तस्मा᳚त्पितृदेव॒त्य॑न्तेनै॒व नव॑छदि॒ तस्मा॒थ्सदः॒ पञ्च॑दश च}]}%॥10॥

%6.2.11.1
शिरो॒ वा ए॒तद्य॒ज्ञस्य॒ यद्ध॑वि॒र्धानं॑ प्रा॒णा उ॑पर॒वा ह॑वि॒र्धाने॑ खायन्ते॒ तस्मा᳚च्छी॒र्॒\mbox{}षन्प्रा॒णा अ॒धस्ता᳚त्खायन्ते॒ तस्मा॑द॒धस्ता᳚च्छी॒र्ष्णः प्रा॒णा र॑क्षो॒हणो॑ वलग॒हनो॑ वैष्ण॒वान्ख॑ना॒मीत्या॑ह वैष्ण॒वा हि दे॒वत॑योपर॒वा असु॑रा॒ वै नि॒र्यन्तो॑ दे॒वानां᳚ प्रा॒णेषु॑ वल॒गान्न्य॑खन॒न्तान्बा॑हुमा॒त्रे\-ऽन्व॑विन्द॒न्तस्मा᳚द्बाहुमा॒त्राः खा॑यन्त इ॒दम॒हं तं व॑ल॒गमुद्व॑पामि~(५६)

%6.2.11.2
यं नः॑ समा॒नो यमस॑मानो निच॒खानेत्या॑ह॒ द्वौ वाव पुरु॑षौ॒ यश्चै॒व स॑मा॒नो यश्चास॑मानो॒ यमे॒वास्मै॒ तौ व॑ल॒गं नि॒खन॑त॒स्तमे॒वोद्व॑पति॒ सं तृ॑णत्ति॒ तस्मा॒थ्सन्तृ॑ण्णा अन्तर॒तः प्रा॒णा न सम्भि॑नत्ति॒ तस्मा॒दस॑म्भिन्नाः प्रा॒णा अ॒पो\-ऽव॑ नयति॒ तस्मा॑दा॒र्द्रा अ॑न्तर॒तः प्रा॒णा यव॑मती॒रव॑ नयति~(५७)

%6.2.11.3
ऊर्ग्वै यवः॑ प्रा॒णा उ॑पर॒वाः प्रा॒णेष्वे॒वोर्जं॑ दधाति ब॒र्॒\mbox{}हिरव॑ स्तृणाति॒ तस्मा᳚ल्लोम॒शा अ॑न्तर॒तः प्रा॒णा आज्ये॑न॒ व्याघा॑रयति॒ तेजो॒ वा आज्यं॑ प्रा॒णा उ॑पर॒वाः प्रा॒णेष्वे॒व तेजो॑ दधाति॒ हनू॒ वा ए॒ते य॒ज्ञस्य॒ यद॑धि॒षव॑णे॒ न सं तृ॑ण॒त्त्यस॑न्तृण्णे॒ हि हनू॒ अथो॒ खलु॑ दीर्घसो॒मे स॒न्तृद्ये॒ धृत्यै॒ शिरो॒ वा ए॒तद्य॒ज्ञस्य॒ यद्ध॑वि॒र्धानम्᳚~(५८)

%6.2.11.4
प्रा॒णा उ॑पर॒वा हनू॑ अधि॒षव॑णे जि॒ह्वा चर्म॒ ग्रावा॑णो॒ दन्ता॒ मुख॑माहव॒नीयो॒ नासि॑कोत्तरवे॒दिरु॒दर॒ꣳ॒ सदो॑ य॒दा खलु॒ वै जि॒ह्वया॑ द॒थ्स्वधि॒ खाद॒त्यथ॒ मुखं॑ गच्छति य॒दा मुखं॒ गच्छ॒त्यथो॒दरं॑ गच्छति॒ तस्मा᳚द्धवि॒र्धाने॒ चर्म॒न्नधि॒ ग्राव॑भिरभि॒षुत्या॑हव॒नीये॑ हु॒त्वा प्र॒त्यञ्चः॑ प॒रेत्य॒ सद॑सि भक्षयन्ति॒ यो वै वि॒राजो॑ यज्ञमु॒खे दोहं॒ वेद॑ दु॒ह ए॒वैना॑मि॒यं वै वि॒राट्तस्यैँ त्वक्चर्मोधो॑\-ऽधि॒षव॑णे॒ स्तना॑ उपर॒वा ग्रावा॑णो व॒थ्सा ऋ॒त्विजो॑ दुहन्ति॒ सोमः॒ पयो॒ य ए॒वं वेद॑ दु॒ह ए॒वैना᳚म्॥~(५९)

%6.3.0.0
{\anuvakamend[{व॒पा॒मि॒ यव॑मती॒रव॑ नयति हवि॒र्धान॑मे॒व त्रयो॑विꣳशतिश्च}]}%॥11॥

%6.3.0.0

{\anuvakamend[{चात्वा॑लाथ्सुव॒र्गाय॒ यद्वै॑सर्ज॒नानि॑ वैष्ण॒व्यर्चा पृ॑थि॒व्यै सा॒ध्या इ॒षे त्वेत्य॒ग्निना॒ पर्य॑ग्नि प॒शोः प॒शुमा॒लभ्य॒ मेद॑सा॒ स्रुचा॒वेका॑\-दश}]}%॥11॥
{\prashnaend{चात्वा॑लाद्दे॒वानु॒पैति॑ मुञ्चति प्रह्रि॒यमा॑णाय॒ पर्य॑ग्नि प॒शुमा॒लभ्य॒ चतु॑ष्पादो॒ द्विष॑ष्टिः॥62॥ चात्वा॑लात्प॒शुषु॑ दधाति॥}}
%%% END PRASHNA

\sect{तृतीयः प्रश्नः}\setcounter{anuvakam}{0}
\dnsub{तैत्तिरीयसंहितायां षष्ठमकाण्डे तृतीयः प्रश्नः}
%6.3.1.0
%6.3.1.1
चात्वा॑ला॒द्धिष्णि॑या॒नुप॑ वपति॒ योनि॒र्वै य॒ज्ञस्य॒ चात्वा॑लं य॒ज्ञस्य॑ सयोनि॒त्वाय॑ दे॒वा वै य॒ज्ञं परा॑जयन्त॒ तमाग्नी᳚ध्रा॒त्पुन॒रपा॑जयन्ने॒तद्वै य॒ज्ञस्याप॑राजितं॒ यदाग्नी᳚ध्रं॒ यदाग्नी᳚ध्रा॒द्धिष्णि॑यान् वि॒हर॑ति॒ यदे॒व य॒ज्ञस्याप॑राजितं॒ तत॑ ए॒वैनं॒ पुन॑स्तनुते परा॒जित्ये॑व॒ खलु॒ वा ए॒ते य॑न्ति॒ ये ब॑हिष्पवमा॒नꣳ सर्प॑न्ति बहिष्पवमा॒ने स्तु॒ते~(१)

%6.3.1.2
आ॒हाग्नी॑द॒ग्नीन् वि ह॑र ब॒र्॒\mbox{}हिः स्तृ॑णाहि पुरो॒डाशा॒ꣳ॒ अलं॑ कु॒र्विति॑ य॒ज्ञमे॒वाप॒जित्य॒ पुन॑स्तन्वा॒ना य॒न्त्यङ्गा॑रै॒र्द्वे सव॑ने॒ वि ह॑रति श॒लाका॑भिस्तृ॒तीयꣳ॑ सशुक्र॒त्वायाथो॒ सम्भ॑रत्ये॒वैन॒द्धिष्णि॑या॒ वा अ॒मुष्मिँ॑ल्लो॒के सोम॑मरक्ष॒न्तेभ्यो\-ऽधि॒ सोम॒माह॑र॒न्तम॑न्व॒वाय॒न्तं पर्य॑विश॒न्॒ य ए॒वं वेद॑ वि॒न्दते᳚~(२)

%6.3.1.3
प॒रि॒वे॒ष्टार॒न्ते सो॑मपी॒थेन॒ व्या᳚र्ध्यन्त॒ ते दे॒वेषु॑ सोमपी॒थमै᳚च्छन्त॒ तां दे॒वा अ॑ब्रुव॒न्द्वेद्वे॒ नाम॑नी कुरुध्व॒मथ॒ प्र वा॒फ्स्यथ॒ न वेत्य॒ग्नयो॒ वा अथ॒ धिष्णि॑या॒स्तस्मा᳚द्द्वि॒नामा᳚ ब्राह्म॒णो\-ऽर्धु॑क॒स्तेषां॒ ये नेदि॑ष्ठं प॒र्यवि॑श॒न्ते सो॑मपी॒थं प्राप्नु॑वन्नाहव॒नीय॑ आग्नी॒ध्रीयो॑ हो॒त्रीयो॑ मार्जा॒लीय॒स्तस्मा॒त्तेषु॑ जुह्वत्यति॒हाय॒ वष॑ट्करोति॒ वि हि~(३)

%6.3.1.4
ए॒ते सो॑मपी॒थेनार्ध्य॑न्त दे॒वा वै याः प्राची॒राहु॑ती॒रजु॑हवु॒र्ये पु॒रस्ता॒दसु॑रा॒ आस॒न्ताꣴस्ताभिः॒ प्राणु॑दन्त॒ याः प्र॒तीची॒र्ये प॒श्चादसु॑रा॒ आस॒न्ताꣴस्ताभि॒रपा॑नुदन्त॒ प्राची॑र॒न्या आहु॑तयो हू॒यन्ते᳚ प्र॒त्यङ्ङासी॑नो॒ धिष्णि॑या॒न्व्याघा॑रयति प॒श्चाच्चै॒व पु॒रस्ता᳚च्च॒ यज॑मानो॒ भ्रातृ॑व्या॒न्प्र णु॑दते॒ तस्मा॒त्परा॑चीः प्र॒जाः प्र वी॑यन्ते प्र॒तीचीः᳚~(४)

%6.3.1.5
जा॒य॒न्ते॒ प्रा॒णा वा ए॒ते यद्धिष्णि॑या॒ यद॑ध्व॒र्युः प्र॒त्यङ्धिष्णि॑यानति॒सर्पे᳚त्प्रा॒णान्थ्सं क॑र्\mbox{}षेत्प्र॒मायु॑कः स्या॒न्नाभि॒र्वा ए॒षा य॒ज्ञस्य॒ यद्धोतो॒र्ध्वः खलु॒ वै नाभ्यै᳚ प्रा॒णो\-ऽवा॑ङपा॒नो यद॑ध्व॒र्युः प्र॒त्यङ्होता॑रमति॒सर्पे॑दपा॒ने प्रा॒णं द॑ध्यात् प्र॒मायु॑कः स्या॒न्नाध्व॒र्युरुप॑ गाये॒द्वाग्वी᳚र्यो॒ वा अ॑ध्व॒र्युर्यद॑ध्व॒र्युरु॑प॒गाये॑दुद्गा॒त्रे~(५)

%6.3.1.6
वाच॒ꣳ॒ सम्प्र य॑च्छेदुप॒दासु॑कास्य॒ वाख्स्या᳚द्ब्रह्मवा॒दिनो॑ वदन्ति॒ नासꣴ॑स्थिते॒ सोमे᳚\-ऽध्व॒र्युः प्र॒त्यङ्ख्सदो\-ऽती॑या॒दथ॑ क॒था दा᳚क्षि॒णानि॒ होतु॑मेति॒ यामो॒ हि स तेषां॒ कस्मा॒ अह॑ दे॒वा यामं॒ वाया॑मं॒ वानु॑ ज्ञास्य॒न्तीत्युत्त॑रे॒णाग्नी᳚ध्रं प॒रीत्य॑ जुहोति दाक्षि॒णानि॒ न प्रा॒णान्थ्सं क॑र्\mbox{}षति॒ न्य॑न्ये धिष्णि॑या उ॒प्यन्ते॒ नान्ये यान्नि॒वप॑ति॒ तेन॒ तान्प्री॑णाति॒ यान्न नि॒वप॑ति॒ यद॑नुदि॒शति॒ तेन॒ तान्॥~(६)

%6.3.2.0
{\anuvakamend[{स्तु॒ते वि॒न्दते॒ हि वी॑यन्ते प्र॒तीची॑रुद्गा॒त्र उ॒प्यन्ते॒ चतु॑र्दश च}]}%~(१)

%6.3.2.1
सु॒व॒र्गाय॒ वा ए॒तानि॑ लो॒काय॑ हूयन्ते॒ यद्वै॑सर्ज॒नानि॒ द्वाभ्यां॒ गार्\mbox{}ह॑पत्ये जुहोति द्वि॒पाद्यज॑मानः॒ प्रति॑ष्ठित्या॒ आग्नी᳚ध्रे जुहोत्य॒न्तरि॑क्ष ए॒वा क्र॑मत आहव॒नीये॑ जुहोति सुव॒र्गमे॒वैनं॑ लो॒कं ग॑मयति दे॒वान् वै सु॑व॒र्गं लो॒कं य॒तो रक्षाꣴ॑स्यजिघाꣳस॒न्ते सोमे॑न॒ राज्ञा॒ रक्षाꣴ॑स्यप॒हत्या॒प्तुमा॒त्मानं॑ कृ॒त्वा सु॑व॒र्गं लो॒कमा॑य॒न्रक्ष॑सा॒मनु॑पलाभा॒यात्तः॒ सोमो॑ भव॒त्यथ॑~(७)

%6.3.2.2
वै॒स॒र्ज॒नानि॑ जुहोति॒ रक्ष॑सा॒मप॑हत्यै॒ त्वꣳ सो॑म तनू॒कृद्भ्य॒ इत्या॑ह तनू॒कृद्ध्ये॑ष द्वेषो᳚भ्यो॒\-ऽन्यकृ॑तेभ्य॒ इत्या॑हा॒न्यकृ॑तानि॒ हि रक्षाꣴ॑स्यु॒रु य॒न्तासि॒ वरू॑थ॒मित्या॑हो॒रु ण॑स्कृ॒धीति॒ वावैतदा॑ह जुषा॒णो अ॒प्तुराज्य॑स्य वे॒त्वित्या॑हा॒प्तुमे॒व यज॑मानं कृ॒त्वा सु॑व॒र्गं लो॒कं ग॑मयति॒ रक्ष॑सा॒मनु॑पलाभा॒या सोमं॑ ददते~(८)

%6.3.2.3
आ ग्राव्ण्ण॒ आ वा॑य॒व्या᳚न्या द्रो॑णकल॒शमुत्पत्नी॒मा न॑य॒न्त्यन्वनाꣳ॑सि॒ प्र व॑र्तयन्ति॒ याव॑दे॒वास्यास्ति॒ तेन॑ स॒ह सु॑व॒र्गं लो॒कमे॑ति॒ नय॑वत्य॒र्चाग्नी᳚ध्रे जुहोति सुव॒र्गस्य॑ लो॒कस्या॒भिनी᳚त्यै॒ ग्राव्ण्णो॑ वाय॒व्या॑नि द्रोणकल॒शमाग्नी᳚ध्र॒ उप॑ वासयति॒ वि ह्ये॑नं॒ तैर्गृ॒ह्णते॒ यथ्स॒होप॑वा॒सये॑दपुवा॒येत॑ सौ॒म्यर्चा प्र पा॑दयति॒ स्वया᳚~(९)

%6.3.2.4
ए॒वैनं॑ दे॒वत॑या॒ प्र पा॑दय॒त्यदि॑त्याः॒ सदो॒\-ऽस्यदि॑त्याः॒ सद॒ आ सी॒देत्या॑ह यथाय॒जुरे॒वैतद्यज॑मानो॒ वा ए॒तस्य॑ पु॒रा गो॒प्ता भ॑वत्ये॒ष वो॑ देव सवितः॒ सोम॒ इत्या॑ह सवि॒तृप्र॑सूत ए॒वैनं॑ दे॒वता᳚भ्यः॒ सम्प्र य॑च्छत्ये॒तत्त्वꣳ सो॑म दे॒वो दे॒वानुपा॑गा॒ इत्या॑ह दे॒वो ह्ये॑ष सन्~(१०)

%6.3.2.5
दे॒वानु॒पैती॒दम॒हं म॑नु॒ष्यो॑ मनु॒ष्या॑नित्या॑ह मनु॒ष्यो  ह्ये॑ष सन्म॑नु॒ष्या॑नु॒पैति॒ यदे॒तद्यजु॒र्न ब्रू॒यादप्र॑जा अप॒शुर्यज॑मानः स्याथ्स॒ह प्र॒जया॑ स॒ह रा॒यस्पोषे॒णेत्या॑ह प्र॒जयै॒व प॒शुभिः॑ स॒हेमं लो॒कमु॒पाव॑र्तते॒ नमो॑ दे॒वेभ्य॒ इत्या॑ह नमस्का॒रो हि दे॒वानाꣴ॑ स्व॒धा पि॒तृभ्य॒ इत्या॑ह स्वधाका॒रो हि~(११)

%6.3.2.6
पि॒तृ॒णामि॒दम॒हं निर्वरु॑णस्य॒ पाशा॒दित्या॑ह वरुणपा॒शादे॒व निर्मु॑च्य॒ते\-ऽग्ने᳚ व्रतपत आ॒त्मनः॒ पूर्वा॑ त॒नूरा॒देयेत्या॑हुः॒ को हि तद्वेद॒ यद्वसी॑या॒न्थ्स्वे वशे॑ भू॒ते पुन॑र्वा॒ ददा॑ति॒ न वेति॒ ग्रावा॑णो॒ वै सोम॑स्य॒ राज्ञो॑ मलिम्लुसे॒ना य ए॒वं वि॒द्वान्ग्राव्ण्ण॒ आग्नी᳚ध्र उपवा॒सय॑ति॒ नैनं॑ मलिम्लुसे॒ना वि॑न्दति॥~(१२)

%6.3.3.0
{\anuvakamend[{अथ॑ ददते॒ स्वया॒ सन्थ्स्व॑धाका॒रो हि वि॑न्दति}]}%~(२)

%6.3.3.1
वै॒ष्ण॒व्यर्चा हु॒त्वा यूप॒मच्छै॑ति वैष्ण॒वो वै दे॒वत॑या॒ यूपः॒ स्वयै॒वैनं॑ दे॒वत॒याच्छै॒त्यत्य॒न्यानगां॒ नान्यानुपा॑गा॒मित्या॒हाति॒ ह्य॑न्यानेति॒ नान्यानु॒पैत्य॒र्वाक्त्वा॒ परै॑रविदं प॒रो\-ऽव॑रै॒रित्या॑हा॒र्वाग्घ्ये॑नं॒ परै᳚र्वि॒न्दति॑ प॒रोव॑रै॒स्तं त्वा॑ जुषे~(१३)

%6.3.3.2
वै॒ष्ण॒वं दे॑वय॒ज्याया॒ इत्या॑ह देवय॒ज्यायै॒ ह्ये॑नं जु॒षते॑ दे॒वस्त्वा॑ सवि॒ता मध्वा॑न॒क्त्वित्या॑ह॒ तेज॑सै॒वैन॑मन॒क्त्योष॑धे॒ त्राय॑स्वैन॒ꣴ॒ स्वधि॑ते॒ मैनꣳ॑ हिꣳसी॒रित्या॑ह॒ वज्रो॒ वै स्वधि॑तिः॒ शान्त्यै॒ स्वधि॑तेर्वृ॒क्षस्य॒ बिभ्य॑तः प्रथ॒मेन॒ शक॑लेन स॒ह तेजः॒ परा॑ पतति॒ यः प्र॑थ॒मः शक॑लः परा॒पते॒त्तमप्या ह॑रे॒थ्सते॑जसम्~(१४)

%6.3.3.3
ए॒वैन॒मा ह॑रती॒मे वै लो॒का यूपा᳚त्प्रय॒तो बि॑भ्यति॒ दिव॒मग्रे॑ण॒ मा ले॑खीर॒न्तरि॑क्षं॒ मध्ये॑न॒ मा हिꣳ॑सी॒रित्या॑है॒भ्य ए॒वैनं॑ लो॒केभ्यः॑ शमयति॒ वन॑स्पते श॒तव॑ल्\mbox{}शो॒ वि रो॒हेत्या॒व्रश्च॑ने जुहोति॒ तस्मा॑दा॒व्रश्च॑नाद्वृ॒क्षाणां॒ भूयाꣳ॑स॒ उत्ति॑ष्ठन्ति स॒हस्र॑वल्\mbox{}शा॒ वि व॒यꣳ रु॑हे॒मेत्या॑हा॒ऽऽशिष॑मे॒वैतामा शा॒स्ते\-ऽन॑क्षसङ्गम्~(१५)

%6.3.3.4
वृ॒श्चे॒द्यद॑क्षस॒ङ्गं वृ॒श्चेद॑धई॒षं यज॑मानस्य प्र॒मायु॑कꣴ स्या॒द्यं का॒मये॒ताप्र॑तिष्ठितः स्या॒दित्या॑रो॒हं तस्मै॑ वृश्चेदे॒ष वै वन॒स्पती॑ना॒मप्र॑तिष्ठि॒तो\-ऽप्र॑तिष्ठित ए॒व भ॑वति॒ यं का॒मये॑ताप॒शुः स्या॒दित्य॑प॒र्णं तस्मै॒ शुष्का᳚ग्रं वृश्चेदे॒ष वै वन॒स्पती॑नामपश॒व्यो॑\-ऽप॒शुरे॒व भ॑वति॒ यं का॒मये॑त पशु॒मान्थ्स्या॒दिति॑ बहुप॒र्णं तस्मै॑ बहुशा॒खं वृ॑श्चेदे॒ष वै~(१६)

%6.3.3.5
वन॒स्पती॑नां पश॒व्यः॑ पशु॒माने॒व भ॑वति॒ प्रति॑ष्ठितं वृश्चेत्प्रति॒ष्ठाका॑मस्यै॒ष वै वन॒स्पती॑नां॒ प्रति॑ष्ठितो॒ यः स॒मे भूम्यै॒ स्वाद्योने॑ रू॒ढः प्रत्ये॒व ति॑ष्ठति॒ यः प्र॒त्यङ्ङुप॑नत॒स्तं वृ॑श्चे॒थ्स हि मेध॑म॒भ्युप॑नतः॒ पञ्चा॑रत्निं॒ तस्मै॑ वृश्चे॒द्यं का॒मये॒तोपै॑न॒मुत्त॑रो य॒ज्ञो न॑मे॒दिति॒ पञ्चा᳚क्षरा प॒ङ्क्तिः पाङ्क्तो॑ य॒ज्ञ उपै॑न॒मुत्त॑रो य॒ज्ञः~(१७)

%6.3.3.6
न॒म॒ति॒ षड॑रत्निं प्रति॒ष्ठाका॑मस्य॒ षड्वा ऋ॒तव॑ ऋ॒तुष्वे॒व प्रति॑ तिष्ठति स॒प्तार॑त्निं प॒शुका॑मस्य स॒प्तप॑दा॒ शक्व॑री प॒शवः॒ शक्व॑री प॒शूने॒वाव॑ रुन्धे॒ नवा॑रत्निं॒ तेज॑स्कामस्य त्रि॒वृता॒ स्तोमे॑न॒ सम्मि॑तं॒ तेज॑स्त्रि॒वृत्ते॑ज॒स्व्ये॑व भ॑व॒त्येका॑\-दशारत्नि\-मिन्द्रि॒यका॑म॒स्यैका॑\-दशाक्षरा त्रि॒ष्टुगि॑न्द्रि॒यं त्रि॒ष्टुगि॑न्द्रिया॒व्ये॑व भ॑वति॒ पञ्च॑दशारत्नि॒म्भ्रातृ॑व्यवतः पञ्चद॒शो वज्रो॒ भ्रातृ॑व्याभिभूत्यै स॒प्तद॑शारत्निं प्र॒जाका॑मस्य सप्तद॒शः प्र॒जा\-प॑तिः प्र॒जा\-प॑ते॒राप्त्या॒ एक॑विꣳशत्यरत्निं प्रति॒ष्ठाका॑मस्यैक\-वि॒ꣳ॒शः स्तोमा॑नां प्रति॒ष्ठा प्रति॑ष्ठित्या अ॒ष्टाश्रि॑र्भवत्य॒ष्टाक्ष॑रा गाय॒त्री तेजो॑ गाय॒त्री गा॑य॒त्री य॑ज्ञमु॒खं तेज॑सै॒व गा॑यत्रि॒या य॑ज्ञमु॒खेन॒ सम्मि॑तः॥~(१८)

%6.3.4.0
{\anuvakamend[{जु॒षे॒ सते॑जस॒मन॑क्षसङ्गं बहुशा॒खं वृ॑श्चेदे॒ष वै य॒ज्ञ उपै॑न॒मुत्त॑रो य॒ज्ञ आप्त्या॒ एका॒न्नविꣳ॑श॒तिश्च॑}]}%~(३)

%6.3.4.1
पृ॒थि॒व्यै त्वा॒न्तरि॑क्षाय त्वा दि॒वे त्वेत्या॑है॒भ्य ए॒वैनं॑ लो॒केभ्यः॒ प्रोक्ष॑ति॒ परा᳚ञ्चं॒ प्रोक्ष॑ति॒ परा॑ङिव॒ हि सु॑व॒र्गो लो॒कः क्रू॒रमि॑व॒ वा ए॒तत्क॑रोति॒ यत्खन॑त्य॒पोव॑ नयति॒ शान्त्यै॒ यव॑मती॒रव॑ नय॒त्यूर्ग्वै यवो॒ यज॑मानेन॒ यूपः॒ सम्मि॑तो॒ यावा॑ने॒व यज॑मान॒स्ताव॑तीमे॒वास्मि॒न्नूर्जं॑ दधाति~(१९)

%6.3.4.2
पि॒तृ॒णाꣳ सद॑नम॒सीति॑ ब॒र्॒\mbox{}हिरव॑ स्तृणाति पितृदेव॒त्य ꣴ॒ ह्ये॑तद्यन्निखा॑तं॒ यद्ब॒र्॒\mbox{}हिरन॑वस्तीर्य मिनु॒यात्पि॑तृदेव॒त्यो॑ निखा॑तः स्याद्ब॒र्॒\mbox{}हिर॑व॒स्तीर्य॑ मिनोत्य॒स्यामे॒वैन॑म्मिनोति यूपशक॒लमवा᳚स्यति॒ सते॑जसमे॒वैन॑म्मिनोति दे॒वस्त्वा॑ सवि॒ता मध्वा॑न॒क्त्वित्या॑ह॒ तेज॑सै॒वैन॑मनक्ति सुपिप्प॒लाभ्य॒स्त्वौष॑धीभ्य॒ इति॑ च॒षालं॒ प्रति॑~(२०)

%6.3.4.3
मु॒ञ्च॒ति॒ तस्मा᳚च्छीर्\mbox{}ष॒त ओष॑धयः॒ फलं॑ गृह्णन्त्य॒नक्ति॒ तेजो॒ वा आज्यं॒ यज॑मानेनाग्नि॒ष्ठाश्रिः॒ सम्मि॑ता॒ यद॑ग्नि॒ष्ठा\-मश्रि॑म॒नक्ति॒ यज॑मानमे॒व तेज॑सानक्त्या॒न्तम॑नक्त्या॒न्तमे॒व यज॑मानं॒ तेज॑सानक्ति स॒र्वतः॒ परि॑ मृश॒त्यप॑रिवर्गमे॒वा\-स्मि॒न्तेजो॑ दधा॒त्युद्दिवꣴ॑ स्तभा॒नान्तरि॑क्षं पृ॒णेत्या॑है॒षां लो॒कानां॒ विधृ॑त्यै वैष्ण॒व्यर्चा~(२१)

%6.3.4.4
क॒ल्प॒य॒ति॒ वै॒ष्ण॒वो वै दे॒वत॑या॒ यूपः॒ स्वयै॒वैनं॑ दे॒वत॑या कल्पयति॒ द्वा\-भ्यां᳚ कल्पयति द्वि॒पाद्यज॑मानः॒ प्रति॑ष्ठित्यै॒ यं का॒मये॑त॒ तेज॑सैनं दे॒वता॑भिरिन्द्रि॒येण॒ व्य॑र्धयेय॒मित्य॑ग्नि॒ष्ठां तस्याश्रि॑माहव॒नीया॑दि॒त्थं वे॒त्थं वाति॑ नावये॒त्तेज॑सै॒वैनं॑ दे॒वता॑भिरिन्द्रि॒येण॒ व्य॑र्धयति॒ यं का॒मये॑त॒ तेज॑सैनं दे॒वता॑भिरिन्द्रि॒येण॒ सम॑र्धयेय॒मिति॑~(२२)

%6.3.4.5
अ॒ग्नि॒ष्ठां तस्याश्रि॑माहव॒नीये॑न॒ सम्मि॑नुया॒त्तेज॑सै॒वैनं॑ दे॒वता॑भिरिन्द्रि॒येण॒ सम॑र्धयति ब्रह्म॒वनिं॑ त्वा क्षत्र॒वनि॒मित्या॑ह यथाय॒जुरे॒वैतत्परि॑ व्यय॒त्यूर्ग्वै र॑श॒ना यज॑मानेन॒ यूपः॒ सम्मि॑तो॒ यज॑मानमे॒वोर्जा सम॑र्धयति नाभिद॒घ्ने परि॑ व्ययति नाभिद॒घ्न ए॒वास्मा॒ ऊर्जं॑ दधाति॒ तस्मा᳚न्नाभिद॒घ्न ऊ॒र्जा भु॑ञ्जते॒ यं का॒मये॑तो॒र्जैनम्᳚~(२३)

%6.3.4.6
व्य॑र्धयेय॒मित्यू॒र्ध्वां वा॒ तस्यावा॑चीं॒ वावो॑हेदू॒र्जैवैनं॒ व्य॑र्धयति॒ यदि॑ का॒मये॑त॒ वर्\mbox{}षु॑कः प॒र्जन्यः॑ स्या॒दित्यवा॑ची॒मवो॑हे॒\-द्वृष्टि॑मे॒व नि य॑च्छति॒ यदि॑ का॒मये॒ताव॑र्\mbox{}षुकः स्या॒दित्यू॒र्ध्वामुदू॑हे॒द्वृष्टि॑मे॒वोद्य॑च्छति पितृ॒णां निखा॑तं मनु॒ष्या॑णामू॒र्ध्वं निखा॑ता॒दा र॑श॒नाया॒ ओष॑धीनाꣳ रश॒ना विश्वे॑षाम्~(२४)

%6.3.4.7
दे॒वाना॑मू॒र्ध्वꣳ र॑श॒नाया॒ आ च॒षाला॒दिन्द्र॑स्य च॒षालꣳ॑ सा॒ध्याना॒मति॑रिक्त॒ꣳ॒ स वा ए॒ष स॑र्वदेव॒त्यो॑ यद्यूपो॒ यद्यूपं॑ मि॒नोति॒ सर्वा॑ ए॒व दे॒वताः᳚ प्रीणाति य॒ज्ञेन॒ वै दे॒वाः सु॑व॒र्गं लो॒कमा॑य॒न्ते॑\-ऽमन्यन्त मनु॒ष्या॑ नो॒\-ऽन्वाभ॑विष्य॒न्तीति॒ ते यूपे॑न योपयि॒त्वा सु॑व॒र्गं लो॒कमा॑य॒न्तमृष॑यो॒ यूपे॑नै॒वानु॒ प्राजा॑न॒न्तद्यूप॑स्य यूप॒त्वम्~(२५)

%6.3.4.8
यद्यूपं॑ मि॒नोति॑ सुव॒र्गस्य॑ लो॒कस्य॒ प्रज्ञा᳚त्यै पु॒रस्ता᳚न्मिनोति पु॒रस्ता॒द्धि य॒ज्ञस्य॑ प्रज्ञा॒यते\-ऽप्र॑ज्ञात॒ꣳ॒ हि तद्यदति॑पन्न आ॒हुरि॒दं का॒र्य॑मासी॒दिति॑ सा॒ध्या वै दे॒वा य॒ज्ञमत्य॑मन्यन्त॒ तान् य॒ज्ञो नास्पृ॑श॒त्तान् यद्य॒ज्ञस्याति॑रिक्त॒मासी॒त्तद॑स्पृश॒\-दति॑रिक्तं॒ वा ए॒तद्य॒ज्ञस्य॒ यद॒ग्नाव॒ग्निं म॑थि॒त्वा प्र॒हर॒त्यति॑रिक्तमे॒तत्~(२६)

%6.3.4.9
यूप॑स्य॒ यदू॒र्ध्वं च॒षाला॒त्तेषां॒ तद्भा॑ग॒धेयं॒ ताने॒व तेन॑ प्रीणाति दे॒वा वै सꣴस्थि॑ते॒ सोमे॒ प्र स्रुचोह॑र॒न्प्र यूपं॒ ते॑\-ऽमन्यन्त यज्ञवेश॒सं वा इ॒दं कु॑र्म॒ इति॒ ते प्र॑स्त॒रꣴ स्रु॒चान्नि॒ष्क्रय॑णमपश्य॒न्थ्स्वरुं॒ यूप॑स्य॒ सꣴस्थि॑ते॒ सोमे॒ प्र प्र॑स्त॒रꣳ हर॑ति जु॒होति॒ स्व॒रुमय॑ज्ञवेशसाय॥~(२७)

%6.3.5.0
{\anuvakamend[{द॒धा॒ति॒ प्रत्यृ॒चा सम॑र्धयेय॒मित्यू॒र्जैनं॒ विश्वे॑षां यूप॒त्वमति॑रिक्तमे॒तद्द्विच॑त्वारिꣳशच्च}]}%~(४)

%6.3.5.1
सा॒ध्या वै दे॒वा अ॒स्मिँल्लो॒क आ॑स॒न्नान्यत्किञ्च॒न मि॒षत्ते᳚\-ऽग्निमे॒वाग्नये॒ मेधा॒याल॑भन्त॒ न ह्य॑न्यदा॑ल॒म्भ्य॑मवि॑न्द॒न्ततो॒ वा इ॒माः प्र॒जाः प्राजा॑यन्त॒ यद॒ग्नाव॒ग्निं म॑थि॒त्वा प्र॒हर॑ति प्र॒जानां᳚ प्र॒जन॑नाय रु॒द्रो वा ए॒ष यद॒ग्निर्यज॑मानः प॒शुर्यत्प॒शुमा॒लभ्या॒ग्निं मन्थे᳚द्रु॒द्राय॒ यज॑मानम्~(२८)

%6.3.5.2
अपि॑ दध्यात्प्र॒मायु॑कः स्या॒दथो॒ खल्वा॑हुर॒ग्निः सर्वा॑ दे॒वता॑ ह॒विरे॒तद्यत्प॒शुरिति॒ यत्प॒शुमा॒लभ्या॒ग्निं मन्थ॑ति ह॒व्यायै॒वास॑न्नाय॒ सर्वा॑ दे॒वता॑ जनयत्युपा॒कृत्यै॒व मन्थ्य॒स्तन्नेवाल॑ब्धं॒ नेवाना॑लब्धम॒ग्नेर्ज॒नित्र॑म॒सीत्या॑हा॒ग्नेर्\mbox{}ह्ये॑तज्ज॒नित्रं॒ वृष॑णौ स्थ॒ इत्या॑ह॒ वृष॑णौ~(२९)

%6.3.5.3
ह्ये॑तावु॒र्वश्य॑स्या॒युर॒सीत्या॑ह मिथुन॒त्वाय॑ घृ॒तेना॒क्ते वृष॑णं दधाथा॒मित्या॑ह॒ वृष॑ण॒ꣴ॒ ह्ये॑ते दधा॑ते॒ ये अ॒ग्निङ्गा॑य॒त्रं छन्दो\-ऽनु॒ प्र जा॑य॒स्वेत्या॑ह॒ छन्दो॑भिरे॒वैनं॒ प्र ज॑नयत्य॒ग्नये॑ म॒थ्यमा॑ना॒यानु॑ ब्रू॒हीत्या॑ह सावि॒त्रीमृच॒मन्वा॑ह सवि॒तृप्र॑सूत ए॒वैनं॑ मन्थति जा॒ताया॑नु ब्रूहि~(३०)

%6.3.5.4
प्र॒ह्रि॒यमा॑णा॒यानु॑ ब्रू॒हीत्या॑ह॒ काण्डे॑काण्ड ए॒वैनं॑ क्रियमा॑णे॒ सम॑र्धयति गाय॒त्रीः सर्वा॒ अन्वा॑ह गाय॒त्रछ॑न्दा॒ वा अ॒ग्निः स्वेनै॒वैनं॒ छन्द॑सा॒ सम॑र्धयत्य॒ग्निः पु॒रा भव॑त्य॒ग्निं म॑थि॒त्वा प्र ह॑रति॒ तौ स॒म्भव॑न्तौ॒ यज॑मानम॒भि सम्भ॑वतो॒ भव॑तं नः॒ सम॑नसा॒वित्या॑ह॒ शान्त्यै᳚ प्र॒हृत्य॑ जुहोति जा॒तायै॒वास्मा॒ अन्न॒मपि॑ दधा॒त्याज्ये॑न जुहोत्ये॒तद्वा अ॒ग्नेः प्रि॒यं धाम॒ यदाज्यं॑ प्रि॒येणै॒वैनं॒ धाम्ना॒ सम॑र्धय॒त्यथो॒ तेज॑सा॥~(३१)

%6.3.6.0
{\anuvakamend[{यज॑मानमाह॒ वृष॑णौ जाता॒यानु॑ब्रू॒ह्यप्य॒ष्टाद॑श च}]}%~(५)

%6.3.6.1
इ॒षे त्वेति॑ ब॒र्॒\mbox{}हिरा द॑त्त इ॒च्छत॑ इव॒ ह्ये॑ष यो यज॑त उप॒वीर॒सीत्या॒होप॒ ह्ये॑नानाक॒रोत्युपो॑ दे॒वान्दैवी॒र्विशः॒ प्रागु॒रित्या॑ह॒ दैवी॒र्॒\mbox{}ह्ये॑ता विशः॑ स॒तीर्दे॒वानु॑प॒यन्ति॒ वह्नी॑रु॒शिज॒ इत्या॑ह॒र्त्विजो॒ वै वह्न॑य उ॒शिज॒स्तस्मा॑दे॒वमा॑ह॒ बृह॑स्पते धा॒रया॒ वसू॒नीति॑~(३२)

%6.3.6.2
आ॒ह॒ ब्रह्म॒ वै दे॒वानां॒ बृह॒स्पति॒र्ब्रह्म॑णै॒वास्मै॑ प॒शूनव॑ रुन्धे ह॒व्या ते᳚ स्वदन्ता॒मित्या॑ह स्व॒दय॑त्ये॒वैना॒न्देव॑ त्वष्ट॒र्वसु॑ र॒ण्वेत्या॑ह॒ त्वष्टा॒ वै प॑शू॒नां मि॑थु॒नानाꣳ॑ रूप॒कृद्रू॒पमे॒व प॒शुषु॑ दधाति॒ रेव॑ती॒ रम॑ध्व॒मित्या॑ह प॒शवो॒ वै रे॒वतीः᳚ प॒शूने॒वास्मै॑ रमयति दे॒वस्य॑ त्वा सवि॒तुः प्र॑स॒व इति॑~(३३)

%6.3.6.3
र॒श॒नामा द॑त्ते॒ प्रसू᳚त्या अ॒श्विनो᳚र्बा॒हुभ्या॒मित्या॑हा॒श्विनौ॒ हि दे॒वाना॑मध्व॒र्यू आस्तां᳚ पू॒ष्णो हस्ता᳚भ्या॒मित्या॑ह॒ यत्या॑ ऋ॒तस्य॑ त्वा देवहविः॒ पाशे॒ना र॑भ॒ इत्या॑ह स॒त्यं वा ऋ॒तꣳ स॒त्येनै॒वैन॑मृ॒तेना र॑भते\-ऽक्ष्ण॒या परि॑ हरति॒ वध्य॒ꣳ॒ हि प्र॒त्यञ्चं॑ प्रतिमु॒ञ्चन्ति॒ व्यावृ॑त्त्यै॒ धर्\mbox{}षा॒ मानु॑षा॒निति॒ नि यु॑नक्ति॒ धृत्या॑ अ॒द्भ्यः~(३४)

%6.3.6.4
त्वौष॑धीभ्यः॒ प्रोक्षा॒मीत्या॑हा॒द्भ्यो ह्ये॑ष ओष॑धीभ्यः स॒म्भव॑ति॒ यत्प॒शुर॒पां पे॒रुर॒सीत्या॑है॒ष ह्य॑पां पा॒ता यो मेधा॑यार॒भ्यते᳚ स्वा॒त्तं चि॒थ्सदे॑वꣳ ह॒व्यमापो॑ देवीः॒ स्वद॑तैन॒मित्या॑ह स्व॒दय॑त्ये॒वैन॑मु॒परि॑ष्टा॒त्प्रोक्ष॑त्यु॒परि॑ष्टादे॒वैनं॒ मेध्यं॑ करोति पा॒यय॑त्यन्तर॒त ए॒वैनं॒ मेध्यं॑ करोत्य॒धस्ता॒दुपो᳚क्षति स॒र्वत॑ ए॒वैनं॒ मेध्यं॑ करोति॥~(३५)

%6.3.7.0
{\anuvakamend[{वसू॒निति॑ प्रस॒व इत्य॒द्भ्यो᳚\-ऽन्तर॒त ए॒वैन॒न्दश॑ च}]}%~(६)

%6.3.7.1
अ॒ग्निना॒ वै होत्रा॑ दे॒वा असु॑रान॒भ्य॑भवन्न॒ग्नये॑ समि॒ध्यमा॑ना॒यानु॑ ब्रू॒हीत्या॑ह॒ भ्रातृ॑व्याभिभूत्यै स॒प्तद॑श सामिधे॒नीरन्वा॑ह सप्तद॒शः प्र॒जा\-प॑तिः प्र॒जा\-प॑ते॒राप्त्यै॑ स॒प्तद॒शान्वा॑ह॒ द्वाद॑श॒ मासाः॒ पञ्च॒र्तवः॒ स सं॑वथ्स॒रः सं॑वथ्स॒रं प्र॒जा अनु॒ प्र जा॑यन्ते प्र॒जानां᳚ प्र॒जन॑नाय दे॒वा वै सा॑मिधे॒नीर॒नूच्य॑ य॒ज्ञं नान्व॑पश्य॒न्थ्स प्र॒जा\-प॑तिस्तू॒ष्णीमा॑घा॒रम्~(३६)

%6.3.7.2
आघा॑रय॒त्ततो॒ वै दे॒वा य॒ज्ञमन्व॑पश्य॒न्॒ यत्तू॒ष्णीमा॑घा॒रमा॑घा॒रय॑ति य॒ज्ञस्यानु॑ख्यात्या॒ असु॑रेषु॒ वै य॒ज्ञ आ॑सी॒त्तं दे॒वास्तू᳚ष्णीꣳहो॒मेना॑वृञ्जत॒ यत्तू॒ष्णीमा॑घा॒रमा॑घा॒रय॑ति॒ भ्रातृ॑व्यस्यै॒व तद्य॒ज्ञं वृ॑ङ्क्ते परि॒धीन्थ्सम्मा᳚र्ष्टि पु॒नात्ये॒वैना॒न्त्रिस्त्रिः॒ सम्मा᳚र्ष्टि॒ त्र्या॑वृ॒द्धि य॒ज्ञो\-ऽथो॒ रक्ष॑सा॒मप॑हत्यै॒ द्वाद॑श॒ सम्प॑द्यन्ते॒ द्वाद॑श~(३७)

%6.3.7.3
मासाः᳚ संवथ्स॒रः सं॑वथ्स॒रमे॒व प्री॑णा॒त्यथो॑ संवथ्स॒रमे॒वास्मा॒ उप॑ दधाति सुव॒र्गस्य॑ लो॒कस्य॒ सम॑ष्ट्यै॒ शिरो॒ वा ए॒तद्य॒ज्ञस्य॒ यदा॑घा॒रो᳚\-ऽग्निः सर्वा॑ दे॒वता॒ यदा॑घा॒रमा॑घा॒रय॑ति शीर्\mbox{}ष॒त ए॒व य॒ज्ञस्य॒ यज॑मानः॒ सर्वा॑ दे॒वता॒ अव॑ रुन्धे॒ शिरो॒ वा ए॒तद्य॒ज्ञस्य॒ यदा॑घा॒र आ॒त्मा प॒शुरा॑घा॒रमा॒घार्य॑ प॒शुꣳ सम॑नक्त्या॒त्मन्ने॒व य॒ज्ञस्य॑~(३८)

%6.3.7.4
शिरः॒ प्रति॑ दधाति॒ सं ते᳚ प्रा॒णो वा॒युना॑ गच्छता॒मित्या॑ह वायुदेव॒त्यो॑ वै प्रा॒णो वा॒यावे॒वास्य॑ प्रा॒णं जु॑होति॒ सं यज॑त्रै॒रङ्गा॑नि॒ सं य॒ज्ञप॑तिरा॒शिषेत्या॑ह य॒ज्ञप॑तिमे॒वास्या॒ऽऽशिषं॑ गमयति वि॒श्वरू॑पो॒ वै त्वा॒ष्ट्र उ॒परि॑ष्टात्प॒शुम॒भ्य॑वमी॒त्तस्मा॑दु॒परि॑ष्टात्प॒शोर्नाव॑ द्यन्ति॒ यदु॒परि॑ष्टात्प॒शुꣳ स॑म॒नक्ति॒ मेध्य॑मे॒व~(३९)

%6.3.7.5
ए॒नं॒ क॒रो॒त्यृ॒त्विजो॑ वृणीते॒ छन्दाꣴ॑स्ये॒व वृ॑णीते स॒प्त वृ॑णीते स॒प्त ग्रा॒म्याः प॒शवः॑ स॒प्तार॒ण्याः स॒प्त छन्दाꣴ॑स्यु॒भय॒स्याव॑रुद्ध्या॒ एका॑\-दश प्रया॒जान् य॑जति॒ दश॒ वै प॒शोः प्रा॒णा आ॒त्मैका॑द॒शो यावा॑ने॒व प॒शुस्तं प्र य॑जति व॒पामेकः॒ परि॑ शय आ॒त्मैवात्मानं॒ परि॑ शये॒ वज्रो॒ वै स्वधि॑ति॒र्वज्रो॑ यूपशक॒लो घृ॒तं खलु॒ वै दे॒वा वज्रं॑ कृ॒त्वा सोम॑मघ्नन्घृ॒तेना॒क्तौ प॒शं त्रा॑येथा॒मित्या॑ह॒ वज्रे॑णै॒वैनं॒ वशे॑ कृ॒त्वा ल॑भते॥~(४०)

%6.3.8.0
{\anuvakamend[{आ॒घा॒रं प॑द्यन्ते॒ द्वाद॑शा॒त्मन्ने॒व य॒ज्ञस्य॒ मेध्य॑मे॒व खलु॒ वा अ॒ष्टाद॑श च}]}%~(७)

%6.3.8.1
पर्य॑ग्नि करोति सर्व॒हुत॑मे॒वैनं॑ करो॒त्यस्क॑न्दा॒यास्क॑न्न॒ꣳ॒ हि तद्यद्धु॒तस्य॒ स्कन्द॑ति॒ त्रिः पर्य॑ग्नि करोति॒ त्र्या॑वृ॒द्धि य॒ज्ञो\-ऽथो॒ रक्ष॑सा॒मप॑हत्यै ब्रह्मवा॒दिनो॑ वदन्त्यन्वा॒रभ्यः॑ प॒शू~(३) र्नान्वा॒रभ्या~(३) इति॑ मृ॒त्यवे॒ वा ए॒ष नी॑यते॒ यत्प॒शुस्तं यद॑न्वा॒रभे॑त प्र॒मायु॑को॒ यज॑मानः स्या॒दथो॒ खल्वा॑हुः सुव॒र्गाय॒ वा ए॒ष लो॒काय॑ नीयते॒ यत्~(४१)

%6.3.8.2
प॒शुरिति॒ यन्नान्वा॒रभे॑त सुव॒र्गाल्लो॒काद्यज॑मानो हीयेत वपा॒श्रप॑णीभ्याम॒न्वार॑भते॒ तन्नेवा॒न्वार॑ब्धं॒ नेवान॑न्वारब्ध॒मुप॒ प्रेष्य॑ होतर्\mbox{}ह॒व्या दे॒वेभ्य॒ इत्या॑हेषि॒तꣳ हि कर्म॑ क्रि॒यते॒ रेव॑तीर्य॒ज्ञप॑तिं प्रिय॒धा वि॑श॒तेत्या॑ह यथाय॒जुरे॒वैतद॒ग्निना॑ पु॒रस्ता॑देति॒ रक्ष॑सा॒मप॑हत्यै पृथि॒व्याः स॒म्पृचः॑ पा॒हीति॑ ब॒र्॒\mbox{}हिः~(४२)

%6.3.8.3
उपा᳚स्य॒त्यस्क॑न्दा॒यास्क॑न्न॒ꣳ॒ हि तद्यद्ब॒र्॒\mbox{}हिषि॒ स्कन्द॒त्यथो॑ बर्\mbox{}हि॒षद॑मे॒वैनं॑ करोति॒ परा॒ङा व॑र्तते\-ऽध्व॒र्युः प॒शोः सं᳚ज्ञ॒प्यमा॑नात्प॒शुभ्य॑ ए॒व तन्नि ह्नु॑त आ॒त्मनोना᳚व्रस्काय॒ गच्छ॑ति॒ श्रियं प्र प॒शूना᳚प्नोति॒ य ए॒वं वेद॑ प॒श्चाल्लो॑का॒ वा ए॒षा प्राच्यु॒दानी॑यते॒ यत्पत्नी॒ नम॑स्त आता॒नेत्या॑हादि॒त्यस्य॒ वै र॒श्मयः॑~(४३)

%6.3.8.4
आ॒ता॒नास्तेभ्य॑ ए॒व नम॑स्करोत्यन॒र्वा प्रेहीत्या॑ह॒ भ्रातृ॑व्यो॒ वा अर्वा॒ भ्रातृ॑व्यापनुत्त्यै घृ॒तस्य॑ कु॒ल्यामनु॑ स॒ह प्र॒जया॑ स॒ह रा॒यस्पोषे॒णेत्या॑हा॒ऽऽशिष॑मे॒वैतामा शा᳚स्त॒ आपो॑ देवीः शुद्धायुव॒ इत्या॑ह यथाय॒जुरे॒वैतत्॥~(४४)

%6.3.9.0
{\anuvakamend[{लो॒काय॑ नीयते॒ यद्ब॒र्॒\mbox{}ही र॒श्मयः॑ स॒प्तत्रिꣳ॑शच्च}]}%~(८)

%6.3.9.1
प॒शोर्वा आल॑ब्धस्य प्रा॒णाञ्छुगृ॑च्छति॒ वाक्त॒ आ प्या॑यतां प्रा॒णस्त॒ आ प्या॑यता॒मित्या॑ह प्रा॒णेभ्य॑ ए॒वास्य॒ शुचꣳ॑ शमयति॒ सा प्रा॒णेभ्यो\-ऽधि॑ पृथि॒वीꣳ शुक्प्र वि॑शति॒ शमहो᳚भ्या॒मिति॒ नि न॑यत्यहोरा॒त्राभ्या॑मे॒व पृ॑थि॒व्यै शुचꣳ॑ शमय॒त्योष॑धे॒ त्राय॑स्वैन॒ꣴ॒ स्वधि॑ते॒ मैनꣳ॑ हिꣳसी॒रित्या॑ह॒ वज्रो॒ वै स्वधि॑तिः~(४५)

%6.3.9.2
शान्त्यै॑ पार्श्व॒त आच्छ्य॑ति मध्य॒तो हि म॑नु॒ष्या॑ आ॒च्छ्यन्ति॑ तिर॒श्चीन॒मा च्छ्य॑त्यनू॒चीन॒ꣳ॒ हि म॑नु॒ष्या॑ आ॒च्छ्यन्ति॒ व्यावृ॑त्त्यै॒ रक्ष॑सां भा॒गो॑\-ऽसीति॑ स्थविम॒तो ब॒र्॒\mbox{}हिर॒क्त्वापा᳚स्यत्य॒स्नैव रक्षाꣳ॑सि नि॒रव॑दयत इ॒दम॒हꣳ रक्षो॑\-ऽध॒मं तमो॑ नयामि॒ यो᳚\-ऽस्मान्द्वेष्टि॒ यं च॑ व॒यं द्वि॒ष्म इत्या॑ह॒ द्वौ वाव पुरु॑षौ॒ यं चै॒व~(४६)

%6.3.9.3
द्वे॒ष्टि॒ यश्चै॑नं॒ द्वेष्टि॒ तावु॒भाव॑ध॒मं तमो॑ नयती॒षे त्वेति॑ व॒पामुत्खि॑दती॒च्छत॑ इव॒ ह्ये॑ष यो यज॑ते॒ यदु॑पतृ॒न्द्याद्रु॒द्रो᳚\-ऽस्य प॒शून्घातु॑कः स्या॒द्यन्नोप॑तृ॒न्द्यादय॑ता स्याद॒न्ययो॑पतृ॒णत्त्य॒न्यया॒ न धृत्यै॑ घृ॒तेन॑ द्यावा\-पृथिवी॒ प्रोर्ण्वा॑था॒मित्या॑ह॒ द्यावा॑पृथि॒वी ए॒व रसे॑नान॒क्त्यछि॑न्नः~(४७)

%6.3.9.4
रायः॑ सु॒वीर॒ इत्या॑ह यथाय॒जुरे॒वैतत्क्रू॒रमि॑व॒ वा ए॒तत्क॑रोति॒ यद्व॒पामु॑त्खि॒दत्यु॒र्व॑न्तरि॑क्ष॒मन्वि॒हीत्या॑ह॒ शान्त्यै॒ प्र वा ए॒षो᳚\-ऽस्माल्लो॒काच्च्य॑वते॒ यः प॒शुं मृ॒त्यवे॑ नी॒यमा॑नमन्वा॒रभ॑ते वपा॒श्रप॑णी॒ पुन॑र॒न्वार॑भते॒\-ऽस्मिन्ने॒व लो॒के प्रति॑ तिष्ठत्य॒ग्निना॑ पु॒रस्ता॑देति॒ रक्ष॑सा॒मप॑हत्या॒ अथो॑ दे॒वता॑ ए॒व ह॒व्येन॑~(४८)

%6.3.9.5
अन्वे॑ति॒ नान्त॒ममङ्गा॑र॒मति॑ हरे॒द्यद॑न्त॒ममङ्गा॑रमति॒हरे᳚द्दे॒वता॒ अति॑ मन्येत॒ वायो॒ वीहि॑ स्तो॒काना॒मित्या॑ह॒ तस्मा॒द्विभ॑क्ताः स्तो॒का अव॑ पद्य॒न्ते\-ऽग्रं॒ वा ए॒तत्प॑शू॒नां यद्व॒पाग्र॒मोष॑धीनां ब॒र्॒\mbox{}हिरग्रे॑णै॒वाग्र॒ꣳ॒ सम॑र्धय॒त्यथो॒ ओष॑धीष्वे॒व प॒शून्प्रति॑\-ष्ठापयति॒ स्वाहा॑कृतीभ्यः॒ प्रेष्येत्या॑ह~(४९)

%6.3.9.6
य॒ज्ञस्य॒ समि॑ष्ट्यै प्राणापा॒नौ वा ए॒तौ प॑शू॒नां यत्पृ॑षदा॒ज्यमा॒त्मा व॒पा पृ॑षदा॒ज्यम॑भि॒घार्य॑ व॒पाम॒भि घा॑रयत्या॒त्मन्ने॒व प॑शू॒नां प्रा॑णापा॒नौ द॑धाति॒ स्वाहो॒र्ध्वन॑भसम्मारु॒तं ग॑च्छत॒मित्या॑हो॒र्ध्वन॑भा ह स्म॒ वै मा॑रु॒तो दे॒वानां᳚ वपा॒श्रप॑णी॒ प्रह॑रति॒ तेनै॒वैने॒ प्र ह॑रति॒ विषू॑ची॒ प्र ह॑रति॒ तस्मा॒द्विष्व॑ञ्चौ प्राणापा॒नौ॥~(५०)

%6.3.10.0
{\anuvakamend[{स्वधि॑तिश्चै॒वाच्छि॑न्नो ह॒व्येने॒ष्येत्या॑ह॒ षट्च॑त्वारिꣳशच्च}]}%~(९)

%6.3.10.1
प॒शुमा॒लभ्य॑ पुरो॒डाशं॒ निर्व॑पति॒ समे॑धमे॒वैन॒मा ल॑भते व॒पया᳚ प्र॒चर्य॑ पुरो॒डाशे॑न॒ प्र च॑र॒त्यूर्ग्वै पु॑रो॒डाश॒ ऊर्ज॑मे॒व प॑शू॒नां म॑ध्य॒तो द॑धा॒त्यथो॑ प॒शोरे॒व छि॒द्रमपि॑ दधाति पृषदा॒ज्यस्यो॑प॒हत्य॒ त्रिः पृ॑च्छति शृ॒तꣳ ह॒वीः~(३) श॑मित॒रिति॒ त्रिष॑त्या॒ हि दे॒वा यो\-ऽशृ॑तꣳ शृ॒तमाह॒ स एन॑सा प्राणापा॒नौ वा ए॒तौ प॑शू॒नाम्~(५१)

%6.3.10.2
यत्पृ॑षदा॒ज्यं प॒शोः खलु॒ वा आल॑ब्धस्य॒ हृद॑यमा॒त्माभि समे॑ति॒ यत्पृ॑षदा॒ज्येन॒ हृद॑यमभिघा॒रय॑त्या॒त्मन्ने॒व प॑शू॒नां प्रा॑णापा॒नौ द॑धाति प॒शुना॒ वै दे॒वाः सु॑व॒र्गं लो॒कमा॑य॒न्ते॑\-ऽमन्यन्त मनु॒ष्या॑ नो॒\-ऽन्वाभ॑विष्य॒न्तीति॒ तस्य॒ शिर॑श्छि॒त्त्वा मेधं॒ प्राक्षा॑रय॒न्थ्स प्र॒क्षो॑\-ऽभव॒त्तत्प्र॒क्षस्य॑ प्रक्ष॒त्वं यत्प्ल॑क्षशा॒खोत्त॑रब॒र्॒\mbox{}हिर्भव॑ति॒ समे॑धस्यै॒व~(५२)

%6.3.10.3
प॒शोरव॑ द्यति प॒शुं वै ह्रि॒यमा॑ण॒ꣳ॒ रक्षा॒ꣴ॒स्यनु॑ सचन्ते\-ऽन्त॒रा यूपं॑ चाहव॒नीयं॑ च हरति॒ रक्ष॑सा॒मप॑हत्यै प॒शोर्वा आल॑ब्धस्य॒ मनो\-ऽप॑ क्रामति म॒नोता॑यै ह॒विषो॑\-ऽवदी॒यमा॑न॒स्यानु॑ ब्रू॒हीत्या॑ह॒ मन॑ ए॒वास्याव॑ रुन्ध॒ एका॑\-दशाव॒दाना॒न्यव॑ द्यति॒ दश॒ वै प॒शोः प्रा॒णा आ॒त्मैका॑द॒शो यावा॑ने॒व प॒शुस्तस्याव॑~(५३)

%6.3.10.4
द्य॒ति॒ हृद॑य॒स्याग्रे\-ऽव॑ द्य॒त्यथ॑ जि॒ह्वाया॒ अथ॒ वक्ष॑सो॒ यद्वै हृद॑येनाभि॒गच्छ॑ति॒ तज्जि॒ह्वया॑ वदति॒ यज्जि॒ह्वया॒ वद॑ति॒ तदुर॒सो\-ऽधि॒ निर्व॑दत्ये॒तद्वै प॒शोर्य॑थापू॒र्वं यस्यै॒वम॑व॒दाय॑ यथा॒काम॒मुत्त॑रेषामव॒द्यति॑ यथापू॒र्वमे॒वास्य॑ प॒शोरव॑त्तं भवति मध्य॒तो गु॒दस्याव॑ द्यति मध्य॒तो हि प्रा॒ण उ॑त्त॒मस्याव॑ द्यति~(५४)

%6.3.10.5
उ॒त्त॒मो हि प्रा॒णो यदीत॑रं॒ यदीत॑रमु॒भय॑मे॒वाजा॑मि॒ जाय॑मानो॒ वै ब्रा᳚ह्म॒णस्त्रि॒भिर्\mbox{}ऋ॑ण॒वा जा॑यते ब्रह्म॒चर्ये॒णर्\mbox{}षि॑भ्यो य॒ज्ञेन॑ दे॒वेभ्यः॑ प्र॒जया॑ पि॒तृभ्य॑ ए॒ष वा अ॑नृ॒णो यः पु॒त्री यज्वा᳚ ब्रह्मचारिवा॒सी तद॑व॒दानै॑रे॒वाव॑ दयते॒ तद॑व॒दाना॑नामवदान॒त्वन्दे॑वासु॒राः संय॑त्ता आस॒न्ते दे॒वा अ॒ग्निम॑ब्रुव॒न्त्वया॑ वी॒रेणासु॑रान॒भि भ॑वा॒मेति॑~(५)

%6.3.10.6
सो᳚\-ऽब्रवी॒द्वरं॑ वृणै प॒शोरु॑द्धा॒रमुद्ध॑रा॒ इति॒ स ए॒तमु॑द्धा॒रमुद॑हरत॒ दोः पू᳚र्वा॒र्धस्य॑ गु॒दं म॑ध्य॒तः श्रोणिं॑ जघना॒र्धस्य॒ ततो॑ दे॒वा अभ॑व॒न्परासु॑रा॒ यत्त्र्य॒ङ्गाणाꣳ॑ समव॒द्यति॒ भ्रातृ॑व्या॒भिभूत्यै॒ भव॑त्या॒त्मना॒ परा᳚स्य॒ भ्रातृ॑व्यो भवत्यक्ष्ण॒याव॑ द्यति॒ तस्मा॑दक्ष्ण॒या प॒शवो\-ऽङ्गा॑नि॒ प्र ह॑रन्ति॒ प्रति॑ष्ठित्यै॥~(५६)

%6.3.11.0
{\anuvakamend[{ए॒तौ प॑शू॒नाꣳ समे॑धस्यै॒व तस्यावो᳚त्त॒मस्याव॑ द्य॒तीति॒ पञ्च॑चत्वारिꣳशच्च}]}%॥10॥

%6.3.11.1
मेद॑सा॒ स्रुचौ॒ प्रोर्णो॑ति॒ मेदो॑रूपा॒ वै प॒शवो॑ रू॒पमे॒व प॒शुषु॑ दधाति यू॒षन्न॑व॒धाय॒ प्रोर्णो॑ति॒ रसो॒ वा ए॒ष प॑शू॒नां यद्यू रस॑मे॒व प॒शुषु॑ दधाति पा॒र्श्वेन॑ वसाहो॒मं प्र यौ॑ति॒ मध्यं॒ वा ए॒तत्प॑शू॒नां यत्पा॒र्श्वꣳ रस॑ ए॒ष प॑शू॒नां यद्वसा॒ यत्पा॒र्श्वेन॑ वसाहो॒मं प्र॒यौति॑ मध्य॒त ए॒व प॑शू॒नाꣳ रसं॑ दधाति॒ घ्नन्ति॑~(५७)

%6.3.11.2
वा ए॒तत्प॒शुं यथ्सं᳚ज्ञ॒पय॑न्त्यै॒न्द्रः खलु॒ वै दे॒वत॑या प्रा॒ण ऐ॒न्द्रो॑\-ऽपा॒न ऐ॒न्द्रः प्रा॒णो अङ्गे॑अङ्गे॒ नि दे᳚ध्य॒दित्या॑ह प्राणापा॒नावे॒व प॒शुषु॑ दधाति॒ देव॑ त्वष्ट॒र्भूरि॑ ते॒ सꣳस॑मे॒त्वित्या॑ह त्वा॒ष्ट्रा हि दे॒वत॑या प॒शवो॒ विषु॑रूपा॒ यथ्सल॑क्ष्माणो॒ भव॒थेत्या॑ह॒ विषु॑रूपा॒ ह्ये॑ते सन्तः॒ सल॑क्ष्माण ए॒तर्\mbox{}हि॒ भव॑न्ति देव॒त्रा यन्तम्᳚~(५८)

%6.3.11.3
अव॑से॒ सखा॒यो\-ऽनु॑ त्वा मा॒ता पि॒तरो॑ मद॒न्त्वित्या॒हानु॑मतमे॒वैनं॑ मा॒त्रा पि॒त्रा सु॑व॒र्गं लो॒कं ग॑मयत्यर्ध॒र्चे व॑साहो॒मं जु॑होत्य॒सौ वा अ॑र्ध॒र्च इ॒यम॑र्ध॒र्च इ॒मे ए॒व रसे॑नानक्ति॒ दिशो॑ जुहोति॒ दिश॑ ए॒व रसे॑नान॒क्त्यथो॑ दि॒ग्भ्य ए॒वोर्ज॒ꣳ॒ रस॒मव॑ रुन्धे प्राणापा॒नौ वा ए॒तौ प॑शू॒नां यत्पृ॑षदा॒ज्यं वा॑नस्प॒त्याः खलु॑~(५९)

%6.3.11.4
वै दे॒वत॑या प॒शवो॒ यत्पृ॑षदा॒ज्यस्यो॑प॒हत्याह॒ वन॒स्पत॒ये\-ऽनु॑ ब्रूहि॒ वन॒स्पत॑ये॒ प्रेष्येति॑ प्राणापा॒नावे॒व प॒शुषु॑ दधात्य॒न्यस्या᳚न्यस्य समव॒त्तꣳ स॒मव॑द्यति॒ तस्मा॒न्नाना॑रूपाः प॒शवो॑ यू॒ष्णोप॑ सिञ्चति॒ रसो॒ वा ए॒ष प॑शू॒नां यद्यू रस॑मे॒व प॒शुषु॑ दधा॒तीडा॒मुप॑ ह्वयते प॒शवो॒ वा इडा॑ प॒शूने॒वोप॑ ह्वयते च॒तुरुप॑ ह्वयते~(६०)

%6.3.11.5
चतु॑ष्पादो॒ हि प॒शवो॒ यं का॒मये॑ताप॒शुः स्या॒दित्य॑मे॒दस्कं॒ तस्मा॒ आ द॑ध्या॒न्मेदो॑रूपा॒ वै प॒शवो॑ रू॒पेणै॒वैनं॑ प॒शुभ्यो॒ निर्भ॑जत्यप॒शुरे॒व भ॑वति॒ यं का॒मये॑त पशु॒मान्थ्स्या॒दिति॒ मेद॑स्व॒त्तस्मा॒ आ द॑ध्या॒न्मेदो॑रूपा॒ वै प॒शवो॑ रू॒पेणै॒वास्मै॑ प॒शूनव॑ रुन्धे पशु॒माने॒व भ॑वति प्र॒जा\-प॑तिर्य॒ज्ञम॑सृजत॒ स आज्यम्᳚~(६१)

%6.3.11.6
पु॒रस्ता॑दसृजत प॒शुं म॑ध्य॒तः पृ॑षदा॒ज्यं प॒श्चात्तस्मा॒दाज्ये॑न प्रया॒जा इ॑ज्यन्ते प॒शुना॑ मध्य॒तः पृ॑षदा॒ज्येना॑नूया॒जास्तस्मा॑दे॒तन्मि॒श्रमि॑व पश्चाथ्सृ॒ष्टꣴ ह्येका॑\-दशानूया॒जान् य॑जति॒ दश॒ वै प॒शोः प्रा॒णा आ॒त्मैका॑द॒शो यावा॑ने॒व प॒शुस्तमनु॑ यजति॒ घ्नन्ति॒ वा ए॒तत्प॒शुं यथ्सं᳚ज्ञ॒पय॑न्ति प्राणापा॒नौ खलु॒ वा ए॒तौ प॑शू॒नां यत्पृ॑षदा॒ज्यं यत्पृ॑षदा॒ज्येना॑नूया॒जान् यज॑ति प्राणापा॒नावे॒व प॒शुषु॑ दधाति॥~(६२)

%6.4.0.0
{\anuvakamend[{घ्नन्ति॒ यन्तं॒ खलु॑ च॒तुरुप॑ ह्वयत॒ आज्यं॒ यत्पृ॑षदा॒ज्येन॒ षट्च॑}]}%॥11॥

%6.4.0.0

{\anuvakamend[{य॒ज्ञेन॒ ता उ॑प॒यड्भि॑र्दे॒वा वै य॒ज्ञमाग्नी᳚ध्रे ब्रह्मवा॒दिनः॒ सत्वै दे॒वस्य॒ ग्रावा॑णं प्रा॒ण उ॑पा॒ꣳ॒श्व॑ग्रा दे॒वा वा उ॑पा॒ꣳ॒शौ वाग्वै मि॒त्रं य॒ज्ञस्य॒ बृह॒स्पति॑र्दे॒वा वा आ᳚ग्रय॒णाग्रा॒नेका॑\-दश}]}%॥11॥
{\prashnaend{य॒ज्ञेन॑ लो॒के प॑शु॒मान्थ्स्या॒थ्सव॑नं॒ माध्य॑न्दिनं॒ वाग्वा अरि॑क्तानि॒ तत्प्र॒जा अ॒भ्येक॑पञ्चा॒शत्॥51॥ य॒ज्ञेन॒ गौर॒भि निव॑र्तते॥}}
%%% END PRASHNA

\sect{चतुर्थः प्रश्नः}\setcounter{anuvakam}{0}
\dnsub{तैत्तिरीयसंहितायां षष्ठमकाण्डे चतुर्थः प्रश्नः}
%6.4.1.0
%6.4.1.1
य॒ज्ञेन॒ वै प्र॒जा\-प॑तिः प्र॒जा अ॑सृजत॒ ता उ॑प॒यड्भि॑रे॒वासृ॑जत॒ यदु॑प॒यज॑ उप॒यज॑ति प्र॒जा ए॒व तद्यज॑मानः सृजते जघना॒र्धादव॑ द्यति जघना॒र्धाद्धि प्र॒जाः प्र॒जाय॑न्ते स्थविम॒तो\-ऽव॑ द्यति स्थविम॒तो हि प्र॒जाः प्र॒जाय॒न्ते\-ऽस॑म्भिन्द॒न्नव॑ द्यति प्रा॒णाना॒मस॑म्भेदाय॒ न प॒र्याव॑र्तयति॒ यत्प॑र्याव॒र्तये॑दुदाव॒र्तः प्र॒जा ग्राहु॑कः स्याथ्समु॒द्रं ग॑च्छ॒ स्वाहेत्या॑ह रेतः॑~(१)

%6.4.1.2
ए॒व तद्द॑धात्य॒न्तरि॑क्षं गच्छ॒ स्वाहेत्या॑हा॒न्तरि॑क्षेणै॒वास्मै᳚ प्र॒जाः प्र ज॑नयत्य॒न्तरि॑क्ष॒ꣴ॒ ह्यनु॑ प्र॒जाः प्र॒जाय॑न्ते दे॒वꣳ स॑वि॒तारं॑ गच्छ॒ स्वाहेत्या॑ह सवि॒तृप्र॑सूत ए॒वास्मै᳚ प्र॒जाः प्र ज॑नयत्यहोरा॒त्रे ग॑च्छ॒ स्वाहेत्या॑हाहोरा॒त्राभ्या॑मे॒वास्मै᳚ प्र॒जाः प्र ज॑नयत्यहोरा॒त्रे ह्यनु॑ प्र॒जाः प्र॒जाय॑न्ते मि॒त्रावरु॑णौ गच्छ॒ स्वाहा᳚~(२)

%6.4.1.3
इत्या॑ह प्र॒जास्वे॒व प्रजा॑तासु प्राणापा॒नौ द॑धाति॒ सोमं॑ गच्छ॒ स्वाहेत्या॑ह सौ॒म्या हि दे॒वत॑या प्र॒जा य॒ज्ञं ग॑च्छ॒ स्वाहेत्या॑ह प्र॒जा ए॒व य॒ज्ञियाः᳚ करोति॒ छन्दाꣳ॑सि गच्छ॒ स्वाहेत्या॑ह प॒शवो॒ वै छन्दाꣳ॑सि प॒शूने॒वाव॑ रुन्धे॒ द्यावा॑पृथि॒वी ग॑च्छ॒ स्वाहेत्या॑ह प्र॒जा ए॒व प्रजा॑ता॒ द्यावा॑पृथि॒वीभ्या॑मुभ॒यतः॒ परि॑ गृह्णाति नभः॑~(३)

%6.4.1.4
दि॒व्यं ग॑च्छ॒ स्वाहेत्या॑ह प्र॒जाभ्य॑ ए॒व प्रजा॑ताभ्यो॒ वृष्टिं॒ नि य॑च्छत्य॒ग्निं वै᳚श्वान॒रं ग॑च्छ॒ स्वाहेत्या॑ह प्र॒जा ए॒व प्रजा॑ता अ॒स्यां प्रति॑\-ष्ठापयति प्रा॒णानां॒ वा ए॒षो\-ऽव॑ द्यति॒ यो॑\-ऽव॒द्यति॑ गु॒दस्य॒ मनो॑ मे॒ हार्दि॑ य॒च्छेत्या॑ह प्रा॒णाने॒व य॑थास्था॒नमुप॑ ह्वयते प॒शोर्वा आल॑ब्धस्य॒ हृद॑य॒ꣳ॒ शुगृ॑च्छति॒ सा हृ॑दयशू॒लम्~(४)

%6.4.1.5
अ॒भि समे॑ति॒ यत्पृ॑थि॒व्याꣳ हृ॑दयशू॒लमु॑द्वा॒सये᳚त्पृथि॒वीꣳ शु॒चार्प॑ये॒द्यद॒फ्स्व॑पः शु॒चार्प॑ये॒च्छुष्क॑स्य चा॒र्द्रस्य॑ च स॒न्धावुद्वा॑सयत्यु॒भय॑स्य॒ शान्त्यै॒ यं द्वि॒ष्यात्तं ध्या॑येच्छु॒चैवैन॑मर्पयति॥~(५)

%6.4.2.0
{\anuvakamend[{रेतो॑ मि॒त्रावरु॑णौ गच्छ॒ स्वाहा॒ नभो॑ हृदयशू॒लं द्वात्रिꣳ॑शच्च}]}%~(१)

%6.4.2.1
दे॒वा वै य॒ज्ञमाग्नी᳚ध्रे॒ व्य॑भजन्त॒ ततो॒ यद॒त्यशि॑ष्यत॒ तद॑ब्रुव॒न्वस॑तु॒ नु न॑ इ॒दमिति॒ तद्व॑सती॒वरी॑णां वसतीवरि॒त्वं तस्मि॑न्प्रा॒तर्न सम॑शक्नुव॒न्तद॒फ्सु प्रावे॑शय॒न्ता व॑सती॒वरी॑रभवन्वसती॒वरी᳚र्गृह्णाति य॒ज्ञो वै व॑सती॒वरी᳚र्य॒ज्ञमे॒वारभ्य॑ गृही॒त्वोप॑ वसति॒ यस्यागृ॑हीता अ॒भि नि॒म्रोचे॒दना॑रब्धो\-ऽस्य य॒ज्ञः स्या᳚त्~(६)

%6.4.2.2
य॒ज्ञं वि च्छि॑न्द्याज्ज्योति॒ष्या॑ वा गृह्णी॒याद्धिर॑ण्यं वाव॒धाय॒ सशु॑क्राणामे॒व गृ॑ह्णाति॒ यो वा᳚ ब्राह्म॒णो ब॑हुया॒जी तस्य॒ कुम्भ्या॑नां गृह्णीया॒थ्स हि गृ॑ही॒तव॑सतीवरीको वसती॒वरी᳚र्गृह्णाति प॒शवो॒ वै व॑सती॒वरीः᳚ प॒शूने॒वारभ्य॑ गृही॒त्वोप॑ वसति॒ यद॑न्वी॒पं तिष्ठ॑न्गृह्णी॒यान्नि॒र्मार्गु॑का अस्मात्प॒शवः॑ स्युः प्रती॒पं तिष्ठ॑न्गृह्णाति प्रति॒रुध्यै॒वास्मै॑ प॒शून्गृ॑ह्णा॒तीन्द्रः॑~(७)

%6.4.2.3
वृ॒त्रम॑ह॒न्थ्सो \-ऽपो \-ऽभ्य॑म्रियत॒ तासां॒ यन्मेध्यं॑ य॒ज्ञिय॒ꣳ॒ सदे॑व॒मासी॒त्तदत्य॑मुच्यत॒ ता वह॑न्तीरभव॒न्वह॑न्तीनां गृह्णाति॒ या ए॒व मेध्या॑ य॒ज्ञियाः॒ सदे॑वा॒ आप॒स्तासा॑मे॒व गृ॑ह्णाति॒ नान्त॒मा वह॑न्ती॒रती॑या॒द्यद॑न्त॒मा वह॑न्तीरती॒याद्य॒ज्ञमति॑ मन्येत॒ न स्था॑व॒राणां᳚ गृह्णीया॒द्वरु॑णगृहीता॒ वै स्था॑व॒रा यथ्स्था॑व॒राणां᳚ गृह्णी॒यात्~(८)

%6.4.2.4
वरु॑णेनास्य य॒ज्ञं ग्रा॑हये॒द्यद्वै दिवा॒ भव॑त्य॒पो रात्रिः॒ प्र वि॑शति॒ तस्मा᳚त्ता॒म्रा आपो॒ दिवा॑ ददृश्रे॒ यन्नक्त॒म्भव॑त्य॒पो\-ऽहः॒ प्र वि॑शति॒ तस्मा᳚च्च॒न्द्रा आपो॒ नक्तं॑ ददृश्रे छा॒यायै॑ चा॒तप॑तश्च सं॒धौ गृ॑ह्णात्यहोरा॒त्रयो॑रे॒वास्मै॒ वर्णं॑ गृह्णाति ह॒विष्म॑तीरि॒मा आप॒ इत्या॑ह ह॒विष्कृ॑तानामे॒व गृ॑ह्णाति ह॒विष्माꣳ॑ अस्तु~(९)

%6.4.2.5
सूर्य॒ इत्या॑ह॒ सशु॑क्राणामे॒व गृ॑ह्णात्यनु॒ष्टुभा॑ गृह्णाति॒ वाग्वा अ॑नु॒ष्टुग्वा॒चैवैनाः॒ सर्व॑या गृह्णाति॒ चतु॑ष्पदय॒र्चा गृ॑ह्णाति॒ त्रिः सा॑दयति स॒प्त सम्प॑द्यन्ते स॒प्तप॑दा॒ शक्व॑री प॒शवः॒ शक्व॑री प॒शूने॒वाव॑ रुन्धे॒\-ऽस्मै वै लो॒काय॒ गार्\mbox{}ह॑पत्य॒ आ धी॑यते॒\-ऽमुष्मा॑ आहव॒नीयो॒ यद्गार्\mbox{}ह॑पत्य उपसा॒दये॑द॒स्मिँल्लो॒के प॑शु॒मान्थ्स्या॒द्ययदा॑हव॒नीये॒\-ऽमुष्मिन्न्॑~(१०)

%6.4.2.6
लो॒के प॑शु॒मान्थ्स्या॑दु॒भयो॒रुप॑ सादयत्यु॒भयो॑रे॒वैनं॑ लो॒कयोः᳚ पशु॒मन्तं॑ करोति स॒र्वतः॒ परि॑ हरति॒ रक्ष॑सा॒मप॑हत्या इन्द्राग्नि॒योर्भा॑ग॒धेयीः॒ स्थेत्या॑ह यथाय॒जुरे॒वैतदाग्नी᳚ध्र॒ उप॑ वासयत्ये॒तद्वै य॒ज्ञस्याप॑राजितं॒ यदाग्नी᳚ध्रं॒ यदे॒व य॒ज्ञस्याप॑राजितं॒ तदे॒वैना॒ उप॑ वासयति॒ यतः॒ खलु॒ वै य॒ज्ञस्य॒ वित॑तस्य॒ न क्रि॒यते॒ तदनु॑ य॒ज्ञꣳ रक्षा॒ꣴ॒स्यव॑ चरन्ति॒ यद्वह॑न्तीनां गृ॒ह्णाति॑ क्रि॒यमा॑णमे॒व तद्य॒ज्ञस्य॑ शये॒ रक्ष॑सा॒मन॑न्ववचाराय॒ न ह्ये॑ता ई॒लय॒न्त्या तृ॑तीयसव॒नात्परि॑ शेरे य॒ज्ञस्य॒ सन्त॑त्यै॥~(११)

%6.4.3.0
{\anuvakamend[{स्या॒दिन्द्रो॑ गृह्णी॒याद॑स्त्व॒मुष्मि॑न्क्रि॒यते॒ षड्विꣳ॑शतिश्च}]}%~(२)

%6.4.3.1
ब्र॒ह्म॒वा॒दिनो॑ वदन्ति॒ स त्वा अ॑ध्व॒र्युः स्या॒द्यः सोम॑मुपाव॒हर॒न्थ्सर्वा᳚भ्यो दे॒वता᳚भ्य उपाव॒हरे॒दिति॑ हृ॒दे त्वेत्या॑ह मनु॒ष्ये᳚भ्य ए॒वैतेन॑ करोति॒ मन॑से॒ त्वेत्या॑ह पि॒तृभ्य॑ ए॒वैतेन॑ करोति दि॒वे त्वा॒ सूर्या॑य॒ त्वेत्या॑ह दे॒वेभ्य॑ ए॒वैतेन॑ करोत्ये॒ताव॑ती॒र्वै दे॒वता॒स्ताभ्य॑ ए॒वैन॒ꣳ॒ सर्वा᳚भ्य उ॒पाव॑हरति पु॒रा वा॒चः~(१२)

%6.4.3.2
प्रव॑दितोः प्रातरनुवा॒कमु॒पा\-क॑रोति॒ याव॑त्ये॒व वाक्तामव॑ रुन्धे॒\-ऽपो\-ऽग्रे॑\-ऽभि॒व्याह॑रति य॒ज्ञो वा आपो॑ य॒ज्ञमे॒वाभि वाचं॒ वि सृ॑जति॒ सर्वा॑णि॒ छन्दा॒ꣴ॒स्यन्वा॑ह प॒शवो॒ वै छन्दाꣳ॑सि प॒शूने॒वाव॑ रुन्धे गायत्रि॒या तेज॑स्कामस्य॒ परि॑ दध्यात्त्रि॒ष्टुभे᳚न्द्रि॒यका॑मस्य॒ जग॑त्या प॒शुका॑मस्यानु॒ष्टुभा᳚ प्रति॒ष्ठाका॑मस्य प॒ङ्क्त्या य॒ज्ञका॑मस्य वि॒राजान्न॑कामस्य शृ॒णोत्व॒ग्निः स॒मिधा॒ हवम्᳚~(१३)

%6.4.3.3
म॒ इत्या॑ह सवि॒तृप्र॑सूत ए॒व दे॒वता᳚भ्यो नि॒वेद्या॒पो\-ऽच्छै᳚त्य॒प इ॑ष्य होत॒रित्या॑हेषि॒तꣳ हि कर्म॑ क्रि॒यते॒ मैत्रा॑वरुणस्य चमसाध्वर्य॒वा द्र॒वेत्या॑ह मि॒त्रावरु॑णौ॒ वा अ॒पां ने॒तारौ॒ ताभ्या॑मे॒वैना॒ अच्छै॑ति॒ देवी॑रापो अपां नपा॒दित्या॒हाहु॑त्यै॒वैना॑ नि॒ष्क्रीय॑ गृह्णा॒त्यथो॑ ह॒विष्कृ॑तानामे॒वाभिघृ॑तानां गृह्णाति~(१४)

%6.4.3.4
कार्\mbox{}षि॑र॒सीत्या॑ह॒ शम॑लमे॒वासा॒मप॑ प्लावयति समु॒द्रस्य॒ वोक्षि॑त्या॒ उन्न॑य॒ इत्या॑ह॒ तस्मा॑द॒द्यमा॑नाः पी॒यमा॑ना॒ आपो॒ न क्षी॑यन्ते॒ योनि॒र्वै य॒ज्ञस्य॒ चात्वा॑लं य॒ज्ञो व॑सती॒वरीर्॑\mbox{}होतृचम॒सं च॑ मैत्रावरुणचम॒सं च॑ स॒ꣴ॒स्पर्श्य॑ वसती॒वरी॒र्व्यान॑यति य॒ज्ञस्य॑ सयोनि॒त्वायाथो॒ स्वादे॒वैना॒ योनेः॒ प्र ज॑नय॒त्यध्व॒र्यो\-ऽवे॑र॒पा~(३) इत्या॑हो॒तेम॑नन्नमुरु॒तेमाः प॒श्येति॒ वावैतदा॑ह॒ यद्य॑ग्निष्टो॒मो जु॒होति॒ यद्यु॒क्थ्यः॑ परि॒धौ नि मा᳚र्ष्टि॒ यद्य॑तिरा॒त्रो यजु॒र्वद॒न्प्र प॑द्यते यज्ञक्रतू॒नां व्यावृ॑त्त्यै॥~(१५)

%6.4.4.0
{\anuvakamend[{वा॒चो हव॑म॒भिघृ॑तानां गृह्णात्यु॒त पञ्च॑विꣳशतिश्च}]}%~(३)

%6.4.4.1
दे॒वस्य॑ त्वा सवि॒तुः प्र॑स॒व इति॒ ग्रावा॑ण॒मा द॑त्ते॒ प्रसू᳚त्या अ॒श्विनो᳚र्बा॒हुभ्या॒मित्या॑हा॒श्विनौ॒ हि दे॒वाना॑मध्व॒र्यू आस्तां᳚ पू॒ष्णो हस्ता᳚भ्या॒मित्या॑ह॒ यत्यै॑ प॒शवो॒ वै सोमो᳚ व्या॒न उ॑पाꣳशु॒सव॑नो॒ यदु॑पाꣳशु॒सव॑नम॒भि मिमी॑ते व्या॒नमे॒व प॒शुषु॑ दधा॒तीन्द्रा॑य॒ त्वेन्द्रा॑य॒ त्वेति॑ मिमीत॒ इन्द्रा॑य॒ हि सोम॑ आह्रि॒यते॒ पञ्च॒ कृत्वो॒ यजु॑षा मिमीते~(१६)

%6.4.4.2
पञ्चा᳚क्षरा प॒ङ्क्तिः पाङ्क्तो॑ य॒ज्ञो य॒ज्ञमे॒वाव॑ रुन्धे॒ पञ्च॒ कृत्व॑स्तू॒ष्णीन्दश॒ सम्प॑द्यन्ते॒ दशा᳚क्षरा वि॒राडन्नं॑ वि॒राड्वि॒राजै॒वान्नाद्य॒मव॑ रुन्धे श्वा॒त्राः स्थ॑ वृत्र॒तुर॒ इत्या॑है॒ष वा अ॒पाꣳ सो॑मपी॒थो य ए॒वं वेद॒ नाफ्स्वार्ति॒मार्च्छ॑ति॒ यत्ते॑ सोम दि॒वि ज्योति॒रित्या॑है॒भ्य ए॒वैनम्᳚~(१७)

%6.4.4.3
लो॒केभ्यः॒ सम्भ॑रति॒ सोमो॒ वै राजा॒ दिशो॒\-ऽभ्य॑ध्याय॒थ्स दिशो\-ऽनु॒ प्रावि॑श॒त्प्रागपा॒गुद॑गध॒रागित्या॑ह दि॒ग्भ्य ए॒वैन॒ꣳ॒ सम्भ॑र॒त्यथो॒ दिश॑ ए॒वास्मा॒ अव॑ रु॒न्धे\-ऽम्ब॒ नि ष्व॒रेत्या॑ह॒ कामु॑का एन॒ꣴ॒ स्त्रियो॑ भवन्ति॒ य ए॒वं वेद॒ यत्ते॑ सो॒मादा᳚भ्यं॒ नाम॒ जागृ॒वीति॑~(१८)

%6.4.4.4
आ॒है॒ष वै सोम॑स्य सोमपी॒थो य ए॒वं वेद॒ न सौ॒म्यामार्ति॒मार्च्छ॑ति॒ घ्नन्ति॒ वा ए॒तथ्सोमं॒ यद॑भिषु॒ण्वन्त्य॒ꣳ॒शूनप॑ गृह्णाति॒ त्राय॑त ए॒वैनं॑ प्रा॒णा वा अ॒ꣳ॒शवः॑ प॒शवः॒ सोमो॒\-ऽꣳ॒शून्पुन॒रपि॑ सृजति प्रा॒णाने॒व प॒शुषु॑ दधाति॒ द्वौद्वा॒वपि॑ सृजति॒ तस्मा॒द्द्वौद्वौ᳚ प्रा॒णाः॥~(१९)

%6.4.5.0
{\anuvakamend[{यजु॑षा मिमीत एनं॒ जागृ॒वीति॒ चतु॑श्चत्वारिꣳशच्च}]}%~(४)

%6.4.5.1
प्रा॒णो वा ए॒ष यदु॑पा॒ꣳ॒शुर्यदु॑पा॒ꣳ॒श्व॑ग्रा॒ ग्रहा॑ गृ॒ह्यन्ते᳚ प्रा॒णमे॒वानु॒ प्र य॑न्त्यरु॒णो ह॑ स्मा॒हौप॑वेशिः प्रातःसव॒न ए॒वाहं य॒ज्ञꣳ सꣴस्था॑पयामि॒ तेन॒ ततः॒ सꣴस्थि॑तेन चरा॒मीत्य॒ष्टौ कृत्वो\-ऽग्रे॒\-ऽभि षु॑णोत्य॒ष्टाक्ष॑रा गाय॒त्री गा॑य॒त्रं प्रा॑तःसव॒नं प्रा॑तःसव॒नमे॒व तेना᳚ऽऽप्नो॒त्येका॑\-दश॒ कृत्वो᳚ द्वि॒तीय॒मेका॑\-दशाक्षरा त्रि॒ष्टुप्त्रैष्टु॑भं॒ माध्य॑न्दिनम्~(२०)

%6.4.5.2
सव॑नं॒ माध्य॑न्दिनमे॒व सव॑नं॒ तेना᳚ऽऽप्नोति॒ द्वाद॑श॒ कृत्व॑स्तृ॒तीयं॒ द्वाद॑शाक्षरा॒ जग॑ती॒ जाग॑तं तृतीयसव॒नन्तृ॑तीयसव॒नमे॒व तेना᳚ऽऽप्नोत्ये॒ताꣳ ह॒ वाव स य॒ज्ञस्य॒ सꣴस्थि॑तिमुवा॒चास्क॑न्दा॒यास्क॑न्न॒ꣳ॒ हि तद्यद्य॒ज्ञस्य॒ सꣴस्थि॑तस्य॒ स्कन्द॒त्यथो॒ खल्वा॑हुर्गाय॒त्री वाव प्रा॑तःसव॒ने नाति॒वाद॒ इत्यन॑तिवादुक एन॒म्भ्रातृ॑व्यो भवति॒ य ए॒वं वेद॒ तस्मा॑द॒ष्टाव॑ष्टौ~(२१)

%6.4.5.3
कृत्वो॑\-ऽभि॒षुत्यं॑ ब्रह्मवा॒दिनो॑ वदन्ति प॒वित्र॑वन्तो॒\-ऽन्ये ग्रहा॑ गृ॒ह्यन्ते॒ किं प॑वित्र उपा॒ꣳ॒शुरिति॒ वाक्प॑वित्र॒ इति॑ ब्रूयात् वा॒चस्पत॑ये पवस्व वाजि॒न्नित्या॑ह वा॒चैवैनं॑ पवयति॒ वृष्णो॑ अ॒ꣳ॒शुभ्या॒मित्या॑ह॒ वृष्णो॒ ह्ये॑ताव॒ꣳ॒शू यौ सोम॑स्य॒ गभ॑स्तिपूत॒ इत्या॑ह॒ गभ॑स्तिना॒ ह्ये॑नं प॒वय॑ति दे॒वो दे॒वानां᳚ प॒वित्र॑म॒सीत्या॑ह दे॒वो ह्ये॑षः~(२२)

%6.4.5.4
सं दे॒वानां᳚ प॒वित्रं॒ येषां᳚ भा॒गो\-ऽसि॒ तेभ्य॒स्त्वेत्या॑ह॒ येषा॒ꣴ॒ ह्ये॑ष भा॒गस्तेभ्य॑ एनं गृ॒ह्णाति॒ स्वां कृ॑तो॒\-ऽसीत्या॑ह प्रा॒णमे॒व स्वम॑कृत॒ मधु॑मतीर्न॒ इष॑स्कृ॒धीत्या॑ह॒ सर्व॑मे॒वास्मा॑ इ॒दꣴ स्व॑दयति॒ विश्वे᳚भ्यस्त्वेन्द्रि॒येभ्यो॑ दि॒व्येभ्यः॒ पार्थि॑वेभ्य॒ इत्या॑हो॒भये᳚ष्वे॒व दे॑वमनु॒ष्येषु॑ प्रा॒णान्द॑धाति॒ मन॑स्त्वा~(२३)

%6.4.5.5
अ॒ष्ट्वित्या॑ह॒ मन॑ ए॒वाश्ञु॑त उ॒र्व॑न्तरि॑क्ष॒मन्वि॒हीत्या॑हान्तरिक्षदेव॒त्यो॑ हि प्रा॒णः स्वाहा᳚ त्वा सुभवः॒ सूर्या॒येत्या॑ह प्रा॒णा वै स्वभ॑वसो दे॒वास्तेष्वे॒व प॒रोक्षं॑ जुहोति दे॒वेभ्य॑स्त्वा मरीचि॒पेभ्य॒ इत्या॑हादि॒त्यस्य॒ वै र॒श्मयो॑ दे॒वा म॑रीचि॒पास्तेषां॒ तद्भा॑ग॒धेय॒न्ताने॒व तेन॑ प्रीणाति॒ यदि॑ का॒मये॑त॒ वर्\mbox{}षु॑कः प॒र्जन्यः॑~(२४)

%6.4.5.6
स्या॒दिति॒ नीचा॒ हस्ते॑न॒ नि मृ॑ज्या॒द्वृष्टि॑मे॒व नि य॑च्छति॒ यदि॑ का॒मये॒ताव॑र्\mbox{}षुकः स्या॒दित्यु॑त्ता॒नेन॒ नि मृ॑ज्या॒द्वृष्टि॑मे॒वोद्य॑च्छति॒ यद्य॑भि॒चरे॑द॒मुं ज॒ह्यथ॑ त्वा होष्या॒मीति॑ ब्रूया॒दाहु॑तिमे॒वैनं॑ प्रे॒फ्सन् ह॑न्ति॒ यदि॑ दू॒रे स्यादा तमि॑तोस्तिष्ठेत्प्रा॒णमे॒वास्या॑नु॒गत्य॑ हन्ति॒ यद्य॑भि॒चरे॑द॒मुष्य॑~(२५)

%6.4.5.7
त्वा॒ प्रा॒णे सा॑दया॒मीति॑ सादये॒दस॑न्नो॒ वै प्रा॒णः प्रा॒णमे॒वास्य॑ सादयति ष॒ड्भिर॒ꣳ॒शुभिः॑ पवयति॒ षड्वा ऋ॒तव॑ ऋ॒तुभि॑रे॒वैनं॑ पवयति॒ त्रिः प॑वयति॒ त्रय॑ इ॒मे लो॒का ए॒भिरे॒वैनं॑ लो॒कैः प॑वयति ब्रह्मवा॒दिनो॑ वदन्ति॒ कस्मा᳚थ्स॒त्यात्त्रयः॑ पशू॒नाꣳ हस्ता॑दाना॒ इति॒ यत्त्रिरु॑पा॒ꣳ॒शुꣳ हस्ते॑न विगृ॒ह्णाति॒ तस्मा॒त्त्रयः॑ पशू॒नाꣳ हस्ता॑दानाः॒ पुरु॑षो ह॒स्ती म॒र्कटः॑॥~(२६)

%6.4.6.0
{\anuvakamend[{माध्य॑न्दिनम॒ष्टाव॑ष्टावे॒ष मन॑स्त्वा प॒र्जन्यो॒\-ऽमुष्य॒ पुरु॑षो॒ द्वे च॑}]}%~(५)

%6.4.6.1
दे॒वा वै यद्य॒ज्ञे\-ऽकु॑र्वत॒ तदसु॑रा अकुर्वत॒ ते दे॒वा उ॑पा॒ꣳ॒शौ य॒ज्ञꣳ स॒ꣴ॒स्थाप्य॑मपश्य॒न्तमु॑पा॒ꣳ॒शौ सम॑स्थापय॒न्ते\-ऽसु॑रा॒ वज्र॑मु॒द्यत्य॑ दे॒वान॒भ्या॑यन्त॒ ते दे॒वा बिभ्य॑त॒ इन्द्र॒मुपा॑धाव॒न्तानिन्द्रो᳚\-ऽन्तर्या॒मेणा॒न्तर॑धत्त॒ तद॑न्तर्या॒मस्या᳚न्तर्याम॒त्वं यद॑न्तर्या॒मो गृ॒ह्यते॒ भ्रातृ॑व्याने॒व तद्यज॑मानो॒\-ऽन्तर्ध॑त्ते॒\-ऽन्तस्ते᳚~(२७)

%6.4.6.2
द॒धा॒मि॒ द्यावा॑पृथि॒वी अ॒न्तरु॒र्व॑न्तरि॑क्ष॒मित्या॑है॒भिरे॒व लो॒कैर्यज॑मानो॒ भ्रातृ॑व्यान॒न्तर्ध॑त्ते॒ ते दे॒वा अ॑मन्य॒न्तेन्द्रो॒ वा इ॒दम॑भू॒द्यद्व॒यꣴ स्म इति॒ ते᳚\-ऽब्रुव॒न्मघ॑व॒न्ननु॑ न॒ आ भ॒जेति॑ स॒जोषा॑ दे॒वैरव॑रैः॒ परै॒श्चेत्य॑ब्रवी॒द्ये चै॒व दे॒वाः परे॒ ये चाव॑रे॒ तानु॒भयान्॑~(२८)

%6.4.6.3
अ॒न्वाभ॑जथ्स॒जोषा॑ दे॒वैरव॑रैः॒ परै॒श्चेत्या॑ह॒ ये चै॒व दे॒वाः परे॒ य चाव॑रे॒ तानु॒भया॑न॒न्वाभ॑जत्यन्तर्या॒मे म॑घवन्मादय॒स्वेत्या॑ह य॒ज्ञादे॒व यज॑मानं॒ नान्तरे᳚त्युपया॒मगृ॑हीतो॒\-ऽसीत्या॑हापा॒नस्य॒ धृत्यै॒ यदु॒भाव॑पवि॒त्रौ गृ॒ह्येया॑तां प्रा॒णम॑पा॒नो\-ऽनु॒ न्यृ॑च्छेत्प्र॒मायु॑कः स्यात्प॒वित्र॑वानन्तर्या॒मो गृ॑ह्यते~(२९)

%6.4.6.4
प्रा॒णा॒पा॒नयो॒र्विधृ॑त्यै प्राणापा॒नौ वा ए॒तौ यदु॑पाꣳश्वन्तर्या॒मौ व्या॒न उ॑पाꣳशु॒सव॑नो॒ यं का॒मये॑त प्र॒मायु॑कः स्या॒दित्यसꣴ॑स्पृष्टौ॒ तस्य॑ सादयेद्व्या॒नेनै॒वास्य॑ प्राणापा॒नौ वि च्छि॑नत्ति ता॒जक्प्रमी॑यते॒ यं का॒मये॑त॒ सर्व॒मायु॑रिया॒दिति॒ सꣴस्पृ॑ष्टौ॒ तस्य॑ सादयेद्व्या॒नेनै॒वास्य॑ प्राणापा॒नौ सं त॑नोति॒ सर्व॒मायु॑रेति॥~(३०)

%6.4.7.0
{\anuvakamend[{त॒ उ॒भया᳚न्गृह्यते॒ चतु॑श्चत्वारिꣳशच्च}]}%~(६)

%6.4.7.1
वाग्वा ए॒षा यदै᳚न्द्रवाय॒वो यदै᳚न्द्रवाय॒वाग्रा॒ ग्रहा॑ गृ॒ह्यन्ते॒ वाच॑मे॒वानु॒ प्र य॑न्ति वा॒युं दे॒वा अ॑ब्रुव॒न्थ्सोम॒ꣳ॒ राजा॑नꣳ हना॒मेति॒ सो᳚\-ऽब्रवी॒द्वरं॑ वृणै॒ मद॑ग्रा ए॒व वो॒ ग्रहा॑ गृह्यान्ता॒ इति॒ तस्मा॑दैन्द्रवाय॒वाग्रा॒ ग्रहा॑ गृह्यन्ते॒ तम॑घ्न॒न्थ्सो॑\-ऽपूय॒त् तं दे॒वा नोपा॑धृष्णुव॒न्ते वा॒युम॑ब्रुवन्नि॒मं नः॑ स्वदय~(३१)

%6.4.7.2
इति॒ सो᳚\-ऽब्रवी॒द्वरं॑ वृणै मद्देव॒त्या᳚न्ये॒व वः॒ पात्रा᳚ण्युच्यान्ता॒ इति॒ तस्मा᳚न्नानादेव॒त्या॑नि॒ सन्ति॑ वाय॒व्या᳚न्युच्यन्ते॒ तमे᳚भ्यो वा॒युरे॒वास्व॑दय॒त्तस्मा॒द्यत्पूय॑ति॒ तत्प्र॑वा॒ते वि ष॑जन्ति वा॒युर्\mbox{}हि तस्य॑ पवयि॒ता स्व॑दयि॒ता तस्य॑ वि॒ग्रह॑णं॒ नावि॑न्द॒न्थ्सा\-ऽदि॑तिरब्रवी॒द्वरं॑ वृणा॒ अथ॒ मया॒ वि गृ॑ह्णीध्वं मद्देव॒त्या॑ ए॒व वः॒ सोमाः᳚~(३२)

%6.4.7.3
स॒न्ना अ॑स॒न्नित्यु॑पया॒मगृ॑हीतो॒\-ऽसीत्या॑हादितिदेव॒त्या᳚स्तेन॒ यानि॒ हि दा॑रु॒मया॑णि॒ पात्रा᳚ण्य॒स्यै तानि॒ योनेः॒ सम्भू॑तानि॒ यानि॑ मृ॒न्मया॑नि सा॒क्षात्तान्य॒स्यै तस्मा॑दे॒वमा॑ह॒ वाग्वै परा॒च्यव्या॑कृतावद॒त्ते दे॒वा इन्द्र॑मब्रुवन्नि॒मां नो॒ वाचं॒ व्याकु॒र्विति॒ सो᳚\-ऽब्रवी॒द्वरं॑ वृणै॒ मह्यं॑ चै॒वैष वा॒यवे॑ च स॒ह गृ॑ह्याता॒ इति॒ तस्मा॑दैन्द्रवाय॒वः स॒ह गृ॑ह्यते॒ तामिन्द्रो॑ मध्य॒तो॑\-ऽव॒क्रम्य॒ व्याक॑रो॒त्तस्मा॑दि॒यं व्याकृ॑ता॒ वागु॑द्यते॒ तस्मा᳚थ्स॒कृदिन्द्रा॑य मध्य॒तो गृ॑ह्यते॒ द्विर्वा॒यवे॒ द्वौ हि स वरा॒ववृ॑णीत॥~(३३)

%6.4.8.0
{\anuvakamend[{स्व॒द॒य॒ सोमाः᳚ स॒हाष्टाविꣳ॑शतिश्च}]}%~(७)

%6.4.8.1
मि॒त्रं दे॒वा अ॑ब्रुव॒न्थ्सोम॒ꣳ॒ राजा॑नꣳ हना॒मेति॒ सो᳚\-ऽब्रवी॒न्नाहꣳ सर्व॑स्य॒ वा अ॒हं मि॒त्रम॒स्मीति॒ तम॑ब्रुव॒न्॒ हना॑मै॒वेति॒ सो᳚\-ऽब्रवी॒द्वरं॑ वृणै॒ पय॑सै॒व मे॒ सोमꣴ॑ श्रीण॒न्निति॒ तस्मा᳚न्मैत्रावरु॒णं पय॑सा श्रीणन्ति॒ तस्मा᳚त्प॒शवो\-ऽपा᳚क्रामन् मि॒त्रः सन्क्रू॒रम॑क॒रिति॑ क्रू॒रमि॑व॒ खलु॒ वा ए॒षः~(३४)

%6.4.8.2
क॒रो॒ति॒ यः सोमे॑न॒ यज॑ते॒ तस्मा᳚त्प॒शवो\-ऽप॑ क्रामन्ति॒ यन्मै᳚त्रावरु॒णं पय॑सा श्री॒णाति॑ प॒शुभि॑रे॒व तन्मि॒त्रꣳ स॑म॒र्धय॑ति प॒शुभि॒र्यज॑मानं पु॒रा खलु॒ वावैवं मि॒त्रो॑\-ऽवे॒दप॒ मत्क्रू॒रं च॒क्रुषः॑ प॒शवः॑ क्रमिष्य॒न्तीति॒ तस्मा॑दे॒वम॑वृणीत॒ वरु॑णं दे॒वा अ॑ब्रुव॒न्त्वयाꣳ॑श॒भुवा॒ सोम॒ꣳ॒ राजा॑नꣳ हना॒मेति॒ सो᳚\-ऽब्रवी॒द्वरं॑ वृणै॒ मह्यं॑ च~(३५)

%6.4.8.3
ए॒वैष मि॒त्राय॑ च स॒ह गृ॑ह्याता॒ इति॒ तस्मा᳚न्मैत्रावरु॒णः स॒ह गृ॑ह्यते॒ तस्मा॒द्राज्ञा॒ राजा॑नमꣳश॒भुवा᳚ घ्नन्ति॒ वैश्ये॑न॒ वैश्यꣳ॑ शू॒द्रेण॑ शू॒द्रन्न वा इ॒दं दिवा॒ न नक्त॑मासी॒दव्या॑वृत्त॒न्ते दे॒वा मि॒त्रावरु॑णावब्रुवन्नि॒दं नो॒ वि वा॑सयत॒मिति॒ ताव॑ब्रूतां॒ वरं॑ वृणावहा॒ एक॑ ए॒वावत्पूर्वो॒ ग्रहो॑ ग्रहो गृह्याता॒ इति॒ तस्मा॑दैन्द्रवाय॒वः पूर्वो॑ मैत्रावरु॒णाद्गृ॑ह्यते प्राणापा॒नौ ह्ये॑तौ यदु॑पाꣳश्वन्तर्या॒मौ मि॒त्रो\-ऽह॒रज॑नय॒द्वरु॑णो॒ रात्रिं॒ ततो॒ वा इ॒दं व्यौ᳚च्छ॒द्यन्मै᳚त्रावरु॒णो गृ॒ह्यते॒ व्यु॑ष्ट्यै॥~(३६)

%6.4.9.0
{\anuvakamend[{ए॒ष चै᳚न्द्रवाय॒वो द्वाविꣳ॑शतिश्च}]}%~(८)

%6.4.9.1
य॒ज्ञस्य॒ शिरो᳚\-ऽच्छिद्यत॒ ते दे॒वा अ॒श्विना॑वब्रुवन्भि॒षजौ॒ वै स्थ॑ इ॒दं य॒ज्ञस्य॒ शिरः॒ प्रति॑ धत्त॒मिति॒ ताव॑ब्रूतां॒ वरं॑ वृणावहै॒ ग्रह॑ ए॒व ना॒वत्रापि॑ गृह्यता॒मिति॒ ताभ्या॑मे॒तमा᳚श्वि॒नम॑गृह्ण॒न्ततो॒ वै तौ य॒ज्ञस्य॒ शिरः॒ प्रत्य॑धत्तां॒ यदा᳚श्वि॒नो गृ॒ह्यते॑ य॒ज्ञस्य॒ निष्कृ॑त्यै॒ तौ दे॒वा अ॑ब्रुव॒न्नपू॑तौ॒ वा इ॒मौ म॑नुष्यच॒रौ~(३७)

%6.4.9.2
भि॒षजा॒विति॒ तस्मा᳚द्ब्राह्म॒णेन॑ भेष॒जं न का॒र्य॑मपू॑तो॒ ह्ये  षो॑\-ऽमे॒ध्यो यो भि॒षक्तौ ब॑हिष्पवमा॒नेन॑ पवयि॒त्वा ताभ्या॑मे॒तमा᳚श्वि॒नम॑गृह्ण॒न्तस्मा᳚द्बहिष्पवमा॒ने स्तु॒त आ᳚श्वि॒नो गृ॑ह्यते॒ तस्मा॑दे॒वं वि॒दुषा॑ बहिष्पवमा॒न उ॑प॒सद्यः॑ प॒वित्रं॒ वै ब॑हिष्पवमा॒न आ॒त्मान॑मे॒व प॑वयते॒ तयो᳚स्त्रे॒धा भैष॑ज्यं॒ वि न्य॑दधुर॒ग्नौ तृ॑तीयम॒फ्सु तृती॑यं ब्राह्म॒णे तृती॑य॒न्तस्मा॑दुदपा॒त्रम्~(३८)

%6.4.9.3
उ॒प॒नि॒धाय॑ ब्राह्म॒णं द॑क्षिण॒तो नि॒षाद्य॑ भेष॒जं कु॑र्या॒द्याव॑दे॒व भे॑ष॒जं तेन॑ करोति स॒मर्धु॑कमस्य कृ॒तं भ॑वति ब्रह्मवा॒दिनो॑ वदन्ति॒ कस्मा᳚थ्स॒त्यादेक॑पात्रा द्विदेव॒त्या॑ गृ॒ह्यन्ते᳚ द्वि॒पात्रा॑ हूयन्त॒ इति॒ यदेक॑पात्रा गृ॒ह्यन्ते॒ तस्मा॒देको᳚\-ऽन्तर॒तः प्रा॒णो द्वि॒पात्रा॑ हूयन्ते॒ तस्मा॒द्द्वौद्वौ॑ ब॒हिष्टा᳚त्प्रा॒णाः प्रा॒णा वा ए॒ते यद्द्वि॑देव॒त्याः᳚ प॒शव॒ इडा॒ यदिडां॒ पूर्वां᳚ द्विदेव॒त्ये᳚भ्य उप॒ह्वये॑त~(३९)

%6.4.9.4
प॒शुभिः॑ प्रा॒णान॒न्तर्द॑धीत प्र॒मायु॑कः स्याद्द्विदेव॒त्या᳚न्भक्षयि॒त्वेडा॒मुप॑ ह्वयते प्रा॒णाने॒वात्मन्धि॒त्वा प॒शूनुप॑ ह्वयते॒ वाग्वा ऐ᳚न्द्रवाय॒वश्चक्षु॑र्मैत्रावरु॒णः श्रोत्र॑माश्वि॒नः पु॒रस्ता॑दैन्द्रवाय॒वं भ॑क्षयति॒ तस्मा᳚त्पु॒रस्ता᳚द्वा॒चा व॑दति पु॒रस्ता᳚न्मैत्रावरु॒णं तस्मा᳚त्पु॒रस्ता॒च्चक्षु॑षा पश्यति स॒र्वतः॑ परि॒हार॑माश्वि॒नं तस्मा᳚थ्स॒र्वतः॒ श्रोत्रे॑ण शृणोति प्रा॒णा वा ए॒ते यद्द्वि॑देव॒त्याः᳚~(४०)

%6.4.9.5
अरि॑क्तानि॒ पात्रा॑णि सादयति॒ तस्मा॒दरि॑क्ता अन्तर॒तः प्रा॒णा यतः॒ खलु॒ वै य॒ज्ञस्य॒ वित॑तस्य॒ न क्रि॒यते॒ तदनु॑ य॒ज्ञꣳ रक्षा॒ꣴ॒स्यव॑ चरन्ति॒ यदरि॑क्तानि॒ पात्रा॑णि सा॒दय॑ति क्रि॒यमा॑णमे॒व तद्य॒ज्ञस्य॑ शये॒ रक्ष॑सा॒मन॑न्ववचाराय॒ दक्षि॑णस्य हवि॒र्धान॒स्योत्त॑रस्यां वर्त॒न्याꣳ सा॑दयति वा॒च्ये॑व वाचं॑ दधा॒त्या तृ॑तीयसव॒नात्परि॑ शेरे य॒ज्ञस्य॒ सन्त॑त्यै॥~(४१)

%6.4.10.0
{\anuvakamend[{म॒नु॒ष्य॒च॒रावु॑दपा॒त्रमु॑प॒ह्वये॑त द्विदेव॒त्याः᳚ षट्च॑त्वारिꣳशच्च}]}%~(९)

%6.4.10.1
बृह॒स्पति॑र्दे॒वानां᳚ पु॒रोहि॑त॒ आसी॒च्छण्डा॒मर्का॒वसु॑राणां॒ ब्रह्म॑ण्वन्तो दे॒वा आस॒न्ब्रह्म॑ण्व॒न्तो\-ऽसु॑रा॒स्ते \-ऽन्यो᳚न्यं नाश॑क्नुवन्न॒भिभ॑वितु॒न्ते दे॒वाः शण्डा॒मर्का॒वुपा॑मन्त्रयन्त॒ ताव॑ब्रूतां॒ वरं॑ वृणावहै॒ ग्रहा॑वे॒व ना॒वत्रापि॑ गृह्येता॒मिति॒ ताभ्या॑मे॒तौ शु॒क्राम॒न्थिना॑वगृह्ण॒न्ततो॑ दे॒वा अभ॑व॒न्परासु॑रा॒ यस्यै॒वं वि॒दुषः॑ शु॒क्राम॒न्थिनौ॑ गृ॒ह्येते॒ भव॑त्या॒त्मना॒ परा᳚~(४२)

%6.4.10.2
अ॒स्य॒ भ्रातृ॑व्यो भवति॒ तौ दे॒वा अ॑प॒नुद्या॒त्मन॒ इन्द्रा॑याजुहवु॒रप॑नुत्तौ॒ शण्डा॒मर्कौ॑ स॒हामुनेति॑ ब्रूया॒द्यं द्वि॒ष्याद्यमे॒व द्वेष्टि॒ तेनै॑नौ स॒हाप॑ नुदते॒ स प्र॑थ॒मः सङ्कृ॑तिर्वि॒श्वक॒र्मेत्ये॒वैना॑वा॒त्मन॒ इन्द्रा॑याजुहवु॒रिन्द्रो॒ ह्ये॑तानि॑ रू॒पाणि॒ करि॑क्र॒दच॑रद॒सौ वा आ॑दि॒त्यः शु॒क्रश्च॒न्द्रमा॑ म॒न्थ्य॑पि॒गृह्य॒ प्राञ्चौ॒ निः~(४३)

%6.4.10.3
क्रा॒म॒त॒स्तस्मा॒त्प्राञ्चौ॒ यन्तौ॒ न प॑श्यन्ति प्र॒त्यञ्चा॑वा॒वृत्य॑ जुहुत॒स्तस्मा᳚त्प्र॒त्यञ्चौ॒ यन्तौ॑ पश्यन्ति॒ चक्षु॑षी॒ वा ए॒ते य॒ज्ञस्य॒ यच्छु॒क्राम॒न्थिनौ॒ नासि॑कोत्तरवे॒दिर॒भितः॑ परि॒क्रम्य॑ जुहुत॒स्तस्मा॑द॒भितो॒ नासि॑कां॒ चक्षु॑षी॒ तस्मा॒न्नासि॑कया॒ चक्षु॑षी॒ विधृ॑ते स॒र्वतः॒ परि॑ क्रामतो॒ रक्ष॑सा॒मप॑हत्यै दे॒वा वै याः प्राची॒राहु॑ती॒रजु॑हवु॒र्ये पु॒रस्ता॒दसु॑रा॒ आस॒न्ताꣴ स्ताभिः॒ प्र~(४४)

%6.4.10.4
अ॒नु॒द॒न्त॒ याः प्र॒तीची॒र्ये प॒श्चादसु॑रा॒ आस॒न्ताꣴस्ताभि॒रपा॑नुदन्त॒ प्राची॑र॒न्या आहु॑तयो हू॒यन्ते᳚ प्र॒त्यञ्चौ॑ शु॒क्राम॒न्थिनौ॑ प॒श्चाच्चै॒व पु॒रस्ता᳚च्च॒ यज॑मानो॒ भ्रातृ॑व्या॒न्प्र णु॑दते॒ तस्मा॒त्परा॑चीः प्र॒जाः प्र वी॑यन्ते प्र॒तीची᳚र्जायन्ते शु॒क्राम॒न्थिनौ॒ वा अनु॑ प्र॒जाः प्र जा॑यन्ते॒\-ऽत्त्रीश्चा॒द्या᳚श्च सु॒वीराः᳚ प्र॒जाः प्र॑ज॒नय॒न्परी॑हि शु॒क्रः शु॒क्रशो॑चिषा~(४५)

%6.4.10.5
सु॒प्र॒जाः प्र॒जाः प्र॑ज॒नय॒न्परी॑हि म॒न्थी म॒न्थिशो॑चि॒षेत्या॑है॒ता वै सु॒वीरा॒ या अ॒त्त्रीरे॒ताः सु॑प्र॒जा या आ॒द्या॑ य ए॒वं वेदा॒त्त्र्य॑स्य प्र॒जा जा॑यते॒ नाद्या᳚ प्र॒जा\-प॑ते॒रक्ष्य॑श्वय॒त्तत्परा॑पत॒त्तद्विक॑ङ्कतं॒ प्रावि॑श॒त्तद्विक॑ङ्कते॒ नार॑मत॒ तद्यवं॒ प्रावि॑श॒त् तद्यवे॑\-ऽरमत॒ तद्यव॑स्य~(४६)

%6.4.10.6
य॒व॒त्वं यद्वैक॑ङ्कतं मन्थिपा॒त्रं भव॑ति॒ सक्तु॑भिः श्री॒णाति॑ प्र॒जा\-प॑तेरे॒व तच्चक्षुः॒ सम्भ॑रति ब्रह्मवा॒दिनो॑ वदन्ति॒ कस्मा᳚थ्स॒त्यान्म॑न्थिपा॒त्रꣳ सदो॒ नाश्ञु॑त॒ इत्या᳚र्तपा॒त्रꣳ हीति॑ ब्रूया॒द्यद॑श्ञुवी॒तान्धो᳚\-ऽध्व॒र्युः स्या॒दार्ति॒मार्च्छे॒त्तस्मा॒न्नाश्ञु॑ते॥~(४७)

%6.4.11.0
{\anuvakamend[{आ॒त्मना॒ परा॒ निष्प्र शु॒क्रशो॑चिषा॒ यव॑स्य स॒प्तत्रिꣳ॑शच्च}]}%॥10॥

%6.4.11.1
दे॒वा वै यद्य॒ज्ञे\-ऽकु॑र्वत॒ तदसु॑रा अकुर्वत॒ ते दे॒वा आ᳚ग्रय॒णाग्रा॒न्ग्रहा॑नपश्य॒न्तान॑गृह्णत॒ ततो॒ वै ते\-ऽग्रं॒ पर्या॑य॒न्॒ यस्यै॒वं वि॒दुष॑ आग्रय॒णाग्रा॒ ग्रहा॑ गृ॒ह्यन्ते\-ऽग्र॑मे॒व स॑मा॒नानां॒ पर्ये॑ति रु॒ग्णव॑त्य॒र्चा भ्रातृ॑व्यवतो गृह्णीया॒द्भ्रातृ॑व्यस्यै॒व रु॒क्त्वाग्रꣳ॑ समा॒नानां॒ पर्ये॑ति॒ ये दे॑वा दि॒व्येका॑\-दश॒ स्थेत्या॑ह~(४८)

%6.4.11.2
ए॒ताव॑ती॒र्वै दे॒वता॒स्ताभ्य॑ ए॒वैन॒ꣳ॒ सर्वा᳚भ्यो गृह्णात्ये॒ष ते॒ योनि॒र्विश्वे᳚भ्यस्त्वा दे॒वेभ्य॒ इत्या॑ह वैश्वदे॒वो ह्ये॑ष दे॒वत॑या॒ वाग्वै दे॒वेभ्यो\-ऽपा᳚क्रामद्य॒ज्ञायाति॑ष्ठमाना॒ ते दे॒वा वा॒च्यप॑क्रान्तायां तू॒ष्णीं ग्रहा॑नगृह्णत॒ सा\-ऽम॑न्यत॒ वाग॒न्तर्य॑न्ति॒ वै मेति॒ साग्र॑य॒णं प्रत्याग॑च्छ॒त्तदा᳚ग्रय॒णस्या᳚ग्रयण॒त्वम्~(४९)

%6.4.11.3
तस्मा॑दाग्रय॒णे वाग्वि सृ॑ज्यते॒ यत्तू॒ष्णीं पूर्वे॒ ग्रहा॑ गृ॒ह्यन्ते॒ यथा᳚ थ्सा॒रीय॑ति म॒ आख॒ इय॑ति॒ नाप॑ राथ्स्या॒मीत्यु॑पावसृ॒जत्ये॒वमे॒व तद॑ध्व॒र्युरा᳚ग्रय॒णं गृ॑ही॒त्वा य॒ज्ञमा॒रभ्य॒ वाचं॒ वि सृ॑जते॒ त्रिर्\mbox{}हिं क॑रोत्युद्गा॒तॄने॒व तद्वृ॑णीते प्र॒जा\-प॑ति॒र्वा ए॒ष यदा᳚ग्रय॒णो यदा᳚ग्रय॒णं गृ॑ही॒त्वा हिं॑क॒रोति॑ प्र॒जा\-प॑तिरे॒व~(५०)

%6.4.11.4
तत्प्र॒जा अ॒भि जि॑घ्रति॒ तस्मा᳚द्व॒थ्सं जा॒तं गौर॒भि जि॑घ्रत्या॒त्मा वा ए॒ष य॒ज्ञस्य॒ यदा᳚ग्रय॒णः सव॑नेसवने॒\-ऽभि गृ॑ह्णात्या॒त्मन्ने॒व य॒ज्ञꣳ सं त॑नोत्यु॒परि॑ष्टा॒दा न॑यति॒ रेत॑ ए॒व तद्द॑धात्य॒धस्ता॒दुप॑ गृह्णाति॒ प्र ज॑नयत्ये॒व तद्ब्र॑ह्मवा॒दिनो॑ वदन्ति॒ कस्मा᳚थ्स॒त्याद्गा॑य॒त्री कनि॑ष्ठा॒ छन्द॑साꣳ स॒ती सर्वा॑णि॒ सव॑नानि वह॒तीत्ये॒ष वै गा॑यत्रि॒यै व॒थ्सो यदा᳚ग्रय॒णस्तमे॒व तद॑भिनि॒वर्त॒ꣳ॒ सर्वा॑णि॒ सव॑नानि वहति॒ तस्मा᳚द्व॒थ्सम॒पाकृ॑तं॒ गौर॒भि नि व॑र्तते॥~(५१)

%6.5.0.0
{\anuvakamend[{आ॒हा॒ग्र॒य॒ण॒त्वं प्र॒जा\-प॑तिरे॒वेति॑ विꣳश॒तिश्च॑}]}%॥11॥

%6.5.0.0

{\anuvakamend[{इन्द्रो॑ वृ॒त्रायायु॒र्वै य॒ज्ञेन॑ सुव॒र्गायेन्द्रो॑ म॒रुद्भि॒रदि॑तिरन्तर्यामपा॒त्रेण॑ प्रा॒ण उ॑पाꣳशुपा॒त्रेणेन्द्रो॑ वृ॒त्रम॑ह॒न्तस्य॒ ग्रहा॒न्॒ वै प्रान्यान्येका॑\-दश}]}%॥11॥ 
{\prashnaend{इन्द्रो॑ वृ॒त्राय॒ पुन॑र्\mbox{}ऋ॒तुना॑ह मिथु॒नं प॒शवो॒ नेष्टः॒ पत्नी॑मुपाꣳश्वन्तर्या॒मयो॒र्द्विच॑त्वारिꣳशत्॥42॥ इन्द्रो॑ वृ॒त्राय॑ पाङ्क्त॒त्वम्॥}}
%%% END PRASHNA

\sect{पञ्चमः प्रश्नः}\setcounter{anuvakam}{0}
\dnsub{तैत्तिरीयसंहितायां षष्ठमकाण्डे पञ्चमः प्रश्नः}
%6.5.1.0
%6.5.1.1
इन्द्रो॑ वृ॒त्राय॒ वज्र॒मुद॑यच्छ॒थ्स वृ॒त्रो वज्रा॒दुद्य॑तादबिभे॒थ्सो᳚\-ऽब्रवी॒न्मा मे॒ प्र हा॒रस्ति॒ वा इ॒दं मयि॑ वी॒र्यं॑ तत्ते॒ प्र दा᳚स्या॒मीति॒ तस्मा॑ उ॒क्थ्यं॑ प्राय॑च्छ॒त्तस्मै᳚ द्वि॒तीय॒मुद॑यच्छ॒थ्सो᳚\-ऽब्रवी॒न्मा मे॒ प्र हा॒रस्ति॒ वा इ॒दं मयि॑ वी॒र्यं॑ तत्ते॒ प्र दा᳚स्या॒मीति॑~(१)

%6.5.1.2
तस्मा॑ उ॒क्थ्य॑मे॒व प्राय॑च्छ॒त्तस्मै॑ तृ॒तीय॒मुद॑यच्छ॒त्तं विष्णु॒रन्व॑तिष्ठत ज॒हीति॒ सो᳚\-ऽब्रवी॒न्मा मे॒ प्र हा॒रस्ति॒ वा इ॒दं मयि॑ वी॒र्यं॑ तत्ते॒ प्र दा᳚स्या॒मीति॒ तस्मा॑ उ॒क्थ्य॑मे॒व प्राय॑च्छ॒त्तं निर्मा॑यं भू॒तम॑हन् य॒ज्ञो हि तस्य॑ मा॒यासी॒द्यदु॒क्थ्यो॑ गृ॒ह्यत॑ इन्द्रि॒यमे॒व~(२)

%6.5.1.3
तद्वी॒र्यं॑ यज॑मानो॒ भ्रातृ॑व्यस्य वृङ्क्त॒ इन्द्रा॑य त्वा बृ॒हद्व॑ते॒ वय॑स्वत॒ इत्या॒हेन्द्रा॑य॒ हि स तं प्राय॑च्छ॒त्तस्मै᳚ त्वा॒ विष्ण॑वे॒ त्वेत्या॑ह॒ यदे॒व विष्णु॑र॒न्वति॑ष्ठत ज॒हीति॒ तस्मा॒द्विष्णु॑म॒न्वाभ॑जति॒ त्रिर्निर्गृ॑ह्णाति॒ त्रिर्\mbox{}हि स तं तस्मै॒ प्राय॑च्छदे॒ष ते॒ योनिः॒ पुन॑र्\mbox{}हविर॒सीत्या॑ह॒ पुनः॑पुनः~(३)

%6.5.1.4
ह्य॑स्मान्निर्गृ॒ह्णाति॒ चक्षु॒र्वा ए॒तद्य॒ज्ञस्य॒ यदु॒क्थ्य॑स्तस्मा॑दु॒क्थ्यꣳ॑ हु॒तꣳ सोमा॑ अ॒न्वाय॑न्ति॒ तस्मा॑दा॒त्मा चक्षु॒रन्वे॑ति॒ तस्मा॒देकं॒ यन्तं॑ ब॒हवो\-ऽनु॑ यन्ति॒ तस्मा॒देको॑ बहू॒नां भ॒द्रो भ॑वति॒ तस्मा॒देको॑ ब॒ह्वीर्जा॒या वि॑न्दते॒ यदि॑ का॒मये॑ताध्व॒र्युरा॒त्मानं॑ यज्ञयश॒सेना᳚र्पयेय॒मित्य॑न्त॒राह॑व॒नीयं॑ च हवि॒र्धानं॑ च॒ तिष्ठ॒न्नव॑ नयेत्~(४)

%6.5.1.5
आ॒त्मान॑मे॒व य॑ज्ञयश॒सेना᳚र्पयति॒ यदि॑ का॒मये॑त॒ यज॑मानं यज्ञयश॒सेना᳚र्पयेय॒मित्य॑न्त॒रा स॑दोहविर्धा॒ने तिष्ठ॒न्नव॑ नये॒द्यज॑मानमे॒व य॑ज्ञयश॒सेना᳚र्पयति॒ यदि॑ का॒मये॑त सद॒स्यान्॑ यज्ञयश॒सेना᳚र्पयेय॒मिति॒ सद॑ आ॒लभ्याव॑ नयेथ्सद॒स्या॑ने॒व य॑ज्ञयश॒सेना᳚र्पयति॥~(५)

%6.5.2.0
{\anuvakamend[{इती᳚न्द्रि॒यमे॒व पुनः॑पुनर्नये॒त्त्रय॑स्त्रिꣳशच्च}]}%~(१)

%6.5.2.1
आयु॒र्वा ए॒तद्य॒ज्ञस्य॒ यद्ध्रु॒व उ॑त्त॒मो ग्रहा॑णां गृह्यते॒ तस्मा॒दायुः॑ प्रा॒णाना॑मुत्त॒मं मू॒र्धानं॑ दि॒वो अ॑र॒तिं पृ॑थि॒व्या इत्या॑ह मू॒र्धान॑मे॒वैनꣳ॑ समा॒नानां᳚ करोति वैश्वान॒रमृ॒ताय॑ जा॒तम॒ग्निमित्या॑ह वैश्वान॒रꣳ हि दे॒वत॒यायु॑रुभ॒यतो॑वैश्वानरो गृह्यते॒ तस्मा॑दुभ॒यतः॑ प्रा॒णा अ॒धस्ता᳚च्चो॒परि॑ष्टाच्चा॒र्धिनो॒\-ऽन्ये ग्रहा॑ गृ॒ह्यन्ते॒\-ऽर्धी ध्रु॒वस्तस्मा᳚त्~(६)

%6.5.2.2
अ॒र्ध्यवा᳚ङ्प्रा॒णो᳚\-ऽन्येषां᳚ प्रा॒णाना॒मुपो᳚प्ते॒\-ऽन्ये ग्रहाः᳚ सा॒द्यन्ते\-ऽनु॑पोप्ते ध्रु॒वस्तस्मा॑द॒स्थ्नान्याः प्र॒जाः प्र॑ति॒तिष्ठ॑न्ति मा॒ꣳ॒सेना॒न्या असु॑रा॒ वा उ॑त्तर॒तः पृ॑थि॒वीं प॒र्याचि॑कीर्\mbox{}ष॒न्तां दे॒वा ध्रु॒वेणा॑दृꣳह॒न्तद्ध्रु॒वस्य॑ ध्रुव॒त्वं यद्ध्रु॒व उ॑त्तर॒तः सा॒द्यते॒ धृत्या॒ आयु॒र्वा ए॒तद्य॒ज्ञस्य॒ यद्ध्रु॒व आ॒त्मा होता॒ यद्धो॑तृचम॒से ध्रु॒वम॑व॒नय॑त्या॒त्मन्ने॒व य॒ज्ञस्य॑~(७)

%6.5.2.3
आयु॑र्दधाति पु॒रस्ता॑दु॒क्थस्या॑व॒नीय॒ इत्या॑हुः पु॒रस्ता॒द्ध्यायु॑षो भु॒ङ्क्ते म॑ध्य॒तो॑\-ऽव॒नीय॒ इत्या॑हुर्मध्य॒मेन॒ ह्यायु॑षो भु॒ङ्क्त उ॑त्तरा॒र्धे॑\-ऽव॒नीय॒ इत्या॑हुरुत्त॒मेन॒ ह्यायु॑षो भु॒ङ्क्ते वै᳚श्वदे॒व्यामृ॒चि श॒स्यमा॑नाया॒मव॑ नयति वैश्वदे॒व्यो॑ वै प्र॒जाः प्र॒जास्वे॒वायु॑र्दधाति॥~(८)

%6.5.3.0
{\anuvakamend[{ध्रु॒वस्तस्मा॑दे॒व य॒ज्ञस्यैका॒न्नच॑त्वारि॒ꣳ॒शच्च॑}]}%~(२)

%6.5.3.1
य॒ज्ञेन॒ वै दे॒वाः सु॑व॒र्गं लो॒कमा॑य॒न्ते॑\-ऽमन्यन्त मनु॒ष्या॑ नो॒\-ऽन्वाभ॑विष्य॒न्तीति॒ ते सं॑वथ्स॒रेण॑ योपयि॒त्वा सु॑व॒र्गं लो॒कमा॑य॒न्तमृष॑य ऋतुग्र॒हैरे॒वानु॒ प्राजा॑न॒न्॒यदृ॑तुग्र॒हा गृ॒ह्यन्ते॑ सुव॒र्गस्य॑ लो॒कस्य॒ प्रज्ञा᳚त्यै॒ द्वाद॑श गृह्यन्ते॒ द्वाद॑श॒ मासाः᳚ संवथ्स॒रः सं॑वथ्स॒रस्य॒ प्रज्ञा᳚त्यै स॒ह प्र॑थ॒मौ गृ॑ह्येते स॒होत्त॒मौ तस्मा॒द्द्वौद्वा॑वृ॒तू उ॑भ॒यतो॑मुखमृतुपा॒त्रं भ॑वति॒ कः~(९)

%6.5.3.2
हि तद्वेद॒ यत॑ ऋतू॒नां मुख॑मृ॒तुना॒ प्रेष्येति॒ षट्कृत्व॑ आह॒ षड्वा ऋ॒तव॑ ऋ॒तूने॒व प्री॑णात्यृ॒तुभि॒रिति॑ च॒तुश्चतु॑ष्पद ए॒व प॒शून्प्री॑णाति॒ द्विः पुन॑र्\mbox{}ऋ॒तुना॑ह द्वि॒पद॑ ए॒व प्री॑णात्यृ॒तुना॒ प्रेष्येति॒ षट्कृत्व॑ आह॒र्तुभि॒रिति॑ च॒तुस्तस्मा॒च्चतु॑ष्पादः प॒शव॑ ऋ॒तूनुप॑ जीवन्ति॒ द्विः~(१०)

%6.5.3.3
पुन॑र्\mbox{}ऋ॒तुना॑ह॒ तस्मा᳚द्द्वि॒पाद॒श्चतु॑ष्पदः प॒शूनुप॑ जीवन्त्यृ॒तुना॒ प्रेष्येति॒ षट्कृत्व॑ आह॒र्तुभि॒रिति॑ च॒तुर्द्विः पुन॑र्\mbox{}ऋ॒तुना॑हा॒क्रम॑णमे॒व तथ्सेतुं॒ यज॑मानः कुरुते सुव॒र्गस्य॑ लो॒कस्य॒ सम॑ष्ट्यै॒ नान्यो᳚न्यमनु॒ प्र प॑द्येत॒ यद॒न्यो᳚\-ऽन्यम॑नु प्र॒पद्ये॑त॒र्तुर्\mbox{}ऋ॒तुमनु॒ प्र प॑द्येत॒र्तवो॒ मोहु॑काः स्युः~(११)

%6.5.3.4
प्रसि॑द्धमे॒वाध्व॒र्युर्दक्षि॑णेन॒ प्र प॑द्यते॒ प्रसि॑द्धं प्रतिप्रस्था॒तोत्त॑रेण॒ तस्मा॑दादि॒त्यः षण्मा॒सो दक्षि॑णेनैति॒ षडुत्त॑रेणोपया॒मगृ॑हीतो\-ऽसि स॒ꣳ॒सर्पो᳚\-ऽस्यꣳहस्प॒त्याय॒ त्वेत्या॒हास्ति॑ त्रयोद॒शो मास॒ इत्या॑हु॒स्तमे॒व तत्प्री॑णाति॥~(१२)

%6.5.4.0
{\anuvakamend[{को जी॑वन्ति॒ द्विः स्यु॒श्चतु॑स्त्रिꣳशच्च}]}%~(३)

%6.5.4.1
सु॒व॒र्गाय॒ वा ए॒ते लो॒काय॑ गृह्यन्ते॒ यदृ॑तुग्र॒हा ज्योति॑रिन्द्रा॒ग्नी यदै᳚न्द्रा॒ग्नमृ॑तुपा॒त्रेण॑ गृ॒ह्णाति॒ ज्योति॑रे॒वास्मा॑ उ॒परि॑ष्टाद्दधाति सुव॒र्गस्य॑ लो॒कस्यानु॑ख्यात्या ओजो॒भृतौ॒ वा ए॒तौ दे॒वानां॒ यदि॑न्द्रा॒ग्नी यदै᳚न्द्रा॒ग्नो गृ॒ह्यत॒ ओज॑ ए॒वाव॑ रुन्धे वैश्वदे॒वꣳ शु॑क्रपा॒त्रेण॑ गृह्णाति वैश्वदे॒व्यो॑ वै प्र॒जा अ॒सावा॑दि॒त्यः शु॒क्रो यद्वै᳚श्वदे॒वꣳ शु॑क्रपा॒त्रेण॑ गृ॒ह्णाति॒ तस्मा॑द॒सावा॑दि॒त्यः~(१३)

%6.5.4.2
सर्वाः᳚ प्र॒जाः प्र॒त्यङ्ङुदे॑ति॒ तस्मा॒थ्सर्व॑ ए॒व म॑न्यते॒ मां प्रत्युद॑गा॒दिति॑ वैश्वदे॒वꣳ शु॑क्रपा॒त्रेण॑ गृह्णाति वैश्वदे॒व्यो॑ वै प्र॒जास्तेजः॑ शु॒क्रो यद्वै᳚श्वदे॒वꣳ शु॑क्रपा॒त्रेण॑ गृ॒ह्णाति॑ प्र॒जास्वे॒व तेजो॑ दधाति॥~(१४)

%6.5.5.0
{\anuvakamend[{तस्मा॑द॒सावा॑दि॒त्यस्त्रि॒ꣳ॒शच्च॑}]}%~(४)

%6.5.5.1
इन्द्रो॑ म॒रुद्भिः॒ सांवि॑द्येन॒ माध्य॑न्दिने॒ सव॑ने वृ॒त्रम॑ह॒न्॒यन्माध्य॑न्दिने॒ सव॑ने मरुत्व॒तीया॑ गृ॒ह्यन्ते॒ वार्त्र॑घ्ना ए॒व ते यज॑मानस्य गृह्यन्ते॒ तस्य॑ वृ॒त्रं ज॒घ्नुष॑ ऋ॒तवो॑\-ऽमुह्य॒न्थ्स ऋ॑तुपा॒त्रेण॑ मरुत्व॒तीया॑नगृह्णा॒त्ततो॒ वै स ऋ॒तून्प्राजा॑ना॒द्यदृ॑तुपा॒त्रेण॑ मरुत्व॒तीया॑ गृ॒ह्यन्त॑ ऋतू॒नां प्रज्ञा᳚त्यै॒ वज्रं॒ वा ए॒तं यज॑मानो॒ भ्रातृ॑व्याय॒ प्र ह॑रति॒ यन्म॑रुत्व॒तीया॒ उदे॒व प्र॑थ॒मेन॑~(१५)

%6.5.5.2
य॒च्छ॒ति॒ प्र ह॑रति द्वि॒तीये॑न स्तृणु॒ते तृ॒तीये॒नायु॑धं॒ वा ए॒तद्यज॑मानः॒ सꣴस्कु॑रुते॒ यन्म॑रुत्व॒तीया॒ धनु॑रे॒व प्र॑थ॒मो ज्या द्वि॒तीय॒ इषु॑स्तृ॒तीयः॒ प्रत्ये॒व प्र॑थ॒मेन॑ धत्ते॒ वि सृ॑जति द्वि॒तीये॑न॒ विध्य॑ति तृ॒तीये॒नेन्द्रो॑ वृ॒त्रꣳ ह॒त्वा परां᳚ परा॒वत॑मगच्छ॒दपा॑राध॒मिति॒ मन्य॑मानः॒ स हरि॑तो\-ऽभव॒थ्स ए॒तान्म॑रुत्व॒तीया॑नात्म॒स्पर॑णानपश्य॒त्तान॑गृह्णीत~(१६)

%6.5.5.3
प्रा॒णमे॒व प्र॑थ॒मेना᳚स्पृणुतापा॒नं द्वि॒तीये॑ना॒ऽऽत्मानं॑ तृ॒तीये॑नात्म॒स्पर॑णा॒ वा ए॒ते यज॑मानस्य गृह्यन्ते॒ यन्म॑रुत्व॒तीयाः᳚ प्रा॒णमे॒व प्र॑थ॒मेन॑ स्पृणुते\-ऽपा॒नं द्वि॒तीये॑ना॒ऽऽत्मानं॑ तृ॒तीये॒नेन्द्रो॑ वृ॒त्रम॑ह॒न्तं दे॒वा अ॑ब्रुवन्म॒हान् वा अ॒यम॑भू॒द्यो वृ॒त्रमव॑धी॒दिति॒ तन्म॑हे॒न्द्रस्य॑ महेन्द्र॒त्वꣳ स ए॒तं मा॑हे॒न्द्रमु॑द्धा॒रमुद॑हरत वृ॒त्रꣳ ह॒त्वान्यासु॑ दे॒वता॒स्वधि॒ यन्मा॑हे॒न्द्रो गृ॒ह्यत॑ उद्धा॒रमे॒व तं यज॑मान॒ उद्ध॑रते॒\-ऽन्यासु॑ प्र॒जास्वधि॑ शुक्रपा॒त्रेण॑ गृह्णाति यजमानदेव॒त्यो॑ वै मा॑हे॒न्द्रस्तेजः॑ शु॒क्रो यन्मा॑हे॒न्द्रꣳ शु॑क्रपा॒त्रेण॑ गृ॒ह्णाति॒ यज॑मान ए॒व तेजो॑ दधाति॥~(१७)

%6.5.6.0
{\anuvakamend[{प्र॒थ॒मेना॑गृह्णीत दे॒वता᳚स्व॒ष्टाविꣳ॑शतिश्च}]}%~(५)

%6.5.6.1
अदि॑तिः पु॒त्रका॑मा सा॒ध्येभ्यो॑ दे॒वेभ्यो᳚ ब्रह्मौद॒नम॑पच॒त्तस्या॑ उ॒च्छेष॑णमददु॒स्तत्प्राश्ञा॒थ्सा रेतो॑\-ऽधत्त॒ तस्यै॑ च॒त्वार॑ आदि॒त्या अ॑जायन्त॒ सा द्वि॒तीय॑मपच॒थ्साम॑न्यतो॒च्छेष॑णान्म इ॒मे᳚\-ऽज्ञत॒ यदग्रे᳚ प्राशि॒ष्यामी॒तो मे॒ वसी॑याꣳसो जनिष्यन्त॒ इति॒ साग्रे॒ प्राश्ञा॒थ्सा रेतो॑\-ऽधत्त॒ तस्यै॒ व्यृ॑द्धमा॒ण्डम॑जायत॒ सादि॒त्येभ्य॑ ए॒व~(१८)

%6.5.6.2
तृ॒तीय॑मपच॒द्भोगा॑य म इ॒दꣴ श्रा॒न्तम॒स्त्विति॒ ते᳚\-ऽब्रुव॒न्वरं॑ वृणामहै॒ यो\-ऽतो॒ जाया॑ता अ॒स्माक॒ꣳ॒ स एको॑\-ऽस॒द्यो᳚\-ऽस्य प्र॒जाया॒मृध्या॑ता अ॒स्माक॒म्भोगा॑य भवा॒दिति॒ ततो॒ विव॑स्वानादि॒त्यो॑\-ऽजायत॒ तस्य॒ वा इ॒यं प्र॒जा यन्म॑नु॒ष्या᳚स्तास्वेक॑ ए॒वर्द्धो यो यज॑ते॒ स दे॒वाना॒म्भोगा॑य भवति दे॒वा वै य॒ज्ञात्~(१९)

%6.5.6.3
रु॒द्रम॒न्तरा॑य॒न्थ्स आ॑दि॒त्यान॒न्वाक्र॑मत॒ ते द्वि॑देव॒त्या᳚न्प्राप॑द्यन्त॒ तान्न प्रति॒ प्राय॑च्छ॒न्तस्मा॒दपि॒ वध्यं॒ प्रप॑न्नं॒ न प्रति॒ प्र य॑च्छन्ति॒ तस्मा᳚द्द्विदेव॒त्ये᳚भ्य आदि॒त्यो निर्गृ॑ह्यते॒ यदु॒च्छेष॑णा॒दजा॑यन्त॒ तस्मा॑दु॒च्छेष॑णाद्गृह्यते ति॒सृभि॑र्\mbox{}ऋ॒ग्भिर्गृ॑ह्णाति मा॒ता पि॒ता पु॒त्रस्तदे॒व तन्मि॑थु॒नमुल्बं॒ गर्भो॑ ज॒रायु॒ तदे॒व तत्~(२०)

%6.5.6.4
मि॒थु॒नं प॒शवो॒ वा ए॒ते यदा॑दि॒त्य ऊर्ग्दधि॑ द॒ध्ना म॑ध्य॒तः श्री॑णा॒त्यूर्ज॑मे॒व प॑शू॒नां म॑ध्य॒तो द॑धाति शृतात॒ङ्क्ये॑न मेध्य॒त्वाय॒ तस्मा॑दा॒मा प॒क्वं दु॑हे प॒शवो॒ वा ए॒ते यदा॑दि॒त्यः प॑रि॒श्रित्य॑ गृह्णाति प्रति॒रुध्यै॒वास्मै॑ प॒शून्गृ॑ह्णाति प॒शवो॒ वा ए॒ते यदा॑दि॒त्य ए॒ष रु॒द्रो यद॒ग्निः प॑रि॒श्रित्य॑ गृह्णाति रु॒द्रादे॒व प॒शून॒न्तर्द॑धाति~(२१)

%6.5.6.5
ए॒ष वै विव॑स्वानादि॒त्यो यदु॑पाꣳशु॒सव॑नः॒ स ए॒तमे॒व सो॑मपी॒थं परि॑ शय॒ आ तृ॑तीयसव॒नाद्विव॑स्व आदित्यै॒ष ते॑ सोमपी॒थ इत्या॑ह॒ विव॑स्वन्तमे॒वाऽऽदि॒त्यꣳ सो॑मपी॒थेन॒ सम॑र्धयति॒ या दि॒व्या वृष्टि॒स्तया᳚ त्वा श्रीणा॒मीति॒ वृष्टि॑कामस्य श्रीणीया॒द्वृष्टि॑मे॒वाव॑ रुन्धे॒ यदि॑ ता॒जक्प्र॒स्कन्दे॒द्वर्\mbox{}षु॑कः प॒र्जन्यः॑ स्या॒द्यदि॑ चि॒रमव॑र्\mbox{}षुको॒ न सा॑दय॒त्यस॑न्ना॒द्धि प्र॒जाः प्र॒जाय॑न्ते॒ नानु॒ वष॑ट्करोति॒ यद॑नुवषट्कु॒॒र्याद्रु॒द्रं प्र॒जा अ॒न्वव॑सृजे॒न्न हु॒त्वान्वी᳚क्षेत॒ यद॒न्वीक्षे॑त॒ चक्षु॑रस्य प्र॒मायु॑कꣴ स्या॒त्तस्मा॒न्नान्वीक्ष्यः॑॥~(२२)

%6.5.7.0
{\anuvakamend[{ए॒व य॒ज्ञाज्ज॒रायु॒ तदे॒व तद॒न्तर्द॑धाति॒ न स॒प्तविꣳ॑शतिश्च}]}%~(६)

%6.5.7.1
अ॒न्त॒र्या॒म॒पा॒त्रेण॑ सावि॒त्रमा᳚ग्रय॒णाद्गृ॑ह्णाति प्र॒जा\-प॑ति॒र्वा ए॒ष यदा᳚ग्रय॒णः प्र॒जानां᳚ प्र॒जन॑नाय॒ न सा॑दय॒त्यस॑न्ना॒द्धि प्र॒जाः प्र॒जाय॑न्ते॒ नानु॒ वष॑ट्करोति॒ यद॑नुवषट्कु॒र्याद्रु॒द्रं प्र॒जा अ॒न्वव॑सृजेदे॒ष वै गा॑य॒त्रो दे॒वानां॒ यथ्स॑वि॒तैष गा॑यत्रि॒यै लो॒के गृ॑ह्यते॒ यदा᳚ग्रय॒णो यद॑न्तर्यामपा॒त्रेण॑ सावि॒त्रमा᳚ग्रय॒णाद्गृ॒ह्णाति॒ स्वादे॒वैनं॒ योने॒र्निर्गृ॑ह्णाति॒ विश्वे᳚~(२३)

%6.5.7.2
दे॒वास्तृ॒तीय॒ꣳ॒ सव॑नं॒ नोद॑यच्छ॒न्ते स॑वि॒तारं॑ प्रातःसव॒नभा॑ग॒ꣳ॒ सन्तं॑ तृतीयसव॒नम॒भि पर्य॑णय॒न्ततो॒ वै ते तृ॒तीय॒ꣳ॒ सव॑न॒मुद॑यच्छ॒न्॒यत्तृ॑तीयसव॒ने सा॑वि॒त्रो गृ॒ह्यते॑ तृ॒तीय॑स्य॒ सव॑न॒स्योद्य॑त्यै सवितृपा॒त्रेण॑ वैश्वदे॒वं क॒लशा᳚द्गृह्णाति वैश्वदे॒व्यो॑ वै प्र॒जा वै᳚श्वदे॒वः क॒लशः॑ सवि॒ता प्र॑स॒वाना॑मीशे॒ यथ्स॑वितृपा॒त्रेण॑ वैश्वदे॒वं क॒लशा᳚द्गृ॒ह्णाति॑ सवि॒तृप्र॑सूत ए॒वास्मै᳚ प्र॒जाः प्र~(२४)

%6.5.7.3
ज॒न॒य॒ति॒ सोमे॒ सोम॑म॒भि गृ॑ह्णाति॒ रेत॑ ए॒व तद्द॑धाति सु॒शर्मा॑सि सुप्रतिष्ठा॒न इत्या॑ह॒ सोमे॒ हि सोम॑मभिगृ॒ह्णाति॒ प्रति॑ष्ठित्या ए॒तस्मि॒न्वा अपि॒ ग्रहे॑ मनु॒ष्ये᳚भ्यो दे॒वेभ्यः॑ पि॒तृभ्यः॑ क्रियते सु॒शर्मा॑सि सुप्रतिष्ठा॒न इत्या॑ह मनु॒ष्ये᳚भ्य ए॒वैतेन॑ करोति बृ॒हदित्या॑ह दे॒वेभ्य॑ ए॒वैतेन॑ करोति॒ नम॒ इत्या॑ह पि॒तृभ्य॑ ए॒वैतेन॑ करोत्ये॒ताव॑ती॒र्वै दे॒वता॒स्ताभ्य॑ ए॒वैन॒ꣳ॒ सर्वा᳚भ्यो गृह्णात्ये॒ष ते॒ योनि॒र्विश्वे᳚भ्यस्त्वा दे॒वेभ्य॒ इत्या॑ह वैश्वदे॒वो ह्ये॑षः॥~(२५)

%6.5.8.0
{\anuvakamend[{विश्वे॒ प्र पि॒तृभ्य॑ ए॒वैतेन॑ करो॒त्येका॒न्नविꣳ॑श॒तिश्च॑}]}%~(७)

%6.5.8.1
प्रा॒णो वा ए॒ष यदु॑पा॒ꣳ॒शुर्यदु॑पाꣳशुपा॒त्रेण॑ प्रथ॒मश्चो᳚त्त॒मश्च॒ ग्रहौ॑ गृ॒ह्येते᳚ प्रा॒णमे॒वानु॑ प्र॒यन्ति॑ प्रा॒णमनूद्य॑न्ति प्र॒जा\-प॑ति॒र्वा ए॒ष यदा᳚ग्रय॒णः प्रा॒ण उ॑पा॒ꣳ॒शुः पत्नीः᳚ प्र॒जाः प्र ज॑नयन्ति॒ यदु॑पाꣳशुपा॒त्रेण॑ पात्नीव॒तमा᳚ग्रय॒णाद्गृ॒ह्णाति॑ प्र॒जानां᳚ प्र॒जन॑नाय॒ तस्मा᳚त्प्रा॒णं प्र॒जा अनु॒ प्र जा॑यन्ते दे॒वा वा इ॒तइ॑तः॒ पत्नीः᳚ सुव॒र्गम्~(२६)

%6.5.8.2
लो॒कम॑जिगाꣳस॒न्ते सु॑व॒र्गं लो॒कं न प्राजा॑न॒न्त ए॒तं पा᳚त्नीव॒तम॑पश्य॒न्तम॑गृह्णत॒ ततो॒ वै ते सु॑व॒र्गं लो॒कं प्राजा॑न॒न्॒ यत्पा᳚त्नीव॒तो गृ॒ह्यते॑ सुव॒र्गस्य॑ लो॒कस्य॒ प्रज्ञा᳚त्यै॒ स सोमो॒ नाति॑ष्ठत स्त्री॒भ्यो गृ॒ह्यमा॑ण॒स्तं घृ॒तं वज्रं॑ कृ॒त्वाघ्न॒न्तं निरि॑न्द्रियं भू॒तम॑गृह्ण॒न्तस्मा॒थ्स्त्रियो॒ निरि॑न्द्रिया॒ अदा॑यादी॒रपि॑ पा॒पात्पु॒ꣳ॒स उप॑स्तितरम्~(२७)

%6.5.8.3
व॒द॒न्ति॒ यद् घृ॒तेन॑ पात्नीव॒तꣴ श्री॒णाति॒ वज्रे॑णै॒वैनं॒ वशे॑ कृ॒त्वा गृ॑ह्णात्युपया॒मगृ॑हीतो॒\-ऽसीत्या॑हे॒यं वा उ॑पया॒मस्तस्मा॑दि॒मां प्र॒जा अनु॒ प्र जा॑यन्ते॒ बृह॒स्पति॑सुतस्य त॒ इत्या॑ह॒ ब्रह्म॒ वै दे॒वानां॒ बृह॒स्पति॒र्ब्रह्म॑णै॒वास्मै᳚ प्र॒जाः प्र ज॑नयतीन्दो॒ इत्या॑ह॒ रेतो॒ वा इन्दू॒ रेत॑ ए॒व तद्द॑धातीन्द्रियाव॒ इति॑~(२८)

%6.5.8.4
आ॒ह॒ प्र॒जा वा इ॑न्द्रि॒यं प्र॒जा ए॒वास्मै॒ प्र ज॑नय॒त्यग्ना(३) इत्या॑हा॒ग्निर्वै रे॑तो॒धाः पत्नी॑व॒ इत्या॑ह मिथुन॒त्वाय॑ स॒जूर्दे॒वेन॒ त्वष्ट्रा॒ सोमं॑ पि॒बेत्या॑ह॒ त्वष्टा॒ वै प॑शू॒नां मि॑थु॒नानाꣳ॑ रूप॒कृद्रू॒पमे॒व प॒शुषु॑ दधाति दे॒वा वै त्वष्टा॑रमजिघाꣳस॒न्थ्स पत्नीः॒ प्राप॑द्यत॒ तं न प्रति॒ प्राय॑च्छ॒न्तस्मा॒दपि॑~(२९)

%6.5.8.5
वध्यं॒ प्रप॑न्नं॒ न प्रति॒ प्र य॑च्छन्ति॒ तस्मा᳚त्पात्नीव॒ते त्वष्ट्रे\-ऽपि॑ गृह्यते॒ न सा॑दय॒त्यस॑न्ना॒द्धि प्र॒जाः प्र॒जाय॑न्ते॒ नानु॒ वष॑ट्करोति॒ यद॑नुवषट्कु॒र्याद्रु॒द्रं प्र॒जा अ॒न्वव॑सृजे॒द्यन्नानु॑वषट्कु॒र्यादशा᳚न्तम॒ग्नीथ्सोमं॑ भक्षयेदुपा॒ꣳ॒श्वनु॒ वष॑ट्करोति॒ न रु॒द्रं प्र॒जा अ॑न्ववसृ॒जति॑ शा॒न्तम॒ग्नीथ्सोम॑म्भक्षय॒त्यग्नी॒न्नेष्टु॑रु॒पस्थ॒मा सी॑द~(३०)

%6.5.8.6
नेष्टः॒ पत्नी॑मु॒दान॒येत्या॑हा॒ग्नीदे॒व नेष्ट॑रि॒ रेतो॒ दधा॑ति॒ नेष्टा॒ पत्नि॑यामुद्गा॒त्रा सं ख्या॑पयति प्र॒जा\-प॑ति॒र्वा ए॒ष यदु॑द्गा॒ता प्र॒जानां᳚ प्र॒जन॑नाया॒प उप॒ प्र व॑र्तयति॒ रेत॑ ए॒व तथ्सि॑ञ्चत्यू॒रुणोप॒ प्र व॑र्तयत्यू॒रुणा॒ हि रेतः॑ सि॒च्यते॑ नग्नं॒कृत्यो॒रुमुप॒ प्र व॑र्तयति य॒दा हि न॒ग्न ऊ॒रुर्भव॒त्यथ॑ मिथु॒नी भ॑व॒तो\-ऽथ॒ रेतः॑ सिच्य॒ते\-ऽथ॑ प्र॒जाः प्र जा॑यन्ते॥~(३१)

%6.5.9.0
{\anuvakamend[{पत्नीः᳚ सुव॒र्गमुप॑स्तितरमिन्द्रियाव॒ इत्यपि॑ सीद मिथु॒न्य॑ष्टौ च॑}]}%~(८)

%6.5.9.1
इन्द्रो॑ वृ॒त्रम॑ह॒न्तस्य॑ शीर्\mbox{}षकपा॒लमुदौ᳚ब्ज॒थ्स द्रो॑णकल॒शो॑\-ऽभव॒त्तस्मा॒थ्सोमः॒ सम॑स्रव॒थ्स हा॑रियोज॒नो॑\-ऽभव॒त्तं व्य॑चिकिथ्सज्जु॒हवा॒नी(३) मा हौ॒षा(३) मिति॒ सो॑\-ऽमन्यत॒ यद्धो॒ष्याम्या॒मꣳ हो᳚ष्यामि॒ यन्न हो॒ष्यामि॑ यज्ञवेश॒सं क॑रिष्या॒मीति॒ तम॑ध्रियत॒ होतु॒ꣳ॒ सो᳚\-ऽग्निर॑ब्रवी॒न्न मय्या॒मꣳ हो᳚ष्य॒सीति॒ तं धा॒नाभि॑रश्रीणात्~(३२)

%6.5.9.2
तꣳ शृ॒तं भू॒तम॑जुहो॒द्यद्धा॒नाभि॑र्\mbox{}हारियोज॒नꣴ श्री॒णाति॑ शृत॒त्वाय॑ शृ॒तमे॒वैनं॑ भू॒तं जु॑होति ब॒ह्वीभिः॑ श्रीणात्ये॒ताव॑ती\-रे॒वास्या॒मुष्मिँ॑ल्लो॒के का॑म॒दुघा॑ भव॒न्त्यथो॒ खल्वा॑हुरे॒ता वा इन्द्र॑स्य॒ पृश्ञ॑यः काम॒दुघा॒ यद्धा॑रियोज॒नीरिति॒ तस्मा᳚द्ब॒ह्वीभिः॑ श्रीणीयादृख्सा॒मे वा इन्द्र॑स्य॒ हरी॑ सोम॒पानौ॒ तयोः᳚ परि॒धय॑ आ॒धानं॒ यदप्र॑हृत्य परि॒धीञ्जु॑हु॒याद॒न्तरा॑धानाभ्याम्~(३३)

%6.5.9.3
घा॒सं प्र य॑च्छेत्प्र॒हृत्य॑ परि॒धीञ्जु॑होति॒ निरा॑धानाभ्यामे॒व घा॒सं प्र य॑च्छत्युन्ने॒ता जु॑होति या॒तया॑मेव॒ ह्ये॑तर्\mbox{}ह्य॑ध्व॒र्युः स्व॒गाकृ॑तो॒ यद॑ध्व॒र्युर्जु॑हु॒याद्यथा॒ विमु॑क्तं॒ पुन॑र्यु॒नक्ति॑ ता॒दृगे॒व तच्छी॒र्॒\mbox{}षन्न॑धिनि॒धाय॑ जुहोति शीर्\mbox{}ष॒तो हि स स॒मभ॑वद्वि॒क्रम्य॑ जुहोति वि॒क्रम्य॒ हीन्द्रो॑ वृ॒त्रमह॒न्थ्समृ॑द्ध्यै प॒शवो॒ वै हा॑रियोज॒नीर्यथ्स॑म्भि॒न्द्यादल्पाः᳚~(३४)

%6.5.9.4
ए॒नं॒ प॒शवो॑ भु॒ञ्जन्त॒ उप॑ तिष्ठेर॒न्॒यन्न स॑म्भि॒न्द्याद्ब॒हव॑ एनं प॒शवो\-ऽभु॑ञ्जन्त॒ उप॑ तिष्ठेर॒न्मन॑सा॒ सम्बा॑धत उ॒भयं॑ करोति ब॒हव॑ ए॒वैनं॑ प॒शवो॑ भु॒ञ्जन्त॒ उप॑ तिष्ठन्त उन्ने॒तर्यु॑पह॒वमि॑च्छन्ते॒ य ए॒व तत्र॑ सोमपी॒थस्तमे॒वाव॑ रुन्धत उत्तरवे॒द्यां नि व॑पति प॒शवो॒ वा उ॑त्तरवे॒दिः प॒शवो॑ हारियोज॒नीः प॒शुष्वे॒व प॒शून्प्रति॑\-ष्ठापयन्ति॥~(३५)

%6.5.10.0
{\anuvakamend[{अ॒श्री॒णा॒द॒न्तरा॑धानाभ्या॒मल्पाः᳚ स्थापयन्ति}]}%~(९)

%6.5.10.1
ग्रहा॒न् वा अनु॑ प्र॒जाः प॒शवः॒ प्र जा॑यन्त उपाꣳश्वन्तर्या॒माव॑जा॒वयः॑ शु॒क्राम॒न्थिनौ॒ पुरु॑षा ऋतुग्र॒हानेक॑शफा आदित्यग्र॒हं गाव॑ आदित्यग्र॒हो भूयि॑ष्ठाभिर्\mbox{}ऋ॒ग्भिर्गृ॑ह्यते॒ तस्मा॒द्गावः॑ पशू॒नां भूयि॑ष्ठा॒ यत्त्रिरु॑पा॒ꣳ॒शुꣳ हस्ते॑न विगृ॒ह्णाति॒ तस्मा॒द्द्वौ त्रीन॒जा ज॒नय॒त्यथाव॑यो॒ भूय॑सीः पि॒ता वा ए॒ष यदा᳚ग्रय॒णः पु॒त्रः क॒लशो॒ यदा᳚ग्रय॒ण उ॑प॒दस्ये᳚त्क॒लशा᳚द्गृह्णीया॒द्यथा॑ पि॒ता~(३६)

%6.5.10.2
पु॒त्रं क्षि॒त उ॑प॒धाव॑ति ता॒दृगे॒व तद्यत्क॒लश॑ उप॒दस्ये॑दाग्रय॒णाद्गृ॑ह्णीया॒द्यथा॑ पु॒त्रः पि॒तरं॑ क्षि॒त उ॑प॒धाव॑ति ता॒दृगे॒व तदा॒त्मा वा ए॒ष य॒ज्ञस्य॒ यदा᳚ग्रय॒णो यद्ग्रहो॑ वा क॒लशो॑ वोप॒दस्ये॑दाग्रय॒णाद्गृ॑ह्णीयादा॒त्मन॑ ए॒वाधि॑ य॒ज्ञं निष्क॑रो॒त्यवि॑ज्ञातो॒ वा ए॒ष गृ॑ह्यते॒ यदा᳚ग्रय॒णः स्था॒ल्या गृ॒ह्णाति॑ वाय॒व्ये॑न जुहोति॒ तस्मा᳚त्~(३७)

%6.5.10.3
गर्भे॒णावि॑ज्ञातेन ब्रह्म॒हाव॑भृ॒थमव॑ यन्ति॒ परा᳚ स्था॒लीरस्य॒न्त्युद्वा॑य॒व्या॑नि हरन्ति॒ तस्मा॒थ्स्त्रियं॑ जा॒तां परा᳚स्य॒न्त्यु\-त्पुमाꣳ॑सꣳ हरन्ति॒ यत्पु॑रो॒रुच॒माह॒ यथा॒ वस्य॑स आ॒हर॑ति ता॒दृगे॒व तद्यद्ग्रहं॑ गृ॒ह्णाति॒ यथा॒ वस्य॑स आ॒हृत्य॒ प्राह॑ ता॒दृगे॒व तद्यथ्सा॒दय॑ति॒ यथा॒ वस्य॑स उपनि॒धाया॑प॒क्राम॑ति ता॒दृगे॒व तद्यद्वै य॒ज्ञस्य॒ साम्ना॒ यजु॑षा क्रि॒यते॑ शिथि॒लं तद्यदृ॒चा तद्दृ॒ढं पु॒रस्ता॑दुपयामा॒ यजु॑षा गृह्यन्त उ॒परि॑ष्टादुपयामा ऋ॒चा य॒ज्ञस्य॒ धृत्यै᳚॥~(३८)

%6.5.11.0
{\anuvakamend[{यथा॑ पि॒ता तस्मा॑दप॒क्राम॑ति ता॒दृगे॒व तद्यद॒ष्टाद॑श च}]}%॥10॥

%6.5.11.1
प्रान्यानि॒ पात्रा॑णि यु॒ज्यन्ते॒ नान्यानि॒ यानि॑ परा॒चीना॑नि प्रयु॒ज्यन्ते॒\-ऽमुमे॒व तैर्लो॒कम॒भि ज॑यति॒ परा॑ङिव॒ ह्य॑सौ लो॒को यानि॒ पुनः॑ प्रयु॒ज्यन्त॑ इ॒ममे॒व तैर्लो॒कम॒भि ज॑यति॒ पुनः॑पुनरिव॒ ह्य॑यं लो॒कः प्रान्यानि॒ पात्रा॑णि यु॒ज्यन्ते॒ नान्यानि॒ यानि॑ परा॒चीना॑नि प्रयु॒ज्यन्ते॒ तान्यन्वोष॑धयः॒ परा॑ भवन्ति॒ यानि॒ पुनः॑~(३९)

%6.5.11.2
प्र॒यु॒ज्यन्ते॒ तान्यन्वोष॑धयः॒ पुन॒रा भ॑वन्ति॒ प्रान्यानि॒ पात्रा॑णि यु॒ज्यन्ते॒ नान्यानि॒ यानि॑ परा॒चीना॑नि प्रयु॒ज्यन्ते॒ तान्यन्वा॑र॒ण्याः प॒शवो\-ऽर॑ण्य॒मप॑ यन्ति॒ यानि॒ पुनः॑ प्रयु॒ज्यन्ते॒ तान्यनु॑ ग्रा॒म्याः प॒शवो॒ ग्राम॑मु॒पाव॑यन्ति॒ यो वै ग्रहा॑णां नि॒दानं॒ वेद॑ नि॒दान॑वान्भव॒त्याज्य॒मित्यु॒क्थं तद्वै ग्रहा॑णां नि॒दानं॒ यदु॑पा॒ꣳ॒शु शꣳस॑ति॒ तत्~(४०)

%6.5.11.3
उ॒पा॒ꣳ॒श्व॒न्त॒र्या॒मयो॒र्यदु॒च्चैस्तदित॑रेषां॒ ग्रहा॑णामे॒तद्वै ग्रहा॑णां नि॒दानं॒ य ए॒वं वेद॑ नि॒दान॑वान्भवति॒ यो वै ग्रहा॑णां मिथु॒नं वेद॒ प्र प्र॒जया॑ प॒शुभि॑र्मिथु॒नैर्जा॑यते स्था॒लीभि॑र॒न्ये ग्रहा॑ गृ॒ह्यन्ते॑ वाय॒व्यै॑र॒न्य ए॒तद्वै ग्रहा॑णां मिथु॒नं य ए॒वं वेद॒ प्र प्र॒जया॑ प॒शुभि॑र्मिथु॒नैर्जा॑यत॒ इन्द्र॒स्त्वष्टुः॒ सोम॑मभी॒षहा॑पिब॒थ्स विष्वङ्ङ्॑~(४१)

%6.5.11.4
व्या᳚र्च्छ॒थ्स आ॒त्मन्ना॒रम॑णं॒ नावि॑न्द॒थ्स ए॒तान॑नुसव॒नं पु॑रो॒डाशा॑नपश्य॒त्तां निर॑वप॒त्तैर्वै स आ॒त्मन्ना॒रम॑णमकुरुत॒ तस्मा॑दनुसव॒नं पु॑रो॒डाशा॒ निरु॑प्यन्ते॒ तस्मा॑दनुसव॒नं पु॑रो॒डाशा॑नां॒ प्राश्ञी॑यादा॒त्मन्ने॒वारम॑णं कुरुते॒ नैन॒ꣳ॒ सोमो\-ऽति॑ पवते ब्रह्मवा॒दिनो॑ वदन्ति॒ नर्चा न यजु॑षा प॒ङ्क्तिरा᳚प्य॒ते\-ऽथ॒ किं य॒ज्ञस्य॑ पाङ्क्त॒त्वमिति॑ धा॒नाः क॑र॒म्भः प॑रिवा॒पः पु॑रो॒डाशः॑ पय॒स्या॑ तेन॑ प॒ङ्क्तिरा᳚प्यते॒ तद्य॒ज्ञस्य॑ पाङ्क्त॒त्वम्॥~(४२)

%6.6.0.0
{\anuvakamend[{भ॒व॒न्ति॒ यानि॒ पुनः॒ शꣳस॑ति॒ तद्विष्व॒ङ्किञ्चतु॑र्दश च}]}%॥11॥

%6.6.0.0

{\anuvakamend[{सु॒व॒र्गाय॒ यद्दा᳚क्षि॒णानि॑ समिष्टय॒जूꣳष्य॑वभृथय॒जूꣳषि॒ स्फ्येन॑ प्र॒जा\-प॑तिरेकाद॒शिनी॒मिन्द्रः॒ पत्नि॑या॒ घ्नन्ति॑ दे॒वा वा इ॑न्द्रि॒यं दे॒वा वा अदा᳚भ्ये दे॒वा वै प्र॒बाहु॑क्प्र॒जा\-प॑तिर्दे॒वेभ्यः॒ स रि॑रिचा॒नः षो॑डश॒धैका॑\-दश}]}%॥11॥ 
{\prashnaend{सु॒व॒र्गाय॑ यजति प्र॒जाः सौ॒म्येन॑ गृह्णी॒यात्प्र॒त्यञ्चं॑ गृह्णी॒यात्प्र॒जां प॒शून्त्रिच॑त्वारिꣳशत्॥43॥ सु॒व॒र्गाय॒ वज्र॑स्य रू॒पꣳ समृ॑द्ध्यै॥}}
%%% END PRASHNA

\sect{षष्ठमः प्रश्नः}\setcounter{anuvakam}{0}
\dnsub{तैत्तिरीयसंहितायां षष्ठमकाण्डे षष्ठमः प्रश्नः}
%6.6.1.0
%6.6.1.1
सु॒व॒र्गाय॒ वा ए॒तानि॑ लो॒काय॑ हूयन्ते॒ यद्दा᳚क्षि॒णानि॒ द्वाभ्यां॒ गार्\mbox{}ह॑पत्ये जुहोति द्वि॒पाद्यज॑मानः॒ प्रति॑ष्ठित्या॒ आग्नी᳚ध्रे जुहोत्य॒न्तरि॑क्ष ए॒वा क्र॑मते॒ सदो॒\-ऽभ्यैति॑ सुव॒र्गमे॒वैनं॑ लो॒कं ग॑मयति सौ॒रीभ्या॑मृ॒ग्भ्यां गार्\mbox{}ह॑पत्ये जुहोत्य॒मुमे॒वैनं॑ लो॒कꣳ स॒मारो॑हयति॒ नय॑वत्य॒र्चाग्नी᳚ध्रे जुहोति सुव॒र्गस्य॑ लो॒कस्या॒भिनी᳚त्यै॒ दिवं॑ गच्छ॒ सुवः॑ प॒तेति॒ हिर॑ण्यम्~(१)

%6.6.1.2
हु॒त्वोद्गृ॑ह्णाति सुव॒र्गमे॒वैनं॑ लो॒कङ्ग॑मयति रू॒पेण॑ वो रू॒पम॒भ्यैमीत्या॑ह रू॒पेण॒ ह्या॑साꣳ रू॒पम॒भ्यैति॒ यद्धिर॑ण्येन तु॒थो वो॑ वि॒श्ववे॑दा॒ वि भ॑ज॒त्वित्या॑ह तु॒थो ह॑ स्म॒ वै वि॒श्ववे॑दा दे॒वानां॒ दक्षि॑णा॒ वि भ॑जति॒ तेनै॒वैना॒ वि भ॑जत्ये॒तत्ते॑ अग्ने॒ राधः॑~(२)

%6.6.1.3
ऐति॒ सोम॑च्युत॒मित्या॑ह॒ सोम॑च्युत॒ꣴ॒ ह्य॑स्य॒ राध॒ ऐति॒ तन्मि॒त्रस्य॑ प॒था न॒येत्या॑ह॒ शान्त्या॑ ऋ॒तस्य॑ प॒था प्रेत॑ च॒न्द्रद॑क्षिणा॒ इत्या॑ह स॒त्यं वा ऋ॒तꣳ स॒त्येनै॒वैना॑ ऋ॒तेन॒ वि भ॑जति य॒ज्ञस्य॑ प॒था सु॑वि॒ता नय॑न्ती॒रित्या॑ह य॒ज्ञस्य॒ ह्ये॑ताः प॒था यन्ति॒ यद्दक्षि॑णा ब्राह्म॒णम॒द्य रा᳚ध्यासम्~(३)

%6.6.1.4
ऋषि॑मार्\mbox{}षे॒यमित्या॑है॒ष वै ब्रा᳚ह्म॒ण ऋषि॑रार्\mbox{}षे॒यो यः शु॑श्रु॒वान्तस्मा॑दे॒वमा॑ह॒ वि सुवः॒ पश्य॒ व्य॑न्तरि॑क्ष॒मित्या॑ह सुव॒र्गमे॒वैनं॑ लो॒कं ग॑मयति॒ यत॑स्व सद॒स्यै॑रित्या॑ह मित्र॒त्वाया॒स्मद्दा᳚त्रा देव॒त्रा ग॑च्छत॒ मधु॑मतीः प्र दा॒तार॒मा वि॑श॒तेत्या॑ह व॒यमि॒ह प्र॑दा॒तारः॒ स्मो᳚\-ऽस्मान॒मुत्र॒ मधु॑मती॒रा वि॑श॒तेति॑~(४)

%6.6.1.5
वावैतदा॑ह॒ हिर॑ण्यं ददाति॒ ज्योति॒र्वै हिर॑ण्यं॒ ज्योति॑रे॒व पु॒रस्ता᳚द्धत्ते सुव॒र्गस्य॑ लो॒कस्यानु॑ख्यात्या अ॒ग्नीधे॑ ददात्य॒ग्निमु॑खाने॒वर्तून्प्री॑णाति ब्र॒ह्मणे॑ ददाति॒ प्रसू᳚त्यै॒ होत्रे॑ ददात्या॒त्मा वा ए॒ष य॒ज्ञस्य॒ यद्धोता॒त्मान॑मे॒व य॒ज्ञस्य॒ दक्षि॑णाभिः॒ सम॑र्धयति॥~(५)

%6.6.2.0
{\anuvakamend[{हिर॑ण्य॒ꣳ॒ राधो॑ राध्यासम॒मुत्र॒ मधु॑मती॒रा वि॑श॒तेत्य॒ष्टात्रिꣳ॑शच्च}]}%~(१)

%6.6.2.1
स॒मि॒ष्ट॒य॒जूꣳषि॑ जुहोति य॒ज्ञस्य॒ समि॑ष्ट्यै॒ यद्वै य॒ज्ञस्य॑ क्रू॒रं यद्विलि॑ष्टं॒ यद॒त्येति॒ यन्नात्येति॒ यद॑तिक॒रोति॒ यन्नापि॑ क॒रोति॒ तदे॒व तैः प्री॑णाति॒ नव॑ जुहोति॒ नव॒ वै पुरु॑षे प्रा॒णाः पुरु॑षेण य॒ज्ञः सम्मि॑तो॒ यावा॑ने॒व य॒ज्ञस्तं प्री॑णाति॒ षडृग्मि॑याणि जुहोति॒ षड्वा ऋ॒तव॑ ऋ॒तूने॒व प्री॑णाति॒ त्रीणि॒ यजूꣳ॑षि~(६)

%6.6.2.2
त्रय॑ इ॒मे लो॒का इ॒माने॒व लो॒कान्प्री॑णाति॒ यज्ञ॑ य॒ज्ञं ग॑च्छ य॒ज्ञप॑तिं ग॒च्छेत्या॑ह य॒ज्ञप॑तिमे॒वैनं॑ गमयति॒ स्वां योनिं॑ ग॒च्छेत्या॑ह॒ स्वामे॒वैनं॒ योनिं॑ गमयत्ये॒ष ते॑ य॒ज्ञो य॑ज्ञपते स॒हसू᳚क्तवाकः सु॒वीर॒ इत्या॑ह॒ यज॑मान ए॒व वी॒र्यं॑ दधाति वासि॒ष्ठो ह॑ सात्यह॒व्यो दे॑वभा॒गं प॑प्रच्छ॒ यथ्सृञ्ज॑यान्बहुया॒जिनो\-ऽयी॑यजो य॒ज्ञे~(७)

%6.6.2.3
य॒ज्ञं प्रत्य॑तिष्ठि॒पा(३)य॒ज्ञप॒ता(३)विति॒ स हो॑वाच य॒ज्ञप॑ता॒विति॑ स॒त्याद्वै सृञ्ज॑याः॒ परा॑ बभूवु॒रिति॑ होवाच य॒ज्ञे वाव य॒ज्ञः प्र॑ति॒ष्ठाप्य॑ आसी॒द्यज॑मान॒स्याप॑राभावा॒येति॒ देवा॑ गातुविदो गा॒तुं वि॒त्त्वा गा॒तुमि॒तेत्या॑ह य॒ज्ञ ए॒व य॒ज्ञं प्रति॑\-ष्ठापयति॒ यज॑मान॒स्याप॑राभावाय॥~(८)

%6.6.3.0
{\anuvakamend[{यजूꣳ॑षि य॒ज्ञ एक॑चत्वारिꣳशच्च}]}%~(२)

%6.6.3.1
अ॒व॒भृ॒थ॒य॒जूꣳषि॑ जुहोति॒ यदे॒वार्वा॒चीन॒मेक॑हायना॒देनः॑ क॒रोति॒ तदे॒व तैरव॑ यजते॒\-ऽपो॑\-ऽवभृ॒थमवै᳚त्य॒फ्सु वै वरु॑णः सा॒क्षादे॒व वरु॑ण॒मव॑ यजते॒ वर्त्म॑ना॒ वा अ॒न्वित्य॑ य॒ज्ञꣳ रक्षाꣳ॑सि जिघाꣳसन्ति॒ साम्ना᳚ प्रस्तो॒तान्ववै॑ति॒ साम॒ वै र॑क्षो॒हा रक्ष॑सा॒मप॑हत्यै॒ त्रिर्नि॒धन॒मुपै॑ति॒ त्रय॑ इ॒मे लो॒का ए॒भ्य ए॒व लो॒केभ्यो॒ रक्षाꣳ॑सि~(९)

%6.6.3.2
अप॑ हन्ति॒ पुरु॑षःपुरुषो नि॒धन॒मुपै॑ति॒ पुरु॑षःपुरुषो॒ हि र॑क्ष॒स्वी रक्ष॑सा॒मप॑हत्या उ॒रुꣳ हि राजा॒ वरु॑णश्च॒कारेत्या॑ह॒ प्रति॑ष्ठित्यै श॒तं ते॑ राजन्भि॒षजः॑ स॒हस्र॒मित्या॑ह भेष॒जमे॒वास्मै॑ करोत्य॒भिष्ठि॑तो॒ वरु॑णस्य॒ पाश॒ इत्या॑ह वरुणपा॒शमे॒वाभि ति॑ष्ठति ब॒र्॒\mbox{}हिर॒भि जु॑हो॒त्याहु॑तीनां॒ प्रति॑ष्ठित्या॒ अथो॑ अग्नि॒वत्ये॒व जु॑हो॒त्यप॑बर्\mbox{}हिषः प्रया॒जान्~(१०)

%6.6.3.3
य॒ज॒ति॒ प्र॒जा वै ब॒र्॒\mbox{}हिः प्र॒जा ए॒व व॑रुणपा॒शान्मु॑ञ्च॒त्याज्य॑भागौ यजति य॒ज्ञस्यै॒व चक्षु॑षी॒ नान्तरे॑ति॒ वरु॑णं यजति वरुणपा॒शादे॒वैनं॑ मुञ्चत्य॒ग्नीवरु॑णौ यजति सा॒क्षादे॒वैनं॑ वरुणपा॒शान्मु॑ञ्च॒त्यप॑बर्\mbox{}हिषावनूया॒जौ य॑जति प्र॒जा वै ब॒र्॒\mbox{}हिः प्र॒जा ए॒व व॑रुणपा॒शान्मु॑ञ्चति च॒तुरः॑ प्रया॒जान् य॑जति॒ द्वाव॑नूया॒जौ षट्थ्सम्प॑द्यन्ते॒ षड्वा ऋ॒तवः॑~(११)

%6.6.3.4
ऋ॒तुष्वे॒व प्रति॑ तिष्ठ॒त्यव॑भृथ निचङ्कु॒णेत्या॑ह यथोदि॒तमे॒व वरु॑ण॒मव॑ यजते समु॒द्रे ते॒ हृद॑यम॒फ्स्व॑न्तरित्या॑ह समु॒द्रे ह्य॑न्तर्वरु॑णः॒ सं त्वा॑ विश॒न्त्वोष॑धीरु॒ताप॒ इत्या॑हा॒द्भिरे॒वैन॒मोष॑धीभिः स॒म्यञ्चं॑ दधाति॒ देवी॑राप ए॒ष वो॒ गर्भ॒ इत्या॑ह यथाय॒जुरे॒वैतत्प॒शवो॒ वै~(१२)

%6.6.3.5
सोमो॒ यद्भि॑न्दू॒नां भ॒क्षये᳚त्पशु॒मान्थ्स्या॒द्वरु॑ण॒स्त्वे॑नं गृह्णीया॒द्यन्न भ॒क्षये॑दप॒शुः स्या॒न्नैनं॒ वरु॑णो गृह्णीयादुप॒स्पृश्य॑मे॒व प॑शु॒मान्भ॑वति॒ नैनं॒ वरु॑णो गृह्णाति॒ प्रति॑युतो॒ वरु॑णस्य॒ पाश॒ इत्या॑ह वरुणपा॒शादे॒व निर्मु॑च्य॒ते\-ऽप्र॑तीक्ष॒मा य॑न्ति॒ वरु॑णस्या॒न्तर्\mbox{}हि॑त्या॒ एधो᳚\-ऽस्येधिषीम॒हीत्या॑ह स॒मिधै॒वाग्निं न॑म॒स्यन्त॑ उ॒पाय॑न्ति॒ तेजो॑\-ऽसि॒ तेजो॒ मयि॑ धे॒हीत्या॑ह॒ तेज॑ ए॒वात्मन्ध॑त्ते॥~(१३)

%6.6.4.0
{\anuvakamend[{रक्षाꣳ॑सि प्रया॒जानृ॒तवो॒ वै न॑म॒स्यन्तो॒ द्वाद॑श च}]}%~(३)

%6.6.4.1
स्फ्येन॒ वेदि॒मुद्ध॑न्ति रथा॒क्षेण॒ वि मि॑मीते॒ यूपं॑ मिनोति त्रि॒वृत॑मे॒व वज्रꣳ॑ स॒म्भृत्य॒ भ्रातृ॑व्याय॒ प्र ह॑रति॒ स्तृत्यै॒ यद॑न्तर्वे॒दि मि॑नु॒याद्दे॑वलो॒कम॒भि ज॑ये॒द्यद्ब॑हिर्वे॒दि म॑नुष्यलो॒कं वे᳚द्य॒न्तस्य॑ सं॒धौ मि॑नोत्यु॒भयो᳚र्लो॒कयो॑र॒भिजि॑त्या॒ उप॑रसम्मितां मिनुयात्पितृलो॒कका॑मस्य रश॒नस॑म्मितां मनुष्यलो॒कका॑मस्य च॒षाल॑सम्मितामिन्द्रि॒यका॑मस्य॒ सर्वा᳚न्थ्स॒मान्प्र॑ति॒ष्ठाका॑मस्य॒ ये त्रयो॑ मध्य॒मास्तान्थ्स॒मान्प॒शुका॑मस्यै॒तान् वै~(१४)

%6.6.4.2
अनु॑ प॒शव॒ उप॑ तिष्ठन्ते पशु॒माने॒व भ॑वति॒ व्यति॑षजे॒दित॑रान्प्र॒जयै॒वैनं॑ प॒शुभि॒र्व्यति॑षजति॒ यं का॒मये॑त प्र॒मायु॑कः स्या॒दिति॑ गर्त॒मितं॒ तस्य॑ मिनुयादुत्तरा॒र्ध्यं॑ वर्\mbox{}षि॑ष्ठ॒मथ॒ ह्रसी॑याꣳसमे॒षा वै ग॑र्त॒मिद्यस्यै॒वं मि॒नोति॑ ता॒जक्प्र मी॑यते दक्षिणा॒र्ध्यं॑ वर्\mbox{}षि॑ष्ठं मिनुयाथ्सुव॒र्गका॑म॒स्याथ॒ ह्रसी॑याꣳसमा॒क्रम॑णमे॒व तथ्सेतुं॒ यज॑मानः कुरुते सुव॒र्गस्य॑ लो॒कस्य॒ सम॑ष्ट्यै~(१५)

%6.6.4.3
यदेक॑स्मि॒न्॒ यूपे॒ द्वे र॑श॒ने प॑रि॒व्यय॑ति॒ तस्मा॒देको॒ द्वे जा॒ये वि॑न्दते॒ यन्नैकाꣳ॑ रश॒नां द्वयो॒र्यूप॑योः परि॒व्यय॑ति॒ तस्मा॒न्नैका॒ द्वौ पती॑ विन्दते॒ यं का॒मये॑त॒ स्त्र्य॑स्य जाये॒तेत्यु॑पा॒न्ते तस्य॒ व्यति॑षजे॒थ्स्त्र्ये॑वास्य॑ जायते॒ यं का॒मये॑त॒ पुमा॑नस्य जाये॒तेत्या॒न्तं तस्य॒ प्र वे᳚ष्टये॒त्पुमा॑ने॒वास्य॑~(१६)

%6.6.4.4
जा॒य॒ते\-ऽसु॑रा॒ वै दे॒वान्द॑क्षिण॒त उपा॑नय॒न्तां दे॒वा उ॑पश॒येनै॒वापा॑नुदन्त॒ तदु॑पश॒यस्यो॑पशय॒त्वं यद्द॑क्षिण॒त उ॑पश॒य उ॑प॒शये॒ भ्रातृ॑व्यापनुत्त्यै॒ सर्वे॒ वा अ॒न्ये यूपाः᳚ पशु॒मन्तो\-ऽथो॑पश॒य ए॒वाप॒शुस्तस्य॒ यज॑मानः प॒शुर्यन्न नि॑र्दि॒शेदार्ति॒\-मार्च्छे॒द्यज॑मानो॒\-ऽसौ ते॑ प॒शुरिति॒ निर्दि॑शे॒द्यं द्वि॒ष्याद्यमे॒व~(१७)

%6.6.4.5
द्वेष्टि॒ तम॑स्मै प॒शुं निर्दि॑शति॒ यदि॒ न द्वि॒ष्यादा॒खुस्ते॑ प॒शुरिति॑ ब्रूया॒न्न ग्रा॒म्यान्प॒शून् हि॒नस्ति॒ नार॒ण्यान्प्र॒जा\-प॑तिः प्र॒जा अ॑सृजत॒ सो᳚\-ऽन्नाद्ये॑न॒ व्या᳚र्ध्यत॒ स ए॒तामे॑काद॒शिनी॑मपश्य॒त्तया॒ वै सो᳚\-ऽन्नाद्य॒मवा॑रुन्ध॒ यद्दश॒ यूपा॒ भव॑न्ति॒ दशा᳚क्षरा वि॒राडन्नं॑ वि॒राड्वि॒राजै॒वान्नाद्य॒मव॑ रुन्धे~(१८)

%6.6.4.6
य ए॑काद॒शः स्तन॑ ए॒वास्यै॒ स दु॒ह ए॒वैनां॒ तेन॒ वज्रो॒ वा ए॒षा सम्मी॑यते॒ यदे॑काद॒शिनी॒ सेश्व॒रा पु॒रस्ता᳚त्प्र॒त्यञ्चं॑ य॒ज्ञꣳ सम्म॑र्दितो॒र्यत्पा᳚त्नीव॒तं मि॒नोति॑ य॒ज्ञस्य॒ प्रत्युत्त॑ब्ध्यै सय॒त्वाय॑॥~(१९)

%6.6.5.0
{\anuvakamend[{वै सम॑ष्ट्यै॒ पुमा॑ने॒वास्य॒ यमे॒व रु॑न्धे त्रि॒ꣳ॒शच्च॑}]}%~(४)

%6.6.5.1
प्र॒जा\-प॑तिः प्र॒जा अ॑सृजत॒ स रि॑रिचा॒नो॑\-ऽमन्यत॒ स ए॒तामे॑काद॒शिनी॑मपश्य॒त्तया॒ वै स आयु॑रिन्द्रि॒यं वी॒र्य॑मा॒त्मन्न॑धत्त प्र॒जा इ॑व॒ खलु॒ वा ए॒ष सृ॑जते॒ यो यज॑ते॒ स ए॒तर्\mbox{}हि॑ रिरिचा॒न इ॑व॒ यदे॒षैका॑द॒शिनी॒ भव॒त्यायु॑रे॒व तये᳚न्द्रि॒यं वी॒र्यं॑ यज॑मान आ॒त्मन्ध॑त्ते॒ प्रैवाऽऽग्ने॒येन॑ वापयति मिथु॒नꣳ सा॑रस्व॒त्या क॑रोति॒ रेतः॑~(२०)

%6.6.5.2
सौ॒म्येन॑ दधाति॒ प्र ज॑नयति पौ॒ष्णेन॑ बार्\mbox{}हस्प॒त्यो भ॑वति॒ ब्रह्म॒ वै दे॒वानां॒ बृह॒स्पति॒र्ब्रह्म॑णै॒वास्मै᳚ प्र॒जाः प्र ज॑नयति वैश्वदे॒वो भ॑वति वैश्वदे॒व्यो॑ वै प्र॒जाः प्र॒जा ए॒वास्मै॒ प्र ज॑नयतीन्द्रि॒यमे॒वैन्द्रेणाव॑रुन्धे॒ विशं॑ मारु॒तेनौजो॒ बल॑मैन्द्रा॒ग्नेन॑ प्रस॒वाय॑ सावि॒त्रो नि॑र्वरुण॒त्वाय॑ वारु॒णो म॑ध्य॒त ऐ॒न्द्रमा ल॑भते मध्य॒त ए॒वेन्द्रि॒यं यज॑माने दधाति~(२१)

%6.6.5.3
पु॒रस्ता॑दै॒न्द्रस्य॑ वैश्वदे॒वमाल॑भते वैश्वदे॒वं वा अन्न॒मन्न॑मे॒व पु॒रस्ता᳚द्धत्ते॒ तस्मा᳚त्पु॒रस्ता॒दन्न॑मद्यत ऐ॒न्द्रमा॒लभ्य॑ मारु॒तमा ल॑भते॒ विड्वै म॒रुतो॒ विश॑मे॒वास्मा॒ अनु॑ बध्नाति॒ यदि॑ का॒मये॑त॒ यो\-ऽव॑गतः॒ सो\-ऽप॑ रुध्यतां॒ यो\-ऽप॑रुद्धः॒ सो\-ऽव॑ गच्छ॒त्वित्यै॒न्द्रस्य॑ लो॒के वा॑रु॒णमा ल॑भेत वारु॒णस्य॑ लो॒क ऐ॒न्द्रम्~(२२)

%6.6.5.4
य ए॒वाव॑गतः॒ सो\-ऽप॑ रुध्यते॒ यो\-ऽप॑रुद्धः॒ सो\-ऽव॑ गच्छति॒ यदि॑ का॒मये॑त प्र॒जा मु॑ह्येयु॒रिति॑ प॒शून्व्यति॑षजेत्प्र॒जा ए॒व मो॑हयति॒ यद॑भिवाह॒तो॑\-ऽपां वा॑रु॒णमा॒लभे॑त प्र॒जा वरु॑णो गृह्णीयाद्दक्षिण॒त उद॑ञ्च॒मा ल॑भते\-ऽपवाह॒तो॑\-ऽ\-पां प्र॒जाना॒मव॑रुणग्राहाय॥~(२३)

%6.6.6.0
{\anuvakamend[{रेतो॒ यज॑माने दधाति लो॒क ऐ॒न्द्रꣳ स॒प्तत्रिꣳ॑शच्च}]}%~(५)

%6.6.6.1
इन्द्रः॒ पत्नि॑या॒ मनु॑मयाजय॒त्तां पर्य॑ग्निकृता॒मुद॑सृज॒त्तया॒ मनु॑रार्ध्नो॒द्यत्पर्य॑ग्निकृतं पात्नीव॒तमु॑थ्सृ॒जति॒ यामे॒व मनु॒र्॒\mbox{}ऋद्धि॒\-मार्ध्नो॒त्तामे॒व यज॑मान ऋध्नोति य॒ज्ञस्य॒ वा अप्र॑तिष्ठिताद्य॒ज्ञः परा॑ भवति य॒ज्ञं प॑रा॒भव॑न्तं॒ यज॑मा॒नो\-ऽनु॒ परा॑ भवति॒ यदाज्ये॑न पात्नीव॒तꣳ सꣴ॑स्था॒पय॑ति य॒ज्ञस्य॒ प्रति॑ष्ठित्यै य॒ज्ञं प्र॑ति॒तिष्ठ॑न्तं॒ यज॑मा॒नो\-ऽनु॒ प्रति॑ तिष्ठती॒ष्टं व॒पया᳚~(२४)

%6.6.6.2
भव॒त्यनि॑ष्टं व॒शयाथ॑ पात्नीव॒तेन॒ प्र च॑रति ती॒र्थ ए॒व प्र च॑र॒त्यथो॑ ए॒तर्\mbox{}ह्ये॒वास्य॒ याम॑स्त्वा॒ष्ट्रो भ॑वति॒ त्वष्टा॒ वै रेत॑सः सि॒क्तस्य॑ रू॒पाणि॒ वि क॑रोति॒ तमे॒व वृ॑षाणं॒ पत्नी॒ष्वपि॑ सृजति॒ सो᳚\-ऽस्मै रू॒पाणि॒ वि क॑रोति॥~(२५)

%6.6.7.0
{\anuvakamend[{व॒पया॒ षट्त्रिꣳ॑शच्च}]}%~(६)

%6.6.7.1
घ्नन्ति॒ वा ए॒तथ्सोमं॒ यद॑भिषु॒ण्वन्ति॒ यथ्सौ॒म्यो भव॑ति॒ यथा॑ मृ॒ताया॑नु॒स्तर॑णीं॒ घ्नन्ति॑ ता॒दृगे॒व तद्यदु॑त्तरा॒र्धे वा॒ मध्ये॑ वा जुहु॒याद्दे॒वता᳚भ्यः स॒मदं॑ दध्याद्दक्षिणा॒र्धे जु॑होत्ये॒षा वै पि॑तृ॒णां दिख्स्वाया॑मे॒व दि॒शि पि॒तॄन्नि॒रव॑दयत उद्गा॒तृभ्यो॑ हरन्ति सामदेव॒त्यो॑ वै सौ॒म्यो यदे॒व साम्न॑श्छम्बट्कु॒र्वन्ति॒ तस्यै॒व स शान्ति॒रव॑~(२६)

%6.6.7.2
ई॒क्ष॒न्ते॒ प॒वित्रं॒ वै सौ॒म्य आ॒त्मान॑मे॒व प॑वयन्ते॒ य आ॒त्मानं॒ न प॑रि॒पश्ये॑दि॒तासुः॑ स्यादभिद॒दिं कृ॒त्वावे᳚क्षेत॒ तस्मि॒न् ह्या᳚त्मानं॑ परि॒पश्य॒त्यथो॑ आ॒त्मान॑मे॒व प॑वयते॒ यो ग॒तम॑नाः॒ स्याथ्सो\-ऽवे᳚क्षेत॒ यन्मे॒ मनः॒ परा॑गतं॒ यद्वा॑ मे॒ अप॑रागतम्। राज्ञा॒ सोमे॑न॒ तद्व॒यम॒स्मासु॑ धारयाम॒सीति॒ मन॑ ए॒वात्मन्दा॑धार~(२७)

%6.6.7.3
न ग॒तम॑ना भव॒त्यप॒ वै तृ॑तीयसव॒ने य॒ज्ञः क्रा॑मतीजा॒नादनी॑जानम॒भ्या᳚ग्नावैष्ण॒व्यर्चा घृ॒तस्य॑ यजत्य॒ग्निः सर्वा॑ दे॒वता॒ विष्णु॑र्य॒ज्ञो दे॒वता᳚श्चै॒व य॒ज्ञं च॑ दाधारोपा॒ꣳ॒शु य॑जति मिथुन॒त्वाय॑ ब्रह्मवा॒दिनो॑ वदन्ति मि॒त्रो य॒ज्ञस्य॒ स्वि॑ष्टं युवते॒ वरु॑णो॒ दुरि॑ष्टं॒ क्व॑ तर्\mbox{}हि॑ य॒ज्ञः क्व॑ यज॑मानो भव॒तीति॒ यन्मै᳚त्रावरु॒णीं व॒शामा॒लभ॑ते मि॒त्रेणै॒व~(२८)

%6.6.7.4
य॒ज्ञस्य॒ स्वि॑ष्टꣳ शमयति॒ वरु॑णेन॒ दुरि॑ष्टं॒ नार्ति॒मार्च्छ॑ति॒ यज॑मानो॒ यथा॒ वै लाङ्ग॑लेनो॒र्वरां᳚ प्रभि॒न्दन्त्ये॒वमृ॑ख्सा॒मे य॒ज्ञं प्र भि॑न्तो॒ यन्मै᳚त्रावरु॒णीं व॒शामा॒लभ॑ते य॒ज्ञायै॒व प्रभि॑न्नाय म॒त्य॑म॒न्ववा᳚स्यति॒ शान्त्यै॑ या॒तया॑मानि॒ वा ए॒तस्य॒ छन्दाꣳ॑सि॒ य ई॑जा॒नश्छन्द॑सामे॒ष रसो॒ यद्व॒शा यन्मै᳚त्रावरु॒णीं व॒शामा॒लभ॑ते॒ छन्दाꣴ॑स्ये॒व पुन॒रा प्री॑णा॒त्यया॑तयामत्वा॒याथो॒ छन्दः॑स्वे॒व रसं॑ दधाति॥~(२९)

%6.6.8.0
{\anuvakamend[{अव॑ दाधार मि॒त्रेणै॒व प्री॑णाति॒ षट्च॑}]}%~(७)

%6.6.8.1
दे॒वा वा इ॑न्द्रि॒यं वी॒र्यं  व्य॑भजन्त॒ ततो॒ यद॒त्यशि॑ष्यत॒ तद॑तिग्रा॒ह्या॑ अभव॒न्तद॑तिग्रा॒ह्या॑णामतिग्राह्य॒त्वं यद॑तिग्रा॒ह्या॑ गृ॒ह्यन्त॑ इन्द्रि॒यमे॒व तद्वी॒र्यं॑ यज॑मान आ॒त्मन्ध॑त्ते॒ तेज॑ आग्ने॒येने᳚न्द्रि॒यमै॒न्द्रेण॑ ब्रह्मवर्च॒सꣳ सौ॒र्येणो॑प॒स्तम्भ॑नं॒ वा ए॒तद्य॒ज्ञस्य॒ यद॑तिग्रा॒ह्या᳚श्च॒क्रे पृ॒ष्ठानि॒ यत्पृष्ठ्ये॒ न गृ॑ह्णी॒यात्प्राञ्चं॑ य॒ज्ञं पृ॒ष्ठानि॒ सꣳ शृ॑णीयु॒र्यदु॒क्थ्ये᳚~(३०)

%6.6.8.2
गृ॒ह्णी॒यात्प्र॒त्यञ्चं॑ य॒ज्ञम॑तिग्रा॒ह्याः᳚ सꣳ शृ॑णीयुर्विश्व॒जिति॒ सर्व॑पृष्ठे ग्रहीत॒व्या॑ य॒ज्ञस्य॑ सवीर्य॒त्वाय॑ प्र॒जा\-प॑तिर्दे॒वेभ्यो॑ य॒ज्ञान्व्यादि॑श॒थ्स प्रि॒यास्त॒नूरप॒ न्य॑धत्त॒ तद॑तिग्रा॒ह्या॑ अभव॒न्वित॑नु॒स्तस्य॑ य॒ज्ञ इत्या॑हु॒र्यस्या॑तिग्रा॒ह्या॑ न गृ॒ह्यन्त॒ इत्यप्य॑ग्निष्टो॒मे ग्र॑हीत॒व्या॑ य॒ज्ञस्य॑ सतनु॒त्वाय॑ दे॒वता॒ वै सर्वाः᳚ स॒दृशी॑रास॒न्ता न व्या॒वृत᳚मगच्छ॒न्ते दे॒वाः~(३१)

%6.6.8.3
ए॒त ए॒तान्ग्रहा॑नपश्य॒न्तान॑गृह्णताग्ने॒यम॒ग्निरै॒न्द्रमिन्द्रः॑ सौ॒र्यꣳ सूर्य॒स्ततो॒ वै ते᳚\-ऽन्याभि॑र्दे॒वता॑भिर्व्या॒वृत॑मगच्छ॒न्॒ यस्यै॒वं वि॒दुष॑ ए॒ते ग्रहा॑ गृ॒ह्यन्ते᳚ व्या॒वृत॑मे॒व पा॒प्मना॒ भ्रातृ॑व्येण गच्छती॒मे लो॒का ज्योति॑ष्मन्तः स॒माव॑द्वीर्याः का॒र्या॑ इत्या॑हुराग्ने॒येना॒स्मिँल्लो॒के ज्योति॑र्धत्त ऐ॒न्द्रेणा॒न्तरि॑क्ष इन्द्रवा॒यू हि स॒युजौ॑ सौ॒र्येणा॒मुष्मिँ॑ल्लो॒के~(३२)

%6.6.8.4
ज्योति॑र्धत्ते॒ ज्योति॑ष्मन्तो\-ऽस्मा इ॒मे लो॒का भ॑वन्ति स॒माव॑द्वीर्यानेनान्कुरुत ए॒तान् वै ग्रहा᳚न्ब॒म्बावि॒श्वव॑यसाववित्तां॒ ताभ्या॑मि॒मे लो॒काः परा᳚ञ्चश्चा॒र्वाञ्च॑श्च॒ प्राभु॒र्यस्यै॒वं वि॒दुष॑ ए॒ते ग्रहा॑ गृ॒ह्यन्ते॒ प्रास्मा॑ इ॒मे लो॒काः परा᳚ञ्चश्चा॒र्वाञ्च॑श्च भान्ति॥~(३३)

%6.6.9.0
{\anuvakamend[{उ॒क्थ्ये॑ दे॒वा अ॒मुष्मिँ॑ल्लो॒क एका॒न्नच॑त्वारि॒ꣳ॒शच्च॑}]}%~(८)

%6.6.9.1
दे॒वा वै यद्य॒ज्ञे\-ऽकु॑र्वत॒ तदसु॑रा अकुर्वत॒ ते दे॒वा अदा᳚भ्ये॒ छन्दाꣳ॑सि॒ सव॑नानि॒ सम॑स्थापय॒न्ततो॑ दे॒वा अभ॑व॒न्परासु॑रा॒ यस्यै॒वं वि॒दुषो\-ऽदा᳚भ्यो गृ॒ह्यते॒ भव॑त्या॒त्मना॒ परा᳚स्य॒ भ्रातृ॑व्यो भवति॒ यद्वै दे॒वा असु॑रा॒नदा᳚भ्ये॒ना\-द॑भ्नुव॒न्तददा᳚भ्यस्यादाभ्य॒त्वं य ए॒वं वेद॑ द॒भ्नोत्ये॒व भ्रातृ॑व्यं॒ नैन॒म्भ्रातृ॑व्यो दभ्नोति~(३४)

%6.6.9.2
ए॒षा वै प्र॒जा\-प॑तेरतिमो॒क्षिणी॒ नाम॑ त॒नूर्यददा᳚भ्य॒ उप॑नद्धस्य गृह्णा॒त्यति॑मुक्त्या॒ अति॑ पा॒प्मान॒म्भ्रातृ॑व्यं मुच्यते॒ य ए॒वं वेद॒ घ्नन्ति॒ वा ए॒तथ्सोमं॒ यद॑भिषु॒ण्वन्ति॒ सोमे॑ ह॒न्यमा॑ने य॒ज्ञो ह॑न्यते य॒ज्ञे यज॑मानो ब्रह्मवा॒दिनो॑ वदन्ति॒ किं तद्य॒ज्ञे यज॑मानः कुरुते॒ येन॒ जीव᳚न्थ्सुव॒र्गं लो॒कमेतीति॑ जीवग्र॒हो वा ए॒ष यददा॒भ्यो\-ऽन॑भिषुतस्य गृह्णाति॒ जीव॑न्तमे॒वैनꣳ॑ सुव॒र्गं लो॒कं ग॑मयति॒ वि वा ए॒तद्य॒ज्ञं छि॑न्दन्ति॒ यददा᳚भ्ये सꣴस्था॒पय॑न्त्य॒ꣳ॒शूनपि॑ सृजति य॒ज्ञस्य॒ सन्त॑त्यै॥~(३५)

%6.6.10.0
{\anuvakamend[{द॒भ्नो॒त्यन॑भिषुतस्य गृह्णा॒त्येका॒न्नविꣳ॑श॒तिश्च॑}]}%~(९)

%6.6.10.1
दे॒वा वै प्र॒बाहु॒ग्ग्रहा॑नगृह्णत॒ स ए॒तं प्र॒जा\-प॑तिर॒ꣳ॒शुम॑पश्य॒त्तम॑गृह्णीत॒ तेन॒ वै स आ᳚र्ध्नो॒द्यस्यै॒वं वि॒दुषो॒\-ऽꣳ॒शुर्गृ॒ह्यत॑ ऋ॒ध्नोत्ये॒व स॒कृद॑भिषुतस्य गृह्णाति स॒कृद्धि स तेनार्ध्नो॒न्मन॑सा गृह्णाति॒ मन॑ इव॒ हि प्र॒जा\-प॑तिः प्र॒जा\-प॑ते॒राप्त्या॒ औदु॑म्बरेण गृह्णा॒त्यूर्ग्वा उ॑दु॒म्बर॒ ऊर्ज॑मे॒वाव॑ रुन्धे॒ चतुः॑स्रक्ति भवति दि॒क्षु~(३६)

%6.6.10.2
ए॒व प्रति॑ तिष्ठति॒ यो वा अ॒ꣳ॒शोरा॒यत॑नं॒ वेदा॒यत॑नवान्भवति वामदे॒व्यमिति॒ साम॒ तद्वा अ॑स्या॒यत॑नं॒ मन॑सा॒ गाय॑मानो गृह्णात्या॒यत॑नवाने॒व भ॑वति॒ यद॑ध्व॒र्युर॒ꣳ॒शुं गृ॒ह्णन्नार्धये॑दु॒भाभ्यां॒ नर्ध्ये॑ताध्व॒र्यवे॑ च॒ यज॑मानाय च॒ यद॒र्धये॑दु॒भाभ्या॑मृध्ये॒तान॑वानं गृह्णाति॒ सैवास्यर्द्धि॒र्॒\mbox{}हिर॑ण्यम॒भि व्य॑नित्य॒मृतं॒ वै हिर॑ण्य॒मायुः॑ प्रा॒ण आयु॑षै॒वामृत॑म॒भि धि॑नोति श॒तमा॑नं भवति श॒तायुः॒ पुरु॑षः श॒तेन्द्रि॑य॒ आयु॑ष्ये॒वेन्द्रि॒ये प्रति॑ तिष्ठति॥~(३७)

%6.6.11.0
{\anuvakamend[{दि॒क्ष्व॑निति विꣳश॒तिश्च॑}]}%॥10॥

%6.6.11.1
प्र॒जा\-प॑तिर्दे॒वेभ्यो॑ य॒ज्ञान्व्यादि॑श॒थ्स रि॑रिचा॒नो॑\-ऽमन्यत॒ स य॒ज्ञानाꣳ॑ षोडश॒धेन्द्रि॒यं वी॒र्य॑मा॒त्मान॑म॒भि सम॑क्खिद॒त् तथ्षो॑ड॒श्य॑भव॒न्न वै षो॑ड॒शी नाम॑ य॒ज्ञो᳚\-ऽस्ति॒ यद्वाव षो॑ड॒शꣴ स्तो॒त्रꣳ षो॑ड॒शꣳ श॒स्त्रं तेन॑ षोड॒शी तथ्षो॑ड॒शिनः॑ षोडशि॒त्वं यथ्षो॑ड॒शी गृ॒ह्यत॑ इन्द्रि॒यमे॒व तद्वी॒र्यं॑ यज॑मान आ॒त्मन्ध॑त्ते दे॒वेभ्यो॒ वै सु॑व॒र्गो लो॒कः~(३८)

%6.6.11.2
न प्राभ॑व॒त्त ए॒तꣳ षो॑ड॒शिन॑मपश्य॒न्तम॑गृह्णत॒ ततो॒ वै तेभ्यः॑ सुव॒र्गो लो॒कः प्राभ॑व॒द्यथ्षो॑ड॒शी गृ॒ह्यते॑ सुव॒र्गस्य॑ लो॒कस्या॒भिजि॑त्या॒ इन्द्रो॒ वै दे॒वाना॑मानुजाव॒र आ॑सी॒थ्स प्र॒जा\-प॑ति॒मुपा॑धाव॒त्तस्मा॑ ए॒तꣳ षो॑ड॒शिनं॒ प्राय॑च्छ॒त्तम॑गृह्णीत॒ ततो॒ वै सो\-ऽग्रं॑ दे॒वता॑नां॒ पर्यै॒द्यस्यै॒वं वि॒दुषः॑ षोड॒शी गृ॒ह्यते᳚~(३९)

%6.6.11.3
अग्र॑मे॒व स॑मा॒नानां॒ पर्ये॑ति प्रातःसव॒ने गृ॑ह्णाति॒ वज्रो॒ वै षो॑ड॒शी वज्रः॑ प्रातःसव॒नꣴ स्वादे॒वैनं॒ योने॒र्निर्गृ॑ह्णाति॒ सव॑नेसवने॒\-ऽभि गृ॑ह्णाति॒ सव॑नाथ्सवनादे॒वैनं॒ प्र ज॑नयति तृतीयसव॒ने प॒शुका॑मस्य गृह्णीया॒द्वज्रो॒ वै षो॑ड॒शी प॒शव॑स्तृतीयसव॒नं वज्रे॑णै॒वास्मै॑ तृतीयसव॒नात्प॒शूनव॑ रुन्धे॒ नोक्थ्ये॑ गृह्णीयात्प्र॒जा वै प॒शव॑ उ॒क्थानि॒ यदु॒क्थ्ये᳚~(४०)

%6.6.11.4
गृ॒ह्णी॒यात्प्र॒जां प॒शून॑स्य॒ निर्द॑हेदतिरा॒त्रे प॒शुका॑मस्य गृह्णीया॒द्वज्रो॒ वै षो॑ड॒शी वज्रे॑णै॒वास्मै॑ प॒शून॑व॒रुध्य॒ रात्रि॑यो॒परि॑ष्टाच्छमय॒त्यप्य॑ग्निष्टो॒मे रा॑ज॒न्य॑स्य गृह्णीयाद्व्या॒वृत्का॑मो॒ हि रा॑ज॒न्यो॑ यज॑ते सा॒ह्न ए॒वास्मै॒ वज्रं॑ गृह्णाति॒ स ए॑नं॒ वज्रो॒ भूत्या॑ इन्द्धे॒ निर्वा दहत्येकवि॒ꣳ॒शꣴ स्तो॒त्रं भ॑वति॒ प्रति॑ष्ठित्यै॒ हरि॑वच्छस्यत॒ इन्द्र॑स्य प्रि॒यं धाम॑~(४१)

%6.6.11.5
उपा᳚प्नोति॒ कनी॑याꣳसि॒ वै दे॒वेषु॒ छन्दा॒ꣴ॒स्यास॒ञ्ज्याया॒ꣴ॒स्यसु॑रेषु॒ ते दे॒वाः कनी॑यसा॒ छन्द॑सा॒ ज्याय॒श्छन्दो॒\-ऽभि व्य॑शꣳस॒न्ततो॒ वै ते\-ऽसु॑राणां लो॒कम॑वृञ्जत॒ यत्कनी॑यसा॒ छन्द॑सा॒ ज्याय॒श्छन्दो॒\-ऽभि वि॒शꣳस॑ति॒ भ्रातृ॑व्यस्यै॒व तल्लो॒कं वृ॑ङ्क्ते॒ षड॒क्षरा॒ण्यति॑ रेचयन्ति॒ षड्वा ऋ॒तव॑ ऋ॒तूने॒व प्री॑णाति च॒त्वारि॒ पूर्वा॒ण्यव॑ कल्पयन्ति~(४२)

%6.6.11.6
चतु॑ष्पद ए॒व प॒शूनव॑ रुन्धे॒ द्वे उत्त॑रे द्वि॒पद॑ ए॒वाव॑ रुन्धे\-ऽनु॒ष्टुभ॑म॒भि सम्पा॑दयन्ति॒ वाग्वा अ॑नु॒ष्टुप्तस्मा᳚त्प्रा॒णानां॒ वागु॑त्त॒मा स॑मयाविषि॒ते सूर्ये॑ षोड॒शिनः॑ स्तो॒त्रमु॒पाक॑रोत्ये॒तस्मि॒न्वै लो॒क इन्द्रो॑ वृ॒त्रम॑हन्थ्सा॒क्षादे॒व वज्र॒म्भ्रातृ॑व्याय॒ प्र ह॑रत्यरुणपिशं॒गो\-ऽश्वो॒ दक्षि॑णै॒तद्वै वज्र॑स्य रू॒पꣳ समृ॑द्ध्यै~(४३)

%7.1.0.0

%7.1.0.0
{\anuvakamend[{लो॒को वि॒दुषः॑ षोड॒शी गृ॒ह्यते॒ यदु॒क्थ्ये॑ धाम॑ कल्पयन्ति स॒प्तच॑त्वारिꣳशच्च}]}%॥11॥

%%% END KANDAM
