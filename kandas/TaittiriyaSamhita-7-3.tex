\chapt{काण्डम् ७}
\sect{तृतीयः प्रश्नः}\setcounter{anuvakam}{0}
\dnsub{तैत्तिरीयसंहितायां सप्तमकाण्डे तृतीयः प्रश्नः}
%7.3.1.1
प्र॒जवं॒ वा ए॒तेन॑ यन्ति॒ यद्द॑श॒ममहः॑ पापाव॒हीयं॒ वा ए॒तेन॑ भवन्ति॒ यद्द॑श॒ममह॒र्यो वै प्र॒जवं॑ य॒तामप॑थेन प्रति॒पद्य॑ते॒ यः स्था॒णुꣳ हन्ति॒ यो भ्रेषं॒ न्येति॒ स ही॑यते॒ स यो वै द॑श॒मे\-ऽह॑न्नविवा॒क्य उ॑पह॒न्यते॒ स ही॑यते॒ तस्मै॒ य उप॑हताय॒ व्याह॒ तमे॒वान्वा॒रभ्य॒ सम॑श्ञु॒ते\-ऽथ॒ यो व्याह॒ सः~(१)

%7.3.1.2
ही॒य॒ते॒ तस्मा᳚द्दश॒मे\-ऽह॑न्नविवा॒क्य उप॑हताय॒ न व्युच्य॒मथो॒ खल्वा॑हुर्य॒ज्ञस्य॒ वै समृ॑द्धेन दे॒वाः सु॑व॒र्गं लो॒कमा॑यन् य॒ज्ञस्य॒ व्यृ॑द्धे॒नासु॑रा॒न्परा॑भावय॒न्निति॒ यत्खलु॒ वै य॒ज्ञस्य॒ समृ॑द्धं॒ तद्यज॑मानस्य॒ यद्व्यृ॑द्धं॒ तद्भ्रातृ॑व्यस्य॒ स यो वै द॑श॒मे\-ऽह॑न्नविवा॒क्य उ॑पह॒न्यते॒ स ए॒वाति॑ रेचयति॒ ते ये बाह्या॑ दृशी॒कवः॑~(२)

%7.3.1.3
स्युस्ते वि ब्रू॑यु॒र्यदि॒ तत्र॒ न वि॒न्देयु॑रन्तःसद॒साद्व्युच्यं॒ यदि॒ तत्र॒ न वि॒न्देयु॑र्गृ॒हप॑तिना॒ व्युच्य॒न्तद्व्युच्य॑मे॒वाथ॒ वा ए॒तथ्स॑र्परा॒ज्ञिया॑ ऋ॒ग्भिः स्तु॑वन्ती॒यं वै सर्प॑तो॒ राज्ञी॒ यद्वा अ॒स्यां किं चार्च॑न्ति॒ यदा॑नृ॒चुस्तेने॒यꣳ स॑र्परा॒ज्ञी ते यदे॒व किं च॑ वा॒चानृ॒चुर्यद॒तो\-ऽध्य॑र्चि॒तारः॑~(३)

%7.3.1.4
तदु॒भय॑मा॒प्त्वाव॒रुध्योत्ति॑ष्ठा॒मेति॒ ताभि॒र्मन॑सा स्तुवते॒ न वा इ॒माम॑श्वर॒थो नाश्व॑तरीर॒थः स॒द्यः पर्या᳚प्तुमर्\mbox{}हति॒ मनो॒ वा इ॒माꣳ स॒द्यः पर्या᳚प्तुमर्\mbox{}हति॒ मनः॒ परि॑भवितु॒मथ॒ ब्रह्म॑ वदन्ति॒ परि॑मिता॒ वा ऋचः॒ परि॑मितानि॒ सामा॑नि॒ परि॑मितानि॒ यजू॒ꣳ॒ष्यथै॒तस्यै॒वान्तो॒ नास्ति॒ यद्ब्रह्म॒ तत्प्र॑तिगृण॒त आ च॑क्षीत॒ स प्र॑तिग॒रः॥~(४)

{\anuvakamend[{व्याह॒ स दृ॑शी॒कवो᳚\-ऽर्चि॒तारः॒ स एक॑ञ्च}]}%~(१)

%7.3.2.1
ब्र॒ह्म॒वा॒दिनो॑ वदन्ति॒ किं द्वा॑दशा॒हस्य॑ प्रथ॒मेनाह्न॒र्त्विजां॒ यज॑मानो वृङ्क्त॒ इति॒ तेज॑ इन्द्रि॒यमिति॒ किं द्वि॒तीये॒नेति॑ प्रा॒णान॒न्नाद्य॒मिति॒ किं तृ॒तीये॒नेति॒ त्रीनि॒माल्लोण॒कानिति॒ किं च॑तु॒र्थेनेति॒ चतु॑ष्पदः प॒शूनिति॒ किं प॑ञ्च॒मेनेति॒ पञ्चा᳚क्षरां प॒ङ्क्तिमिति॒ किꣳ ष॒ष्ठेनेति॒ षडृ॒तूनिति॒ किꣳ स॑प्त॒मेनेति॑ स॒प्तप॑दा॒ꣳ॒ शक्व॑री॒मिति॑~(५)

%7.3.2.2
किम॑ष्ट॒मेनेत्य॒ष्टाक्ष॑रां गाय॒त्रीमिति॒ किं न॑व॒मेनेति॑ त्रि॒वृत॒ꣴ॒ स्तोम॒मिति॒ किं द॑श॒मेनेति॒ दशा᳚क्षरां वि॒राज॒मिति॒ किमे॑काद॒शेनेत्येका॑\-दशाक्षरां त्रि॒ष्टुभ॒मिति॒ किं द्वा॑द॒शेनेति॒ द्वाद॑शाक्षरां॒ जग॑ती॒मित्ये॒ताव॒द्वा अ॑स्ति॒ याव॑दे॒तद्याव॑दे॒वास्ति॒ तदे॑षां वृङ्क्ते॥~(६)

{\anuvakamend[{शक्व॑री॒मित्येक॑चत्वारिꣳशच्च}]}%~(२)

%7.3.3.1
ए॒ष वा आ॒प्तो द्वा॑दशा॒हो यत् त्र॑योदशरा॒त्रः स॑मा॒नꣴ ह्ये॑तदह॒र्यत्प्रा॑य॒णीय॑श्चोदय॒नीय॑श्च॒ त्र्य॑तिरात्रो भवति॒ त्रय॑ इ॒मे लो॒का ए॒षां लो॒काना॒माप्त्यै᳚ प्रा॒णो वै प्र॑थ॒मो॑\-ऽतिरा॒त्रो व्या॒नो द्वि॒तीयो॑\-ऽपा॒नस्तृ॒तीयः॑ प्राणापानोदा॒नेष्वे॒वान्नाद्ये॒ प्रति॑ तिष्ठन्ति॒ सर्व॒मायु॑र्यन्ति॒ य ए॒वं वि॒द्वाꣳ॑सस्त्रयोदशरा॒त्रमास॑ते॒ तदा॑हु॒र्वाग्वा ए॒षा वित॑ता~(७)

%7.3.3.2
यद्द्वा॑दशा॒हस्तां विच्छि॑न्द्यु॒र्यन्मध्ये॑\-ऽतिरा॒त्रं कु॒र्युरु॑प॒दासु॑का गृ॒हप॑ते॒र्वाख्स्या॑दु॒परि॑ष्टाच्छन्दो॒मानां᳚ महाव्र॒तं कु॑र्वन्ति॒ सन्त॑तामे॒व वाच॒मव॑ रुन्ध॒ते\-ऽनु॑पदासुका गृ॒हप॑ते॒र्वाग्भ॑वति प॒शवो॒ वै छ॑न्दो॒मा अन्नं॑ महाव्र॒तं यदु॒परि॑ष्टाच्छन्दो॒मानां᳚ महाव्र॒तं कु॒र्वन्ति॑ प॒शुषु॑ चै॒वान्नाद्ये॑ च॒ प्रति॑ तिष्ठन्ति॥~(८)

{\anuvakamend[{वित॑ता॒ त्रिच॑त्वारिꣳशच्च}]}%~(३)

%7.3.4.1
आ॒दि॒त्या अ॑कामयन्तो॒भयो᳚र्लो॒कयोर्॑ऋध्नुया॒मेति॒ त ए॒तं च॑तुर्दशरा॒त्रम॑पश्य॒न्तमाह॑र॒न्तेना॑यजन्त॒ ततो॒ वै त उ॒भयो᳚र्लो॒कयो॑रार्ध्नुवन्न॒स्मिꣴश्चा॒मुष्मिꣴ॑श्च॒ य ए॒वं वि॒द्वाꣳस॑श्चतुर्दशरा॒त्रमास॑त उ॒भयो॑रे॒व लो॒कयोर्॑\mbox{}॑ध्नुवन्त्य॒\-स्मिꣴश्चा॒मुष्मिꣴ॑श्च चतुर्दशरा॒त्रो भ॑वति स॒प्त ग्रा॒म्या ओष॑धयः स॒प्तार॒ण्या उ॒भयी॑षा॒मव॑रुद्ध्यै॒ यत्प॑रा॒चीना॑नि पृ॒ष्ठानि॑~(९)

%7.3.4.2
भव॑न्त्य॒मुमे॒व तैर्लो॒कम॒भि ज॑यन्ति॒ यत्प्र॑ती॒चीना॑नि पृ॒ष्ठानि॒ भव॑न्ती॒ममे॒व तैर्लो॒कम॒भि ज॑यन्ति त्रयस्त्रि॒ꣳ॒शौ म॑ध्य॒तः स्तोमौ॑ भवतः॒ साम्रा᳚ज्यमे॒व ग॑च्छन्त्यधिरा॒जौ भ॑वतो\-ऽधिरा॒जा ए॒व स॑मा॒नानां᳚ भवन्त्यतिरा॒त्राव॒भितो॑ भवतः॒ परि॑गृहीत्यै॥~(१०)

{\anuvakamend[{पृ॒ष्ठानि॒ चतु॑स्त्रिꣳशच्च}]}%~(४)

%7.3.5.1
प्र॒जा\-प॑तिः सुव॒र्गं लो॒कमै॒त्तं दे॒वा अन्वा॑य॒न्ताना॑दि॒त्याश्च॑ प॒शव॒श्चान्वा॑य॒न्ते दे॒वा अ॑ब्रुव॒न्॒ यान्प॒शूनु॒पाजी॑विष्म॒ त इ॒मे᳚\-ऽन्वाग्म॒न्निति॒ तेभ्य॑ ए॒तं च॑तुर्दशरा॒त्रं प्रत्यौ॑ह॒न्त आ॑दि॒त्याः पृ॒ष्ठैः सु॑व॒र्गं लो॒कमारो॑हन्त्र्य॒हाभ्या॑म॒स्मिँल्लो॒के प॒शून्प्रत्यौ॑हन्पृ॒ष्ठैरा॑दि॒त्या अ॒मुष्मिँ॑ल्लो॒क आर्ध्नु॑वन्त्र्य॒हाभ्या॑म॒स्मिन्~(११)

%7.3.5.2
लो॒के प॒शवो॒ य ए॒वं वि॒द्वाꣳस॑श्चतुर्दशरा॒त्रमास॑त उ॒भयो॑रे॒व लो॒कयोर्॑ऋध्नुवन्त्य॒स्मिꣴश्चा॒मुष्मिꣴ॑श्च पृ॒ष्ठैरे॒वामुष्मिँ॑ल्लो॒क ऋ॑ध्नु॒वन्ति॑ त्र्य॒हाभ्या॑म॒स्मिँल्लो॒के ज्योति॒र्गौरायु॒रिति॑ त्र्य॒हो भ॑वती॒यं वाव ज्योति॑र॒न्तरि॑क्षं॒ गौर॒सावायु॑रि॒माने॒व लो॒कान॒भ्यारो॑हन्ति॒ यद॒न्यतः॑ पृ॒ष्ठानि॒ स्युर्विवि॑वधꣴ स्या॒न्मध्ये॑ पृ॒ष्ठानि॑ भवन्ति सविवध॒त्वाय॑~(१२)

%7.3.5.3
ओजो॒ वै वी॒र्यं॑ पृ॒ष्ठान्योज॑ ए॒व वी॒र्यं॑ मध्य॒तो द॑धते बृहद्रथन्त॒रा\-भ्यां᳚ यन्ती॒यं वाव र॑थन्त॒रम॒सौ बृ॒हदा॒भ्यामे॒व य॒न्त्यथो॑ अ॒नयो॑रे॒व प्रति॑ तिष्ठन्त्ये॒ते वै य॒ज्ञस्या᳚ञ्ज॒साय॑नी स्रु॒ती ताभ्या॑मे॒व सु॑व॒र्गं लो॒कं य॑न्ति॒ परा᳚ञ्चो॒ वा ए॒ते सु॑व॒र्गं लो॒कम॒भ्यारो॑हन्ति॒ ये प॑रा॒चीना॑नि पृ॒ष्ठान्यु॑प॒यन्ति॑ प्र॒त्यङ्त्र्य॒हो भ॑वति प्र॒त्यव॑रूढ्या॒ अथो॒ प्रति॑ष्ठित्या उ॒भयो᳚र्लो॒कयोर्॑\mbox{}॑ऋद्ध्वोत्ति॑ष्ठन्ति॒ चतु॑र्दशै॒तास्तासां॒ या दश॒ दशा᳚क्षरा वि॒राडन्नं॑ वि॒राड्वि॒राजै॒वान्नाद्य॒मव॑ रुन्धते॒ याश्चत॑स्र॒श्चत॑स्रो॒ दिशो॑ दि॒क्ष्वे॑व प्रति॑ तिष्ठन्त्यतिरा॒त्राव॒भितो॑ भवतः॒ परि॑गृहीत्यै॥~(१३)

{\anuvakamend[{आर्ध्नु॑वन्त्र्य॒हाभ्या॑म॒स्मिन्थ्स॑विवध॒त्वाय॒ प्रति॑ष्ठित्या॒ एक॑त्रिꣳशच्च}]}%~(५)

%7.3.6.1
इन्द्रो॒ वै स॒दृङ्दे॒वता॑भिरासी॒थ्स न व्या॒वृत॑मगच्छ॒थ्स प्र॒जा\-प॑ति॒मुपा॑धाव॒त्तस्मा॑ ए॒तं प॑ञ्चदशरा॒त्रं प्राय॑च्छ॒त्तमाह॑र॒त् तेना॑यजत॒ ततो॒ वै सो᳚\-ऽन्याभि॑र्दे॒वता॑भिर्व्या॒वृत॑मगच्छ॒द्य ए॒वं वि॒द्वाꣳसः॑ पञ्चदशरा॒त्रमास॑ते व्या॒वृत॑मे॒व पा॒प्मना॒ भ्रातृ॑व्येण गच्छन्ति॒ ज्योति॒र्गौरायु॒रिति॑ त्र्य॒हो भ॑वती॒यं वाव ज्योति॑र॒न्तरि॑क्षम्~(१४)

%7.3.6.2
गौर॒सावायु॑रे॒ष्वे॑व लो॒केषु॒ प्रति॑ तिष्ठ॒न्त्यस॑त्रं॒ वा ए॒तद्यद॑छन्दो॒मं यच्छ॑न्दो॒मा भव॑न्ति॒ तेन॑ स॒त्रं दे॒वता॑ ए॒व पृ॒ष्ठैरव॑ रुन्धते प॒शूञ्छ॑न्दो॒मैरोजो॒ वा वी॒र्यं॑ पृ॒ष्ठानि॑ प॒शव॑श्छन्दो॒मा ओज॑स्ये॒व वी॒र्ये॑ प॒शुषु॒ प्रति॑ तिष्ठन्ति पञ्चदशरा॒त्रो भ॑वति पञ्चद॒शो वज्रो॒ वज्र॑मे॒व भ्रातृ॑व्येभ्यः॒ प्र ह॑रन्त्यतिरा॒त्राव॒भितो॑ भवत इन्द्रि॒यस्य॒ परि॑गृहीत्यै॥~(१५)

{\anuvakamend[{अ॒न्तरि॑क्षमिन्द्रि॒यस्यैक॑ञ्च}]}%~(६)

%7.3.7.1
इन्द्रो॒ वै शि॑थि॒ल इ॒वाप्र॑तिष्ठित आसी॒थ्सो\-ऽसु॑रेभ्यो\-ऽबिभे॒थ्स प्र॒जा\-प॑ति॒मुपा॑धाव॒त्तस्मा॑ ए॒तं प॑ञ्चदशरा॒त्रं वज्रं॒ प्राय॑च्छ॒त् तेनासु॑रान्परा॒भाव्य॑ वि॒जित्य॒ श्रिय॑मगच्छदग्नि॒ष्टुता॑ पा॒प्मानं॒ निर॑दहत पञ्चदशरा॒त्रेणौजो॒ बल॑मिन्द्रि॒यं वी॒र्य॑मा॒त्मन्न॑धत्त॒ य ए॒वं वि॒द्वाꣳसः॑ पञ्चदशरा॒त्रमास॑ते॒ भ्रातृ॑व्याने॒व प॑रा॒भाव्य॑ वि॒जित्य॒ श्रियं॑ गच्छन्त्यग्नि॒ष्टुता॑ पा॒प्मानं॒ निः~(१६)

%7.3.7.2
द॒ह॒न्ते॒ प॒ञ्च॒द॒श॒रा॒त्रेणौजो॒ बल॑मिन्द्रि॒यं वी॒र्य॑मा॒त्मन्द॑धत ए॒ता ए॒व प॑श॒व्याः᳚ पञ्च॑दश॒ वा अ॑र्धमा॒सस्य॒ रात्र॑यो\-ऽ\-र्धमास॒शः सं॑वथ्स॒र आ᳚प्यते संवथ्स॒रं प॒शवो\-ऽनु॒ प्र जा॑यन्ते॒ तस्मा᳚त्पश॒व्या॑ ए॒ता ए॒व सु॑व॒र्ग्याः᳚ पञ्च॑दश॒ वा अ॑र्धमा॒सस्य॒ रात्र॑यो\-ऽर्धमास॒शः सं॑वथ्स॒र आ᳚प्यते संवथ्स॒रः सु॑व॒र्गो लो॒कस्तस्मा᳚थ्सुव॒र्ग्या᳚ ज्योति॒र्गौरायु॒रिति॑ त्र्य॒हो भ॑वती॒यं वाव ज्योति॑र॒न्तरि॑क्षम्~(१७)

%7.3.7.3
गौर॒सावायु॑रि॒माने॒व लो॒कान॒भ्यारो॑हन्ति॒ यद॒न्यतः॑ पृ॒ष्ठानि॒ स्युर्विवि॑वधꣴ स्या॒न्मध्ये॑ पृ॒ष्ठानि॑ भवन्ति सविवध॒त्वायौजो॒ वै वी॒र्यं॑ पृ॒ष्ठान्योज॑ ए॒व वी॒र्यं॑ मध्य॒तो द॑धते बृहद्रथन्त॒रा\-भ्यां᳚ यन्ती॒यं वाव र॑थन्त॒रम॒सौ बृ॒हदा॒भ्यामे॒व य॒न्त्यथो॑ अ॒नयो॑रे॒व प्रति॑ तिष्ठन्त्ये॒ते वै य॒ज्ञस्या᳚ञ्ज॒साय॑नी स्रु॒ती ताभ्या॑मे॒व सु॑व॒र्गं लो॒कम्~(१८)

%7.3.7.4
य॒न्ति॒ परा᳚ञ्चो॒ वा ए॒ते सु॑व॒र्गं लो॒कम॒भ्यारो॑हन्ति॒ ये प॑रा॒चीना॑नि पृ॒ष्ठान्यु॑प॒यन्ति॑ प्र॒त्यङ्त्र्य॒हो भ॑वति प्र॒त्यव॑रूढ्या॒ अथो॒ प्रति॑ष्ठित्या उ॒भयो᳚र्लो॒कयोर्॑ऋ॒द्ध्वोत्ति॑ष्ठन्ति॒ पञ्च॑दशै॒तास्तासां॒ या दश॒ दशा᳚क्षरा वि॒राडन्नं॑ वि॒राड्वि॒राजै॒वान्नाद्य॒\-मव॑ रुन्धते॒ याः पञ्च॒ पञ्च॒ दिशो॑ दि॒क्ष्वे॑व प्रति॑ तिष्ठन्त्यतिरा॒त्राव॒भितो॑ भवत इन्द्रि॒यस्य॑ वी॒र्य॑स्य प्र॒जायै॑ पशू॒नां परि॑गृहीत्यै॥~(१९)

{\anuvakamend[{ग॒च्छ॒न्त्य॒ग्नि॒ष्टुता॑ पा॒प्मान॒न्निर॒न्तरि॑क्षं लो॒कं प्र॒जायै॒ द्वे च॑}]}%~(७)

%7.3.8.1
प्र॒जा\-प॑तिरकामयतान्ना॒दः स्या॒मिति॒ स ए॒तꣳ स॑प्तदशरा॒त्रम॑पश्य॒त्तमाह॑र॒त्तेना॑यजत॒ ततो॒ वै सो᳚\-ऽन्ना॒दो॑\-ऽभव॒द्य ए॒वं वि॒द्वाꣳसः॑ सप्तदशरा॒त्रमास॑ते\-ऽन्ना॒दा ए॒व भ॑वन्ति पञ्चा॒हो भ॑वति॒ पञ्च॒ वा ऋ॒तवः॑ संवथ्स॒र ऋ॒तुष्वे॒व सं॑वथ्स॒रे प्रति॑ तिष्ठ॒न्त्यथो॒ पञ्चा᳚क्षरा प॒ङ्क्तिः पाङ्क्तो॑ य॒ज्ञो य॒ज्ञमे॒वाव॑ रुन्ध॒ते\-ऽस॑त्रं॒ वा ए॒तत्~(२०)

%7.3.8.2
यद॑छन्दो॒मं यच्छ॑न्दो॒मा भव॑न्ति॒ तेन॑ स॒त्रं दे॒वता॑ ए॒व पृ॒ष्ठैरव॑ रुन्धते प॒शूञ्छ॑न्दो॒मैरोजो॒ वै वी॒र्यं॑ पृ॒ष्ठानि॑ प॒शव॑श्छन्दो॒मा ओज॑स्ये॒व वी॒र्ये॑ प॒शुषु॒ प्रति॑ तिष्ठन्ति सप्तदशरा॒त्रो भ॑वति सप्तद॒शः प्र॒जा\-प॑तिः प्र॒जा\-प॑ते॒राप्त्या॑ अतिरा॒त्राव॒भितो॑ भवतो॒\-ऽन्नाद्य॑स्य॒ परि॑गृहीत्यै॥~(२१)

{\anuvakamend[{ए॒तथ्स॒प्तत्रिꣴ॑श्चच्च}]}%~(८)

%7.3.9.1
सा वि॒राड्वि॒क्रम्या॑तिष्ठ॒द्ब्रह्म॑णा दे॒वेष्वन्ने॒नासु॑रेषु॒ ते दे॒वा अ॑कामयन्तो॒भय॒ꣳ॒ सं वृ॑ञ्जीमहि॒ ब्रह्म॒ चान्नं॒ चेति॒ त ए॒ता विꣳ॑श॒तिꣳ रात्री॑रपश्य॒न्ततो॒ वै त उ॒भय॒ꣳ॒ सम॑वृञ्जत॒ ब्रह्म॒ चान्नं॑ च ब्रह्मवर्च॒सिनो᳚\-ऽन्ना॒दा अ॑भव॒न्॒ य ए॒वं वि॒द्वाꣳस॑ ए॒ता आस॑त उ॒भय॑मे॒व सं वृ॑ञ्जते॒ ब्रह्म॒ चान्नं॑ च~(२२)

%7.3.9.2
ब्र॒ह्म॒व॒र्च॒सिनो᳚\-ऽन्ना॒दा भ॑वन्ति॒ द्वे वा ए॒ते वि॒राजौ॒ तयो॑रे॒व नाना॒ प्रति॑ तिष्ठन्ति वि॒ꣳ॒शो वै पुरु॑षो॒ दश॒ हस्त्या॑ अ॒ङ्गुल॑यो॒ दश॒ पद्या॒ यावा॑ने॒व पुरु॑ष॒स्तमा॒प्त्वोत्ति॑ष्ठन्ति॒ ज्योति॒र्गौरायु॒रिति॑ त्र्य॒हा भ॑वन्ती॒यं वाव ज्योति॑र॒न्तरि॑क्षं॒ गौर॒सावायु॑रि॒माने॒व लो॒कान॒भ्यारो॑हन्त्यभिपू॒र्वं त्र्य॒हा भ॑वन्त्यभिपू॒र्वमे॒व सु॑व॒र्गम्~(२३)

%7.3.9.3
लो॒कम॒भ्यारो॑हन्ति॒ यद॒न्यतः॑ पृ॒ष्ठानि॒ स्युर्विवि॑वधꣴ स्या॒न्मध्ये॑ पृ॒ष्ठानि॑ भवन्ति सविवध॒त्वायौजो॒ वै वी॒र्यं॑ पृ॒ष्ठान्योज॑ ए॒व वी॒र्यं॑ मध्य॒तो द॑धते बृहद्रथन्त॒रा\-भ्यां᳚ यन्ती॒यं वाव र॑थन्त॒रम॒सौ बृ॒हदाभ्यामे॒व य॒न्त्यथो॑ अ॒नयो॑रे॒व प्रति॑ तिष्ठन्त्ये॒ते वै य॒ज्ञस्या᳚ञ्ज॒साय॑नी स्रु॒ती ताभ्या॑मे॒व सु॑व॒र्गं लो॒कं य॑न्ति॒ परा᳚ञ्चो॒ वा ए॒ते सु॑व॒र्गं लो॒कम॒भ्यारो॑हन्ति॒ ये प॑रा॒चीना॑नि पृ॒ष्ठान्यु॑प॒यन्ति॑ प्र॒त्यङ्त्र्य॒हो भ॑वति प्र॒त्यव॑रूढ्या॒ अथो॒ प्रति॑ष्ठित्या उ॒भयो᳚र्लो॒कयोर्॑ ऋ॒द्ध्वोत्ति॑ष्ठन्त्यतिरा॒त्राव॒भितो॑ भवतो ब्रह्मवर्च॒सस्या॒न्नाद्य॑स्य॒ परि॑गृहीत्यै॥~(२४)

{\anuvakamend[{वृ॒ञ्ज॒ते॒ ब्रह्म॒ चान्न॑ञ्च सुव॒र्गमे॒ते सु॑व॒र्गन्त्रयो॑विꣳशतिश्च}]}%~(९)

%7.3.10.1
अ॒सावा॑दि॒त्यो᳚\-ऽस्मिँल्लो॒क आ॑सी॒त्तं दे॒वाः पृ॒ष्ठैः प॑रि॒गृह्य॑ सुव॒र्गं लो॒कम॑गमय॒न्परै॑र॒वस्ता॒त्पर्य॑गृह्णन्दिवाकी॒र्त्ये॑न सुव॒र्गे लो॒के प्रत्य॑स्थापय॒न्परैः᳚ प॒रस्ता॒त्पर्य॑गृह्णन्पृ॒ष्ठैरु॒पावा॑रोह॒न्थ्स वा अ॒सावा॑दि॒त्यो॑\-ऽमुष्मिँ॑ल्लो॒के परै॑रुभ॒यतः॒ परि॑गृहीतो॒ यत्पृ॒ष्ठानि॒ भव॑न्ति सुव॒र्गमे॒व तैर्लो॒कं यज॑माना यन्ति॒ परै॑र॒वस्ता॒त्परि॑ गृह्णन्ति दिवाकी॒र्त्ये॑न~(२५)

%7.3.10.2
सु॒व॒र्गे लो॒के प्रति॑ तिष्ठन्ति॒ परैः᳚ प॒रस्ता॒त्परि॑ गृह्णन्ति पृ॒ष्ठैरु॒पाव॑रोहन्ति॒ यत्परे॑ प॒रस्ता॒न्न स्युः परा᳚ञ्चः सुव॒र्गाल्लो॒कान्निष्प॑द्येर॒न्॒ यद॒वस्ता॒न्न स्युः प्र॒जा निर्द॑हेयुर॒भितो॑ दिवाकी॒र्त्यं॑ परः॑सामानो भवन्ति सुव॒र्ग ए॒वैनां᳚ लो॒क उ॑भ॒यतः॒ परि॑ गृह्णन्ति॒ यज॑माना॒ वै दि॑वाकी॒र्त्यꣳ॑ संवथ्स॒रः परः॑सामानो॒\-ऽभितो॑ दिवाकी॒र्त्यं॑ परः॑ सामानो भवन्ति संवथ्स॒र ए॒वोभ॒यतः॑~(२६)

%7.3.10.3
प्रति॑ तिष्ठन्ति पृ॒ष्ठं वै दि॑वाकी॒र्त्यं॑ पा॒र्श्वे परः॑सामानो॒\-ऽभितो॑ दिवाकी॒र्त्यं॑ परः॑सामानो भवन्ति॒ तस्मा॑द॒भितः॑ पृ॒ष्ठं पा॒र्श्वे भूयि॑ष्ठा॒ ग्रहा॑ गृह्यन्ते॒ भूयि॑ष्ठꣳ शस्यते य॒ज्ञस्यै॒व तन्म॑ध्य॒तो ग्र॒न्थं ग्र॑थ्न॒न्त्यवि॑स्रꣳसाय स॒प्त गृ॑ह्यन्ते स॒प्त वै शी॑र्\mbox{}ष॒ण्याः᳚ प्रा॒णाः प्रा॒णाने॒व यज॑मानेषु दधति॒ यत्प॑रा॒चीना॑नि पृ॒ष्ठानि॒ भव॑न्त्य॒मुमे॒व तैर्लो॒कम॒भ्यारो॑हन्ति॒ यदि॒मं लो॒कं न~(२७)

%7.3.10.4
प्र॒त्य॒व॒रोहे॑यु॒रुद्वा॒ माद्ये॑यु॒र्यज॑मानाः॒ प्र वा॑ मीयेर॒न्॒ यत्प्र॑ती॒चीना॑नि पृ॒ष्ठानि॒ भव॑न्ती॒ममे॒व तैर्लो॒कं प्र॒त्यव॑रोह॒न्त्यथो॑ अ॒स्मिन्ने॒व लो॒के प्रति॑ तिष्ठ॒न्त्यनु॑न्मादा॒येन्द्रो॒ वा अप्र॑तिष्ठित आसी॒थ्स प्र॒जा\-प॑ति॒मुपा॑धाव॒त्तस्मा॑ ए॒तमे॑कविꣳशतिरा॒त्रं प्राय॑च्छ॒त्तमाह॑र॒त्तेना॑यजत॒ ततो॒ वै स प्रत्य॑तिष्ठ॒द्ये ब॑हुया॒जिनो\-ऽप्र॑तिष्ठिताः~(२८)

%7.3.10.5
स्युस्त ए॑कविꣳशतिरा॒त्रमा॑सीर॒न्द्वाद॑श॒ मासाः॒ पञ्च॒र्तव॒स्त्रय॑ इ॒मे लो॒का अ॒सावा॑दि॒त्य ए॑कवि॒ꣳ॒श ए॒ताव॑न्तो॒ वै दे॑वलो॒कास्तेष्वे॒व य॑थापू॒र्वं प्रति॑ तिष्ठन्त्य॒सावा॑दि॒त्यो न व्य॑रोचत॒ स प्र॒जा\-प॑ति॒मुपा॑धाव॒त्तस्मा॑ ए॒तमे॑कविꣳशति\-रा॒त्रं प्राय॑च्छ॒त्तमाह॑र॒त्तेना॑यजत॒ ततो॒ वै सो॑\-ऽरोचत॒ य ए॒वं वि॒द्वाꣳ॑स एकविꣳशतिरा॒त्रमास॑ते॒ रोच॑न्त ए॒वैक॑विꣳशतिरा॒त्रो भ॑वति॒ रुग्वा ए॑कवि॒ꣳ॒शो रुच॑मे॒व ग॑च्छ॒न्त्यथो᳚ प्रति॒ष्ठामे॒व प्र॑ति॒ष्ठा ह्ये॑कवि॒ꣳ॒शो॑-\-ऽतिरा॒त्राव॒भितो॑ भवतो ब्रह्मवर्च॒सस्य॒ परि॑गृहीत्यै॥~(२९)

{\anuvakamend[{गृ॒ह्ण॒न्ति॒ दि॒वा॒की॒र्त्ये॑नै॒वोभ॒यतो॒ नाप्र॑तिष्ठिता॒ आस॑त॒ एक॑विꣳशतिश्च}]}%॥10॥

%7.3.11.1
अ॒र्वाङ्य॒ज्ञः सं क्रा॑मत्व॒मुष्मा॒दधि॒ माम॒भि। ऋषी॑णां॒ यः पु॒रोहि॑तः। निर्दे॑वं॒ निर्वी॑रं कृ॒त्वा विष्क॑न्धं॒ तस्मि॑न् हीयतां॒ यो᳚\-ऽस्मान्द्वेष्टि॑। शरी॑रं यज्ञशम॒लं कुसी॑दं॒ तस्मि᳚न्थ्सीदतु॒ यो᳚\-ऽस्मान्द्वेष्टि॑। यज्ञ॑ य॒ज्ञस्य॒ यत्तेज॒स्तेन॒ सं क्रा॑म॒ माम॒भि। ब्रा॒ह्म॒णानृ॒त्विजो॑ दे॒वान् य॒ज्ञस्य॒ तप॑सा॒ ते सवा॒हमा हु॑वे। इ॒ष्टेन॑ प॒क्वमुप॑~(३०)

%7.3.11.2
ते॒ हु॒वे॒ स॒वा॒हम्। सन्ते॑ वृञ्जे सुकृ॒तꣳ सं प्र॒जां प॒शून्। प्रै॒षान्थ्सा॑मिधे॒नीरा॑घा॒रावाज्य॑भागा॒वाश्रु॑तं प्र॒त्याश्रु॑त॒मा शृ॑णामि ते। प्र॒या॒जा॒नू॒या॒जान्थ्स्वि॑ष्ट॒कृत॒मिडा॑मा॒शिष॒ आ वृ॑ञ्जे॒ सुवः॑। अ॒ग्निनेन्द्रे॑ण॒ सोमे॑न॒ सर॑स्वत्या॒ विष्णु॑ना दे॒वता॑भिः। या॒ज्या॒नु॒वा॒क्या᳚भ्या॒मुप॑ ते हुवे स्वा॒हं य॒ज्ञमा द॑दे ते॒ वष॑ट्कृतम्। स्तु॒तꣳ श॒स्त्रं प्र॑तिग॒रं ग्रह॒मिडा॑मा॒शिषः॑~(३१)

%7.3.11.3
आ वृ॑ञ्जे॒ सुवः॑। प॒त्नी॒सं॒या॒जानुप॑ ते हुवे सवा॒हꣳ स॑मिष्टय॒जुरा द॑दे॒ तव॑। प॒शून्थ्सु॒तं पु॑रो॒डाशा॒न्थ्सव॑ना॒न्योत य॒ज्ञम्। दे॒वान्थ्सेन्द्रा॒नुप॑ ते हुवे सवा॒हम॒ग्निमु॑खा॒न्थ्सोम॑वतो॒ ये च॒ विश्वे᳚॥~(३२)

{\anuvakamend[{उप॒ ग्रह॒मिडा॑मा॒शिषो॒ द्वात्रिꣳ॑शच्च}]}%॥11॥

%7.3.12.1
भू॒तं भव्य॑म्भवि॒ष्यद्वष॒ट्थ्\-स्वाहा॒ नम॒ ऋख्साम॒ यजु॒र्वष॒ट्थ्\-स्वाहा॒ नमो॑ गाय॒त्री त्रि॒ष्टुब्जग॑ती॒ वष॒ट्थ्\-स्वाहा॒ नमः॑ पृथि॒व्य॑न्तरि॑क्षं॒ द्यौर्वष॒ट्थ्\-स्वाहा॒ नमो॒\-ऽग्निर्वा॒युः सूर्यो॒ वष॒ट्थ्\-स्वाहा॒ नमः॑ प्रा॒णो व्या॒नो॑\-ऽपा॒नो वष॒ट्थ्\-स्वाहा॒ नमो\-ऽन्नं॑ कृ॒षिर्वृष्टि॒र्वष॒ट्थ्\-स्वाहा॒ नमः॑ पि॒ता पु॒त्रः पौत्रो॒ वष॒ट्थ्\-स्वाहा॒ नमो॒ भूर्भुवः॒सुव॒र्वष॒ट्थ्\-स्वाहा॒ नमः॑॥~(३३)

{\anuvakamend[{भुव॑श्च॒त्वारि॑ च}]}%॥12॥

%7.3.13.1
आ मे॑ गृ॒हा भ॑वं᳚ त्वा प्र॒जा म॒ आ मा॑ य॒ज्ञो वि॑शतु वी॒र्या॑वान्। आपो॑ दे॒वीर्य॒ज्ञिया॒ मा वि॑शन्तु स॒हस्र॑स्य मा भू॒मा मा प्र हा॑सीत्। आ मे॒ ग्रहो॑ भव॒त्वा पु॑रो॒रुख्स्तु॑तश॒स्त्रे मा वि॑शताꣳ स॒मीची᳚। आ॒दि॒त्या रु॒द्रा वस॑वो मे सद॒स्याः᳚ स॒हस्र॑स्य मा भू॒मा मा प्र हा॑सीत्। आ मा᳚ग्निष्टो॒मो वि॑शतू॒क्थ्य॑श्चातिरा॒त्रो मा वि॑शत्वापिशर्व॒रः। ति॒रोअ॑ह्निया मा॒ सुहु॑ता॒ आ वि॑शन्तु स॒हस्र॑स्य मा भू॒मा मा॒ प्र हा॑सीत्॥~(३४)

{\anuvakamend[{अ॒ग्नि॒ष्टो॒मो वि॑शत्व॒ष्टाद॑श च}]}%॥13॥

%7.3.14.1
अ॒ग्निना॒ तपो\-ऽन्व॑भवद्वा॒चा ब्रह्म॑ म॒णिना॑ रू॒पाणीन्द्रे॑ण दे॒वान् वाते॑न प्रा॒णान्थ्सूर्ये॑ण॒ द्याञ्च॒न्द्रम॑सा॒ नक्ष॑त्राणि य॒मेन॑ पितॄन्राज्ञा॑ मनु॒ष्या᳚न्फ॒लेन॑ नादे॒यान॑जग॒रेण॑ स॒र्पान्व्या॒घ्रेणा॑र॒ण्यान्प॒शूञ्छ्ये॒नेन॑ पत॒त्रिणो॒ वृष्णाश्वा॑नृष॒भेण॒ गा ब॒स्तेना॒जा वृ॒ष्णिनावी᳚र्व्री॒हिणान्ना॑नि॒ यवे॒नौष॑धीर्न्य॒ग्रोधे॑न॒ वन॒स्पती॑नुदु॒म्बरे॒णोर्जं॑ गायत्रि॒या छन्दाꣳ॑सि त्रि॒वृता॒ स्तोमा᳚न्ब्राह्म॒णेन॒ वाचम्᳚॥~(३५)

{\anuvakamend[{ब्रा॒ह्म॒णेनैक॑ञ्च}]}%॥14॥

%7.3.15.1
स्वाहा॒धिमाधी॑ताय॒ स्वाहा॒ स्वाहाधी॑तं॒ मन॑से॒ स्वाहा॒ स्वाहा॒ मनः॑ प्र॒जा\-प॑तये॒ स्वाहा॒ काय॒ स्वाहा॒ कस्मै॒ स्वाहा॑ कत॒मस्मै॒ स्वाहादि॑त्यै॒ स्वाहादि॑त्यै म॒ह्यै᳚ स्वाहादि॑त्यै सुमृडी॒कायै॒ स्वाहा॒ सर॑स्वत्यै॒ स्वाहा॒ सर॑स्वत्यै बृह॒त्यै᳚ स्वाहा॒ सर॑स्वत्यै पाव॒कायै॒ स्वाहा॑ पू॒ष्णे स्वाहा॑ पू॒ष्णे प्र॑प॒थ्या॑य॒ स्वाहा॑ पू॒ष्णे न॒रन्धि॑षाय॒ स्वाहा॒ त्वष्ट्रे॒ स्वाहा॒ त्वष्ट्रे॑ तु॒रीपा॑य॒ स्वाहा॒ त्वष्ट्रे॑ पुरु॒रूपा॑य॒ स्वाहा॒ विष्ण॑वे॒ स्वाहा॒ विष्ण॑वे निखुर्य॒पाय॒ स्वाहा॒ विष्ण॑वे निभूय॒पाय॒ स्वाहा॒ सर्वस्मै॒ स्वाहा᳚॥~(३६)

{\anuvakamend[{पु॒रु॒रूपा॑य॒ स्वाहा॒ दश॑ च}]}%॥15॥

%7.3.16.1
द॒द्भ्यः स्वाहा॒ हनू᳚भ्या॒ꣴ॒ स्वाहोष्ठा᳚भ्या॒ꣴ॒ स्वाहा॒ मुखा॑य॒ स्वाहा॒ नासि॑काभ्या॒ꣴ॒ स्वाहा॒क्षीभ्या॒ꣴ॒ स्वाहा॒ कर्णा᳚भ्या॒ꣴ॒ स्वाहा॑ पा॒र इ॒क्षवो॑\-ऽवा॒र्ये᳚भ्यः॒ पक्ष्म॑भ्यः॒ स्वाहा॑वा॒र इ॒क्षवः॑ पा॒र्ये᳚भ्यः॒ पक्ष्म॑भ्यः॒ स्वाहा॑ शी॒र्॒\mbox{}ष्णे स्वाहा᳚ भ्रू॒भ्याꣴ स्वाहा॑ ल॒लाटा॑य॒ स्वाहा॑ मू॒र्ध्ने स्वाहा॑ म॒स्तिष्का॑य॒ स्वाहा॒ केशे᳚भ्यः॒ स्वाहा॒ वहा॑य॒ स्वाहा᳚ ग्री॒वाभ्यः॒ स्वाहा᳚ स्क॒न्धेभ्यः॒ स्वाहा॒ कीक॑साभ्यः॒ स्वाहा॑ पृ॒ष्टीभ्यः॒ स्वाहा॑ पाज॒स्या॑य॒ स्वाहा॑ पा॒र्श्वाभ्या॒ꣴ॒ स्वाहा᳚~(३७)

%7.3.16.2
अꣳसा᳚भ्या॒ꣴ॒ स्वाहा॑ दो॒षभ्या॒ꣴ॒ स्वाहा॑ बा॒हुभ्या॒ꣴ॒ स्वाहा॒ जङ्घा᳚भ्या॒ꣴ॒ स्वाहा॒ श्रोणी᳚भ्या॒ꣴ॒ स्वाहो॒रुभ्या॒ꣴ॒ स्वाहा᳚ष्ठी॒वद्भ्या॒ꣴ॒ स्वाहा॒ जङ्घा᳚भ्या॒ꣴ॒ स्वाहा॑ भ॒सदे॒ स्वाहा॑ शिख॒ण्डेभ्यः॒ स्वाहा॑ वाल॒धाना॑य॒ स्वाहा॒ण्डाभ्या॒ꣴ॒ स्वाहा॒ शेपा॑य॒ स्वाहा॒ रेत॑से॒ स्वाहा᳚ प्र॒जाभ्यः॒ स्वाहा᳚ प्र॒जन॑नाय॒ स्वाहा॑ प॒द्भ्यः स्वाहा॑ श॒फेभ्यः॒ स्वाहा॒ लोम॑भ्यः॒ स्वाहा᳚ त्व॒चे स्वाहा॒ लोहि॑ताय॒ स्वाहा॑ मा॒ꣳ॒साय॒ स्वाहा॒ स्नाव॑भ्यः॒ स्वाहा॒स्थभ्यः॒ स्वाहा॑ म॒ज्जभ्यः॒ स्वाहाङ्गे᳚भ्यः॒ स्वाहा॒त्मने॒ स्वाहा॒ सर्व॑स्मै॒ स्वाहा᳚॥~(३८)

{\anuvakamend[{पा॒र्श्वाभ्या॒ꣴ॒ स्वाहा॑ म॒ज्जभ्यः॒ स्वाहा॒ षट्च॑}]}%॥16॥

%7.3.17.1
अ॒ञ्ज्ये॒ताय॒ स्वाहा᳚ञ्जिस॒क्थाय॒ स्वाहा॑ शिति॒पदे॒ स्वाहा॒ शिति॑ककुदे॒ स्वाहा॑ शिति॒रन्ध्रा॑य॒ स्वाहा॑ शितिपृ॒ष्ठाय॒ स्वाहा॑ शि॒त्यꣳसा॑य॒ स्वाहा॑ पुष्प॒कर्णा॑य॒ स्वाहा॑ शि॒त्योष्ठा॑य॒ स्वाहा॑ शिति॒भ्रवे॒ स्वाहा॒ शिति॑भसदे॒ स्वाहा᳚ श्वे॒तानू॑काशाय॒ स्वाहा॒ञ्जये॒ स्वाहा॑ ल॒लामा॑य॒ स्वाहासि॑तज्ञवे॒ स्वाहा॑ कृष्णै॒ताय॒ स्वाहा॑ रोहितै॒ताय॒ स्वाहा॑रुणै॒ताय॒ स्वाहे॒दृशा॑य॒ स्वाहा॑ की॒दृशा॑य॒ स्वाहा॑ ता॒दृशा॑य॒ स्वाहा॑ स॒दृशा॑य॒ स्वाहा॒ विस॑दृशाय॒ स्वाहा॒ सुस॑दृ॒शाय॒ स्वाहा॑ रू॒पाय॒ स्वाहा॒ सर्व॑स्मै॒ स्वाहा᳚॥~(३९)

{\anuvakamend[{रू॒पाय॒ स्वाहा॒ द्वे च॑}]}%॥17॥

%7.3.18.1
कृ॒ष्णाय॒ स्वाहा᳚ श्वे॒ताय॒ स्वाहा॑ पि॒शङ्गा॑य॒ स्वाहा॑ सा॒रङ्गा॑य॒ स्वाहा॑रु॒णाय॒ स्वाहा॑ गौ॒राय॒ स्वाहा॑ ब॒भ्रवे॒ स्वाहा॑ नकु॒लाय॒ स्वाहा॒ रोहि॑ताय॒ स्वाहा॒ शोणा॑य॒ स्वाहा᳚ श्या॒वाय॒ स्वाहा᳚ श्या॒माय॒ स्वाहा॑ पाक॒लाय॒ स्वाहा॑ सुरू॒पाय॒ स्वाहानु॑रूपाय॒ स्वाहा॒ विरू॑पाय॒ स्वाहा॒ सरू॑पाय॒ स्वाहा॒ प्रति॑रूपाय॒ स्वाहा॑ श॒बला॑य॒ स्वाहा॑ कम॒लाय॒ स्वाहा॒ पृश्ञ॑ये॒ स्वाहा॑ पृश्ञिस॒क्थाय॒ स्वाहा॒ सर्व॑स्मै॒ स्वाहा᳚॥~(४०)

{\anuvakamend[{कृ॒ष्णाय॒ षट्च॑त्वारिꣳशत्}]}%॥18॥

%7.3.19.1
ओष॑धीभ्यः॒ स्वाहा॒ मूले᳚भ्यः॒ स्वाहा॒ तूले᳚भ्यः॒ स्वाहा॒ काण्डे᳚भ्यः॒ स्वाहा॒ वल्\mbox{}शे᳚भ्यः॒ स्वाहा॒ पुष्पे᳚भ्यः॒ स्वाहा॒ फले᳚भ्यः॒ स्वाहा॑ गृही॒तेभ्यः॒ स्वाहागृ॑हीतेभ्यः॒ स्वाहाव॑पन्नेभ्यः॒ स्वाहा॒ शया॑नेभ्यः॒ स्वाहा॒ सर्व॑स्मै॒ स्वाहा᳚॥~(४१)

{\anuvakamend[{ओष॑धीभ्य॒श्चतु॑र्विꣳशतिः}]}%॥19॥

%7.3.20.1
वन॒स्पति॑भ्यः॒ स्वाहा॒ मूले᳚भ्यः॒ स्वाहा॒ तूले᳚भ्यः॒ स्वाहा॒ स्कन्धो᳚भ्यः॒ स्वाहा॒ शाखा᳚भ्यः॒ स्वाहा॑ प॒र्णेभ्यः॒ स्वाहा॒ पुष्पे᳚भ्यः॒ स्वाहा॒ फले᳚भ्यः॒ स्वाहा॑ गृही॒तेभ्यः॒ स्वाहागृ॑हीतेभ्यः॒ स्वाहाव॑पन्नेभ्यः॒ स्वाहा॒ शया॑नेभ्यः॒ स्वाहा॑ शि॒ष्टाय॒ स्वाहाति॑शिष्टाय॒ स्वाहा॒ परि॑शिष्टाय॒ स्वाहा॒ सꣳशि॑ष्टाय॒ स्वाहोच्छि॑ष्टाय॒ स्वाहा॑ रि॒क्ताय॒ स्वाहारि॑क्ताय॒ स्वाहा॒ प्ररि॑क्ताय॒ स्वाहा॒ सꣳरि॑क्ताय॒ स्वाहोद्रि॑क्ताय॒ स्वाहा॒ सर्व॑स्मै॒ स्वाहा᳚॥~(४२)

{\anuvakamend[{वन॒स्पति॑भ्यः॒ स्कन्धो᳚भ्यः शि॒ष्टाय॑ रि॒क्ताय॒ षट्च॑त्वारिꣳशत्}]}%॥20॥

\prashnaend{प्र॒जवं॑ ब्रह्मवा॒दिनः॒ किमे॒ष वा आ॒प्त आ॑दि॒त्या उ॒भयोः᳚ प्र॒जा\-प॑ति॒रन्वा॑य॒न्निन्द्रो॒ वै स॒दृङ्ङिन्द्रो॒ वै शि॑थि॒लः प्र॒जा\-प॑तिरकामयतान्ना॒दः सा वि॒राड॒सावा॑दि॒त्यो᳚\-ऽर्वाङ्भू॒तमा मे॒\-ऽग्निना॒ स्वाहा॒धिन्द॒द्भ्यो᳚\-ऽञ्ज्ये॒ताय॑ कृ॒ष्णायौष॑धीभ्यो॒ वन॒स्पति॑भ्यो विꣳश॒तिः॥२०॥}{प्र॒जवं॑ प्र॒जा\-प॑ति॒र्यद॑छन्दो॒मन्ते॑ हुवे सवा॒हमोष॑धीभ्यो॒ द्विच॑त्वारिꣳशत्॥४२॥}{प्र॒जव॒ꣳ॒ सर्व॑स्मै॒ स्वाहा᳚॥}%%७-३
{हरिः॑ ॐ}{॥कृष्ण-यजुर्वेदीय-तैत्तिरीय-संहितायां सप्तमकाण्डे तृतीयः प्रश्नः समाप्तः॥७-३॥}
%%% END PRASHNA
