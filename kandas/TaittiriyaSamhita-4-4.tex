\chapt{काण्डम् ४}
\sect{चतुर्थः प्रश्नः}\setcounter{anuvakam}{0}
\dnsub{तैत्तिरीयसंहितायां चतुर्थकाण्डे चतुर्थः प्रश्नः}
%4.4.1.1
र॒श्मिर॑सि॒ क्षया॑य त्वा॒ क्षयं॑ जिन्व॒ प्रेति॑रसि॒ धर्मा॑य त्वा॒ धर्मं॑ जि॒न्वान्वि॑तिरसि दि॒वे त्वा॒ दिवं॑ जिन्व स॒न्धिर॑स्य॒न्तरि॑क्षाय त्वा॒ऽन्तरि॑क्षं जिन्व प्रति॒धिर॑सि पृथि॒व्यै त्वा॑ पृथि॒वीं जि॑न्व विष्ट॒म्भो॑\-ऽसि॒ वृष्ट्यै᳚ त्वा॒ वृष्टिं॑ जिन्व प्र॒वास्यह्ने॒ त्वाह॑र्जिन्वानु॒वासि॒ रात्रि॑यै त्वा॒ रात्रिं॑ जिन्वो॒शिग॑सि॒~(१)\ip

%4.4.1.2
वसु॑भ्यस्त्वा॒ वसू᳚ञ्जिन्व प्रके॒तो॑\-ऽसि रु॒द्रेभ्य॑स्त्वा रु॒द्राञ्जि॑न्व सुदी॒तिर॑स्याऽऽदि॒त्येभ्य॑स्त्वा\-ऽऽ\-दि॒त्याञ्जि॒न्वौजो॑\-ऽसि पि॒तृभ्य॑स्त्वा पि॒तॄञ्जि॑न्व॒ तन्तु॑रसि प्र॒जाभ्य॑स्त्वा प्र॒जा जि॑न्व पृतना॒षाड॑सि प॒शुभ्य॑स्त्वा प॒शूञ्जि॑न्व रे॒वद॒स्योष॑धीभ्य॒स्त्वौष॑धीर्जिन्वाभि॒जिद॑सि यु॒क्तग्रा॒वेन्द्रा॑य॒ त्वेन्द्रं॑ जि॒न्वाधि॑\-पतिरसि प्रा॒णाय॑~(२)\ip

%4.4.1.3
त्वा प्रा॒णं जि॑न्व य॒न्तास्य॑पा॒नाय॑ त्वाऽपा॒नं जि॑न्व स॒ꣳ॒सर्पो॑\-ऽसि॒ चक्षु॑षे त्वा॒ चक्षु॑र्जिन्व वयो॒धा अ॑सि॒ श्रोत्रा॑य त्वा॒ श्रोत्रं॑ जिन्व त्रि॒वृद॑सि प्र॒वृद॑सि सं॒वृद॑सि वि॒वृद॑सि सꣳरो॒हो॑\-ऽसि नीरो॒हो॑\-ऽसि प्ररो॒हो᳚\-ऽस्यनुरो॒हो॑\-ऽसि वसु॒को॑\-ऽसि॒ वेष॑श्रिरसि॒ वस्य॑ष्टिरसि॥~(३)\ip

{\anuvakamend[{उ॒शिग॑सि प्रा॒णाय॒ त्रिच॑त्वारिꣳशच्च}]}

%4.4.2.1
राज्ञ्य॑सि॒ प्राची॒ दिग्वस॑वस्ते दे॒वा अधि॑पतयो॒\-ऽग्निर्\mbox{}हे॑ती॒नां प्र॑तिध॒र्ता त्रि॒वृत्त्वा॒ स्तोमः॑ पृथि॒व्याꣴ श्र॑य॒त्वाज्य॑मु॒क्थ\-मव्य॑थयथ्स्तभ्नातु रथन्त॒रꣳ साम॒ प्रति॑ष्ठित्यै वि॒राड॑सि दक्षि॒णा दिग्रु॒द्रास्ते॑ दे॒वा अधि॑पतय॒ इन्द्रो॑ हेती॒नां प्र॑तिध॒र्ता प॑ञ्चद॒शस्त्वा॒ स्तोमः॑ पृथि॒व्याꣴ श्र॑यतु॒ प्रउ॑गमु॒क्थमव्य॑थयथ्स्तभ्नातु बृ॒हथ्साम॒ प्रति॑ष्ठित्यै स॒म्राड॑सि प्र॒तीची॒ दि-~(४)\ip

%4.4.2.2
गादि॒त्यास्ते॑ दे॒वा अधि॑पतयः॒ सोमो॑ हेती॒नां प्र॑तिध॒र्ता स॑प्तद॒शस्त्वा॒ स्तोमः॑ पृथि॒व्याꣴ श्र॑यतु मरुत्व॒तीय॑मु॒क्थ\-मव्य॑थयथ्स्तभ्नातु वैरू॒पꣳ साम॒ प्रति॑ष्ठित्यै स्व॒राड॒स्युदी॑ची॒ दिग्विश्वे॑ ते दे॒वा अधि॑पतयो॒ वरु॑णो हेती॒नां प्र॑तिध॒र्तैक॑\-वि॒ꣳ॒शस्त्वा॒ स्तोमः॑ पृथि॒व्याꣴ श्र॑यतु॒ निष्के॑वल्यमु॒क्थमव्य॑थयथ्स्तभ्नातु वैरा॒जꣳ साम॒ प्रति॑ष्ठित्या॒ अधि॑पत्न्यसि बृह॒ती दिङ्म॒रुत॑स्ते दे॒वा अधि॑पतयो॒~(५)\ip

%4.4.2.3
बृह॒स्पति॑र्\mbox{}हेती॒नां प्र॑तिध॒र्ता त्रि॑णवत्रयस्त्रि॒ꣳ॒शौ त्वा॒ स्तोमौ॑ पृथि॒व्याꣴ श्र॑यतां वैश्वदेवाग्निमारु॒ते उ॒क्थे अव्य॑थयन्ती स्तभ्नीताꣳ शाक्वररैव॒ते साम॑नी॒ प्रति॑ष्ठित्या अ॒न्तरि॑क्षा॒यर्\mbox{}ष॑यस्त्वा प्रथम॒जा दे॒वेषु॑ दि॒वो मात्र॑या वरि॒णा प्र॑थन्तु विध॒र्ता चा॒यमधि॑\-पतिश्च॒ ते त्वा॒ सर्वे॑ संविदा॒ना नाक॑स्य पृ॒ष्ठे सु॑व॒र्गे लो॒के यज॑मानं च सादयन्तु॥~(६)\ip

{\anuvakamend[{प्र॒तीची॒ दिङ्म॒रुत॑स्ते दे॒वा अधि॑पतयश्चत्वारि॒ꣳ॒शच्च॑}]}

%4.4.3.1
अ॒यं पु॒रो हरि॑केशः॒ सूर्य॑रश्मि॒स्तस्य॑ रथगृ॒थ्सश्च॒ रथौ॑जाश्च सेनानिग्राम॒ण्यौ॑ पुञ्जिकस्थ॒ला च॑ कृतस्थ॒ला चा᳚फ्स॒रसौ॑ यातु॒धाना॑ हे॒ती रक्षाꣳ॑सि॒ प्रहे॑तिर॒यं द॑क्षि॒णा वि॒श्वक॑र्मा॒ तस्य॑ रथस्व॒नश्च॒ रथे॑चित्रश्च सेनानिग्राम॒ण्यौ॑ मेन॒का च॑ सहज॒न्या चा᳚फ्स॒रसौ॑ द॒ङ्क्ष्णवः॑ प॒शवो॑ हे॒तिः पौरु॑षेयो व॒धः प्रहे॑तिर॒यं प॒श्चाद्वि॒श्वव्य॑चा॒स्तस्य॒ रथ॑प्रोत॒श्चास॑मरथश्च सेनानिग्राम॒ण्यौ᳚ प्र॒म्लोच॑न्ती चा-~(७)\ip

%4.4.3.2
नु॒म्लोच॑न्ती चाफ्स॒रसौ॑ स॒र्पा हे॒तिर्व्या॒घ्राः प्रहे॑तिर॒यमु॑त्त॒राथ्\-सं॒यद्व॑सु॒स्तस्य॑ सेन॒जिच्च॑ सु॒षेण॑श्च सेनानिग्राम॒ण्यौ॑ वि॒श्वाची॑ च घृ॒ताची॑ चाफ्स॒रसा॒वापो॑ हे॒तिर्वातः॒ प्रहे॑ति\-र॒यमु॒पर्य॒र्वाग्व॑सु॒स्तस्य॒ तार्क्ष्य॒श्चारि॑ष्टनेमिश्च सेनानिग्राम॒ण्या॑\-वु॒र्वशी॑ च पू॒र्वचि॑त्तिश्चाफ्स॒रसौ॑ वि॒द्युद्धे॒तिर॑व॒स्फूर्ज॒न्प्रहे॑ति॒स्तेभ्यो॒ नम॒स्ते नो॑ मृडयन्तु॒ ते यं~(८)\ip

%4.4.3.3
द्वि॒ष्मो यश्च॑ नो॒ द्वेष्टि॒ तं वो॒ जम्भे॑ दधाम्या॒योस्त्वा॒ सद॑ने सादया॒म्यव॑तश्छा॒यायां॒ नमः॑ समु॒द्राय॒ नमः॑ समु॒द्रस्य॒ चक्ष॑से परमे॒ष्ठी त्वा॑ सादयतु दि॒वः पृ॒ष्ठे व्यच॑स्वतीं॒ प्रथ॑स्वतीं वि॒भूम॑तीं प्र॒भूम॑तीं परि॒भूम॑तीं॒ दिवं॑ यच्छ॒ दिवं॑ दृꣳह॒ दिवं॒ मा हिꣳ॑सी॒र्विश्व॑स्मै प्रा॒णाया॑पा॒नाय॑ व्या॒नायो॑दा॒नाय॑ प्रति॒ष्ठायै॑ च॒रित्रा॑य॒ सूर्य॑स्त्वा॒ऽभि पा॑तु म॒ह्या स्व॒स्त्या छ॒र्दिषा॒ शन्त॑मेन॒ तया॑ दे॒वत॑या\-ऽ\-ङ्गिर॒स्वद्ध्रु॒वा सी॑द। प्रोथ॒दश्वो॒ न यव॑से अवि॒ष्यन् य॒दा म॒हः सं॒वर॑णा॒द्व्यस्था᳚त्। आद॑स्य॒ वातो॒ अनु॑ वाति शो॒चिरध॑ स्म ते॒ व्रज॑नं कृ॒ष्णम॑स्ति॥~(९)\ip

{\anuvakamend[{प्र॒म्लोच॑न्ती च॒ यꣴ स्व॒स्त्याष्टाविꣳ॑शतिश्च}]}

%4.4.4.1
अ॒ग्निर्मू॒र्धा दि॒वः क॒कुत्पतिः॑ पृथि॒व्या अ॒यम्। अ॒पाꣳ रेताꣳ॑सि जिन्वति॥ त्वाम॑ग्ने॒ पुष्क॑रा॒दध्यथ॑र्वा॒ निर॑मन्थत। मू॒र्ध्नो विश्व॑स्य वा॒घतः॑॥ अ॒यम॒ग्निः स॑ह॒स्रिणो॒ वाज॑स्य श॒तिन॒स्पतिः॑। मू॒र्धा क॒वी र॑यी॒णाम्॥ भुवो॑ य॒ज्ञस्य॒ रज॑सश्च ने॒ता यत्रा॑ नि॒युद्भिः॒ सच॑से शि॒वाभिः॑। दि॒वि मू॒र्धानं॑ दधिषे सुव॒र्॒\mbox{}षां जि॒ह्वाम॑ग्ने चकृषे हव्य॒वाहम्᳚॥ अबो᳚ध्य॒ग्निः स॒मिधा॒ जना॑नां॒~(१०)\ip

%4.4.4.2
प्रति॑ धे॒नुमि॑वाय॒तीमु॒षासम्᳚। य॒ह्वा इ॑व॒ प्र व॒यामु॒ज्जिहा॑नाः॒ प्र भा॒नवः॑ सिस्रते॒ नाक॒मच्छ॑। अवो॑चाम क॒वये॒ मेध्या॑य॒ वचो॑ व॒न्दारु॑ वृष॒भाय॒ वृष्णे᳚। गवि॑ष्ठिरो॒ नम॑सा॒ स्तोम॑म॒ग्नौ दि॒वीव॑ रु॒क्ममु॒र्व्यञ्च॑मश्रेत्। जन॑स्य गो॒पा अ॑जनिष्ट॒ जागृ॑विर॒ग्निः सु॒दक्षः॑ सुवि॒ताय॒ नव्य॑से। घृ॒तप्र॑तीको बृह॒ता दि॑वि॒स्पृशा᳚ द्यु॒मद्वि भा॑ति भर॒तेभ्यः॒ शुचिः॑। त्वाम॑ग्ने॒ अङ्गि॑रसो॒~(११)\ip

%4.4.4.3
गुहा॑ हि॒तमन्व॑विन्दञ्छिश्रिया॒णं वने॑वने। स जा॑यसे म॒थ्यमा॑नः॒ सहो॑ म॒हत्त्वामा॑हुः॒ सह॑सस्पु॒त्रम॑ङ्गिरः। य॒ज्ञस्य॑ के॒तुं प्र॑थ॒मं पु॒रोहि॑तम॒ग्निं नर॑स्त्रिषध॒स्थे समि॑न्धते। इन्द्रे॑ण दे॒वैः स॒रथ॒ꣳ॒ स ब॒र्॒\mbox{}हिषि॒ सीद॒न्नि होता॑ य॒जथा॑य सु॒क्रतुः॑। त्वं चि॑त्रश्रवस्तम॒ हव॑न्ते वि॒क्षु ज॒न्तवः॑। शो॒चिष्के॑शं पुरुप्रि॒याग्ने॑ ह॒व्याय॒ वोढ॑वे। सखा॑यः॒ सं वः॑ स॒म्यञ्च॒मिष॒ꣴ॒~(१२)\ip

%4.4.4.4
स्तोमं॑ चा॒ग्नये᳚। वर्\mbox{}षि॑ष्ठाय क्षिती॒नामू॒र्जो नप्त्रे॒ सह॑स्वते। सꣳस॒मिद्यु॑वसे वृष॒न्नग्ने॒ विश्वा᳚न्य॒र्य आ। इ॒डस्प॒दे समि॑ध्यसे॒ स नो॒ वसू॒न्या भ॑र। ए॒ना वो॑ अ॒ग्निं नम॑सो॒र्जो नपा॑त॒मा हु॑वे। प्रि॒यं चेति॑ष्ठमर॒तिꣴ स्व॑ध्व॒रं विश्व॑स्य दू॒तम॒मृतम्᳚। स यो॑जते अरु॒षो वि॒श्वभो॑जसा॒ स दु॑द्रव॒थ्\-स्वा॑हुतः। सु॒ब्रह्मा॑ य॒ज्ञः सु॒शमी॒~(१३)\ip

%4.4.4.5
वसू॑नां दे॒वꣳ राधो॒ जना॑नाम्। उद॑स्य शो॒चिर॑स्थादा॒जुह्वा॑नस्य मी॒ढुषः॑। उद्धू॒मासो॑ अरु॒षासो॑ दिवि॒स्पृशः॒ सम॒ग्निमि॑न्धते॒ नरः॑। अग्ने॒ वाज॑स्य॒ गोम॑त॒ ईशा॑नः सहसो यहो। अ॒स्मे धे॑हि जातवेदो॒ महि॒ श्रवः॑। स इ॑धा॒नो वसु॑ष्क॒विर॒ग्निरी॒डेन्यो॑ गि॒रा। रे॒वद॒स्मभ्यं॑ पुर्वणीक दीदिहि। क्ष॒पो रा॑जन्नु॒त त्मनाऽग्ने॒ वस्तो॑रु॒तोषसः॑। स ति॑ग्मजम्भ~(१४)\ip

%4.4.4.6
र॒क्षसो॑ दह॒ प्रति॑। आ ते॑ अग्न इधीमहि द्यु॒मन्तं॑ देवा॒जरम्᳚। यद्ध॒ स्या ते॒ पनी॑यसी स॒मिद्दी॒दय॑ति॒ द्यवीषꣴ॑ स्तो॒तृभ्य॒ आ भ॑र। आ ते॑ अग्न ऋ॒चा ह॒विः शु॒क्रस्य॑ ज्योतिषस्पते। सुश्च॑न्द्र॒ दस्म॒ विश्प॑ते॒ हव्य॑वा॒ट्तुभ्यꣳ॑ हूयत॒ इषꣴ॑ स्तो॒तृभ्य॒ आ भ॑र। उ॒भे सु॑श्चन्द्र स॒र्पिषो॒ दर्वी᳚ श्रीणीष आ॒सनि॑। उ॒तो न॒ उत्पु॑पूर्या~-~(१५)\ip

%4.4.4.7
उ॒क्थेषु॑ शवसस्पत॒ इषꣴ॑ स्तो॒तृभ्य॒ आ भ॑र। अग्ने॒ तम॒द्याश्वं॒ न स्तोमैः॒ क्रतुं॒ न भ॒द्रꣳ हृ॑दि॒स्पृशम्᳚। ऋ॒ध्यामा॑ त॒ ओहैः᳚। अधा॒ ह्य॑ग्ने॒ क्रतो᳚र्भ॒द्रस्य॒ दक्ष॑स्य सा॒धोः। र॒थीर्\mbox{}ऋ॒तस्य॑ बृह॒तो ब॒भूथ॑। आ॒भिष्टे॑ अ॒द्य गी॒र्भिर्गृ॒णन्तो\-ऽग्ने॒ दाशे॑म। प्र ते॑ दि॒वो न स्त॑नयन्ति॒ शुष्माः᳚। ए॒भिर्नो॑ अ॒र्कैर्भवा॑ नो अ॒र्वाङ्~(१६)\ip

%4.4.4.8
ख्सुव॒र्न ज्योतिः॑। अग्ने॒ विश्वे॑भिः सु॒मना॒ अनी॑कैः। अ॒ग्निꣳ होता॑रं मन्ये॒ दास्व॑न्तं॒ वसोः᳚ सू॒नुꣳ सह॑सो जा॒तवे॑दसम्। विप्रं॒ न जा॒तवे॑दसम्। य ऊ॒र्ध्वया᳚ स्वध्व॒रो दे॒वो दे॒वाच्या॑ कृ॒पा। घृ॒तस्य॒ विभ्रा᳚ष्टि॒मनु॑ शु॒क्रशो॑चिष आ॒जुह्वा॑नस्य स॒र्पिषः॑। अग्ने॒ त्वं नो॒ अन्त॑मः। उ॒त त्रा॒ता शि॒वो भ॑व वरू॒थ्यः॑। तं त्वा॑ शोचिष्ठ दीदिवः। सु॒म्नाय॑ नू॒नमी॑महे॒ सखि॑भ्यः। वसु॑र॒ग्निर्वसु॑श्रवाः। अच्छा॑ नक्षि द्यु॒मत्त॑मो र॒यिं दाः᳚॥~(१७)\ip

{\anuvakamend[{जना॑ना॒मङ्गि॑रस॒ इषꣳ॑ सु॒शमी॑ तिग्मजम्भ पुपूर्या अ॒र्वाङ्वसु॑श्रवाः॒ पञ़्च॑ च}]}

%4.4.5.1
इ॒न्द्रा॒ग्नि\-भ्यां᳚ त्वा स॒युजा॑ यु॒जा यु॑नज्म्याघा॒राभ्यां॒ तेज॑सा॒ वर्च॑सो॒क्थेभिः॒ स्तोमे॑भि॒श्छन्दो॑भी र॒य्यै पोषा॑य सजा॒तानां᳚ मध्यम॒स्थेया॑य॒ मया᳚ त्वा स॒युजा॑ यु॒जा यु॑नज्म्य॒म्बा दु॒ला नि॑त॒त्निर॒भ्रय॑न्ती मे॒घय॑न्ती व॒र्॒\mbox{}षय॑न्ती चुपु॒णीका॒ नामा॑सि प्र॒जा\-प॑तिना त्वा॒ विश्वा॑भिर्धी॒भिरुप॑ दधामि पृथि॒व्यु॑दपु॒रमन्ने॑न वि॒ष्टा म॑नु॒ष्या᳚स्ते गो॒प्तारो॒\-ऽग्निर्विय॑त्तो\-ऽस्यां॒ ताम॒हं प्र प॑द्ये॒ सा~(१८)\ip

%4.4.5.2
मे॒ शर्म॑ च॒ वर्म॑ चा॒स्त्वधि॑द्यौर॒न्तरि॑क्षं॒ ब्रह्म॑णा वि॒ष्टा म॒रुत॑स्ते गो॒प्तारो॑ वा॒युर्विय॑त्तो\-ऽस्यां॒ ताम॒हं प्र प॑द्ये॒ सा मे॒ शर्म॑ च॒ वर्म॑ चास्तु॒ द्यौरप॑राजिता॒मृते॑न वि॒ष्टा\-ऽ\-ऽ\-दि॒त्यास्ते॑ गो॒प्तारः॒ सूर्यो॒ विय॑त्तो\-ऽस्यां॒ ताम॒हं प्र प॑द्ये॒ सा मे॒ शर्म॑ च॒ वर्म॑ चास्तु॥~(१९)\ip

{\anuvakamend[{सा\-ऽष्टाच॑त्वारिꣳशच्च}]}

%4.4.6.1
बृह॒स्पति॑स्त्वा सादयतु पृथि॒व्याः पृ॒ष्ठे ज्योति॑ष्मतीं॒ विश्व॑स्मै प्रा॒णाया॑पा॒नाय॒ विश्वं॒ ज्योति॑र्यच्छा॒ग्निस्ते\-ऽधि॑\-पतिर्वि॒श्वक॑र्मा त्वा सादयत्व॒न्त\-रि॑क्षस्य पृ॒ष्ठे ज्योति॑ष्मतीं॒ विश्व॑स्मै प्रा॒णाया॑पा॒नाय॒ विश्वं॒ ज्योति॑र्यच्छ वा॒युस्ते\-ऽधि॑\-पतिः प्र॒जा\-प॑तिस्त्वा सादयतु दि॒वः पृ॒ष्ठे ज्योति॑ष्मतीं॒ विश्व॑स्मै प्रा॒णाया॑पा॒नाय॒ विश्वं॒ ज्योति॑र्यच्छ परमे॒ष्ठी ते\-ऽधि॑\-पतिः पुरोवात॒सनि॑रस्यभ्र॒सनि॑रसि विद्यु॒थ्सनि॑-~(२०)\ip

%4.4.6.2
रसि स्तनयित्नु॒सनि॑रसि वृष्टि॒सनि॑रस्य॒ग्नेर्यान्य॑सि दे॒वाना॑\-मग्ने॒यान्य॑सि वा॒योर्यान्य॑सि दे॒वानां᳚ वायो॒यान्य॑स्य॒न्त\-रि॑क्षस्य॒ यान्य॑सि दे॒वाना॑\-मन्तरिक्ष॒यान्य॑स्य॒न्तरि॑क्षमस्य॒न्तरि॑क्षाय त्वा सलि॒लाय॑ त्वा॒ सर्णी॑काय त्वा॒ सती॑काय त्वा॒ केता॑य त्वा॒ प्रचे॑तसे त्वा॒ विव॑स्वते त्वा दि॒वस्त्वा॒ ज्योति॑ष आदि॒त्येभ्य॑स्त्व॒र्चे त्वा॑ रु॒चे त्वा᳚ द्यु॒ते त्वा॑ भा॒से त्वा॒ ज्योति॑षे त्वा यशो॒दां त्वा॒ यश॑सि तेजो॒दां त्वा॒ तेज॑सि पयो॒दां त्वा॒ पय॑सि वर्चो॒दां त्वा॒ वर्च॑सि द्रविणो॒दां त्वा॒ द्रवि॑णे सादयामि॒ तेनर्\mbox{}षि॑णा॒ तेन॒ ब्रह्म॑णा॒ तया॑ दे॒वत॑या\-ऽ\-ङ्गिर॒स्वद्ध्रु॒वा सी॑द॥~(२१)\ip

{\anuvakamend[{वि॒द्यु॒थ्सनि॑र्द्यु॒त्वैका॒न्नत्रि॒ꣳ॒शच्च॑}]}

%4.4.7.1
भू॒य॒स्कृद॑सि वरिव॒स्कृद॑सि॒ प्राच्य॑स्यू॒र्ध्वास्य॑न्तरिक्ष॒सद॑स्य॒न्त\-रि॑क्षे सीदाफ्सु॒षद॑सि श्येन॒सद॑सि गृध्र॒सद॑सि सुपर्ण॒सद॑सि नाक॒सद॑सि पृथि॒व्यास्त्वा॒ द्रवि॑णे सादयाम्य॒न्त\-रि॑क्षस्य त्वा॒ द्रवि॑णे सादयामि दि॒वस्त्वा॒ द्रवि॑णे सादयामि दि॒शां त्वा॒ द्रवि॑णे सादयामि द्रविणो॒दां त्वा॒ द्रवि॑णे सादयामि प्रा॒णं मे॑ पाह्यपा॒नं मे॑ पाहि व्या॒नं मे॑~(२२)\ip

%4.4.7.2
पा॒ह्यायु॑र्मे पाहि वि॒श्वायु॑र्मे पाहि स॒र्वायु॑र्मे पा॒ह्यग्ने॒ यत्ते॒ पर॒ꣳ॒ हृन्नाम॒ तावेहि॒ सꣳ र॑भावहै॒ पाञ्च॑जन्ये॒ष्वप्ये᳚ध्यग्ने॒ यावा॒ अया॑वा॒ एवा॒ ऊमाः॒ सब्दः॒ सग॑रः सु॒मेकः॑॥~(२३)\ip

{\anuvakamend[{व्या॒नं मे॒ द्वात्रिꣳ॑शच्च}]}

%4.4.8.1
अ॒ग्निना॑ विश्वा॒षाट्थ्सूर्ये॑ण स्व॒राट्क्रत्वा॒ शची॒पति॑र्\mbox{}ऋष॒भेण॒ त्वष्टा॑ य॒ज्ञेन॑ म॒घवा॒न्दक्षि॑णया सुव॒र्गो म॒न्युना॑ वृत्र॒हा सौहा᳚र्द्येन तनू॒धा अन्ने॑न॒ गयः॑ पृथि॒व्यास॑नोदृ॒ग्भिर॑न्ना॒दो व॑षट्का॒रेण॒र्द्धः साम्ना॑ तनू॒पा वि॒राजा॒ ज्योति॑ष्मा॒न् ब्रह्म॑णा सोम॒पा गोभि॑र्य॒ज्ञं दा॑धार क्ष॒त्रेण॑ मनु॒ष्या॑नश्वे॑न च॒ रथे॑न च व॒ज्र्यृ॑तुभिः॑ प्र॒भुः सं॑वथ्स॒रेण॑ परि॒भूस्तप॒सा\-ऽ\-ना॑धृष्टः॒ सूर्यः॒ सन्त॒नूभिः॑॥~(२४)\ip

{\anuvakamend[{अ॒ग्निनैका॒न्नप॑ञ्चा॒शत्}]}

%4.4.9.1
प्र॒जा\-प॑ति॒र्मन॒सा\-ऽ\-न्धो\-ऽच्छे॑तो धा॒ता दी॒क्षायाꣳ॑ सवि॒ता भृ॒त्यां पू॒षा सो॑म॒क्रय॑ण्यां॒ वरु॑ण॒ उप॑न॒द्धो\-ऽसु॑रः क्री॒यमा॑णो मि॒त्रः क्री॒तः शि॑पिवि॒ष्ट आसा॑दितो न॒रन्धि॑षः प्रो॒ह्यमा॒णो\-ऽधि॑\-पति॒राग॑तः प्र॒जा\-प॑तिः प्रणी॒यमा॑नो॒\-ऽग्निराग्नी᳚ध्रे॒ बृह॒स्पति॒राग्नी᳚ध्रात्प्रणी॒यमा॑न॒ इन्द्रो॑ हवि॒र्धाने\-ऽदि॑ति॒रासा॑दितो॒ विष्णु॑रुपावह्रि॒यमा॒णो\-ऽथ॒र्वोपो᳚त्तो य॒मो॑\-ऽभिषु॑तो\-ऽपूत॒पा आ॑धू॒यमा॑नो वा॒युः पू॒यमा॑नो मि॒त्रः क्षी॑र॒श्रीर्म॒न्थी स॑क्तु॒श्रीर्वै᳚श्वदे॒व उन्नी॑तो रु॒द्र आहु॑तो वा॒युरावृ॑त्तो नृ॒चक्षाः॒ प्रति॑ख्यातो भ॒क्ष आग॑तः पितृ॒णां ना॑राश॒ꣳ॒सो\-ऽसु॒रात्तः॒ सिन्धु॑रवभृ॒थम॑वप्र॒यन्थ्स॑मु॒द्रो\-ऽव॑गतः सलि॒लः प्रप्लु॑तः॒ सुव॑रु॒दृचं॑ ग॒तः॥~(२५)\ip

{\anuvakamend[{रु॒द्र एक॑विꣳशतिश्च}]}

%4.4.10.1
कृत्ति॑का॒ नक्ष॑त्रम॒ग्निर्दे॒वता॒ऽग्ने रुचः॑ स्थ प्र॒जा\-प॑तेर्धा॒तुः सोम॑स्य॒र्चे त्वा॑ रु॒चे त्वा᳚ द्यु॒ते त्वा॑ भा॒से त्वा॒ ज्योति॑षे त्वा रोहि॒णी नक्ष॑त्रं प्र॒जा\-प॑तिर्दे॒वता॑ मृगशी॒र्॒\mbox{}षं नक्ष॑त्र॒ꣳ॒ सोमो॑ दे॒वता॒ऽऽर्द्रा नक्ष॑त्रꣳ रु॒द्रो दे॒वता॒ पुन॑र्वसू॒ नक्ष॑त्र॒मदि॑ति\-र्दे॒वता॑ ति॒ष्यो॑ नक्ष॑त्रं॒ बृह॒स्पति॑र्दे॒वता᳚ऽऽश्रे॒षा नक्ष॑त्रꣳ स॒र्पा दे॒वता॑ म॒घा नक्ष॑त्रं पि॒तरो॑ दे॒वता॒ फल्गु॑नी॒ नक्ष॑त्र-~(२६)\ip

%4.4.10.2
मर्य॒मा दे॒वता॒ फल्गु॑नी॒ नक्ष॑त्रं॒ भगो॑ दे॒वता॒ हस्तो॒ नक्ष॑त्रꣳ सवि॒ता दे॒वता॑ चि॒त्रा नक्ष॑त्र॒मिन्द्रो॑ दे॒वता᳚ स्वा॒ती नक्ष॑त्रं वा॒युर्दे॒वता॒ विशा॑खे॒ नक्ष॑त्रमिन्द्रा॒ग्नी दे॒वता॑\-ऽनूरा॒धा नक्ष॑त्रं मि॒त्रो दे॒वता॑ रोहि॒णी नक्ष॑त्र॒मिन्द्रो॑ दे॒वता॑ वि॒चृतौ॒ नक्ष॑त्रं पि॒तरो॑ दे॒वता॑ऽषा॒ढा नक्ष॑त्र॒मापो॑ दे॒वता॑ऽषा॒ढा नक्ष॑त्रं॒ विश्वे॑ दे॒वा दे॒वता᳚ श्रो॒णा नक्ष॑त्रं॒ विष्णु॑र्दे॒वता॒ श्रवि॑ष्ठा॒ नक्ष॑त्रं॒ वस॑वो~(२७)\ip

%4.4.10.3
दे॒वता॑ श॒तभि॑ष॒ङ्नक्ष॑त्र॒मिन्द्रो॑ दे॒वता᳚ प्रोष्ठप॒दा नक्ष॑त्रम॒ज एक॑पाद्दे॒वता᳚ प्रोष्ठप॒दा नक्ष॑त्र॒महि॑र्बु॒ध्नियो॑ दे॒वता॑ रे॒वती॒ नक्ष॑त्रं पू॒षा दे॒वता᳚\-ऽ\-श्व॒युजौ॒ नक्ष॑त्रम॒श्विनौ॑ दे॒वता॑ऽप॒\-भर॑णी॒र्नक्ष॑त्रं य॒मो दे॒वता॑ पू॒र्णा प॒श्चाद्यत्ते॑ दे॒वा अद॑धुः॥~(२८)\ip

{\anuvakamend[{फल्गु॑नी॒ नक्ष॑त्रं॒ वस॑व॒स्त्रय॑स्त्रिꣳशच्च}]}%॥10॥

%4.4.11.1
मधु॑श्च॒ माध॑वश्च॒ वास॑न्तिकावृ॒तू शु॒क्रश्च॒ शुचि॑श्च॒ ग्रैष्मा॑वृ॒तू नभ॑श्च नभ॒स्य॑श्च॒ वार्\mbox{}षि॑कावृ॒तू इ॒षश्चो॒र्जश्च॑ शार॒दावृ॒तू सह॑श्च सह॒स्य॑श्च॒ हैम॑न्तिकावृ॒तू तप॑श्च तप॒स्य॑श्च शैशि॒रावृ॒तू अ॒ग्नेर॑न्तःश्ले॒षो॑\-ऽसि॒ कल्पे॑तां॒ द्यावा॑\-पृथि॒वी कल्प॑न्ता॒माप॒ ओष॑धीः॒ कल्प॑न्ताम॒ग्नयः॒ पृथ॒ङ्मम॒ ज्यैष्ठ्या॑य॒ सव्र॑ता॒~-~(२९)\ip

%4.4.11.2
ये᳚\-ऽग्नयः॒ सम॑नसो\-ऽन्त॒रा द्यावा॑\-पृथि॒वी शै॑शि॒रावृ॒तू अ॒भि कल्प॑माना॒ इन्द्र॑मिव दे॒वा अ॒भि सं वि॑शन्तु सं॒यच्च॒ प्रचे॑ताश्चा॒ग्नेः सोम॑स्य॒ सूर्य॑स्यो॒ग्रा च॑ भी॒मा च॑ पितृ॒णां य॒मस्येन्द्र॑स्य ध्रु॒वा च॑ पृथि॒वी च॑ दे॒वस्य॑ सवि॒तुर्म॒रुतां॒ वरु॑णस्य ध॒र्त्री च॒ धरि॑त्री च मि॒त्रावरु॑णयोर्मि॒त्रस्य॑ धा॒तुः प्राची॑ च प्र॒तीची॑ च॒ वसू॑नाꣳ रु॒द्राणा॑-~(३०)\ip

%4.4.11.3
मादि॒त्यानां॒ ते ते\-ऽधि॑पतय॒स्तेभ्यो॒ नम॒स्ते नो॑ मृडयन्तु॒ ते यं द्वि॒ष्मो यश्च॑ नो॒ द्वेष्टि॒ तं वो॒ जम्भे॑ दधामि स॒हस्र॑स्य प्र॒मा अ॑सि स॒हस्र॑स्य प्रति॒मा अ॑सि स॒हस्र॑स्य वि॒मा अ॑सि स॒हस्र॑स्यो॒न्मा अ॑सि साह॒स्रो॑\-ऽसि स॒हस्रा॑य त्वे॒मा मे॑ अग्न॒ इष्ट॑का धे॒नवः॑ स॒न्त्वेका॑ च श॒तं च॑ स॒हस्रं॑ चा॒युतं॑ च~(३१)\ip

%4.4.11.4
नि॒युतं॑ च प्र॒युतं॒ चार्बु॑दं च॒ न्य॑र्बुदं च समु॒द्रश्च॒ मध्यं॒ चान्त॑श्च परा॒र्धश्चे॒मा मे॑ अग्न॒ इष्ट॑का धे॒नवः॑ सन्तु ष॒ष्टिः स॒हस्र॑म॒युत॒मक्षी॑यमाणा ऋत॒स्थाः स्थ॑र्ता॒वृधो॑ घृत॒श्चुतो॑ मधु॒श्चुत॒ ऊर्ज॑स्वतीः स्वधा॒विनी॒स्ता मे॑ अग्न॒ इष्ट॑का धे॒नवः॑ सन्तु वि॒राजो॒ नाम॑ काम॒दुघा॑ अ॒मुत्रा॒मुष्मिँ॑ल्लो॒के॥~(३२)\ip

{\anuvakamend[{सव्र॑ता रु॒द्राणा॑म॒युतं॑ च॒ पञ्च॑चत्वारिꣳशच्च}]}%॥11॥

%4.4.12.1
स॒मिद्दि॒शामा॒शया॑ नः सुव॒र्विन्मधो॒रतो॒ माध॑वः पात्व॒स्मान्। अ॒ग्निर्दे॒वो दु॒ष्टरी॑तु॒रदा᳚भ्य इ॒दं क्ष॒त्रꣳ र॑क्षतु॒ पात्व॒स्मान्। र॒थ॒न्त॒रꣳ साम॑भिः पात्व॒स्मान्गा॑य॒त्री छन्द॑सां वि॒श्वरू॑पा। त्रि॒वृन्नो॑ वि॒ष्ठया॒ स्तोमो॒ अह्नाꣳ॑ समु॒द्रो वात॑ इ॒दमोजः॑ पिपर्तु। उ॒ग्रा दि॒शाम॒भिभू॑तिर्वयो॒धाः शुचिः॑ शु॒क्रे अह॑न्योज॒सीना᳚। इन्द्राधि॑\-पतिः पिपृता॒दतो॑ नो॒ महि॑~(३३)\ip

%4.4.12.2
क्ष॒त्रं वि॒श्वतो॑ धारये॒दम्। बृ॒हथ्साम॑ क्षत्र॒भृद्वृ॒द्धवृ॑ष्णियं त्रि॒ष्टुभौजः॑ शुभि॒तमु॒ग्रवी॑रम्। इन्द्र॒ स्तोमे॑न पञ्चद॒शेन॒ मध्य॑मि॒दं वाते॑न॒ सग॑रेण रक्ष। प्राची॑ दि॒शाꣳ स॒हय॑शा॒ यश॑स्वती॒ विश्वे॑ देवाः प्रा॒वृषाह्ना॒ꣳ॒ सुव॑र्वती। इ॒दं क्ष॒त्रं दु॒ष्टर॑म॒स्त्वोजो\-ऽना॑धृष्टꣳ सह॒स्रिय॒ꣳ॒ सह॑स्वत्। वै॒रू॒पे साम॑न्नि॒ह तच्छ॑केम॒ जग॑त्यैनं वि॒क्ष्वा वे॑शयामः। विश्वे॑ देवाः सप्तद॒शेन॒~(३४)\ip

%4.4.12.3
वर्च॑ इ॒दं क्ष॒त्रꣳ स॑लि॒लवा॑तमु॒ग्रम्। ध॒र्त्री दि॒शां क्ष॒त्रमि॒दं दा॑धारोप॒स्थाशा॑नां मि॒त्रव॑द॒स्त्वोजः॑। मित्रा॑वरुणा श॒रदाह्नां᳚ चिकित्नू अ॒स्मै रा॒ष्ट्राय॒ महि॒ शर्म॑ यच्छतम्। वै॒रा॒जे साम॒न्नधि॑ मे मनी॒षाऽनु॒ष्टुभा॒ सम्भृ॑तं वी॒र्यꣳ॑ सहः॑। इ॒दं क्ष॒त्रं मि॒त्रव॑दा॒र्द्रदा॑नु॒ मित्रा॑वरुणा॒ रक्ष॑त॒माधि॑पत्यैः। स॒म्राड्दि॒शाꣳ स॒हसा᳚म्नी॒ सह॑स्वत्यृ॒तुर्\mbox{}हे॑म॒न्तो वि॒ष्ठया॑ नः पिपर्तु। अ॒व॒स्युवा॑ता~-~(३५)\ip

%4.4.12.4
बृह॒तीर्नु शक्व॑रीरि॒मं य॒ज्ञम॑वन्तु नो घृ॒ताचीः᳚। सुव॑र्वती सु॒दुघा॑ नः॒ पय॑स्वती दि॒शां दे॒व्य॑वतु नो घृ॒ताची᳚। त्वं गो॒पाः पु॑रए॒तोत प॒श्चाद्बृह॑स्पते॒ याम्यां᳚ युङ्ग्धि॒ वाचम्᳚। ऊ॒र्ध्वा दि॒शाꣳ रन्ति॒राशौष॑धीनाꣳ संवथ्स॒रेण॑ सवि॒ता नो॒ अह्ना᳚म्। रे॒वथ्सामाति॑च्छन्दा उ॒ छन्दोऽजा॑तशत्रुः स्यो॒ना नो॑ अस्तु। स्तोम॑त्रयस्त्रिꣳशे॒ भुव॑नस्य पत्नि॒ विव॑स्वद्वाते अ॒भि नो॑~(३६)\ip

%4.4.12.5
गृणाहि। घृ॒तव॑ती सवित॒राधि॑पत्यैः॒ पय॑स्वती॒ रन्ति॒राशा॑ नो अस्तु। ध्रु॒वा दि॒शां वि॑ष्णुप॒त्न्यघो॑रा॒स्येशा॑ना॒ सह॑सो॒ या म॒नोता᳚। बृह॒स्पति॑र्मात॒रिश्वो॒त वा॒युः स॑न्धुवा॒ना वाता॑ अ॒भि नो॑ गृणन्तु। वि॒ष्ट॒म्भो दि॒वो ध॒रुणः॑ पृथि॒व्या अ॒स्येशा॑ना॒ जग॑तो॒ विष्णु॑पत्नी। वि॒श्वव्य॑चा इ॒षय॑न्ती॒ सुभू॑तिः शि॒वा नो॑ अ॒स्त्वदि॑तिरु॒पस्थे᳚। वै॒श्वा॒न॒रो न॑ ऊ॒त्या पृ॒ष्टो दि॒व्यनु॑ नो॒\-ऽद्यानु॑मति॒रन्विद॑नुमते॒ त्वं कया॑ नश्चि॒त्र आ भु॑व॒त्को अ॒द्य यु॑ङ्क्ते॥~(३७)\ip

{\anuvakamend[{महि॑ सप्तद॒शेना॑व॒स्युवा॑ता अ॒भि नो\-ऽनु॑ न॒श्चतु॑र्दश च}]}%॥12॥

\prashnaend{र॒श्मिर॑सि॒ राज्ञ्य॑स्य॒यं पु॒रो हरि॑केशो॒\-ऽग्निर्मू॒र्धेन्द्रा॒ग्निभ्यां॒ बृह॒स्पति॑र्भूय॒स्कृद॑स्य॒ग्निना॑ विश्वा॒षाट्प्र॒जा\-प॑ति॒र्मन॑सा॒ कृत्ति॑का॒ मधु॑श्च स॒मिद्दि॒शां द्वाद॑श॥१२॥}{र॒श्मिर॑सि॒ प्रति॑ धे॒नुम॑सि स्तनयित्नु॒सनि॑रस्यादि॒त्यानाꣳ॑ स॒प्तत्रिꣳ॑शत्॥३७॥}{र॒श्मिर॑सि॒ को अ॒द्य यु॑ङ्क्ते॥}%%४-४
{हरिः॑ ॐ}{॥कृष्ण-यजुर्वेदीय-तैत्तिरीय-संहितायां चतुर्थकाण्डे चतुर्थः प्रश्नः समाप्तः॥४-४॥}
%%% END PRASHNA
