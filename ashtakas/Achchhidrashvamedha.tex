\sect{सप्तमः प्रश्नः}
\setcounter{anuvakam}{0}
\dnsub{तैत्तिरीयब्राह्मणे तृतीयाष्टके सप्तमः प्रपाठकः}

%3.7.1.1
सर्वा॒न्॒ वा ए॒षो᳚\-ऽग्नौ कामा॒न्प्रवे॑शयति।
यो᳚ऽग्नीन॑न्वा॒धाय॑ व्र॒तमु॒पैति॑।
सयदनि॑ष्ट्वा प्रया॒यात्।
अका॑मप्रीता एनं॒ कामा॒ नानु॒प्रया॑युः।
अ॒ते॒जा अ॑वी॒र्यः॑ स्यात्।
स जु॑हुयात्।
तुभ्यं॒ ता अ॑ङ्गिरस्तम।
विश्वाः᳚ सुक्षि॒तयः॒ पृथ॑क्।
अग्ने॒ कामा॑य येमिर॒ इति॑।
कामा॑ने॒वास्मि॑न्दधाति॥१॥\ip

%3.7.1.2
काम॑प्रीता एनं॒ कामा॒ अनु॒ प्रया᳚न्ति।
ते॒ज॒स्वी वी॒र्या॑वान्भवति।
सन्त॑ति॒र्वा ए॒षा य॒ज्ञस्य॑।
यो᳚ऽग्नीन॑न्वा॒धाय॑ व्र॒तमु॒पैति॑।
स यदु॒द्वाय॑ति।
विच्छि॑त्तिरे॒वास्य॒ सा।
तं प्राञ्च॑मु॒द्धृत्य॑।
मन॒सोप॑तिष्ठेत।
मनो॒ वै प्र॒जा\-प॑तिः।
प्रा॒जा॒प॒त्यो य॒ज्ञः॥२॥\ip

%3.7.1.3
मन॑सै॒व य॒ज्ञꣳ सं त॑नोति।
भूरित्या॑ह।
भू॒तो वै प्र॒जा\-प॑तिः।
भूति॑मे॒वोपै॑ति।
वि वा ए॒ष इ॑न्द्रि॒येण॑ वी॒र्ये॑णर्ध्यते।
यस्याऽऽहि॑ताग्नेर॒ग्निर॑प॒क्षाय॑ति।
याव॒च्छम्य॑या प्र॒विध्ये᳚त्।
यदि॒ ताव॑दप॒क्षाये᳚त्।
तꣳ सम्भ॑रेत्।
इ॒दं त॒ एकं॑ प॒र उ॑ त॒ एकम्᳚॥३॥\ip

%3.7.1.4
तृ॒तीये॑न॒ ज्योति॑षा॒ संवि॑शस्व।
सं॒वेश॑नस्त॒नुवै॒ चारु॑रेधि।
प्रि॒ये दे॒वानां᳚ पर॒मे ज॒नित्र॒ इति॑।
ब्रह्म॑णै॒वैन॒ꣳ॒ सम्भ॑रति।
सैव ततः॒ प्राय॑श्चित्तिः।
यदि॑ परस्त॒राम॑प॒क्षाये᳚त्।
अ॒नु॒प्र॒यायाव॑स्येत्।
सो ए॒व ततः॒ प्राय॑श्चित्तिः।
ओष॑धी॒र्वा ए॒तस्य॑ प॒शून्पयः॒ प्रवि॑शति।
यस्य॑ ह॒विषे॑ व॒थ्सा अ॒पाकृ॑ता॒ धय॑न्ति॥४॥\ip

%3.7.1.5
तान् यद्दु॒ह्यात्।
या॒तया᳚म्ना ह॒विषा॑ यजेत।
यन्न दु॒ह्यात्।
य॒ज्ञ॒प॒रुर॒न्तरि॑यात्।
वा॒य॒व्यां᳚ यवा॒गूं निर्व॑पेत्।
वा॒युर्वै पय॑सः प्रदापयि॒ता।
स ए॒वास्मै॒ पयः॒ प्रदा॑पयति।
पयो॒ वा ओष॑धयः।
पयः॒ पयः॑।
पय॑सै॒वास्मै॒ पयो\-ऽव॑ रुन्धे॥५॥\ip

%3.7.1.6
अथोत्त॑रस्मै ह॒विषे॑ व॒थ्सान॒पाकु॑र्यात्।
सैव ततः॒ प्राय॑श्चित्तिः।
अ॒न्य॒त॒रान् वा ए॒ष दे॒वान्भा॑ग॒धेये॑न॒ व्य॑र्धयति।
ये यज॑मानस्य सा॒यं गृ॒हमा॒ गच्छ॑न्ति।
यस्य॑ सायं दु॒ग्धꣳ ह॒विरार्ति॑मा॒र्च्छति॑।
इन्द्रा॑य व्री॒हीन्नि॒रुप्योप॑ वसेत्।
पयो॒ वा ओष॑धयः।
पय॑ ए॒वाऽऽरभ्य॑ गृही॒त्वोप॑ वसति।
यत्प्रा॒तः स्यात्।
तच्छृ॒तं कु॑र्यात्॥६॥\ip

%3.7.1.7
अथेत॑र ऐ॒न्द्रः पु॑रो॒डाशः॑ स्यात्।
इ॒न्द्रि॒ये ए॒वास्मै॑ स॒मीची॑ दधाति।
पयो॒ वा ओष॑धयः।
पयः॒ पयः॑।
पय॑सै॒वास्मै॒ पयो\-ऽव॑ रुन्धे।
अथोत्त॑रस्मै ह॒विषे॑ व॒थ्सान॒पाकु॑र्यात्।
सैव ततः॒ प्राय॑श्चित्तिः।
उ॒भया॒न्॒ वा ए॒ष दे॒वान्भा॑ग॒धेये॑न॒ व्य॑र्धयति।
ये यज॑मानस्य सा॒यं च॑ प्रा॒तश्च॑ गृ॒हमा॒ गच्छ॑न्ति।
यस्यो॒भयꣳ॑ ह॒विरार्ति॑मा॒र्च्छति॑॥७॥\ip

%3.7.1.8
ऐ॒न्द्रं पञ्च॑शरावमोद॒नं निर्व॑पेत्।
अ॒ग्निं दे॒वता॑नां प्रथ॒मं य॑जेत्।
अ॒ग्निमु॑खा ए॒व दे॒वताः᳚ प्रीणाति।
अ॒ग्निं वा अन्व॒न्या दे॒वताः᳚।
इन्द्र॒मन्व॒न्याः।
ता ए॒वोभयीः᳚ प्रीणाति।
पयो॒ वा ओष॑धयः।
पयः॒ पयः॑।
पय॑सै॒वास्मै॒ पयो\-ऽव॑ रुन्धे।
अथोत्त॑रस्मै ह॒विषे॑ व॒थ्सान॒पाकु॑र्यात्॥८॥\ip

%3.7.1.9
सैव ततः॒ प्राय॑श्चित्तिः।
अ॒र्धो वा ए॒तस्य॑ य॒ज्ञस्य॑ मीयते।
यस्य॒ व्रत्ये\-ऽह॒न्पत्न्य॑नालम्भु॒का भव॑ति।
ताम॑प॒रुध्य॑ यजेत।
सर्वे॑णै॒व य॒ज्ञेन॑ यजते।
तामि॒ष्ट्वोप॑ ह्वयेत।
अमू॒हम॑स्मि।
सा त्वम्।
द्यौर॒हम्।
पृ॒थि॒वी त्वम्।
सामा॒हम्।
ऋक्त्वम्।
तावेहि॒ सम्भ॑वाव।
स॒ह रेतो॑ दधावहै।
पु॒ꣳ॒से पु॒त्राय॒ वेत्त॑वै।
रा॒यस्पोषा॑य सुप्रजा॒स्त्वाय॑ सु॒वीर्या॒येति॑।
अ॒र्ध ए॒वैना॒मुप॑ ह्वयते।
सैव ततः॒ प्राय॑श्चित्तिः॥९॥\ip\anuvakamend[द॒धा॒ति॒ य॒ज्ञ उ॑त॒ एक॒न्धय॑न्ति रुन्धे कुर्यादा॒र्च्छत्य॒पाकु॑र्यात्पृथि॒वी त्वम॒ष्टौ च॑ (सर्वा॒न्॒ वि वै यदि॑ परस्त॒रामोष॑धीरन्यत॒रानु॒भया॑न॒र्धो वै॥)]

%3.7.2.1
यद्विष्ष॑ण्णेन जुहु॒यात्।
अप्र॑जा अप॒शुर्यज॑मानः स्यात्।
यदना॑यतने नि॒नये᳚त्।
अ॒ना॒य॒त॒नः स्या᳚त्।
प्रा॒जा॒प॒त्यय॒र्चा व॑ल्मीकव॒पाया॒मव॑ नयेत्।
प्रा॒जा॒प॒त्यो वै व॒ल्मीकः॑।
य॒ज्ञः प्र॒जा\-प॑तिः।
प्र॒जा\-प॑तावे॒व य॒ज्ञं प्रति॑\-ष्ठापयति।
भूरित्या॑ह।
भू॒तो वै प्र॒जा\-प॑तिः॥१०॥\ip

%3.7.2.2
भूति॑मे॒वोपै॑ति।
तत्कृ॒त्वा।
अ॒न्यां दु॒ग्ध्वा पुन॑र्‌\mbox{}होत॒व्यम्᳚।
सैव ततः॒ प्राय॑श्चित्तिः।
यत्की॒टाव॑पन्नेन जुहु॒यात्।
अप्र॑जा अप॒शुर्यज॑मानः स्यात्।
यदना॑यतने नि॒नये᳚त्।
अ॒ना॒य॒त॒नः स्या᳚त्।
म॒ध्य॒मेन॑ प॒र्णेन॑ द्यावापृथि॒व्य॑य॒र्चा\-ऽन्तः॑ परि॒धि निन॑येत्।
द्यावा॑पृथि॒व्योरे॒वैन॒त्प्रति॑\-ष्ठापयति॥११॥\ip

%3.7.2.3
तत्कृ॒त्वा।
अ॒न्यां दु॒ग्ध्वा पुन॑र्\mbox{}होत॒व्यम्᳚।
सैव ततः॒ प्राय॑श्चित्तिः।
यदव॑वृष्टेन जुहु॒यात्।
अप॑रूपमस्या॒ऽ॒ऽ॒त्मञ्जा॑येत।
कि॒लासो॑ वा॒स्याद॑र्\mbox{}श॒सो वा᳚।
यत्प्रत्ये॒यात्।
य॒ज्ञं वि\-च्छि॑न्द्यात्।
स जु॑हुयात्।
मि॒त्रो जना᳚न्कल्पयति प्रजा॒नन्॥१२॥\ip

%3.7.2.4
मि॒त्रो दा॑धार पृथि॒वीमु॒त द्याम्।
मि॒त्रः कृ॒ष्टीरनि॑मिषा॒ऽभि च॑ष्टे।
स॒त्याय॑ ह॒व्यं घृ॒तव॑ज्जुहो॒तेति॑।
मि॒त्रेणै॒वैन॑त्कल्पयति।
तत्कृ॒त्वा।
अ॒न्यां दु॒ग्ध्वा पुन॑र्\mbox{}होत॒व्यम्᳚।
सैव ततः॒ प्राय॑श्चित्तिः।
यत्पूर्व॑स्या॒माहु॑त्याꣳ हु॒ताया॒मुत्त॒रा\-ऽऽहु॑तिः॒ स्कन्दे᳚त्।
द्वि॒पाद्भिः॑ प॒शुभि॒र्यज॑मानो॒ व्यृ॑ध्येत।
यदुत्त॑रया॒ऽभि जु॑हु॒यात्॥१३॥\ip

%3.7.2.5
चतु॑ष्पाद्भिः प॒शुभि॒र्यज॑मानो॒ व्यृ॑ध्येत।
यत्र॒ वेत्थ॑ वनस्पते दे॒वानां॒ गुह्या॒ नामा॑नि।
तत्र॑ ह॒व्यानि॑ गाम॒येति॑ वानस्प॒त्यय॒र्चा स॒मिध॑मा॒धाय॑।
तू॒ष्णीमे॒व पुन॑र्जुहुयात्।
वन॒स्पति॑नै॒व य॒ज्ञस्यार्तां॒ चाना᳚र्तां॒ चाऽऽहु॑ती॒ वि दा॑धार।
तत्कृ॒त्वा।
अ॒न्यां दु॒ग्ध्वा पुन॑र्\mbox{}होत॒व्यम्᳚।
सैव ततः॒ प्राय॑श्चित्तिः।
यत्पु॒रा प्र॑या॒जेभ्यः॒ प्राङङ्गा॑रः॒ स्कन्दे᳚त्।
अ॒ध्व॒र्यवे॑ च॒ यज॑मानाय॒ चाकꣴ॑ स्यात्॥१४॥\ip

%3.7.2.6
यद्द॑क्षि॒णा।
ब्र॒ह्मणे॑ च॒ यज॑मानाय॒ चाकꣴ॑ स्यात्।
यत्प्र॒त्यक्।
होत्रे॑ च॒ पत्नि॑यै च॒ यज॑मानाय॒ चाकꣴ॑ स्यात्।
यदुदङ्ङ्॑।
अ॒ग्नीधे॑ च प॒शुभ्य॑श्च॒ यज॑मानाय॒ चाकꣴ॑ स्यात्।
यद॑भिजुहु॒यात्।
रु॒द्रो᳚ऽस्य प॒शून्घातु॑कः स्यात्।
यन्नाभि॑जुहु॒यात्।
अशा᳚न्तः॒ प्रह्रि॑येत॥१५॥\ip

%3.7.2.7
स्रु॒वस्य॒ बुध्ने॑नाभि॒निद॑ध्यात्।
मा त॑मो॒ मा य॒ज्ञस्त॑म॒न्मा यज॑मानस्तमत्।
नम॑स्ते अस्त्वाय॒ते।
नमो॑ रुद्र पराय॒ते।
नमो॒ यत्र॑ नि॒षीद॑सि।
अ॒मुं मा हिꣳ॑सीर॒मुं मा हिꣳ॑सी॒रिति॒ येन॒ स्कन्दे᳚त्।
तं प्रह॑रेत्।
स॒हस्र॑शृङ्गो वृष॒भो जा॒तवे॑दाः।
स्तोम॑पृष्ठो घृ॒तवा᳚न्थ्सु॒प्रती॑कः।
मा नो॑ हासीन्मेत्थि॒तो नेत्त्वा॒ जहा॑म।
गो॒पो॒षं नो॑ वीरपो॒षं च॑ य॒च्छेति॑।
ब्रह्म॑णै॒वैनं॒ प्र ह॑रति।
सैव ततः॒ प्राय॑श्चित्तिः॥१६॥\ip\anuvakamend[वै प्र॒जा\-प॑तिः स्थापयति प्रजा॒नन्न॒भि जु॑हु॒याथ्स्या᳚द्ध्रियेत॒ जहा॑म॒ त्रीणि॑ च (यद्विष्ष॑ण्णेन प्राजाप॒त्यया॒ यत्की॒टा म॑ध्य॒मेन॒ यदव॑वृष्टेन॒ यत्पूर्व॑स्यां॒ यत्पु॒रा प्र॑या॒जेभ्यः॒ प्राङङ्गा॑रो॒ यद्द॑क्षि॒णा यत्प्र॒त्यग्यदुदङ्ङ्॑॥)]

%3.7.3.1
वि वा ए॒ष इ॑न्द्रि॒येण॑ वी॒र्ये॑णर्ध्यते।
यस्याऽऽहि॑ताग्ने\-र॒ग्निर्म॒थ्य\-मा॑नो॒ न जाय॑ते।
यत्रा॒न्यं पश्ये᳚त्।
तत॑ आ॒हृत्य॑ होत॒व्यम्᳚।
अ॒ग्नावे॒वास्या᳚ग्निहो॒त्रꣳ हु॒तं भ॑वति।
यद्य॒न्यन्न वि॒न्देत्।
अ॒जायाꣳ॑ होत॒व्यम्᳚।
आ॒ग्ने॒यी वा ए॒षा।
यद॒जा।
अ॒ग्नावे॒वास्या᳚ग्निहो॒त्रꣳ हु॒तं भ॑वति॥१७॥\ip

%3.7.3.2
अ॒जस्य॒ तु नाश्ञी॑यात्।
यद॒जस्या᳚श्ञी॒यात्।
यामे॒वाग्नावाहु॑तिं जुहु॒यात्।
ताम॑द्यात्।
तस्मा॑द॒जस्य॒ नाश्यम्᳚।
यद्य॒जान्न वि॒न्देत्।
ब्रा॒ह्म॒णस्य॒ दक्षि॑णे॒ हस्ते॑ होत॒व्यम्᳚।
ए॒ष वा अ॒ग्निर्वै᳚श्वान॒रः।
यद्ब्रा᳚ह्म॒णः।
अ॒ग्नावे॒वास्या᳚ग्निहो॒त्रꣳ हु॒तं भ॑वति॥१८॥\ip

%3.7.3.3
ब्रा॒ह्म॒णं तु व॑स॒त्यै॑ नाप॑ रुन्ध्यात्।
यद्ब्रा᳚ह्म॒णं व॑स॒त्या अ॑परु॒न्ध्यात्।
यस्मि॑न्ने॒वाग्नावाहु॑तिं जुहु॒यात्।
तं भा॑ग॒धेये॑न॒ व्य॑र्धयेत्।
तस्मा᳚द्ब्राह्म॒णो व॑स॒त्यै॑ नाप॒रुध्यः॑।
यदि॑ ब्राह्म॒णं न वि॒न्देत्।
द॒र्भ॒स्त॒म्बे हो॑त॒व्यम्᳚।
अ॒ग्नि॒वान् वै द॑र्भस्त॒म्बः।
अ॒ग्नावे॒वास्या᳚ग्निहो॒त्रꣳ हु॒तं भ॑वति।
द॒र्भाꣴस्तु नाध्या॑सीत॥१९॥\ip

%3.7.3.4
यद्द॒र्भान॒ध्यासी॑त।
यामे॒वाग्नावाहु॑तिं जुहु॒यात्।
तामध्या॑सीत।
तस्मा᳚द्द॒र्भा नाध्या॑सित॒व्याः᳚।
यदि॑ द॒र्भान्न वि॒न्देत्।
अ॒फ्सु हो॑त॒व्यम्᳚।
आपो॒ वै सर्वा॑ दे॒वताः᳚।
दे॒वता᳚स्वे॒वास्या᳚ग्निहो॒त्रꣳ हु॒तं भ॑वति।
आप॒स्तु न परि॑चक्षीत।
यदापः॑ परि॒चक्षी॑त॥२०॥\ip

%3.7.3.5
यामे॒वाफ्स्वाहु॑तिं जुहु॒यात्।
तां परि॑चक्षीत।
तस्मा॒दापो॒ न प॑रि॒चक्ष्याः᳚।
मेध्या॑ च॒ वा ए॒तस्या॑मे॒ध्या च॑ त॒नुवौ॒ सꣳ सृ॑ज्येते।
यस्याऽऽहि॑ताग्नेर॒न्यैर॒ग्निभि॑र॒ग्नयः॑ सꣳसृ॒ज्यन्ते᳚।
अ॒ग्नये॒ विवि॑चये पुरो॒डाश॑म॒ष्टा\-क॑पालं॒ निर्व॑पेत्।
मेध्यां चै॒वास्या॑मे॒ध्यां च॑ त॒नुवौ॒ व्याव॑र्तयति।
अ॒ग्नये᳚ व्र॒तप॑तये पुरो॒डाश॑म॒ष्टा\-क॑पालं॒ निर्व॑पेत्।
अ॒ग्निमे॒व व्र॒तप॑ति॒ꣴ॒ स्वेन॑ भाग॒धेये॒नोप॑ धावति।
स ए॒वैनं॑ व्र॒तमा ल॑म्भयति॥२१॥\ip

%3.7.3.6
गर्भ॒ꣴ॒ स्रव॑न्तमग॒दम॑कः।
अ॒ग्निरिन्द्र॒स्त्वष्टा॒ बृह॒स्पतिः॑।
पृ॒थि॒व्यामव॑ चुश्चोतै॒तत्।
नाभि\-प्राप्नो॑ति॒ निर्‌\mbox{}ऋ॑तिं परा॒चैः।
रेतो॒ वा ए॒तद्वाजि॑न॒माहि॑ताग्नेः।
यद॑ग्निहो॒त्रम्।
तद्यथ्स्रवे᳚त्।
रेतो᳚\-ऽस्य॒ वाजि॑नꣴ स्रवेत्।
गर्भ॒ꣴ॒ स्रव॑न्तमग॒दम॑क॒रित्या॑ह।
रेत॑ ए॒वास्मि॒न्वाजि॑नं दधाति॥२२॥\ip

%3.7.3.7
अ॒ग्निरित्या॑ह।
अ॒ग्निर्वै रे॑तो॒धाः।
रेत॑ ए॒व तद्द॑धाति।
इन्द्र॒ इत्या॑ह।
इ॒न्द्रि॒यमे॒वास्मि॑न्दधाति।
त्वष्टेत्या॑ह।
त्वष्टा॒ वै प॑शू॒नां मि॑थु॒नानाꣳ॑ रूप॒कृत्।
रू॒पमे॒व प॒शुषु॑ दधाति।
बृह॒स्पति॒रित्या॑ह।
ब्रह्म॒ वै दे॒वानां॒ बृह॒स्पतिः॑।
ब्रह्म॑णै॒वास्मै᳚ प्र॒जाः प्र ज॑नयति।
पृ॒थि॒व्यामव॑ चुश्चोतै॒तदित्या॑ह।
अ॒स्यामे॒वैन॒त्प्रति॑\-ष्ठापयति।
नाभिप्राप्नो॑ति॒ निर्‌\mbox{}ऋ॑तिं परा॒चैरित्या॑ह।
रक्ष॑सा॒मप॑हत्यै॥२३॥\ip\anuvakamend[अ॒जा\-ऽग्नावे॒वास्या᳚ग्निहो॒त्रꣳ हु॒तं भ॑वति भवत्यासीत परि॒चक्षी॑त लम्भयति दधाति दे॒वानां॒ बृह॒स्पतिः॒ पञ्च॑ च  (वि वै यद्य॒न्यम॒जायां᳚ ब्राह्म॒णस्य॑ दर्भस्त॒म्बे᳚\-ऽफ्सु हो॑त॒व्यम्᳚।)]

%3.7.4.1
याः पु॒रस्ता᳚त्प्र॒स्रव॑न्ति।
उ॒परि॑ष्टाथ्स॒र्वत॑श्च॒ याः।
ताभी॑ र॒श्मिप॑वित्राभिः।
श्र॒द्धां य॒ज्ञमा र॑भे।
देवा॑ गातुविदः।
गा॒तुं य॒ज्ञाय॑ विन्दत।
मन॑स॒स्पति॑ना दे॒वेन॑।
वाता᳚द्य॒ज्ञः प्र यु॑ज्यताम्।
तृ॒तीय॑स्यै दि॒वः।
गा॒य॒त्रि॒या सोम॒ आभृ॑तः॥२४॥\ip

%3.7.4.2
सो॒म॒पी॒थाय॒ सन्न॑यितुम्।
वक॑ल॒मन्त॑र॒मा द॑दे।
आपो॑ देवीः शु॒द्धाः स्थ॑।
इ॒मा पात्रा॑णि शुन्धत।
उ॒पा॒त॒ङ्क्या॑य दे॒वाना᳚म्।
प॒र्ण॒व॒ल्कमु॒त शु॑न्धत।
पयो॑ गृ॒हेषु॒ पयो॑ अघ्नि॒यासु॑।
पयो॑ व॒थ्सेषु॒ पय॒ इन्द्रा॑य ह॒विषे᳚ ध्रियस्व।
गा॒य॒त्री प॑र्णव॒ल्केन॑।
पयः॒ सोमं॑ करोत्वि॒मम्॥२५॥\ip

%3.7.4.3
अ॒ग्निं गृ॑ह्णामि सु॒रथं॒ यो म॑यो॒भूः।
य उ॒द्यन्त॑मा॒रोह॑ति॒ सूर्य॒मह्ने᳚।
आ॒दि॒त्यं ज्योति॑षां॒ ज्योति॑रुत्त॒मम्।
श्वो य॒ज्ञाय॑ रमतां दे॒वता᳚भ्यः।
वसू᳚न्रु॒द्राना॑दि॒त्यान्।
इन्द्रे॑ण स॒ह दे॒वताः᳚।
ताः पूर्वः॒ परि॑ गृह्णामि।
स्व आ॒यत॑ने मनी॒षया᳚।
इ॒मामूर्जं॑ पञ्चद॒शीं ये प्रवि॑ष्टाः।
तान्दे॒वान्परि॑ गृह्णामि॒ पूर्वः॑॥२६॥\ip

%3.7.4.4
अ॒ग्निर्‌\mbox{}ह॑व्य॒वाडि॒ह ताना व॑हतु।
पौ॒र्ण॒मा॒सꣳ ह॒विरि॒दमे॑षां॒ मयि॑।
आ॒मा॒वा॒स्यꣳ॑ ह॒विरि॒दमे॑षां॒ मयि॑।
अ॒न्त॒रा\-ऽग्नी प॒शवः॑।
दे॒व॒स॒ꣳ॒सद॒मा ग॑मन्।
तान्पूर्वः॒ परि॑ गृह्णामि।
स्व आ॒यत॑ने मनी॒षया᳚।
इ॒ह प्र॒जा वि॒श्वरू॑पा रमन्ताम्।
अ॒ग्निं गृ॒हप॑तिम॒भि सं॒वसा॑नाः।
ताः पूर्वः॒ परि॑ गृह्णामि॥२७॥\ip

%3.7.4.5
स्व आ॒यत॑ने मनी॒षया᳚।
इ॒ह प॒शवो॑ वि॒श्वरू॑पा रमन्ताम्।
अ॒ग्निं गृ॒हप॑तिम॒भि सं॒वसा॑नाः।
तान्पूर्वः॒ परि॑ गृह्णामि।
स्व आ॒यत॑ने मनी॒षया᳚।
अ॒यं पि॑तृ॒णाम॒ग्निः।
अवा᳚ड्ढ॒व्या पि॒तृभ्य॒ आ।
तं पूर्वः॒ परि॑ गृह्णामि।
अवि॑षन्नः पि॒तुं क॑रत्।
अज॑स्रं॒ त्वाꣳ स॑भापा॒लाः॥२८॥\ip

%3.7.4.6
वि॒ज॒यभा॑ग॒ꣳ॒ समि॑न्धताम्।
अग्ने॑ दी॒दा॑य मे सभ्य।
विजि॑त्यै श॒रदः॑ श॒तम्।
अन्न॑मावस॒थीयम्᳚।
अ॒भि ह॑राणि श॒रदः॑ श॒तम्।
आ॒व॒स॒थे श्रियं॒ मन्त्रम्᳚।
अहि॑र्बु॒ध्नियो॒ नि य॑च्छतु।
इ॒द\-म॒ह\-म॒ग्नि\-ज्ये᳚ष्ठेभ्यः।
वसु॑भ्यो य॒ज्ञं प्रब्र॑वीमि।
इ॒द\-म॒ह\-मिन्द्र॑\-ज्येष्ठेभ्यः॥२९॥\ip

%3.7.4.7
रु॒द्रेभ्यो॑ य॒ज्ञं प्र ब्र॑वीमि।
इ॒दम॒हं वरु॑णज्येष्ठेभ्यः।
आ॒दि॒त्येभ्यो॑ य॒ज्ञं प्र ब्र॑वीमि।
पय॑स्वती॒रोष॑धयः।
पय॑स्वद्वी॒रुधां॒ पयः॑।
अ॒पां पय॑सो॒ यत्पयः॑।
तेन॒ मामि॑न्द्र॒ सꣳ सृ॑ज।
अग्ने᳚ व्रतपते व्र॒तं च॑रिष्यामि।
तच्छ॑केयं॒ तन्मे॑ राध्यताम्।
वायो᳚ व्रतपत॒ आदि॑त्य व्रतपते॥३०॥\ip

%3.7.4.8
व्र॒तानां᳚ व्रतपते व्र॒तं च॑रिष्यामि।
तच्छ॑केयं॒ तन्मे॑ राध्यताम्।
इ॒मां प्राची॒मुदी॑चीम्।
इष॒मूर्ज॑म॒भि सꣴस्कृ॑ताम्।
ब॒हु॒प॒र्णामशु॑ष्काग्राम्।
हरा॑मि पशु॒पाम॒हम्।
यत्कृष्णो॑ रू॒पं कृ॒त्वा।
प्रावि॑श॒स्त्वं वन॒स्पतीन्॑।
तत॒स्त्वामे॑कविꣳशति॒धा।
सम्भ॑रामि सुस॒म्भृता᳚॥३१॥\ip

%3.7.4.9
त्रीन्प॑रि॒धीꣴस्ति॒स्रः स॒मिधः॑।
य॒ज्ञायु॑रनु\-सञ्च॒रान्।
उ॒प॒वे॒षं मेक्ष॑णं॒ धृष्टिम्᳚।
सं भ॑रामि सुस॒म्भृता᳚।
या जा॒ता ओष॑धयः।
दे॒वेभ्य॑स्त्रियु॒गं पु॒रा।
तासां॒ पर्व॑ राध्यासम्।
प॒रि॒स्त॒रमा॒हरन्॑।
अ॒पां मेध्यं॑ य॒ज्ञियम्᳚।
सदे॑वꣳ शि॒वम॑स्तु मे॥३२॥\ip

%3.7.4.10
आ॒च्छे॒त्ता वो॒ मा रि॑षम्।
जीवा॑नि श॒रदः॑ श॒तम्।
अप॑रिमितानां॒ परि॑मिताः।
सन्न॑ह्ये सुकृ॒ताय॒ कम्।
एनो॒ मा निगा᳚ङ्कत॒मच्च॒नाहम्।
पुन॑रु॒त्थाय॑ बहु॒ला भ॑वन्तु।
स॒कृ॒दा॒च्छि॒न्नं ब॒र्॒हिरूर्णा॑मृदु।
स्यो॒नं पि॒तृभ्य॑स्त्वा भराम्य॒हम्।
अ॒स्मिन्थ्सी॑दन्तु मे पि॒तरः॑ सो॒म्याः।
पि॒ता॒म॒हाः प्रपि॑तामहाश्चानु॒गैः स॒ह॥३३॥\ip

%3.7.4.11
त्रि॒वृत्प॑ला॒शे द॒र्भः।
इया᳚न्प्रादे॒शस॑म्मितः।
य॒ज्ञे प॒वित्रं॒ पोतृ॑तमम्।
पयो॑ ह॒व्यं क॑रोतु मे।
इ॒मौ प्रा॑णापा॒नौ।
य॒ज्ञस्याङ्गा॑नि सर्व॒शः।
आ॒प्या॒यय॑न्तौ॒ सञ्च॑रताम्।
प॒वित्रे॑ हव्य॒शोध॑ने।
प॒वित्रे᳚ स्थो वैष्ण॒वी।
वा॒युर्वां॒ मन॑सा पुनातु॥३४॥\ip

%3.7.4.12
अ॒यं प्रा॒णश्चा॑पा॒नश्च॑।
यज॑मान॒मपि॑ गच्छताम्।
य॒ज्ञे ह्यभू॑तां॒ पोता॑रौ।
प॒वित्रे॑ हव्य॒शोध॑ने।
त्वया॒ वेदिं॑ विविदुः पृथि॒वीम्।
त्वया॑ य॒ज्ञो जा॑यते विश्व॒दानिः॑।
अच्छि॑द्रं य॒ज्ञमन्वे॑षि वि॒द्वान्।
त्वया॒ होता॒ सं त॑नोत्यर्धमा॒सान्।
त्र॒य॒स्त्रि॒ꣳ॒शो॑ऽसि॒ तन्तू॑नाम्।
प॒वित्रे॑ण स॒हाग॑हि॥३५॥\ip

%3.7.4.13
शि॒वेयꣳ रज्जु॑रभि॒धानी᳚।
अ॒घ्नि॒यामुप॑ सेवताम्।
अप्र॑स्रꣳसाय य॒ज्ञस्य॑।
उ॒खे उप॑दधाम्य॒हम्।
प॒शुभिः॒ सन्नी॑तं बिभृताम्।
इन्द्रा॑य शृ॒तं दधि॑।
उ॒प॒वे॒षो॑ऽसि य॒ज्ञाय॑।
त्वां प॑रिवे॒षम॑धारयन्।
इन्द्रा॑य ह॒विः कृ॒ण्वन्तः॑।
शि॒वः श॒ग्मो भ॑वासि नः॥३६॥\ip

%3.7.4.14
अमृ॑न्मयन्देवपा॒त्रम्।
य॒ज्ञस्याऽऽयु॑षि॒ प्र यु॑ज्यताम्।
ति॒रः॒ प॒वि॒त्रमति॑नीताः।
आपो॑ धारय॒ माति॑गुः।
दे॒वेन॑ सवि॒त्रोत्पू॑ताः।
वसोः॒ सूर्य॑स्य र॒श्मिभिः॑।
गां दो॑हपवि॒त्रे रज्जुम्᳚।
सर्वा॒ पात्रा॑णि शुन्धत।
ए॒ता आ च॑रन्ति॒ मधु॑म॒द्दुहा॑नाः।
प्र॒जाव॑तीर्य॒शसो॑ वि॒श्वरू॑पाः॥३७॥\ip

%3.7.4.15
ब॒ह्वीर्भव॑न्ती॒रुप॒जाय॑मानाः।
इ॒ह व॒ इन्द्रो॑ रमयतु गावः।
पू॒षा स्थ॑।
अ॒य॒क्ष्मा वः॑ प्र॒जया॒ सꣳ सृ॑जामि।
रा॒यस्पोषे॑ण बहु॒लाभव॑न्तीः।
ऊर्जं॒ पयः॒ पिन्व॑माना घृ॒तं च॑।
जी॒वो जीव॑न्ती॒रुप॑वः सदेयम्।
द्यौश्चे॒मं य॒ज्ञं पृ॑थि॒वी च॒ सन्दु॑हाताम्।
धा॒ता सोमे॑न स॒ह वाते॑न वा॒युः।
यज॑मानाय॒ द्रवि॑णं दधातु॥३८॥\ip

%3.7.4.16
उथ्सं॑ दुहन्ति क॒लशं॒ चतु॑र्बिलम्।
इडां᳚ दे॒वीं मधु॑मतीꣳ सुव॒र्विदम्᳚।
तदि॑न्द्रा॒ग्नी जि॑न्वतꣳ सू॒नृता॑वत्।
तद्यज॑मान\-ममृत॒त्वे द॑धातु।
काम॑धुक्षः॒ प्र णो᳚ ब्रूहि।
इन्द्रा॑य ह॒विरि॑न्द्रि॒यम्।
अ॒मूं यस्यां᳚ दे॒वाना᳚म्।
म॒नु॒ष्या॑णां॒ पयो॑ हि॒तम्।
ब॒हु दु॒ग्धीन्द्रा॑य दे॒वेभ्यः॑।
ह॒व्यमा प्या॑यतां॒ पुनः॑॥३९॥\ip

%3.7.4.17
व॒थ्सेभ्यो॑ मनु॒ष्ये᳚भ्यः।
पु॒न॒र्दो॒हाय॑ कल्पताम्।
य॒ज्ञस्य॒ सन्त॑\-ति\-रसि।
य॒ज्ञस्य॑ त्वा॒ सन्त॑\-ति॒\-मनु॒ सं त॑नोमि।
अद॑स्तमसि॒ विष्ण॑वे त्वा।
य॒ज्ञायापि॑ दधाम्य॒हम्।
अ॒द्भिररि॑क्तेन॒ पात्रे॑ण।
याः पू॒ताः प॑रि॒शेर॑ते।
अ॒यं पयः॒ सोमं॑ कृ॒त्वा।
स्वां योनि॒मपि॑ गच्छतु॥४०॥\ip

%3.7.4.18
प॒र्ण॒व॒ल्कः प॒वित्रम्᳚।
सौ॒म्यः सोमा॒द्धि निर्मि॑तः।
इ॒मौ प॒र्णं च॑ द॒र्भं च॑।
दे॒वानाꣳ॑ हव्य॒शोध॑नौ।
प्रा॒त॒र्वे॒षाय॑ गोपाय।
विष्णो॑ ह॒व्यꣳ हि रक्ष॑सि।
उ॒भाव॒ग्नी उ॑पस्तृण॒ते।
दे॒वता॒ उप॑वसन्तु मे।
अ॒हं ग्रा॒म्यानुप॑ वसामि।
मह्यं॒ गोप॑तये प॒शून्॥४१॥\ip\anuvakamend[आभृ॑त इ॒मं गृ॑ह्णामि॒ पूर्व॒स्ताः पूर्वः॒ परि॑गृह्णामि सभापा॒ला इन्द्र॑ज्येष्ठेभ्य॒ आदि॑त्य व्रतपते सुस॒म्भृता॑ मे स॒ह पु॑नातु गहि नो वि॒श्वरू॑पा दधातु॒ पुन॑र्गच्छतु प॒शून् (याः पु॒रस्ता॑दि॒मामूर्ज॑मि॒ह प्र॒जा इ॒ह प॒शवो॒ऽयं पि॑तृ॒णाम॒ग्निः।)]

%3.7.5.1
देवा॑ दे॒वेषु॒ परा᳚क्रमध्वम्।
प्रथ॑मा द्वि॒तीये॑षु।
द्विती॑यास्तृ॒तीये॑षु।
त्रिरे॑कादशा इ॒ह मा॑ऽवत।
इ॒दꣳ श॑केयं॒ यदि॒दं क॒रोमि॑।
आ॒त्मा क॑रोत्वा॒त्मने᳚।
इ॒दं क॑रिष्ये भेष॒जम्।
इ॒दं मे॑ विश्वभेषजा।
अश्वि॑ना॒ प्राव॑तं यु॒वम्।
इ॒दम॒हꣳ सेना॑या अ॒भीत्व॑र्यै॥४२॥\ip

%3.7.5.2
मुख॒मपो॑हामि।
सूर्य॑ ज्योति॒र्वि भा॑हि।
म॒ह॒त इ॑न्द्रि॒याय॑।
आ प्या॑यतां घृ॒तयो॑निः।
अ॒ग्निर्‌\mbox{}ह॒व्याऽनु॑ मन्यताम्।
खम॑ङ्क्ष्व॒ त्वच॑मङ्क्ष्व।
सु॒रू॒पं त्वा॑ वसु॒विदम्᳚।
प॒शू॒नां तेज॑सा।
अ॒ग्नये॒ जुष्ट॑म॒भि घा॑रयामि।
स्यो॒नं ते॒ सद॑नं करोमि॥४३॥\ip

%3.7.5.3
घृ॒तस्य॒ धार॑या सु॒शेवं॑ कल्पयामि।
तस्मि᳚न्थ्सीदा॒मृते॒ प्रति॑ तिष्ठ।
व्री॒ही॒णां मे॑ध सुमन॒स्यमा॑नः।
आ॒र्द्रः प्र॑थस्नु॒र्भुव॑नस्य गो॒पाः।
शृ॒त उथ्स्ना॑ति जनि॒ता म॑ती॒नाम्।
यस्त॑ आ॒त्मा प॒शुषु॒ प्रवि॑ष्टः।
दे॒वानां᳚ वि॒ष्ठामनु॒ यो वि॑त॒स्थे।
आ॒त्म॒न्वान्थ्सो॑म घृ॒तवा॒न्॒ हि भू॒त्वा।
दे॒वान्ग॑च्छ॒ सुव॑र्विन्द॒ यज॑मानाय॒ मह्यम्᳚।
इरा॒ भूतिः॑ पृथि॒व्यै रसो॒ मोत्क्र॑मीत्॥४४॥\ip

%3.7.5.4
देवाः᳚ पितरः॒ पित॑रो देवाः।
यो॑ऽहम॑स्मि॒ स सन् य॑जे।
यस्या᳚स्मि॒ न तम॒न्तरे॑मि।
स्वं म॑ इ॒ष्टꣴ स्वं द॒त्तम्।
स्वं पू॒र्तꣴ स्वꣴ श्रा॒न्तम्।
स्वꣳ हु॒तम्।
तस्य॑ मे॒\-ऽग्निरु॑पद्र॒ष्टा।
वा॒युरु॑पश्रो॒ता।
आ॒दि॒त्यो॑\-ऽनुख्या॒ता।
द्यौः पि॒ता॥४५॥\ip

%3.7.5.5
पृ॒थि॒वी मा॒ता।
प्र॒जा\-प॑ति॒र्बन्धुः॑।
य ए॒वास्मि॒ स सन् य॑जे।
मा भेर्मा संवि॑क्था॒ मा त्वा॑ हिꣳसिषम्।
मा ते॒ तेजोऽप॑ क्रमीत्।
भ॒र॒तमुद्ध॑रे॒मनु॑षिञ्च।
अ॒व॒दाना॑नि ते प्र॒त्यव॑दास्यामि।
नम॑स्ते अस्तु॒ मा मा॑ हिꣳसीः।
यद॑व॒दाना॑नि तेऽव॒द्यन्।
विलो॒माका॑र्‌\mbox{}षमा॒त्मनः॑॥४६॥\ip

%3.7.5.6
आज्ये॑न॒ प्रत्य॑नज्म्येनत्।
तत्त॒ आ प्या॑यतां॒ पुनः॑।
अज्या॑यो यवमा॒त्रात्।
आ॒व्या॒धात्कृ॑त्यतामि॒दम्।
मा रू॑रुपाम य॒ज्ञस्य॑।
शु॒द्धꣴ स्वि॑ष्टमि॒दꣳ ह॒विः।
मनु॑ना दृ॒ष्टां घृ॒तप॑दीम्।
मि॒त्रावरु॑णसमीरिताम्।
द॒क्षि॒णा॒र्धादस॑म्भिन्दन्।
अव॑द्याम्ये\-क॒तोमु॑खाम्॥४७॥\ip

%3.7.5.7
इडे॑ भा॒गं जु॑षस्व नः।
जिन्व॒ गा जिन्वार्व॑तः।
तस्या᳚स्ते भक्षि॒वाणः॑ स्याम।
स॒र्वात्मा॑नः स॒र्वग॑णाः।
ब्रध्न॒ पिन्व॑स्व।
दद॑तो मे॒ मा क्षा॑यि।
कु॒र्व॒तो मे॒ मोप॑दसत्।
दि॒शां कॢप्ति॑रसि।
दिशो॑ मे कल्पन्ताम्।
कल्प॑न्तां मे॒ दिशः॑॥४८॥\ip

%3.7.5.8
दैवी᳚श्च॒ मानु॑षीश्च।
अ॒हो॒रा॒त्रे मे॑ कल्पेताम्।
अ॒र्ध॒मा॒सा मे॑ कल्पन्ताम्।
मासा॑ मे कल्पन्ताम्।
ऋ॒तवो॑ मे कल्पन्ताम्।
सं॒व॒थ्स॒रो मे॑ कल्पताम्।
कॢप्ति॑रसि॒ कल्प॑तां मे।
आशा॑नां त्वा\-ऽऽशापा॒लेभ्यः॑।
च॒तुर्भ्यो॑ अ॒मृते᳚भ्यः।
इ॒दं भू॒तस्याध्य॑क्षेभ्यः॥४९॥\ip

%3.7.5.9
वि॒धेम॑ ह॒विषा॑ व॒यम्।
भज॑तां भा॒गी भा॒गम्।
मा भा॒गो\-ऽभ॑क्त।
निर॑भा॒गं भ॑जामः।
अ॒पस्पि॑न्व।
ओष॑धीर्जिन्व।
द्वि॒पात्पा॑हि।
चतु॑ष्पादव।
दि॒वो वृष्टि॒मेर॑य।
ब्रा॒ह्म॒णाना॑मि॒दꣳ ह॒विः॥५०॥\ip

%3.7.5.10
सो॒म्यानाꣳ॑ सोमपी॒थिना᳚म्।
निर्भ॒क्तो ब्रा᳚ह्मणः।
नेहा ब्रा᳚ह्मणस्यास्ति।
सम॑ङ्क्तां ब॒र्॒हिर्‌\mbox{}ह॒विषा॑ घृ॒तेन॑।
समा॑दि॒त्यैर्वसु॑भिः॒ सं म॒रुद्भिः॑।
समिन्द्रे॑ण॒ विश्वे॑भिर्दे॒वेभि॑रङ्क्ताम्।
दि॒व्यं नभो॑ गच्छतु॒ यथ्स्वाहा᳚।
इ॒न्द्रा॒णीवा॑विध॒वा भू॑यासम्।
अदि॑तिरिव सुपु॒त्रा।
अ॒स्थू॒रि त्वा॑ गार्‌\mbox{}हपत्य॥५१॥\ip

%3.7.5.11
उप॒निष॑दे सुप्रजा॒स्त्वाय॑।
सं पत्नी॒ पत्या॑ सुकृ॒तेन॑ गच्छताम्।
य॒ज्ञस्य॑ यु॒क्तौ धुर्या॑वभूताम्।
स॒ञ्जा॒ना॒नौ विज॑हता॒मरा॑तीः।
दि॒वि ज्योति॑र॒जर॒मा र॑भेताम्।
दश॑ते त॒नुवो॑ यज्ञ य॒ज्ञियाः᳚।
ताः प्री॑णातु॒ यज॑मानो घृ॒तेन॑।
ना॒रि॒ष्ठयोः᳚ प्र॒शिष॒मीड॑मानः।
दे॒वानां॒ दैव्येऽपि॒ यज॑मानो॒\-ऽमृतो॑\-ऽभूत्।
यं वां᳚ दे॒वा अ॑कल्पयन्॥५२॥\ip

%3.7.5.12
ऊ॒र्जो भा॒गꣳ श॑तक्रतू।
ए॒तद्वां॒ तेन॑ प्रीणानि।
तेन॑ तृप्यतमꣳहहौ।
अ॒हं दे॒वानाꣳ॑ सु॒कृता॑मस्मि लो॒के।
ममे॒दमि॒ष्टं न मिथु॑र्भवाति।
अ॒हं ना॑रि॒ष्ठावनु॑ यजामि वि॒द्वान्।
यदा᳚भ्या॒मिन्द्रो॒ अद॑धाद्भाग॒धेयम्᳚।
अदा॑रसृद्भवत देवसोम।
अ॒स्मिन् य॒ज्ञे म॑रुतो मृडता नः।
मा नो॑ विदद॒भिभा॒मो अश॑स्तिः॥५३॥\ip

%3.7.5.13
मा नो॑ विदद्वृ॒जना॒ द्वेष्या॒ या।
ऋ॒ष॒भं वा॒जिनं॑ व॒यम्।
पू॒र्णमा॑सं यजामहे।
स नो॑ दोहताꣳ सु॒वीर्यम्᳚।
रा॒यस्पोषꣳ॑ सह॒स्रिणम्᳚।
प्रा॒णाय॑ सु॒राध॑से।
पू॒र्णमा॑साय॒ स्वाहा᳚।
अ॒मा॒वा॒स्या॑ सु॒भगा॑ सु॒शेवा᳚।
धे॒नुरि॑व॒ भूय॑ आ॒प्याय॑माना।
सा नो॑ दोहताꣳ सु॒वीर्यम्᳚।
रा॒यस्पोषꣳ॑ सह॒स्रिणम्᳚।
अ॒पा॒नाय॑ सु॒राध॑से।
अ॒मा॒वा॒स्या॑यै॒ स्वाहा᳚।
अ॒भि स्तृ॑णीहि॒ परि॑ धेहि॒ वेदिम्᳚।
जा॒मिं मा हिꣳ॑सीरमु॒या शया॑ना।
हो॒तृ॒षद॑ना॒ हरि॑ताः सु॒वर्णाः᳚।
नि॒ष्का इ॒मे यज॑मानस्य ब्र॒ध्ने॥५४॥\ip\anuvakamend[अ॒भीत्व॑र्यै करोमि क्रमीत्पि॒ता\-ऽऽत्मन॑ एक॒तो मु॑खां मे॒ दिशो\-ऽध्य॑क्षेभ्यो ह॒विर्गा॑र्‌\mbox{}हपत्या कल्पय॒न्नश॑स्तिः॒ सा नो॑ दोहताꣳ सु॒वीर्यꣳ॑ स॒प्त च॑]

%3.7.6.1
परि॑स्तृणीत॒ परि॑धत्ता॒ग्निम्।
परि॑हितो॒\-ऽग्निर्यज॑मानं भुनक्तु।
अ॒पाꣳ रस॒ ओष॑धीनाꣳ सु॒वर्णः॑।
नि॒ष्का इ॒मे यज॑मानस्य सन्तु काम॒दुघाः᳚।
अ॒मुत्रा॒मुष्मिँ॑ल्लो॒के।
भूप॑ते॒ भुव॑नपते।
म॒ह॒तो भू॒तस्य॑ पते।
ब्र॒ह्माणं॑ त्वा वृणीमहे।
अ॒हं भूप॑तिर॒हं भुव॑नपतिः।
अ॒हं म॑ह॒तो भू॒तस्य॒ पतिः॑॥५५॥\ip

%3.7.6.2
दे॒वेन॑ सवि॒त्रा प्रसू॑त॒ आर्त्वि॑ज्यं करिष्यामि।
देव॑ सवितरे॒तं त्वा॑ वृणते।
बृह॒स्पतिं॒ दैव्यं॑ ब्र॒ह्माणम्᳚।
तद॒हं मन॑से॒ प्र ब्र॑वीमि।
मनो॑ गायत्रि॒यै।
गा॒य॒त्री त्रि॒ष्टुभे᳚।
त्रि॒ष्टुब्जग॑त्यै।
जग॑त्यनु॒ष्टुभे᳚।
अ॒नु॒ष्टुक्प॒ङ्क्त्यै।
प॒ङ्क्तिः प्र॒जा\-प॑तये॥५६॥\ip

%3.7.6.3
प्र॒जा\-प॑ति॒र्विश्वे᳚भ्यो दे॒वेभ्यः॑।
विश्वे॑ दे॒वा बृह॒स्पत॑ये।
बृह॒स्पति॒र्ब्रह्म॑णे।
ब्रह्म॒ भूर्भुवः॒ सुवः॑।
बृह॒स्पति॑र्दे॒वानां᳚ ब्र॒ह्मा।
अ॒हं म॑नु॒ष्या॑णाम्।
बृह॑स्पते य॒ज्ञं गो॑पाय।
इ॒दं तस्मै॑ ह॒र्म्यं क॑रोमि।
यो वो॑ देवा॒श्चर॑ति ब्रह्म॒चर्यम्᳚।
मे॒धा॒वी दि॒क्षु मन॑सा तप॒स्वी॥५७॥\ip

%3.7.6.4
अ॒न्तर्दू॒तश्च॑रति॒ मानु॑षीषु।
चतुः॑ शिखण्डा युव॒तिः सु॒पेशाः᳚।
घृ॒तप्र॑तीका॒ भुव॑नस्य॒ मध्ये᳚।
म॒र्मृ॒ज्यमा॑ना मह॒ते सौभ॑गाय।
मह्यं॑ धुक्ष्व॒ यज॑मानाय॒ कामान्॑।
भूमि॑र्भू॒त्वा म॑हि॒मानं॑ पुपोष।
ततो॑ दे॒वी व॑र्धयते॒ पयाꣳ॑सि।
य॒ज्ञिया॑ य॒ज्ञं वि च॒ यन्ति॒ शं च॑।
ओष॑धी॒राप॑ इ॒ह शक्व॑रीश्च।
यो मा॑ हृ॒दा मन॑सा॒ यश्च॑ वा॒चा॥५८॥\ip

%3.7.6.5
यो ब्रह्म॑णा॒ कर्म॑णा॒ द्वेष्टि॑ देवाः।
यः श्रु॒तेन॒ हृद॑येनेष्ण॒ता च॑।
तस्ये᳚न्द्र॒ वज्रे॑ण॒ शिर॑श्छिनद्मि।
ऊर्णा॑मृदु॒ प्रथ॑मानꣴ स्यो॒नम्।
दे॒वेभ्यो॒ जुष्ट॒ꣳ॒ सद॑नाय ब॒र्॒हिः।
सु॒व॒र्गे लो॒के यज॑मान॒ꣳ॒ हि धे॒हि।
मां नाक॑स्य पृ॒ष्ठे प॑र॒मे व्यो॑मन्।
चतुः॑ शिखण्डा युव॒तिः सु॒पेशाः᳚।
घृ॒तप्र॑तीका व॒युना॑नि वस्ते।
साऽऽस्ती॒र्यमा॑णा मह॒ते सौभ॑गाय॥५९॥\ip

%3.7.6.6
सा मे॑ धुक्ष्व॒ यज॑मानाय॒ कामान्॑।
शि॒वा च॑ मे श॒ग्मा चै॑धि।
स्यो॒ना च॑ मे सु॒षदा॑ चैधि।
ऊर्ज॑स्वती च मे॒ पय॑स्वती चैधि।
इष॒मूर्जं॑ मे पिन्वस्व।
ब्रह्म॒ तेजो॑ मे पिन्वस्व।
क्ष॒त्रमोजो॑ मे पिन्वस्व।
विशं॒ पुष्टिं॑ मे पिन्वस्व।
आयु॑र॒न्नाद्यं॑ मे पिन्वस्व।
प्र॒जां प॒शून्मे॑ पिन्वस्व॥६०॥\ip

%3.7.6.7
अ॒स्मिन् य॒ज्ञ उप॒ भूय॒ इन्नु मे᳚।
अवि॑क्षोभाय परि॒धीं द॑धामि।
ध॒र्ता ध॒रुणो॒ धरी॑यान्।
अ॒ग्निर्द्वेषाꣳ॑सि॒ निरि॒तो नु॑दातै।
विच्छि॑नद्मि॒ विधृ॑तीभ्याꣳ स॒पत्नान्॑।
जा॒तान्भ्रातृ॑व्या॒न्॒ ये च॑ जनि॒ष्यमा॑णाः।
वि॒शो य॒न्त्राभ्यां॒ विध॑माम्येनान्।
अ॒हꣴ स्वाना॑मुत्त॒मो॑\-ऽसानि देवाः।
वि॒शो य॒न्त्रे नु॒दमा॑ने॒ अरा॑तिम्।
विश्वं॑ पा॒प्मान॒मम॑तिं दुर्मरा॒युम्॥६१॥\ip

%3.7.6.8
सीद॑न्ती दे॒वी सु॑कृ॒तस्य॑ लो॒के।
धृती᳚ स्थो॒ विधृ॑ती॒ स्वधृ॑ती।
प्रा॒णान्मयि॑ धारयतम्।
प्र॒जां मयि॑ धारयतम्।
प॒शून्मयि॑ धारयतम्।
अ॒यं प्र॑स्त॒र उ॒भय॑स्य ध॒र्ता।
ध॒र्ता प्र॑या॒जाना॑मु॒तानू॑या॒जाना᳚म्।
स दा॑धार स॒मिधो॑ वि॒श्वरू॑पाः।
तस्मि॒न्थ्स्रुचो॒ अध्या सा॑दयामि।
आ रो॑ह प॒थो जु॑हु देव॒यानान्॑॥६२॥\ip

%3.7.6.9
यत्रर्‌\mbox{}ष॑यः प्रथम॒जा ये पु॑रा॒णाः।
हिर॑ण्यपक्षा\-ऽजि॒रा सम्भृ॑ताङ्गा।
वहा॑सि मा सु॒कृतां॒ यत्र॑ लो॒काः।
अवा॒हं बा॑ध उप॒भृता॑ स॒पत्नान्॑।
जा॒तान्भ्रातृ॑व्या॒न्॒ ये च॑ जनि॒ष्यमा॑णाः।
दोहै॑ य॒ज्ञꣳ सु॒दुघा॑मिव धे॒नुम्।
अ॒हमुत्त॑रो भूयासम्।
अध॑रे॒ मथ्स॒पत्नाः᳚।
यो मा॑ वा॒चा मन॑सा दुर्मरा॒युः।
हृ॒दा\-ऽरा॑ती॒याद॑भि॒दास॑दग्ने॥६३॥\ip

%3.7.6.10
इ॒दम॑स्य चि॒त्तमध॑रं ध्रु॒वायाः᳚।
अ॒हमुत्त॑रो भूयासम्।
अध॑रे॒ मथ्स॒पत्नाः᳚।
ऋ॒ष॒भो॑ऽसि शाक्व॒रः।
घृ॒ताची॑नाꣳ सू॒नुः।
प्रि॒येण॒ नाम्ना᳚ प्रि॒ये सद॑सि सीद।
स्यो॒नो मे॑ सीद सु॒षदः॑ पृथि॒व्याम्।
प्रथ॑यि प्र॒जया॑ प॒शुभिः॑ सुव॒र्गे लो॒के।
दि॒वि सी॑द पृथि॒व्याम॒न्तरि॑क्षे।
अ॒हमुत्त॑रो भूयासम्॥६४॥\ip\phantom{त्त॑}%for spacing!

%3.7.6.11
अध॑रे॒ मथ्स॒पत्नाः᳚।
इ॒यꣴ स्था॒ली घृ॒तस्य॑ पू॒र्णा।
अच्छि॑न्नपयाः श॒तधा॑र॒ उथ्सः॑।
मा॒रु॒तेन॒ शर्म॑णा॒ दैव्ये॑न।
य॒ज्ञो॑ऽसि स॒र्वतः॑ श्रि॒तः।
स॒र्वतो॒ मां भू॒तं भ॑वि॒ष्यच्छ्र॑यताम्।
श॒तं मे॑ सन्त्वा॒शिषः॑।
स॒हस्रं॑ मे सन्तु सू॒नृताः᳚।
इरा॑वतीः पशु॒मतीः᳚।
प्र॒जा\-प॑तिरसि स॒र्वतः॑ श्रि॒तः॥६५॥\ip

%3.7.6.12
स॒र्वतो॒ मां भू॒तं भ॑वि॒ष्यच्छ्र॑यताम्।
श॒तं मे॑ सन्त्वा॒शिषः॑।
स॒हस्रं॑ मे सन्तु सू॒नृताः᳚।
इरा॑वतीः पशु॒मतीः᳚।
इ॒दमि॑न्द्रि॒यम॒मृतं॑ वी॒र्यम्᳚।
अ॒नेनेन्द्रा॑य प॒शवो॑\-ऽचिकिथ्सन्।
तेन॑ देवा अव॒तोप॒ माम्।
इ॒हेष॒मूर्जं॒ यशः॒ सह॒ ओजः॑ सनेयम्।
शृ॒तं मयि॑ श्रयताम्।
यत्पृ॑थि॒वीमच॑र॒त्तत्प्रवि॑ष्टम्॥६६॥\ip

%3.7.6.13
येनासि॑ञ्च॒द्बल॒मिन्द्रे᳚ प्र॒जा\-प॑तिः।
इ॒दं तच्छु॒क्रं मधु॑ वा॒जिनी॑वत्।
येनो॒परि॑ष्टा॒दधि॑नोन्महे॒\-न्द्रम्।
दधि॒ मां धि॑नोतु।
अ॒यं वे॒दः पृ॑थि॒वीमन्व॑विन्दत्।
गुहा॑ स॒तीं गह॑ने॒ गह्व॑रेषु।
स वि॑न्दतु॒ यज॑मानाय लो॒कम्।
अच्छि॑द्रं य॒ज्ञं भूरि॑कर्मा करोतु।
अ॒यं य॒ज्ञः सम॑सदद्ध॒विष्मान्॑।
ऋ॒चा साम्ना॒ यजु॑षा दे॒वता॑भिः॥६७॥\ip

%3.7.6.14
तेन॑ लो॒कान्थ्सूर्य॑वतो जयेम।
इन्द्र॑स्य स॒ख्यम॑मृत॒त्वम॑\-श्याम्।
यो नः॒ कनी॑य इ॒ह का॒मया॑तै।
अ॒स्मिन् य॒ज्ञे यज॑मानाय॒ मह्यम्᳚।
अप॒ तमि॑न्द्रा॒ग्नी भुव॑नान्नुदेताम्।
अ॒हं प्र॒जां वी॒रव॑तीं विदेय।
अग्ने॑ वाजजित्।
वाजं॑ त्वा सरि॒ष्यन्तम्᳚।
वाजं॑ जे॒ष्यन्तम्᳚।
वा॒जिनं॑ वाज॒जितम्᳚॥६८॥\ip

%3.7.6.15
वा॒ज॒जि॒त्यायै॒ सं मा᳚र्ज्मि।
अ॒ग्निम॑न्ना॒दम॒न्नाद्या॑य।
उप॑हूतो॒ द्यौः पि॒ता।
उप॒ मां द्यौः पि॒ता ह्व॑यताम्।
अ॒ग्निराग्नी᳚ध्रात्।
आयु॑षे॒ वर्च॑से।
जी॒वात्वै पुण्या॑य।
उप॑हूता पृथि॒वी मा॒ता।
उप॒ मां मा॒ता पृ॑थि॒वी ह्व॑यताम्।
अ॒ग्निराग्नी᳚ध्रात्॥६९॥\ip

%3.7.6.16
आयु॑षे॒ वर्च॑से।
जी॒वात्वै पुण्या॑य।
मनो॒ ज्योति॑र्जुषता॒मा\-ज्यम्᳚।
विच्छि॑न्नं य॒ज्ञꣳ समि॒मं द॑धातु।
बृह॒स्पति॑स्तनुतामि॒मं नः॑।
विश्वे॑ दे॒वा इ॒ह मा॑दयन्ताम्।
यन्ते॑ अग्न आवृ॒श्चामि॑।
अ॒हं वा᳚ क्षिपि॒तश्चरन्॑।
प्र॒जां च॒ तस्य॒ मूलं॑ च।
नी॒चैर्दे॑वा॒ नि वृ॑श्चत॥७०॥\ip

%3.7.6.17
अग्ने॒ यो नो॑\-ऽभि॒दास॑ति।
स॒मा॒नो यश्च॒ निष्ट्यः॑।
इ॒ध्मस्ये॑व प्र॒क्षाय॑तः।
मा तस्योच्छे॑षि॒ किञ्च॒न।
यो मां द्वेष्टि॑ जातवेदः।
यं चा॒ऽऽहं द्वेष्मि॒ यश्च॒ माम्।
सर्वा॒ꣴ॒स्तान॑ग्ने॒ सन्द॑ह।
याꣴश्चा॒हं द्वेष्मि॒ ये च॒ माम्।
अग्ने॑ वाजजित्।
वाजं॑ त्वा ससृ॒वाꣳसम्᳚॥७१॥\ip

%3.7.6.18
वाजं॑ जिगि॒वाꣳसम्᳚।
वा॒जिनं॑ वाज॒जितम्᳚।
वा॒ज॒जि॒त्यायै॒ सम्मा᳚र्ज्मि।
अ॒ग्निम॑न्ना॒दम॒न्नाद्या॑य।
वेदि॑र्ब॒र्॒हिः शृ॒तꣳ ह॒विः।
इ॒ध्मः प॑रि॒धयः॒ स्रुचः॑।
आज्यं॑ य॒ज्ञ ऋचो॒ यजुः॑।
या॒ज्या᳚श्च वषट्का॒राः।
सं मे॒ सन्न॑तयो नमन्ताम्।
इ॒ध्म॒स॒न्नह॑ने हु॒ते॥७२॥\ip

%3.7.6.19
दि॒वः खीलो\-ऽव॑ततः।
पृ॒थि॒व्या अध्युत्थि॑तः।
तेना॑ स॒हस्र॑काण्डेन।
द्वि॒षन्तꣳ॑ शोचयामसि।
द्वि॒षन्मे॑ ब॒हु शो॑चतु।
ओष॑धे॒ मो अ॒हꣳ शु॑चम्।
यज्ञ॒ नम॑स्ते यज्ञ।
नमो॒ नम॑श्च ते यज्ञ।
शि॒वेन॑ मे॒ सन्ति॑ष्ठस्व।
स्यो॒नेन॑ मे॒ सन्ति॑ष्ठस्व॥७३॥\ip

%3.7.6.20
सु॒भू॒तेन॑ मे॒ सन्ति॑ष्ठस्व।
ब्र॒ह्म॒व॒र्च॒सेन॑ मे॒ सन्ति॑ष्ठस्व।
य॒ज्ञस्यर्द्धि॒मनु॒ सन्ति॑ष्ठस्व।
उप॑ ते यज्ञ॒ नमः॑।
उप॑ ते॒ नमः॑।
उप॑ ते॒ नमः॑।
त्रिष्फ॒लीक्रि॒यमा॑णानाम्।
यो न्य॒ङ्गो अ॑व॒शिष्य॑ते।
रक्ष॑सां भाग॒धेयम्᳚।
आप॒स्तत्प्र व॑हतादि॒तः॥७४॥\ip

%3.7.6.21
उ॒लूख॑ले॒ मुस॑ले॒ यच्च॒ शूर्पे᳚।
आ॒शि॒श्लेष॑ दृ॒षदि॒ यत्क॒पाले᳚।
अ॒व॒प्रुषो॑ वि॒प्रुषः॒ संय॑जामि।
विश्वे॑ दे॒वा ह॒विरि॒दं जु॑षन्ताम्।
य॒ज्ञे या वि॒प्रुषः॒ सन्ति॑ ब॒ह्वीः।
अ॒ग्नौ ताः सर्वाः॒ स्वि॑ष्टाः॒ सुहु॑ता जुहोमि।
उ॒द्यन्न॒द्यमि॑त्र महः।
स॒पत्ना᳚न्मे अनीनशः।
दिवै॑नान् वि॒द्युता॑ जहि।
नि॒म्रोच॒न्नध॑रान्कृधि॥७५॥\ip

%3.7.6.22
उ॒द्यन्न॒द्य वि नो॑ भज।
पि॒ता पु॒त्रेभ्यो॒ यथा᳚।
दी॒र्घा॒यु॒त्वस्य॑ हेशिषे।
तस्य॑ नो देहि सूर्य।

\dnsub{हृद्रोगघ्न-मन्त्राः}

उ॒द्यन्न॒द्य मि॑त्रमहः।
आ॒रोह॒न्नुत्त॑रां॒ दिवम्᳚।
हृ॒द्रो॒गं मम॑ सूर्य।
ह॒रि॒माणं॑ च नाशय।
शुके॑षु मे हरि॒माणम्᳚।
रो॒प॒णाका॑सु दध्मसि॥७६॥\ip

%3.7.6.23
अथो॑ हारिद्र॒वेषु॑ मे।
ह॒रि॒माणं॒ नि द॑ध्मसि।
उद॑गाद॒यमा॑दि॒त्यः।
विश्वे॑न॒ सह॑सा स॒ह।
द्वि॒षन्तं॒ मम॑ र॒न्धयन्॑।
मो अ॒हं द्वि॑ष॒तो र॑धम्।

{\small \closesub{}}

यो नः॒ शपा॒दश॑पतः।
यश्च॑ नः॒ शप॑तः॒ शपा᳚त्।
उ॒षाश्च॒ तस्मै॑ नि॒म्रुक्च॑।
सर्वं॑ पा॒पꣳ समू॑हताम्॥७७॥\ip

%3.7.6.24
यो नः॑ स॒पत्नो॒ यो रणः॑।
मर्तो॑\-ऽभि॒दास॑ति देवाः।
इ॒ध्मस्ये॑व प्र॒क्षाय॑तः।
मा तस्योच्छे॑षि॒ किञ्च॒न।
अव॑सृष्टः॒ परा॑पत।
श॒रो ब्रह्म॑सꣳशितः।
गच्छा॒\-ऽमित्रा॒न्प्र वि॑श।
मैषां॒ कञ्च॒नोच्छि॑षः॥७८॥\ip\anuvakamend[पतिः॑ प्र॒जा\-प॑तये तप॒स्वी वा॒चा सौभ॑गाय प॒शून्मे॑ पिन्वस्व दुर्मरा॒युं दे॑व॒याना॑नग्ने॒\-ऽन्तरि॑क्षे॒\-ऽहमुत्त॑रो भूयासं प्र॒जा\-प॑तिरसि स॒र्वतः॑ श्रि॒तः प्रवि॑ष्टं दे॒वता॑भिर्वाज॒जितं॑ पृथि॒वी ह्व॑यताम॒ग्निराग्नी᳚ध्राद्वृश्चत ससृ॒वाꣳसꣳ॑ हु॒ते स्यो॒नेन॑ मे॒ सन्ति॑ष्ठस्वे॒तः कृ॑धि दध्मस्यूहताम॒ष्टौ च॑]

%3.7.7.1
सक्षे॒दं प॑श्य।
विध॑र्तरि॒दं प॑श्य।
नाके॒दं प॑श्य।
र॒मतिः॒ पनि॑ष्ठा।
ऋ॒तं वर्‌\mbox{}षि॑ष्ठम्।
अ॒मृता॒यान्या॒हुः।
सूर्यो॒ वरि॑ष्ठो अ॒क्षभि॒र्विभा॑ति।
अनु॒ द्यावा॑पृथि॒वी दे॒वपु॑त्रे।
दी॒क्षाऽसि॒ तप॑सो॒ योनिः॑।
तपो॑ऽसि॒ ब्रह्म॑णो॒ योनिः॑॥७९॥\ip

%3.7.7.2
ब्रह्मा॑सि क्ष॒त्रस्य॒ योनिः॑।
क्ष॒त्रम॑स्यृ॒तस्य॒ योनिः॑।
ऋ॒तम॑सि॒ भूरा र॑भे।
श्र॒द्धां मन॑सा।
दी॒क्षां तप॑सा।
विश्व॑स्य॒ भुव॑न॒स्याधि॑पत्नीम्।
सर्वे॒ कामा॒ यज॑मानस्य सन्तु।
वातं॑ प्रा॒णं मन॑सा॒\-ऽन्वा र॑भामहे।
प्र॒जा\-प॑तिं॒ यो भुव॑नस्य गो॒पाः।
स नो॑ मृ॒त्योस्त्रा॑यतां॒ पात्वꣳह॑सः॥८०॥\ip

%3.7.7.3
ज्योग्जी॒वा ज॒राम॑शीमहि।
इन्द्र॑ शाक्वर गाय॒त्रीं प्रप॑द्ये।
तान्ते॑ युनज्मि।
इन्द्र॑ शाक्वर त्रि॒ष्टुभं॒ प्रप॑द्ये।
तान्ते॑ युनज्मि।
इन्द्र॑ शाक्वर॒ जग॑तीं॒ प्रप॑द्ये।
तान्ते॑ युनज्मि।
इन्द्र॑ शाक्वरानु॒ष्टुभं॒ प्रप॑द्ये।
तान्ते॑ युनज्मि।
इन्द्र॑ शाक्वर प॒ङ्क्तिं प्रप॑द्ये॥८१॥\ip

%3.7.7.4
तान्ते॑ युनज्मि।
आऽहं दी॒क्षाम॑रुहमृ॒तस्य॒ पत्नी᳚म्।
गा॒य॒त्रेण॒ छन्द॑सा॒ ब्रह्म॑णा च।
ऋ॒तꣳ स॒त्ये॑\-ऽधायि।
स॒त्यमृ॒ते॑\-ऽधायि।
ऋ॒तं च॑ मे स॒त्यं चा॑भूताम्।
ज्योति॑रभूव॒ꣳ॒ सुव॑रगमम्।
सु॒व॒र्गं लो॒कं नाक॑स्य पृ॒ष्ठम्।
ब्र॒ध्नस्य॑ वि॒ष्टप॑मगमम्।
पृ॒थि॒वी दी॒क्षा॥८२॥\ip

%3.7.7.5
तया॒ऽग्निर्दी॒क्षया॑ दीक्षि॒तः।
यया॒ऽग्निर्दी॒क्षया॑ दीक्षि॒तः।
तया᳚ त्वा दी॒क्षया॑ दीक्षयामि।
अ॒न्तरि॑क्षं दी॒क्षा।
तया॑ वा॒युर्दी॒क्षया॑ दीक्षि॒तः।
यया॑ वा॒युर्दी॒क्षया॑ दीक्षि॒तः।
तया᳚ त्वा दी॒क्षया॑ दीक्षयामि।
द्यौर्दी॒क्षा।
तया॑ऽऽदि॒त्यो दी॒क्षया॑ दीक्षि॒तः।
यया॑ऽऽदि॒त्यो दी॒क्षया॑ दीक्षि॒तः॥८३॥\ip

%3.7.7.6
तया᳚ त्वा दी॒क्षया॑ दीक्षयामि।
दिशो॑ दी॒क्षा।
तया॑ च॒न्द्रमा॑ दी॒क्षया॑ दीक्षि॒तः।
यया॑ च॒न्द्रमा॑ दी॒क्षया॑ दीक्षि॒तः।
तया᳚ त्वा दी॒क्षया॑ दीक्षयामि।
आपो॑ दी॒क्षा।
तया॒ वरु॑णो॒ राजा॑ दी॒क्षया॑ दीक्षि॒तः।
यया॒ वरु॑णो॒ राजा॑ दी॒क्षया॑ दीक्षि॒तः।
तया᳚ त्वा दी॒क्षया॑ दीक्षयामि।
ओष॑धयो दी॒क्षा॥८४॥\ip

%3.7.7.7
तया॒ सोमो॒ राजा॑ दी॒क्षया॑ दीक्षि॒तः।
यया॒ सोमो॒ राजा॑ दी॒क्षया॑ दीक्षि॒तः।
तया᳚ त्वा दी॒क्षया॑ दीक्षयामि।
वाग्दी॒क्षा।
तया᳚ प्रा॒णो दी॒क्षया॑ दीक्षि॒तः।
यया᳚ प्रा॒णो दी॒क्षया॑ दीक्षि॒तः।
तया᳚ त्वा दी॒क्षया॑ दीक्षयामि।
पृ॒थि॒वी त्वा॒ दीक्ष॑\-माण॒\-मनु॑ दीक्षताम्।
अ॒न्तरि॑क्षं त्वा॒ दीक्ष॑\-माण॒\-मनु॑ दीक्षताम्।
द्यौस्त्वा॒ दीक्ष॑\-माण॒\-मनु॑ दीक्षताम्॥८५॥\ip

%3.7.7.8
दिश॑स्त्वा॒ दीक्ष॑\-माण॒\-मनु॑ दीक्षन्ताम्।
आप॑स्त्वा॒ दीक्ष॑\-माण॒\-मनु॑ दीक्षन्ताम्।
ओष॑धयस्त्वा॒ दीक्ष॑\-माण॒\-मनु॑ दीक्षन्ताम्।
वाक्त्वा॒ दीक्ष॑\-माण॒\-मनु॑ दीक्षताम्।
ऋच॑स्त्वा॒ दीक्ष॑\-माण॒\-मनु॑ दीक्षन्ताम्।
सामा॑नि त्वा॒ दीक्ष॑\-माण॒\-मनु॑ दीक्षन्ताम्।
यजूꣳ॑षि त्वा॒ दीक्ष॑\-माण॒\-मनु॑ दीक्षन्ताम्।
अह॑श्च॒ रात्रि॑श्च।
कृ॒षिश्च॒ वृष्टि॑श्च।
त्विषि॒श्चाप॑चितिश्च॥८६॥\ip

%3.7.7.9
आप॒श्चौष॑धयश्च।
ऊर्क्च॑ सू॒नृता॑ च।
तास्त्वा॒ दीक्ष॑\-माण॒\-मनु॑ दीक्षन्ताम्।
स्वे दक्षे॒ दक्ष॑पिते॒ह सी॑द।
दे॒वानाꣳ॑ सु॒म्नो म॑ह॒ते रणा॑य।
स्वा॒स॒स्थस्त॒नुवा॒ संवि॑शस्व।
पि॒तेवै॑धि सू॒नव॒ आ सु॒शेवः॑।
शि॒वो मा॑ शि॒वमा वि॑श।
स॒त्यं म॑ आ॒त्मा।
श्र॒द्धा मेऽक्षि॑तिः॥८७॥\ip

%3.7.7.10
तपो॑ मे प्रति॒ष्ठा।
स॒वि॒तृप्र॑सूता मा॒ दिशो॑ दीक्षयन्तु।
स॒त्यम॑स्मि।
अ॒हं त्वद॑स्मि॒ मद॑सि॒ त्वमे॒तत्।
ममा॑सि॒ योनि॒स्तव॒ योनि॑रस्मि।
ममै॒व सन्वह॑ ह॒व्यान्य॑ग्ने।
पु॒त्रः पि॒त्रे लो॑क॒कृज्जा॑तवेदः।
आ॒जुह्वा॑नः सु॒प्रती॑कः पु॒रस्ता᳚त्।
अग्ने॒ स्वां योनि॒मा सी॑द सा॒ध्या।
अ॒स्मिन्थ्स॒धस्थे॒ अध्युत्त॑रस्मिन्॥८८॥\ip

%3.7.7.11
विश्वे॑ देवा॒ यज॑मानश्च सीदत।
एक॑मि॒षे विष्णु॒स्त्वा\-ऽन्वे॑तु।
द्वे ऊ॒र्जे विष्णु॒स्त्वा\-ऽन्वे॑तु।
त्रीणि॑ व्र॒ताय॒ विष्णु॒स्त्वा\-ऽन्वे॑तु।
च॒त्वारि॒ मायो॑भवाय॒ विष्णु॒स्त्वा\-ऽन्वे॑तु।
पञ्च॑ प॒शुभ्यो॒ विष्णु॒स्त्वा\-ऽन्वे॑तु।
षड्रा॒यस्पोषा॑य॒ विष्णु॒स्त्वा\-ऽन्वे॑तु।
स॒प्त स॒प्तभ्यो॒ होत्रा᳚भ्यो॒ विष्णु॒स्त्वा\-ऽन्वे॑तु।
सखा॑यः स॒प्तप॑दा अभूम।
स॒ख्यं ते॑ गमेयम्॥८९॥\ip

%3.7.7.12
स॒ख्यात्ते॒ मा यो॑षम्।
स॒ख्यान्मे॒ मा यो᳚ष्ठाः।
साऽसि॑ सुब्रह्मण्ये।
तस्या᳚स्ते पृथि॒वी पादः॑।
साऽसि॑ सुब्रह्मण्ये।
तस्या᳚स्ते॒\-ऽन्तरि॑क्षं॒ पादः॑।
साऽसि॑ सुब्रह्मण्ये।
तस्या᳚स्ते॒ द्यौः पादः॑।
साऽसि॑ सुब्रह्मण्ये।
तस्या᳚स्ते॒ दिशः॒ पादः॑॥९०॥\ip

%3.7.7.13
प॒रोर॑जास्ते पञ्च॒मः पादः॑।
सा न॒ इष॒मूर्जं॑ धुक्ष्व।
तेज॑ इन्द्रि॒यम्।
ब्र॒ह्म॒व॒र्च॒सम॒न्नाद्यम्᳚।
वि मि॑मे त्वा॒ पय॑स्वतीम्।
दे॒वानां धे॒नुꣳ सु॒दुघा॒मन॑पस्फुरन्तीम्।
इन्द्रः॒ सोमं॑ पिबतु।
क्षेमो॑ अस्तु नः।
इ॒मान्न॑राः कृणुत॒ वेदि॒मेत्य॑।
वसु॑मतीꣳ रु॒द्रव॑तीमादि॒त्यव॑तीम्॥९१॥\ip

%3.7.7.14
वर्ष्म॑न्दि॒वः।
नाभा॑ पृथि॒व्याः।
यथा॒ऽयं यज॑मानो॒ न रिष्ये᳚त्।
दे॒वस्य॑ सवि॒तुः स॒वे।
चतुः॑ शिखण्डा युव॒तिः सु॒पेशाः᳚।
घृ॒तप्र॑तीका॒ भुव॑नस्य॒ मध्ये᳚।
तस्याꣳ॑ सुप॒र्णावधि॒ यौ निवि॑ष्टौ।
तयो᳚र्दे॒वाना॒मधि॑ भाग॒धेयम्᳚।
अप॒ जन्यं॑ भ॒यं नु॑द।
अप॑ च॒क्राणि॑ वर्तय।
गृ॒हꣳ सोम॑स्य गच्छतम्।
न वा उ॑ वे॒तन्म्रि॑यसे॒ न रि॑ष्यसि।
दे॒वाꣳ इदे॑षि प॒थिभिः॑ सु॒गेभिः॑।
यत्र॒ यन्ति॑ सु॒कृतो॒ नापि॑ दु॒ष्कृतः॑।
तत्र॑ त्वा दे॒वः स॑वि॒ता द॑धातु॥९२॥\ip\anuvakamend[ब्रह्म॑णो॒ योनि॒रꣳह॑सः प॒ङ्क्तिं प्रप॑द्ये दी॒क्षा यया॑\-ऽऽदि॒त्यो दी॒क्षया॑ दीक्षि॒तस्तया᳚ त्वा दी॒क्षया॑ दीक्षया॒म्योष॑धयो दी॒क्षा द्यौस्त्वा॒ दीक्ष॑\-माण॒\-मनु॑ दीक्षता॒मप॑चिति॒श्चाक्षि॑ति॒रुत्त॑रस्मिन्गमेयं॒ दिशः॒ पाद॑ आदि॒त्यव॑तीं वर्तय॒ पञ्च॑ च]

%3.7.8.1
यद॒स्य पा॒रे रज॑सः।
शु॒क्रं  ज्योति॒रजा॑यत।
तन्नः॑ पर्‌\mbox{}ष॒दति॒ द्विषः॑।
अग्ने॑ वैश्वानर॒ स्वाहा᳚।
यस्मा᳚द्भी॒षा\-ऽवा॑शिष्ठाः।
ततो॑ नो॒ अभ॑यं कृधि।
प्र॒जाभ्यः॒ सर्वा᳚भ्यो मृड।
नमो॑ रु॒द्राय॑ मी॒ढुषे᳚।
यस्मा᳚द्भी॒षा न्यष॑दः।
ततो॑ नो॒ अभ॑यं कृधि॥९३॥\ip

%3.7.8.2
प्र॒जाभ्यः॒ सर्वा᳚भ्यो मृड।
नमो॑ रु॒द्राय॑ मी॒ढुषे᳚।
उदु॑स्र तिष्ठ॒ प्रति॑ तिष्ठ॒ मारि॑षः।
मेमं य॒ज्ञं यज॑मानं च रीरिषः।
सु॒व॒र्गे लो॒के यज॑मान॒ꣳ॒ हि धे॒हि।
शन्न॑ एधि द्वि॒पदे॒ शं चतु॑ष्पदे।
यस्मा᳚द्भी॒षा\-ऽवे॑पिष्ठाः प॒लायि॑ष्ठाः स॒मज्ञा᳚स्थाः।
ततो॑ नो॒ अभ॑यं कृधि।
प्र॒जाभ्यः॒ सर्वा᳚भ्यो मृड।
नमो॑ रु॒द्राय॑ मी॒ढुषे᳚॥९४॥\ip

%3.7.8.3
य इ॒दमकः॑।
तस्मै॒ नमः॑।
तस्मै॒ स्वाहा᳚।
न वा उ॑वे॒तन्म्रि॑यसे।
आशा॑नां त्वा॒ विश्वा॒ आशाः᳚।
य॒ज्ञस्य॒ हि स्थ ऋ॒त्वियौ᳚।
इन्द्रा᳚ग्नी॒ चेत॑नस्य च।
हु॒ता॒हु॒तस्य॑ तृप्यतम्।
अहु॑तस्य हु॒तस्य॑ च।
हु॒तस्य॒ चाहु॑तस्य च।
अहु॑तस्य हु॒तस्य॑ च।
इन्द्रा᳚ग्नी अ॒स्य सोम॑स्य।
वी॒तं पि॑बतं जु॒षेथा᳚म्।
मा यज॑मानं॒ तमो॑ विदत्।
मर्त्विजो॒ मो इ॒माः प्र॒जाः।
मा यः सोम॑मि॒मं पिबा᳚त्।
सꣳसृ॑ष्टमु॒भयं॑ कृ॒तम्॥९५॥\ip\anuvakamend[कृ॒धि॒ मी॒ढुषे\-ऽहु॑तस्य च स॒प्त च॑]

%3.7.9.1
अ॒ना॒गस॑स्त्वा व॒यम्।
इन्द्रे॑ण॒ प्रेषि॑ता॒ उप॑।
वा॒युष्टे॑ अस्त्वꣳश॒भूः।
मि॒त्रस्ते॑ अस्त्वꣳश॒भूः।
वरु॑णस्ते अस्त्वꣳश॒भूः।
अपा᳚ङ्क्षया॒ ऋत॑स्य गर्भाः।
भुव॑नस्य गोपाः॒ श्येना॑ अतिथयः।
पर्व॑तानां ककुभः प्र॒युतो॑ नपातारः।
व॒ग्नुनेन्द्रꣴ॑ ह्वयत।
घोषे॒णामी॑वाꣴश्चातयत॥९६॥\ip

%3.7.9.2
यु॒क्ताः स्थ॒ वह॑त।
दे॒वा ग्रावा॑ण॒ इन्दु॒रिन्द्र॒ इत्य॑वादिषुः।
एन्द्र॑मचुच्यवुः पर॒मस्याः᳚ परा॒वतः॑।
आऽस्माथ्स॒धस्था᳚त्।
ओरोर॒न्तरि॑क्षात्।
आ सु॑भू॒तम॑सुषवुः।
ब्र॒ह्म॒व॒र्च॒सं म॒ आसु॑षवुः।
स॒म॒रे रक्षाꣴ॑स्यवधिषुः।
अप॑हतं ब्रह्म॒ज्यस्य॑।
वाक्च॑ त्वा॒ मन॑श्च श्रीणीताम्॥९७॥\ip

%3.7.9.3
प्रा॒णश्च॑ त्वा\-ऽपा॒नश्च॑ श्रीणीताम्।
चक्षु॑श्च त्वा॒ श्रोत्रं॑ च श्रीणीताम्।
दक्ष॑श्च त्वा॒ बलं॑ च श्रीणीताम्।
ओज॑श्च त्वा॒ सह॑श्च श्रीणीताम्।
आयु॑श्च त्वा\-ऽज॒रा च॑ श्रीणीताम्।
आ॒त्मा च॑ त्वा त॒नूश्च॑ श्रीणीताम्।
शृ॒तो॑ऽसि शृ॒तं कृ॑तः।
शृ॒ताय॑ त्वा शृ॒तेभ्य॑स्त्वा।
यमिन्द्र॑मा॒हुर्वरु॑णं॒ यमा॒हुः।
यं मि॒त्रमा॒हुर्यमु॑ स॒त्यमा॒हुः॥९८॥\ip

%3.7.9.4
यो दे॒वानां᳚ दे॒वत॑मस्तपो॒जाः।
तस्मै᳚ त्वा॒ तेभ्य॑स्त्वा।
मयि॒ त्यदि॑न्द्रि॒यं म॒हत्।
मयि॒ दक्षो॒ मयि॒ क्रतुः॑।
मयि॑ धायि सु॒वीर्यम्᳚।
त्रिशु॑ग्घ॒र्मो वि भा॑तु मे।
आकू᳚त्या॒ मन॑सा स॒ह।
वि॒राजा॒ ज्योति॑षा स॒ह।
य॒ज्ञेन॒ पय॑सा स॒ह।
तस्य॒ दोह॑मशीमहि॥९९॥\ip

%3.7.9.5
तस्य॑ सु॒म्नम॑शीमहि।
तस्य॑ भ॒क्षम॑शीमहि।
वाग्जु॑षा॒णा सोम॑स्य तृप्यतु।
मि॒त्रो जना॒न्प्र स मि॑त्र।
यस्मा॒न्न जा॒तः परो॑ अ॒न्यो अस्ति॑।
य आ॑वि॒वेश॒ भुव॑नानि॒ विश्वा᳚।
प्र॒जा\-प॑तिः प्र॒जया॑ संविदा॒नः।
त्रीणि॒ ज्योतीꣳ॑षि सचते॒ स षो॑ड॒शी।
ए॒ष ब्र॒ह्मा य ऋ॒त्वियः॑।
इन्द्रो॒ नाम॑ श्रु॒तो ग॒णे॥१००॥\ip

%3.7.9.6
प्र ते॑ म॒हे वि॒दथे॑ शꣳसिष॒ꣳ॒ हरी᳚।
य ऋ॒त्वियः॒ प्र ते॑ वन्वे।
व॒नुषो॑ हर्य॒तं मदम्᳚।
इन्द्रो॒ नाम॑ घृ॒तं नयः।
हरि॑भि॒श्चारु॒ सेच॑ते।
श्रु॒तो ग॒ण आ त्वा॑ विशन्तु।
हरि॑वर्पस॒ङ्गिरः॑।
इन्द्राधि॑प॒ते\-ऽधि॑पति॒स्त्वं दे॒वाना॑मसि।
अधि॑पतिं॒ माम्।
आयु॑ष्मन्तं॒ वर्च॑स्वन्तं मनु॒ष्ये॑षु कुरु॥१०१॥\ip

%3.7.9.7
इन्द्र॑श्च स॒म्राड्वरु॑णश्च॒ राजा᳚।
तौ ते॑ भ॒क्षं च॑क्रतु॒रग्र॑ ए॒तम्।
तयो॒रनु॑\- भ॒क्षं भ॑क्षयामि।
वाग्जु॑षा॒णा सोम॑स्य तृप्यतु।
प्र॒जा\-प॑तिर्वि॒श्वक॑र्मा।
तस्य॒ मनो॑ दे॒वं य॒ज्ञेन॑ राध्यासम्।
अ॒र्थे॒गा अ॒स्य ज॑हितः।
अ॒व॒सान॑पते\-ऽव॒सानं॑ मे विन्द।
नमो॑ रु॒द्राय॑ वास्तो॒ष्पत॑ये।
आय॑ने वि॒द्रव॑णे॥१०२॥\ip

%3.7.9.8
उ॒द्याने॒ यत्प॒राय॑णे।
आ॒वर्त॑ने वि॒वर्त॑ने।
यो गो॑पा॒यति॒ तꣳ हु॑वे।
यान्य॑पा॒मित्या॒न्य\-प्र॑तीत्ता॒\-न्यस्मि॑।
य॒मस्य॑ ब॒लिना॒ चरा॑मि।
इ॒हैव सन्तः॒ प्रति॒ तद्या॑तयामः।
जी॒वा जी॒वेभ्यो॒ नि ह॑राम एनत्।
अ॒नृ॒णा अ॒स्मिन्न॑नृ॒णाः पर॑स्मिन्।
तृ॒तीये॑ लो॒के अ॑नृ॒णाः स्या॑म।
ये दे॑व॒याना॑ उ॒त पि॑तृ॒याणाः᳚॥१०३॥\ip

%3.7.9.9
सर्वा᳚न्प॒थो अ॑नृ॒णा आक्षी॑येम।
इ॒दमू॒नु श्रेयो॑\-ऽव॒सान॒मा ग॑न्म।
शि॒वे नो॒ द्यावा॑पृथि॒वी उ॒भे इ॒मे।
गोम॒द्धन॑व॒दश्व॑व॒दूर्ज॑स्वत्।
सु॒वीरा॑ वी॒रैरनु॒\- सञ्च॑रेम।
अ॒र्कः प॒वित्र॒ꣳ॒ रज॑सो वि॒मानः॑।
पु॒नाति॑ दे॒वानां॒ भुव॑नानि॒ विश्वा᳚।
द्यावा॑पृथि॒वी पय॑सा संविदा॒ने।
घृ॒तं दु॑हाते अ॒मृतं॒ प्रपी॑ने।
प॒वित्र॑म॒र्को रज॑सो वि॒मानः॑।
पु॒नाति॑ दे॒वानां॒ भुव॑नानि॒ विश्वा᳚।
सुव॒र्ज्योति॒र्यशो॑ म॒हत्।
अ॒शी॒महि॑ गा॒धमु॒त प्र॑ति॒ष्ठाम्॥१०४॥\ip\anuvakamend[चा॒त॒य॒त॒ श्री॒णी॒ता॒ꣳ॒ स॒त्यमा॒हुर॑शीमहि ग॒णे कु॑रु वि॒द्रव॑णे पितृ॒याणा॑ अ॒र्को रज॑सो वि॒मान॒स्त्रीणि॑ च]

%3.7.10.1
उद॑स्ताम्फ्सीथ्सवि॒ता मि॒त्रो अ॑र्य॒मा।
सर्वा॑न॒मित्रा॑न\-वधीद्यु॒गेन॑।
बृ॒हन्तं॒ माम॑करद्वी॒र\-व॑न्तम्।
र॒थ॒न्त॒रे श्र॑यस्व॒ स्वाहा॑ पृथि॒व्याम्।
वा॒म॒दे॒व्ये श्र॑यस्व॒ स्वाहा॒\-ऽन्तरि॑क्षे।
बृ॒ह॒ति श्र॑यस्व॒ स्वाहा॑ दि॒वि।
बृ॒ह॒ता त्वोप॑स्तभ्नोमि।
आ त्वा॑ ददे॒ यश॑से वी॒र्या॑य च।
अ॒स्मास्व॑घ्निया यू॒यं द॑धाथेन्द्रि॒यं पयः॑।
यस्ते᳚ द्र॒फ्सो यस्त॑ उद॒र्॒षः॥१०५॥\ip

%3.7.10.2
दैव्यः॑ के॒तुर्विश्वं॒ भुव॑नमावि॒वेश॑।
स नः॑ पा॒ह्यरि॑ष्ट्यै॒ स्वाहा᳚।
अनु॑ मा॒ सर्वो॑ य॒ज्ञो॑\-ऽयमे॑तु।
विश्वे॑ दे॒वा म॒रुतः॒ सामा॒र्कः।
आ॒प्रिय॒श्छन्दाꣳ॑सि नि॒विदो॒ यजूꣳ॑षि।
अ॒स्यै पृ॑थि॒व्यै यद्य॒ज्ञियम्᳚।
प्र॒जा\-प॑तेर्वर्त॒निमनु॑ वर्तस्व।
अनु॑वी॒रैरनु॑\- राध्याम॒ गोभिः॑।
अन्वश्वै॒रनु॒\- सर्वै॑रु पु॒ष्टैः।
अनु॑ प्र॒जया\-ऽन्वि॑न्द्रि॒येण॑॥१०६॥\ip

%3.7.10.3
दे॒वा नो॑ य॒ज्ञमृ॑जु॒धा न॑यन्तु।
प्रति॑\-क्ष॒त्रे प्रति॑ तिष्ठामि रा॒ष्ट्रे।
प्रत्यश्वे॑षु॒ प्रति॑ तिष्ठामि॒ गोषु॑।
प्रति॑ प्र॒जायां॒ प्रति॑ तिष्ठामि॒ भव्ये᳚।
विश्व॑म॒न्याऽभि॑ वावृ॒धे।
तद॒न्यस्या॒मधि॑श्रि॒तम्।
दि॒वे च॑ वि॒श्वक॑र्मणे।
पृ॒थि॒व्यै चा॑करं॒ नमः॑।
अस्का॒न्द्यौः पृ॑थि॒वीम्।
अस्का॑नृष॒भो युवा॒गाः॥१०७॥\ip

%3.7.10.4
स्क॒न्नेमा विश्वा॒ भुव॑ना।
स्क॒न्नो य॒ज्ञः प्र ज॑नयतु।
अस्का॒नज॑नि॒ प्राज॑नि।
आ स्क॒न्नाज्जा॑यते॒ वृषा᳚।
स्क॒न्नात्प्र ज॑निषीमहि।
ये दे॒वा येषा॑मि॒दं भा॑ग॒धेयं॑ ब॒भूव॑।
येषां᳚ प्रया॒जा उ॒तानू॑या॒जाः।
इन्द्र॑ज्येष्ठेभ्यो॒ वरु॑णराजभ्यः।
अ॒ग्निहो॑तृभ्यो दे॒वेभ्यः॒ स्वाहा᳚।
उ॒त त्या नो॒ दिवा॑ म॒तिः॥१०८॥\ip

%3.7.10.5
अदि॑तिरू॒त्या ग॑मत्।
सा शन्ता॑ची॒ मय॑स्करत्।
अप॒ स्रिधः॑।
उ॒त त्या दैव्या॑ भि॒षजा᳚।
शन्न॑स्करतो अ॒श्विना᳚।
यू॒याता॑म॒स्मद्रपः॑।
अप॒ स्रिधः॑।
शम॒ग्निर॒ग्निभि॑स्करत्।
शन्न॑स्तपतु॒ सूर्यः॑।
शं वातो॑ वात्वर॒पाः॥१०९॥\ip

%3.7.10.6
अप॒ स्रिधः॑।
तदित्प॒दं न विचि॑केत वि॒द्वान्।
यन्मृ॒तः पुन॑र॒प्येति॑ जी॒वान्।
त्रि॒वृद्यद्भुव॑नस्य रथ॒वृत्।
जी॒वो गर्भो॒ न मृ॒तः स जी॑वात्।
प्रत्य॑स्मै॒ पिपी॑षते।
विश्वा॑नि वि॒दुषे॑ भर।
अ॒र॒ङ्ग॒माय॒ जग्म॑वे।
अप॑श्चाद्दघ्वने॒ नरे᳚।
इन्दु॒रिन्दु॒मवा॑गात्।
इन्दो॒रिन्द्रो॑\-ऽपात्।
तस्य॑ त इन्द॒विन्द्र॑पीतस्य॒ मधु॑मतः।
उप॑हूत॒स्योप॑हूतो भक्षयामि॥११०॥\ip\anuvakamend[उ॒द॒र्॒ष इ॑न्द्रि॒येण॒ गा म॒तिर॑र॒पा अ॑गा॒त्रीणि॑ च]

%3.7.11.1
ब्रह्म॑ प्रति॒ष्ठा मन॑सो॒ ब्रह्म॑ वा॒चः।
ब्रह्म॑ य॒ज्ञानाꣳ॑ ह॒विषा॒माज्य॑स्य।
अति॑रिक्तं॒ कर्म॑णो॒ यच्च॑ ही॒नम्।
य॒ज्ञः पर्वा॑णि प्रति॒रन्ने॑ति क॒ल्पयन्॑।
स्वाहा॑कृ॒ता\-ऽऽहु॑तिरेतु दे॒वान्।
आश्रा॑वितम॒त्याश्रा॑वितम्।
वष॑ट्कृतम॒त्यनू᳚क्तं च य॒ज्ञे।
अति॑रिक्तं॒ कर्म॑णो॒ यच्च॑ ही॒नम्।
य॒ज्ञः पर्वा॑णि प्रति॒रन्ने॑ति क॒ल्पयन्॑।
स्वाहा॑कृ॒ता\-ऽऽहु॑तिरेतु दे॒वान्॥१११॥\ip

%3.7.11.2
यद्वो॑ देवा अतिपा॒दया॑नि।
वा॒चा चि॒त्प्रय॑तं देव॒हेड॑नम्।
अ॒रा॒यो अ॒स्माꣳ अ॒भिदु॑च्छुना॒यते᳚।
अ॒न्यत्रा॒स्मन्म॑रुत॒स्तन्निधे॑\-तन।
त॒तं म॒ आप॒स्तदु॑ तायते॒ पुनः॑।
स्वादि॑ष्ठा धी॒तिरु॒चथा॑य शस्यते।
अ॒यꣳ स॑मु॒द्र उ॒त वि॒श्वभे॑षजः।
स्वाहा॑कृतस्य॒ समु॑तृप्णुतर्भुवः।
उद्व॒यं तम॑स॒स्परि॑।
उदु॒त्यं चि॒त्रम्॥११२॥\ip

%3.7.11.3
इ॒मं मे॑ वरुण॒ तत्त्वा॑ यामि।
त्वन्नो॑ अग्ने॒ स त्वन्नो॑ अग्ने।
त्वम॑ग्ने अ॒यासि॒ प्रजा॑पते।
इ॒मं जी॒वेभ्यः॑ परि॒धिं द॑धामि।
मैषान्नु॑गा॒दप॑रो॒ अर्ध॑मे॒तम्।
श॒तं जी॑वन्तु श॒रदः॑ पुरू॒चीः।
ति॒रो मृ॒त्युं द॑धतां॒ पर्व॑तेन।
इ॒ष्टेभ्यः॒ स्वाहा॒ वष॒डनि॑ष्टेभ्यः॒ स्वाहा᳚।
भे॒ष॒जं दुरि॑ष्ट्यै॒ स्वाहा॒ निष्कृ॑त्यै॒ स्वाहा᳚।
दौरा᳚र्ध्यै॒ स्वाहा॒ दैवी᳚भ्यस्त॒नूभ्यः॒ स्वाहा᳚॥११३॥\ip

%3.7.11.4
ऋद्ध्यै॒ स्वाहा॒ समृ॑द्ध्यै॒ स्वाहा᳚।
यत॑ इन्द्र॒ भया॑महे।
ततो॑ नो॒ अभ॑यं कृधि।
मघ॑वञ्छ॒ग्धि तव॒ तन्न॑ ऊ॒तये᳚।
वि द्विषो॒ वि मृधो॑ जहि।
स्व॒स्ति॒दा वि॒शस्पतिः॑।
वृ॒त्र॒हा वि मृधो॑ व॒शी।
वृषेन्द्रः॑ पु॒र ए॑तु नः।
स्व॒स्ति॒दा अ॑भयङ्क॒रः।
आ॒भिर्गी॒र्भिर्यदतो॑ न ऊ॒नम्॥११४॥\ip

%3.7.11.5
आप्या॑यय हरिवो॒ वर्ध॑मानः।
य॒दा स्तो॒तृभ्यो॒ महि॑ गो॒त्रा रु॒जासि॑।
भू॒यि॒ष्ठ॒भाजो॒ अध॑ ते स्याम।
अना᳚ज्ञातं॒ यदाज्ञा॑तम्।
य॒ज्ञस्य॑ क्रि॒यते॒ मिथु॑।
अग्ने॒ तद॑स्य कल्पय।
त्वꣳ हि वेत्थ॑ यथात॒थम्।
पुरु॑षसम्मितो य॒ज्ञः।
य॒ज्ञः पुरु॑षसम्मितः।
अग्ने॒ तद॑स्य कल्पय।
त्वꣳ हि वेत्थ॑ यथात॒थम्।
यत्पा॑क॒त्रा मन॑सा दी॒नद॑क्षा॒ न।
य॒ज्ञस्य॑ म॒न्वते॒ मर्ता॑सः।
अ॒ग्निष्टद्धोता᳚ क्रतु॒विद्वि॑जा॒नन्।
यजि॑ष्ठो दे॒वाꣳ ऋ॑तु॒शो य॑जाति॥११५॥\ip\anuvakamend[दे॒वाꣴश्चि॒त्रं त॒नूभ्यः॒ स्वाहो॒नं पुरु॑षसम्मि॒तो\-ऽग्ने॒ तद॑स्य कल्पय॒ पञ्च॑ च]

%3.7.12.1
यद्दे॑वा देव॒हेड॑नम्।
देवा॑सश्चकृ॒मा व॒यम्।
आदि॑त्या॒\-स्तस्मा᳚न्मा मुञ्चत।
ऋ॒तस्य॒र्तेन॒ मामु॒त।
देवा॑ जीवनका॒म्या यत्।
वा॒चा\-ऽनृ॑तमूदि॒म।
अ॒ग्निर्मा॒ तस्मा॒देन॑सः।
गार्‌\mbox{}ह॑पत्यः॒ प्रमु॑ञ्चतु।
दु॒रि॒ता यानि॑ चकृ॒म।
क॒रोतु॒ माम॑ने॒नसम्᳚॥११६॥\ip

%3.7.12.2
ऋ॒तेन॑ द्यावापृथिवी।
ऋ॒तेन॒ त्वꣳ स॑रस्वति।
ऋ॒तान्मा॑ मुञ्च॒ताꣳह॑सः।
यद॒न्यकृ॑तमारि॒म।
स॒जा॒त॒श॒ꣳ॒सादु॒त वा॑ जामिश॒ꣳ॒सात्।
ज्याय॑सः॒ शꣳसा॑दु॒त वा॒ कनी॑यसः।
अना᳚ज्ञातं दे॒वकृ॑तं॒ यदेनः॑।
तस्मा॒त्त्वम॒स्माञ्जा॑तवेदो मुमुग्धि।
यद्वा॒चा यन्मन॑सा।
बा॒हुभ्या॑मू॒रुभ्या॑मष्ठी॒वद्भ्या᳚म्॥११७॥\ip

%3.7.12.3
शि॒श्ञैर्यदनृ॑तं चकृ॒मा व॒यम्।
अ॒ग्निर्मा॒ तस्मा॒देन॑सः।
यद्धस्ता᳚भ्यां च॒कर॒ किल्बि॑षाणि।
अ॒क्षाणां᳚ व॒ग्नुमु॑प॒जिघ्न॑मानः।
दू॒रे॒प॒श्या च॑ राष्ट्र॒भृच्च॑।
तान्य॑फ्स॒रसा॒वनु॑दत्तामृ॒णानि॑।
अदी᳚व्यन्नृ॒णं यद॒हं च॒कार॑।
यद्वादा᳚स्यन्थ्सञ्ज॒गारा॒ जने᳚भ्यः।
अ॒ग्निर्मा॒ तस्मा॒देन॑सः।
यन्मयि॑ मा॒ता गर्भे॑ स॒ति॥११८॥\ip

%3.7.12.4
एन॑श्च॒कार॒ यत्पि॒ता।
अ॒ग्निर्मा॒ तस्मा॒देन॑सः।
यदा॑ पि॒पेष॑ मा॒तरं॑ पि॒तरम्᳚।
पु॒त्रः प्रमु॑दितो॒ धयन्॑।
अहिꣳ॑सितौ पि॒तरौ॒ मया॒ तत्।
तद॑ग्ने अनृ॒णो भ॑वामि।
यद॒न्तरि॑क्षं पृथि॒वीमु॒त द्याम्।
यन्मा॒तरं॑ पि॒तरं॑ वा जिहिꣳसि॒म।
अ॒ग्निर्मा॒ तस्मा॒देन॑सः।
यदा॒शसा॑ नि॒शसा॒ यत्प॑रा॒शसा᳚॥११९॥\ip

%3.7.12.5
यदेन॑श्चकृ॒मा नूत॑नं॒ यत्पु॑रा॒णम्।
अ॒ग्निर्मा॒ तस्मा॒देन॑सः।
अति॑ क्रामामि दुरि॒तं यदेनः॑।
जहा॑मि रि॒प्रं प॑र॒मे स॒धस्थे᳚।
यत्र॒ यन्ति॑ सु॒कृतो॒ नापि॑ दु॒ष्कृतः॑।
तमा रो॑हामि सु॒कृतां॒ नु लो॒कम्।
त्रि॒ते दे॒वा अ॑मृजतै॒तदेनः॑।
त्रि॒त ए॒तन्म॑नु॒ष्ये॑षु मामृजे।
ततो॑ मा॒ यदि॒ किञ्चि॑दान॒शे।
अ॒ग्निर्मा॒ तस्मा॒देन॑सः॥१२०॥\ip

%3.7.12.6
गार्‌\mbox{}ह॑पत्यः॒ प्रमु॑ञ्चतु।
दु॒रि॒ता यानि॑ चकृ॒म।
क॒रोतु॒ माम॑ने॒नसम्᳚।
दि॒वि जा॒ता अ॒फ्सु जा॒ताः।
या जा॒ता ओष॑धीभ्यः।
अथो॒ या अ॑ग्नि॒जा आपः॑।
ता नः॑ शुन्धन्तु॒ शुन्ध॑नीः।
यदापो॒ नक्तं॑ दुरि॒तं चरा॑म।
यद्वा॒ दिवा॒ नूत॑नं॒ यत्पु॑रा॒णम्।
हिर॑ण्यवर्णा॒स्तत॒ उत्पु॑नीत नः।
इ॒मं मे॑ वरुण॒ तत्त्वा॑ यामि।
त्वन्नो॑ अग्ने॒ स त्वन्नो॑ अग्ने।
त्वम॑ग्ने अ॒यासि॑॥१२१॥\ip\anuvakamend[अ॒ने॒नस॑मष्ठी॒वद्भ्याꣳ॑ स॒ति प॑रा॒शसा॑\-ऽऽन॒शे᳚\-ऽग्निर्मा॒ तस्मा॒देन॑सः पुनीत न॒स्त्रीणि॑ च (यद्दे॑वा॒ देवा॑ ऋ॒तेन॑ सजातश॒ꣳ॒साद्यद्वा॒चा यद्धस्ता᳚भ्या॒मदी᳚व्यं॒ यन्मयि॑ मा॒ता यदा॑ पि॒पेष॒ यद॒न्तरि॑क्षं॒ यदा॒शसाऽति॑ क्रामामि त्रि॒ते दे॒वा दि॒वि जा॒ता अ॒फ्सु जा॒ता यदाप॑ इ॒मं मे॑ वरुण॒ तत्त्वा॑ यामि॒ त्वन्नो॑ अग्ने॒ स त्वन्नो॑ अग्ने॒ त्वम॑ग्ने अ॒यासि॑।
)]

%3.7.13.1
यत्ते॒ ग्राव्ण्णा॑ चिच्छि॒दुः सो॑म राजन्।
प्रि॒याण्यङ्गा॑नि॒ स्वधि॑ता॒ परूꣳ॑षि।
तथ्सन्ध॒थ्स्वाज्ये॑नो॒त व॑र्धयस्व।
अ॒ना॒गसो॒ अध॒मिथ्स॒ङ्क्षये॑म।
यत्ते॒ ग्रावा॑ बा॒हुच्यु॑तो॒ अचु॑च्यवुः।
नरो॒ यत्ते॑ दुदु॒हुर्दक्षि॑णेन।
तत्त॒ आप्या॑यतां॒ तत्ते᳚।
निष्ट्या॑यतां देव सोम।
यत्ते॒ त्वचं॑ बिभि॒दुर्यच्च॒ योनिम्᳚।
यदा॒स्थाना॒त्प्रच्यु॑तो॒ वेन॑सि॒ त्मना᳚॥१२२॥\ip

%3.7.13.2
त्वया॒ तथ्सो॑म गु॒प्तम॑स्तु नः।
सा नः॑ स॒न्धास॑त्पर॒मे व्यो॑मन्।
अहा॒च्छरी॑रं॒ पय॑सा स॒मेत्य॑।
अ॒न्यो᳚न्यो भवति॒ वर्णो॑ अस्य।
तस्मि॑न्व॒यमुप॑हूता॒स्तव॑ स्मः।
आ नो॑ भज॒ सद॑सि वि॒श्वरू॑पे।
नृ॒चक्षाः॒ सोम॑ उ॒त शु॒श्रुग॑स्तु।
मा नो॒ वि हा॑सी॒द्गिर॑ आवृणा॒नः।
अना॑गास्त॒नुवो॑ वावृधा॒नः।
आ नो॑ रू॒पं व॑हतु॒ जाय॑मानः॥१२३॥\ip

%3.7.13.3
उप॑ क्षरन्ति जु॒ह्वो॑ घृ॒तेन॑।
प्रि॒याण्यङ्गा॑नि॒ तव॑ व॒र्धय॑न्तीः।
तस्मै॑ ते सोम॒ नम॒ इद्वष॑ट्च।
उप॑ मा राजन्थ्सुकृ॒ते ह्व॑यस्व।
सं प्रा॑णापा॒नाभ्या॒ꣳ॒ समु॒ चक्षु॑षा॒ त्वम्।
सꣴ श्रोत्रे॑ण गच्छस्व सोम राजन्।
यत्त॒ आस्थि॑त॒ꣳ॒ शमु॒ तत्ते॑ अस्तु।
जा॒नी॒तान्नः॑ स॒ङ्गम॑ने पथी॒नाम्।
ए॒तं जा॑नीतात्पर॒मे व्यो॑मन्।
वृकाः᳚ सधस्था वि॒द रू॒पम॑स्य॥१२४॥\ip

%3.7.13.4
यदा॒गच्छा᳚त्प॒थिभि॑र्देव॒यानैः᳚।
इ॒ष्टा॒पू॒र्ते कृ॑णुतादा॒विर॑स्मै।
अरि॑ष्टो राजन्नग॒दः परे॑हि।
नम॑स्ते अस्तु॒ चक्ष॑से रघूय॒ते।
नाक॒मारो॑ह स॒ह यज॑मानेन।
सूर्यं॑ गच्छतात्पर॒मे व्यो॑मन्।
अभू᳚द्दे॒वः स॑वि॒ता वन्द्यो॒नु नः॑।
इ॒दानी॒मह्न॑ उप॒वाच्यो॒ नृभिः॑।
वि यो रत्ना॒ भज॑ति मान॒वेभ्यः॑।
श्रेष्ठं॑ नो॒ अत्र॒ द्रवि॑णं॒ यथा॒ दध॑त्।
उप॑ नो मित्रावरुणावि॒हाव॑तम्।
अ॒न्वादी᳚ध्याथामि॒ह नः॑ सखाया।
आ॒दि॒त्यानां॒ प्रसि॑तिर्\mbox{}हे॒तिः।
उ॒ग्रा श॒तापा᳚ष्ठा घ॒विषा॒ परि॑ णो वृणक्तु।
आप्या॑यस्व॒ सन्ते᳚॥१२५॥\ip\anuvakamend[त्मना॒ जाय॑मानो\-ऽस्य॒ दध॒त्पञ्च॑ च]

%3.7.14.1
यद्दि॑दी॒क्षे मन॑सा॒ यच्च॑ वा॒चा।
यद्वा᳚ प्रा॒णैश्चक्षु॑षा॒ यच्च॒ श्रोत्रे॑ण।
यद्रेत॑सा मिथु॒नेनाप्या॒त्मना᳚।
अ॒द्भ्यो लो॒का द॑धिरे॒ तेज॑ इन्द्रि॒यम्।
शु॒क्रा दी॒क्षायै॒ तप॑सो वि॒मोच॑नीः।
आपो॑ विमो॒क्त्रीर्मयि॒ तेज॑ इन्द्रि॒यम्।
यदृ॒चा साम्ना॒ यजु॑षा।
प॒शू॒नां चर्म॑न् ह॒विषा॑ दिदी॒क्षे।
यच्छन्दो॑भि॒रोष॑धीभि॒र्वन॒स्पतौ᳚।
अ॒द्भ्यो लो॒का द॑धिरे॒ तेज॑ इन्द्रि॒यम्॥१२६॥\ip

%3.7.14.2
शु॒क्रा दी॒क्षायै॒ तप॑सो वि॒मोच॑नीः।
आपो॑ विमो॒क्त्रीर्मयि॒ तेज॑ इन्द्रि॒यम्।
येन॒ ब्रह्म॒ येन॑ क्ष॒त्रम्।
येने᳚न्द्रा॒ग्नी प्र॒जा\-प॑तिः॒ सोमो॒ वरु॑णो॒ येन॒ राजा᳚।
विश्वे॑ दे॒वा ऋष॑यो॒ येन॑ प्रा॒णाः।
अ॒द्भ्यो लो॒का द॑धिरे॒ तेज॑ इन्द्रि॒यम्।
शु॒क्रा दी॒क्षायै॒ तप॑सो वि॒मोच॑नीः।
आपो॑ विमो॒क्त्रीर्मयि॒ तेज॑ इन्द्रि॒यम्।
अ॒पां पुष्प॑म॒स्योष॑धीना॒ꣳ॒ रसः॑।
सोम॑स्य प्रि॒यं धाम॑॥१२७॥\ip

%3.7.14.3
अ॒ग्नेः प्रि॒यत॑मꣳ ह॒विः स्वाहा᳚।
अ॒पां पुष्प॑म॒स्योष॑धीना॒ꣳ॒ रसः॑।
सोम॑स्य प्रि॒यं धाम॑।
इन्द्र॑स्य प्रि॒यत॑मꣳ ह॒विः स्वाहा᳚।
अ॒पां पुष्प॑म॒स्योष॑धीना॒ꣳ॒ रसः॑।
सोम॑स्य प्रि॒यं धाम॑।
विश्वे॑षां दे॒वानां᳚ प्रि॒यत॑मꣳ ह॒विः स्वाहा᳚।
व॒यꣳ सो॑म व्र॒ते तव॑।
मन॑स्त॒नूषु॒ पिप्र॑तः।
प्र॒जाव॑न्तो अशीमहि॥१२८॥\ip

%3.7.14.4
दे॒वेभ्यः॑ पि॒तृभ्यः॒ स्वाहा᳚।
सो॒म्येभ्यः॑ पि॒तृभ्यः॒ स्वाहा᳚।
क॒व्येभ्यः॑ पि॒तृभ्यः॒ स्वाहा᳚।
देवा॑स इ॒ह मा॑दयध्वम्।
सोम्या॑स इ॒ह मा॑दयध्वम्।
कव्या॑स इ॒ह मा॑दयध्वम्।
अन॑न्तरिताः पि॒तरः॑ सो॒म्याः सो॑मपी॒थात्।
अपै॑तु मृ॒त्युर॒मृतं॑ न॒ आगन्॑।
वै॒व॒स्व॒तो नो॒ अभ॑यं कृणोतु।
प॒र्णं वन॒स्पते॑रिव॥१२९॥\ip

%3.7.14.5
अ॒भि नः॑ शीयताꣳ र॒यिः।
सच॑तां नः॒ शची॒पतिः॑।
परं॑ मृत्यो॒ अनु॒ परे॑हि॒ पन्था᳚म्।
यस्ते॒ स्व इत॑रो देव॒याना᳚त्।
चक्षु॑ष्मते शृण्व॒ते ते᳚ ब्रवीमि।
मा नः॑ प्र॒जाꣳ री॑रिषो॒ मोत वी॒रान्।
इ॒दमू॒नु श्रेयो॑व॒सान॒माग॑न्म।
यद्गो॒जिद्ध॑न॒जिद॑श्व॒जिद्यत्।
प॒र्णं वन॒स्पते॑रिव।
अ॒भि नः॑ शीयताꣳ र॒यिः।
सच॑तां नः॒ शची॒पतिः॑॥१३०॥\ip\anuvakamend[वन॒स्पता॑व॒द्भ्यो लो॒का द॑धिरे॒ तेज॑ इन्द्रि॒यं धामा॑शीमहीवा॒भिनः॑ शीयताꣳ र॒यिरेकं॑ च]


\prashnaend{सर्वा॒न्॒ यद्विष्ष॑ण्णेन॒ वि वै याः पु॒रस्ता॒द्देवा॑ दे॒वेषु॒ परि॑स्तृणीत॒ सक्षे॒दं यद॒स्य पा॒रे॑\-ऽना॒गस॒ उद॑स्ताम्फ्सी॒द्ब्रह्म॑ प्रति॒ष्ठा यद्दे॑वा॒ यत्ते॒ ग्राव्ण्णा॒ यद्दि॑दी॒क्षे चतु॑र्दश॥१४॥}{सर्वा॒न्भूति॑मे॒व यामे॒वाफ्स्वाहु॑तिं व्र॒तानां᳚ पर्णव॒ल्कः सो॒म्याना॑म॒स्मिन्‌ य॒ज्ञे\-ऽग्ने॒ यो नो॒ ज्योग्जी॒वाः प॒रोर॑जाः॒ प्रते॑महे॒ ब्रह्म॑ प्रति॒ष्ठा गार्‌\mbox{}ह॑पत्यस्त्रि॒ꣳ॒शदु॑त्तरश॒तम्॥१३०॥}{सर्वा॒ञ्छची॒पतिः॑॥}{हरिः॑ ओम्॥}{इति श्रीकृष्णयजुर्वेदीयतैत्तिरीयब्राह्मणे तृतीयाष्टके सप्तमः प्रपाठकः समाप्तः॥}
\sect{अष्टमः प्रश्नः}
\setcounter{anuvakam}{0}
\dnsub{तैत्तिरीयब्राह्मणे तृतीयाष्टके अष्टमः प्रपाठकः}

%3.8.1.1
सा॒ङ्ग्र॒ह॒ण्येष्ट्या॑ यजते।
इ॒माञ्ज॒नता॒ꣳ॒ सङ्गृ॑ह्णा॒नीति॑।
द्वाद॑शारत्नी रश॒ना भ॑वति।
द्वाद॑श॒ मासाः᳚ संवथ्स॒रः।
सं॒व॒थ्स॒रमे॒वाव॑ रुन्धे।
मौ॒ञ्जी भ॑वति।
ऊर्ग्वै मुञ्जाः᳚।
ऊर्ज॑\-मे॒वाव॑ रुन्धे।
चि॒त्रा नक्ष॑त्रं भवति।
चि॒त्रं वा ए॒तत्कर्म॑॥१॥\ip

%3.8.1.2
यद॑श्वमे॒धः समृ॑द्ध्यै।
पुण्य॑नाम देव॒यज॑नम॒ध्यव॑स्यति।
पुण्या॑मे॒व तेन॑ की॒र्तिम॒भि ज॑यति।
अप॑दातीनृ॒त्विजः॑ स॒माव॑ह॒न्त्या सु॑ब्रह्म॒ण्यायाः᳚।
सु॒व॒र्गस्य॑ लो॒कस्य॒ सम॑ष्ट्यै।
के॒श॒श्म॒श्रु व॑पते।
न॒खानि॒ नि कृ॑न्तते।
द॒तो धा॑वते।
स्नाति॑।
अह॑तं॒ वासः॒ परि॑धत्ते।
पा॒प्मनो\-ऽप॑हत्यै।
वाचं॑ य॒त्वोप॑ वसति।
सु॒व॒र्गस्य॑ लो॒कस्य॒ गुप्त्यै᳚।
रात्रिं॑ जाग॒रय॑न्त आसते।
सु॒व॒र्गस्य॑ लो॒कस्य॒ सम॑ष्ट्यै॥२॥\ip\anuvakamend[कर्म॑ धत्ते॒ पञ्च॑ च]

%3.8.2.1
चतु॑ष्टय्य॒ आपो॑ भवन्ति।
चतुः॑ शफो॒ वा अश्वः॑ प्राजाप॒त्यः समृ॑द्ध्यै।
ता दि॒ग्भ्यः स॒माभृ॑ता भवन्ति।
दि॒क्षु वा आपः॑।
अन्नं॒ वा आपः॑।
अ॒द्भ्यो वा अन्नं॑ जायते।
यदे॒वाद्भ्यो\-ऽन्नं॒ जाय॑ते।
तदव॑ रुन्धे।
तासु॑ ब्रह्मौद॒नं प॑चति।
रेत॑ ए॒व तद्द॑धाति॥३॥\ip

%3.8.2.2
चतुः॑ शरावो भवति।
दि॒क्ष्वे॑व प्रति॑ तिष्ठति।
उ॒भ॒यतो॑रु॒क्मौ भ॑वतः।
उ॒भ॒यत॑ ए॒वास्मि॒न्रुचं॑ दधाति।
उद्ध॑रति शृत॒त्वाय॑।
स॒र्पिष्वा᳚न्भवति मेध्य॒त्वाय॑।
च॒त्वार॑ आर्\mbox{}षे॒याः प्राश्ञ॑न्ति।
दि॒शामे॒व ज्योति॑षि जुहोति।
च॒त्वारि॒ हिर॑ण्यानि ददाति।
दि॒शामे॒व ज्योती॒ꣴ॒ष्यव॑ रुन्धे॥४॥\ip

%3.8.2.3
यदाज्य॑मु॒च्छिष्य॑ते।
तस्मि॑न्रश॒नान्यु॑नत्ति।
प्र॒जा\-प॑ति॒र्वा ओ॑द॒नः।
रेत॒ आज्यम्᳚।
यदाज्ये॑ रश॒नान्यु॒नत्ति॑।
प्र॒जा\-प॑तिमे॒व रेत॑सा॒ सम॑र्धयति।
द॒र्भ॒मयी॑ रश॒ना भ॑वति।
ब॒हु वा ए॒ष कु॑च॒रो॑ मे॒ध्यमुप॑गच्छति।
यदश्वः॑।
प॒वित्रं॒ वै द॒र्भाः॥५॥\ip

%3.8.2.4
यद्द॑र्भ॒मयी॑ रश॒ना भव॑ति।
पु॒नात्ये॒वैनम्᳚।
पू॒तमे॑नं॒ मेध्य॒मा ल॑भते।
अश्व॑स्य॒ वा आल॑ब्धस्य महि॒मोद॑क्रामत्।
स म॒हर्त्वि॑जः॒ प्रावि॑शत्।
तन्म॒हर्त्वि॑जां महर्त्वि॒क्त्वम्।
यन्म॒हर्त्वि॑जः प्रा॒श्ञन्ति॑।
म॒हि॒मान॑मे॒वास्मि॒न्तद्द॑धति।
अश्व॑स्य॒ वा आल॑ब्धस्य॒ रेत॒ उद॑क्रामत्।
तथ्सु॒वर्ण॒ꣳ॒ हिर॑ण्यमभवत्।
यथ्सु॒वर्ण॒ꣳ॒ हिर॑ण्यं॒ ददा॑ति।
रेत॑ ए॒व तद्द॑धाति।
ओ॒द॒ने द॑दाति।
रेतो॒ वा ओ॑द॒नः।
रेतो॒ हिर॑ण्यम्।
रेत॑सै॒वास्मि॒न्रेतो॑ दधाति॥६॥\ip\anuvakamend[द॒धा॒ति॒ रु॒न्धे॒ द॒र्भा अ॑भव॒थ्षट् च॑]

%3.8.3.1
यो वै ब्रह्म॑णे दे॒वेभ्यः॑ प्र॒जा\-प॑त॒ये\-ऽप्र॑तिप्रो॒च्याश्वं॒ मेध्यं॑ ब॒ध्नाति॑।
आ दे॒वता᳚भ्यो वृश्च्यते।
पापी॑यान्भवति।
यः प्र॑ति॒प्रोच्य॑।
न दे॒वता᳚भ्य॒ आवृ॑श्च्यते।
वसी॑यान्भवति।
यदाह॑।
ब्रह्म॒न्नश्वं॒ मेध्यं॑ भन्थ्स्यामि दे॒वेभ्यः॑ प्र॒जा\-प॑तये॒ तेन॑ राध्यास॒मिति॑।
ब्रह्म॒ वै ब्र॒ह्मा।
ब्रह्म॑ण ए॒व दे॒वेभ्यः॑ प्र॒जा\-प॑तये प्रति॒प्रोच्याश्वं॒ मेध्यं॑ बध्नाति॥७॥\ip

%3.8.3.2
न दे॒वता᳚भ्य॒ आ वृ॑श्च्यते।
वसी॑यान्भवति।
दे॒वस्य॑ त्वा सवि॒तुः प्र॑स॒व इति॑ रश॒नामाद॑त्ते॒ प्रसू᳚त्यै।
अ॒श्विनो᳚र्बा॒हुभ्या॒मित्या॑ह।
अ॒श्विनौ॒ हि दे॒वाना॑मध्व॒र्यू आस्ता᳚म्।
पू॒ष्णो हस्ता᳚भ्या॒मित्या॑ह॒ यत्यै᳚।
व्यृ॑द्धं॒ वा ए॒तद्य॒ज्ञस्य॑।
यद॑य॒जुष्के॑ण क्रि॒यते᳚।
इ॒माम॑गृभ्णन्रश॒नामृ॒तस्ये\-त्यधि॑ वदति॒ यजु॑ष्कृत्यै।
य॒ज्ञस्य॒ समृ॑द्ध्यै॥८॥\ip

%3.8.3.3
तदा॑हुः।
द्वाद॑शारत्नी रश॒ना क॑र्त॒व्या(३) त्रयो॑दशार॒त्नी(३)\-रिति॑।
ऋ॒ष॒भो वा ए॒ष ऋ॑तू॒नाम्।
यथ्सं॑वथ्स॒रः।
तस्य॑ त्रयोद॒शो मासो॑ वि॒ष्टपम्᳚।
ऋ॒ष॒भ ए॒ष य॒ज्ञाना᳚म्।
यद॑श्वमे॒धः।
यथा॒ वा ऋ॑ष॒भस्य॑ वि॒ष्टपम्᳚।
ए॒वमे॒तस्य॑ वि॒ष्टपम्᳚।
त्र॒यो॒द॒शम॑र॒त्निꣳ र॑श॒नाया॑मु॒पा द॑धाति॥९॥\ip

%3.8.3.4
यथ॑र्\mbox{}ष॒भस्य॑ वि॒ष्टपꣳ॑ सꣴस्क॒रोति॑।
ता॒दृगे॒व तत्।
पूर्व॒ आयु॑षि वि॒दथे॑षु क॒व्येत्या॑ह।
आयु॑रे॒वास्मि॑न्दधाति।
तया॑ दे॒वाः सु॒तमा ब॑भूवु॒रित्या॑ह।
भूति॑मे॒वोपाव॑र्तते।
ऋ॒तस्य॒ साम᳚न्थ्स॒रमा॒रप॒न्तीत्या॑ह।
स॒त्यं वा ऋ॒तम्।
स॒त्येनै॒वैन॑मृ॒तेनार॑भते।
अ॒भि॒धा अ॒सीत्या॑ह॥१०॥\ip

%3.8.3.5
तस्मा॑दश्वमेधया॒जी सर्वा॑णि भू॒तान्य॒भि भ॑वति।
भुव॑नम॒सी\-त्या॑ह।
भू॒मान॑मे॒वोपै॑ति।
य॒न्ता\-ऽसीत्या॑ह।
य॒न्तार॑मे॒वैनं॑ करोति।
ध॒र्ता\-ऽसीत्या॑ह।
ध॒र्तार॑मे॒वैनं॑ करोति।
सो᳚ऽग्निं वै᳚श्वान॒रमित्या॑ह।
अ॒ग्नावे॒वैनं॑ वैश्वान॒रे जु॑होति।
सप्र॑थस॒मित्या॑ह॥११॥\ip

%3.8.3.6
प्र॒जयै॒वैनं॑ प॒शुभिः॑ प्रथयति।
स्वाहा॑कृत॒ इत्या॑ह।
होम॑ ए॒वास्यै॒षः।
पृ॒थि॒व्यामित्या॑ह।
अ॒स्यामे॒वैनं॒ प्रति॑\-ष्ठापयति।
य॒न्ता राड्य॒न्ताऽसि॒ यम॑नो ध॒र्ताऽसि॑ ध॒रुण॒ इत्या॑ह।
रू॒पमे॒वास्यै॒तन्म॑हि॒मानं॒ व्याच॑ष्टे।
कृ॒ष्यै त्वा॒ क्षेमा॑य त्वा र॒य्यै त्वा॒ पोषा॑य॒ त्वेत्या॑ह।
आ॒\-मे॒वैतामा शा᳚स्ते।
स्व॒गा त्वा॑ दे॒वेभ्य॒ इत्या॑ह।
दे॒वेभ्य॑ ए॒वैनꣴ॑ स्व॒गा क॑रोति।
स्वाहा᳚ त्वा प्र॒जा\-प॑तय॒ इत्या॑ह।
प्रा॒जा॒प॒त्यो वा अश्वः॑।
यस्या॑ ए॒व दे॒वता॑या आल॒भ्यते᳚।
तयै॒वैन॒ꣳ॒ सम॑र्धयति॥१२॥\ip\anuvakamend[ब॒ध्ना॒ति॒ समृ॑द्ध्या उ॒पाद॑धात्य॒सीत्या॑ह॒ सप्र॑थस॒मित्या॑ह दे॒वेभ्य॒ इत्या॑ह॒ पञ्च॑ च]

%3.8.4.1
यः पि॒तुर॑नु॒जायाः᳚ पु॒त्रः।
स पु॒रस्ता᳚न्नयति।
यो मा॒तुर॑नु॒जायाः᳚ पु॒त्रः।
स प॒श्चान्न॑यति।
विष्व॑ञ्चमे॒वास्मा᳚त्पा॒प्मानं॒ विवृ॑हतः।
यो अर्व॑न्तं॒ जिघाꣳ॑सति॒ तम॒भ्य॑मीति॒ वरु॑ण॒ इति॒ श्वानं॑ चतुर॒क्षं प्रसौ॑ति।
प॒रो मर्तः॑ प॒रः श्वेति॒ शुन॑श्चतुर॒क्षस्य॒ प्रह॑न्ति।
श्वेव॒ वै पा॒प्मा भ्रातृ॑व्यः।
पा॒प्मान॑मे॒वास्य॒ भ्रातृ॑व्यꣳ हन्ति।
सै॒ध्र॒कं मुस॑लं भवति॥१३॥\ip

%3.8.4.2
कर्म॑कर्मै॒वास्मै॑ साधयति।
पौ॒ꣴ॒श्च॒ले॒यो ह॑न्ति।
पु॒ꣴ॒श्च॒ल्वां वै दे॒वाः शुचं॒ न्य॑दधुः।
शु॒चैवास्य॒ शुचꣳ॑ हन्ति।
पा॒प्मा वा ए॒तमी᳚फ्स॒तीत्या॑हुः।
यो᳚ऽश्वमे॒धेन॒ यज॑त॒ इति॑।
अश्व॑स्याधस्प॒दमु\-पा᳚स्यति।
व॒ज्री वा अश्वः॑ प्राजाप॒त्यः।
वज्रे॑णै॒व पा॒प्मानं॒ भ्रातृ॑व्य॒मव॑\-क्रामति।
द॒क्षि॒णाऽप॑ प्लावयति॥१४॥\ip

%3.8.4.3
पा॒प्मान॑मे॒वास्मा॒च्छम॑ल॒मप॑ प्लावयति।
ऐ॒षी॒क उ॑दू॒हो भ॑वति।
आयु॒र्वा इ॒षीकाः᳚।
आयु॑रे॒वास्मि॑न्दधति।
अ॒मृतं॒ वा इ॒षीकाः᳚।
अ॒मृत॑मे॒वास्मि॑न्दधति।
वे॒त॒स॒शा॒खोप॒सम्ब॑द्धा भवति।
अ॒फ्सुयो॑नि॒र्वा अश्वः॑।
अ॒फ्सु॒जो वे॑त॒सः।
स्वादे॒वैनं॒ योने॒र्निर्मि॑मीते।
पु॒रस्ता᳚त्प्र॒त्यञ्च॑म॒भ्युदू॑हति।
पु॒रस्ता॑दे॒वास्मि॑न्प्र॒तीच्य॒मृतं॑ दधाति।
अ॒हं च॒ त्वं च॑ वृत्रह॒न्निति॑ ब्र॒ह्मा यज॑मानस्य॒ हस्तं॑ गृह्णाति।
ब्र॒ह्म॒क्ष॒त्रे ए॒व सन्द॑धाति।
अ॒भिक्रत्वे᳚न्द्र भू॒रध॒ज्मन्नित्य॑ध्व॒र्युर्यज॑मानं वाचयत्य॒भिजि॑त्यै॥१५॥\ip\anuvakamend[भ॒व॒ति॒ प्ला॒व॒य॒ति॒ मि॒मी॒ते॒ पञ्च॑ च]

%3.8.5.1
च॒त्वार॑ ऋ॒त्विजः॒ समु॑क्षन्ति।
आ॒भ्य ए॒वैनं॑ चत॒सृभ्यो॑ दि॒ग्भ्यो॑\-ऽभि समी॑रयन्ति।
श॒तेन॑ राजपु॒त्रैः स॒हाध्व॒र्युः।
पु॒रस्ता᳚त्\-प्र॒त्यङ्तिष्ठ॒न्प्रोक्ष॑ति।
अ॒नेनाश्वे॑न॒ मेध्ये॑ने॒ष्ट्वा।
अ॒यꣳ राजा॑ वृ॒त्रं व॑ध्या॒दिति॑।
रा॒ज्यं वा अ॑ध्व॒र्युः।
क्ष॒त्रꣳ रा॑जपु॒त्रः।
रा॒ज्ये\-नै॒वास्मि॑न्क्ष॒त्रं द॑धाति।
श॒तेना॑रा॒जभि॑रु॒ग्रैः स॒ह ब्र॒ह्मा॥१६॥\ip

%3.8.5.2
द॒क्षि॒ण॒त उद॒ङ्तिष्ठ॒न्प्रोक्ष॑ति।
अ॒नेनाश्वे॑न॒ मेध्ये॑ने॒ष्ट्वा।
अ॒यꣳ राजा᳚\-ऽप्रतिधृ॒ष्यो᳚\-ऽस्त्विति॑।
बलं॒ वै ब्र॒ह्मा।
बल॑मरा॒जोग्रः।
बले॑\-नै॒वा\-स्मि॒न्बलं॑ दधाति।
श॒तेन॑ सूतग्राम॒णिभिः॑ स॒ह होता᳚।
प॒श्चात्प्राङ्तिष्ठ॒न्प्रोक्ष॑ति।
अ॒नेनाश्वे॑न॒ मेध्ये॑ने॒ष्ट्वा।
अ॒यꣳ राजा॒\-ऽस्यै वि॒शः॥१७॥\ip

%3.8.5.3
ब॒हु॒ग्वै ब॑ह्व॒श्वायै॑ बह्वजावि॒कायै᳚।
ब॒हु॒व्री॒हि॒य॒वायै॑ बहुमाष\-ति॒लायै᳚।
ब॒हु॒हि॒र॒ण्यायै॑ बहुह॒स्तिका॑यै।
ब॒हु॒दा॒स॒\-पू॒रु॒षायै॑ रयि॒मत्यै॒ पुष्टि॑मत्यै।
ब॒हु॒रा॒य॒स्पो॒षायै॒ राजा॒स्त्विति॑।
भू॒मा वै होता᳚।
भू॒मा सू॑तग्राम॒ण्यः॑।
भू॒म्नैवास्मि॑न्भू॒मानं॑ दधाति।
श॒तेन॑ क्षत्तसङ्ग्रही॒तृभिः॑ स॒होद्गा॒ता।
उ॒त्त॒र॒तो द॑क्षि॒णा तिष्ठ॒न्प्रोक्ष॑ति॥१८॥\ip

%3.8.5.4
अ॒नेनाश्वे॑न॒ मेध्ये॑ने॒ष्ट्वा।
अ॒यꣳ राजा॒ सर्व॒मायु॑रे॒त्विति॑।
आयु॒र्वा उ॑द्गा॒ता।
आयुः॑ क्षत्तसङ्ग्रही॒तारः॑।
आयु॑षै॒वास्मि॒न्नायु॑र्द\-धाति।
श॒तꣳ\-श॑तं भवन्ति।
श॒तायुः॒ पुरु॑षः श॒तेन्द्रि॑यः।
आयु॑ष्ये॒वेन्द्रि॒ये प्रति॑ तिष्ठति।
च॒तुः॒ श॒ता भ॑वन्ति।
चत॑स्रो॒ दिशः॑।
दि॒क्ष्वे॑व प्रति॑ तिष्ठति॥१९॥\ip\anuvakamend[ब्र॒ह्मा वि॒श उ॑क्षति॒ दिश॒ एकं॑ च]

%3.8.6.1
यथा॒ वै ह॒विषो॑ गृही॒तस्य॒ स्कन्द॑ति।
ए॒वं वा ए॒तदश्व॑स्य स्कन्दति।
यन्नि॒क्तमना॑लब्धमुथ्सृ॒जन्ति॑।
यथ्स्तोक्या॑ अ॒न्वाह॑।
स॒र्व॒हुत॑मे॒वैनं॑ करो॒त्यस्क॑न्दाय।
अस्क॑न्न॒ꣳ॒ हि तत्।
यद्धु॒तस्य॒ स्कन्द॑ति।
स॒हस्र॒मन्वा॑ह।
स॒हस्र॑सम्मितः सुव॒र्गो लो॒कः।
सु॒व॒र्गस्य॑ लो॒कस्या॒भिजि॑त्यै॥२०॥\ip

%3.8.6.2
यत्परि॑मिता अनुब्रू॒यात्।
परि॑मित॒मव॑ रुन्धीत।
अप॑रिमिता॒ अन्वा॑ह।
अप॑रिमितः सुव॒र्गो लो॒कः।
सु॒व॒र्गस्य॑ लो॒कस्य॒ सम॑ष्ट्यै।
स्तोक्या॑ जुहोति।
या ए॒व वर्ष्या॒ आपः॑।
ता अव॑ रुन्धे।
अ॒स्यां जु॑होति।
इ॒यं वा अ॒ग्निर्वै᳚श्वान॒रः॥२१॥\ip

%3.8.6.3
अ॒स्यामे॒वैनाः॒ प्रति॑\-ष्ठापयति।
उ॒वाच॑ ह प्र॒जा\-प॑तिः।
स्तोक्या॑सु॒ वा अ॒हम॑श्वमे॒धꣳ सꣴस्था॑पयामि।
तेन॒ ततः॒ सꣴस्थि॑तेन चरा॒मीति॑।
अ॒ग्नये॒ स्वाहेत्या॑ह।
अ॒ग्नय॑ ए॒वैनं॑ जुहोति।
सोमा॑य॒ स्वाहेत्या॑ह।
सोमा॑यै॒वैनं॑ जुहोति।
स॒वि॒त्रे स्वाहेत्या॑ह।
स॒वि॒त्र ए॒वैनं॑ जुहोति॥२२॥\ip

%3.8.6.4
सर॑स्वत्यै॒ स्वाहेत्या॑ह।
सर॑स्वत्या ए॒वैनं॑ जुहोति।
पू॒ष्णे स्वाहेत्या॑ह।
पू॒ष्ण ए॒वैनं॑ जुहोति।
बृह॒स्पत॑ये॒ स्वाहेत्या॑ह।
बृह॒स्पत॑य ए॒वैनं॑ जुहोति।
अ॒पां मोदा॑य॒ स्वाहेत्या॑ह।
अ॒द्भ्य ए॒वैनं॑ जुहोति।
वा॒यवे॒ स्वाहेत्या॑ह।
वा॒यव॑ ए॒वैनं॑ जुहोति॥२३॥\ip

%3.8.6.5
मि॒त्राय॒ स्वाहेत्या॑ह।
मि॒त्रायै॒वैनं॑ जुहोति।
वरु॑णाय॒ स्वाहेत्या॑ह।
वरु॑णायै॒वैनं॑ जुहोति।
ए॒ताभ्य॑ ए॒वैनं॑ दे॒वता᳚भ्यो जुहोति।
दश॑दश स॒म्पादं॑ जुहोति।
दशा᳚क्षरा वि॒राट्।
अन्नं॑ वि॒राट्।
वि॒राजै॒वान्नाद्य॒मव॑ रुन्धे।
प्र वा ए॒षो᳚\-ऽस्माल्लो॒काच्च्य॑वते।
यः परा॑ची॒राहु॑तीर्जु॒होति॑।
पुनः॑ पुनरभ्या॒वर्तं॑ जुहोति।
अ॒स्मिन्ने॒व लो॒के प्रति॑ तिष्ठति।
ए॒ताꣳ ह॒ वाव सो᳚\-ऽश्वमे॒धस्य॒ सꣴस्थि॑तिमुवा॒चास्क॑न्दाय।
अस्क॑न्न॒ꣳ॒ हि तत्।
यद्य॒ज्ञस्य॒ सꣴस्थि॑तस्य॒ स्कन्द॑ति॥२४॥\ip\anuvakamend[अ॒भिजि॑त्यै वैश्वान॒रः स॑वि॒त्र ए॒वैनं॑ जुहोति वा॒यव॑ ए॒वैनं॑ जुहोति च्यवते॒ षट् च॑]

%3.8.7.1
प्र॒जा\-प॑तये त्वा॒ जुष्टं॒ प्रोक्षा॒मीति॑ पु॒रस्ता᳚त्प्र॒त्यङ्तिष्ठ॒न्प्रोक्ष॑ति।
प्र॒जा\-प॑ति॒र्वै दे॒वाना॑मन्ना॒दो वी॒र्या॑वान्।
अ॒न्नाद्य॑मे॒वास्मि॑न्वी॒र्यं॑ दधाति।
तस्मा॒दश्वः॑ पशू॒नाम॑न्ना॒दो वी॒र्या॑वत्तमः।
इ॒न्द्रा॒ग्निभ्यां॒ त्वेति॑ दक्षिण॒तः।
इ॒न्द्रा॒ग्नी वै दे॒वाना॒मोजि॑ष्ठौ॒ बलि॑ष्ठौ।
ओज॑ ए॒वास्मि॒न्बलं॑ दधाति।
तस्मा॒दश्वः॑ पशू॒नामोजि॑ष्ठो॒ बलि॑ष्ठः।
वा॒यवे॒ त्वेति॑ प॒श्चात्।
वा॒युर्वै दे॒वाना॑मा॒शुः सा॑रसा॒रित॑मः॥२५॥\ip

%3.8.7.2
ज॒वमे॒वास्मि॑न्दधाति।
तस्मा॒दश्वः॑ पशू॒नामा॒शुः सा॑रसा॒रित॑मः।
विश्वे᳚भ्यस्त्वा दे॒वेभ्य॒ इत्यु॑त्तर॒तः।
विश्वे॒ वै दे॒वा दे॒वानां᳚ यश॒स्वित॑माः।
यश॑ ए॒वास्मि॑न्दधाति।
तस्मा॒दश्वः॑ पशू॒नां य॑श॒स्वित॑मः।
दे॒वेभ्य॒स्त्वेत्य॒धस्ता᳚त्।
दे॒वा वै दे॒वाना॒मप॑चिततमाः।
अप॑चितिमे॒वास्मि॑न्दधाति।
तस्मा॒दश्वः॑ पशू॒नामप॑चिततमः॥२६॥\ip

%3.8.7.3
सर्वे᳚भ्यस्त्वा दे॒वेभ्य॒ इत्यु॒परि॑ष्टात्।
सर्वे॒ वै दे॒वास्त्विषि॑मन्तो हर॒स्विनः॑।
त्विषि॑मे॒वास्मि॒न्॒ हरो॑ दधाति।
तस्मा॒दश्वः॑ पशू॒नां त्विषि॑मान्‌ हर॒स्वित॑मः।
दि॒वे त्वा॒\-ऽन्तरि॑क्षाय त्वा पृथि॒व्यै त्वेत्या॑ह।
ए॒भ्य ए॒वैनं॑ लो॒केभ्यः॒ प्रोक्ष॑ति।
स॒ते त्वा\-ऽस॑ते त्वा॒\-ऽद्भ्यस्त्वौष॑धीभ्यस्त्वा॒ विश्वे᳚भ्यस्त्वा भू॒तेभ्य॒ इत्या॑ह।
तस्मा॑दश्वमेधया॒जिन॒ꣳ॒ सर्वा॑णि भू॒तान्युप॑जीवन्ति।
ब्र॒ह्म॒वा॒दिनो॑ वदन्ति।
यत्प्रा॑जाप॒त्यो\-ऽश्वः॑।
अथ॒ कस्मा॑देनम॒न्याभ्यो॑ दे॒वता॒भ्योऽपि॒ प्रोक्ष॒तीति॑।
अश्वे॒ वै सर्वा॑ दे॒वता॑ अ॒न्वाय॑त्ताः।
तं यद्विश्वे᳚भ्यस्त्वा भू॒तेभ्य॒ इति॑ प्रो॒क्षति॑।
दे॒वता॑ ए॒वास्मि॑न्न॒न्वा या॑तयति।
तस्मा॒दश्वे॒ सर्वा॑ दे॒वता॑ अ॒न्वाय॑त्ताः॥२७॥\ip\anuvakamend[सा॒र॒सा॒रित॒मो\-ऽप॑चिततमः प्राजाप॒त्यो\-ऽश्वः॒ पञ्च॑ च]

%3.8.8.1
यथा॒ वै ह॒विषो॑ गृही॒तस्य॒ स्कन्द॑ति।
ए॒वं वा ए॒तदश्व॑स्य स्कन्दति।
यत्प्रोक्षि॑त॒मना॑लब्धमुथ्सृ॒जन्ति॑।
यद॑श्वचरि॒तानि॑ जु॒होति॑।
स॒र्व॒हुत॑मे॒वैनं॑ करो॒त्यस्क॑न्दाय।
अस्क॑न्न॒ꣳ॒ हि तत्।
यद्धु॒तस्य॒ स्कन्द॑ति।
ई॒ङ्का॒राय॒ स्वाहेङ्कृ॑ताय॒ स्वाहेत्या॑ह।
ए॒तानि॒ वा अ॑श्वचरि॒तानि॑।
च॒रि॒तैरे॒वैन॒ꣳ॒ सम॑र्धयति॥२८॥\ip

%3.8.8.2
तदा॑हुः।
अना॑हुतयो॒ वा अ॑श्वचरि॒तानि॑।
नैता हो॑त॒व्या॑ इति॑।
अथो॒ खल्वा॑हुः।
हो॒त॒व्या॑ ए॒व।
अत्र॒ वावैवं वि॒द्वान॑श्वमे॒धꣳ सꣴस्था॑पयति।
यद॑श्वचरि॒तानि॑ जु॒होति॑।
तस्मा᳚द्धोत॒व्या॑ इति॑।
ब॒हि॒र्धा वा ए॑नमे॒तदा॒यत॑नाद्दधाति।
भ्रातृ॑व्यमस्मै जनयति॥२९॥\ip

%3.8.8.3
यस्या॑नायत॒ने᳚\-ऽन्यत्रा॒ग्नेराहु॑तीर्जु॒होति॑।
सा॒वि॒त्रि॒या इष्ट्याः᳚ पु॒रस्ता᳚थ्स्विष्ट॒कृतः॑।
आ॒ह॒व॒नीये᳚\-ऽश्वचरि॒तानि॑ जुहोति।
आ॒यत॑न ए॒वास्याऽऽहु॑तीर्जुहोति।
नास्मै॒ भ्रातृ॑व्यं जनयति।
तदा॑हुः।
य॒ज्ञ॒\-मु॒खेय॑ज्ञमुखे होत॒व्याः᳚।
य॒ज्ञस्य॒ कॢप्त्यै᳚।
सु॒व॒र्गस्य॑ लो॒कस्यानु॑ख्यात्या॒ इति॑।
अथो॒ खल्वा॑हुः॥३०॥\ip

%3.8.8.4
यद्य॑ज्ञमु॒खेय॑ज्ञमुखे जुहु॒यात्।
प॒शुभि॒र्यज॑मानं॒ व्य॑र्धयेत्।
अव॑ सुव॒र्गाल्लो॒कात्प॑द्येत।
पापी॑यान्थ्स्या॒दिति॑।
स॒कृदे॒व हो॑त॒व्याः᳚।
न यज॑मानं प॒शुभि॒र्व्य॑र्धयति।
अ॒भि सु॑व॒र्गं लो॒कं ज॑यति।
न पापी॑यान्भवति।
अ॒ष्टाच॑त्वारिꣳशतमश्वरू॒पाणि॑ जुहोति।
अ॒ष्टाच॑त्वारिꣳशदक्षरा॒ जग॑ती।
जाग॒तो\-ऽश्वः॑ प्राजाप॒त्यः समृ॑द्ध्यै।
एक॒मति॑रिक्तं जुहोति।
तस्मा॒देकः॑ प्र॒जास्वर्धु॑कः॥३१॥\ip\anuvakamend[अ॒र्ध॒य॒ति॒ ज॒न॒य॒ति॒ खल्वा॑हु॒र्जग॑ती॒ त्रीणि॑ च]

%3.8.9.1
वि॒भूर्मा॒त्रा प्र॒भूः पि॒त्रेत्या॑ह।
इ॒यं वै मा॒ता।
अ॒सौ पि॒ता।
आ॒भ्यामे॒वैनं॒ परि॑ददाति।
अश्वो॑ऽसि॒ हयो॒\-ऽसीत्या॑ह।
शास्त्ये॒वैन॑मे॒तत्।
तस्मा᳚च्छि॒ष्टाः प्र॒जा जा॑यन्ते।
अत्यो॒\-ऽसीत्या॑ह।
तस्मा॒दश्वः॒ सर्वा᳚न्प॒शूनत्ये॑ति।
तस्मा॒दश्वः॒ सर्वे॑षां पशू॒नाꣴ श्रैष्ठ्यं॑ गच्छति॥३२॥\ip

%3.8.9.2
प्र यशः॒ श्रैष्ठ्य॑माप्नोति।
य ए॒वं वेद॑।
नरो॒ऽस्यर्वा॑ऽसि॒ सप्ति॑रसि वा॒ज्य॑सीत्या॑ह।
रू॒पमे॒वास्यै॒तन्म॑हि॒मानं॒ व्याच॑ष्टे।
ययु॒र्नामा॒सीत्या॑ह।
ए॒तद्वा अश्व॑स्य प्रि॒यं ना॑म॒धेयम्᳚।
प्रि॒येणै॒वैनं॑ नाम॒धेये॑ना॒भि व॑दति।
तस्मा॒दप्या॑मि॒त्रौ स॒ङ्गत्य॑।
नाम्ना॒ चेद्‌ध्वये॑ते।
मि॒त्रमे॒व भ॑वतः॥३३॥\ip

%3.8.9.3
आ॒दि॒त्यानां॒ पत्वा\-ऽन्वि॒हीत्या॑ह।
आ॒दि॒त्याने॒वैनं॑ गमयति।
अ॒ग्नये॒ स्वाहा॒ स्वाहे᳚न्द्रा॒ग्निभ्या॒मिति॑ पूर्वहो॒मां जु॑होति।
पूर्व॑ ए॒व द्वि॒षन्तं॒ भ्रातृ॑व्य॒मति॑ क्रामति।
भूर॑सि भु॒वे त्वा॒ भव्या॑य त्वा भविष्य॒ते त्वेत्युथ्सृ॑जति सर्व॒त्वाय॑।
देवा॑ आशापाला ए॒तं दे॒वेभ्यो\-ऽश्वं॒ मेधा॑य॒ प्रोक्षि॑तं गोपाय॒तेत्या॑ह।
श॒तं वै तल्प्या॑ राजपु॒त्रा दे॒वा आ॑शापा॒लाः।
तेभ्य॑ ए॒वैनं॒ परि॑ ददाति।
ई॒श्व॒रो वा अश्वः॒ प्रमु॑क्तः॒ परां᳚ परा॒वतं॒ गन्तोः᳚।
इ॒ह धृतिः॒ स्वाहे॒ह विधृ॑तिः॒ स्वाहे॒ह रन्तिः॒ स्वाहे॒ह रम॑तिः॒ स्वाहेति॑ चतृ॒षु प॒थ्सु जु॑होति॥३४॥\ip

%3.8.9.4
ए॒ता वा अश्व॑स्य॒ बन्ध॑नम्।
ताभि॑रे॒वैनं॑ बध्नाति।
तस्मा॒दश्वः॒ प्रमु॑क्तो॒ बन्ध॑न॒मा ग॑च्छति।
तस्मा॒दश्वः॒ प्रमु॑क्तो॒ बन्ध॑नं॒ न ज॑हाति।
रा॒ष्ट्रं वा अ॑श्वमे॒धः।
रा॒ष्ट्रे खलु॒ वा ए॒ते व्याय॑च्छन्ते।
येऽश्वं॒ मेध्य॒ꣳ॒ रक्ष॑न्ति।
तेषां॒ य उ॒दृचं॒ गच्छ॑न्ति।
रा॒ष्ट्रादे॒व ते रा॒ष्ट्रं ग॑च्छन्ति।
अथ॒ य उ॒दृचं॒ न गच्छ॑न्ति॥३५॥\ip

%3.8.9.5
रा॒ष्ट्रादे॒व ते व्यव॑च्छिद्यन्ते।
परा॒ वा ए॒ष सि॑च्यते।
यो॑ऽब॒लो᳚\-ऽश्वमे॒धेन॒ यज॑ते।
यद॒मित्रा॒ अश्वं॑ वि॒न्देरन्॑।
ह॒न्येता᳚स्य य॒ज्ञः।
च॒तुः॒ श॒ता र॑क्षन्ति।
य॒ज्ञस्याघा॑ताय।
अथा॒न्यमा॒नीय॒ प्रोक्षे॑युः।
सैव ततः॒ प्राय॑श्चित्तिः॥३६॥\ip\anuvakamend[ग॒च्छ॒ति॒ भ॒व॒तः॒ प॒थ्सु जु॑होति॒ न गच्छ॑न्ति॒ नव॑ च]

%3.8.10.1
प्र॒जा\-प॑तिरकामयताश्वमे॒धेन॑ यजे॒येति॑।
स तपो॑\-ऽतप्यत।
तस्य॑ तेपा॒नस्य॑।
स॒प्तात्मनो॑ दे॒वता॒ उद॑क्रामन्।
सा दी॒क्षा\-ऽभ॑वत्।
स ए॒तानि॑ वैश्वदे॒वान्य॑पश्यत्।
तान्य॑जुहोत्।
तैर्वै स दी॒क्षामवा॑रुन्ध।
यद्वै᳚श्वदे॒वानि॑ जु॒होति॑।
दी॒क्षामे॒व तैर्यज॑मा॒नोऽव॑ रुन्धे॥३७॥\ip

%3.8.10.2
स॒प्त जु॑होति।
स॒प्त हि ता दे॒वता॑ उ॒दक्रा॑मन्।
अ॒न्व॒हं जु॑होति।
अ॒न्व॒हमे॒व दी॒क्षामव॑ रुन्धे।
त्रीणि॑ वैश्वदे॒वानि॑ जुहोति।
च॒त्वार्यौ᳚द्ग्रह॒णानि॑।
स॒प्त सम्प॑द्यन्ते।
स॒प्त वै शी॑र्‌\mbox{}ष॒ण्याः᳚ प्रा॒णाः।
प्रा॒णा दी॒क्षा।
प्रा॒णैरे॒व प्रा॒णान्दी॒क्षामव॑ रुन्धे॥३८॥\ip

%3.8.10.3
एक॑विꣳशतिं वैश्वदे॒वानि॑ जुहोति।
एक॑विꣳशति॒र्वै दे॑वलो॒काः।
द्वाद॑श॒ मासाः॒ पञ्च॒र्तवः॑।
त्रय॑ इ॒मे लो॒काः।
अ॒सावा॑दि॒त्य ए॑कवि॒ꣳ॒शः।
ए॒ष सु॑व॒र्गो लो॒कः।
तद्दैव्यं॑ क्ष॒त्रम्।
सा श्रीः।
तद्ब्र॒ध्नस्य॑ वि॒ष्टपम्᳚।
तथ्स्वारा᳚ज्यमुच्यते॥३९॥\ip

%3.8.10.4
त्रि॒ꣳ॒शत॑मौद्ग्रह॒णानि॑ जुहोति।
त्रि॒ꣳ॒शद॑क्षरा वि॒राट्।
अन्नं॑ वि॒राट्।
वि॒राजै॒वान्नाद्य॒मव॑ रुन्धे।
त्रे॒धा वि॒भज्य॑ दे॒वतां᳚ जुहोति।
त्र्या॑वृतो॒ वै दे॒वाः।
त्र्या॑वृत इ॒मे लो॒काः।
ए॒षां लो॒काना॒माप्त्यै᳚।
ए॒षां लो॒कानां॒ कॢप्त्यै᳚।
अप॒ वा ए॒तस्मा᳚त्प्रा॒णाः क्रा॑मन्ति॥४०॥\ip

%3.8.10.5
यो दी॒क्षाम॑तिरे॒चय॑ति।
स॒प्ता॒हं प्रच॑रन्ति।
स॒प्त वै शी॑र्\mbox{}ष॒ण्याः᳚ प्रा॒णाः।
प्रा॒णा दी॒क्षा।
प्रा॒णैरे॒व प्रा॒णान्दी॒क्षामव॑ रुन्धे।
पू॒र्णा॒हु॒तिमु॑त्त॒मां जु॑होति।
सर्वं॒ वै पू᳚र्णाहु॒तिः।
सर्व॑\-मे॒वा\-प्नो॑ति।
अथो॑ इ॒यं वै पू᳚र्णाहु॒तिः।
अ॒स्यामे॒व प्रति॑ तिष्ठति॥४१॥\ip\anuvakamend[रु॒न्धे॒ प्रा॒णान्दी॒क्षामव॑ रुन्ध उच्यते क्रामन्ति तिष्ठति]

%3.8.11.1
प्र॒जा\-प॑तिरश्वमे॒धम॑\-सृजत।
तꣳ सृ॒ष्टं न किञ्च॒नोद॑यच्छत्।
तं वै᳚श्वदे॒वान्ये॒वोद॑यच्छन्।
यद्वै᳚श्वदे॒वानि॑ जु॒होति॑।
य॒ज्ञस्योद्य॑त्यै।
स्वाहा॒\-ऽऽधिमाधी॑ताय॒ स्वाहा᳚।
स्वाहा\-ऽधी॑तं॒ मन॑से॒ स्वाहा᳚।
स्वाहा॒ मनः॑ प्र॒जा\-प॑तये॒ स्वाहा᳚।
काय॒ स्वाहा॒ कस्मै॒ स्वाहा॑ कत॒मस्मै॒ स्वाहेति॑ प्राजाप॒त्ये मुख्ये॑ भवतः।
प्र॒जा\-प॑तिमुखाभिरे॒वैनं॑ दे॒वता॑भि॒रुद्य॑च्छते॥४२॥\ip

%3.8.11.2
अदि॑त्यै॒ स्वाहा\-ऽदि॑त्यै म॒ह्यै᳚ स्वाहा\-ऽदि॑त्यै सुमृडी॒कायै॒ स्वाहेत्या॑ह।
इ॒यं वा अदि॑तिः।
अ॒स्या ए॒वैनं॑ प्रति॒ष्ठायोद्य॑च्छते।
सर॑स्वत्यै॒ स्वाहा॒ सर॑स्वत्यै बृह॒त्यै᳚ स्वाहा॒ सर॑स्वत्यै पाव॒कायै॒ स्वाहेत्या॑ह।
वाग्वै सर॑स्वती।
वा॒चैवैन॒मुद्य॑च्छते।
पू॒ष्णे स्वाहा॑ पू॒ष्णे प्र॑प॒थ्या॑य॒ स्वाहा॑ पू॒ष्णे न॒रन्धि॑षाय॒ स्वाहेत्या॑ह।
प॒शवो॒ वै पू॒षा।
प॒शुभि॑रे॒वैन॒मुद्य॑च्छते।
त्वष्ट्रे॒ स्वाहा॒ त्वष्ट्रे॑ तु॒रीपा॑य॒ स्वाहा॒ त्वष्ट्रे॑ पुरु॒रूपा॑य॒ स्वाहेत्या॑ह।
त्वष्टा॒ वै प॑शू॒नां मि॑थु॒नानाꣳ॑ रूप॒कृत्।
रू॒पमे॒व प॒शुषु॑ दधाति।
अथो॑ रू॒पैरे॒वैन॒मुद्य॑च्छते।
विष्ण॑वे॒ स्वाहा॒ विष्ण॑वे निखुर्य॒पाय॒ स्वाहा॒ विष्ण॑वे निभूय॒पाय॒ स्वाहेत्या॑ह।
य॒ज्ञो वै विष्णुः॑।
य॒ज्ञायै॒वैन॒मुद्य॑च्छते।
पू॒र्णा॒हु॒तिमु॑त्त॒मां जु॑होति।
प्रत्युत्त॑ब्ध्यै सय॒त्वाय॑॥४३॥\ip\anuvakamend[य॒च्छ॒ते॒ पु॒रु॒रूपा॑य॒ स्वाहेत्या॑हा॒ष्टौ च॑]

%3.8.12.1
सा॒वि॒त्रम॒ष्टा\-क॑पालं प्रा॒तर्निर्व॑पति।
अ॒ष्टाक्ष॑रा गाय॒त्री।
गा॒य॒त्रं प्रा॑तः सव॒नम्।
प्रा॒तः॒ स॒व॒नादे॒वैनं॑ गायत्रि॒याश्छन्द॒सो\-ऽधि॒ निर्मि॑मीते।
अथो᳚ प्रातः सव॒नमे॒व तेना᳚ऽऽप्नोति।
गा॒य॒त्रीं छन्दः॑।
स॒वि॒त्रे प्र॑सवि॒त्र एका॑\-दश\-कपालं म॒ध्यन्दि॑ने।
एका॑दशाक्षरा त्रि॒ष्टुप्।
त्रैष्टु॑भं॒ माध्यं॑ दिन॒ꣳ॒ सव॑नम्।
माध्यं॑ दिनादे॒वैन॒ꣳ॒ सव॑नात्त्रि॒ष्टुभ॒श्छन्द॒सोऽधि॒ निर्मि॑मीते॥४४॥\ip

%3.8.12.2
अथो॒ माध्यं॑ दिनमे॒व सव॑नं॒ तेना᳚ऽऽप्नोति।
त्रि॒ष्टुभं॒ छन्दः॑।
स॒वि॒त्र आ॑सवि॒त्रे द्वाद॑शकपालमपरा॒ह्णे।
द्वाद॑शाक्षरा॒ जग॑ती।
जाग॑तं तृतीयसव॒नम्।
तृ॒ती॒य॒स॒व॒नादे॒वैनं॒ जग॑त्या॒श्छन्द॒सोऽधि॒ निर्मि॑मीते।
अथो॑ तृतीयसव॒नमे॒व तेना᳚ऽऽप्नोति।
जग॑तीं॒ छन्दः॑।
ई॒श्व॒रो वा अश्वः॒ प्रमु॑क्तः॒ परां᳚ परा॒वतं॒ गन्तोः᳚।
इ॒ह धृतिः॒ स्वाहे॒ह विधृ॑तिः॒ स्वाहे॒ह रन्तिः॒ स्वाहे॒ह रम॑तिः॒ स्वाहेति॒ चत॑स्र॒ आहु॑तीर्जुहोति॥४५॥\ip

%3.8.12.3
चत॑स्रो॒ दिशः॑।
दि॒ग्भिरे॒वैनं॒ परि॑गृह्णाति।
आश्व॑त्थो व्र॒जो भ॑वति।
प्र॒जा\-प॑तिर्दे॒वेभ्यो॒ निला॑यत।
अश्वो॑ रू॒पं कृ॒त्वा।
सो᳚ऽश्व॒त्थे सं॑वथ्स॒रम॑तिष्ठत्।
तद॑श्व॒त्थस्या᳚श्वत्थ॒त्वम्।
यदाश्व॑त्थो व्र॒जो भव॑ति।
स्व ए॒वैनं॒ योनौ॒ प्रति॑\-ष्ठापयति॥४६॥\ip\anuvakamend[त्रि॒ष्टुभ॒श्छन्द॒सोऽधि॒ निर्मि॑मीते जुहोति॒ नव॑ च]

%3.8.13.1
आ ब्रह्म॑न्ब्राह्म॒णो ब्र॑ह्म\-वर्च॒सी जा॑यता॒मित्या॑ह।
ब्रा॒ह्म॒ण ए॒व ब्र॑ह्म\-वर्च॒सं द॑धाति।
तस्मा᳚त्पु॒रा ब्रा᳚ह्म॒णो ब्र॑ह्म\-वर्च॒स्य॑जायत।
आऽस्मिन्रा॒ष्ट्रे रा॑ज॒न्य॑ इष॒व्यः॑ शूरो॑ महार॒थो जा॑यता॒मित्या॑ह।
रा॒ज॒न्य॑ ए॒व शौ॒र्यं म॑हि॒मानं॑ दधाति।
तस्मा᳚त्पु॒रा रा॑ज॒न्य॑ इष॒व्यः॑ शूरो॑ महार॒थो॑\-ऽजायत।
दोग्ध्री॑ धे॒नुरित्या॑ह।
धे॒न्वामे॒व पयो॑ दधाति।
तस्मा᳚त्पु॒रा दोग्ध्री॑ धे॒नुर॑जायत।
वोढा॑\-ऽन॒ड्वानित्या॑ह॥४७॥\ip

%3.8.13.2
अ॒न॒डुह्ये॒व वी॒र्यं॑ दधाति।
तस्मा᳚त्पु॒रा वोढा॑\-ऽन॒ड्वान॑जायत।
आ॒शुः सप्ति॒रित्या॑ह।
अश्व॑ ए॒व ज॒वं द॑धाति।
तस्मा᳚त्पु॒रा\-ऽऽशुरश्वो॑\-ऽजायत।
पुर॑न्धि॒र्योषेत्या॑ह।
यो॒षित्ये॒व रू॒पं द॑धाति।
तस्मा॒थ्स्त्री यु॑व॒तिः प्रि॒या भावु॑का।
जि॒ष्णू र॑थे॒ष्ठा इत्या॑ह।
आ ह॒ वै तत्र॑ जि॒ष्णू र॑थे॒ष्ठा जा॑यते॥४८॥\ip

%3.8.13.3
यत्रै॒तेन॑ य॒ज्ञेन॒ यज॑न्ते।
स॒भेयो॒ युवेत्या॑ह।
यो वै पू᳚र्ववय॒सी।
स स॒भेयो॒ युवा᳚।
तस्मा॒द्युवा॒ पुमा᳚न्प्रि॒यो भावु॑कः।
आऽस्य यज॑मानस्य वी॒रो जा॑यता॒मित्या॑ह।
आ ह॒ वै तत्र॒ यज॑मानस्य वी॒रो जा॑यते।
यत्रै॒तेन॑ य॒ज्ञेन॒ यज॑न्ते।
नि॒का॒मेनि॑कामे नः प॒र्जन्यो॑ वर्\mbox{}ष॒त्वित्या॑ह।
नि॒का॒मेनि॑कामे ह॒ वै तत्र॑ प॒र्जन्यो॑ वर्\mbox{}षति।
यत्रै॒तेन॑ य॒ज्ञेन॒ यज॑न्ते।
फ॒लिन्यो॑ न॒ ओष॑धयः पच्यन्ता॒मित्या॑ह।
फ॒लिन्यो॑ ह॒ वै तत्रौष॑धयः पच्यन्ते।
यत्रै॒तेन॑ य॒ज्ञेन॒ यज॑न्ते।
यो॒ग॒क्षे॒मो नः॑ कल्पता॒मित्या॑ह।
कल्प॑ते ह॒ वै तत्र॑ प्र॒जाभ्यो॑ योगक्षे॒मः।
यत्रै॒तेन॑ य॒ज्ञेन॒ यज॑न्ते॥४९॥\ip\anuvakamend[अ॒न॒ड्वानित्या॑ह जायते वर्‌\mbox{}षति स॒प्त च॑]

%3.8.14.1
प्र॒जा\-प॑तिर्दे॒वेभ्यो॑ य॒ज्ञान्व्यादि॑शत्।
स आ॒त्मन्न॑श्वमे॒धम॑धत्त।
तं दे॒वा अ॑ब्रुवन्।
ए॒ष वाव य॒ज्ञः।
यद॑श्वमे॒धः।
अप्ये॒व नोऽत्रा॒स्त्विति॑।
तेभ्य॑ ए॒तान॑न्नहो॒मान्प्राय॑च्छत्।
तान॑जुहोत्।
तैर्वै स दे॒वान॑प्रीणात्।
यद॑न्नहो॒मां जु॒होति॑॥५०॥\ip

%3.8.14.2
दे॒वाने॒व तैर्यज॑मानः प्रीणाति।
आज्ये॑न जुहोति।
अ॒ग्नेर्वा ए॒तद्रू॒पम्।
यदाज्यम्᳚।
यदाज्ये॑न जु॒होति॑।
अ॒ग्निमे॒व तत्प्री॑णाति।
मधु॑ना जुहोति।
म॒ह॒त्यै वा ए॒तद्दे॒वता॑यै रू॒पम्।
यन्मधु॑।
यन्मधु॑ना जु॒होति॑॥५१॥\ip

%3.8.14.3
म॒ह॒तीमे॒व तद्दे॒वतां᳚ प्रीणाति।
त॒ण्डु॒लैर्जु॑होति।
वसू॑नां॒ वा ए॒तद्रू॒पम्।
यत्त॑ण्डु॒लाः।
यत्त॑ण्डु॒लैर्जु॒होति॑।
वसू॑ने॒व तत्प्री॑णाति।
पृथु॑कैर्जुहोति।
रु॒द्राणां॒ वा ए॒तद्रू॒पम्।
यत्पृथु॑काः।
यत्पृथु॑कैर्जु॒होति॑॥५२॥\ip

%3.8.14.4
रु॒द्राने॒व तत्प्री॑णाति।
ला॒जैर्जु॑होति।
आ॒दि॒त्यानां॒ वा ए॒तद्रू॒पम्।
यल्ला॒जाः।
यल्ला॒जैर्जु॒होति॑।
आ॒दि॒त्याने॒व तत्प्री॑णाति।
क॒रम्बै᳚र्जुहोति।
विश्वे॑षां॒ वा ए॒तद्दे॒वानाꣳ॑ रू॒पम्।
यत्क॒रम्बाः᳚।
यत्क॒रम्बै᳚र्जु॒होति॑॥५३॥\ip

%3.8.14.5
विश्वा॑ने॒व तद्दे॒वान्प्री॑णाति।
धा॒नाभि॑र्जुहोति।
नक्ष॑त्राणां॒ वा ए॒तद्रू॒पम्।
यद्धा॒नाः।
यद्धा॒नाभि॑र्जु॒होति॑।
नक्ष॑त्राण्ये॒व तत्प्री॑णाति।
सक्तु॑भिर्जुहोति।
प्र॒जा\-प॑ते॒र्वा ए॒तद्रू॒पम्।
यथ्सक्त॑वः।
यथ्सक्तु॑भिर्जु॒होति॑॥५४॥\ip

%3.8.14.6
प्र॒जा॑पतिमे॒व तत्प्री॑णाति।
म॒सूस्यै᳚र्जुहोति।
सर्वा॑सां॒ वा ए॒तद्दे॒वता॑नाꣳ रू॒पम्।
यन्म॒सूस्या॑नि।
यन्म॒सूस्यै᳚र्जु॒होति॑।
सर्वा॑ ए॒व तद्दे॒वताः᳚ प्रीणाति।
प्रि॒य॒ङ्गु॒त॒ण्डु॒लैर्जु॑होति।
प्रि॒याङ्गा॑ ह॒ वै नामै॒ते।
ए॒तैर्वै दे॒वा अश्व॒स्याङ्गा॑नि॒ सम॑दधुः।
यत्प्रि॑यङ्गुतण्डु॒लैर्जु॒होति॑।
अश्व॑स्यै॒वाङ्गा॑नि॒ सन्द॑धाति।
दशान्ना॑नि जुहोति।
दशा᳚क्षरा वि॒राट्।
वि॒राट्कृ॒थ्स्नस्या॒न्नाद्य॒स्या\-व॑\-रुद्ध्यै॥५५॥\ip\anuvakamend[जु॒होति॒ मधु॑ना जु॒होति॒ पृथु॑कैर्जु॒होति॑ क॒रम्बै᳚र्जु॒होति॒ सक्तु॑भिर्जु॒होति॑ प्रियङ्गुतण्डु॒लैर्जु॒होति॑ च॒त्वारि॑ च (अ॒न्नहो॒मानाऽऽज्ये॑ना॒ग्नेर्मधु॑ना तण़्डु॒लैः पृथु॑कैर्ला॒जैः क॒रम्बै᳚र्धा॒नाभिः॒ सक्तु॑भिर्म॒सूस्यैः᳚ प्रियङ्गुतण्डु॒लैर्द॒शान्ना॑नि॒ द्वाद॑श।)]

%3.8.15.1
प्र॒जा\-प॑तिरश्वमे॒धम॑\-सृजत।
तꣳ सृ॒ष्टꣳ रक्षाꣴ॑स्यजिघाꣳसन्।
स ए॒तान्प्र॒जा\-प॑तिर्न॒क्तꣳ हो॒मान॑पश्यत्।
तान॑जुहोत्।
तैर्वै स य॒ज्ञाद्रक्षा॒ꣴ॒स्यपा॑हन्।
यन्न॑क्तꣳ हो॒मां जु॒होति॑।
य॒ज्ञादे॒व तैर्यज॑मानो॒ रक्षा॒ꣴ॒स्यप॑ हन्ति।
आज्ये॑न जुहोति।
वज्रो॒ वा आज्यम्᳚।
वज्रे॑णै॒व य॒ज्ञाद्रक्षा॒ꣴ॒स्यप॑ हन्ति॥५६॥\ip

%3.8.15.2
आज्य॑स्य प्रति॒पदं॑ करोति।
प्रा॒णो वा आज्यम्᳚।
मु॒ख॒त ए॒वास्य॑ प्रा॒णं द॑धाति।
अ॒न्न॒हो॒माञ्जु॑होति।
शरी॑रवदे॒वाव॑ रुन्धे।
व्य॒त्यासं॑ जुहोति।
उ॒भय॒स्या\-व॑\-रुद्ध्यै।
नक्तं॑ जुहोति।
रक्ष॑सा॒मप॑हत्यै।
आज्ये॑नान्त॒तो जु॑होति॥५७॥\ip

%3.8.15.3
प्रा॒णो वा आज्यम्᳚।
उ॒भ॒यत॑ ए॒वास्य॑ प्रा॒णं द॑धाति।
पु॒रस्ता᳚च्चो॒परि॑ष्टाच्च।
एक॑स्मै॒ स्वाहेत्या॑ह।
अ॒स्मिन्ने॒व लो॒के प्रति॑ तिष्ठति।
द्वाभ्या॒ꣴ॒ स्वाहेत्या॑ह।
अ॒मुष्मि॑न्ने॒व लो॒के प्रति॑ तिष्ठति।
उ॒भयो॑रे॒व लो॒कयोः॒ प्रति॑ तिष्ठति।
अ॒स्मिꣴश्चा॒मुष्मिꣴ॑श्च।
श॒ताय॒ स्वाहेत्या॑ह।
श॒तायु॒र्वै पुरु॑षः श॒तवी᳚र्यः।
आयु॑रे॒व वी॒र्य॑मव॑ रुन्धे।
स॒हस्रा॑य॒ स्वाहेत्या॑ह।
आयु॒र्वै स॒हस्रम्᳚।
आयु॑रे॒वाव॑ रुन्धे।
सर्व॑स्मै॒ स्वाहेत्या॑ह।
अप॑रिमितमे॒वाव॑ रुन्धे॥५८॥\ip\anuvakamend[ए॒व य॒ज्ञाद्रक्षा॒ꣴ॒स्यप॑हन्त्यन्त॒तो जु॑होति श॒ताय॒ स्वाहेत्या॑ह स॒प्त च॑]

%3.8.16.1
प्र॒जा\-प॑तिं॒ वा ए॒ष ई᳚फ्स॒तीत्या॑हुः।
यो᳚ऽश्वमे॒धेन॒ यज॑त॒ इति॑।
अथो॑ आहुः।
सर्वा॑णि भू॒तानीति॑।
एक॑स्मै॒ स्वाहेत्या॑ह।
प्र॒जा\-प॑ति॒र्वा एकः॑।
तमे॒वाऽऽप्नो॑ति।
एक॑स्मै॒ स्वाहा॒ द्वाभ्या॒ꣴ॒ स्वाहेत्य॑भिपू॒र्वमाहु॑तीर्जुहोति।
अ॒भि॒पू॒र्वमे॒व सु॑व॒र्गं लो॒कमे॑ति।
ए॒को॒त्त॒रं जु॑होति॥५९॥\ip

%3.8.16.2
ए॒क॒वदे॒व सु॑व॒र्गं लो॒कमे॑ति।
सन्त॑तं जुहोति।
सु॒व॒र्गस्य॑ लो॒कस्य॒ सन्त॑त्यै।
श॒ताय॒ स्वाहेत्या॑ह।
श॒तायु॒र्वै पुरु॑षः श॒तवी᳚र्यः।
आयु॑रे॒व वी॒र्य॑मव॑ रुन्धे।
स॒हस्रा॑य॒ स्वाहेत्या॑ह।
आयु॒र्वै स॒हस्रम्᳚।
आयु॑रे॒वाव॑ रुन्धे।
अ॒युता॑य॒ स्वाहा॑ नि॒युता॑य॒ स्वाहा᳚ प्र॒युता॑य॒ स्वाहेत्या॑ह॥६०॥\ip

%3.8.16.3
त्रय॑ इ॒मे लो॒काः।
इ॒माने॒व लो॒कानव॑ रुन्धे।
अर्बु॑दाय॒ स्वाहेत्या॑ह।
वाग्वा अर्बु॑दम्।
वाच॑मे॒वाव॑ रुन्धे।
न्य॑र्बुदाय॒ स्वाहेत्या॑ह।
यो वै वा॒चो भू॒मा।
तन्न्य॑र्बुदम्।
वा॒च ए॒व भू॒मान॒मव॑ रुन्धे।
स॒मु॒द्राय॒ स्वाहेत्या॑ह॥६१॥\ip

%3.8.16.4
स॒मु॒द्रमे॒वाऽऽप्नो॑ति।
मध्या॑य॒ स्वाहेत्या॑ह।
मध्य॑मे॒वाऽऽप्नो॑ति।
अन्ता॑य॒ स्वाहेत्या॑ह।
अन्त॑मे॒वाऽऽप्नो॑ति।
प॒रा॒र्धाय॒ स्वाहेत्या॑ह।
प॒रा॒र्धमे॒वाऽऽप्नो॑ति।
उ॒षसे॒ स्वाहा॒ व्यु॑ष्ट्यै॒ स्वाहेत्या॑ह।
रात्रि॒र्वा उ॒षाः।
अह॒र्व्यु॑ष्टिः।
अ॒हो॒रा॒त्रे ए॒वाव॑ रुन्धे।
अथो॑ अहोरा॒त्रयो॑रे॒व प्रति॑ तिष्ठति।
ता यदु॒भयी॒र्दिवा॑ वा॒ नक्तं॑ वा जुहु॒यात्।
अ॒हो॒रा॒त्रे मो॑हयेत्।
उ॒षसे॒ स्वाहा॒ व्यु॑ष्ट्यै॒ स्वाहो॑देष्य॒ते स्वाहो᳚द्य॒ते स्वाहेत्यनु॑दिते जुहोति।
उदि॑ताय॒ स्वाहा॑ सुव॒र्गाय॒ स्वाहा॑ लो॒काय॒ स्वाहेत्युदि॑ते जुहोति।
अ॒हो॒रा॒त्रयो॒रव्य॑तिमोहाय॥६२॥\ip\anuvakamend[ए॒को॒त्त॒रं जु॑होति प्र॒युता॑य॒ स्वाहेत्या॑ह समु॒द्राय॒ स्वाहेत्या॒हाह॒र्व्यु॑ष्टिः स॒प्त च॑]

%3.8.17.1
वि॒भूर्मा॒त्रा प्र॒भूः पि॒त्रेत्य॑श्वना॒मानि॑ जुहोति।
उ॒भयो॑रे॒वैनं॑ लो॒कयो᳚र्नाम॒धेयं॑ गमयति।
आय॑नाय॒ स्वाहा॒ प्राय॑णाय॒ स्वाहेत्यु॑द्द्रा॒वाञ्जु॑होति।
सर्व॑मे॒वैन॒मस्क॑न्नꣳ सुव॒र्गं लो॒कं ग॑मयति।
अ॒ग्नये॒ स्वाहा॒ सोमा॑य॒ स्वाहेति॑ पूर्वहो॒माञ्जु॑होति।
पूर्व॑ ए॒व द्वि॒षन्तं॒ भ्रातृ॑व्य॒मति॑ क्रामति।
पृ॒थि॒व्यै स्वाहा॒\-ऽन्तरि॑क्षाय॒ स्वाहेत्या॑ह।
य॒था॒\-य॒जु\-रे॒वै\-तत्।
अ॒ग्नये॒ स्वाहा॒ सोमा॑य॒ स्वाहेति॑ पूर्वदी॒क्षा जु॑होति।
पूर्व॑ ए॒व द्वि॒षन्तं॒ भ्रातृ॑व्य॒मति॑ क्रामति॥६३॥\ip

%3.8.17.2
पृ॒थि॒व्यै स्वाहा॒\-ऽन्तरि॑क्षाय॒ स्वाहेत्ये॑कवि॒ꣳ॒शिनीं᳚ दी॒क्षां जु॑होति।
एक॑विꣳशति॒र्वै दे॑वलो॒काः।
द्वाद॑श॒ मासाः॒ पञ्च॒र्तवः॑।
त्रय॑ इ॒मे लो॒काः।
अ॒सावा॑दि॒त्य ए॑कवि॒ꣳ॒शः।
ए॒ष सु॑व॒र्गो लो॒कः।
सु॒व॒र्गस्य॑ लो॒कस्य॒ सम॑ष्ट्यै।
भुवो॑ दे॒वानां॒ कर्म॒णेत्यृ॑तुदी॒क्षा जु॑होति।
ऋ॒तूने॒वास्मै॑ कल्पयति।
अ॒ग्नये॒ स्वाहा॑ वा॒यवे॒ स्वाहेति॑ जुहो॒त्यन॑न्तरित्यै॥६४॥\ip

%3.8.17.3
अ॒र्वाङ्य॒ज्ञः सङ्क्रा॑म॒त्वित्याप्ती᳚र्जुहोति।
सु॒व॒र्गस्य॑ लो॒कस्याप्त्यै᳚।
भू॒तं भव्यं॑ भवि॒ष्यदिति॒ पर्या᳚प्तीर्जुहोति।
सु॒व॒र्गस्य॑ लो॒कस्य॒ पर्या᳚प्त्यै।
आ मे॑ गृ॒हा भ॑व॒न्त्वित्या॒भूर्जु॑होति।
सु॒व॒र्गस्य॑ लो॒कस्याभू᳚त्यै।
अ॒ग्निना॒ तपो\-ऽन्व॑भव॒दित्य॑नु॒भूर्जु॑होति।
सु॒व॒र्गस्य॑ लो॒कस्यानु॑भूत्यै।
स्वाहा॒\-ऽऽधिमाधी॑ताय॒ स्वाहेति॒ सम॑स्तानि वैश्वदे॒वानि॑ जुहोति।
सम॑स्तमे॒व द्वि॒षन्तं॒ भ्रातृ॑व्य॒मति॑ क्रामति॥६५॥\ip

%3.8.17.4
द॒द्भ्यः स्वाहा॒ हनू᳚भ्या॒ꣴ॒ स्वाहेत्य॑ङ्गहो॒माञ्जु॑होति।
अङ्गे॑अङ्गे॒ वै पुरु॑षस्य पा॒प्मोप॑श्लिष्टः।
अङ्गा॑दङ्गादे॒वैनं॑ पा॒प्मन॒स्तेन॑ मुञ्चति।
अ॒ञ्ज्ये॒ताय॒ स्वाहा॑ कृ॒ष्णाय॒ स्वाहा᳚ श्वे॒ताय॒ स्वाहेत्य॑श्वरू॒पाणि॑ जुहोति।
रू॒पैरे॒वैन॒ꣳ॒ सम॑र्धयति।
ओष॑धीभ्यः॒ स्वाहा॒ मूले᳚भ्यः॒ स्वाहेत्यो॑षधिहो॒माञ्जु॑होति।
द्व॒य्यो वा ओष॑धयः।
पुष्पे᳚भ्यो॒\-ऽन्याः फलं॑ गृ॒ह्णन्ति॑।
मूले᳚भ्यो॒\-ऽन्याः।
ता ए॒वोभयी॒रव॑ रुन्धे॥६६॥\ip

%3.8.17.5
वन॒स्पति॑भ्यः॒ स्वाहेति॑ वनस्पतिहो॒माञ्जु॑होति।
आ॒र॒ण्यस्या॒\-न्नाद्य॒स्या\-व॑\-रुद्ध्यै।
मे॒षस्त्वा॑ पच॒तैर॑व॒त्वित्यपा᳚व्यानि जुहोति।
प्रा॒णा वै दे॒वा अपा᳚व्याः।
प्रा॒णाने॒वाव॑ रुन्धे।
कूप्या᳚भ्यः॒ स्वाहा॒ऽद्भ्यः स्वाहेत्य॒पाꣳ होमा᳚ञ्जुहोति।
अ॒फ्सु वा आपः॑।
अन्नं॒ वा आपः॑।
अ॒द्भ्यो वा अन्नं॑ जायते।
यदे॒वाद्भ्यो\-ऽन्नं॒ जाय॑ते।
तदव॑ रुन्धे॥६७॥\ip\anuvakamend[पू॒र्व॒दी॒क्षा जु॑होति॒ पूर्व॑ ए॒व द्वि॒षन्तं॒ भ्रातृ॑व्य॒मति॑ क्राम॒त्यन॑न्तरित्यै क्रामति रुन्धे॒ जाय॑त॒ एकं॑ च]

%3.8.18.1
अम्भाꣳ॑सि जुहोति।
अ॒यं वै लो॒को\-ऽम्भाꣳ॑सि।
तस्य॒ वस॒वो\-ऽधि॑पतयः।
अ॒ग्निर्ज्योतिः॑।
यदम्भाꣳ॑सि जु॒होति॑।
इ॒ममे॒व लो॒कमव॑ रुन्धे।
वसू॑ना॒ꣳ॒ सायु॑ज्यं गच्छति।
अ॒ग्निं ज्योति॒रव॑ रुन्धे।
नभाꣳ॑सि जुहोति।
अ॒न्तरि॑क्षं॒ वै नभाꣳ॑सि॥६८॥\ip

%3.8.18.2
तस्य॑ रु॒द्रा अधि॑पतयः।
वा॒युर्ज्योतिः॑।
यन्नभाꣳ॑सि जु॒होति॑।
अ॒न्तरि॑क्षमे॒वाव॑ रुन्धे।
रु॒द्राणा॒ꣳ॒ सायु॑ज्यं गच्छति।
वा॒युं ज्योति॒रव॑ रुन्धे।
महाꣳ॑सि जुहोति।
अ॒सौ वै लो॒को महाꣳ॑सि।
तस्या॑दि॒त्या अधि॑पतयः।
सूर्यो॒ ज्योतिः॑॥६९॥\ip

%3.8.18.3
यन्महाꣳ॑सि जु॒होति॑।
अ॒मुमे॒व लो॒कमव॑ रुन्धे।
आ॒दि॒त्याना॒ꣳ॒ सायु॑ज्यं गच्छति।
सूर्यं॒ ज्योति॒रव॑ रुन्धे।
नमो॒ राज्ञे॒ नमो॒ वरु॑णा॒येति॑ य॒व्यानि॑ जुहोति।
अ॒न्नाद्य॒स्या\-व॑\-रुद्ध्यै।
म॒यो॒भूर्वातो॑ अ॒भि वा॑तू॒स्रा इति॑ ग॒व्यानि॑ जुहोति।
प॒शू॒नामव॑रुद्ध्यै।
प्रा॒णाय॒ स्वाहा᳚ व्या॒नाय॒ स्वाहेति॑ सन्ततिहो॒माञ्जु॑होति।
सु॒व॒र्गस्य॑ लो॒कस्य॒ सन्त॑त्यै॥७०॥\ip

%3.8.18.4
सि॒ताय॒ स्वाहा\-ऽसि॑ताय॒ स्वाहेति॒ प्रमु॑क्तीर्जुहोति।
सु॒व॒र्गस्य॑ लो॒कस्य॒ प्रमु॑क्त्यै।
पृ॒थि॒व्यै स्वाहा॒\-ऽन्तरि॑क्षाय॒ स्वाहेत्या॑ह।
य॒था॒\-य॒जु\-रे॒वै\-तत्।
द॒त्वते॒ स्वाहा॑\-ऽद॒न्तका॑य॒ स्वाहेति॑ शरीरहो॒माञ्जु॑होति।
पि॒तृ॒लो॒कमे॒व तैर्यज॑मा॒नोऽव॑ रुन्धे।
कस्त्वा॑ युनक्ति॒ स त्वा॑ युन॒क्त्विति॑ परि॒धीन् यु॑नक्ति।
इ॒मे वै लो॒काः प॑रि॒धयः॑।
इ॒माने॒वास्मै॑ लो॒कान् यु॑नक्ति।
सु॒व॒र्गस्य॑ लो॒कस्य॒ सम॑ष्ट्यै॥७१॥\ip

%3.8.18.5
यः प्रा॑ण॒तो य आ᳚त्म॒दा इति॑ महि॒मानौ॑ जुहोति।
सु॒व॒र्गो वै लो॒को महः॑।
सु॒व॒र्गमे॒व ताभ्यां᳚ लो॒कं यज॑मा॒नोऽव॑ रुन्धे।
आ ब्रह्म॑न्ब्राह्म॒णो ब्र॑ह्म\-वर्च॒सी जा॑यता॒मिति॒ सम॑स्तानि ब्रह्मवर्च॒सानि॑ जुहोति।
ब्र॒ह्म॒व॒र्चसमे॒व तैर्यज॑मा॒नोऽव॑ रुन्धे।
जज्ञि॒ बीज॒मिति॑ जुहो॒त्यन॑न्तरित्यै।
अ॒ग्नये॒ सम॑नमत्पृथि॒व्यै सम॑नम॒दिति॑ सन्नतिहो॒माञ्जु॑होति।
सु॒व॒र्गस्य॑ लो॒कस्य॒ सन्न॑त्यै।
भू॒ताय॒ स्वाहा॑ भविष्य॒ते स्वाहेति॑ भूताभ॒व्यौ होमौ॑ जुहोति।
अ॒यं वै लो॒को भू॒तम्॥७२॥\ip

%3.8.18.6
अ॒सौ भ॑वि॒ष्यत्।
अ॒नयो॑रे॒व लो॒कयोः॒ प्रति॑ तिष्ठति।
सर्व॒स्याऽऽप्त्यै᳚।
सर्व॒स्या\-व॑\-रुद्ध्यै।
यदक्र॑न्दः प्रथ॒मं जाय॑मान॒ इत्य॑श्वस्तो॒मीयं॑ जुहोति।
सर्व॒स्याऽऽप्त्यै᳚।
सर्व॑स्य॒ जित्यै᳚।
सर्व॑मे॒व तेना᳚ऽऽप्नोति।
सर्वं॑ जयति।
यो᳚ऽश्वमे॒धेन॒ यज॑ते॥७३॥\ip

%3.8.18.7
य उ॑ चैनमे॒वं वेद॑।
य॒ज्ञꣳ रक्षाꣴ॑स्यजिघाꣳसन्।
स ए॒तान्प्र॒जा\-प॑तिर्नक्तꣳहो॒मान॑पश्यत्।
तान॑जुहोत्।
तैर्वै स य॒ज्ञाद्रक्षा॒ꣴ॒स्यपा॑हन्।
यन्न॑क्तꣳहो॒माञ्जु॒होति॑।
य॒ज्ञादे॒व तैर्यज॑मानो॒ रक्षा॒ꣴ॒स्यप॑हन्ति।
उ॒षसे॒ स्वाहा॒ व्यु॑ष्ट्यै॒ स्वाहेत्य॑न्त॒तो जु॑होति।
सु॒व॒र्गस्य॑ लो॒कस्य॒ सम॑ष्ट्यै॥७४॥\ip\anuvakamend[वै नभाꣳ॑सि॒ सूर्यो॒ ज्योतिः॒ सन्त॑त्यै॒ सम॑ष्ट्यै भू॒तं यज॑ते॒ नव॑ च]

%3.8.19.1
ए॒क॒यू॒पो वै॑काद॒शिनी॑ वा।
अ॒न्येषां᳚ य॒ज्ञानां॒ यूपा॑ भवन्ति।
ए॒क॒वि॒ꣳ॒शिन्य॑श्वमे॒धस्य॑।
सु॒व॒र्गस्य॑ लो॒कस्या॒भिजि॑त्यै।
बै॒ल्॒वो वा॑ खादि॒रो वा॑ पाला॒शो वा᳚।
अ॒न्येषां᳚ यज्ञक्रतू॒नां यूपा॑ भवन्ति।
राज्जु॑दाल॒ एक॑विꣳशत्यरत्निरश्वमे॒धस्य॑।
सु॒व॒र्गस्य॑ लो॒कस्य॒ सम॑ष्ट्यै।
नान्येषां᳚ पशू॒नां ते॑ज॒न्या अ॑व॒द्यन्ति॑।
अव॑द्य॒न्त्यश्व॑स्य॥७५॥\ip

%3.8.19.2
पा॒प्मा वै ते॑ज॒नी।
पा॒प्मनो\-ऽप॑हत्यै।
प्ल॒क्ष॒शा॒खाया॑म॒न्येषां᳚ पशू॒नाम॑व॒द्यन्ति॑।
वे॒त॒स॒शा॒खाया॒मश्व॑स्य।
अ॒फ्सुयो॑नि॒र्वा अश्वः॑।
अ॒फ्सु॒जो वे॑त॒सः।
स्व ए॒वास्य॒ योना॒वव॑ द्यति।
यूपे॑षु ग्रा॒म्यान्\-प॒शून्नि॑यु॒ञ्जन्ति॑।
आ॒रो॒केष्वा॑र॒ण्यान्धा॑रयन्ति।
प॒शू॒नां व्यावृ॑त्त्यै।
आ ग्रा॒म्यान्प॒शूल्लँभ॑न्ते।
प्रार॒ण्यान्थ्सृ॑जन्ति।
पा॒प्मनो\-ऽप॑हत्यै॥७६॥\ip\anuvakamend[अश्व॑स्य॒ व्यावृ॑त्त्यै॒ त्रीणि॑ च]

%3.8.20.1
राज्जु॑दालमग्नि॒ष्ठं मि॑नोति।
भ्रू॒ण॒ह॒त्याया॒ अप॑हत्यै।
पौतु॑द्रवाव॒भितो॑ भवतः।
पुण्य॑स्य ग॒न्धस्या\-व॑\-रुद्ध्यै।
भ्रू॒ण॒ह॒त्या\-मे॒वा\-स्मा॑दप॒हत्य॑।
पुण्ये॑न ग॒न्धेनो॑भ॒यतः॒ परि॑ गृह्णाति।
षड्बै॒ल्॒वा भ॑वन्ति।
ब्र॒ह्म॒व॒र्च॒सस्या\-व॑\-रुद्ध्यै।
षट्खा॑दि॒राः।
तेज॒सो\-ऽव॑रुद्ध्यै॥७७॥\ip

%3.8.20.2
षट्पा॑ला॒शाः।
सो॒म॒पी॒थस्या\-व॑\-रुद्ध्यै।
एक॑विꣳशतिः॒ सम्प॑द्यन्ते।
एक॑विꣳशति॒र्वै दे॑वलो॒काः।
द्वाद॑श॒ मासाः॒ पञ्च॒र्तवः॑।
त्रय॑ इ॒मे लो॒काः।
अ॒सावा॑दि॒त्य एक॑वि॒ꣳ॒शः।
ए॒ष सु॑व॒र्गो लो॒कः।
सु॒व॒र्गस्य॑ लो॒कस्य॒ सम॑ष्ट्यै।
श॒तं प॒शवो॑ भवन्ति॥७८॥\ip

%3.8.20.3
श॒तायुः॒ पुरु॑षः श॒तेन्द्रि॑यः।
आयु॑ष्ये॒वेन्द्रि॒ये प्रति॑ तिष्ठति।
सर्वं॒ वा अ॑श्वमे॒ध्याप्नो॑ति।
अप॑रिमिता भवन्ति।
अप॑रिमित॒स्या\-व॑\-रुद्ध्यै।
ब्र॒ह्म॒वा॒दिनो॑ वदन्ति।
कस्मा᳚थ्स॒त्यात्।
द॒क्षि॒ण॒तो᳚\-ऽन्येषां᳚ पशू॒ना\-म॑व॒\-द्यन्ति॑।
उ॒त्त॒र॒तो\-ऽश्व॒स्येति॑।
वा॒रु॒णो वा अश्वः॑॥७९॥\ip

%3.8.20.4
ए॒षा वै वरु॑णस्य॒ दिक्।
स्वाया॑मे॒वास्य॑ दि॒श्यव॑द्यति।
यदित॑रेषां पशू॒नाम॑व॒द्यति॑।
श॒त॒दे॒व॒त्यं॑ तेनाव॑ रुन्धे।
चि॒ते᳚\-ऽग्नावधि॑ वैत॒से कटे\-ऽश्वं॑ चिनोति।
अ॒फ्सुयो॑नि॒र्वा अश्वः॑।
अ॒फ्सु॒जो वे॑त॒सः।
स्व ए॒वैनं॒ योनौ॒ प्रति॑\-ष्ठापयति।
पु॒रस्ता᳚त्प्र॒त्यञ्चं॑ तूप॒रं चि॑नोति।
प॒श्चात्प्रा॒चीनं॑ गोमृ॒गम्॥८०॥\ip

%3.8.20.5
प्रा॒णा॒पा॒नावे॒वास्मि᳚न्थ्स॒म्यञ्चौ॑ दधाति।
अश्वं॑ तूप॒रं गो॑मृ॒गमिति॑ सर्व॒हुत॑ ए॒ताञ्जु॑होति।
ए॒षां लो॒काना॑म॒भिजि॑त्यै।
आ॒त्मना॒ऽभि जु॑\-होति।
सात्मा॑नमे॒वैन॒ꣳ॒ सत॑नुं करोति।
सात्मा॒\-ऽमुष्मिँ॑ल्लो॒के भ॑वति।
य ए॒वं वेद॑।
अथो॒ वसो॑रे॒व धारां॒ तेनाव॑ रुन्धे।
इ॒लु॒\-वर्दा॑य॒ स्वाहा॑ बलि॒वर्दा॑य॒ स्वाहेत्या॑ह।
सं॒व॒थ्स॒रो वा इ॑लु॒वर्दः॑।
प॒रि॒\-व॒थ्स॒रो ब॑लि॒वर्दः॑।
सं॒व॒थ्स॒रा\-दे॒व प॑रि\-वथ्स॒रा\-दायु॒रव॑ रुन्धे।
आयु॑\-रे॒वा\-स्मि॑न्दधाति।
तस्मा॑दश्वमेधया॒जी ज॒रसा॑ वि॒स्रसा॒मुं लो॒कमे॑ति॥८१॥\ip\anuvakamend[तेज॒सो\-ऽव॑रुद्ध्यै भव॒न्त्यश्वो॑ गोमृ॒गमि॑लु॒वर्द॑श्च॒त्वारि॑ च]

%3.8.21.1
ए॒क॒वि॒ꣳ॒शो᳚\-ऽग्निर्भ॑वति।
ए॒क॒वि॒ꣳ॒शः स्तोमः॑।
एक॑\-विꣳशति॒र्यूपाः᳚।
यथा॒ वा अश्वा॑ वर्\mbox{}ष॒भा वा॒ वृषा॑णः सꣴस्फु॒रेरन्॑।
ए॒वमे॒व तथ्स्तोमाः॒ सꣴस्फु॑रन्ते।
यदे॑कवि॒ꣳ॒शाः।
ते यथ्स॑मृ॒च्छेरन्॑।
ह॒न्येता᳚स्य य॒ज्ञः।
द्वा॒द॒श ए॒वाग्निः स्या॒दित्या॑हुः।
द्वा॒द॒शः स्तोमः॑॥८२॥\ip

%3.8.21.2
एका॑दश॒ यूपाः᳚।
यद्द्वा॑द॒शो᳚\-ऽग्निर्भव॑ति।
द्वाद॑श॒ मासाः᳚ संवथ्स॒रः।
सं॒व॒थ्स॒रेणै॒वास्मा॒ अन्न॒मव॑ रुन्धे।
यद्दश॒ यूपा॒ भव॑न्ति।
दशा᳚क्षरा वि॒राट्।
अन्नं॑ वि॒राट्।
वि॒राजै॒वान्नाद्य॒मव॑ रुन्धे।
य ए॑काद॒शः।
स्तन॑ ए॒वास्यै॒ सः॥८३॥\ip

%3.8.21.3
दु॒ह ए॒वैनां॒ तेन॑।
तदा॑हुः।
यद्द्वा॑द॒शो᳚\-ऽग्निः स्या᳚द्द्वाद॒शः स्तोम॒ एका॑दश॒ यूपाः᳚।
यथा॒ स्थूरि॑णा या॒यात्।
ता॒दृक्तत्।
ए॒क॒वि॒ꣳ॒श ए॒वाग्निः स्या॒दित्या॑हुः।
ए॒क॒वि॒ꣳ॒शः स्तोमः॑।
एक॑विꣳशति॒र्यूपाः᳚।
यथा॒ प्रष्टि॑भि॒र्याति॑।
ता॒दृगे॒व तत्॥८४॥\ip

%3.8.21.4
यो वा अ॑श्वमे॒धे ति॒स्रः क॒कुभो॒ वेद॑।
क॒कुद्ध॒ राज्ञां᳚ भवति।
ए॒क॒वि॒ꣳ॒शो᳚\-ऽग्निर्भ॑वति।
ए॒क॒वि॒ꣳ॒शः स्तोमः॑।
एक॑विꣳशति॒र्यूपाः᳚।
ए॒ता वा अ॑श्वमे॒धे ति॒स्रः क॒कुभः॑।
य ए॒वं वेद॑।
क॒कुद्ध॒ राज्ञां᳚ भवति।
यो वा अ॑श्वमे॒धे त्रीणि॑ शी॒र्॒षाणि॒ वेद॑।
शिरो॑ ह॒ राज्ञां᳚ भवति।
ए॒क॒वि॒ꣳ॒शो᳚\-ऽग्निर्भ॑वति।
ए॒क॒वि॒ꣳ॒शः स्तोमः॑।
एक॑विꣳशति॒र्यूपाः᳚।
ए॒तानि॒ वा अ॑श्वमे॒धे त्रीणि॑ शी॒र्॒षाणि॑।
य ए॒वं वेद॑।
शिरो॑ ह॒ राज्ञां᳚ भवति॥८५॥\ip\anuvakamend[द्वा॒द॒शः स्तोमः॒ स ए॒व तच्छिरो॑ ह॒ राज्ञां᳚ भवति॒ षट् च॑]

%3.8.22.1
दे॒वा वा अ॑श्वमे॒धे पव॑माने।
सु॒व॒र्गं लो॒कं न प्राजा॑नन्।
तमश्वः॒ प्राजा॑नात्।
यद॑श्वमे॒धे\-ऽश्वे॑न॒ मेध्ये॒नोद॑ञ्चो बहिष्पवमा॒नꣳ सर्प॑न्ति।
सु॒व॒र्गस्य॑ लो॒कस्य॒ प्रज्ञा᳚त्यै।
न वै म॑नु॒ष्यः॑ सुव॒र्गं लो॒कमञ्ज॑सा वेद।
अश्वो॒ वै सु॑व॒र्गं लो॒कमञ्ज॑सा वेद।
यदु॑द्गा॒तोद्गाये᳚त्।
यथा क्षे᳚त्रज्ञो॒\-ऽन्येन॑ प॒था प्र॑तिपा॒दये᳚त्।
ता॒दृक्तत्॥८६॥\ip

%3.8.22.2
उ॒द्गा॒तार॑मप॒रुध्य॑।
अश्व॑मुद्गी॒थाय॑ वृणीते।
यथा᳚ क्षेत्र॒ज्ञो\-ऽञ्ज॑सा॒ नय॑ति।
ए॒वमे॒वैन॒मश्वः॑ सुव॒र्गं लो॒कमञ्ज॑सा नयति।
पुच्छ॑म॒न्वा र॑भन्ते।
सु॒व॒र्गस्य॑ लो॒कस्य॒ सम॑ष्ट्यै।
हिं क॑रोति।
सामै॒वाकः॑।
हिं क॑रोति।
उ॒द्गी॒थ ए॒वास्य॒ सः॥८७॥\ip

%3.8.22.3
वड॑बा॒ उप॑ रुन्धन्ति।
मि॒थु॒न॒त्वाय॒ प्रजा᳚त्यै।
अथो॒ यथो॑पगा॒तार॑ उप॒गाय॑न्ति।
ता॒दृगे॒व तत्।
उद॑गासी॒दश्वो॒ मेध्य॒ इत्या॑ह।
प्रा॒जा॒प॒त्यो वा अश्वः॑।
प्र॒जा\-प॑तिरुद्गी॒थः।
उ॒द्गी॒थमे॒वाव॑ रुन्धे।
अथो॑ ऋख्सा॒मयो॑रे॒व प्रति॑ तिष्ठति।
हिर॑ण्येनो॒पाक॑रोति।
ज्योति॒र्वै हिर॑ण्यम्।
ज्योति॑रे॒व मु॑ख॒तो द॑धाति।
यज॑माने च प्र॒जासु॑ च।
अथो॒ हिर॑ण्यज्योतिरे॒व यज॑मानः सुव॒र्गं लो॒कमे॑ति॥८८॥\ip\anuvakamend[तथ्स उ॒पाक॑रोति च॒त्वारि॑ च]

%3.8.23.1
पुरु॑षो॒ वै य॒ज्ञः।
य॒ज्ञः प्र॒जा\-प॑तिः।
यदश्वे॑ प॒शून्नि॑यु॒ञ्जन्ति॑।
य॒ज्ञादे॒व तद्य॒ज्ञं प्रयु॑ङ्क्ते।
अश्वं॑ तूप॒रं गो॑मृ॒गम्।
तान॑ग्नि॒ष्ठ आल॑भते।
से॒ना॒मु॒खमे॒व तथ्सꣴश्य॑ति।
तस्मा᳚द्राजमु॒खं भी॒ष्मं भावु॑कम्।
आ॒ग्ने॒यं कृ॒ष्णग्री॑वं पु॒रस्ता᳚ल्ल॒लाटे᳚।
पू॒र्वा॒ग्निमे॒व तं कु॑रुते॥८९॥\ip

%3.8.23.2
तस्मा᳚त्पूर्वा॒ग्निं पु॒रस्ता᳚थ्स्थापयन्ति।
पौ॒ष्णम॒न्वञ्चम्᳚।
अन्नं॒ वै पू॒षा।
तस्मा᳚त्पूर्वा॒ग्नावा॑\-हा॒र्य॑मा ह॑रन्ति।
ऐ॒न्द्रा॒पौ॒ष्णमु॒परि॑ष्टात्।
ऐ॒न्द्रो वै रा॑ज॒न्यो\-ऽन्नं॑ पू॒षा।
अ॒न्नाद्ये॑नै॒वैन॑मुभ॒यतः॒ परि॑ गृह्णाति।
तस्मा᳚द्राज॒न्यो᳚\-ऽन्ना॒दो भावु॑कः।
आ॒ग्ने॒यौ कृ॒ष्णग्री॑वौ बाहु॒वोः।
बा॒हु॒वोरे॒व वी॒र्यं॑ धत्ते॥९०॥\ip

%3.8.23.3
तस्मा᳚द्राज॒न्यो॑ बाहुब॒लीभावु॑कः।
त्वा॒ष्ट्रौ लो॑मशस॒क्थौ स॒क्थ्योः।
स॒क्थ्योरे॒व वी॒र्यं॑ धत्ते।
तस्मा᳚द्राज॒न्य॑ ऊरुब॒लीभावु॑कः।
शि॒ति॒पृ॒ष्ठौ बा॑र्\mbox{}हस्प॒त्यौ पृ॒ष्ठे।
ब्र॒ह्म॒व॒र्च॒समे॒वोपरि॑ष्टाद्धत्ते।
अथो॑ क॒वचे॑ ए॒वैते अ॒भितः॒ पर्यू॑हते।
तस्मा᳚द्राज॒न्यः॑ सन्न॑द्धो वी॒र्यं॑ करोति।
धा॒त्रे पृ॑षोद॒रम॒धस्ता᳚त्।
प्र॒ति॒ष्ठामे॒वैतां कु॑रुते।
अथो॑ इ॒यं वै धा॒ता।
अ॒स्यामे॒व प्रति॑ तिष्ठति।
सौ॒र्यं ब॒लक्षं॒ पुच्छे᳚।
उ॒थ्से॒धमे॒व तं कु॑रुते।
तस्मा॑दुथ्से॒धं भ॒ये प्र॒जा अ॒भिसꣴश्र॑यन्ति॥९१॥\ip\anuvakamend[कु॒रु॒ते॒ ध॒त्ते॒ कु॒रु॒ते॒ पञ्च॑ च]


\prashnaend{सा॒ङ्ग्र॒ह॒ण्या चतु॑ष्टय्यो॒ यो वै यः पि॒तुश्च॒त्वारो॒ यथा॑ नि॒क्तं प्र॒जा\-प॑तये त्वा॒ यथा॒ प्रोक्षि॑तं वि॒भूरा॑ह प्र॒जा\-प॑तिरकामयताश्वमे॒धेन॑ प्र॒जा\-प॑ति॒र्न किञ्च॒न सा॑वि॒त्रमा ब्रह्म॑न्प्र॒जा\-प॑तिर्दे॒वैभ्यः॑ प्र॒जा\-प॑ती॒ रक्षाꣳ॑सि प्र॒जा\-प॑तिमीफ्सति वि॒भूर॑श्वना॒मान्यम्भाꣴ॑स्येकयू॒पो राज्जु॑दालमेकवि॒ꣳ॒शो दे॒वाः पुरु॑ष॒स्त्रयो॑विꣳशतिः॥२३॥}{सा॒ङ्ग्रह॒ण्या तस्मा॑दश्वमेधया॒जी यत्परि॑मिता॒ यद्य॑ज्ञमु॒खे यो दी॒क्षां दे॒वाने॒व त्रय॑ इ॒मे सि॒ताय॑ प्राणापा॒नावे॒वास्मि॒न्तस्मा᳚द्राज॒न्य॑ एक॑नवतिः॥९१॥}{सा॒ङ्ग्र॒ह॒ण्या सꣴश्र॑यन्ति॥}{हरिः॑ ओम्॥}{इति श्रीकृष्णयजुर्वेदीयतैत्तिरीयब्राह्मणे तृतीयाष्टके अष्टमः प्रपाठकः समाप्तः॥}
\sect{नवमः प्रश्नः}
\setcounter{anuvakam}{0}
\dnsub{तैत्तिरीयब्राह्मणे तृतीयाष्टके नवमः प्रपाठकः}

%3.9.1.1
प्र॒जा\-प॑तिरश्वमे॒धम॑\-सृजत।
सो᳚ऽस्माथ्सृ॒ष्टो\-ऽपा᳚क्रामत्।
तम॑ष्टा\-द॒शिभि॒\-रनु॒\- प्रायु॑ङ्क्त।
तमा᳚प्नोत्।
तमा॒प्त्वा\-ऽष्टा॑द॒शिभि॒रवा॑\-रुन्ध।
यद॑ष्टा\-द॒शिन॑ आल॒भ्यन्ते᳚।
य॒ज्ञमे॒व तैरा॒प्त्वा यज॑मा॒नो\-ऽव॑ रुन्धे।
सं॒व॒थ्स॒\-र\-स्य॒ वा ए॒षा प्र॑ति॒मा।
यद॑ष्टाद॒शिनः॑।
द्वाद॑श॒ मासाः॒ पञ्च॒र्तवः॑॥१॥\ip

%3.9.1.2
सं॒व॒थ्स॒रो᳚\-ऽष्टाद॒शः।
यद॑ष्टाद॒शिन॑ आल॒भ्यन्ते᳚।
सं॒व॒थ्स॒रमे॒व तैरा॒प्त्वा यज॑मा॒नो\-ऽव॑ रुन्धे।
अ॒ग्नि॒ष्ठे᳚\-ऽन्यान्प॒शूनु॑पाक॒रोति॑।
इत॑रेषु॒ यूपे᳚ष्वष्टाद॒शिनो\-ऽजा॑मित्वाय।
नव॑न॒वाल॑भ्यन्ते सवीर्य॒त्वाय॑।
यदा॑र॒ण्यैः सꣴ॑स्था॒पये᳚त्।
व्यव॑स्येतां पितापु॒त्रौ।
व्यध्वा॑नः क्रामेयुः।
विदू॑रं॒ ग्राम॑योर्ग्रामा॒न्तौ स्या॑ताम्॥२॥\ip

%3.9.1.3
ऋ॒क्षीकाः᳚ पुरुषव्या॒घ्राः प॑रिमो॒षिण॑ आव्या॒धिनी॒स्तस्क॑रा॒ अर॑ण्ये॒ष्वाजा॑येरन्।
तदा॑हुः।
अप॑शवो॒ वा ए॒ते।
यदा॑र॒ण्याः।
यदा॑र॒ण्यैः सꣴ॑स्था॒पये᳚त्।
क्षि॒प्रे यज॑मान॒मर॑ण्यं मृ॒तꣳ ह॑रेयुः।
अर॑ण्यायतना॒ ह्या॑र॒ण्याः प॒शव॒ इति॑।
यत्प॒शून्नालभे॑त।
अन॑वरुद्धा अस्य प॒शवः॑ स्युः।
यत्पर्य॑ग्निकृतानुथ्सृ॒जेत्॥३॥\ip

%3.9.1.4
य॒ज्ञ॒वे॒श॒सं कु॑र्यात्।
यत्प॒शूना॒लभ॑ते।
तेनै॒व प॒शूनव॑ रुन्धे।
यत्पर्य॑ग्निकृतानुथ्सृ॒जत्य\-य॑ज्ञ\-वेशसाय।
अव॑रुद्धा अस्य प॒शवो॒ भव॑न्ति।
न य॑ज्ञवेश॒सं भ॑वति।
न यज॑मान॒मर॑ण्यं मृ॒तꣳ ह॑रन्ति।
ग्रा॒म्यैः सꣴ स्था॑पयति।
ए॒ते वै प॒शवः॒ क्षेमो॒ नाम॑।
सं पि॑तापु॒त्रावव॑स्यतः।
समध्वा॑नः क्रामन्ति।
स॒म॒न्ति॒कं ग्राम॑योर्ग्रामा॒न्तौ भ॑वतः।
नर्क्षीकाः᳚ पुरुषव्या॒घ्राः प॑रिमो॒षिण॑ आव्या॒धिनी॒स्तस्क॑रा॒ अर॑ण्ये॒ष्वाजा॑यन्ते॥४॥\ip\anuvakamend[ऋ॒तवः॑ स्यातामुथ्सृ॒जेथ्स्य॑त॒स्त्रीणि॑ च]

%3.9.2.1
प्र॒जा\-प॑तिरकामयतो॒भौ लो॒कावव॑ रुन्धी॒येति॑।
स ए॒ता\-नु॒भया᳚न्प॒शून॑\-पश्यत्।
ग्रा॒म्याꣴश्चा॑\-र॒ण्याꣴश्च॑।
तानाल॑भत।
तैर्वै स उ॒भौ लो॒का\-ववा॑\-रुन्ध।
ग्रा॒म्यैरे॒व प॒शुभि॑रि॒मं लो॒कमवा॑रुन्ध।
आ॒र॒ण्यै\-र॒मुम्।
यद् ग्रा॒म्यान्प॒शूना॒लभ॑ते।
इ॒ममे॒व तैर्लो॒कमव॑ रुन्धे।
यदा॑र॒ण्यान्॥५॥\ip

%3.9.2.2
अ॒मुं तैः।
अन॑वरुद्धो॒ वा ए॒तस्य॑ संवथ्स॒र इत्या॑हुः।
य इ॒तइ॑तश्चातुर्मा॒स्यानि॑ संवथ्स॒रं प्र॑यु॒ङ्क्त इति॑।
ए॒तावा॒न्॒ वै सं॑वथ्स॒रः।
यच्चा॑तुर्मा॒स्यानि॑।
यदे॒ते चा॑तुर्मा॒स्याः प॒शव॑ आल॒भ्यन्ते᳚।
प्र॒त्यक्ष॑मे॒व तैः सं॑वथ्स॒रं यज॑मा॒नो\-ऽव॑ रुन्धे।
वि वा ए॒ष प्र॒जया॑ प॒शुभि॑र्‌\mbox{}ऋध्यते।
यः सं॑वथ्स॒रं प्र॑यु॒ङ्क्ते।
सं॒व॒थ्स॒रः सु॑व॒र्गो लो॒कः॥६॥\ip

%3.9.2.3
सु॒व॒र्गं तु लो॒कं नाप॑राध्नोति।
प्र॒जा वै प॒शव॑ एकाद॒शिनी᳚।
यदे॒त ऐ॑कादशि॒नाः प॒शव॑ आल॒भ्यन्ते᳚।
सा॒क्षादे॒व प्र॒जां प॒शून् यज॑मा॒नो\-ऽव॑ रुन्धे।
प्र॒जा\-प॑तिर्वि॒राज॑म\-सृजत।
सा सृ॒ष्टा\-ऽश्व॑मे॒धं प्रावि॑शत्।
तान्द॒शिभि॒रनु॒\- प्रायु॑ङ्क्त।
तामा᳚प्नोत्।
तामा॒प्त्वा द॒शिभि॒रवा॑रुन्ध।
यद्द॒शिन॑ आल॒भ्यन्ते᳚॥७॥\ip

%3.9.2.4
वि॒राज॑मे॒व तैरा॒प्त्वा यज॑मा॒नो\-ऽव॑ रुन्धे।
एका॑दश द॒शत॒ आल॑भ्यन्ते।
एका॑दशाक्षरा त्रि॒ष्टुप्।
त्रैष्टु॑भाः प॒शवः॑।
प॒शूने॒वाव॑ रुन्धे।
वै॒श्व॒दे॒वो वा अश्वः॑।
ना॒ना॒दे॒व॒त्याः᳚ प॒शवो॑ भवन्ति।
अश्व॑स्य सर्व॒त्वाय॑।
नाना॑रूपा भवन्ति।
तस्मा॒न्नाना॑रूपाः प॒शवः॑।
ब॒हु॒रू॒पा भ॑वन्ति।
तस्मा᳚द्बहुरू॒पाः प॒शवः॒ समृ॑द्ध्यै॥८॥\ip\anuvakamend[आ॒र॒ण्याँल्लो॒को द॒शिन॑ आल॒भ्यन्ते॒ नाना॑रूपाः प॒शवो॒ द्वे च॑]

%3.9.3.1
अ॒स्मै वै लो॒काय॑ ग्रा॒म्याः प॒शव॒ आल॑भ्यन्ते।
अ॒मुष्मा॑ आर॒ण्याः।
यद्ग्रा॒म्यान्प॒शूना॒लभ॑ते।
इ॒ममे॒व तैर्लो॒कमव॑ रुन्धे।
यदा॑र॒ण्यान्।
अ॒मुं तैः।
उ॒भया᳚न्प॒शूनाल॑भते।
गा॒म्याꣴश्चा॑र॒ण्याꣴश्च॑।
उ॒भयो᳚र्लो॒कयो॒रव॑रुद्ध्यै।
उ॒भया᳚न्प॒शूना\-ल॑भते॥९॥\ip

%3.9.3.2
ग्रा॒म्याꣴश्चा॑\-र॒ण्याꣴश्च॑।
उ॒भय॑स्या॒न्नाद्य॒स्या\-व॑\-रुद्ध्यै।
उ॒भया᳚न्प॒शू\-नाल॑भते।
ग्रा॒म्याꣴश्चा॑\-र॒ण्याꣴश्च॑।
उ॒भये॑षां पशू॒नामव॑रुद्ध्यै।
त्रय॑स्त्रयो भवन्ति।
त्रय॑ इ॒मे लो॒काः।
ए॒षां लो॒काना॒माप्त्यै᳚।
ब्र॒ह्म॒वा॒दिनो॑ वदन्ति।
कस्मा᳚थ्स॒त्यात्॥१०॥\ip

%3.9.3.3
अ॒स्मिँल्लो॒के ब॒हवः॒ कामा॒ इति॑।
यथ्स॑मा॒नीभ्यो॑ दे॒वता᳚भ्यो॒\-ऽन्ये᳚\-ऽन्ये प॒शव॑ आल॒भ्यन्ते᳚।
अ॒स्मिन्ने॒व तल्लो॒के कामा᳚न्दधाति।
तस्मा॑द॒स्मिँल्लो॒के ब॒हवः॒ कामाः᳚।
त्र॒या॒णां त्र॑याणाꣳ स॒ह व॒पा जु॑होति।
त्र्या॑वृतो॒ वै दे॒वाः।
त्र्या॑वृत इ॒मे लो॒काः।
ए॒षां लो॒काना॒माप्त्यै᳚।
ए॒षां लो॒कानां॒ कॢप्त्यै᳚।
पर्य॑ग्निकृतानार॒ण्या\-नुथ्सृ॑ज॒न्त्यहिꣳ॑सायै॥११॥\ip\anuvakamend[अव॑रुद्ध्या उ॒भया᳚न्प॒शूनाल॑भते स॒त्यादहिꣳ॑सायै]

%3.9.4.1
यु॒ञ्जन्ति॑ ब्र॒ध्नमित्या॑ह।
अ॒सौ वा आ॑दि॒त्यो ब्र॒ध्नः।
आ॒दि॒त्यमे॒वास्मै॑ युनक्ति।
अ॒रु॒षमित्या॑ह।
अ॒ग्निर्वा अ॑रु॒षः।
अ॒ग्निमे॒वास्मै॑ युनक्ति।
चर॑न्त॒मित्या॑ह।
वा॒युर्वै चरन्॑।
वा॒युमे॒वास्मै॑ युनक्ति।
परि॑त॒स्थुष॒ इत्या॑ह॥१२॥\ip

%3.9.4.2
इ॒मे वै लो॒काः परि॑त॒स्थुषः॑।
इ॒माने॒वास्मै॑ लो॒कान् यु॑नक्ति।
रोच॑न्ते रोच॒ना दि॒वीत्या॑ह।
नक्ष॑त्राणि॒ वै रो॑च॒ना दि॒वि।
नक्ष॑त्राण्ये॒वास्मै॑ रोचयति।
यु॒ञ्जन्त्य॑स्य॒ काम्येत्या॑ह।
कामा॑ने॒वास्मै॑ युनक्ति।
हरी॒ विप॑क्ष॒सेत्या॑ह।
इ॒मे वै हरी॒ विप॑क्षसा।
इ॒मे ए॒वास्मै॑ युनक्ति॥१३॥\ip

%3.9.4.3
शोणा॑ धृ॒ष्णू नृ॒वाह॒सेत्या॑ह।
अ॒हो॒रा॒त्रे वै नृ॒वाह॑सा।
अ॒हो॒रा॒त्रे ए॒वास्मै॑ युनक्ति।
ए॒ता ए॒वास्मै॑ दे॒वता॑ युनक्ति।
सु॒व॒र्गस्य॑ लो॒कस्य॒ सम॑ष्ट्यै।
के॒तुं कृ॒ण्वन्न॑के॒तव॒ इति॑ ध्व॒जं प्रति॑\-मुञ्चति।
यश॑ ए॒वैन॒ꣳ॒ राज्ञां गमयति।
जी॒मूत॑स्येव भवति॒ प्रती॑क॒मित्या॑ह।
य॒था॒\-य॒जु\-रे॒वै\-तत्।
ये ते॒ पन्था॑नः सवितः पू॒र्व्यास॒ इत्य॑ध्व॒र्युर्यज॑मानं वाचयत्य॒भिजि॑त्यै॥१४॥\ip

%3.9.4.4
परा॒ वा ए॒तस्य॑ य॒ज्ञ ए॑ति।
यस्य॑ प॒शुरु॒पाकृ॑तो॒\-ऽन्यत्र॒ वेद्या॒ एति॑।
ए॒तꣴस्तो॑तरे॒तेन॑ प॒था पुन॒रश्व॒माव॑र्तयासि न॒ इत्या॑ह।
वा॒युर्वै स्तोता᳚।
वा॒युमे॒वास्य॑ प॒रस्ता᳚द्दधा॒त्यावृ॑त्त्यै।
यथा॒ वै ह॒विषो॑ गृही॒तस्य॒ स्कन्द॑ति।
ए॒वं वा ए॒तदश्व॑स्य स्कन्दति।
यद॑स्यो॒पाकृ॑तस्य॒ लोमा॑नि॒ शीय॑न्ते।
यद्वाले॑षु का॒चाना॒वय॑न्ति।
लोमा᳚न्ये॒वास्य॒ तथ्सम्भ॑रन्ति॥१५॥\ip

%3.9.4.5
भूर्भुवः॒ सुव॒रिति॑ प्राजाप॒त्याभि॒राव॑यन्ति।
प्रा॒जा॒प॒त्यो वा अश्वः॑।
स्वयै॒वैनं॑ दे॒वत॑या॒ सम॑र्धयन्ति।
भूरिति॒ महि॑षी।
भुव॒ इति॑ वा॒वाता᳚।
सुव॒रिति॑ परिवृ॒क्ती।
ए॒षां लो॒काना॑म॒भिजि॑त्यै।
हि॒र॒ण्ययाः᳚ का॒चा भ॑वन्ति।
ज्योति॒र्वै हिर॑ण्यम्।
रा॒ष्ट्रम॑श्वमे॒धः॥१६॥\ip

%3.9.4.6
ज्योति॑श्चै॒वास्मै॑ रा॒ष्ट्रं च॑ स॒मीची॑ दधाति।
स॒हस्रं॑ भवन्ति।
स॒हस्र॑सम्मितः सुव॒र्गो लो॒कः।
सु॒व॒र्गस्य॑ लो॒कस्या॒भिजि॑त्यै।
अप॒ वा ए॒तस्मा॒त्तेज॑ इन्द्रि॒यं प॒शवः॒ श्रीः क्रा॑मन्ति।
यो᳚ऽश्वमे॒धेन॒ यज॑ते।
वस॑वस्त्वा\-ऽञ्जन्तु गाय॒त्रेण॒ छन्द॒सेति॒ महि॑ष्य॒भ्य॑नक्ति।
तेजो॒ वा आज्यम्᳚।
तेजो॑ गाय॒त्री।
तेज॑सै॒वास्मै॒ तेजो\-ऽव॑ रुन्धे॥१७॥\ip

%3.9.4.7
रु॒द्रास्त्वा᳚ञ्जन्तु॒ त्रैष्टु॑भेन॒ छन्द॒सेति॑ वा॒वाता᳚।
तेजो॒ वा आज्यम्᳚।
इ॒न्द्रि॒यं त्रि॒ष्टुप्।
तेज॑सै॒वास्मा॑ इन्द्रि॒यमव॑ रुन्धे।
आ॒दि॒त्यास्त्वा᳚\-ऽञ्जन्तु॒ जाग॑तेन॒ छन्द॒सेति॑ परिवृ॒क्ती।
तेजो॒ वा आज्यम्᳚।
प॒शवो॒ जग॑ती।
तेज॑सै॒वास्मे॑ प॒शूनव॑ रुन्धे।
पत्न॑यो॒\-ऽभ्य॑ञ्जन्ति।
श्रि॒या वा ए॒तद्रू॒पम्॥१८॥\ip

%3.9.4.8
यत्पत्न॑यः।
श्रिय॑मे॒वास्मि॒न्तद्द॑धति।
नास्मा॒त्तेज॑ इन्द्रि॒यं प॒शवः॒ श्रीरप॑ क्रामन्ति।
लाजी(३)ञ्छाची(३)न् यशो॑म॒माँ(४) इत्यति॑रिक्त॒मन्न॒मश्वा॑यो॒पाह॑रन्ति।
प्र॒जामे॒वान्ना॒दीं कु॑र्वते।
ए॒तद्दे॑वा॒ अन्न॑मत्तै॒तदन्न॑मद्धि प्रजापत॒ इत्या॑ह।
प्र॒जाया॑मे॒वान्नाद्यं॑ दधते।
यदि॒ नाव॒जिघ्रे᳚त्।
अ॒ग्निः प॒शुरा॑सी॒दित्यव॑घ्रापयेत्।
अव॑ है॒व जि॑घ्रति।
आक्रान्॑ वा॒जी क्रमै॒रत्य॑क्रमीद्वा॒जी द्यौस्ते॑ पृ॒ष्ठं पृ॑थि॒वी स॒धस्थ॒मित्यश्व॒मनु॑मन्त्रयते।
ए॒षां लो॒काना॑म॒भिजि॑त्यै।
समि॑द्धो अ॒ञ्जन्कृद॑रं मती॒नामित्यश्व॑स्या॒प्रियो॑ भवन्ति सरूप॒त्वाय॑॥१९॥\ip\anuvakamend[परि॑त॒स्थुष॒ इत्या॑हे॒मे ए॒वास्मै॑ युनक्त्य॒भिजि॑त्यै भरन्त्यश्वमे॒धो रु॑न्धे रू॒पञ्जि॑घ्रति॒ त्रीणि॑ च]

%3.9.5.1
तेज॑सा॒ वा ए॒ष ब्र॑ह्म\-वर्च॒सेन॒ व्यृ॑द्ध्यते।
यो᳚ऽश्वमे॒धेन॒ यज॑ते।
होता॑ च ब्र॒ह्मा च॑ ब्र॒ह्मोद्यं॑ वदतः।
तेज॑सा चै॒वैनं॑ ब्रह्मवर्च॒सेन॑ च॒ सम॑र्धयतः।
द॒क्षि॒ण॒तो ब्र॒ह्मा भ॑वति।
द॒क्षि॒ण॒त आ॑यतनो॒ वै ब्र॒ह्मा।
बा॒र्॒ह॒स्प॒त्यो वै ब्र॒ह्मा।
ब्र॒ह्म॒व॒र्च॒समे॒वास्य॑ दक्षिण॒तो द॑धाति।
तस्मा॒द्दक्षि॒णो\-ऽर्धो᳚ ब्रह्मवर्च॒सित॑रः।
उ॒त्त॒र॒तो होता॑ भवति॥२०॥\ip

%3.9.5.2
उ॒त्त॒र॒त आ॑यतनो॒ वै होता᳚।
आ॒ग्ने॒यो वै होता᳚।
तेजो॒ वा अ॒ग्निः।
तेज॑ ए॒वास्यो᳚त्तर॒तो द॑धाति।
तस्मा॒दुत्त॒रो\-ऽर्ध॑स्तेज॒स्वित॑रः।
यूप॑म॒भितो॑ वदतः।
य॒ज॒मा॒न॒दे॒व॒त्यो॑ वै यूपः॑।
यज॑मानमे॒व तेज॑सा च ब्रह्मवर्च॒सेन॑ च॒ सम॑र्धयतः।
किꣴ स्वि॑दासीत्पू॒र्वचि॑त्ति॒रित्या॑ह।
द्यौर्वै वृष्टिः॑ पू॒र्वचि॑त्तिः॥२१॥\ip

%3.9.5.3
दिव॑मे॒व वृष्टि॒मव॑ रुन्धे।
किꣴ स्वि॑दासीद्बृ॒हद्वय॒ इत्या॑ह।
अश्वो॒ वै बृ॒हद्वयः॑।
अश्व॑मे॒वाव॑ रुन्धे।
किꣴ स्वि॑दासीत्पिशङ्गि॒लेत्या॑ह।
रात्रि॒र्वै पि॑शङ्गि॒ला।
रात्रि॑मे॒वाव॑ रुन्धे।
किꣴ स्वि॑दासीत्पिलिप्पि॒ले\-त्या॑ह।
श्रीर्वै पि॑लिप्पि॒ला।
अ॒न्नाद्य॑मे॒वाव॑ रुन्धे॥२२॥\ip

%3.9.5.4
कः स्वि॑देका॒की च॑र॒तीत्या॑ह।
अ॒सौ वा आ॑दि॒त्य ए॑का॒की च॑रति।
तेज॑ ए॒वाव॑ रुन्धे।
क उ॑स्विज्जायते॒ पुन॒रित्या॑ह।
च॒न्द्रमा॒ वै जा॑यते॒ पुनः॑।
आयु॑रे॒वाव॑ रुन्धे।
किꣴ स्वि॑द्धि॒मस्य॑ भेष॒जमित्या॑ह।
अ॒ग्निर्वै हि॒मस्य॑ भेष॒जम्।
ब्र॒ह्म॒व॒र्च॒समे॒वाव॑ रुन्धे।
किꣴ स्वि॑दा॒वप॑नं म॒हदित्या॑ह॥२३॥\ip

%3.9.5.5
अ॒यं वै लो॒क आ॒वप॑नं म॒हत्।
अ॒स्मिन्ने॒व लो॒के प्रति॑ तिष्ठति।
पृ॒च्छामि॑ त्वा॒ पर॒मन्तं॑ पृथि॒व्या इत्या॑ह।
वेदि॒र्वै परो\-ऽन्तः॑ पृथि॒व्याः।
वेदि॑मे॒वाव॑ रुन्धे।
पृ॒च्छामि॑ त्वा॒ भुव॑नस्य॒ नाभि॒मित्या॑ह।
य॒ज्ञो वै भुव॑नस्य॒ नाभिः॑।
य॒ज्ञमे॒वाव॑ रुन्धे।
पृ॒च्छामि॑ त्वा॒ वृष्णो॒ अश्व॑स्य॒ रेत॒ इत्या॑ह।
सोमो॒ वै वृष्णो॒ अश्व॑स्य॒ रेतः॑।
सो॒म॒पी॒थमे॒वाव॑ रुन्धे।
पृ॒च्छामि॑ वा॒चः प॑र॒मं व्यो॑मेत्या॑ह।
ब्रह्म॒ वै वा॒चः प॑र॒मं व्यो॑म।
ब्र॒ह्म॒व॒र्च॒समे॒वाव॑ रुन्धे॥२४॥\ip\anuvakamend[होता॑ भवति॒ वै वृष्टिः॑ पू॒र्वचि॑त्तिर॒न्नाद्य॑मे॒वाव॑ रुन्धे म॒हदित्या॑ह॒ सोमो॒ वै वृष्णो॒ अश्व॑स्य॒ रेत॑श्च॒त्वारि॑ च]

%3.9.6.1
अप॒ वा ए॒तस्मा᳚त्प्रा॒णाः क्रा॑मन्ति।
यो᳚ऽश्वमे॒धेन॒ यज॑ते।
प्रा॒णाय॒ स्वाहा᳚ व्या॒नाय॒ स्वाहेति॑ संज्ञ॒प्यमा॑न॒ आहु॑तीर्जुहोति।
प्रा॒णाने॒वास्मि॑न्दधाति।
नास्मा᳚त्प्रा॒णा अप॑क्रामन्ति।
अव॑न्तीः॒ स्थाव॑न्तीस्त्वा\-ऽवन्तु।
प्रि॒यं त्वा᳚ प्रि॒याणा᳚म्।
वर्‌\mbox{}षि॑ष्ठ॒माप्या॑नाम्।
नि॒धी॒नां त्वा॑ निधि॒पतिꣳ॑ हवामहे वसो म॒मेत्या॑ह।
अपै॒वास्मै॒ तद्ध्नु॑वते॥२५॥\ip

%3.9.6.2
अथो॑ धु॒वन्त्ये॒वैनम्᳚।
अथो॒ न्ये॑वास्मै᳚ ह्नुवते।
त्रिः परि॑यन्ति।
त्रय॑ इ॒मे लो॒काः।
ए॒भ्य ए॒वैनं॑ लो॒केभ्यो॑ धुवते।
त्रिः पुनः॒ परि॑यन्ति।
षट्थ्सम्प॑द्यन्ते।
षड्वा ऋ॒तवः॑।
ऋ॒तुभि॑रे॒वैनं॑ धुवते।
अप॒ वा ए॒तेभ्यः॑ प्रा॒णाः क्रा॑मन्ति॥२६॥\ip

%3.9.6.3
ये य॒ज्ञे धुव॑नं त॒न्वते᳚।
न॒व॒कृत्वः॒ परि॑यन्ति।
नव॒ वै पुरु॑षे प्रा॒णाः।
प्रा॒णाने॒वाऽऽत्मन्द॑धते।
नैभ्यः॑ प्रा॒णा अप॑क्रामन्ति।
अम्बे॒ अम्बा॒ल्यम्बि॑क॒ इति॒ पत्नी॑मु॒दान॑यति।
अह्व॑तै॒वैना᳚म्।
सुभ॑गे॒ काम्पी॑लवासि॒नीत्या॑ह।
तप॑ ए॒वैना॒मुप॑नयति।
सु॒व॒र्गे लो॒के सम्प्रोर्ण्वा॑था॒मित्या॑ह॥२७॥\ip

%3.9.6.4
सु॒व॒र्गमे॒वैनां᳚ लो॒कं ग॑मयति।
आऽहम॑जानि गर्भ॒धमा त्वम॑जाऽसि गर्भ॒धमित्या॑ह।
प्र॒जा वै प॒शवो॒ गर्भः॑।
प्र॒जामे॒व प॒शूना॒त्मन्ध॑त्ते।
दे॒वा वा अ॑श्वमे॒धे पव॑माने।
सु॒व॒र्गं लो॒कं न प्राजा॑नन्।
तमश्वः॒ प्राजा॑नात्।
यथ्सू॒चीभि॑रसिप॒थान्क॒ल्पय॑न्ति।
सु॒व॒र्गस्य॑ लो॒कस्य॒ प्रज्ञा᳚त्यै।
गा॒य॒त्री त्रि॒ष्टुब्जग॒तीत्या॑ह॥२८॥\ip

%3.9.6.5
य॒था॒\-य॒जु\-रे॒वै\-तत्।
त्र॒य्यः सू॒च्यो॑ भवन्ति।
अ॒य॒स्मय्यो॑ रज॒ता हरि॑ण्यः।
अ॒स्य वै लो॒कस्य॑ रू॒पम॑य॒स्मय्यः॑।
अ॒न्तरि॑क्षस्य रज॒ताः।
दि॒वो हरि॑ण्यः।
दिशो॒ वा अ॑य॒स्मय्यः॑।
अ॒वा॒न्त॒र॒दि॒शा र॑ज॒ताः।
ऊ॒र्ध्वा हरि॑ण्यः।
दिश॑ ए॒वास्मै॑ कल्पयति।
कस्त्वा᳚ छ्यति॒ कस्त्वा॒ विशा॒स्तीत्या॒हाहिꣳ॑सायै॥२९॥\ip\anuvakamend[ह्नु॒व॒ते॒ क्रा॒म॒न्त्यू॒र्ण्वा॒था॒मित्या॑ह॒ जग॒तीत्या॑ह कल्पय॒त्येकं॑ च]

%3.9.7.1
अप॒ वा ए॒तस्मा॒च्छ्री रा॒ष्ट्रं क्रा॑मति।
यो᳚ऽश्वमे॒धेन॒ यज॑ते।
ऊ॒र्ध्वामे॑ना॒मुच्छ्र॑यता॒दित्या॑ह।
श्रीर्वै रा॒ष्ट्रम॑श्वमे॒धः।
श्रिय॑मे॒वास्मै॑ रा॒ष्ट्रमू॒र्ध्वमुच्छ्र॑यति।
वे॒णु॒भा॒रङ्गि॒रावि॒वेत्या॑ह।
रा॒ष्ट्रं वै भा॒रः।
रा॒ष्ट्रमे॒वास्मै॒ पर्यू॑हति।
अथा᳚स्या॒ मध्य॑मेधता॒मित्या॑ह।
श्रीर्वै रा॒ष्ट्रस्य॒ मध्यम्᳚॥३०॥\ip

%3.9.7.2
श्रिय॑मे॒वाव॑ रुन्धे।
शी॒ते वाते॑ पु॒नन्नि॒वेत्या॑ह।
क्षेमो॒ वै रा॒ष्ट्रस्य॑ शी॒तो वातः॑।
क्षेम॑मे॒वाव॑ रुन्धे।
यद्ध॑रि॒णी यव॒मत्तीत्या॑ह।
विड्वै ह॑रि॒णी।
रा॒ष्ट्रं यवः॑।
विशं॑ चै॒वास्मै॑ रा॒ष्ट्रं च॑ स॒मीची॑ दधाति।
न पु॒ष्टं प॒शु म॑न्यत॒ इत्या॑ह।
तस्मा॒द्राजा॑ प॒शून्न पुष्य॑ति॥३१॥\ip

%3.9.7.3
शू॒द्रा यदर्य॑जारा॒ न पोषा॑य धनाय॒तीत्या॑ह।
तस्मा᳚द्वैशीपु॒त्रं नाभिषि॑ञ्चन्ते।
इ॒यं य॒का श॑कुन्ति॒केत्या॑ह।
विड्वै श॑कुन्ति॒का।
रा॒ष्ट्रम॑श्वमे॒धः।
विशं॑ चै॒वास्मै॑ रा॒ष्ट्रं च॑ स॒मीची॑ दधाति।
आ॒हल॒मिति॒ सर्प॒तीत्या॑ह।
तस्मा᳚द्रा॒ष्ट्राय॒ विशः॑ सर्पन्ति।
आह॑तं ग॒भे पस॒ इत्या॑ह।
विड्वै गभः॑॥३२॥\ip

%3.9.7.4
रा॒ष्ट्रं पसः॑।
रा॒ष्ट्रमे॒व वि॒श्याह॑न्ति।
तस्मा᳚द्रा॒ष्ट्रं विशं॒ घातु॑कम्।
मा॒ता च॑ ते पि॒ता च॑ त॒ इत्या॑ह।
इ॒यं वै मा॒ता।
अ॒सौ पि॒ता।
आ॒भ्यामे॒वैनं॒ परि॑ददाति।
अग्रं॑ वृ॒क्षस्य॑ रोहत॒ इत्या॑ह।
श्रीर्वै वृ॒क्षस्याग्रम्᳚।
श्रि॒य॑मे॒वाव॑ रुन्धे॥३३॥\ip

%3.9.7.5
प्रसु॑ला॒मीति॑ ते पि॒ता ग॒भे मु॒ष्टिम॑तꣳसय॒दित्या॑ह।
विड्वै गभः॑।
रा॒ष्ट्रं मु॒ष्टिः।
रा॒ष्ट्रमे॒व वि॒श्याह॑न्ति।
तस्मा᳚द्रा॒ष्ट्रं विशं॒ घातु॑कम्।
अप॒ वा ए॒तेभ्यः॑ प्रा॒णाः क्रा॑मन्ति।
ये य॒ज्ञे\-ऽपू॑तं॒ वद॑न्ति।
द॒धि॒क्राव्ण्णो॑ अकारिष॒मिति॑ सुरभि॒मती॒मृचं॑ वदन्ति।
प्रा॒णा वै सु॑र॒भयः॑।
प्रा॒णाने॒वाऽऽत्मन्द॑धते।
नैभ्यः॑ प्रा॒णा अप॑क्रामन्ति।
आपो॒ हि ष्ठा म॑यो॒भुव॒ इत्य॒द्भिर्मा᳚र्जयन्ते।
आपो॒ वै सर्वा॑ दे॒वताः᳚।
दे॒वता॑भिरे॒वाऽऽत्मानं॑ पवयन्ते॥३४॥\ip\anuvakamend[रा॒ष्ट्रस्य॒ मध्यं॒ पुष्य॑ति॒ गभो॑ रुन्धे दधते च॒त्वारि॑ च]

%3.9.8.1
प्र॒जा\-प॑तिः प्र॒जाः सृ॒ष्ट्वा प्रे॒णाऽनु॒ प्रावि॑शत्।
ताभ्यः॒ पुनः॒ सम्भ॑वितुं॒ नाश॑क्नोत्।
सो᳚ऽब्रवीत्।
ऋ॒ध्नव॒दिथ्सः।
यो मे॒तः पुनः॑ स॒म्भर॒दिति॑।
तं दे॒वा अ॑श्वमे॒धेनै॒व सम॑भरन्।
ततो॒ वै त आ᳚र्ध्नुवन्।
यो᳚ऽश्वमे॒धेन॒ यज॑ते।
प्र॒जा\-प॑तिमे॒व सम्भ॑रत्यृ॒ध्नोति॑।
पुरु॑ष॒माल॑भते॥३५॥\ip

%3.9.8.2
वै॒रा॒जो वै पुरु॑षः।
वि॒राज॑मे॒वाल॑भते।
अथो॒ अन्नं॒ वै वि॒राट्।
अन्न॑मे॒वाव॑ रुन्धे।
अश्व॒माल॑भते।
प्रा॒जा॒प॒त्यो वा अश्वः॑।
प्र॒जा\-प॑तिमे॒वाल॑भते।
अथो॒ श्रीर्वा एक॑शफम्।
श्रिय॑मे॒वाव॑ रुन्धे।
गामाल॑भते॥३६॥\ip

%3.9.8.3
य॒ज्ञो वै गौः।
य॒ज्ञमे॒वाल॑भते।
अथो॒ अन्नं॒ वै गौः।
अन्न॑मे॒वाव॑ रुन्धे।
अ॒जा॒वी आल॑भते भू॒म्ने।
अथो॒ पुष्टि॒र्वै भू॒मा।
पुष्टि॑मे॒वाव॑ रुन्धे।
पर्य॑ग्निकृतं॒ पुरु॑षं चार॒ण्याꣴश्चोथ्सृ॑ज॒न्त्यहिꣳ॑सायै।
उ॒भौ वा ए॒तौ प॒शू आल॑भ्येते।
यश्चा॑व॒मो यश्च॑ पर॒मः।
ते᳚ऽस्यो॒भये॑ य॒ज्ञे ब॒द्धाः।
अ॒भीष्टा॑ अ॒भिप्री॑ताः।
अ॒भिजि॑ता अ॒भिहु॑ता भवन्ति।
नैनं॑ द॒ङ्क्ष्णवः॑ प॒शवो॑ य॒ज्ञे ब॒द्धाः।
अ॒भीष्टा॑ अ॒भिप्री॑ताः।
अ॒भिजि॑ता अ॒भिहु॑ता हिꣳसन्ति।
यो᳚ऽश्वमे॒धेन॒ यज॑ते।
य उ॑ चैनमे॒वं वेद॑॥३७॥\ip\anuvakamend[ल॒भ॒ते॒ गामाल॑भते पर॒मो᳚\-ऽष्टौ च॑]

%3.9.9.1
प्र॒थ॒मेन॒ वा ए॒ष स्तोमे॑न रा॒ध्वा।
च॒तु॒ष्टो॒मेन॑ कृ॒तेनाया॑ना॒मुत्त॒रे\-हन्॑।
ए॒क॒वि॒ꣳ॒शे प्र॑ति॒ष्ठायां॒ प्रति॑ तिष्ठति।
ए॒क॒वि॒ꣳ॒शात्प्र॑ति॒ष्ठाया॑ ऋ॒तून॒न्वारो॑हति।
ऋ॒तवो॒ वै पृ॒ष्ठानि॑।
ऋ॒तवः॑ संवथ्स॒रः।
ऋ॒तुष्वे॒व सं॑वथ्स॒रे प्र॑ति॒ष्ठाय॑।
दे॒वता॑ अ॒भ्यारो॑हति।
शक्व॑रयः पृ॒ष्ठं भ॑वन्त्य॒न्यद॑न्य॒च्छन्दः॑।
अ॒न्ये᳚\-ऽन्ये॒ वा ए॒ते प॒शव॒ आल॑भ्यन्ते॥३८॥\ip

%3.9.9.2
उ॒तेव॑ ग्रा॒म्याः।
उ॒तेवा॑र॒ण्याः।
अह॑रे॒व रू॒पेण॒ सम॑र्धयति।
अथो॒ अह्न॑ ए॒वैष ब॒लिर्‌\mbox{}ह्रि॑यते।
तदा॑हुः।
अप॑शवो॒ वा ए॒ते।
यद॑जा॒वय॑श्चार॒ण्याश्च॑।
ए॒ते वै सर्वे॑ प॒शवः॑।
यद्ग॒व्या इति॑।
ग॒व्यान्प॒शूनु॑त्त॒मेऽहं॒ नाल॑भते॥३९॥\ip

%3.9.9.3
तेनै॒वोभया᳚न्प॒शूनव॑ रुन्धे।
प्रा॒जा॒प॒त्या भ॑वन्ति।
अन॑भि\-जितस्या॒भि\-जि॑त्यै।
सौ॒रीर्नव॑ श्वे॒ता व॒शा अ॑नूब॒न्ध्या॑ भवन्ति।
अ॒न्त॒त ए॒व ब्र॑ह्म\-वर्च॒समव॑ रुन्धे।
सोमा॑य स्व॒राज्ञे॑\-ऽनोवा॒हाव॑न॒ड्वाहा॒विति॑ द्व॒न्द्विनः॑ प॒शूनाल॑भते।
अ॒हो॒रा॒त्राणा॑म॒भिजि॑त्यै।
प॒शुभि॒र्वा ए॒ष व्यृ॑ध्यते।
यो᳚ऽश्वमे॒धेन॒ यज॑ते।
छ॒ग॒लं क॒ल्माषं॑ किकिदी॒विं वि॑दी॒गय॒मिति॑ त्वा॒ष्ट्रान्प॒शूना ल॑भते।
प॒शुभि॑रे॒वाऽऽत्मान॒ꣳ॒ सम॑र्धयति।
ऋ॒तुभि॒र्वा ए॒ष व्यृ॑ध्यते।
यो᳚ऽश्वमे॒धेन॒ यज॑ते।
पि॒शङ्गा॒स्त्रयो॑ वास॒न्ता इत्यृ॑तुप॒शूनाल॑भते।
ऋ॒तुभि॑रे॒वाऽऽत्मान॒ꣳ॒ सम॑र्धयति।
आ वा ए॒ष प॒शुभ्यो॑ वृश्च्यते।
यो᳚ऽश्वमे॒धेन॒ यज॑ते।
पर्य॑ग्निकृता॒ उथ्सृ॑ज॒न्त्यना᳚व्रस्काय॥४०॥\ip\anuvakamend[ल॒भ्य॒न्ते॒ ल॒भ॒ते॒ त्वा॒ष्ट्रान्प॒शूनाल॑भते॒\-ऽष्टौ च॑]

%3.9.10.1
प्र॒जा\-प॑तिरकामयत म॒हान॑न्ना॒दः स्या॒मिति॑।
स ए॒ताव॑श्वमे॒धे म॑हि॒माना॑वपश्यत्।
ताव॑गृह्णीत।
ततो॒ वै स म॒हान॑न्ना॒दो॑\-ऽभवत्।
यः का॒मये॑त म॒हान॑न्ना॒दः स्या॒मिति॑।
स ए॒ताव॑श्वमे॒धे म॑हि॒मानौ॑ गृह्णीत।
म॒हाने॒वान्ना॒दो भ॑वति।
य॒ज॒मा॒न॒दे॒व॒त्या॑ वै व॒पा।
राजा॑ महि॒मा।
यद्व॒पां म॑हि॒म्नोभ॒यतः॑ परि॒यज॑ति।
यज॑मानमे॒व रा॒ज्येनो॑भ॒यतः॒ परि॑गृह्णाति।
पु॒रस्ता᳚थ्स्वाहाकारा॒ वा अ॒न्ये दे॒वाः।
उ॒परि॑ष्टाथ्स्वाहाकारा अ॒न्ये।
ते वा ए॒ते\-ऽश्व॑ ए॒व मेध्य॑ उ॒भये\-ऽव॑रुध्यन्ते।
यद्व॒पां म॑हि॒म्नोभ॒यतः॑ परि॒यज॑ति।
ताने॒वोभया᳚न्प्रीणाति॥४१॥\ip\anuvakamend[प॒रि॒यज॑ति॒ षट्च॑]

%3.9.11.1
वै॒श्व॒दे॒वो वा अश्वः॑।
तं यत्प्रा॑जाप॒त्यं कु॒र्यात्।
या दे॒वता॒ अपि॑भागाः।
ता भा॑ग॒धेये॑न॒ व्य॑र्धयेत्।
दे॒वता᳚भ्यः स॒मदं॑ दध्यात्।
स्ते॒गान्दꣴष्ट्रा᳚भ्यां म॒ण्डूकां॒ जम्भ्ये॑भि॒रिति॑।
आज्य॑मव॒दानं॑ कृ॒त्वा प्र॑तिस॒ङ्ख्याय॒माहु॑तीर्जुहोति।
या ए॒व दे॒वता॒ अपि॑भागाः।
ता भा॑ग॒धेये॑न॒ सम॑र्धयति।
न दे॒वता᳚भ्यः स॒मदं॑ दधाति॥४२॥\ip

%3.9.11.2
चतु॑र्दशै॒तान॑नुवा॒काञ्जु॑हो॒त्यन॑न्तरित्यै।
प्र॒या॒साय॒ स्वाहेति॑ पञ्चद॒शम्।
पञ्च॑दश॒ वा अ॑र्धमा॒सस्य॒ रात्र॑यः।
अ॒र्ध॒मा॒स॒शः सं॑वथ्स॒र आ᳚प्यते।
दे॒वा॒सु॒राः संय॑त्ता आसन्।
ते᳚ऽब्रुवन्न॒ग्नयः॑ स्विष्ट॒कृतः॑।
अश्व॑स्य॒ मेध्य॑स्य व॒यमु॑द्धा॒रमुद्ध॑रामहै।
अथै॒तान॒भि भ॑वा॒मेति॑।
ते लोहि॑त॒मुद॑हरन्त।
ततो॑ दे॒वा अभ॑वन्॥४३॥\ip

%3.9.11.3
पराऽसु॑राः।
यथ्स्वि॑ष्ट॒कृद्भ्यो॒ लोहि॑तं जु॒होति॒ भ्रातृ॑व्याऽभिभूत्यै।
भव॑त्या॒त्मना᳚।
परा᳚ऽस्य॒ भ्रातृ॑व्यो भवति।
गो॒मृ॒ग॒क॒ण्ठेन॑ प्रथ॒मामाहु॑तिं जुहोति।
प॒शवो॒ वै गो॑मृ॒गः।
रु॒द्रो᳚\-ऽग्निः स्वि॑ष्ट॒कृत्।
रु॒द्रादे॒व प॒शून॒न्तर्द॑धाति।
अथो॒ यत्रै॒षा\-ऽऽहु॑तिर्‌\mbox{}हू॒यते᳚।
न तत्र॑ रु॒द्रः प॒शून॒भिम॑न्यते॥४४॥\ip

%3.9.11.4
अ॒श्व॒श॒फेन॑ द्वि॒तीया॒माहु॑तिं जुहोति।
प॒शवो॒ वा एक॑शफम्।
रु॒द्रो᳚\-ऽग्निः स्वि॑ष्ट॒कृत्।
रु॒द्रादे॒व प॒शून॒न्तर्द॑धाति।
अथो॒ यत्रै॒षा\-ऽऽहु॑तिर्‌\mbox{}हू॒यते᳚।
न तत्र॑ रु॒द्रः प॒शून॒भिम॑न्यते।
अ॒य॒स्मये॑न कम॒ण्डलु॑ना तृ॒तीया᳚म्।
आहु॑तिं जुहोत्याया॒स्यो॑ वै प्र॒जाः।
रु॒द्रो᳚\-ऽग्निः स्वि॑ष्ट॒कृत्।
रु॒द्रादे॒व प्र॒जा अ॒न्तर्द॑धाति।
अथो॒ यत्रै॒षा\-ऽऽहु॑तिर्‌\mbox{}हू॒यते᳚।
न तत्र॑ रु॒द्रः प्र॒जा अ॒भिम॑न्यते॥४५॥\ip\anuvakamend[द॒धा॒त्यभ॑वन्मन्यते प्र॒जा अ॒न्तर्द॑धाति॒ द्वे च॑ ]

%3.9.12.1
अश्व॑स्य॒ वा आल॑ब्धस्य॒ मेध॒ उद॑क्रामत्।
तद॑श्वस्तो॒मीय॑\-मभवत्।
यद॑श्वस्तो॒मीयं॑ जु॒होति॑।
समे॑धमे॒वैन॒माल॑भते।
आज्ये॑न जुहोति।
मेधो॒ वा आज्यम्᳚।
मेधो᳚\-ऽश्वस्तो॒मीयम्᳚।
मेधे॑नै॒वास्मि॒न्मेधं॑ दधाति।
षट्त्रिꣳ॑शतं जुहोति।
षट्त्रिꣳ॑शदक्षरा बृह॒ती॥४६॥\ip

%3.9.12.2
बार्‌\mbox{}ह॑ताः प॒शवः॑।
सा प॑शू॒नां मात्रा᳚।
प॒शूने॒व मात्र॑या॒ सम॑र्धयति।
तायद्भूय॑सीर्वा॒ कनी॑यसीर्वा जुहु॒यात्।
प॒शून्मात्र॑या॒ व्य॑र्धयेत्।
षट्त्रिꣳ॑शतं जुहोति।
षट्त्रिꣳ॑शदक्षरा बृह॒ती।
बार्‌\mbox{}ह॑ताः प॒शवः॑।
सा प॑शू॒नां मात्रा᳚।
प॒शूने॒व मात्र॑या॒ सम॑र्धयति॥४७॥\ip

%3.9.12.3
अ॒श्व॒स्तो॒मीयꣳ॑ हु॒त्वा द्वि॒पदा॑ जुहोति।
द्वि॒पाद्वै पुरु॑षो॒ द्विप्र॑तिष्ठः।
तदे॑नं प्रति॒ष्ठया॒ सम॑र्धयति।
तदा॑हुः।
अ॒श्व॒स्तो॒मीयं॒ पूर्वꣳ॑ होत॒व्याँ(३)न्द्वि॒पदा(३) इति॑।
अश्वो॒ वा अ॑श्वस्तो॒मीयम्᳚।
पुरु॑षो द्वि॒पदाः᳚।
अ॒श्व॒स्तो॒मीयꣳ॑ हु॒त्वा द्वि॒पदा॑ जुहोति।
तस्मा᳚द्\-द्वि॒पाच्चतु॑ष्पादमत्ति।
अथो᳚ द्वि॒पद्ये॒व चतु॑ष्पदः॒ प्रति॑\-ष्ठापयति।
द्वि॒पदा॑ हु॒त्वा।
नान्यामुत्त॑रा॒माहु॑तिं जुहुयात्।
यद॒न्यामुत्त॑रा॒माहु॑तिं जुहु॒यात्।
प्र प्र॑ति॒ष्ठाया᳚श्च्यवेत।
द्वि॒पदा॑ अन्त॒तो जु॑होति॒ प्रति॑\-ष्ठित्यै॥४८॥\ip\anuvakamend[बृ॒ह॒त्य॑र्धयति स्थापयति॒ पञ्च॑ च]

%3.9.13.1
प्र॒जा\-प॑तिरश्वमे॒धम॑\-सृजत।
सो᳚ऽस्माथ्सृ॒ष्टो\-ऽपा᳚क्रामत्।
तं य॑ज्ञ\-क्र॒तुभि॒रन्वै᳚च्छत्।
तं य॑ज्ञ\-क्र॒तुभि॒र्नान्व॑विन्दत्।
तमिष्टि॑भि॒रन्वै᳚च्छत्।
तमिष्टि॑भि॒रन्व॑विन्दत्।
तदिष्टी॑नामिष्टि॒\-त्वम्।
यथ्सं॑वथ्स॒रमिष्टि॑\-भि॒र्यज॑ते।
अश्व॑मे॒व तदन्वि॑च्छति।
सा॒वि॒त्रियो॑ भवन्ति॥४९॥\ip

%3.9.13.2
इ॒यं वै स॑वि॒ता।
यो वा अ॒स्यान्नश्य॑ति॒ यो नि॒लय॑ते।
अ॒स्यां वाव तं वि॑न्दन्ति।
न वा इ॒मां कश्च॒नेत्या॑हुः।
ति॒र्यङ्नोर्ध्वोत्ये॑तुमर्ह॒तीति॑।
यथ्सा॑वि॒त्रियो॒ भव॑न्ति।
स॒वि॒तृ\-प्र॑सूत ए॒वैन॑मिच्छति।
ई॒श्व॒रो वा अश्वः॒ प्रमु॑क्तः॒ परां᳚ परा॒वतं॒ गन्तोः᳚।
यथ्सा॒यं धृती᳚र्जु॒होति॑।
अश्व॑स्य॒ यत्यै॒ धृत्यै᳚॥५०॥\ip

%3.9.13.3
यत्प्रा॒तरिष्टि॑भि॒र्यज॑ते।
अश्व॑मे॒व तदन्वि॑च्छति।
यथ्सा॒यं धृती᳚र्जु॒होति॑।
अश्व॑स्यै॒व यत्यै॒ धृत्यै᳚।
तस्मा᳚थ्सा॒यं प्र॒जाः क्षे॒म्या॑ भवन्ति।
यत्प्रा॒तरिष्टि॑भि॒र्यज॑ते।
अश्व॑मे॒व तदन्वि॑च्छति।
तस्मा॒द्दिवा॑ नष्टै॒ष ए॑ति।
यत्प्रा॒तरिष्टि॑भि॒र्यज॑ते सा॒यं धृती᳚र्जु॒होति॑।
अ॒हो॒रा॒त्राभ्या॑मे॒वैन॒मन्वि॑च्छति।
अथो॑ अहोरा॒त्राभ्या॑मे॒वास्मै॑ योग\-क्षे॒मं क॑ल्पयति॥५१॥\ip\anuvakamend[भ॒व॒न्ति॒ धृत्या॑ एन॒मन्वि॑च्छ॒त्येकं॑ च]

%3.9.14.1
अप॒ वा ए॒तस्मा॒च्छ्री रा॒ष्ट्रं क्रा॑मति।
यो᳚ऽश्वमे॒धेन॒ यज॑ते।
ब्रा॒ह्म॒णौ वी॑णागा॒थिनौ॑ गायतः।
श्रि॒या वा ए॒तद्रू॒पम्।
यद्वीणा᳚।
श्रिय॑मे॒वास्मि॒न्तद्ध॑त्तः।
य॒दा खलु॒ वै पुरु॑षः॒ श्रिय॑मश्ञु॒ते।
वीणा᳚\-ऽस्मै वाद्यते।
तदा॑हुः।
यदु॒भौ ब्रा᳚ह्म॒णौ गाये॑ताम्॥५२॥\ip

%3.9.14.2
प्र॒भ्रꣳशु॑कास्मा॒च्छ्रीः स्या᳚त्।
न वै ब्रा᳚ह्म॒णे श्री र॑मत॒ इति॑।
ब्रा॒ह्म॒णो᳚\-ऽन्यो गाये᳚त्।
रा॒ज॒न्यो᳚\-ऽन्यः।
ब्रह्म॒ वै ब्रा᳚ह्म॒णः।
क्ष॒त्रꣳ रा॑ज॒न्यः॑।
तथा॑ हास्य॒ ब्रह्म॑णा च क्ष॒त्रेण॑ चोभ॒यतः॒ श्रीः परि॑गृहीता भवति।
तदा॑हुः।
यदु॒भौ दिवा॒ गाये॑ताम्।
अपा᳚स्माद्रा॒ष्ट्रं क्रा॑मेत्॥५३॥\ip

%3.9.14.3
न वै ब्रा᳚ह्म॒णे रा॒ष्ट्रꣳ र॑मत॒ इति॑।
य॒दा खलु॒ वै राजा॑ का॒मय॑ते।
अथ॑ ब्राह्म॒णं जि॑नाति।
दिवा᳚ ब्राह्म॒णो गा॑येत्।
नक्तꣳ॑ राज॒न्यः॑।
ब्रह्म॑णो॒ वै रू॒पमहः॑।
क्ष॒त्रस्य॒ रात्रिः॑।
तथा॑ हास्य॒ ब्रह्म॑णा च क्ष॒त्रेण॑ चोभ॒यतो॑ रा॒ष्ट्रं परि॑गृहीतं भवति।
इत्य॑ददा॒ इत्य॑यजथा॒ इत्य॑पच॒ इति॑ ब्राह्म॒णो गाये᳚त्।
इ॒ष्टा॒पू॒र्तं वै ब्रा᳚ह्म॒णस्य॑॥५४॥\ip

%3.9.14.4
इ॒ष्टा॒पू॒र्तेनै॒वैन॒ꣳ॒ स सम॑र्धयति।
इत्य॑जिना॒ इत्य॑युध्यथा॒ इत्य॒मुꣳ स॑ङ्ग्रा॒मम॑ह॒न्निति॑ राज॒न्यः॑।
यु॒द्धं वै रा॑ज॒न्य॑स्य।
यु॒द्धेनै॒वैन॒ꣳ॒ स सम॑र्धयति।
अकॢ॑प्ता॒ वा ए॒तस्य॒र्तव॒ इत्या॑हुः।
यो᳚ऽश्वमे॒धेन॒ यज॑त॒ इति॑।
ति॒स्रो᳚\-ऽन्यो गाय॑ति ति॒स्रो᳚\-ऽन्यः।
षट्थ्सम्प॑द्यन्ते।
षड्वा ऋ॒तवः॑।
ऋ॒तूने॒वास्मै॑ कल्पयतः।
ताभ्याꣳ॑ स॒ꣴ॒स्थाया᳚म्।
अ॒नो॒यु॒क्ते च॑ श॒ते च॑ ददाति।
श॒तायुः॒ पुरु॑षः श॒तेन्द्रि॑यः।
आयु॑ष्ये॒वेन्द्रि॒ये प्रति॑ तिष्ठति॥५५॥\ip\anuvakamend[गाये॑ताङ्क्रामेद्ब्राह्म॒णस्य॑ कल्पयतश्च॒त्वारि॑ च]

%3.9.15.1
सर्वे॑षु॒ वा ए॒षु लो॒केषु॑ मृ॒त्यवो॒\-ऽन्वाय॑त्ताः।
तेभ्यो॒ यदाहु॑ती॒र्न जु॑हु॒यात्।
लो॒केलो॑क एनं मृ॒त्युर्वि॑न्देत्।
मृ॒त्यवे॒ स्वाहा॑ मृ॒त्यवे॒ स्वाहेत्य॑भिपू॒र्वमाहु॑तीर्जुहोति।
लो॒काल्लो॑कादे॒व मृ॒त्युमव॑यजते।
नैनं॑ लो॒केलो॑के मृ॒त्युर्वि॑न्दति।
यद॒मुष्मै॒ स्वाहा॒\-ऽमुष्मै॒ स्वाहेति॒ जुह्व॑थ्स॒ञ्चक्षी॑त।
ब॒हुं मृ॒त्युम॒मित्रं॑ कुर्वीत।
मृ॒त्यवे॒ स्वाहेत्येक॑स्मा ए॒वैकां᳚ जुहुयात्।
एको॒ वा अ॒मुष्मिँ॑ल्लो॒के मृ॒त्युः॥५६॥\ip

%3.9.15.2
अ॒श॒न॒या॒ मृ॒त्युरे॒व।
तमे॒वामुष्मिँ॑ल्लो॒के\-ऽव॑यजते।
भ्रू॒ण॒ह॒त्यायै॒ स्वाहेत्य॑वभृ॒थ आहु॑तिं जुहोति।
भ्रू॒ण॒ह॒त्यामे॒वाव॑ यजते।
तदा॑हुः।
यद्भ्रू॑णह॒त्या पा॒त्र्याऽथ॑।
कस्मा᳚द्य॒ज्ञेऽपि॑ क्रियत॒ इति॑।
अमृ॑त्यु॒र्वा अ॒न्यो भ्रू॑णह॒त्याया॒ इत्या॑हुः।
भ्रू॒ण॒ह॒त्या वाव मृ॒त्युरिति॑।
यद्भ्रू॑णह॒त्यायै॒ स्वाहेत्य॑वभृ॒थ आहु॑तिं जु॒होति॑॥५७॥\ip

%3.9.15.3
मृ॒त्युमे॒वाऽऽहु॑त्या तर्पयि॒त्वा प॑रि॒पाणं॑ कृ॒त्वा।
भ्रू॒ण॒घ्ने भे॑ष॒जं क॑रोति।
ए॒ताꣳ ह॒ वै मु॑ण्डि॒भ औ॑दन्य॒वः।
भ्रू॒ण॒ह॒त्यायै॒ प्राय॑श्चित्तिं वि॒दां च॑कार।
यो हा॒स्यापि॑ प्र॒जायां᳚ ब्राह्म॒णꣳ हन्ति॑।
सर्व॑स्मै॒ तस्मै॑ भेष॒जं क॑रोति।
जु॒म्ब॒काय॒ स्वाहेत्य॑वभृ॒थ उ॑त्त॒मामाहु॑तिं जुहोति।
वरु॑णो॒ वै जु॑म्ब॒कः।
अ॒न्त॒त ए॒व वरु॑ण॒मव॑यजते।
ख॒ल॒तेर्वि॑क्लि॒धस्य॑ शु॒क्लस्य॑ पिङ्गा॒क्षस्य॑ मू॒र्धं जु॑होति।
ए॒तद्वै वरु॑णस्य रू॒पम्।
रू॒पेणै॒व वरु॑ण॒मव॑यजते॥५८॥\ip\anuvakamend[लो॒के मृ॒त्युर्जु॒होति॑ मू॒र्धं जु॑होति॒ द्वे च॑]

%3.9.16.1
वा॒रु॒णो वा अश्वः॑।
तं दे॒वत॑या॒ व्य॑र्धयति।
यत्प्रा॑जाप॒त्यं क॒रोति॑।
नमो॒ राज्ञे॒ नमो॒ वरु॑णा॒येत्या॑ह।
वा॒रु॒णो वा अश्वः॑।
स्वयै॒वैनं॑ दे॒वत॑या॒ सम॑र्धयति।
नमोऽश्वा॑य॒ नमः॑ प्र॒जा\-प॑तय॒ इत्या॑ह।
प्रा॒जा॒प॒त्यो वा अश्वः॑।
स्वयै॒वैनं॑ दे॒वत॑या॒ सम॑र्धयति।
नमोऽधि॑पतय॒ इत्या॑ह॥५९॥\ip

%3.9.16.2
धर्मो॒ वा अधि॑पतिः।
धर्म॑मे॒वाव॑ रुन्धे।
अधि॑पतिर॒स्यधि॑पतिं मा कु॒र्वधि॑पतिर॒हं प्र॒जानां᳚ भूयास॒मित्या॑ह।
अधि॑पतिमे॒वैनꣳ॑ समा॒नानां᳚ करोति।
मां धे॑हि॒ मयि॑ धे॒हीत्या॑ह।
आ॒शिष॑\-मे॒वैतामा शा᳚स्ते।
उ॒पाकृ॑ताय॒ स्वाहेत्यु॒पाकृ॑ते जुहोति।
आल॑ब्धाय॒ स्वाहेति॒ नियु॑क्ते जुहोति।
हु॒ताय॒ स्वाहेति॑ हु॒ते जु॑होति।
ए॒षां लो॒काना॑म॒भिजि॑त्यै॥६०॥\ip

%3.9.16.3
प्र वा ए॒ष ए॒भ्यो लो॒केभ्य॑श्च्यवते।
यो᳚ऽश्वमे॒धेन॒ यज॑ते।
आ॒ग्ने॒यमै᳚न्द्रा॒ग्नमा᳚श्वि॒नम्।
तान्प॒शूनाल॑भते॒ प्रति॑\-ष्ठित्यै।
यदा᳚ग्ने॒यो भव॑ति।
अ॒ग्निः सर्वा॑ दे॒वताः᳚।
दे॒वता॑ ए॒वाव॑ रुन्धे।
ब्रह्म॒ वा अ॒ग्निः।
क्ष॒त्रमिन्द्रः॑।
यदै᳚न्द्रा॒ग्नो भव॑ति॥६१॥\ip

%3.9.16.4
ब्र॒ह्म॒क्ष॒त्रे ए॒वाव॑ रुन्धे।
यदा᳚श्वि॒नो भव॑ति।
आ॒शिषा॒मव॑रुद्ध्यै।
त्रयो॑ भवन्ति।
त्रय॑ इ॒मे लो॒काः।
ए॒ष्वे॑व लो॒केषु॒ प्रति॑ तिष्ठति।
अ॒ग्नये\-ऽꣳ॑हो॒मुचे॒\-ऽष्टा\-क॑पाल॒ इति॒ दश॑हविष॒मिष्टिं॒ निर्व॑पति।
दशा᳚क्षरा वि॒राट्।
अन्नं॑ वि॒राट्।
वि॒राजै॒वान्नाद्य॒मव॑ रुन्धे।
अ॒ग्नेर्म॑न्वे प्रथ॒मस्य॒ प्रचे॑तस॒ इति॑ याज्यानुवा॒क्या॑ भवन्ति सर्व॒त्वाय॑॥६२॥\ip\anuvakamend[अधि॑पतय॒ इत्या॑हा॒भि॑जित्या ऐन्द्रा॒ग्नो भव॑ति रुन्ध॒ एकं॑ च]

%3.9.17.1
यद्यश्व॑मुप॒तप॑द्वि॒न्देत्।
आ॒ग्ने॒यम॒ष्टा\-क॑पालं॒ निर्व॑पेत्।
सौ॒म्यं च॒रुम्।
सा॒वि॒त्रम॒ष्टा\-क॑पालम्।
यदा᳚ग्ने॒यो भव॑ति।
अ॒ग्निः सर्वा॑ दे॒वताः᳚।
दे॒वता॑भिरे॒वैनं॑ भिषज्यति।
यथ्सौ॒म्यो भव॑ति।
सोमो॒ वा ओष॑धीना॒ꣳ॒ राजा᳚।
याभ्य॑ ए॒वैनं॑ वि॒न्दति॑॥६३॥\ip

%3.9.17.2
ताभि॑रे॒वैनं॑ भिषज्यति।
यथ्सा॑वि॒त्रो भव॑ति।
स॒वि॒तृप्र॑सूत ए॒वैनं॑ भिषज्यति।
ए॒ताभि॑रे॒वैनं॑ दे॒वता॑भिर्भिषज्यति।
अ॒ग॒दो है॒व भ॑वति।
पौ॒ष्णं च॒रुं निर्व॑पेत्।
यदि॑ श्लो॒णः स्यात्।
पू॒षा वै श्लौण्य॑स्य भि॒षक्।
स ए॒वैनं॑ भिषज्यति।
अश्लो॑णो है॒व भ॑वति॥६४॥\ip

%3.9.17.3
रौ॒द्रं च॒रुं निर्व॑पेत्।
यदि॑ मह॒ती दे॒वता॑\-ऽभि॒मन्ये॑त।
ए॒त॒द्दे॒व॒त्यो॑ वा अश्वः॑।
स्वयै॒वैनं॑ दे॒वत॑या भिषज्यति।
अ॒ग॒दो है॒व भ॑वति।
वै॒श्वा॒न॒रं द्वाद॑शकपालं॒ निर्व॑पेन्मृगाख॒रे यदि॒ नाऽऽगच्छे᳚त्।
इ॒यं वा अ॒ग्निर्वै᳚श्वान॒रः।
इ॒यमे॒वैन॑म॒र्चिभ्यां᳚ परि॒रोध॒मान॑यति।
आहै॒व सुत्य॒मह॑र्गच्छति।
यद्य॑धी॒यात्॥६५॥\ip

%3.9.17.4
अ॒ग्नये\-ऽꣳ॑हो॒मुचे॒\-ऽष्टा\-क॑पालः।
सौ॒र्यं पयः॑।
वा॒य॒व्य॑ आज्य॑भागः।
यज॑मानो॒ वा अश्वः॑।
अꣳह॑सा॒ वा ए॒ष गृ॑ही॒तः।
यस्याश्वो॒ मेधा॑य॒ प्रोक्षि॑तो॒\-ऽध्येति॑।
यदꣳ॑हो॒मुचे॑ नि॒र्वप॑ति।
अꣳह॑स ए॒व तेन॑ मुच्यते।
यज॑मानो॒ वा अश्वः॑।
रेत॑सा॒ वा ए॒ष व्यृ॑ध्यते॥६६॥\ip

%3.9.17.5
यस्याश्वो॒ मेधा॑य॒ प्रोक्षि॑तो॒\-ऽध्येति॑।
सौ॒र्यꣳ रेतः॑।
यथ्सौ॒र्यं पयो॒ भव॑ति।
रेत॑सै॒वैन॒ꣳ॒ स सम॑र्धयति।
यज॑मानो॒ वा अश्वः॑।
गर्भै॒र्वा ए॒ष व्यृ॑ध्यते।
यस्याश्वो॒ मेधा॑य॒ प्रोक्षि॑तो॒\-ऽध्येति॑।
वा॒य॒व्या॑ गर्भाः᳚।
यद्वा॑य॒व्य॑ आज्य॑भागो॒ भव॑ति।
गर्भै॑रे॒वैन॒ꣳ॒ स सम॑र्धयति।
अथो॒ यस्यै॒षा\-ऽश्व॑मे॒धे प्राय॑श्चित्तिः क्रि॒यते᳚।
इ॒ष्ट्वा वसी॑यान्भवति॥६७॥\ip\anuvakamend[वि॒न्दत्यश्लो॑णो है॒व भ॑वत्यधी॒यादृ॑ध्यते॒ गर्भै॑रे॒वैन॒ꣳ॒ स सम॑र्धयति॒ द्वे च॑]

%3.9.18.1
तदा॑हुः।
द्वाद॑श ब्रह्मौद॒नान्थ्सꣴस्थि॑ते॒ निर्व॑पेत्।
द्वा॒द॒शभि॒र्वेष्टि॑\-भिर्यजे॒तेति॑।
यदिष्टि॑भि॒र्यजे॑त।
उ॒प॒नामु॑क एनं य॒ज्ञः स्या᳚त्।
पापी॑या॒ꣴ॒स्तु स्या᳚त्।
आ॒प्तानि॒ वा ए॒तस्य॒ छन्दाꣳ॑सि।
य ई॑जा॒नः।
तानि॒ क ए॒ताव॑दाशु॒ पुनः॒ प्रयु॑ञ्जी॒तेति॑।
सर्वा॒ वै सꣴस्थि॑ते य॒ज्ञे वागा᳚प्यते॥६८॥\ip

%3.9.18.2
साप्ता भ॑वति या॒तया᳚म्नी।
क्रू॒रीकृ॑तेव॒ हि भव॒त्यरु॑ष्कृता।
सा न पुनः॑ प्र॒युज्येत्या॑हुः।
द्वाद॑शै॒व ब्र॑ह्मौद॒नान्थ्सꣴस्थि॑ते॒ निर्व॑पेत्।
प्र॒जा\-प॑ति॒र्वा ओ॑द॒नः।
य॒ज्ञः प्र॒जा\-प॑तिः।
उ॒प॒नामु॑क एनं य॒ज्ञो भ॑वति।
न पापी॑यान्भवति।
द्वाद॑श भवन्ति।
द्वाद॑श॒मासाः᳚ संवथ्स॒रः।
सं॒व॒थ्स॒र ए॒व प्रति॑ तिष्ठति॥६९॥\ip\anuvakamend[आ॒प्य॒ते॒ सं॒व॒थ्स॒र एकं॑ च]

%3.9.19.1
ए॒ष वै वि॒भूर्नाम॑ य॒ज्ञः।
सर्वꣳ॑ ह॒ वै तत्र॑ वि॒भु भ॑वति।
यत्रै॒\-तेन॑ य॒ज्ञेन॒ यज॑न्ते।
ए॒ष वै प्र॒भूर्नाम॑ य॒ज्ञः।
सर्वꣳ॑ ह॒ वै तत्र॑ प्र॒भु भ॑वति।
यत्रै॒तेन॑ य॒ज्ञेन॒ यज॑न्ते।
ए॒ष वा ऊर्ज॑स्वा॒न्नाम॑ य॒ज्ञः।
सर्वꣳ॑ ह॒ वै तत्रो\-र्ज॑\-स्वद्\-भवति।
यत्रै॒तेन॑ य॒ज्ञेन॒ यज॑न्ते।
ए॒ष वै पय॑स्वा॒न्नाम॑ य॒ज्ञः॥७०॥\ip

%3.9.19.2
सर्वꣳ॑ ह॒ वै तत्र॒ पय॑स्वद्भवति।
यत्रै॒तेन॑ य॒ज्ञेन॒ यज॑न्ते।
ए॒ष वै विधृ॑तो॒ नाम॑ य॒ज्ञः।
सर्वꣳ॑ ह॒ वै तत्र॒ विधृ॑तं भवति।
यत्रै॒तेन॑ य॒ज्ञेन॒ यज॑न्ते।
ए॒ष वै व्यावृ॑त्तो॒ नाम॑ य॒ज्ञः।
सर्वꣳ॑ ह॒ वै तत्र॒ व्यावृ॑त्तं भवति।
यत्रै॒तेन॑ य॒ज्ञेन॒ यज॑न्ते।
ए॒ष वै प्रति॑\-ष्ठितो॒ नाम॑ य॒ज्ञः।
सर्वꣳ॑ ह॒ वै तत्र॒ प्रति॑\-ष्ठितं भवति॥७१॥\ip

%3.9.19.3
यत्रै॒तेन॑ य॒ज्ञेन॒ यज॑न्ते।
ए॒ष वै ते॑ज॒स्वी नाम॑ य॒ज्ञः।
सर्वꣳ॑ ह॒ वै तत्र॑ तेज॒स्वि भ॑वति।
यत्रै॒तेन॑ य॒ज्ञेन॒ यज॑न्ते।
ए॒ष वै ब्र॑ह्म\-वर्च॒सी नाम॑ य॒ज्ञः।
आ ह॒ वै तत्र॑ ब्राह्म॒णो ब्र॑ह्म\-वर्च॒सी जा॑यते।
यत्रै॒तेन॑ य॒ज्ञेन॒ यज॑न्ते।
ए॒ष वा अ॑तिव्या॒धी नाम॑ य॒ज्ञः।
आ ह॒ वै तत्र॑ राज॒न्यो॑\-ऽतिव्या॒धी जा॑यते।
यत्रै॒तेन॑ य॒ज्ञेन॒ यज॑न्ते।
ए॒ष वै दी॒र्घो नाम॑ य॒ज्ञः।
दी॒र्घायु॑षो ह॒ वै तत्र॑ मनु॒ष्या॑ भवन्ति।
यत्रै॒तेन॑ य॒ज्ञेन॒ यज॑न्ते।
ए॒ष वै कॢ॒प्तो नाम॑ य॒ज्ञः।
कल्प॑ते ह॒ वै तत्र॑ प्र॒जाभ्यो॑ योगक्षे॒मः।
यत्रै॒तेन॑ य॒ज्ञेन॒ यज॑न्ते॥७२॥\ip\anuvakamend[पय॑स्वा॒न्नाम॑ य॒ज्ञः प्रति॑\-ष्ठितं भवति॒ यत्रै॒तेन॑ य॒ज्ञेन॒ यज॑न्ते॒ षट्च॑ (ए॒ष वै विभूः प्र॒भूरूर्ज॑स्वा॒न्पय॑स्वा॒न् विधृ॑तो॒ व्यावृ॑त्तः॒ प्रति॑\-ष्ठितस्तेज॒स्वी ब्र॑ह्म\-वर्च॒स्य॑तिव्या॒धी दी॒र्घः कॢ॒प्तो द्वाद॑श॥)]

%3.9.20.1
ता॒र्प्येणाश्व॒ꣳ॒ संज्ञ॑पयन्ति।
य॒ज्ञो वै ता॒र्प्यम्।
य॒ज्ञेनै॒वैन॒ꣳ॒ सम॑र्धयन्ति।
या॒मेन॒ साम्ना᳚ प्रस्तो॒ता\-ऽनूप॑तिष्ठते।
य॒म॒लो॒कमे॒वैनं॑ गमयति।
ता॒र्प्ये च॑ कृत्यधीवा॒से चाश्व॒ꣳ॒ संज्ञ॑पयन्ति।
ए॒तद्वै प॑शू॒नाꣳ रू॒पम्।
रू॒पेणै॒व प॒शूनव॑ रुन्धे।
हि॒र॒ण्य॒क॒शि॒पु भ॑वति।
तेज॒सो\-ऽव॑रुद्ध्यै॥७३॥\ip

%3.9.20.2
रु॒क्मो भ॑वति।
सु॒व॒र्गस्य॑ लो॒कस्यानु॑ख्यात्यै।
अश्वो॑ भवति।
प्र॒जा\-प॑ते॒राप्त्यै᳚।
अ॒स्य वै लो॒कस्य॑ रू॒पं ता॒र्प्यम्।
अ॒न्तरि॑क्षस्य कृत्यधीवा॒सः।
दि॒वो हि॑रण्यकशि॒पु।
आ॒दि॒त्यस्य॑ रु॒क्मः।
प्र॒जा\-प॑ते॒रश्वः॑।
इ॒ममे॒व लो॒कं ता॒र्प्येणा᳚ऽऽप्नोति॥७४॥\ip

%3.9.20.3
अ॒न्तरि॑क्षं कृत्यधीवा॒सेन॑।
दिवꣳ॑ हिरण्यकशि॒पुना᳚।
आ॒दि॒त्यꣳ रु॒क्मेण॑।
अश्वे॑नै॒व मेध्ये॑न प्र॒जा\-प॑तेः॒ सायु॑ज्यꣳ स\-लो॒क\-ता॑\-माप्नोति।
ए॒तासा॑मे॒व दे॒वता॑ना॒ꣳ॒ सायु॑ज्यम्।
सा॒र्ष्टिताꣳ॑ समान\-लो॒क\-ता॑\-माप्नोति।
यो᳚ऽश्वमे॒धेन॒ यज॑ते।
य उ॑ चैनमे॒वं वेद॑॥७५॥\ip\anuvakamend[अव॑रुध्या आप्नोत्य॒ष्टौ च॑]

%3.9.21.1
आ॒दि॒त्याश्चाङ्गि॑रसश्च सुव॒र्गे लो॒के᳚\-ऽस्पर्धन्त।
तेऽङ्गि॑रस आदि॒त्येभ्यः॑।
अ॒मुमा॑दि॒त्यमश्वꣴ॑ श्वे॒तं भू॒तं दक्षि॑णामनयन्।
ते᳚ऽब्रुवन्।
यन्नो ने᳚ष्ट।
स वर्यो॑ भू॒दिति॑।
तस्मा॒दश्व॒ꣳ॒ सव॒र्येत्याह्व॑यन्ति।
तस्मा᳚द्य॒ज्ञे वरो॑ दीयते।
यत्प्र॒जा\-प॑ति॒रा\-ल॒ब्धो\-ऽश्वो\-ऽभ॑वत्।
तस्मा॒दश्वो॒ नाम॑॥७६॥\ip

%3.9.21.2
यच्छ्वय॒दरु॒रासी᳚त्।
तस्मा॒दर्वा॒ नाम॑।
यथ्स॒द्यो वाजा᳚न्थ्स॒म\-ज॑यत्।
तस्मा᳚द्वा॒जी नाम॑।
यदसु॑राणां लो॒कानाद॑त्त।
तस्मा॑दादि॒त्यो नाम॑।
अ॒ग्निर्वा अ॑श्वमे॒धस्य॒ योनि॑रा॒\-यत॑नम्।
सूर्यो॒\-ऽग्नेर्योनि॑रा॒\-यत॑नम्।
यद॑श्वमे॒धे᳚\-ऽग्नौ चित्य॑ उत्तरवे॒दिमु॑प॒वप॑ति।
योनि॑मन्तमे॒वैन॑मा॒यत॑नवन्तं करोति॥७७॥\ip

%3.9.21.3
योनि॑माना॒यत॑नवान्भवति।
य ए॒वं वेद॑।
प्रा॒णा॒पा॒नौ वा ए॒तौ दे॒वाना᳚म्।
यद॑र्काश्वमे॒धौ।
प्रा॒णा॒पा॒नावे॒वाव॑ रुन्धे।
ओजो॒ बलं॒ वा ए॒तौ दे॒वाना᳚म्।
यद॑र्काश्वमे॒धौ।
ओजो॒ बल॑मे॒वाव॑ रुन्धे।
अ॒ग्निर्वा अ॑श्वमे॒धस्य॒ योनि॑रा॒यत॑नम्।
सूर्यो॒\-ऽग्नेर्योनि॑रा॒\-यत॑नम्।
यद॑श्वमे॒धे᳚\-ऽग्नौ चित्य॑ उत्तरवे॒दिं चि॒नोति॑।
ताव॑र्काश्वमे॒धौ।
अ॒र्का॒श्व॒मे॒धावे॒वाव॑ रुन्धे।
अथो॑ अर्काश्वमे॒धयो॑रे॒व प्रति॑ तिष्ठति॥७८॥\ip\anuvakamend[नाम॑ करोति॒ सूर्यो॒\-ऽग्नेर्योनि॑रा॒यत॑नञ्च॒त्वारि॑ च]

%3.9.22.1
प्र॒जा\-प॑तिं॒ वै दे॒वाः पि॒तरम्᳚।
प॒शुं भू॒तं मेधा॒याऽऽऽल॑भन्त।
तमा॒लभ्योपा॑वसन्।
प्रा॒तर्यष्टा᳚स्मह॒ इति॑।
एकं॒ वा ए॒तद्दे॒वाना॒महः॑।
यथ्सं॑वथ्स॒रः।
तस्मा॒दश्वः॑ पु॒रस्ता᳚थ्संवथ्स॒र आल॑भ्यते।
यत्प्र॒जा\-प॑ति॒\-रा\-ल॒ब्धो\-ऽश्वो\-ऽभ॑वत्।
तस्मा॒दश्वः॑।
यथ्स॒द्यो मेधो\-ऽभ॑वत्॥७९॥\ip

%3.9.22.2
तस्मा॑दश्वमे॒धः।
वेदु॒को\-ऽश्व॑मा॒शुं भ॑वति।
य ए॒वं वेद॑।
यद्वै तत्प्र॒जा\-प॑ति॒राल॒ब्धो\-ऽश्वो\-ऽभ॑वत्।
तस्मा॒दश्वः॑ प्र॒जा\-प॑तेः पशू॒नामनु॑रूपतमः।
आऽस्य॑ पु॒त्रः प्रति॑\-रूपो जायते।
य ए॒वं वेद॑।
सर्वा॑णि भू॒तानि॑ स॒म्भृत्याऽऽल॑भते।
समे॑नं दे॒वास्तेज॑से ब्रह्मवर्च॒साय॑ भरन्ति।
यो᳚ऽश्वमे॒धेन॒ यज॑ते॥८०॥\ip

%3.9.22.3
य उ॑ चैनमे॒वं वेद॑।
ए॒तद्वै तद्दे॒वा ए॒तान्दे॒वता᳚म्।
प॒शुं भू॒तं मेधा॒याऽऽऽल॑भन्त।
य॒ज्ञमे॒व।
य॒ज्ञेन॑ य॒ज्ञम॑यजन्त दे॒वाः।
का॒म॒प्रं य॒ज्ञम॑कुर्वत।
ते॑ऽमृत॒त्वम॑कामयन्त।
ते॑ऽमृत॒त्वम॑गच्छन्।
यो᳚ऽश्वमे॒धेन॒ यज॑ते।
दे॒वाना॑मे॒वाय॑नेनैति॥८१॥\ip

%3.9.22.4
प्रा॒जा॒प॒त्येनै॒व य॒ज्ञेन॑ यजते काम॒प्रेण॑।
अपु॑नर्मारमे॒व ग॑च्छति।
ए॒तस्य॒ वै रू॒पेण॑ पु॒रस्ता᳚त्प्राजाप॒त्यमृ॑ष॒भं तू॑प॒रं ब॑हुरू॒पमाल॑भते।
सर्वे᳚भ्यः॒ कामे᳚भ्यः।
सर्व॒स्याऽऽप्त्यै᳚।
सर्व॑स्य॒ जित्यै᳚।
सर्व॑मे॒व तेना᳚\-ऽऽ\-प्नोति।
सर्वं॑ जयति।
यो᳚ऽश्वमे॒धेन॒ यज॑ते।
य उ॑ चैनमे॒वं वेद॑॥८२॥\ip\anuvakamend[मेधो\-ऽभ॑व॒द्यज॑त एति॒ वेद॑]

%3.9.23.1
यो वा अश्व॑स्य॒ मेध्य॑स्य॒ लोम॑नी॒ वेद॑।
अश्व॑स्यै॒व मेध्य॑स्य॒ लोमं॑ लोमं जुहोति।
अ॒हो॒रा॒त्रे वा अश्व॑स्य॒ मेध्य॑स्य॒ लोम॑नी।
यथ्सा॒यं प्रा॑तर्जु॒होति॑।
अश्व॑स्यै॒व मेध्य॑स्य॒ लोमं॑ लोमं जुहोति।
ए॒तद॑नुकृति ह स्म॒ वै पु॒रा।
अश्व॑स्य॒ मेध्य॑स्य॒ लोमं॑ लोमं जुह्वति।
यो वा अश्व॑स्य॒ मेध्य॑स्य प॒दे वेद॑।
अश्व॑स्यै॒व मेध्य॑स्य प॒देप॑दे जुहोति।
द॒र्॒श॒पू॒र्ण॒मा॒सौ वा अश्व॑स्य॒ मेध्य॑स्य प॒दे॥८३॥\ip

%3.9.23.2
यद्द॑र्‌\mbox{}शपूर्णमा॒सौ यज॑ते।
अश्व॑स्यै॒व मेध्य॑स्य प॒देप॑दे जुहोति।
ए॒तद॑नुकृति ह स्म॒ वै पु॒रा।
अश्व॑स्य॒ मेध्य॑स्य प॒देप॑दे जुह्वति।
यो वा अश्व॑स्य॒ मेध्य॑स्य वि॒वर्त॑नं॒ वेद॑।
अश्व॑स्यै॒व मेध्य॑स्य वि॒वर्त॑नेविवर्तने जुहोति।
अ॒सौ वा आ॑दि॒त्यो\-ऽश्वः॑।
स आ॑हव॒नीय॒माग॑च्छति।
तद्विव॑र्तते।
यद॑ग्निहो॒त्रं जु॒होति॑।
अश्व॑स्यै॒व मेध्य॑स्य वि॒वर्त॑नेविवर्तने जुहोति।
ए॒तद॑नुकृति ह स्म॒ वै पु॒रा।
अ॑श्वस्य॒ मेध्य॑स्य वि॒वर्त॑नेविवर्तने जुह्वति॥८४॥\ip\anuvakamend[प॒दे अ॑ग्निहो॒त्रं जु॒होति॒ त्रीणि॑ च]

\prashnaend{प्र॒जा\-प॑ति॒स्तम॑ष्टादशिभिः॑ प्र॒जा\-प॑तिरकामयतो॒भाव॒स्मै यु॒ञ्जन्ति॒ तेज॒सा\-ऽप॑प्राणा अप॒श्रीरू॒र्ध्वां प्र॒जा\-प॑तिः प्रे॒णाऽनु॑ प्रथ॒मेन॑ प्र॒जा\-प॑तिरकामयत म॒हान्वै᳚श्वदे॒वो वा अश्वो\-ऽश्व॑स्य प्र॒जा\-प॑ति॒स्तं य॑ज्ञक्र॒तुभि॒रप॒श्रीर्ब्रा᳚ह्म॒णौ सर्वे॑षु वारु॒णो यद्यश्व॒न्तदा॑हुरे॒ष वै वि॒भूस्ता॒र्प्येणा॑दि॒त्याः प्र॒जा\-प॑तिं पि॒तरं॒ यो वा अश्व॑स्य॒ मेध्य॑स्य॒ लोम॑नी॒ त्रयो॑विꣳशतिः॥२३॥}{प्र॒जा\-प॑तिर॒स्मिँल्लो॒क उ॑त्तर॒तः श्रिय॑मे॒व प्र॒जा\-प॑तिरकामयत म॒हान्यत्प्रा॒तः प्र वा ए॒ष ए॒भ्यो लो॒केभ्यः॒ सर्वꣳ॑ ह॒ वै तत्र॒ पयः॑ स्व॒द्य उ॑ चैनमे॒वं वेद॑ च॒त्वार्यशी॑तिः॥८४॥}{प्र॒जा\-प॑तिरश्वमे॒धं जु॑ह्वति॥}{हरिः॑ ओम्॥}{इति श्रीकृष्णयजुर्वेदीयतैत्तिरीयब्राह्मणे तृतीयाष्टके नवमः प्रपाठकः समाप्तः॥}
