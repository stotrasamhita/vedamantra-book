% !TeX program = XeLaTeX
% !TeX root = ../vedamantrabook.tex

\chapt{यजुर्वेद उपाकर्म}

\sect{कामोकार्षीन्मन्त्र जपम्}

\sect{महा-सङ्कल्पम्}
आचम्य, दर्भेषु आसीनः, दर्भान् धारयमाणः । शुक्लाम्बरधरं + शान्तये । प्राणानायम्य । 
    ममोपात्त समस्तदुरितक्षयद्वारा श्रीपरमेश्वरप्रीत्यर्थं   तदेव लग्नं सुदिनं तदेव ताराबलं चन्द्रबलं तदेव विद्याबलं दैवबलं तदेव लक्ष्मीपते ते अङ्घ्रियुगं स्मरामि।
    \twolineshloka*
    {अपवित्रः पवित्रो वा सर्वावस्थागतोऽपि वा}
    {यः स्मरेत्पुण्डरीकाक्षं स बाह्याभ्यन्तरः शुचिः}

     \twolineshloka*
  {मानसं वाचिकं पापं कर्मणा समुपार्जितम्}
{श्रीरामस्मरणेनैव व्यपोहति न संशयः }

श्रीराम राम राम। 

    \twolineshloka*
{तिथिर्विष्णुस्तथा वारो नक्षत्रं विष्णुरेव च}
{योगश्च करणं चैव सर्वं विष्णुमयं जगत्}

श्रीगोविन्द गोविन्द गोविन्द। 

अद्य श्री भगवतः आदिविष्णोः आदिनारायणस्य अचिन्त्यया अपरिमितया शक्त्या भ्रियमाणस्य महाजलौघस्य मध्ये
परिभ्रमताम् अनेककोटिब्रह्माण्डानाम् एकतमे पृथिव्यप्तेजो-वाय्वाकाशाहङ्कार-महदव्यक्तैः आवरणैः
आवृतेऽस्मिन् महति ब्रह्माण्डकरण्डमध्ये भूमण्डले आधारशक्ति आदिकूर्म अनन्तादि अष्टदिग्गजोपरि
प्रतिष्ठितस्य अतल वितल सुतल रसातल तलातल महातल पाताळाख्य लोकसप्तकस्य उपरितले पुण्यकृताम् निवासभूते
भुवर्लोक सुवर्लोक महोलोक जनोलोक तपोलोक सत्यलोकाख्य लोकषट्कस्य अधोभागे महानाळायमानफणिराजशेषस्य सहस्र
फणामणि मण्डल मण्डिते दिग्दन्ति शुण्डादण्ड उत्तम्भिते पञ्चाशत्कोटियोजन विस्तीर्णे लोकालोक अचलेन
वलयिते लवणेक्षु सुरासर्पि दधि क्षीर शुद्धोदकार्णवैः परिवृते
जम्बू-प्लक्ष-कुश-क्रौञ्च-शाक-शाल्मलि-पुष्कराख्य-सप्तद्वीपानां मध्ये जम्बूद्वीपे
भारत-किम्पुरुष-हरि-इलावृत-भद्राश्व-केतुमाल-हिरण्मय-रमणक-कुरु-वर्षाख्य नववर्षाणां मध्ये भारतवर्षे
इन्द्रकशेरु-ताम्र-गभस्ति-पुन्नाग-गन्धर्व-सौम्य-वरुण-भरत-खण्डानां मध्ये भरतखण्डे
सुमेरु-निषध-हेमकूट-हिमाचल-माल्यवत्-पारियात्रक -गन्धमादन-कैलास-विन्ध्याचलादि-महाशैलमध्ये
दण्डकारण्य-चम्पकारण्य-विन्ध्यारण्य-वीक्षारण्य-श्वेतारण्य-वेदारण्यादि-अनेकपुण्यारण्यानां मध्ये
कर्मभूमौ दण्डकारण्ये समभूमिरेखायाः दक्षिणदिग्भागे श्रीशैलस्य आग्नेयदिग्भागे रामसेतोः उत्तरदिग्भागे
गङ्गा-यमुना-सरस्वती-भीमरथी-गौतमी-नर्मदा-गण्डकी-कृष्णवेणी-तुङ्गभद्रा-चन्द्रभागा-मलापहा-कावेरी-कपिला-ताम्रपर्णी-वेगवती-पिनाकिनी-क्षीरनद्यादि
अनेक-महानदी-विराजिते
इन्द्रप्रस्थ-यमप्रस्थ-अवन्तिकापुरी-हस्तिनापुरी-अयोध्यापुरी-मथुरापुरी-मायापुरी-काशीपुरी-काञ्चीपुरी-द्वारकादि-अनेकपुण्यपुरी-विराजिते
वाराणसी-चिदम्बर-श्रीशैल-अहोबिल-वेङ्कटाचल-रामसेतु-जम्बुकेश्वर-कुम्भकोण-हालास्य-गोकर्ण-अनन्तशयन-गया-प्रयागादि
अनेकपुण्यक्षेत्र-परिवृते सकलजगत्स्रष्टुः परार्धद्वयजीविनः ब्रह्मणः प्रथमे परार्धे पञ्चाशत्
अब्दात्मके अतीते द्वितीये परार्धे पञ्चाशद्-अब्दादौ प्रथमे वर्षे प्रथमे मासे प्रथमे पक्षे प्रथमे
दिवसे अह्नि द्वितीये यामे तृतीये मुहूर्ते
पार्थिव-कूर्म-प्रलयानन्त-श्वेतवराह-ब्राह्म-सावित्र्याख्य-सप्त-कल्पानां मध्ये श्वेतवराहकल्पे
स्वायम्भुव-स्वारोचिष-उत्तम-तामस-रैवत-चाक्षुषाख्येषु षट्सु मनुषु अतीतेषु सप्तमे वैवस्वतमन्वन्तरे
अष्टाविंशतितमे कलियुगे प्रथमे पादे युधिष्ठिर-विक्रम-शालिवाहन-विजय-अभिनन्दन-नागार्जुन-कलिभूपाख्य
शकपुरुष मध्यपरिगणितेन शालिवाहनशके बौद्धावतारे ब्राह्म दैव पित्र्य प्राजापत्य बार्हस्पत्य सौर चान्द्र
सावन नक्षत्राख्य नवमान मध्य परिगणितेन सौर-चान्द्रमाण-द्वयेन प्रवर्तमाने प्रभवादीनां षष्ट्याः
संवत्सराणां मध्ये ...नाम संवत्सरे दक्षिणायने ग्रीष्मऋतौ सिंह मासे शुक्लपक्षे पौर्णमास्यां शुभतिथौ
...वासरयुक्तायां ...नक्षत्रयुक्तायां ...योग...करण युक्तायाम् एवंगुणविशेषणविशिष्टायाम् अस्यां
पौर्णमास्यां शुभतिथौ  अनादि-अविद्या-वासनया प्रवर्तमाने अस्मिन् महति संसारचक्रे विचित्राभिः
कर्मगतिभिः विचित्रासु योनिषु पुनः पुनः अनेकधा जनित्वा केनापि पुण्यकर्मविशेषेण
इदानीन्तन-मानुष-द्विजजन्मविशेषं प्राप्तवतः मम जन्माभ्यासात् जन्मप्रभृति एतत्क्षणपर्यन्तं बाल्ये वयसि
कौमारे यौवने वार्धके च जाग्रत्-स्वप्न-सुषुप्ति-अवस्थासु मनो-वाक्-कायैः
कर्मेन्द्रिय-ज्ञानेन्द्रिय-व्यापारैश्च सम्भावितानां रहस्यकृतानां प्रकाशकृतानां ब्रह्महनन सुरापान
स्वर्णस्तेय गुरुदारगमन तत्संसर्गाख्यानां महापातकानां, महापातक अनुमन्तृत्वादीनां अतिपातकानां,
सोमयागस्थ क्षत्रिय वैश्य वधादीनां समपातकानां, गोवधादीनां उपपातकानां मार्जारवधादीनां सङ्कलीकरणानां,
कृमिकीट वधादीनां मलिनीकरणानां, निन्दित धनादान उपजीवनादीनां अपात्रीकरणानां, मद्याघ्राणनादीनां
प्रकीर्णकानां, ज्ञानतः सकृत्कृतानां, अज्ञानतः असकृत्कृतानां, अत्यन्ताभ्यस्तानां निरन्तराभ्यस्तानां
चिरकालाभ्यस्तानां नवानां नवविधानां बहूनां बहुविधानां सर्वेषां पापानां सद्यः अपनोदनार्थं
भास्करक्षेत्रे  विनायकादिसमस्तहरिहरदेवतासन्निधौ ... पौर्णमास्याम् अध्यायोपक्रमकर्म करिष्ये ।  तदङ्गं
शरीरशुद्ध्यर्थं शुद्धोदकस्नानम् अहं करिष्ये ।
    
    \twolineshloka*
    {अतिक्रूर महाकाय कल्पान्त दहनोपम}
     {भैरवाय नमस्तुभ्यम् अनुज्ञां दातुम् अर्हसि}

    
\sect{यज्ञोपवीत-धारणम्}

\sect{काण्डऋषि तर्पणम्}

\sect{वेदव्यास (काण्डऋषि पूजा)}

\sect{होमम्}

\sect{वेदारम्भम्}