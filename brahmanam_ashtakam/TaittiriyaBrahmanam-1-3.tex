\sect{तृतीयः प्रश्नः}
\setcounter{anuvakam}{0}
\dnsub{तैत्तिरीयब्राह्मणे प्रथमाष्टके तृतीयः प्रपाठकः}

%1.3.1.1
दे॒वा॒सु॒राः संय॑त्ता आसन्। ते दे॒वा वि॑ज॒यमु॑प॒यन्त॑। अ॒ग्नीषोम॑योस्तेज॒स्विनीस्त॒नूः संन्य॑दधत। इ॒दमु॑ नो भविष्यति। यदि॑ नो जे॒ष्यन्तीति॑। तेना॒ग्नीषोमा॒वपाक्रामताम्। ते दे॒वा वि॒जित्य॑। अ॒ग्नीषोमा॒वन्वैच्छन्। तेऽग्निमन्व॑\-विन्दन्नृ॒तुषूत्स॑न्नम्। तस्य॒ विभ॑क्तीभिस्तेज॒स्विनीस्त॒नू\-रवा॑रुन्धत॥१॥

%1.3.1.2
ते सोम॒मन्व॑विन्दन्। तम॑घ्नन्। तस्य॑ यथाऽभि॒ज्ञायं॑ त॒नूर्व्य॑गृह्णत। ते ग्रहा॑ अभवन्। तद्ग्रहा॑णां ग्रह॒त्वम्। यस्यै॒वं वि॒दुषो॒ ग्रहा॑ गृ॒ह्यन्ते। तस्य॒ त्वे॑व गृ॑ही॒ताः। नानाऽऽग्नेयं पुनरा॒धेये॑ कुर्यात्। यदनाग्नेयं पुनरा॒धेये॑ कु॒र्यात्। व्यृ॑द्धमे॒व तत्॥२॥

%1.3.1.3
अनाग्नेयं॒ वा ए॒तत्क्रि॑यते। यत्स॒मिध॒स्तनू॒नपा॑तमि॒डो ब॒र्\mbox{}हिर्य॑जति। उ॒भावाग्ने॒यावाज्य॑भागौ स्याताम्। अनाज्यभागौ भवत॒ इत्या॑हुः। यदु॒भावाग्ने॒याव॒न्वञ्चा॒विति॑। अ॒ग्नये॒ पव॑माना॒योत्त॑रः स्यात्। यत्पव॑मानाय। तेनाज्य॑भागः। तेन॑ सौ॒म्यः। बुध॑न्वत्याग्ने॒यस्याज्य॑भागस्य पुरोऽनुवा॒क्या॑ भवति॥३॥

%1.3.1.4
यथा॑ सु॒प्तं बो॒धय॑ति। ता॒दृगे॒व तत्। अ॒ग्निन्य॑क्ताः पत्नीसंया॒जाना॒मृच॑ स्युः। तेनाग्ने॒य सर्वं॑ भवति। ए॒क॒धा॒ ते॑ज॒स्विनीं दे॒वता॒मुपै॒तीत्या॑हुः। सैन॑मीश्व॒रा प्र॒दह॒ इति॑। नेति॑ ब्रूयात्। प्र॒जन॑नं॒ वा अ॒ग्निः। प्र॒जन॑नमे॒वोपै॒तीति॑। कृ॒तय॑जु॒ सम्भृ॑तसम्भार॒ इत्या॑हुः॥४॥

%1.3.1.5
न स॒म्भृत्या सम्भा॒राः। न यजु॑ का॒र्य॑मिति॑। अथो॒ खलु॑। स॒म्भृत्या॑ ए॒व सं॑भा॒राः। का॒र्यं॑ यजु॑। पु॒न॒रा॒धेय॑स्य॒ समृ॑द्ध्यै। तेनो॑पा॒शु प्रच॑रति। एष्य॑ इव॒ वा ए॒षः। यत्पु॑नरा॒धेय॑। यथो॑पा॒शु न॒ष्टमि॒च्छति॑॥५॥

%1.3.1.6
ता॒दृगे॒व तत्। उ॒च्चैः स्वि॑ष्ट॒कृत॒मुत्सृ॑जति। यथा॑ न॒ष्टं वि॒त्त्वा प्राहा॒यमिति॑। ता॒दृगे॒व तत्। ए॒क॒धा ते॑ज॒स्विनीं दे॒वता॒मुपै॒तीत्या॑हुः। सैन॑मीश्व॒रा प्र॒दह॒ इति॑। तत्तथा॒ नोपै॑ति। प्र॒या॒जा॒नू॒या॒जेष्वे॒व विभ॑क्तीः कुर्यात्। य॒था॒पू॒र्वमाज्य॑भागौ॒ स्याताम्। ए॒वं प॑त्नीसंया॒जाः॥६॥

%1.3.1.7
तद्वैश्वान॒रव॑त्प्र॒जन॑नवत्तर॒मुपै॒तीति॑। तदा॑हुः। व्यृ॑द्धं॒ वा ए॒तत्। अनाग्नेयं॒ वा ए॒तत्क्रि॑यत॒ इति॑। नेति॑ ब्रूयात्। अ॒ग्निं प्र॑थ॒मं विभ॑क्तीनां यजति। अ॒ग्निमु॑त्त॒मं प॑त्नीसंया॒जानाम्। तेनाग्ने॒यम्। तेन॒ समृ॑द्धं क्रियत॒ इति॑॥७॥\anuvakamend[अ॒रु॒न्ध॒तै॒व तद्भ॑वति॒ सम्भृ॑तसम्भार॒ इत्या॑हुरि॒च्छति॑ पत्नीसंया॒जा नव॑ च]

%1.3.2.1
दे॒वा वै यथा॒दर्\mbox{}शं॑ य॒ज्ञानाह॑रन्त। योऽग्निष्टो॒मम्। य उ॒क्थ्यम्। यो॑ऽतिरा॒त्रम्। ते स॒हैव सर्वे॑ वाज॒पेय॑मपश्यन्। ते। अ॒न्योऽन्यस्मै॒ नाति॑ष्ठन्त। अ॒हम॒नेन॑ यजा॒ इति॑। तेऽब्रुवन्। आ॒जिम॒स्य धा॑वा॒मेति॑॥८॥

%1.3.2.2
तस्मि॑न्ना॒जिम॑धावन्। तं बृह॒स्पति॒रुद॑जयत्। तेना॑यजत। स स्वाराज्यमगच्छत्। तमिन्द्रोऽब्रवीत्। माम॒नेन॑ याज॒येति॑। तेनेन्द्र॑मयाजयत्। सोऽग्रं॑ दे॒वता॑नां॒ पर्यैत्। अग॑च्छ॒त्स्वाराज्यम्। अति॑ष्ठन्तास्मै॒ ज्यैष्ठ्या॑य॥९॥

%1.3.2.3
य ए॒वं वि॒द्वान् वा॑ज॒पेये॑न॒ यज॑ते। गच्छ॑ति॒ स्वाराज्यम्। अग्र समा॒नानां॒ पर्ये॑ति। तिष्ठ॑न्तेऽस्मै॒ ज्यैष्ठ्या॑य। स वा ए॒ष ब्राह्म॒णस्य॑ चै॒व रा॑ज॒न्य॑स्य च य॒ज्ञः। तं वा ए॒तं वा॑ज॒पेय॒ इत्या॑हुः। वा॒जाप्यो॒ वा ए॒षः। वाज॒ ह्ये॑तेन॑ दे॒वा ऐप्स\sn{}। सोमो॒ वै वा॑ज॒पेय॑। यो वै सोमं॑ वाज॒पेयं॒ वेद॑॥१०॥

%1.3.2.4
वा॒ज्ये॑वैनं॑ पी॒त्वा भ॑वति। आऽस्य॑ वा॒जी जा॑यते। अन्नं॒ वै वा॑ज॒पेय॑। य ए॒वं वेद॑। अत्यन्नम्। आऽस्यान्ना॒दो जा॑यते। ब्रह्म॒ वै वा॑ज॒पेय॑। य ए॒वं वेद॑। अत्ति॒ ब्रह्म॒णाऽन्नम्। आऽस्य॑ ब्र॒ह्मा जा॑यते ॥११॥

%1.3.2.5
वाग्वै वाज॑स्य प्रस॒वः। य ए॒वं वेद॑। क॒रोति॑ वा॒चा वी॒र्यम्। ऐनं॑ वा॒चा ग॑च्छति। अपि॑वतीं॒ वाचं॑ वदति। प्र॒जाप॑तिर्दे॒वेभ्यो॑ य॒ज्ञान्व्यादि॑शत्। स आ॒त्मन्वा॑ज॒पेय॑मधत्त। तं दे॒वा अ॑ब्रुवन्। ए॒ष वाव य॒ज्ञः। यद्वा॑ज॒पेय॑॥१२॥

%1.3.2.6
अप्ये॒व नोऽत्रा॒स्त्विति॑। तेभ्य॑ ए॒ता उज्जि॑ती॒ प्राय॑च्छत्। ता वा ए॒ता उज्जि॑तयो॒ व्याख्या॑यन्ते। य॒ज्ञस्य॑ सर्व॒त्वाय॑। दे॒वता॑ना॒मनि॑र्भागाय। दे॒वा वै ब्रह्म॑ण॒श्चान्न॑स्य च॒ शम॑ल॒मपाघ्नन्। यद्ब्रह्म॑ण॒ शम॑ल॒मासीत्। सा गाथा॑ नाराश॒स्य॑भवत्। यदन्न॑स्य। सा सुरा॥१३॥

%1.3.2.7
तस्मा॒द्गाय॑तश्च म॒त्तस्य॑ च॒ न प्र॑ति॒गृह्यम्। यत्प्र॑तिगृह्णी॒यात्। शम॑लं॒ प्रति॑गृह्णीयात्। सर्वा॒ वा ए॒तस्य॒ वाचोऽव॑रुद्धाः। यो वा॑जपेयया॒जी। या पृ॑थि॒व्यां याऽग्नौ या र॑थन्त॒रे। याऽन्तरि॑क्षे॒ या वा॒यौ या वा॑मदे॒व्ये। या दि॒वि याऽऽदि॒त्ये या बृ॑ह॒ति। याऽप्सु यौष॑धीषु॒ या वन॒स्पति॑षु। तस्माद्वाजपेयया॒ज्यार्त्वि॑जीनः। सर्वा॒ ह्य॑स्य॒ वाचोऽव॑रुद्धाः॥१४॥\anuvakamend[धा॒वा॒मेति॒ ज्यैष्ठ्या॑य॒ वेद॑ ब्र॒ह्मा जा॑यते वाज॒पेय॒ सुराऽऽर्त्वि॑जीन॒ एकं॑ च]

%1.3.3.1
दे॒वा वै यद॒न्यैर्ग्रहैर्य॒ज्ञस्य॒ नावारु॑न्धत। तद॑तिग्रा॒ह्यै॑रति॒\-गृह्या\-वा॑रुन्धत। तद॑तिग्र॒ह्या॑णामतिग्राह्य॒त्वम्। यद॑तिग्रा॒ह्या॑ गृ॒ह्यन्ते। यदे॒वान्यैर्ग्रहैर्य॒ज्ञस्य॒ नाव॑रु॒न्धे। तदे॒व तैर॑ति॒गृह्या\-व॑रुन्धे। पञ्च॑ गृह्यन्ते। पाङ्क्तो॑ य॒ज्ञः। यावा॑ने॒व य॒ज्ञः। तमा॒प्त्वाऽव॑रुन्धे॥१५॥

%1.3.3.2
सर्व॑ ऐ॒न्द्रा भ॑वन्ति। ए॒क॒धैव यज॑मान इन्द्रि॒यं द॑धति। स॒प्तद॑श प्राजाप॒त्या ग्रहा॑ गृह्यन्ते। स॒प्त॒द॒शः प्र॒जाप॑तिः। प्र॒जाप॑ते॒राप्त्यै। एक॑य॒र्चा गृ॑ह्णाति। ए॒क॒धैव यज॑माने वी॒र्यं॑ दधाति। सो॒म॒ग्र॒हाश्च॑ सुराग्र॒हाश्च॑ गृह्णाति। ए॒तद्वै दे॒वानां पर॒ममन्नम्। यत्सोम॑॥१६॥

%1.3.3.3
ए॒तन्म॑नु॒ष्या॑णाम्। यत्सुरा। प॒र॒मेणै॒वास्मा॑ अ॒न्नाद्ये॒नाव॑र\-म॒न्नाद्य॒मव॑रुन्धे। सो॒म॒ग्र॒हान्गृ॑ह्णाति। ब्रह्म॑णो॒ वा ए॒तत्तेज॑। यत्सोम॑। ब्रह्म॑ण ए॒व तेज॑सा॒ तेजो॒ यज॑माने दधाति। सु॒रा॒ग्र॒हान्गृ॑ह्णाति। अन्न॑स्य॒ वा ए॒तच्छम॑लम्। यत्सुरा॥१७॥

%1.3.3.4
अन्न॑स्यै॒व शम॑लेन॒ शम॑लं॒ यज॑माना॒दप॑हन्ति। सो॒म॒ग्र॒हाश्च॑ सुराग्र॒हाश्च॑ गृह्णाति। पुमा॒न्॒ वै सोम॑। स्त्री सुरा। तन्मि॑थु॒नम्। मि॒थु॒नमे॒वास्य॒ तद्य॒ज्ञे क॑रोति प्र॒जन॑नाय। आ॒त्मान॑मे॒व सो॑मग्र॒हैः स्पृ॑णोति। जा॒या सु॑राग्र॒हैः। तस्माद्वाजपेयया॒ज्य॑मुष्मि॑ल्लोँ॒के स्त्रिय॒ सम्भ॑वति। वा॒ज॒पेया॑भिजित॒ ह्य॑स्य॥१८॥

%1.3.3.5
पूर्वे॑ सोमग्र॒हा गृ॑ह्यन्ते। अप॑रे सुराग्र॒हाः। पु॒रो॒ऽक्ष सो॑मग्र॒हान्त्सा॑दयति। प॒श्चा॒द॒क्ष सु॑राग्र॒हान्। पा॒प॒व॒स्य॒सस्य॒ विधृ॑त्यै। ए॒ष वै यज॑मानः। यत्सोम॑। अन्न॒ सुरा। सो॒म॒ग्र॒हाश्च॑ सुराग्र॒हाश्च॒ व्यति॑षजति। अ॒न्नाद्ये॑नै॒वैनं॒ व्यति॑षजति॥१९॥

%1.3.3.6
सं॒पृच॑ स्थ॒ सं मा॑ भ॒द्रेण॑ पृ॒ङ्क्तेत्या॑ह। अन्नं॒ वै भ॒द्रम्। अ॒न्नाद्ये॑नै॒वैन॒ ससृ॑जति। अन्न॑स्य॒ वा ए॒तच्छम॑लम्। यत्सुरा। पा॒प्मेव॒ खलु॒ वै शम॑लम्। पा॒प्मना॒ वा ए॑नमे॒तच्छम॑लेन॒ व्यति॑षजति। यत्सो॑मग्र॒हाश्च॑ सुराग्र॒हाश्च॑ व्यति॒षज॑ति। वि॒पृच॑ स्थ॒ वि मा॑ पा॒प्मना॑ पृ॒ङ्क्तेत्या॑ह। पा॒प्मनै॒वैन॒ शम॑लेन॒ व्याव॑र्तयति॥२०॥

%1.3.3.7
तस्माद्वाजपेयया॒जी पू॒तो मेध्यो॑ दक्षि॒ण्य॑। प्राङुद्द्र॑वति सोमग्र॒हैः। अ॒मुमे॒व तैर्लो॒कम॒भिज॑यति। प्र॒त्यङ्ख्सु॑राग्र॒हैः। इ॒ममे॒व तैर्लो॒कम॒भिज॑यति। प्रति॑ष्ठन्ति सोमग्र॒हैः। याव॑दे॒व स॒त्यम्। तेन॑ सूयते। वा॒ज॒सृद्भ्य॑ सुराग्र॒हान् ह॑रन्ति। अनृ॑तेनै॒व विश॒ ससृ॑जति। हि॒र॒ण्य॒पा॒त्रं मधो पू॒र्णं द॑दाति। म॒ध॒व्यो॑ऽसा॒नीति॑। ए॒क॒धा ब्र॒ह्मण॒ उप॑ हरति। ए॒क॒धैव यज॑मान॒ आयु॒स्तेजो॑ दधाति॥२१॥\anuvakamend[आ॒प्त्वाऽव॑रुन्धे॒ सोम॒ शम॑लं॒ यत्सुरा॒ ह्य॑स्यैनं॒ व्यति॑षजति॒ व्याव॑र्तयति सृजति च॒त्वारि॑ च]

%1.3.4.1
ब्र॒ह्म॒वा॒दिनो॑ वदन्ति। नाग्नि॑ष्टो॒मो नोक्थ्य॑। न षो॑ड॒शी नाति॑रा॒त्रः। अथ॒ कस्माद्वाज॒पेये॒ सर्वे॑ यज्ञक्र॒तवोऽव॑रुध्यन्त॒ इति॑। प॒शुभि॒रिति॑ ब्रूयात्। आ॒ग्ने॒यं प॒शुमाल॑भते। अ॒ग्नि॒ष्टो॒ममे॒व तेनाव॑रुन्धे। ऐ॒न्द्रा॒ग्नेनो॒क्थ्यम्। ऐ॒न्द्रेण॑ षोड॒शिन॑ स्तो॒त्रम्। सा॒र॒स्व॒त्याऽति॑रा॒त्रम्॥२२॥

%1.3.4.2
मा॒रु॒त्या बृ॑ह॒तः स्तो॒त्रम्। ए॒ताव॑न्तो॒ वै य॑ज्ञक्र॒तव॑। तान्प॒शुभि॑रे॒वाव॑रुन्धे। आ॒त्मान॑मे॒व स्पृ॑णोत्यग्निष्टो॒मेन॑। प्रा॒णा॒पा॒नावु॒क्थ्ये॑न। वी॒र्य षोड॒शिन॑ स्तो॒त्रेण॑। वाच॑मतिरा॒त्रेण॑। प्र॒जां बृ॑ह॒तः स्तो॒त्रेण॑। इ॒ममे॒व लो॒कम॒भिज॑यत्यग्निष्टो॒मेन॑। अ॒न्तरि॑क्षमु॒क्थ्ये॑न॥२३॥

%1.3.4.3
सु॒व॒र्गं लो॒क षो॑ड॒शिन॑ स्तो॒त्रेण॑। दे॒व॒याना॑ने॒व प॒थ आरो॑हत्यतिरा॒त्रेण॑। नाक रोहति बृह॒तः स्तो॒त्रेण॑। तेज॑ ए॒वात्मन्ध॑त्त आग्ने॒येन॑ प॒शुना। ओजो॒ बल॑मैन्द्रा॒ग्नेन॑। इ॒न्द्रि॒यमै॒न्द्रेण॑। वाच सारस्व॒त्या। उ॒भावे॒व दे॑वलो॒कं च॑ मनुष्यलो॒कं चा॒भिज॑यति मारु॒त्या व॒शया। स॒प्तद॑श प्राजाप॒त्यान्प॒शूनाल॑भते। स॒प्त॒द॒शः प्र॒जाप॑तिः॥२४॥

%1.3.4.4
प्र॒जाप॑ते॒राप्त्यै। श्या॒मा एक॑रूपा भवन्ति। ए॒वमि॑व॒ हि प्र॒जाप॑ति॒ समृ॑द्ध्यै। तान्पर्य॑ग्निकृता॒नुत्सृ॑जति। म॒रुतो॑ य॒ज्ञम॑जिघासन्प्र॒जाप॑तेः। तेभ्य॑ ए॒तां मा॑रु॒तीं व॒शामाल॑भत। तयै॒वैना॑नशमयत्। मा॒रु॒त्या प्र॒चर्य॑। ए॒तान्त्संज्ञ॑पयेत्। म॒रुत॑ ए॒व श॑मयि॒त्वा॥२५॥

%1.3.4.5
ए॒तैः प्रच॑रति। य॒ज्ञस्याघा॑ताय। ए॒क॒धा व॒पा जु॑होति। ए॒क॒दे॒व॒त्या॑ हि। ए॒ते। अथो॑ एक॒धैव यज॑माने वी॒र्यं॑ दधाति। नै॒वा॒रेण॑ स॒प्तद॑शशरावेणै॒तर्\mbox{}हि॒ प्रच॑रति। ए॒तत्पु॑रोडाशा॒ ह्ये॑ते। अथो॑ पशू॒नामे॒व छि॒द्रमपि॑दधाति। सा॒र॒स्व॒त्योत्त॒मया॒ प्रच॑रति। वाग्वै सर॑स्वती। तस्मात्प्रा॒णानां॒ वागु॑त्त॒मा। अथो प्र॒जाप॑तावे॒व य॒ज्ञं प्रति॑ष्ठापयति। प्र॒जाप॑ति॒र्‌हि वाक्। अप॑न्नदती भवति। तस्मान्मनु॒ष्या सर्वां॒ वाचं॑ वदन्ति॥२६॥\anuvakamend[अ॒ति॒रा॒त्रम॒न्तरि॑क्षमु॒क्थ्ये॑न प्र॒जाप॑तिः शमयि॒त्वोत्त॒मया॒ प्रच॑रति॒ षट् च॑]

%1.3.5.1
सा॒वि॒त्रं जु॑होति॒ कर्म॑णः कर्मणः पु॒रस्तात्। कस्तद्वे॒देत्या॑हुः। यद्वा॑ज॒पेय॑स्य॒ पूर्वं॒ यदप॑र॒मिति॑। स॒वि॒तृप्र॑सूत ए॒व य॑थापू॒र्वं कर्मा॑णि करोति। सव॑नेसवने जुहोति। आ॒क्रम॑णमे॒व तत्सेतुं॒ यज॑मानः कुरुते। सु॒व॒र्गस्य॑ लो॒कस्य॒ सम॑ष्ट्यै। वा॒चस्पति॒र्वाच॑म॒द्य स्व॑दाति न॒ इत्या॑ह। वाग्वै दे॒वानां पु॒राऽन्न॑मासीत्। वाच॑मे॒वास्मा॒ अन्न स्वदयति॥२७॥

%1.3.5.2
इन्द्र॑स्य॒ वज्रो॑ऽसि॒ वार्त्र॑घ्न॒ इति॒ रथ॑मु॒पाव॑हरति॒ विजि॑त्यै। वाज॑स्य॒ नु प्र॑स॒वे मा॒तरं॑ म॒हीमित्या॑ह। यच्चै॒वेयम्। यच्चा॒स्यामधि॑। तदे॒वाव॑रुन्धे। अथो॒ तस्मि॑न्ने॒वोभये॒ऽभिषि॑च्यते। अ॒प्स्व॑न्तर॒मृत॑म॒प्सु भे॑ष॒जमित्यश्वान्पल्पूलयति। अ॒प्सु वा अश्व॑स्य॒ तृती॑यं॒ प्रवि॑ष्टम्। तद॑नु॒वेन॒न्वव॑प्लवते। यद॒प्सु प॑ल्पू॒लय॑ति॥२८॥

%1.3.5.3
यदे॒वास्या॒प्सु प्रवि॑ष्टम्। तदे॒वाव॑रुन्धे। ब॒हु वा अश्वो॑ऽमे॒ध्यमुप॑गच्छति। यद॒प्सु प॑ल्पू॒लय॑ति। मेध्या॑ने॒वै\-नान्करोति। वा॒युर्वा त्वा॒ मनु॑र्वा॒ त्वेत्या॑ह। ए॒ता वा ए॒तं दे॒वता॒ अग्रे॒ अश्व॑मयुञ्जन्। ताभि॑रे॒वैनान्॑ युनक्ति। स॒वस्योज्जि॑त्यै। यजु॑षा युनक्ति॒ व्यावृ॑त्त्यै॥२९॥

%1.3.5.4
अपान्नपादाशुहेम॒न्निति॒ सम्मार्ष्टि। मेध्या॑ने॒वैनान्करोति। अथो॒ स्तौत्ये॒वैना॑ना॒जि स॑रिष्य॒तः। वि॒ष्णु॒क्र॒मान्क्र॑मते। विष्णु॑रे॒व भू॒त्वेमाल्लोँ॒कान॒भिज॑यति। वै॒श्व॒दे॒वो वै रथ॑। अ॒ङ्कौ न्य॒ङ्काव॒भितो॒ रथं॒ यावित्या॑ह। या ए॒व दे॒वता॒ रथे॒ प्रवि॑ष्टाः। ताभ्य॑ ए॒व नम॑स्करोति। आ॒त्मनोऽनार्त्यै। अश॑मरथम्भावुकोऽस्य॒ रथो॑ भवति। य ए॒वं वेद॑॥३०॥\anuvakamend[स्व॒द॒य॒ति॒ प॒ल्पू॒लय॑ति॒ व्यावृ॑त्त्या॒ अनार्त्यै॒ द्वे च॑]

%1.3.6.1
दे॒वस्या॒ह स॑वि॒तुः प्र॑स॒वे बृह॒स्पति॑ना वाज॒जिता॒ वाजं॑ जेष॒मित्या॑ह। स॒वि॒तृप्र॑सूत ए॒व ब्रह्म॑णा॒ वाज॒मुज्ज॑यति। दे॒वस्या॒ह स॑वि॒तुः प्र॑स॒वे बृह॒स्पति॑ना वाज॒जिता॒ वर्\mbox{}षि॑ष्ठं॒ नाक रुहेय॒मित्या॑ह। स॒वि॒तृप्र॑सूत ए॒व ब्रह्म॑णा॒ वर्\mbox{}षि॑ष्ठं॒ नाक रोहति। चात्वा॑ले रथच॒क्रं निमि॑त रोहति। अतो॒ वा अङ्गि॑रस उत्त॒माः सु॑व॒र्गं लो॒कमा॑यन्। सा॒क्षादे॒व यज॑मानः सुव॒र्गं लो॒कमे॑ति। आवेष्टयति। वज्रो॒ वै रथ॑। वज्रे॑णै॒व दिशो॒ऽभिज॑यति॥३१॥

%1.3.6.2
वा॒जिना॒ साम॑ गायते। अन्नं॒ वै वाज॑। अन्न॑मे॒वाव॑रुन्धे। वा॒चो वर्ष्म॑ दे॒वेभ्योऽपाक्रामत्। तद्वन॒स्पती॒न्प्रावि॑शत्। सैषा वाग्वन॒स्पति॑षु वदति। या दु॑न्दु॒भौ। तस्माद्दुन्दु॒भिः सर्वा॒ वाचोऽति॑वदति। दु॒न्दु॒भीन्त्स॒माघ्न॑न्ति। प॒र॒मा वा ए॒षा वाक्॥३२॥

%1.3.6.3
या दु॑न्दु॒भौ। प॒र॒मयै॒व वा॒चाऽव॑रां॒ वाच॒म॑वरुन्धे। अथो॑ वा॒च ए॒व वर्ष्म॒ यज॑मा॒नोऽव॑रुन्धे। इन्द्रा॑य॒ वाचं॑ वद॒तेन्द्रं॒ वाजं॑ जापय॒तेन्द्रो॒ वाज॑मजयि॒दित्या॑ह। ए॒ष वा ए॒तर्\mbox{}हीन्द्र॑। यो यज॑ते। यज॑मान ए॒व वाज॒मुज्ज॑यति। स॒प्तद॑श प्रव्या॒धाना॒जिं धा॑वन्ति। स॒प्त॒द॒श स्तो॒त्रं भ॑वति। स॒प्तद॑शसप्तदश दीयन्ते॥३३॥

%1.3.6.4
स॒प्त॒द॒शः प्र॒जाप॑तिः। प्र॒जा॑पते॒राप्त्यै। अर्वा॑ऽसि॒ सप्ति॑रसि वा॒ज्य॑सीत्या॑ह। अ॒ग्निर्वा अर्वा। वा॒युः सप्ति॑। आ॒दि॒त्यो वा॒जी। ए॒ताभि॑रे॒वास्मै॑ दे॒वता॑भिर्देवर॒थं यु॑नक्ति। प्र॒ष्टि॒वा॒हिनं॑ युनक्ति। प्र॒ष्टि॒वा॒ही वै दे॑वर॒थः। दे॒व॒र॒थमे॒वास्मै॑ युनक्ति॥३४॥

%1.3.6.5
वाजि॑नो॒ वाजं॑ धावत॒ काष्ठां गच्छ॒तेत्या॑ह। सु॒व॒र्गो वै लो॒कः काष्ठा। सु॒व॒र्गमे॒व लो॒कं य॑न्ति। सु॒व॒र्गं वा ए॒ते लो॒कं य॑न्ति। य आ॒जिं धाव॑न्ति। प्राञ्चो॑ धावन्ति। प्राङि॑व॒ हि सु॑व॒र्गो लो॒कः। च॒त॒सृभि॒रनु॑ मन्त्रयते। च॒त्वारि॒ छन्दासि। छन्दो॑भिरे॒वैनान्त्सुव॒र्गं लो॒कं ग॑मयति॥३५॥

%1.3.6.6
प्र वा ए॒तेऽस्माल्लो॒काच्च्य॑वन्ते। य आ॒जिं धाव॑न्ति। उद॑ञ्च॒ आव॑र्तन्ते। अ॒स्मादे॒व तेन॑ लो॒कान्नय॑न्ति। र॒थ॒वि॒मो॒च॒नीयं॑ जुहोति॒ प्रति॑ष्ठित्त्यै। आ मा॒ वाज॑स्य प्रस॒वो ज॑गम्या॒दित्या॑ह। अन्नं॒ वै वाज॑। अन्न॑मे॒वाव॑रुन्धे। य॒था॒लो॒कं वा ए॒त उज्ज॑यन्ति। य आ॒जिं धाव॑न्ति॥३६॥

%1.3.6.7
कृ॒ष्णलं॑कृष्णलं वाज॒सृद्भ्य॒ प्रय॑च्छति। यमे॒व ते वाजं॑ लो॒कमु॒ज्जय॑न्ति। तं प॑रि॒क्रीयाव॑रुन्धे। ए॒क॒धा ब्र॒ह्मण॒ उप॑हरति। ए॒क॒धैव यज॑माने वी॒र्यं॑ दधाति। दे॒वा वा ओष॑धीष्वा॒जिम॑युः। ता बृह॒स्पति॒रुद॑जयत्। स नी॒वारा॒न्निर॑वृणीत। तन्नी॒वारा॑णां नीवार॒त्वम्। नै॒वा॒रश्च॒रुर्भ॑वति॥३७॥

%1.3.6.8
ए॒तद्वै दे॒वानां पर॒ममन्नम्। यन्नी॒वारा। प॒र॒मेणै॒वास्मा॑ अ॒न्नाद्ये॒नाव॑रम॒न्नाद्य॒मव॑रुन्धे। स॒प्तद॑शशरावो भवति। स॒प्त॒द॒शः प्र॒जाप॑तिः। प्र॒जाप॑ते॒राप्त्ये। क्षी॒रे भ॑वति। रुच॑मे॒वास्मि॑न्दधाति। स॒र्पिष्वान्भवति मेध्य॒त्वाय॑। बा॒र्॒ह॒स्प॒त्यो वा ए॒ष दे॒वत॑या॥३८॥

%1.3.6.9
यो वा॑ज॒पेये॑न॒ यज॑ते। बा॒र्॒ह॒स्प॒त्य ए॒ष च॒रुः। अश्वान्त्सरिष्य॒तः स॒स्रुष॒श्चाव॑ घ्रापयति। यमे॒व ते वाजं॑ लो॒कमु॒ज्जय॑न्ति। तमे॒वाव॑रुन्धे। अजी॑जिपत वनस्पतय॒ इन्द्रं॒ वाजं॒ विमु॑च्यध्व॒मिति॑ दुन्दु॒भीन् विमु॑ञ्चति। यमे॒व ते वाजं॑ लो॒कमि॑न्द्रि॒यं दु॑न्दु॒भय॑ उ॒ज्जय॑न्ति। तमे॒वाव॑रुन्धे॥३९॥\anuvakamend[अ॒भिज॑यति॒ वा ए॒षा वाग्दी॑यन्तेऽस्मै युनक्ति गमयति॒ य आ॒जिं धाव॑न्ति भवति दे॒वत॑या॒ऽष्टौ च॑]

%1.3.7.1
ता॒र्प्यं यज॑मानं॒ परि॑धापयति। य॒ज्ञो वै ता॒र्प्यम्। य॒ज्ञेनै॒वैन॒ सम॑र्धयति। द॒र्भ॒मयं॒ परि॑धापयति। प॒वित्रं॒ वै द॒र्भाः। पु॒नात्ये॒वैनम्। वाजं॒ वा ए॒षोऽव॑रुरुत्सते। यो वा॑ज॒पये॑न॒ यज॑ते। ओष॑धय॒ खलु॒ वै वाज॑। यद्द॑र्भ॒मयं॑ परिधा॒पय॑ति॥४०॥

%1.3.7.2
वाज॒स्याव॑रुद्ध्यै। जाय॒ एहि॒ सुवो॒ रोहा॒वेत्या॑ह। पत्नि॑या ए॒वैष य॒ज्ञस्यान्वार॒म्भोऽन॑वच्छित्त्यै। स॒प्तद॑शारत्नि॒र्यूपो॑ भवति। स॒प्त॒द॒शः प्र॒जाप॑तिः। प्र॒जाप॑ते॒राप्त्यै। तू॒प॒रश्चतु॑रश्रिर्भवति। गौ॒धू॒मं च॒षालम्। न वा ए॒ते व्री॒हयो॒ न यवा। यद्गो॒धूमा॥४१॥

%1.3.7.3
ए॒वमि॑व॒ हि प्र॒जाप॑ति॒ समृ॑द्ध्यै। अथो॑ अ॒मुमे॒वास्मै॑ लो॒कमन्न॑वन्तं करोति। वासो॑भिर्वेष्टयति। ए॒ष वै यज॑मानः। यद्यूप॑। स॒र्व॒दे॒व॒त्यं॑ वास॑। सर्वा॑भिरे॒वैनं॑ दे॒वता॑भि॒ सम॑र्धयति। अथो॑ आ॒क्रम॑णमे॒व तत्सेतुं॒ यज॑मानः कुरुते। सु॒व॒र्गस्य॑ लो॒कस्य॒ सम॑ष्ट्यै। द्वाद॑श वाजप्रस॒वीया॑नि जुहोति॥४२॥

%1.3.7.4
द्वाद॑श॒ मासा संवत्स॒रः। सं॒व॒त्स॒रमे॒व प्री॑णाति। अथो॑ संवत्स॒रमे॒वास्मा॒ उप॑दधाति। सु॒व॒र्गस्य॑ लो॒कस्य॒ सम॑ष्ट्यै। द॒शभि॒ कल्पै॑ रोहति। नव॒ वै पुरु॑षे प्रा॒णाः। नाभि॑र्दश॒मी। प्रा॒णाने॒व य॑थास्था॒नं क॑ल्पयि॒त्वा। सु॒व॒र्गं लो॒कमे॑ति। ए॒ताव॒द्वै पुरु॑षस्य॒ स्वम्॥४३॥

%1.3.7.5
याव॑त्प्रा॒णाः। याव॑दे॒वास्यास्ति॑। तेन॑ स॒ह सु॑व॒र्गं लो॒कमे॑ति। सुव॑र्दे॒वा अ॑ग॒न्मेत्या॑ह। सु॒व॒र्गमे॒व लो॒कमे॑ति। अ॒मृता॑ अभू॒मेत्या॑ह। अ॒मृत॑मिव॒ हि सु॑व॒र्गो लो॒कः। प्र॒जाप॑तेः प्र॒जा अ॑भू॒मेत्या॑ह। प्रा॒जा॒प॒त्यो वा अ॒यं लो॒कः। अ॒स्मादे॒व तेन॑ लो॒कान्नैति॑॥४४॥

%1.3.7.6
सम॒हं प्र॒जया॒ सं मया प्र॒जेत्या॑ह। आ॒शिष॑मे॒वैतामा शास्ते। आ॒स॒पु॒टैर्घ्न॑न्ति। अन्नं॒ वा इ॒यम्। अ॒न्नाद्ये॑नै॒वैन॒ सम॑र्धयन्ति। ऊषैर्घ्नन्ति। ए॒ते हि सा॒क्षादन्नम्। यदूषा। सा॒क्षादे॒वैन॑म॒न्नाद्ये॑न॒ सम॑र्धयन्ति। पु॒रस्तात्प्र॒त्यञ्चं घ्नन्ति॥४५॥

%1.3.7.7
पु॒रस्ता॒द्धि प्र॑ती॒चीन॒मन्न॑म॒द्यते। शी॒र्॒ष॒तो घ्न॑न्ति। शी॒र्॒ष॒तो ह्यन्न॑म॒द्यते। दि॒ग्भ्यो घ्न॑न्ति। दि॒ग्भ्य ए॒वास्मा॑ अ॒न्नाद्य॒मव॑रुन्धते। ई॒श्व॒रो वा ए॒ष पराङ्प्र॒दघ॑। यो यूप॒ रोह॑ति। हिर॑ण्यम॒ध्यव॑रोहति। अ॒मृतं॒ वै हिर॑ण्यम्। अ॒मृत सुव॒र्गो लो॒कः॥४६॥

%1.3.7.8
अ॒मृत॑ ए॒व सु॑व॒र्गे लो॒के प्रति॑तिष्ठति। श॒तमा॑नं भवति। श॒तायु॒ पुरु॑षः श॒तेन्द्रि॑यः। आयु॑ष्ये॒वेन्द्रि॒ये प्रति॑ तिष्ठति। पुष्ट्यै॒ वा ए॒तद्रू॒पम्। यद॒जा। त्रिः सं॑वत्स॒रस्या॒न्यान्प॒शून्परि॒ प्रजा॑यते। ब॒स्ता॒जि॒नम॒ध्यव॑ रोहति। पुष्ट्या॑मे॒व प्र॒जन॑ने॒ प्रति॑तिष्ठति॥४७॥\anuvakamend[प॒रि॒धा॒पय॑ति गो॒धूमा॑ जुहोति॒ स्वं नैति॑ प्र॒त्यञ्चं घ्नन्ति लो॒को नव॑ च]

%1.3.8.1
स॒प्तान्न॑हो॒माञ्जु॑होति। स॒प्त वा अन्ना॑नि। याव॑न्त्ये॒वान्ना॑नि। तान्ये॒वाव॑रुन्धे। स॒प्त ग्रा॒म्या ओष॑धयः। स॒प्तार॒ण्याः। उ॒भयी॑षा॒मव॑रुद्ध्यै। अन्न॑स्यान्नस्य जुहोति। अन्न॑स्यान्न॒स्या\-व॑रुद्ध्यै। यद्वा॑जपेयया॒ज्यन॑वरुद्धस्याश्नी॒यात्॥४८॥

%1.3.8.2
अव॑रुद्धेन॒ व्यृ॑द्ध्येत। सर्व॑स्य समव॒दाय॑ जुहोति। अन॑वरुद्ध॒स्याव॑रुद्ध्यै। औदु॑म्बरेण स्रु॒वेण॑ जुहोति। ऊर्ग्वा अन्न॑मुदु॒म्बर॑। ऊ॒र्ज ए॒वान्नाद्य॒स्याव॑रुद्ध्यै। दे॒वस्य॑ त्वा सवि॒तुः प्र॑स॒व इत्या॑ह। स॒वि॒तृप्र॑सूत ए॒वैनं॒ ब्रह्म॑णा दे॒वता॑भिर॒भिषि॑ञ्चति। अन्न॑स्यान्नस्या॒भिषि॑ञ्चति। अन्न॑स्यान्न॒स्याव॑रुद्ध्यै॥४९॥

%1.3.8.3
पु॒रस्तात्प्र॒त्यञ्च॑म॒भिषि॑ञ्चति। पु॒रस्ता॒द्धि प्र॑ती॒चीन॒मन्न॑म॒द्यते। शी॒र्॒ष॒तो॑ऽभिषि॑ञ्चति। शी॒र्॒ष॒तो ह्यन्न॑म॒द्यते। आ मुखा॑द॒न्वव॑स्रावयति। मु॒ख॒त ए॒वास्मा॑ अ॒न्नाद्यं॑ दधाति। अ॒ग्नेस्त्वा॒ साम्राज्येना॒भिषि॑ञ्चा॒मीत्या॑ह। ए॒ष वा अ॒ग्नेः स॒वः। तेनै॒वैन॑म॒भिषि॑ञ्चति। इन्द्र॑स्य त्वा॒ साम्राज्येना॒भिषि॑ञ्चा॒मीत्या॑ह॥५०॥

%1.3.8.4
इ॒न्द्रि॒यमे॒वास्मि॑न्ने॒तेन॑ दधाति। बृह॒स्पतेस्त्वा॒ साम्राज्येना॒भि\-षि॑ञ्चा॒मीत्या॑ह। ब्रह्म॒ वै दे॒वानां॒ बृह॒स्पति॑। ब्रह्म॑णै॒वैन॑म॒भि\-षि॑ञ्चति। सो॒म॒ग्र॒हाश्चा॑वदानी॒यानि॑ च॒र्त्विग्भ्य॒ उप॑हरन्ति। अ॒मुमे॒व तैर्लो॒कमन्न॑वन्तं करोति। सु॒रा॒ग्र॒हाश्चा॑नवदानी॒यानि॑ च वाज॒सृद्भ्य॑। इ॒ममे॒व तैर्लो॒कमन्न॑वन्तं करोति। अथो॑ उ॒भयीष्वे॒वाभिषि॑च्यते। वि॒मा॒थं कु॑र्वते वाज॒सृत॑॥५१॥

%1.3.8.5
इ॒न्द्रि॒यस्याव॑रुद्ध्यै। अनि॑रुक्ताभिः प्रातः सव॒ने स्तु॑वते। अनि॑रुक्तः प्र॒जाप॑तिः। प्र॒जाप॑ते॒राप्त्यै। वाज॑वतीभि॒र्माध्य॑न्दिने। अन्नं॒ वै वाज॑। अन्न॑मे॒वाव॑रुन्धे। शि॒पि॒वि॒ष्टव॑तीभिस्तृतीय\-सव॒ने। य॒ज्ञो वै विष्णु॑। प॒शव॒ शिपि॑। य॒ज्ञ ए॒व प॒शुषु॒ प्रति॑ तिष्ठति। बृ॒हदन्त्यं॑ भवति। अन्त॑मे॒वैन श्रि॒यै ग॑मयति॥५२॥\anuvakamend[अ॒श्नी॒यादन्न॑स्यान्न॒स्याव॑रुद्ध्या॒ इन्द्र॑स्य त्वा॒ साम्राज्येना॒भिषि॑ञ्चा॒मीत्या॑ह वाज॒सृत॒ शिपि॒स्त्रीणि॑ च]

%1.3.9.1
नृ॒षदं॒ त्वेत्या॑ह। प्र॒जा वै नॄन्। प्र॒जाना॑मे॒वैतेन॑ सूयते। द्रु॒षद॒मित्या॑ह। वन॒स्पत॑यो॒ वै द्रु। वन॒स्पती॑नामे॒वैतेन॑ सूयते। भु॒व॒न॒सद॒मित्या॑ह। य॒दा वै वसी॑या॒न्भव॑ति। भुव॑नमग॒न्निति॒ वै तमा॑हुः। भुव॑नमे॒वैतेन॑ गच्छति॥५३॥

%1.3.9.2
अ॒प्सु॒षदं॑ त्वा घृत॒सद॒मित्या॑ह। अ॒पामे॒वैतेन॑ घृ॒तस्य॑ सूयते। व्यो॒म॒सद॒मित्या॑ह। य॒दा वै वसी॑या॒न्भव॑ति। व्यो॑माग॒न्निति॒ वै तमा॑हुः। व्यो॑मै॒वैते॑न गच्छति। पृ॒थि॒वि॒षदं॑ त्वाऽन्तरिक्ष॒सद॒मित्या॑ह। ए॒षामे॒वैतेन॑ लो॒काना सूयते। तस्माद्वाजपेयया॒जी न कंच॒न प्र॒त्यव॑रोहति। अपी॑व॒ हि दे॒वता॑ना सू॒यते॥५४॥

%1.3.9.3
ना॒क॒सद॒मित्या॑ह। य॒दा वै वसी॑या॒न्भव॑ति। नाक॑मग॒न्निति॒ वै तमा॑हुः। नाक॑मे॒वैतेन॑ गच्छति। ये ग्रहा पञ्चज॒नीना॒ इत्या॑ह। प॒ञ्च॒ज॒नाना॑मे॒वैतेन॑ सूयते। अ॒पा रस॒मुद्व॑यस॒मित्या॑ह। अ॒पामे॒वैतेन॒ रस॑स्य सूयते। सूर्य॑रश्मि स॒माभृ॑त॒मित्या॑ह सशुक्र॒त्वाय॑॥५५॥\anuvakamend[ग॒च्छ॒ति॒ सू॒यते॒ नव॑ च]

%1.3.10.1
इन्द्रो॑ वृ॒त्र ह॒त्वा। असु॑रान्परा॒भाव्य॑। सो॑ऽमावा॒स्यां प्रत्याग॑च्छत्। ते पि॒तर॑ पूर्वे॒द्युराग॑च्छन्। पि॒तॄन् य॒ज्ञो॑ऽगच्छत्। तं दे॒वाः पुन॑रयाचन्त। तमेभ्यो॒ न पुन॑रददुः। तेऽब्रुव॒न्वरं॑ वृणामहै। अथ॑ व॒ पुन॑र्दास्यामः। अ॒स्मभ्य॑मे॒व पूर्वे॒द्युः क्रि॑याता॒ इति॑॥५६॥

%1.3.10.2
तमेभ्य॒ पुन॑रददुः। तस्मात्पि॒तृभ्य॑ पूर्वे॒द्युः क्रि॑यते। यत्पि॒तृभ्य॑ पूर्वे॒द्युः क॒रोति॑। पि॒तृभ्य॑ ए॒व तद्य॒ज्ञं नि॒ष्क्रीय॒ यज॑मान॒ प्रत॑नुते। सोमा॑य पि॒तृपी॑ताय स्व॒धा नम॒ इत्या॑ह। पि॒तुरे॒वाधि॑ सोमपी॒थमव॑रुन्धे। न हि पि॒ता प्र॒मीय॑माण॒ आहै॒ष सो॑मपी॒थ इति॑। इ॒न्द्रि॒यं वै सो॑मपी॒थः। इ॒न्द्रि॒यमे॒व सो॑मपी॒थमव॑ रुन्धे। तेनेन्द्रि॒येण॑ द्वि॒तीयां जा॒याम॒भ्य॑श्नुते॥५७॥

%1.3.10.3
ए॒तद्वै ब्राह्म॑णं पु॒रा वा॑जवश्रव॒सा वि॒दाम॑क्रन्। तस्मा॒त्ते द्वेद्वे॑ जा॒ये अ॒भ्याक्षत। य ए॒वं वेद॑। अ॒भि द्वि॒तीयां॑ जा॒याम॑श्नुते। अ॒ग्नये॑ कव्य॒वाह॑नाय स्व॒धा नम॒ इत्या॑ह। य ए॒व पि॑तृ॒णाम॒ग्निः। तं प्री॑णाति। ति॒स्र आहु॑तीर्जुहोति। त्रिर्निद॑धाति। षट्त्संप॑द्यन्ते॥५८॥

%1.3.10.4
षड्वा ऋ॒तव॑। ऋ॒तूने॒व प्री॑णाति। तू॒ष्णीं मेक्ष॑ण॒माद॑धाति। अस्ति॑ वा॒ हि ष॒ष्ठ ऋ॒तुर्न वा। दे॒वान् वै पि॒तॄन्प्री॒तान्। म॒नु॒ष्या पि॒तरोऽनु॒ प्रपि॑पते। ति॒स्र आहु॑तीर्जुहोति। त्रिर्निद॑धाति। षट्त्संप॑द्यन्ते। षड्वा ऋ॒तव॑॥५९॥

%1.3.10.5
ऋ॒तव॒ खलु॒ वै दे॒वाः पि॒तर॑। ऋ॒तूने॒व दे॒वान्पि॒तॄन्प्री॑णाति। तान्प्री॒तान्। म॒नु॒ष्या पि॒तरोऽनु॒ प्रपि॑पते। स॒कृ॒दा॒च्छि॒न्नं ब॒र्\mbox{}हिर्भ॑वति। स॒कृदि॑व॒ हि पि॒तर॑। त्रिर्निद॑धाति। तृ॒तीये॒ वा इ॒तो लो॒के पि॒तर॑। ताने॒व प्री॑णाति। परा॒ङाव॑र्तते॥६०॥

%1.3.10.6
ह्लीका॒ हि पि॒तर॑। ओष्मणो व्या॒वृत॒ उपास्ते। ऊ॒ष्मभा॑गा॒ हि पि॒तर॑। ब्र॒ह्म॒वा॒दिनो॑ वदन्ति। प्राश्या (३) न्न प्राश्या (३) मिति॑। यत्प्राश्नी॒यात्। जन्य॒मन्न॑मद्यात्। प्र॒मायु॑कः स्यात्। यन्न प्राश्नी॒यात्। अह॑विः स्यात्॥६१॥

%1.3.10.7
पि॒तृभ्य॒ आवृ॑श्च्येत। अ॒व॒घ्रेय॑मे॒व। तन्नेव॒ प्राशि॑तं॒ नेवाप्रा॑शितम्। वी॒रं वा॒ वै पि॒तर॑ प्र॒यन्तो॒ हर॑न्ति। वी॒रं वा॑ ददति। द॒शां छि॑नत्ति। हर॑णभागा॒ हि पि॒तर॑। पि॒तॄने॒व नि॒रव॑दयते। उत्त॑र॒ आयु॑षि॒ लोम॑ छिन्दीत। पि॒तृ॒णा ह्ये॑तर्\mbox{}हि॒ नेदी॑यः॥६२॥

%1.3.10.8
नम॑स्करोति। न॒म॒स्का॒रो हि पि॑तृ॒णाम्। नमो॑ वः पितरो॒ रसा॑य। नमो॑ वः पितर॒ शुष्मा॑य। नमो॑ वः पितरो जी॒वाय॑। नमो॑ वः पितरः स्व॒धायै। नमो॑ वः पितरो म॒न्यवे। नमो॑ वः पितरो घो॒राय॑। पित॑रो॒ नमो॑ वः। य ए॒तस्मि॑ल्लोँ॒के स्थ॥६३॥

%1.3.10.9
यु॒ष्मास्तेऽनु॑। येऽस्मिल्लोँ॒के। मां तेऽनु॑। य ए॒तस्मि॑ल्लोँ॒के स्थ। यू॒यं तेषां॒ वसि॑ष्ठा भूयास्त। येऽस्मिल्लोँ॒के। अ॒हं तेषां॒ वसि॑ष्ठो भूयास॒मित्या॑ह। वसि॑ष्ठः समा॒नानां भवति। य ए॒वं वि॒द्वान्पि॒तृभ्य॑ क॒रोति॑। ए॒ष वै म॑नु॒ष्या॑णां य॒ज्ञः॥६४॥

%1.3.10.10
दे॒वानां॒ वा इत॑रे य॒ज्ञाः। तेन॒ वा ए॒तत्पि॑तृलो॒के च॑रति। यत्पि॒तृभ्य॑ क॒रोति॑। स ईश्व॒रः प्रमे॑तोः। प्रा॒जा॒प॒त्यय॒र्चा पुन॒रैति॑। य॒ज्ञो वै प्र॒जाप॑तिः। य॒ज्ञेनै॒व स॒ह पुन॒रैति॑। न प्र॒मायु॑को भवति। पि॒तृ॒लो॒के वा ए॒तद्यज॑मानश्चरति। यत्पि॒तृभ्य॑ क॒रोति॑। स ईश्व॒र आर्ति॒मार्तो। प्र॒जाप॑ति॒स्त्वावैनं॒ तत॒ उन्ने॑तुमर्\mbox{}ह॒तीत्या॑हुः। यत्प्रा॑जाप॒त्यय॒र्चा पुन॒रैति॑। प्र॒जाप॑तिरे॒वैनं॒ तत॒ उन्न॑यति। नार्ति॒मार्च्छ॑ति॒ यज॑मानः॥६५॥\anuvakamend[इत्य॑श्नुते पद्यन्ते पद्यन्ते॒ षड्वा ऋ॒तवो॑ वर्त॒तेऽह॑विः स्या॒न्नेदी॑य॒ स्थ य॒ज्ञो यज॑मानश्चरति॒ यत्पि॒तृभ्य॑ क॒रोति॒ पञ्च॑ च]




\prashnaend{दे॒वा॒सु॒रा अ॒ग्नीषोम॑योर्दे॒वा वै यथा॒दर्\mbox{}शं॑ दे॒वा वै यद॒न्यैर्ग्रहै॑र्ब्रह्मवा॒दिनो॒ नाग्नि॑ष्टो॒मो न सा॑वि॒त्रं दे॒वस्या॒हं ता॒र्प्य स॒प्तान्न॑हो॒मान्नृ॒षदं॒ त्वेन्द्रो॑ वृ॒त्र ह॒त्वा दश॑॥१०॥}{दे॒वा॒सु॒रा वा॒ज्ये॑वैनं॒ तस्माद्वाजपेयया॒जी दे॒वस्या॒हं वाज॒स्याव॑रुद्ध्या इन्द्रि॒यमे॒वास्मि॒न्॒ ह्लीका॒ हि पि॒तर॒ पञ्च॑षष्टिः॥६५॥}{दे॒वा॒सु॒रा यज॑मानः॥}{हरि॑ ओम्॥}{इति श्रीकृष्णयजुर्वेदीयतैत्तिरीयब्राह्मणे प्रथमाष्टके तृतीयः प्रपाठकः समाप्तः॥}
\clearpage
