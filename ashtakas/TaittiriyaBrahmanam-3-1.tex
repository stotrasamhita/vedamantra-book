\sect{प्रथमः प्रश्नः}
\setcounter{anuvakam}{0}
\dnsub{तैत्तिरीयब्राह्मणे तृतीयाष्टके प्रथमः प्रपाठकः}

%3.1.1.1
अ॒ग्निर्न॑ पातु॒ कृत्ति॑काः। नक्ष॑त्रन्दे॒वमि॑न्द्रि॒यम्। इ॒दमा॑सां विचक्ष॒णम्। ह॒विरा॒सं जु॑होतन। यस्य॒ भान्ति॑ र॒श्मयो॒ यस्य के॒तव॑। यस्ये॒मा विश्वा॒ भुव॑नानि॒ सर्वा। स कृत्ति॑काभिर॒भि सं॒वसा॑नः। अ॒ग्निर्नो॑ दे॒वः सु॑वि॒ते द॑धातु। प्र॒जाप॑ते रोहि॒णी वे॑तु॒ पत्नी। वि॒श्वरू॑पा बृह॒ती चि॒त्रभा॑नुः॥१॥

%3.1.1.2
सा नो॑ य॒ज्ञस्य॑ सुवि॒ते द॑धातु। यथा॒ जीवे॑म श॒रद॒ सवी॑राः। रो॒हि॒णी दे॒व्युद॑गात्पु॒रस्तात्। विश्वा॑ रू॒पाणि॑ प्रति॒मोद॑माना। प्र॒जाप॑ति ह॒विषा॑ व॒र्धय॑न्ती। प्रि॒या दे॒वाना॒मुप॑यातु य॒ज्ञम्। सोमो॒ राजा॑ मृगशी॒र्॒षेण॒ आग\sn{}। शि॒वं नक्ष॑त्रं प्रि॒यम॑स्य॒ धाम॑। आ॒प्याय॑मानो बहु॒धा जने॑षु। रेत॑ प्र॒जां यज॑माने दधातु॥२॥

%3.1.1.3
यत्ते॒ न॑क्षत्रं मृगशी॒र्‌षमस्ति॑। प्रि॒य रा॑जन्प्रि॒यत॑मं प्रि॒याणाम्। तस्मै॑ ते सोम ह॒विषा॑ विधेम। शन्न॑ एधि द्वि॒पदे॒ शञ्चतु॑ष्पदे। आ॒र्द्रया॑ रु॒द्रः प्रथ॑मान एति। श्रेष्ठो॑ दे॒वानां॒ पति॑रघ्नि॒यानाम्। नक्ष॑त्रमस्य ह॒विषा॑ विधेम। मा न॑ प्र॒जा री॑रिष॒न्मोत वी॒रान्। हे॒ती रु॒द्रस्य॒ परि॑ णो वृणक्तु। आ॒र्द्रा नक्ष॑त्रं जुषता ह॒विर्न॑॥३॥

%3.1.1.4
प्र॒मु॒ञ्चमा॑नौ दुरि॒तानि॒ विश्वा। अपा॒घशसन्नुदता॒मरा॑तिम्। पुन॑र्नो दे॒व्यदि॑तिः स्पृणोतु। पुन॑र्वसू न॒ पुन॒रेतां य॒ज्ञम्। पुन॑र्नो दे॒वा अ॒भिय॑न्तु॒ सर्वे। पुन॑ पुनर्वो ह॒विषा॑ यजामः। ए॒वा न दे॒व्यदि॑तिरन॒र्वा। विश्व॑स्य भ॒र्त्री जग॑तः प्रति॒ष्ठा। पुन॑र्वसू ह॒विषा॑ व॒र्धय॑न्ती। प्रि॒यन्दे॒वाना॒मप्ये॑तु॒ पाथ॑॥४॥

%3.1.1.5
बृह॒स्पति॑ प्रथ॒मं जाय॑मानः। ति॒ष्यं नक्ष॑त्रम॒भिसम्ब॑भूव। श्रेष्ठो॑ दे॒वानां॒ पृत॑नासु जि॒ष्णुः। दिशोऽनु॒ सर्वा॒ अभ॑यन्नो अस्तु। ति॒ष्य॑ पु॒रस्ता॑दु॒त म॑ध्य॒तो न॑। बृह॒स्पति॑र्न॒ परि॑ पातु प॒श्चात्। बाधे॑ता॒न्द्वेषो॒ अभ॑यङ्कृणुताम्। सु॒वीर्य॑स्य॒ पत॑यः स्याम। इ॒द स॒र्पेभ्यो॑ ह॒विर॑स्तु॒ जुष्टम्। आ॒श्रे॒षा येषा॑मनु॒यन्ति॒ चेत॑॥५॥

%3.1.1.6
ये अ॒न्तरि॑क्षं पृथि॒वीङ्क्षि॒यन्ति॑। ते न॑ स॒र्पासो॒ हव॒माग॑मिष्ठाः। ये रो॑च॒ने सूर्य॒स्यापि॑ स॒र्पाः। ये दिवं॑ दे॒वीमनु॑ स॒ञ्चर॑न्ति। येषा॑माश्रे॒षा अ॑नु॒यन्ति॒ कामम्। तेभ्य॑ स॒र्पेभ्यो॒ मधु॑मज्जुहोमि। उप॑हूताः पि॒तरो॒ ये म॒घासु॑। मनो॑जवसः सु॒कृत॑ सुकृ॒त्याः। ते नो॒ नक्ष॑त्रे॒ हव॒माग॑मिष्ठाः। स्व॒धाभि॑र्य॒ज्ञं प्रय॑तं जुषन्ताम्॥६॥

%3.1.1.7
ये अ॑ग्निद॒ग्धा येऽन॑ग्निदग्धाः। ये॑ऽमुं लो॒कं पि॒तर॑ क्षि॒यन्ति॑। याश्च॑ वि॒द्म या उ॑ च॒ न प्र॑वि॒द्म। म॒घासु॑ य॒ज्ञ सुकृ॑तं जुषन्ताम्। गवां॒ पति॒ फल्गु॑नीनामसि॒ त्वम्। तद॑र्यमन्वरुण मित्र॒ चारु॑। तन्त्वा॑ व॒य स॑नि॒तार सनी॒नाम्। जी॒वा जीव॑न्त॒मुप॒ संवि॑शेम। येने॒मा विश्वा॒ भुव॑नानि॒ सञ्जि॑ता। यस्य॑ दे॒वा अ॑नु सं॒ यन्ति॒ चेत॑॥७॥

%3.1.1.8
अ॒र्य॒मा राजा॒ऽजर॒स्तुवि॑ष्मान्। फल्गु॑नीनामृष॒भो रो॑रवीति। श्रेष्ठो॑ दे॒वानां भगवो भगासि। तत्त्वा॑ विदु॒ फल्गु॑नी॒स्तस्य॑ वित्तात्। अ॒स्मभ्यं॑ क्ष॒त्रम॒जर सु॒वीर्यम्। गोम॒दश्व॑व॒दुप॒ सन्नु॑दे॒ह। भगो॑ ह दा॒ता भग॒ इत्प्र॑दा॒ता। भगो॑ दे॒वीः फल्गु॑नी॒रा वि॑वेश। भग॒स्येत्तं प्र॑स॒वं ग॑मेम। यत्र॑ दे॒वैः स॑ध॒मादं॑ मदेम॥८॥

%3.1.1.9
आया॑तु दे॒वः स॑वि॒तोप॑यातु। हि॒र॒ण्यये॑न सु॒वृता॒ रथे॑न। वह॒न्॒ हस्त सु॒भगं॑ विद्म॒नाप॑सम्। प्र॒यच्छ॑न्तं॒ पपु॑रिं॒ पुण्य॒मच्छ॑। हस्त॒ प्रय॑च्छत्व॒मृतं॒ वसी॑यः। दक्षि॑णेन॒ प्रति॑गृभ्णीम एनत्। दा॒तार॑म॒द्य स॑वि॒ता वि॑देय। यो नो॒ हस्ता॑य प्रसु॒वाति॑ य॒ज्ञम्। त्वष्टा॒ नक्ष॑त्रम॒भ्ये॑ति चि॒त्राम्। सु॒भ स॑सं युव॒ति रोच॑मानाम्॥९॥

%3.1.1.10
नि॒वे॒शय॑न्न॒मृता॒न्मर्त्याश्च। रू॒पाणि॑ पि॒शन्भुव॑नानि॒ विश्वा। तन्न॒स्त्वष्टा॒ तदु॑ चि॒त्रा विच॑ष्टाम्। तन्नक्ष॑त्रं भूरि॒दा अ॑स्तु॒ मह्यम्। तन्न॑ प्र॒जां वी॒रव॑ती सनोतु। गोभि॑र्नो॒ अश्वै॒ सम॑नक्तु य॒ज्ञम्। वा॒युर्नक्ष॑त्रम॒भ्ये॑ति॒ निष्ट्याम्। ति॒ग्मशृ॑ङ्गो वृष॒भो रोरु॑वाणः। स॒मी॒रय॒न्भुव॑ना मात॒रिश्वा। अप॒ द्वेषासि नुदता॒मरा॑तीः॥१०॥

%3.1.1.11
तन्नो॑ वा॒युस्तदु॒ निष्ट्या॑ शृणोतु। तन्नक्ष॑त्रं भूरि॒दा अ॑स्तु॒ मह्यम्। तन्नो॑ दे॒वासो॒ अनु॑जानन्तु॒ कामम्। यथा॒ तरे॑म दुरि॒तानि॒ विश्वा। दू॒रम॒स्मच्छत्र॑वो यन्तु भी॒ताः। तदि॑न्द्रा॒ग्नी कृ॑णुता॒न्तद्विशा॑खे। तन्नो॑ दे॒वा अनु॑मदन्तु य॒ज्ञम्। प॒श्चात्पु॒रस्ता॒दभ॑यन्नो अस्तु। नक्ष॑त्राणा॒मधि॑पत्नी॒ विशा॑खे। श्रेष्ठा॑विन्द्रा॒ग्नी भुव॑नस्य गो॒पौ॥११॥

%3.1.1.12
विषू॑च॒ शत्रू॑नप॒ बाध॑मानौ। अप॒ क्षुध॑न्नुदता॒मरा॑तिम्। पू॒र्णा प॒श्चादु॒त पू॒र्णा पु॒रस्तात्। उन्म॑ध्य॒तः पौर्णमा॒सी जि॑गाय। तस्यान्दे॒वा अधि॑ सं॒वस॑न्तः। उ॒त्त॒मे नाक॑ इ॒ह मा॑दयन्ताम्। पृ॒थ्वी सु॒वर्चा॑ युव॒तिः स॒जोषा। पौ॒र्ण॒मा॒स्युद॑गा॒च्छोभ॑माना। आ॒प्या॒यय॑न्ती दुरि॒तानि॒ विश्वा। उ॒रुन्दुहां॒ यज॑मानाय य॒ज्ञम्॥१२॥\anuvakamend[चि॒त्रभा॑नु॒र्यज॑माने दधातु ह॒विर्न॒ पाथ॒श्चेतो॑ जुषन्ता॒ञ्चेतो॑ मदेम॒ रोच॑माना॒मरा॑तीर्गो॒पौ य॒ज्ञम्]

%3.1.2.1
ऋ॒द्ध्यास्म॑ ह॒व्यैर्नम॑सोप॒सद्य॑। मि॒त्रन्दे॒वं मि॑त्र॒धेय॑न्नो अस्तु। अ॒नू॒रा॒धान् ह॒विषा॑ व॒र्धय॑न्तः। श॒तञ्जी॑वेम श॒रद॒ सवी॑राः। चि॒त्रं नक्ष॑त्र॒मुद॑गात्पु॒रस्तात्। अ॒नू॒रा॒धास॒ इति॒ यद्वद॑न्ति। तन्मि॒त्र ए॑ति प॒थिभि॑र्देव॒यानै। हि॒र॒ण्ययै॒र्वित॑तैर॒न्तरि॑क्षे। इन्द्रो ज्ये॒ष्ठामनु॒ नक्ष॑त्रमेति। यस्मि॑न्वृ॒त्रं वृ॑त्र॒तूर्ये॑ त॒तार॑॥१३॥

%3.1.2.2
तस्मि॑न्व॒यम॒मृत॒न्दुहा॑नाः। क्षुध॑न्तरेम॒ दुरि॑ति॒न्दुरि॑ष्टिम्। पु॒र॒न्द॒राय॑ वृष॒भाय॑ धृ॒ष्णवे। अषा॑ढाय॒ सह॑मानाय मी॒ढुषे। इन्द्रा॑य ज्ये॒ष्ठा मधु॑म॒द्दुहा॑ना। उ॒रुं कृ॑णोतु॒ यज॑मानाय लो॒कम्। मूलं॑ प्र॒जां वी॒रव॑तीं विदेय। पराच्येतु॒ निर्‌ऋ॑तिः परा॒चा। गोभि॒र्नक्ष॑त्रं प॒शुभि॒ सम॑क्तम्। अह॑र्भूया॒द्यज॑मानाय॒ मह्यम्॥१४॥

%3.1.2.3
अह॑र्नो अ॒द्य सु॑वि॒ते द॑धातु। मूलं॒ नक्ष॑त्र॒मिति॒ यद्वद॑न्ति। परा॑चीं वा॒चा निर्‌ऋ॑तिन्नुदामि। शि॒वं प्र॒जायै॑ शि॒वम॑स्तु॒ मह्यम्। या दि॒व्या आप॒ पय॑सा सम्बभू॒वुः। या अ॒न्तरि॑क्ष उ॒त पार्थि॑वी॒र्याः। यासा॑मषा॒ढा अ॑नु॒यन्ति॒ कामम्। ता न॒ आप॒ श स्यो॒ना भ॑वन्तु। याश्च॒ कूप्या॒ याश्च॑ ना॒द्या समु॒द्रिया। याश्च॑ वैश॒न्तीरु॒त प्रा॑स॒चीर्याः॥१५॥

%3.1.2.4
यासा॑मषा॒ढा मधु॑ भ॒क्षय॑न्ति॒। ता न॒ आप॒ श स्यो॒ना भ॑वन्तु। तन्नो॒ विश्वे॒ उप॑ शृण्वन्तु दे॒वाः। तद॑षा॒ढा अ॒भिसंय॑न्तु य॒ज्ञम्। तन्नक्ष॑त्रं प्रथतां प॒शुभ्य॑। कृ॒षिर्वृ॒ष्टिर्यज॑मानाय कल्पताम्। शु॒भ्राः क॒न्या॑ युव॒तय॑ सु॒पेश॑सः। क॒र्म॒कृत॑ सु॒कृतो॑ वी॒र्या॑वतीः। विश्वान्दे॒वान् ह॒विषा॑ व॒र्धय॑न्तीः। अ॒षा॒ढाः काम॒मुप॑ यान्तु य॒ज्ञम्॥१६॥

%3.1.2.5
यस्मि॒न्ब्रह्मा॒ऽभ्यज॑य॒त्सर्व॑मे॒तत्। अ॒मुं च॑ लो॒कमि॒दमू॑ च॒ सर्वम्। तन्नो॒ नक्ष॑त्रमभि॒जिद्वि॒जित्य॑। श्रिय॑न्दधा॒त्वहृ॑णीयमानम्। उ॒भौ लो॒कौ ब्रह्म॑णा॒ सञ्जि॑ते॒मौ। तन्नो॒ नक्ष॑त्रमभि॒जिद्विच॑ष्टाम्। तस्मि॑न्व॒यं पृत॑ना॒ सञ्ज॑येम। तन्नो॑ दे॒वासो॒ अनु॑जानन्तु॒ कामम्। शृ॒ण्वन्ति॑ श्रो॒णाम॒मृत॑स्य गो॒पाम्। पुण्या॑मस्या॒ उप॑शृणोमि॒ वाचम्॥१७॥

%3.1.2.6
म॒हीं देवीं विष्णु॑पत्नीमजू॒र्याम्। प्र॒तीची॑मेना ह॒विषा॑ यजामः। त्रे॒धा विष्णु॑रुरुगा॒यो विच॑क्रमे। म॒हीन्दिवं॑ पृथि॒वीम॒न्तरि॑क्षम्। तच्छ्रो॒णैति॒ श्रव॑ इ॒च्छमा॑ना। पुण्य॒ श्लोकं॒ यज॑मानाय कृण्व॒ती। अ॒ष्टौ दे॒वा वस॑वः सो॒म्यास॑। चत॑स्रो दे॒वीर॒जरा॒ श्रवि॑ष्ठाः। ते य॒ज्ञं पान्तु॒ रज॑सः प॒रस्तात्। सं॒व॒त्स॒रीण॑म॒मृत स्व॒स्ति॥१८॥

%3.1.2.7
य॒ज्ञं न॑ पान्तु॒ वस॑वः पु॒रस्तात्। द॒क्षि॒ण॒तो॑ऽभिय॑न्तु॒ श्रवि॑ष्ठाः। पुण्यं॒ नक्ष॑त्रम॒भि संवि॑शाम। मा नो॒ अरा॑तिर॒घश॒साऽग\sn{}। क्ष॒त्रस्य॒ राजा॒ वरु॑णोऽधिरा॒जः। नक्ष॑त्राणा श॒तभि॑ष॒ग्वसि॑ष्ठः। तौ दे॒वेभ्य॑ कृणुतो दी॒र्घमायु॑। श॒त स॒हस्रा॑ भेष॒जानि॑ धत्तः। य॒ज्ञन्नो॒ राजा॒ वरु॑ण॒ उप॑यातु। तन्नो॒ विश्वे॑ अ॒भि संय॑न्तु दे॒वाः॥१९॥

%3.1.2.8
तन्नो॒ नक्ष॑त्र श॒तभि॑षग्जुषा॒णम्। दी॒र्घमायु॒ प्रति॑रद्भेष॒जानि॑। अ॒ज एक॑पा॒दुद॑गात्पु॒रस्तात्। विश्वा॑ भू॒तानि॑ प्रति॒मोद॑मानः। तस्य॑ दे॒वाः प्र॑स॒वं य॑न्ति॒ सर्वे। प्रो॒ष्ठ॒प॒दासो॑ अ॒मृत॑स्य गो॒पाः। वि॒भ्राज॑मानः समिधा॒न उ॒ग्रः। आऽन्तरि॑क्षमरुह॒दग॒न्द्याम्। त सूर्यं॑ दे॒वम॒जमेक॑पादम्। प्रो॒ष्ठ॒प॒दासो॒ अनु॑यन्ति॒ सर्वे॥२०॥

%3.1.2.9
अहि॑र्बु॒ध्निय॒ प्रथ॑मान एति। श्रेष्ठो॑ दे॒वाना॑मु॒त मानु॑षाणाम्। तं ब्राह्म॒णाः सो॑म॒पाः सो॒म्यास॑। प्रो॒ष्ठ॒प॒दासो॑ अ॒भि र॑क्षन्ति॒ सर्वे। च॒त्वार॒ एक॑म॒भिकर्म॑ दे॒वाः। प्रो॒ष्ठ॒प॒दास॒ इति॒ यान् वद॑न्ति। ते बु॒ध्नियं॑ परि॒षद्य स्तु॒वन्त॑। अहि रक्षन्ति॒ नम॑सोप॒सद्य॑। पू॒षा रे॒वत्यन्वे॑ति॒ पन्थाम्। पु॒ष्टि॒पती॑ पशु॒पा वाज॑बस्त्यौ॥२१॥

%3.1.2.10
इ॒मानि॑ ह॒व्या प्रय॑ता जुषा॒णा। सु॒गैर्नो॒ यानै॒रुप॑यातां य॒ज्ञम्। क्षु॒द्रान्प॒शून्र॑क्षतु रे॒वती॑ नः। गावो॑ नो॒ अश्वा॒ अन्वे॑तु पू॒षा। अन्न॒ रक्ष॑न्तौ बहु॒धा विरू॑पम्। वाज सनुतां॒ यज॑मानाय य॒ज्ञम्। तद॒श्विना॑वश्व॒युजोप॑याताम्। शुभ॒ङ्गमि॑ष्ठौ सु॒यमे॑भि॒रश्वै। स्वं नक्ष॑त्र ह॒विषा॒ यज॑न्तौ। मध्वा॒ संपृ॑क्तौ॒ यजु॑षा॒ सम॑क्तौ॥२२॥

%3.1.2.11
यौ दे॒वानां भि॒षजौ॑ हव्यवा॒हौ। विश्व॑स्य दू॒ताव॒मृत॑स्य गो॒पौ। तौ नक्ष॑त्रं जुजुषा॒णोप॑याताम्। नमो॒ऽश्विभ्याङ्कृणुमोऽश्व॒युग्भ्याम्। अप॑ पा॒प्मानं॒ भर॑णीर्भरन्तु। तद्य॒मो राजा॒ भग॑वा॒न्॒ विच॑ष्टाम्। लो॒कस्य॒ राजा॑ मह॒तो म॒हान् हि। सु॒गन्नः पन्था॒मभ॑यङ्कृणोतु। यस्मि॒न्नक्ष॑त्रे य॒म एति॒ राजा। यस्मि॑न्नेनम॒भ्यषि॑ञ्चन्त दे॒वाः। तद॑स्य चि॒त्र ह॒विषा॑ यजाम। अप॑ पा॒प्मानं॒ भर॑णीर्भरन्तु। नि॒वेश॑नी॒ यत्ते॑ दे॒वा अद॑धुः॥२३॥\anuvakamend[त॒तार॒ मह्यं॑ प्रास॒चीर्या यान्तु य॒ज्ञं वाच स्व॒स्ति दे॒वा अनु॑यन्ति॒ सर्वे॒ वाज॑बस्त्यौ॒ सम॑क्तौ दे॒वास्त्रीणि॑ च]

%3.1.3.1
नवो॑नवो भवति॒ जाय॑मानो॒ यमा॑दि॒त्या अ॒शुमाप्या॒यय॑न्ति। ये विरू॑पे॒ सम॑नसा स॒व्व्यँय॑न्ती। स॒मा॒नन्तन्तुं॑ परितात॒ना ते। वि॒भू प्र॒भू अ॑नु॒भू वि॒श्वतो॑ हुवे। ते नो॒ नक्ष॑त्रे॒ हव॒माग॑मेतम्। व॒यन्दे॒वी ब्रह्म॑णा संविदा॒नाः। सु॒रत्ना॑सो दे॒ववी॑ति॒न्दधा॑नाः। अ॒हो॒रा॒त्रे ह॒विषा॑ व॒र्धय॑न्तः। अति॑ पा॒प्मान॒मति॑ मुक्त्या गमेम। प्रत्यु॑वदृश्याय॒ती॥२४॥

%3.1.3.2
व्यु॒च्छन्ती॑ दुहि॒ता दि॒वः। अ॒पो म॒ही वृ॑णुते॒ चक्षु॑षा। तमो॒ ज्योति॑ष्कृणोति सू॒नरी। उदु॒स्रिया सचते॒ सूर्य॑। सचा॑ उ॒द्यन्नक्ष॑त्रमर्चि॒मत्। तवेदु॑षो॒ व्युषि॒ सूर्य॑स्य च। सं भ॒क्तेन॑ गमेमहि। तन्नो॒ नक्ष॑त्रमर्चि॒मत्। भा॒नु॒मत्तेज॑ उ॒च्चर॑त्। उप॑य॒ज्ञमि॒हाग॑मत्॥२५॥

%3.1.3.3
प्र नक्ष॑त्राय दे॒वाय॑। इन्द्रा॒येन्दु हवामहे। सन॑ सवि॒ता सु॑वत्स॒निम्। पु॒ष्टि॒दां वी॒रव॑त्तमम्। उदु॒त्यञ्चि॒त्रम्। अदि॑तिर्न उरुष्यतु म॒हीमू॒ षु मा॒तरम्। इ॒दं विष्णु॒ प्रतद्विष्णु॑। अ॒ग्निर्मू॒र्धा भुव॑। अनु॑नो॒ऽद्यानु॑मति॒रन्विद॑नुमते॒ त्वम्। ह॒व्य॒वाह॒ स्वि॑ष्टम्॥२६॥\anuvakamend[आ॒य॒त्य॑गम॒त्स्वि॑ष्टम्]

%3.1.4.1
अ॒ग्निर्वा अ॑कामयत। अ॒न्ना॒दो दे॒वानास्या॒मिति॑। स ए॒तम॒ग्नये॒ कृत्ति॑काभ्यः पुरो॒डाश॑म॒ष्टाक॑पालं॒ निर॑वपत्। ततो॒ वै सोऽन्ना॒दो दे॒वाना॑मभवत्। अ॒ग्निर्वै दे॒वाना॑मन्ना॒दः। यथा॑ ह॒ वा अ॒ग्निर्दे॒वाना॑मन्ना॒दः। ए॒व ह॒ वा ए॒ष म॑नु॒ष्या॑णां भवति। य ए॒तेन॑ ह॒विषा॒ यज॑ते। य उ॑ चैनदे॒वं वेद॑। सोऽत्र॑ जुहोति। अ॒ग्नये॒ स्वाहा॒ कृत्ति॑काभ्य॒ स्वाहा। अ॒म्बायै॒ स्वाहा॑ दु॒लायै॒ स्वाहा। नि॒त॒त्न्यै स्वाहा॒ऽभ्रय॑न्त्यै॒ स्वाहा। मे॒घय॑न्त्यै॒ स्वाहा॑ व॒र्॒षय॑न्त्यै॒ स्वाहा। चु॒पु॒णीका॑यै॒ स्वाहेति॑॥२७॥

%3.1.4.2
प्र॒जाप॑तिः प्र॒जा अ॑सृजत। ता अ॑स्मात्सृ॒ष्टाः परा॑चीरायन्। तासा रोहि॒णीम॒भ्य॑ध्यायत्। सो॑ऽकामयत। उप॒ मा व॑र्तेत। समे॑नया गच्छे॒येति॑। स ए॒तं प्र॒जाप॑तये रोहि॒ण्यै च॒रुन्निर॑वपत्। ततो॒ वै सा तमु॒पाव॑र्तत। समे॑नयागच्छत। उप॑ ह॒ वा ए॑नं प्रि॒यमाव॑र्तते। सं प्रि॒येण॑ गच्छते। य ए॒तेन॑ ह॒विषा॒ यज॑ते। य उ॑चैनदे॒वं वेद॑। सोऽत्र॑ जुहोति। प्र॒जाप॑तये॒ स्वाहा॑ रोहि॒ण्यै स्वाहा। रोच॑मानायै॒ स्वाहा प्र॒जाभ्य॒ स्वाहेति॑॥२८॥

%3.1.4.3
सोमो॒ वा अ॑कामयत। ओष॑धीना रा॒ज्यम॒भिज॑येय॒मिति॑। स ए॒त सोमा॑य मृगशी॒र्॒षाय॑ श्यामा॒कञ्च॒रुं पय॑सि॒ निर॑वपत्। ततो॒ वै स ओष॑धीना रा॒ज्यम॒भ्य॑जयत्। स॒मा॒नाना ह॒ वै रा॒ज्यम॒भिज॑यति। य ए॒तेन॑ ह॒विषा॒ यज॑ते। य उ॑ चैनदे॒वं वेद॑। सोऽत्र॑ जुहोति। सोमा॑य॒ स्वाहा॑ मृगशी॒र्‌षाय॒ स्वाहा। इ॒न्व॒काभ्य॒ स्वाहौष॑धीभ्य॒ स्वाहा। रा॒ज्याय॒ स्वाहा॒ऽभिजि॑त्यै॒ स्वाहेति॑॥२९॥

%3.1.4.4
रु॒द्रो वा अ॑कामयत। प॒शु॒मान्त्स्या॒मिति॑। स ए॒त रु॒द्राया॒र्द्रायै॒ प्रैय्य॑ङ्गवञ्च॒रुं पय॑सि॒ निर॑वपत्। ततो॒ वै स प॑शु॒मान॑भवत्। प॒शु॒मान् ह॒ वै भ॑वति। य ए॒तेन॑ ह॒विषा॒ यज॑ते। य उ॑ चैनदे॒वं वेद॑। सोऽत्र॑ जुहोति। रु॒द्राय॒ स्वाहा॒ऽर्द्रायै॒ स्वाहा। पिन्व॑मानायै॒ स्वाहा॑ प॒शुभ्य॒ स्वाहेति॑॥३०॥

%3.1.4.5
ऋ॒क्षा वा इ॒यम॑लो॒मका॑ऽऽसीत्। साऽका॑मयत। ओष॑धीभि॒र्वन॒स्पति॑भि॒ प्रजा॑ये॒येति॑। सैतमदि॑त्यै॒ पुन॑र्वसुभ्याञ्च॒रुन्निर॑वपत्। ततो॒ वा इ॒यमोष॑धीभि॒र्वन॒स्पति॑भि॒ प्राजा॑यत। प्रजा॑यते ह॒ वै प्र॒जया॑ प॒शुभि॑। य ए॒तेन॑ ह॒विषा॒ यज॑ते। य उ॑ चैनदे॒वं वेद॑। सोऽत्र॑ जुहोति। अदि॑त्यै॒ स्वाहा॒ पुन॑र्वसुभ्याम्। स्वाहा भूत्यै॒ स्वाहा॒ प्रजात्यै॒ स्वाहेति॑॥३१॥

%3.1.4.6
बृह॒स्पति॒र्वा अ॑कामयत। ब्र॒ह्म॒व॒र्च॒सी स्या॒मिति॑। स ए॒तं बृह॒स्पत॑ये ति॒ष्या॑य नैवा॒रञ्च॒रुं पय॑सि॒ निर॑वपत्। ततो॒ वै स ब्र॑ह्मवर्च॒स्य॑भवत्। ब्र॒ह्म॒व॒र्च॒सी ह॒ वै भ॑वति। य ए॒तेन॑ ह॒विषा॒ यज॑ते। य उ॑ चैनदे॒वं वेद॑। सोऽत्र॑ जुहोति। बृह॒स्पत॑ये॒ स्वाहा॑ ति॒ष्या॑य॒ स्वाहा। ब्र॒ह्म॒व॒र्च॒साय॒ स्वाहेति॑॥३२॥

%3.1.4.7
दे॒वा॒सु॒राः संय॑त्ता आसन्। ते दे॒वाः स॒र्पेभ्य॑ आश्रे॒षाभ्य॒ आज्ये॑ कर॒म्भन्निर॑वपन्। ताने॒ताभि॑रे॒वदे॒वता॑भि॒रुपा॑नयन्। ए॒ताभि॑र्‌ह॒ वै दे॒वता॑भिर्द्वि॒षन्तं॒ भ्रातृ॑व्य॒मुप॑नयति। य ए॒तेन॑ ह॒विषा॒ यज॑ते। य उ॑ चैनदे॒वं वेद॑। सोऽत्र॑ जुहोति। स॒र्पेभ्य॒ स्वाहाऽऽश्रे॒षाभ्य॒ स्वाहा। द॒न्द॒शूकेभ्य॒ स्वाहेति॑॥३३॥

%3.1.4.8
पि॒तरो॒ वा अ॑कामयन्त। पि॒तृ॒लो॒क ऋ॑ध्नुया॒मेति॑। त ए॒तं पि॒तृभ्यो॑ म॒घाभ्य॑ पुरो॒डाश॒ षट्क॑पालं॒ निर॑वपन्। ततो॒ वै ते पि॑तृलो॒क आर्ध्नुवन्। पि॒तृ॒लो॒के ह॒ वा ऋ॑ध्नोति। य ए॒तेन॑ ह॒विषा॒ यज॑ते। य उ॑ चैनदे॒वं वेद॑। सोऽत्र॑ जुहोति। पि॒तृभ्य॒ स्वाहा॑ म॒घाभ्य॑। स्वाहा॑ऽन॒घाभ्य॒ स्वाहा॑ग॒दाभ्य॑। स्वाहा॑ऽरुन्ध॒तीभ्य॒ स्वाहेति॑॥३४॥

%3.1.4.9
अ॒र्य॒मा वा अ॑कामयत। प॒शु॒मान्त्स्या॒मिति॑। स ए॒तम॑र्य॒म्णे फल्गु॑नीभ्याञ्च॒रुन्निर॑वपत्। ततो॒ वै स प॑शु॒मान॑भवत्। प॒शुमान् ह॒ वै भ॑वति। य ए॒तेन॑ ह॒विषा॒ यज॑ते। य उ॑ चैनदे॒वं वेद॑। सोऽत्र॑ जुहोति। अ॒र्य॒म्णे स्वाहा॒ फल्गु॑नीभ्या॒ स्वाहा। प॒शुभ्य॒ स्वाहेति॑॥३५॥

%3.1.4.10
भगो॒ वा अ॑कामयत। भ॒गी श्रे॒ष्ठी दे॒वानास्या॒मिति॑। स ए॒तं भगा॑य॒ फल्गु॑नीभ्याञ्च॒रुन्निर॑वपत्। ततो॒ वै स भ॒गी श्रे॒ष्ठी दे॒वाना॑मभवत्। भ॒गी ह॒ वै श्रे॒ष्ठी स॑मा॒नानां भवति। य ए॒तेन॑ ह॒विषा॒ यज॑ते। य उ॑ चैनदे॒वं वेद॑। सोऽत्र॑ जुहोति। भगा॑य॒ स्वाहा॒ फल्गु॑नीभ्या॒ स्वाहा। श्रैष्ठ्या॑य॒ स्वाहेति॑॥३६॥

%3.1.4.11
स॒वि॒ता वा अ॑कामयत। श्रन्मे॑ दे॒वा दधी॑रन्। स॒वि॒ता स्या॒मिति॑। स ए॒त स॑वि॒त्रे हस्ता॑य पुरो॒डाशं॒ द्वाद॑शकपालं॒ निर॑वपदाशू॒नां व्री॑ही॒णाम्। ततो॒ वै तस्मै॒ श्रद्दे॒वा अद॑धत। स॒वि॒ताऽभ॑वत्। श्रद्ध॒वा अ॑स्मै मनु॒ष्या॑ दधते। स॒वि॒ता स॑मा॒नानां भवति। य ए॒तेन॑ ह॒विषा॒ यज॑ते। य उ॑ चैनदे॒वं वेद॑। सोऽत्र॑ जुहोति। स॒वि॒त्रे स्वाहा॒ हस्ता॑य। स्वाहा॑ दद॒ते स्वाहा॑ पृण॒ते। स्वाहा प्र॒यच्छ॑ते॒ स्वाहा प्रतिगृभ्ण॒ते स्वाहेति॑ ॥३७॥

%3.1.4.12
त्वष्टा॒ वा अ॑कामयत। चि॒त्रं प्र॒जां वि॑न्दे॒येति॑। स ए॒तन्त्वष्ट्रे॑ चि॒त्रायै॑ पुरो॒डाश॑म॒ष्टाक॑पालं॒ निर॑वपत्। ततो॒ वै स चि॒त्रं प्र॒जाम॑विन्दत। चि॒त्र ह॒ वै प्र॒जां वि॑न्दते। य ए॒तेन॑ ह॒विषा॒ यज॑ते। य उ॑ चैनदे॒वं वेद॑। सोऽत्र॑ जुहोति। त्वष्ट्रे॒ स्वाहा॑ चि॒त्रायै॒ स्वाहा। चैत्रा॑य॒ स्वाहा प्र॒जायै॒ स्वाहेति॑॥३८॥

%3.1.4.13
वा॒युर्वा अ॑कामयत। का॒म॒चार॑मे॒षु लो॒केष्व॒भिज॑येय॒मिति॑। स ए॒तद्वा॒यवे॒ निष्ट्या॑यै गृ॒ष्ट्यै दु॒ग्धं पयो॒ निर॑वपत्। ततो॒ वै स का॑म॒चार॑मे॒षु लो॒केष्व॒भ्य॑जयत्। का॒म॒चार ह॒ वा ए॒षु लो॒केष्व॒भिज॑यति। य ए॒तेन॑ ह॒विषा॒ यज॑ते। य उ॑ चैनदे॒वं वेद॑। सोऽत्र॑ जुहोति। वा॒यवे॒ स्वाहा॒ निष्ट्या॑यै॒ स्वाहा। का॒म॒चारा॑य॒ स्वाहा॒ऽभिजि॑त्यै॒ स्वाहेति॑॥३९॥

%3.1.4.14
इ॒न्द्रा॒ग्नी वा अ॑कामयेताम्। श्रैष्ठ्यं॑ दे॒वाना॑म॒भिज॑ये॒वेति॑। तावे॒तमि॑न्द्रा॒ग्निभ्यां॒ विशा॑खाभ्यां पुरो॒डाश॒मेका॑दशकपालं॒ निर॑वपताम्। ततो॒ वै तौ श्रैष्ठ्यं॑ दे॒वाना॑म॒भ्य॑जयताम्। श्रैष्ठ्य ह॒ वै स॑मा॒नाना॑म॒भि ज॑यति। य ए॒तेन॑ ह॒विषा॒ यज॑ते। य उ॑ चैनदे॒वं वेद॑। सोऽत्र॑ जुहोति। इ॒न्द्रा॒ग्निभ्या॒ स्वाहा॒ विशा॑खाभ्या॒ स्वाहा। श्रैष्ठ्या॑य॒ स्वाहा॒ऽभिजि॑त्यै॒ स्वाहेति॑॥४०॥

%3.1.4.15
अथै॒तत्पौर्णमा॒स्या आज्यं॒ निर्व॑पति। कामो॒ वै पौर्णमा॒सी। काम॒ आज्यम्। कामे॑नै॒व काम॒ सम॑र्धयति। क्षि॒प्रमे॑न॒ सकाम॒ उप॑नमति। येन॒ कामे॑न॒ यज॑ते। सोऽत्र॑ जुहोति। पौ॒र्ण॒मा॒स्यै स्वाहा॒ कामा॑य॒ स्वाहाऽऽग॑त्यै॒ स्वाहेति॑॥४१॥\anuvakamend[अ॒ग्निः पञ्च॑दश प्र॒जाप॑ति॒ष्षोड॑श॒ सोम॒ एका॑दश रु॒द्रो दश॒र्क्षैका॑दश॒ बृह॒स्पति॒र्दश॑ देवासु॒रा नव॑ पि॒तर॒ एका॑दशार्य॒मा भगो॒ दश॑ दश सवि॒ता चतु॑र्दश॒ त्वष्टा॑ वा॒युरि॑न्द्रा॒ग्नी दश॑ द॒शाथै॒तत्पौर्णमा॒स्या अ॒ष्टौ पञ्च॑दश]

%3.1.5.1
मि॒त्रो वा अ॑कामयत। मि॒त्र॒धेय॑मे॒षु लो॒केष्व॒भिज॑येय॒मिति॑। स ए॒तं मि॒त्राया॑नूरा॒धेभ्य॑श्च॒रुन्निर॑वपत्। ततो॒ वै स मि॑त्र॒धेय॑मे॒षुलो॒केष्व॒भ्य॑जयत्। मि॒त्र॒धेय ह॒ वा ए॒षु लो॒केष्व॒भिज॑यति। य ए॒तेन॑ ह॒विषा॒ यज॑ते। य उ॑ चैनदे॒वं वेद॑। सोऽत्र॑ जुहोति। मि॒त्राय॒ स्वाहा॑ऽनूरा॒धेभ्य॒ स्वाहा। मि॒त्र॒धेया॑य॒ स्वाहा॒ऽभिजि॑त्यै॒ स्वाहेति॑॥४२॥

%3.1.5.2
इन्द्रो॒ वा अ॑कामयत। ज्यैष्ठ्यं॑ दे॒वाना॑म॒भिज॑येय॒मिति॑। स ए॒तमिन्द्रा॑य ज्ये॒ष्ठायै॑ पुरो॒डाश॒मेका॑दशकपालं॒ निर॑वपन्म॒हाव्री॑हीणाम्। ततो॒ वै स ज्यैष्ठ्यं॑ दे॒वाना॑म॒भ्य॑जयत्। ज्यैष्ठ्य ह॒ वै स॑मा॒नाना॑म॒भिज॑यति। य ए॒तेन॑ ह॒विषा॒ यज॑ते। य उ॑ चैनदे॒वं वेद॑। सोऽत्र॑ जुहोति। इन्द्रा॑य॒ स्वाहा ज्ये॒ष्ठायै॒ स्वाहा। ज्यैष्ठ्या॑य॒ स्वाहा॒भिजि॑त्यै॒ स्वाहेति॑॥४३॥

%3.1.5.3
प्र॒जाप॑ति॒र्वा अ॑कामयत। मूलं॑ प्र॒जां वि॑न्दे॒येति॑। स ए॒तं प्र॒जाप॑तये॒ मूला॑य च॒रुन्निर॑वपत्। ततो॒ वै स मूलं॑ प्र॒जाम॑विन्दत। मूल ह॒ वै प्र॒जां वि॑न्दते। य ए॒तेन॑ ह॒विषा॒ यज॑ते। य उ॑ चैनदे॒वं वेद॑। सोऽत्र॑ जुहोति। प्र॒जाप॑तये॒ स्वाहा॒ मूला॑य॒ स्वाहा। प्र॒जायै॒ स्वाहेति॑॥४४॥

%3.1.5.4
आपो॒ वा अ॑कामयन्त। स॒मु॒द्रङ्काम॑म॒भिज॑ये॒मेति॑। ता ए॒तम॒द्भ्यो॑ऽषा॒ढाभ्य॑श्च॒रुन्निर॑वपन्। ततो॒ वै ताः स॑मु॒द्रङ्काम॑म॒भ्य॑जयन्। स॒मु॒द्र ह॒ वै काम॑म॒भिज॑यति। य ए॒तेन॑ ह॒विषा॒ यज॑ते। य उ॑ चैनदे॒वं वेद॑। सोऽत्र॑ जुहोति। अ॒द्भ्यः स्वाहा॑ऽषा॒ढाभ्य॒ स्वाहा। स॒मु॒द्राय॒ स्वाहा॒ कामा॑य॒ स्वाहा। अ॒भिजि॑त्यै॒ स्वाहेति॑॥४५॥

%3.1.5.5
विश्वे॒ वै दे॒वा अ॑कामयन्त। अ॒न॒प॒ज॒य्यं ज॑ये॒मेति॑। त ए॒तं विश्वेभ्यो दे॒वेभ्यो॑ऽषा॒ढाभ्य॑श्च॒रुन्निर॑वपन्। ततो॒ वै ते॑ऽनपज॒य्यम॑जयन्। अ॒न॒प॒ज॒य्य ह॒ वै ज॑यति। य ए॒तेन॑ ह॒विषा॒ यज॑ते। य उ॑ चैनदे॒वं वेद॑। सोऽत्र॑ जुहोति। विश्वेभ्यो दे॒वेभ्य॒ स्वाहा॑ऽषा॒ढाभ्य॒ स्वाहा। अ॒न॒प॒ज॒य्याय॒ स्वाहा॒ जित्यै॒ स्वाहेति॑॥४६॥

%3.1.5.6
ब्रह्म॒ वा अ॑कामयत। ब्र॒ह्म॒लो॒कम॒भिज॑येय॒मिति॑। तदे॒तं ब्रह्म॑णेऽभि॒जिते॑ च॒रुन्निर॑वपत्। ततो॒ वै तद्ब्र॑ह्मलो॒कम॒भ्य॑जयत्। ब्र॒ह्म॒लो॒क ह॒ वा अ॒भिज॑यति। य ए॒तेन॑ ह॒विषा॒ यज॑ते। य उ॑ चैनदे॒वं वेद॑। सोऽत्र॑ जुहोति। ब्रह्म॑णे॒ स्वाहा॑ऽभि॒जिते॒ स्वाहा। ब्र॒ह्म॒लो॒काय॒ स्वाहा॒ऽभिजि॑त्यै॒ स्वाहेति॑॥४७॥

%3.1.5.7
विष्णु॒र्वा अ॑कामयत। पुण्य॒ श्लोक शृण्वीय। न मा॑ पा॒पी की॒र्तिराग॑च्छे॒दिति॑। स ए॒तं विष्ण॑वे श्रो॒णायै॑ पुरो॒डाश॑न्त्रिकपा॒लन्निर॑वपत्। ततो॒ वै स पुण्य॒ श्लोक॑मशृणुत। नैनं॑ पा॒पी की॒र्तिराग॑च्छत्। पुण्य ह॒ वै श्लोक शृणुते। नैनं॑ पा॒पी की॒र्तिराग॑च्छति। य ए॒तेन॑ ह॒विषा॒ यज॑ते। य उ॑ चैनदे॒वं वेद॑। सोऽत्र॑ जुहोति। विष्ण॑वे॒ स्वाहा श्रो॒णायै॒ स्वाहा। श्लोका॑य॒ स्वाहा श्रु॒ताय॒ स्वाहेति॑॥४८॥

%3.1.5.8
वस॑वो॒ वा अ॑कामयन्त। अग्रं॑ दे॒वता॑नां॒ परी॑या॒मेति॑। त ए॒तं वसु॑भ्य॒ श्रवि॑ष्ठाभ्यः पुरो॒डाश॑म॒ष्टाक॑पालं॒ निर॑वपन्। ततो॒ वै तेऽग्रं॑ दे॒वता॑नां॒ पर्या॑यन्। अग्र ह॒ वै स॑मा॒नानां॒ पर्ये॑ति। य ए॒तेन॑ ह॒विषा॒ यज॑ते। य उ॑ चैनदे॒वं वेद॑। सोऽत्र॑ जुहोति। वसु॑भ्य॒ स्वाहा॒ श्रवि॑ष्ठाभ्य॒ स्वाहा। अग्रा॑य॒ स्वाहा॒ परीत्यै॒ स्वाहेति॑॥४९॥

%3.1.5.9
इन्द्रो॒ वा अ॑कामयत। दृ॒ढोऽशि॑थिलः स्या॒मिति॑। स ए॒तं वरु॑णाय श॒तभि॑षजे भेष॒जेभ्य॑ पुरो॒डाशं॒ दश॑कपालं॒ निर॑वपत्कृ॒ष्णानां व्रीही॒णाम्। ततो॒ वै स दृ॒ढोऽशि॑थिलोऽभवत्। दृ॒ढो ह॒ वा अशि॑थिलो भवति। य ए॒तेन॑ ह॒विषा॒ यज॑ते। य उ॑ चैनदे॒वं वेद॑। सोऽत्र॑ जुहोति। वरु॑णाय॒ स्वाहा॑ श॒तभि॑षजे॒ स्वाहा। भे॒ष॒जेभ्य॒ स्वाहेति॑॥५०॥

%3.1.5.10
अ॒जो वा एक॑पादकामयत। ते॒ज॒स्वी ब्र॑ह्मवर्च॒सी स्या॒मिति॑। स ए॒तम॒जायैक॑पदे प्रोष्ठप॒देभ्य॑श्च॒रुन्निर॑वपत्। ततो॒ वै स ते॑ज॒स्वी ब्र॑ह्मवर्च॒स्य॑भवत्। ते॒ज॒स्वी ह॒ वै ब्र॑ह्मवर्च॒सी भ॑वति। य ए॒तेन॑ ह॒विषा॒ यजते। य उ॑ चैनदे॒वं वेद॑। सोऽत्र॑ जुहोति। अ॒जायैक॑पदे॒ स्वाहा प्रोष्ठप॒देभ्य॒ स्वाहा। तेज॑से॒ स्वाहा ब्रह्मवर्च॒साय॒ स्वाहेति॑॥५१॥

%3.1.5.11
अहि॒र्वै बु॒ध्नियो॑ऽकामयत। इ॒मां प्र॑ति॒ष्ठां वि॑न्दे॒येति॑। स ए॒तमह॑ये बु॒ध्निया॑य प्रोष्ठप॒देभ्य॑ पुरो॒डाशं॒ भूमि॑कपालं॒ निर॑वत्। ततो॒ वै स इ॒मां प्र॑ति॒ष्ठाम॑विन्दत। इ॒मा ह॒ वै प्र॑ति॒ष्ठां वि॑न्दते। य ए॒तेन॑ ह॒विषा॒ यज॑ते। य उ॑ चैनदे॒वं वेद॑। सोऽत्र॑ जुहोति। अह॑ये बु॒ध्निया॑य॒ स्वाहा प्रोष्ठप॒देभ्य॒ स्वाहा। प्र॒ति॒ष्ठायै॒ स्वाहेति॑॥५२॥

%3.1.5.12
पू॒षा वा अ॑कामयत। प॒शु॒मान्त्स्या॒मिति॑। स ए॒तं पू॒ष्णे रे॒वत्यै॑ च॒रुन्निर॑वपत्। ततो॒ वै स प॑शु॒मान॑भवत्। प॒शु॒मान् ह॒ वै भ॑वति। य ए॒तेन॑ ह॒विषा॒ यज॑ते। य उ॑ चैनदे॒वं वेद॑। सोऽत्र॑ जुहोति। पू॒ष्णे स्वाहा॑ रे॒वत्यै॒ स्वाहा। प॒शुभ्य॒ स्वाहेति॑॥५३॥

%3.1.5.13
अ॒श्विनौ॒ वा अ॑कामयेताम्। श्रो॒त्र॒स्विना॒वब॑धिरौ स्या॒वेति॑। तावे॒तम॒श्विभ्या॑मश्व॒युग्भ्यां पुरो॒डाश॑न्द्विकपा॒लन्निर॑वपताम्। ततो॒ वै तौ श्रोत्र॒स्विना॒वब॑धिरावभवताम्। श्रो॒त्र॒स्वी ह॒ वा अब॑धिरो भवति। य एते॒न॑ ह॒विषा॒ यज॑ते। य उ॑ चैनदे॒वं वेद॑। सोऽत्र॑ जुहोति। अ॒श्विभ्या॒ स्वाहाऽश्व॒युग्भ्या॒ स्वाहा। श्रोत्रा॑य॒ स्वाहा॒ श्रुत्यै॒ स्वाहेति॑॥५४॥

%3.1.5.14
य॒मो वा अ॑कामयत। पि॒तृ॒णा रा॒ज्यम॒भिज॑येय॒मिति॑। स ए॒तं य॒माया॑प॒भर॑णीभ्यश्च॒रुन्निर॑पवत्। ततो॒ वै स पि॑तृ॒णा रा॒ज्यम॒भ्य॑जयत्। स॒मा॒नाना ह॒ वै रा॒ज्यम॒भि ज॑यति। य ए॒तेन॑ ह॒विषा॒ यज॑ते। य उ॑ चैनदे॒वं वेद॑। सोऽत्र॑ जुहोति। य॒माय॒ स्वाहा॑ऽप॒भर॑णीभ्य॒ स्वाहा। रा॒ज्याय॒ स्वाहा॒भिजि॑त्यै॒ स्वाहेति॑॥५५॥

%3.1.5.15
अथै॒तद॑मावा॒स्या॑या॒ आज्यं॒ निर्व॑पति। कामो॒ वा अ॑मावा॒स्या। काम॒ आज्यम्। कामे॑नै॒व काम॒ सम॑र्धयति। क्षि॒प्रमे॑न॒ सकाम॒ उप॑नमति। येन॒ कामे॑न॒ यज॑ते। सोऽत्र॑ जुहोति। अ॒मा॒वा॒स्या॑यै॒ स्वाहा॒ कामा॑य॒ स्वाहाऽऽग॑त्यै॒ स्वाहेति॑॥५६॥\anuvakamend[मि॒त्र इन्द्र॑ प्र॒जाप॑ति॒र्दश॑ द॒शाप॒ एका॑दश॒ विश्वे॒ ब्रह्म॒ दश॑दश॒ विष्णु॒स्त्रयो॑दश॒ वस॑व॒ इन्द्रो॒ऽजोऽहि॒र्वै बु॒ध्निय॑ पू॒षाऽश्विनौ॑ य॒मो दश॑ द॒शाथै॒तद॑मावा॒स्या॑या अ॒ष्टौ पञ्च॑दश]

%3.1.6.1
च॒न्द्रमा॒ वा अ॑कामयत। अ॒होरा॒त्रान॑र्धमा॒सान्मासा॑नृ॒तून्त्सं॑वत्स॒रमा॒प्त्वा। च॒न्द्रम॑स॒ सायु॑ज्य सलो॒कता॑माप्नुया॒मिति॑। स ए॒तञ्च॒न्द्रम॑से प्रती॒दृश्या॑यै पुरो॒डाशं॒ पञ्च॑दशकपालं॒ निर॑वपत्। ततो॒ वै सो॑ऽहोरा॒त्रान॑र्धमा॒सान्मासा॑नृ॒तून्त्सं॑वत्स॒रमा॒प्त्वा। च॒न्द्रम॑स॒ सायु॑ज्य सलो॒कता॑माप्नोत्। अ॒हो॒रा॒त्रान् ह॒ वा अ॑र्धमा॒सान्मासा॑नृ॒तून्त्सं॑वत्स॒रमा॒प्त्वा। च॒न्द्रम॑स॒ सायु॑ज्य सलो॒कता॑माप्नोति। य ए॒तेन॑ ह॒विषा॒ यज॑ते। य उ॑ चैनदे॒वं वेद॑। सोऽत्र॑ जुहोति। च॒न्द्रम॑से॒ स्वाहा प्रती॒दृश्या॑यै॒ स्वाहा। अ॒हो॒रा॒त्रेभ्य॒ स्वाहाऽर्धमा॒सेभ्य॒ स्वाहा। मासेभ्य॒ स्वाह॒र्तुभ्य॒ स्वाहा। सं॒व॒त्स॒राय॒ स्वाहेति॑॥५७॥

%3.1.6.2
अ॒हो॒रा॒त्रे वा अ॑कामयेताम्। अत्य॑होरा॒त्रे मु॑च्येवहि। न ना॑वहोरा॒त्रे आप्नुयाता॒मिति॑। ते ए॒तम॑होरा॒त्राभ्यां च॒रुन्निर॑वपताम्। द्व॒यानाव्व्रीँही॒णाम्। शु॒क्लानां च कृ॒ष्णानां च। स॒वा॒त्योर्दु॒ग्धे। श्वे॒तायै॑ च कृ॒ष्णायै॑ च। ततो॒ वै ते अत्य॑होरा॒त्रे अ॑मुच्येते। नैने॑ अहोरा॒त्रे आप्नुताम्। अति॑ ह॒ वा अ॑होरा॒त्रे मु॑च्यते। नैन॑महोरा॒त्रे आप्नुतः। य ए॒तेन॑ ह॒विषा॒ यज॑ते। य उ॑ चैनदे॒वं वेद॑। सोऽत्र॑ जुहोति। अह्ने॒ स्वाहा॒ रात्रि॑यै॒ स्वाहा। अति॑मुक्त्यै॒ स्वाहेति॑॥५८॥

%3.1.6.3
उ॒षा वा अ॑कामयत। प्रि॒याऽऽदि॒त्यस्य॑ सु॒भगा स्या॒मिति॑। सैतमु॒षसे॑ च॒रुन्निर॑वपत्। ततो॒ वै सा प्रि॒याऽऽदि॒त्यस्य॑ सु॒भगा॑ऽभवत्। प्रि॒यो ह॒ वै स॑मा॒नाना सु॒भगो॑ भवति। य ए॒तेन॑ ह॒विषा॒ यज॑ते। य उ॑ चैनदे॒वं वेद॑। सोऽत्र॑ जुहोति। उ॒षसे॒ स्वाहा॒ व्यु॑ष्ट्यै॒ स्वाहा। व्यू॒षुष्यै॒ स्वाहा व्यु॒च्छन्त्यै॒ स्वाहा। व्यु॑ष्टायै॒ स्वाहेति॑॥५९॥

%3.1.6.4
अथै॒तस्मै॒ नक्ष॑त्राय च॒रुनिर्व॑पति। यथा॒ त्वन्दे॒वाना॒मसि॑। ए॒वम॒हं म॑नु॒ष्या॑णां भूयास॒मिति॑। यथा॑ ह॒ वा ए॒तद्दे॒वानाम्। ए॒व ह॒ वा ए॒ष म॑नु॒ष्या॑णां भवति। य ए॒तेन॑ ह॒विषा॒ यज॑ते। य उ॑ चैनदे॒वं वेद॑। सोऽत्र॑ जुहोति। नक्ष॑त्राय॒ स्वाहो॑देष्य॒ते स्वाहा। उ॒द्य॒ते स्वाहोदि॑ताय॒ स्वाहा। हर॑से॒ स्वाहा॒ भर॑से॒ स्वाहा। भ्राज॑से॒ स्वाहा॒ तेज॑से॒ स्वाहा। तप॑से॒ स्वाहा ब्रह्मवर्च॒साय॒ स्वाहेति॑॥६०॥

%3.1.6.5
सूर्यो॒ वा अ॑कामयत। नक्ष॑त्राणां प्रति॒ष्ठा स्या॒मिति॑। स ए॒त सूर्या॑य॒ नक्ष॑त्रेभ्यश्च॒रुन्निर॑वपत्। ततो॒ वै स नक्ष॑त्राणां प्रति॒ष्ठाऽभ॑वत्। प्र॒ति॒ष्ठा ह॒ वै स॑मा॒नानां भवति। य ए॒तेन॑ ह॒विषा॒ यज॑ते। य उ॑ चैनदे॒वं वेद॑। सोऽत्र॑ जुहोति। सूर्या॑य॒ स्वाहा॒ नक्ष॑त्रेभ्य॒ स्वाहा। प्र॒ति॒ष्ठायै॒ स्वाहेति॑॥६१॥

%3.1.6.6
अथै॒तमदि॑त्यै च॒रुन्निर्व॑पति। इ॒यं वा अदि॑तिः। अ॒स्यामे॒व प्रति॑तिष्ठति। सोऽत्र॑ जुहोति। अदि॑त्यै॒ स्वाहा प्रति॒ष्ठायै॒ स्वाहेति॑॥६२॥

%3.1.6.7
अथै॒तं विष्ण॑वे च॒रुन्निर्व॑पति। य॒ज्ञो वै विष्णु॑। य॒ज्ञ ए॒वान्त॒तः प्रति॑तिष्ठति। सोऽत्र॑ जुहोति। विष्ण॑वे॒ स्वाहा॑ य॒ज्ञाय॒ स्वाहा। प्र॒ति॒ष्ठायै॒ स्वाहेति॑॥६३॥\anuvakamend[च॒न्द्रमा॒ पञ्च॑दशाहोरा॒त्रे स॒प्तद॑शो॒षा एका॑द॒शाथै॒तस्मै॒ नक्ष॑त्राय॒ त्रयो॑दश॒ सूर्यो॒ दशाथै॒तमदि॑त्यै॒ पञ्चाथै॒तं विष्ण॑वे॒ षट्त्स॒प्त (स॒वि॒ताऽऽशू॒नाव्व्रीँ॑ही॒णामिन्द्रो॑ म॒हाव्री॑हीणा॒मिन्द्र॑ कृ॒ष्णानाव्व्रीँही॒णाम॑होरा॒त्रे द्व॒यानाव्व्रीँही॒णाम्। पि॒तर॒ष्षट्क॑पाल सवि॒ता द्वाद॑शकपालमिन्द्रा॒ग्नी एका॑दशकपाल॒मिन्द्र॒ एका॑दशकपाल॒मिन्द्रो॒ दश॑कपालं॒ विष्णु॑स्त्रिकपा॒लमहि॒र्भूमि॑कपालम॒श्विनौ द्विकपा॒लञ्च॒न्द्रमा॒ पञ्च॑दशकपालम॒ग्निस्त्वष्टा॒ वस॑वो॒ऽष्टाक॑पालम॒न्यत्र॑ च॒रुम्। रु॒द्रोऽर्य॒मा पू॒षा प॑शु॒मान्त्स्या॒ सोमो॑ रु॒द्रो बृह॒स्पति॒ पय॑सि वा॒युः पय॒ सोमो॑ वा॒युरि॑न्द्रा॒ग्नी मि॒त्र इन्द्र॒ आपो॒ ब्रह्म॑ य॒मो॑ऽभिजि॑त्यै॒ त्वष्टा प्र॒जाप॑तिः प्र॒जायै॑ पौर्णमा॒स्या अ॑मावा॒स्या॑या॒ अग॑त्यै॒ विश्वे॒ जित्या॑ अ॒श्विनौ॒ श्रुत्यै। ब्रह्म॒ तदे॒तं विष्णु॒ स ए॒तं वा॒युः स ए॒तदाप॒स्ताः। पि॒तरो॒ विश्वे॒ वस॑वोऽकामयन्त॒ मेति॒ त ए॒तन्निर॑वपन्। आपो॑ऽकामयन्त॒ मेति॒ ता ए॒तन्निर॑वपन्। इ॒न्द्रा॒ग्नी अ॒श्विना॑वकामयेतां॒ वेति॒ तावे॒तन्निर॑वपताम्। अ॒हो॒रा॒त्रे वा अ॑कामयेता॒मिति॒ ते ए॒तन्निर॑वपताम्। अ॒न्यत्रा॑कामय॒तेति॒ स ए॒तन्निर॑वपत्। इ॒न्द्रा॒ग्नी श्रैष्ठ्य॒मिन्द्रो॒ ज्यैष्ठ्य॒मिन्द्रो॑ दृ॒ढः। अहि॒ सूर्योऽदि॑त्यै॒ विष्ण॑वे प्रति॒ष्ठायै। सोमो॑ य॒मः स॑मा॒नानाम्। अ॒ग्निर्नो॑ रीरिषद॒न्यत्र॑ रीरिषः ॥ )]




\prashnaend{अ॒ग्निर्न॑ ऋ॒ध्यास्म॒ नवो॑नवो॒ऽग्निर्मि॒त्रश्च॒न्द्रमा॒ष्षट्॥६॥}{अ॒ग्निर्न॒स्तन्नो॑ वा॒युरहि॑र्बु॒ध्निय॑ ऋ॒क्षा वा इ॒यमथै॒तत्पौर्णमा॒स्या अ॒जो वा एक॑पा॒त्सूर्य॒स्त्रिष॑ष्टिः॥६३॥}{अ॒ग्निर्न॑ पातु प्रति॒ष्ठायै॒ स्वाहेति॑॥}{हरि॑ ओम्॥}{इति श्रीकृष्णयजुर्वेदीयतैत्तिरीयब्राह्मणे तृतीयाष्टके प्रथमः प्रपाठकः समाप्तः॥}
\clearpage
