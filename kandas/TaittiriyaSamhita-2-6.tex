\chapt{काण्डम् २}
\sect{षष्ठमः प्रश्नः}\setcounter{anuvakam}{0}
\dnsub{तैत्तिरीयसंहितायां द्वितीयकाण्डे षष्ठमः प्रश्नः}
%2.6.1.1
स॒मिधो॑ यजति वस॒न्तमे॒वर्तू॒नामव॑ रुन्धे॒ तनू॒नपा॑तं यजति ग्री॒ष्ममे॒वाव॑ रुन्ध इ॒डो य॑जति व॒र्॒\mbox{}षा ए॒वाव॑ रुन्धे ब॒र्॒\mbox{}हिर्य॑जति श॒रद॑मे॒वाव॑ रुन्धे स्वाहाका॒रं य॑जति हेम॒न्तमे॒वाव॑ रुन्धे॒ तस्मा॒थ्\-स्वाहा॑कृता॒ हेम॑न्प॒शवो\-ऽव॑ सीदन्ति स॒मिधो॑ यजत्यु॒षस॑ ए॒व दे॒वता॑ना॒मव॑ रुन्धे॒ तनू॒नपा॑तं यजति य॒ज्ञमे॒वाव॑ रुन्ध~-~(१)\ip

%2.6.1.2
इ॒डो य॑जति प॒शूने॒वाव॑ रुन्धे ब॒र्॒\mbox{}हिर्य॑जति प्र॒जामे॒वाव॑ रुन्धे स॒मान॑यत उप॒भृत॒स्तेजो॒ वा आज्यं॑ प्र॒जा ब॒र्॒\mbox{}हिः प्र॒जास्वे॒व तेजो॑ दधाति स्वाहाका॒रं य॑जति॒ वाच॑मे॒वाव॑ रुन्धे॒ दश॒ सं प॑द्यन्ते॒ दशा᳚क्षरा वि॒राड्वि॒राजै॒वान्नाद्य॒मव॑ रुन्धे स॒मिधो॑ यजत्य॒स्मिन्ने॒व लो॒के प्रति॑ तिष्ठति॒ तनू॒नपा॑तं यजति~(२)\ip

%2.6.1.3
य॒ज्ञ ए॒वान्तरि॑क्षे॒ प्रति॑ तिष्ठती॒डो य॑जति प॒शुष्वे॒व प्रति॑ तिष्ठति ब॒र्॒\mbox{}हिर्य॑जति॒ य ए॒व दे॑व॒यानाः॒ पन्था॑न॒स्तेष्वे॒व प्रति॑ तिष्ठति स्वाहाका॒रं य॑जति सुव॒र्ग ए॒व लो॒के प्रति॑ तिष्ठत्ये॒ताव॑न्तो॒ वै दे॑व\-लो॒का\-स्तेष्वे॒व य॑थापू॒र्वं प्रति॑ तिष्ठति देवासु॒रा ए॒षु लो॒केष्व॑स्पर्धन्त॒ ते दे॒वाः प्र॑या॒जैरे॒भ्यो लो॒केभ्यो\-ऽसु॑रा॒न्प्राणु॑दन्त॒ तत्प्र॑या॒जानां᳚~(३)\ip

%2.6.1.4
प्रयाज॒त्वं यस्यै॒वं वि॒दुषः॑ प्रया॒जा इ॒ज्यन्ते॒ प्रैभ्यो लो॒केभ्यो॒ भ्रातृ॑व्यान्नुदते\-ऽभि॒क्रामं॑ जुहोत्य॒भिजि॑त्यै॒ यो वै प्र॑या॒जानां᳚ मिथु॒नं वेद॒ प्र प्र॒जया॑ प॒शुभि॑र्मिथु॒नैर्जा॑यते स॒मिधो॑ ब॒ह्वीरि॑व यजति॒ तनू॒नपा॑त॒मेक॑मिव मिथु॒नं तदि॒डो ब॒ह्वीरि॑व यजति ब॒र्॒\mbox{}हिरेक॑मिव मिथु॒नं तदे॒तद्वै प्र॑या॒जानां᳚ मिथु॒नं य ए॒वं वेद॒ प्र~(४)\ip

%2.6.1.5
प्र॒जया॑ प॒शुभि॑र्मिथु॒नैर्जा॑यते दे॒वानां॒ वा अनि॑ष्टा दे॒वता॒ आस॒\-न्नथा\-सु॑रा य॒ज्ञम॑\-जिघाꣳस॒न्ते दे॒वा गा॑य॒त्रीं व्यौ॑ह॒न् पञ्चा॒क्षरा॑णि प्रा॒ची\-ना॑नि॒ त्रीणि॑ प्रती॒चीना॑नि॒ ततो॒ वर्म॑ य॒ज्ञायाभ॑व॒द्वर्म॒ यज॑मानाय॒ यत्प्र॑या\-जा\-नू\-या॒जा इ॒ज्यन्ते॒ वर्मै॒व तद्य॒ज्ञाय॑ क्रियते॒ वर्म॒ यज॑मानाय॒ भ्रातृ॑व्या\-भिभूत्यै॒ तस्मा॒द्वरू॑थं पु॒रस्ता॒द्वर्\mbox{}षी॑यः प॒श्चाद्ध्रसी॑यो दे॒वा वै पु॒रा रक्षो᳚भ्य॒~-~(५)\ip

%2.6.1.6
इति॑ स्वाहा\-का॒रेण॑ प्रया॒जेषु॑ य॒ज्ञꣳ स॒ꣴ॒स्थाप्य॑मपश्य॒न्तꣴ स्वा॑हा\-का॒रेण॑ प्रया॒जेषु॒ सम॑स्थापय॒न्वि वा ए॒तद्य॒ज्ञं छि॑न्दन्ति॒ यथ्\-स्वा॑हा\-का॒रेण॑ प्रया॒जेषु॑ सꣴस्था॒पय॑न्ति प्रया॒जानि॒ष्ट्वा ह॒वीꣴष्य॒भि घा॑रयति य॒ज्ञस्य॒ सन्त॑त्या॒ अथो॑ ह॒विरे॒वाक॒रथो॑ यथापू॒र्वमुपै॑ति पि॒ता वै प्र॑या॒जाः प्र॒जा\-ऽनू॑या॒जा यत्प्र॑या॒जानि॒ष्ट्वा ह॒वीꣴष्य॑भिघा॒रय॑ति पि॒तैव तत्पु॒त्रेण॒ साधा॑रणं~(६)\ip

%2.6.1.7
कुरुते॒ तस्मा॑दाहु॒र्यश्चै॒वं वेद॒ यश्च॒ न क॒था पु॒त्रस्य॒ केव॑लं क॒था साधा॑रणं पि॒तुरित्यस्क॑न्नमे॒व तद्यत्प्र॑या॒जेष्वि॒ष्टेषु॒ स्कन्द॑ति गाय॒त्र्ये॑व तेन॒ गर्भं॑ धत्ते॒ सा प्र॒जां प॒शून् यज॑मानाय॒ प्र ज॑नयति॥~(७)\ip

{\anuvakamend[{य॒ज॒ति॒ य॒ज्ञमेवा॒व॑\-रुन्धे॒ तनू॒नपा॑तं यजति प्रया॒जाना॑मे॒वं वेद॒ प्र रक्षो᳚भ्यः॒ साधा॑रणं॒ पञ्च॑त्रिꣳशच्च}]}

%2.6.2.1
चक्षु॑षी॒ वा ए॒ते य॒ज्ञस्य॒ यदाज्य॑भागौ॒ यदाज्य॑भागौ॒ यज॑ति॒ चक्षु॑षी ए॒व तद्य॒ज्ञस्य॒ प्रति॑ दधाति पूर्वा॒र्धे जु॑होति॒ तस्मा᳚त्पूर्वा॒र्धे चक्षु॑षी प्र॒बाहु॑ग्जुहोति॒ तस्मा᳚त्प्र॒बाहु॒क्चक्षु॑षी देवलो॒कं वा अ॒ग्निना॒ यज॑मा॒नो\-ऽनु॑ पश्यति पितृलो॒कꣳ सोमे॑नोत्तरा॒र्धे᳚\-ऽग्नये॑ जुहोति दक्षिणा॒र्धे सोमा॑यै॒वमि॑व॒ हीमौ लो॒काव॒नयो᳚र्लो॒कयो॒रनु॑ख्यात्यै॒ राजा॑नौ॒ वा ए॒तौ दे॒वता॑नां॒~(८)\ip

%2.6.2.2
यद॒ग्नी\-षोमा॑वन्त॒रा दे॒वता॑ इज्येते दे॒वता॑नां॒ विधृ॑त्यै॒ तस्मा॒द्राज्ञा॑ मनु॒ष्या॑ विधृ॑ता ब्रह्मवा॒दिनो॑ वदन्ति॒ किं तद्य॒ज्ञे यज॑मानः कुरुते॒ येना॒न्यतो॑दतश्च प॒शून्दा॒धारो॑भ॒यतो॑\-दत॒श्चेत्यृच॑\-म॒नूच्याज्य॑\-भागस्य जुषा॒णेन॑ यजति॒ तेना॒न्यतो॑दतो दाधा॒रर्च॑म॒नूच्य॑ ह॒विष॑ ऋ॒चा य॑जति॒ तेनो॑भ॒यतो॑दतो दाधार मूर्ध॒न्वती॑ पुरोनुवा॒क्या॑ भवति मू॒र्धान॑मे॒वैनꣳ॑ समा॒नानां᳚ करोति~(९)\ip

%2.6.2.3
नि॒युत्व॑त्या यजति॒ भ्रातृ॑व्यस्यै॒व प॒शून्नि यु॑वते के॒शिनꣳ॑ ह दा॒र्भ्यं के॒शी सात्य॑कामिरुवाच स॒प्तप॑दां ते॒ शक्व॑री॒ꣴ॒ श्वो य॒ज्ञे प्र॑यो॒क्तासे॒ यस्यै॑ वी॒र्ये॑ण॒ प्र जा॒तान्भ्रातृ॑व्यान्नु॒दते॒ प्रति॑ जनि॒ष्यमा॑णा॒न्॒ यस्यै॑ वी॒र्ये॑णो॒भयो᳚र्लो॒कयो॒र्ज्योति॑र्ध॒त्ते यस्यै॑ वी॒र्ये॑ण पूर्वा॒र्धेना॑न॒ड्वान्भु॒नक्ति॑ जघना॒र्धेन॑ धे॒नुरिति॑ पु॒रस्ता᳚ल्लक्ष्मा पुरोनुवा॒क्या॑ भवति जा॒ताने॒व भ्रातृ॑व्या॒न्प्र णु॑दत उ॒परि॑ष्टाल्लक्ष्मा~(१०)\ip

%2.6.2.4
या॒ज्या॑ जनि॒ष्यमा॑णाने॒व प्रति॑ नुदते पु॒रस्ता᳚ल्लक्ष्मा पुरोनुवा॒क्या॑ भवत्य॒स्मिन्ने॒व लो॒के ज्योति॑र्धत्त उ॒परि॑ष्टाल्लक्ष्मा या॒ज्या॑मुष्मि॑न्ने॒व लो॒के ज्योति॑र्धत्ते॒ ज्योति॑ष्मन्तावस्मा इ॒मौ लो॒कौ भ॑वतो॒ य ए॒वं वेद॑ पु॒रस्ता᳚ल्लक्ष्मा पुरोनुवा॒क्या॑ भवति॒ तस्मा᳚त्पूर्वा॒र्धेना॑न॒ड्वान्भु॑नक्त्यु॒परि॑ष्टाल्लक्ष्मा या॒ज्या॑ तस्मा᳚ज्जघना॒र्धेन॑ धे॒नुर्य ए॒वं वेद॑ भु॒ङ्क्त ए॑नमे॒तौ वज्र॒ आज्यं॒ वज्र॒ आज्य॑भागौ॒~(११)\ip

%2.6.2.5
वज्रो॑ वषट्का॒रस्त्रि॒वृत॑मे॒व वज्रꣳ॑ स॒म्भृत्य॒ भ्रातृ॑व्याय॒ प्र ह॑र॒त्यछ॑म्बट्कारमप॒गूर्य॒ वष॑ट्करोति॒ स्तृत्यै॑ गाय॒त्री पु॑रोनुवा॒क्या॑ भवति त्रि॒ष्टुग्या॒ज्या᳚ ब्रह्म॑न्ने॒व क्ष॒त्रम॒न्वार॑म्भयति॒ तस्मा᳚द्ब्राह्म॒णो मुख्यो॒ मुख्यो॑ भवति॒ य ए॒वं वेद॒ प्रैवैनं॑ पुरोनुवा॒क्य॑या\-ऽऽ\-ह॒ प्र ण॑यति या॒ज्य॑या ग॒मय॑ति वषट्का॒रेणैवैनं॑ पुरोनुवा॒क्य॑या दत्ते॒ प्र य॑च्छति या॒ज्य॑या॒ प्रति॑~(१२)\ip

%2.6.2.6
वषट्का॒रेण॑ स्थापयति त्रि॒पदा॑ पुरोनुवा॒क्या॑ भवति॒ त्रय॑ इ॒मे लो॒का ए॒ष्वे॑व लो॒केषु॒ प्रति॑ तिष्ठति॒ चतु॑ष्पदा या॒ज्या॑ चतु॑ष्पद ए॒व प॒शूनव॑ रुन्धे द्व्यक्ष॒रो व॑षट्का॒रो द्वि॒पाद्यज॑मानः प॒शुष्वे॒वोपरि॑ष्टा॒त्प्रति॑ तिष्ठति गाय॒त्री पु॑रोनुवा॒क्या॑ भवति त्रि॒ष्टुग्या॒ज्यै॑षा वै स॒प्तप॑दा॒ शक्व॑री॒ यद्वा ए॒तया॑ दे॒वा अशि॑क्ष॒न्तद॑शक्नुव॒न्॒ य ए॒वं वेद॑ श॒क्नोत्ये॒व यच्छिक्ष॑ति॥~(१३)\ip

{\anuvakamend[{दे॒वता॑नाङ्करोत्यु॒परि॑ष्टाल्ल॒क्ष्मा\-ऽऽ\-ज्य॑भागौ॒ प्रति॑ श॒क्नोत्ये॒व द्वे च॑}]}

%2.6.3.1
प्र॒जा\-प॑तिर्दे॒वेभ्यो॑ य॒ज्ञान्व्यादि॑श॒थ्स आ॒त्मन्नाज्य॑\-मधत्त॒ तं दे॒वा अ॑ब्रुवन्ने॒ष वाव य॒ज्ञो यदाज्य॒मप्ये॒व नो\-ऽत्रा॒स्त्विति॒ सो᳚\-ऽब्रवी॒द्यजान्॑ व॒ आज्य॑\-भागा॒वुप॑ स्तृणान॒भि घा॑रया॒निति॒ तस्मा॒द्\-यज॒न्त्याज्य॑\-भागा॒वुप॑ स्तृणन्त्य॒भि घा॑रयन्ति ब्रह्मवा॒दिनो॑ वदन्ति॒ कस्मा᳚थ्स॒त्याद्या॒तया॑\-मान्य॒न्यानि॑ ह॒वीꣴष्यया॑त\-याम॒माज्य॒\-मिति॑ प्राजाप॒त्य-~(१४)\ip

%2.6.3.2
मिति॑ ब्रूया॒दया॑तयामा॒ हि दे॒वानां᳚ प्र॒जा\-प॑ति॒रिति॒ छन्दाꣳ॑सि दे॒वेभ्यो\-ऽपा᳚क्राम॒न्न वो॑\-ऽभा॒गानि॑ ह॒व्यं व॑क्ष्याम॒ इति॒ तेभ्य॑ ए॒तच्च॑तुरव॒त् तम॑धारयन्पुरोनुवा॒क्या॑यै या॒ज्या॑यै दे॒वता॑यै वषट्का॒राय॒ यच्च॑तुरव॒त्तं जु॒होति॒ छन्दाꣴ॑स्ये॒व तत्प्री॑णाति॒ तान्य॑स्य प्री॒तानि॑ दे॒वेभ्यो॑ ह॒व्यं व॑ह॒न्त्यङ्गि॑रसो॒ वा इ॒त उ॑त्त॒माः सु॑व॒र्गं लो॒कमा॑य॒न्तदृष॑यो यज्ञवा॒स्त्व॑भ्य॒वाय॒न्ते॑-~(१५)\ip

%2.6.3.3
ऽपश्यन्पुरो॒डाशं॑ कू॒र्मं भू॒तꣳ सर्प॑न्तं॒ तम॑ब्रुव॒न्निन्द्रा॑य ध्रियस्व॒ बृह॒स्पत॑ये ध्रियस्व॒ विश्वे᳚भ्यो दे॒वेभ्यो᳚ ध्रिय॒स्वेति॒ स नाध्रि॑यत॒ तम॑ब्रुवन्न॒ग्नये᳚ ध्रिय॒स्वेति॒ सो᳚\-ऽग्नये᳚\-ऽध्रियत॒ यदा᳚ग्ने॒यो᳚\-ऽष्टाक॑पालो\-ऽमावा॒स्या॑यां च पौर्णमा॒स्यां चा᳚च्यु॒तो भव॑ति सुव॒र्गस्य॑ लो॒कस्या॒भिजि॑त्यै॒ तम॑ब्रुवन्क॒थाहा᳚स्था॒ इत्यनु॑पाक्तो\-ऽभूव॒मित्य॑ब्रवी॒द्यथाक्षो\-ऽनु॑पाक्तो॒~-~(१६)\ip

%2.6.3.4
ऽवार्च्छ॑त्ये॒वमवा॑र॒मित्यु॒परि॑ष्टाद॒भ्यज्या॒धस्ता॒दुपा॑नक्ति सुव॒र्गस्य॑ लो॒कस्य॒ सम॑ष्ट्यै॒ सर्वा॑णि क॒पाला᳚न्य॒भि प्र॑थयति॒ ताव॑तः पुरो॒\-डाशा॑\-न॒मुष्मिँ॑ल्लो॒के॑\-ऽभि ज॑यति॒ यो विद॑ग्धः॒ स नैर्॑\mbox{}ऋ॒तो यो\-ऽशृ॑तः॒ स रौ॒द्रो यः शृ॒तः स सदे॑व॒स्तस्मा॒दवि॑दहता शृत॒ङ्कृत्यः॑ सदेव॒त्वाय॒ भस्म॑ना॒\-ऽभि वा॑सयति॒ तस्मा᳚न्मा॒ꣳ॒सेनास्थि॑ छ॒न्नं वे॒देना॒भि वा॑सयति॒ तस्मा॒त्~(१७)\ip

%2.6.3.5
केशैः॒ शिर॑श्छ॒न्नं प्रच्यु॑तं॒ वा ए॒तद॒स्माल्लो॒कादग॑तं देवलो॒कं यच्छृ॒तꣳ ह॒विरन॑भि\-घारितमभि॒\-घार्योद्वा॑सयति देव॒त्रैवैन॑द्गमयति॒ यद्येकं॑ क॒पालं॒ नश्ये॒देको॒ मासः॑ संवथ्स॒रस्यान॑वेतः॒ स्यादथ॒ यज॑मानः॒ प्र मी॑येत॒ यद्द्वे नश्ये॑तां॒ द्वौ मासौ॑ संवथ्स॒रस्यान॑वेतौ॒ स्याता॒मथ॒ यज॑मानः॒ प्र मी॑येत स॒ङ्ख्यायोद्वा॑सयति॒ यज॑मानस्य~(१८)\ip

%2.6.3.6
गोपी॒थाय॒ यदि॒ नश्ये॑दाश्वि॒नं द्वि॑कपा॒लं निर्व॑पेद् द्यावा\-पृथि॒व्य॑मेक॑कपालम॒श्विनौ॒ वै दे॒वानां᳚ भि॒षजौ॒ ताभ्या॑मे॒वास्मै॑ भेष॒जं क॑रोति द्यावा\-पृथि॒व्य॑ एक॑कपालो भवत्य॒नयो॒र्वा ए॒तन्न॑श्यति॒ यन्नश्य॑त्य॒नयो॑रे॒वैन॑द्विन्दति॒ प्रति॑ष्ठित्यै॥~(१९)\ip

{\anuvakamend[{प्रा॒जा॒प॒त्यन्ते\-ऽक्षो\-ऽनु॑पाक्तो वे॒देना॒\-ऽभि वा॑सयति॒ तस्मा॒द्यज॑मानस्य॒ द्वात्रिꣳ॑शच्च}]}

%2.6.4.1
दे॒वस्य॑ त्वा सवि॒तुः प्र॑स॒व इति॒ स्फ्यमा द॑त्ते॒ प्रसू᳚त्या अ॒श्विनो᳚र्बा॒हुभ्या॒मित्या॑हा॒श्विनौ॒ हि दे॒वाना॑मध्व॒र्यू आस्तां᳚ पू॒ष्णो हस्ता᳚भ्या॒मित्या॑ह॒ यत्यै॑ श॒तभृ॑ष्टिरसि वानस्प॒त्यो द्वि॑ष॒तो व॒ध इत्या॑ह॒ वज्र॑मे॒व तथ्सꣴ श्य॑ति॒ भ्रातृ॑व्याय प्रहरि॒ष्यन्थ्स्त॑म्बय॒जुर्\mbox{}ह॑रत्ये॒ताव॑ती॒ वै पृ॑थि॒वी याव॑ती॒ वेदि॒स्तस्या॑ ए॒ताव॑त ए॒व भ्रातृ॑व्यं॒ निर्भ॑जति॒~(२०)\ip

%2.6.4.2
तस्मा॒न्नाभा॒गं निर्भ॑जन्ति॒ त्रिर्\mbox{}ह॑रति॒ त्रय॑ इ॒मे लो॒का ए॒भ्य ए॒वैनं॑ लो॒केभ्यो॒ निर्भ॑जति तू॒ष्णीं च॑तु॒र्थꣳ ह॑र॒त्यप॑रिमितादे॒वैनं॒ निर्भ॑ज॒त्युद्ध॑न्ति॒ यदे॒वास्या॑ अमे॒ध्यं तदप॑ ह॒न्त्युद्ध॑न्ति॒ तस्मा॒दोष॑धयः॒ परा॑ भवन्ति॒ मूलं॑ छिनत्ति॒ भ्रातृ॑व्यस्यै॒व मूलं॑ छिनत्ति पितृदेव॒त्याति॑खा॒तेय॑तीं खनति प्र॒जा\-प॑तिना~(२१)\ip

%2.6.4.3
यज्ञमु॒खेन॒ सम्मि॑ता॒मा प्र॑ति॒ष्ठायै॑ खनति॒ यज॑मानमे॒व प्र॑ति॒ष्ठां ग॑मयति दक्षिण॒तो वर्\mbox{}षी॑यसीं करोति देव॒यज॑नस्यै॒व रू॒पम॑कः॒ पुरी॑षवतीं करोति प्र॒जा वै प॒शवः॒ पुरी॑षं प्र॒जयै॒वैनं॑ प॒शुभिः॒ पुरी॑षवन्तं करो॒त्युत्त॑रं परिग्रा॒हं परि॑ गृह्णात्ये॒ताव॑ती॒ वै पृ॑थि॒वी याव॑ती॒ वेदि॒स्तस्या॑ ए॒ताव॑त ए॒व भ्रातृ॑व्यं नि॒र्भज्या॒\-ऽऽ\-त्मन॒ उत्त॑रं परिग्रा॒हं परि॑ गृह्णाति क्रू॒रमि॑व॒ वा~-~(२२)\ip

%2.6.4.4
ए॒तत्क॑रोति॒ यद्वेदिं॑ क॒रोति॒ धा अ॑सि स्व॒धा अ॒सीति॑ योयुप्यते॒ शान्त्यै॒ प्रोक्ष॑णी॒रा सा॑दय॒त्यापो॒ वै र॑क्षो॒घ्नी रक्ष॑सा॒मप॑हत्यै॒ स्फ्यस्य॒ वर्त्म᳚न्थ्सादयति य॒ज्ञस्य॒ सन्त॑त्यै॒ यं द्वि॒ष्यात्तं ध्या॑येच्छु॒चैवैन॑मर्पयति॥~(२३)\ip

{\anuvakamend[{भ॒ज॒ति॒ प्र॒जा\-प॑तिनेव॒ वै त्रय॑स्त्रिꣳशच्च}]}

%2.6.5.1
ब्र॒ह्म॒वा॒दिनो॑ वदन्त्य॒द्भिर्\mbox{}ह॒वीꣳषि॒ प्रौक्षीः॒ केना॒\-ऽप इति॒ ब्रह्म॒णेति॑ ब्रूयाद॒द्भिर्\mbox{}ह्ये॑व ह॒वीꣳषि॑ प्रो॒क्षति॒ ब्रह्म॑णा॒\-ऽप इ॒ध्माब॒र्॒\mbox{}हिः प्रोक्ष॑ति॒ मेध्य॑मे॒वैन॑त्करोति॒ वेदिं॒ प्रोक्ष॑त्यृ॒क्षा वा ए॒षा\-ऽलो॒मका॑\-ऽमे॒ध्या यद्वेदि॒र्मेध्या॑मे॒वैनां᳚ करोति दि॒वे त्वा॒\-ऽ\-न्तरि॑क्षाय त्वा पृथि॒व्यै त्वेति॑ ब॒र्॒\mbox{}हिरा॒साद्य॒ प्रो-~(२४)\ip

%2.6.5.2
क्ष॑त्ये॒भ्य ए॒वैन॑ल्लो॒केभ्यः॒ प्रोक्ष॑ति क्रू॒रमि॑व॒ वा ए॒तत्क॑रोति॒ यत्खन॑त्य॒पो नि न॑यति॒ शान्त्यै॑ पु॒रस्ता᳚त्प्रस्त॒रं गृ॑ह्णाति॒ मुख्य॑मे॒वैनं॑ करो॒तीय॑न्तं गृह्णाति प्र॒जा\-प॑तिना यज्ञमु॒खेन॒ सम्मि॑तं ब॒र्॒\mbox{}हिः स्तृ॑णाति प्र॒जा वै ब॒र्॒\mbox{}हिः पृ॑थि॒वी वेदिः॑ प्र॒जा ए॒व पृ॑थि॒व्यां प्रति॑\-ष्ठा\-पय॒त्यन॑ति\-दृश्ञꣴ स्तृणाति प्र॒जयै॒वैनं॑ प॒शुभि॒रन॑तिदृश्ञं करो॒-~(२५)\ip

%2.6.5.3
त्युत्त॑रं ब॒र्॒\mbox{}हिषः॑ प्रस्त॒रꣳ सा॑दयति प्र॒जा वै ब॒र्॒\mbox{}हिर्यज॑मानः प्रस्त॒रो यज॑मानमे॒वा\-य॑जमाना॒दुत्त॑रं करोति॒ तस्मा॒द्यज॑मा॒नो\-ऽय॑जमाना॒दुत्त॑रो॒\-ऽन्तर्द॑धाति॒ व्यावृ॑त्त्या अ॒नक्ति॑ ह॒विष्कृ॑तमे॒वैनꣳ॑ सुव॒र्गं लो॒कं ग॑मयति त्रे॒धान॑क्ति॒ त्रय॑ इ॒मे लो॒का ए॒भ्य ए॒वैनं॑ लो॒केभ्यो॑\-ऽनक्ति॒ न प्रति॑ शृणाति॒ यत्प्र॑तिशृणी॒यादनू᳚र्ध्वं भावुकं॒ यज॑मानस्य स्यादु॒परी॑व॒ प्र ह॑र-~(२६)\ip

%2.6.5.4
त्यु॒परी॑व॒ हि सु॑व॒र्गो लो॒को नि य॑च्छति॒ वृष्टि॑मे॒वास्मै॒ नि य॑च्छति॒ नात्य॑ग्रं॒ प्र ह॑रे॒द्यदत्य॑ग्रं प्र॒हरे॑दत्यासा॒रिण्य॑ध्व॒र्यो\-र्नाशु॑का स्या॒न्न पु॒रस्ता॒त्प्रत्य॑स्ये॒द्यत्पु॒रस्ता᳚त्प्र॒त्यस्ये᳚थ्सुव॒र्गाल्लो॒काद्यज॑\-मानं॒ प्रति॑ नुदे॒त्प्राञ्चं॒ प्र ह॑रति॒ यज॑मानमे॒व सु॑व॒र्गं लो॒कं ग॑मयति॒ न विष्व॑ञ्चं॒ वि यु॑या॒द्यद्विष्व॑ञ्चं वियु॒याथ्~(२७)\ip

%2.6.5.5
स्त्र्य॑स्य जायेतो॒र्ध्वमुद्यौ᳚त्यू॒र्ध्वमि॑व॒ हि पु॒ꣳ॒सः पुमा॑ने॒वास्य॑ जायते॒ यथ्स्फ्येन॑ वोपवे॒षेण॑ वा योयु॒प्येत॒ स्तृति॑रे॒वास्य॒ सा हस्ते॑न योयुप्यते॒ यज॑मानस्य गोपी॒थाय॑ ब्रह्मवा॒दिनो॑ वदन्ति॒ किं य॒ज्ञस्य॒ यज॑मान॒ इति॑ प्रस्त॒र इति॒ तस्य॒ क्व॑ सुव॒र्गो लो॒क इत्या॑हव॒नीय॒ इति॑ ब्रूया॒द्यत्प्र॑स्त॒रमा॑हव॒नीये᳚ प्र॒हर॑ति॒ यज॑मानमे॒व~(२८)\ip

%2.6.5.6
सु॑व॒र्गं लो॒कं ग॑मयति॒ वि वा ए॒तद्यज॑मानो लिशते॒ यत्प्र॑स्त॒रं यो॑यु॒प्यन्ते॑ ब॒र्॒\mbox{}हिरनु॒ प्रह॑रति॒ शान्त्या॑ अनारम्भ॒ण इ॑व॒ वा ए॒तर्\mbox{}ह्य॑ध्व॒र्युः स ई᳚श्व॒रो वे॑प॒नो भवि॑तोर्ध्रु॒वा\-ऽसीती॒माम॒भि मृ॑शती॒यं वै ध्रु॒वा\-ऽस्यामे॒व प्रति॑ तिष्ठति॒ न वे॑प॒नो भ॑व॒त्यगा(३)न॑ग्नी॒दित्या॑ह॒ यद्ब्रू॒यादग॑न्न॒ग्निरित्य॒ग्नाव॒ग्निं ग॑मये॒न्निर्यज॑मानꣳ सुव॒र्गाल्लो॒काद्भ॑जे॒द\-ग॒न्नित्ये॒व ब्रू॑या॒द्यज॑मानमे॒व सु॑व॒र्गं लो॒कं ग॑मयति॥~(२९)\ip

{\anuvakamend[{आ॒साद्य॒ प्रान॑तिदृश्ञं करोति हरति वियु॒याद्यज॑मानमे॒वाग्निरिति॑ स॒प्तद॑श च}]}

%2.6.6.1
अ॒ग्नेस्त्रयो॒ ज्यायाꣳ॑सो॒ भ्रात॑र आस॒न्ते दे॒वेभ्यो॑ ह॒व्यं वह॑न्तः॒ प्रामी॑यन्त॒ सो᳚\-ऽग्निर॑बिभेदि॒त्थं वाव स्य आर्ति॒मारि॑ष्य॒तीति॒ स निला॑यत॒ सो॑\-ऽपः प्रावि॑श॒त्तं दे॒वताः॒ प्रैष॑मैच्छ॒न्तं मथ्स्यः॒ प्राब्र॑वी॒त्तम॑शपद्धि॒याधि॑या त्वा वध्यासु॒र्यो मा॒ प्रावो॑च॒ इति॒ तस्मा॒न्मथ्स्यं॑ धि॒याधि॑या घ्नन्ति श॒प्तो~-~(३०)\ip

%2.6.6.2
हि तमन्व॑विन्द॒न्तम॑ब्रुव॒न्नुप॑ न॒ आ व॑र्तस्व ह॒व्यं नो॑ व॒हेति॒ सो᳚\-ऽब्रवी॒द्वरं॑ वृणै॒ यदे॒व गृ॑ही॒तस्याहु॑तस्य बहिःपरि॒धि स्कन्दा॒त्तन्मे॒ भ्रातृ॑णां भाग॒धेय॑मस॒दिति॒ तस्मा॒द्यद् गृ॑ही॒तस्याहु॑तस्य बहिःपरि॒धि स्कन्द॑ति॒ तेषां॒ तद्भा॑ग॒धेयं॒ ताने॒व तेन॑ प्रीणाति परि॒धीन्परि॑ दधाति॒ रक्ष॑सा॒मप॑हत्यै॒ सꣴ स्प॑र्\mbox{}शयति॒~(३१)\ip

%2.6.6.3
रक्ष॑सा॒मन॑न्ववचाराय॒ न पु॒रस्ता॒त्परि॑ दधात्यादि॒त्यो ह्ये॑वोद्यन्पु॒रस्ता॒द्\-रक्षाꣴ॑स्यप॒हन्त्यू॒र्ध्वे स॒मिधा॒वा द॑धात्यु॒परि॑ष्टादे॒व रक्षा॒ꣴ॒स्यप॑ हन्ति॒ यजु॑षा॒\-ऽन्यां तू॒ष्णीम॒न्यां मि॑थुन॒त्वाय॒ द्वे आ द॑धाति द्वि॒पाद्यज॑मानः॒ प्रति॑ष्ठित्यै ब्रह्मवा॒दिनो॑ वदन्ति॒ स त्वै य॑जेत॒ यो य॒ज्ञस्या\-ऽऽ\-र्त्या॒ वसी॑या॒न्थ्स्यादिति॒ भूप॑तये॒ स्वाहा॒ भुव॑नपतये॒ स्वाहा॑ भू॒तानां॒~(३२)\ip

%2.6.6.4
पत॑ये॒ स्वाहेति॑ स्क॒न्नमनु॑ मन्त्रयेत य॒ज्ञस्यै॒व तदार्त्या॒ यज॑मानो॒ वसी॑यान्भवति॒ भूय॑सी॒र्॒\mbox{}हि दे॒वताः᳚ प्री॒णाति॑ जा॒मि वा ए॒तद्य॒ज्ञस्य॑ क्रियते॒ यद॒न्वञ्चौ॑ पुरो॒डाशा॑वुपाꣳशुया॒जम॑\-न्त॒रा य॑ज॒त्यजा॑मित्वा॒याथो॑ मिथुन॒त्वाया॒ग्निर॒मुष्मिँ॑ल्लो॒क आसी᳚द्य॒मो᳚\-ऽस्मिन्ते दे॒वा अ॑ब्रुव॒न्नेते॒मौ वि पर्यू॑हा॒मेत्य॒न्नाद्ये॑न दे॒वा अ॒ग्नि-~(३३)\ip

%2.6.6.5
मु॒पाम॑न्त्रयन्त रा॒ज्येन॑ पि॒तरो॑ य॒मं तस्मा॑द॒ग्निर्दे॒वाना॑मन्ना॒दो य॒मः पि॑तृ॒णाꣳ राजा॒ य ए॒वं वेद॒ प्र रा॒ज्यम॒न्नाद्य॑माप्नोति॒ तस्मा॑ ए॒तद्भा॑ग॒धेयं॒ प्राय॑च्छ॒न्॒ यद॒ग्नये᳚ स्विष्ट॒कृते॑\-ऽव॒द्यन्ति॒ यद॒ग्नये᳚ स्विष्ट॒कृते॑\-ऽव॒द्यति॑ भाग॒धेये॑नै॒व तद्रु॒द्रꣳ सम॑र्धयति स॒कृथ्स॑कृ॒दव॑ द्यति स॒कृदि॑व॒ हि रु॒द्र उ॑त्तरा॒र्धादव॑ द्यत्ये॒षा वै रु॒द्रस्य॒~(३४)\ip

%2.6.6.6
दिख्स्वाया॑मे॒व दि॒शि रु॒द्रं नि॒रव॑दयते॒ द्विर॒भि घा॑रयति चतुरव॒त्तस्या\-ऽऽ\-प्त्यै॑ प॒शवो॒ वै पूर्वा॒ आहु॑तय ए॒ष रु॒द्रो यद॒ग्निर्यत्पूर्वा॒ आहु॑तीर॒भि जु॑हु॒याद्रु॒द्राय॑ प॒शूनपि॑ दध्यादप॒शुर्यज॑मानः स्यादति॒हाय॒ पूर्वा॒ आहु॑तीर्जुहोति पशू॒नां गो॑पी॒थाय॑॥~(३५)\ip

{\anuvakamend[{श॒प्तः स्प॑र्\mbox{}शयति भू॒ताना॑म॒ग्निꣳ रु॒द्रस्य॑ स॒प्तत्रिꣳ॑शच्च}]}

%2.6.7.1
मनुः॑ पृथि॒व्या य॒ज्ञिय॑मैच्छ॒थ्स घृ॒तं निषि॑क्तमविन्द॒थ्सो᳚\-ऽब्रवी॒त्को᳚\-ऽस्येश्व॒रो य॒ज्ञे\-ऽपि॒ कर्तो॒रिति॒ ताव॑ब्रूतां मि॒त्रावरु॑णौ॒ गोरे॒वावमी᳚श्व॒रौ कर्तोः᳚ स्व॒ इति॒ तौ ततो॒ गाꣳ समै॑रयता॒ꣳ॒ सा यत्र॑यत्र॒ न्यक्रा॑म॒त्ततो॑ घृ॒तम॑पीड्यत॒ तस्मा᳚द् घृ॒तप॑द्युच्यते॒ तद॑स्यै॒ जन्मोप॑हूतꣳ रथन्त॒रꣳ स॒ह पृ॑थि॒व्येत्या॑हे॒-~(३६)\ip

%2.6.7.2
यं वै र॑थन्त॒रमि॒मामे॒व स॒हान्नाद्ये॒नोप॑ ह्वयत॒ उप॑हूतं वामदे॒व्यꣳ स॒हान्तरि॑क्षे॒णेत्या॑ह प॒शवो॒ वै वा॑मदे॒व्यं प॒शूने॒व स॒हान्तरि॑क्षे॒णोप॑ ह्वयत॒ उप॑हूतं बृ॒हथ्स॒ह दि॒वेत्या॑है॒रं वै बृ॒हदिरा॑मे॒व स॒ह दि॒वोप॑ ह्वयत॒ उप॑हूताः स॒प्त होत्रा॒ इत्या॑ह॒ होत्रा॑ ए॒वोप॑ ह्वयत॒ उप॑हूता धे॒नुः~(३७)\ip

%2.6.7.3
स॒हर्\mbox{}ष॒भेत्या॑ह मिथु॒नमे॒वोप॑ ह्वयत॒ उप॑हूतो भ॒क्षः सखेत्या॑ह सोमपी॒थमे॒वोप॑ ह्वयत॒ उप॑हू॒ताँ~(४) हो इत्या॑हा॒\-ऽ॒\-ऽ॒त्मान॑मे॒वोप॑ ह्वयत आ॒त्मा ह्युप॑हूतानां॒ वसि॑ष्ठ॒ इडा॒मुप॑ ह्वयते प॒शवो॒ वा इडा॑ प॒शूने॒वोप॑ ह्वयते च॒तुरुप॑ ह्वयते॒ चतु॑ष्पादो॒ हि प॒शवो॑ मान॒वीत्या॑ह॒ मनु॒र्॒\mbox{}ह्ये॑ता-~(३८)\ip

%2.6.7.4
मग्रे\-ऽप॑श्यद् घृ॒तप॒दीत्या॑ह॒ यदे॒वास्यै॑ प॒दाद् घृ॒तमपी᳚ड्यत॒ तस्मा॑दे॒वमा॑ह मैत्रावरु॒णीत्या॑ह मि॒त्रावरु॑णौ॒ ह्ये॑नाꣳ स॒मैर॑यतां॒ ब्रह्म॑ दे॒वकृ॑त॒मुप॑हूत॒मित्या॑ह॒ ब्रह्मै॒वोप॑ ह्वयते॒ दैव्या॑ अध्व॒र्यव॒ उप॑हूता॒ उप॑हूता मनु॒ष्या॑ इत्या॑ह देवमनु॒ष्याने॒वोप॑ ह्वयते॒ य इ॒मं य॒ज्ञमवा॒न्॒ ये य॒ज्ञप॑तिं॒ वर्धा॒नित्या॑ह~(३९)\ip

%2.6.7.5
य॒ज्ञाय॑ चै॒व यज॑मानाय चा॒\-ऽऽ\-शिष॒मा शा᳚स्त॒ उप॑हूते॒ द्यावा॑\-पृथि॒वी इत्या॑ह॒ द्यावा॑\-पृथि॒वी ए॒वोप॑ ह्वयते पूर्व॒जे ऋ॒ताव॑री॒ इत्या॑ह पूर्व॒जे ह्ये॑ते ऋ॒ताव॑री दे॒वी दे॒वपु॑त्रे॒ इत्या॑ह दे॒वी ह्ये॑ते दे॒वपु॑त्रे॒ उप॑हूतो॒\-ऽयं यज॑मान॒ इत्या॑ह॒ यज॑मानमे॒वोप॑ ह्वयत॒ उत्त॑रस्यां देवय॒ज्याया॒मुप॑हूतो॒ भूय॑सि हवि॒ष्कर॑ण॒ उप॑हूतो दि॒व्ये धाम॒न्नुप॑हूत॒~-~(४०)\ip

%2.6.7.6
इत्या॑ह प्र॒जा वा उत्त॑रा देवय॒ज्या प॒शवो॒ भूयो॑ हवि॒ष्कर॑णꣳ सुव॒र्गो लो॒को दि॒व्यं धामे॒दम॑सी॒दम॒सीत्ये॒व य॒ज्ञस्य॑ प्रि॒यं धामोप॑ ह्वयते॒ विश्व॑मस्य प्रि॒यमुप॑हूत॒मित्या॒हाछ॑म्बट्कारमे॒वोप॑ ह्वयते॥~(४१)\ip

{\anuvakamend[{आ॒ह॒ धे॒नुरे॒तां वर्धा॒नित्या॑ह॒ धाम॒न्नुप॑हूत॒श्चतु॑स्त्रिꣳशच्च}]}

%2.6.8.1
प॒शवो॒ वा इडा᳚ स्व॒यमा द॑त्ते॒ काम॑मे॒वा\-ऽऽ\-त्मना॑ पशू॒नामा द॑त्ते॒ न ह्य॑न्यः कामं॑ पशू॒नां प्र॒यच्छ॑ति वा॒चस्पत॑ये त्वा हु॒तं प्राश्ञा॒मीत्या॑ह॒ वाच॑मे॒व भा॑ग॒धेये॑न प्रीणाति॒ सद॑स॒स्पत॑ये त्वा हु॒तं प्राश्ञा॒मीत्या॑ह स्व॒गाकृ॑त्यै चतुरव॒त्तं भ॑वति ह॒विर्वै च॑तुरव॒त्तं प॒शव॑श्चतुरव॒त्तं यद्धोता᳚ प्राश्ञी॒याद्धोता-~(४२)\ip

%2.6.8.2
ऽ\-ऽर्ति॒मार्च्छे॒द्यद॒ग्नौ जु॑हु॒याद्रु॒द्राय॑ प॒शूनपि॑ दध्यादप॒शुर्यज॑मानः स्याद्वा॒चस्पत॑ये त्वा हु॒तं प्राश्ञा॒मीत्या॑ह प॒रोक्ष॑मे॒वैन॑ज्जुहोति॒ सद॑स॒स्पत॑ये त्वा हु॒तं प्राश्ञा॒मीत्या॑ह स्व॒गाकृ॑त्यै॒ प्राश्ञ॑न्ति ती॒र्थ ए॒व प्राश्ञ॑न्ति॒ दक्षि॑णां ददाति ती॒र्थ ए॒व दक्षि॑णां ददाति॒ वि वा ए॒तद्य॒ज्ञं~(४३)\ip

%2.6.8.3
छि॑न्दन्ति॒ यन्म॑ध्य॒तः प्रा॒श्ञन्त्य॒द्भिर्मा᳚र्जयन्त॒ आपो॒ वै सर्वा॑ दे॒वता॑ दे॒वता॑भिरे॒व य॒ज्ञꣳ सं त॑न्वन्ति दे॒वा वै य॒ज्ञाद्रु॒द्रम॒न्तरा॑य॒न्थ्स य॒ज्ञम॑विध्य॒त्तं दे॒वा अ॒भि सम॑गच्छन्त॒ कल्प॑तां न इ॒दमिति॒ ते᳚\-ऽब्रुव॒न्थ्स्वि॑ष्टं॒ वै न॑ इ॒दं भ॑विष्यति॒ यदि॒मꣳ रा॑धयि॒ष्याम॒ इति॒ तथ्स्वि॑ष्ट॒कृतः॑ स्विष्टकृ॒त्त्वं तस्या\-ऽऽ\-वि॑द्धं॒ नि-~(४४)\ip

%2.6.8.4
र॑कृन्त॒न्॒ यवे॑न॒ सम्मि॑तं॒ तस्मा᳚द्यवमा॒त्रमव॑ द्ये॒द्यज्ज्यायो॑\-ऽव॒द्येद्रो॒पये॒त्तद्य॒ज्ञस्य॒ यदुप॑ च स्तृणी॒याद॒भि च॑ घा॒रये॑दुभयतः सꣴश्वा॒यि कु॑र्यादव॒दाया॒भि घा॑रयति॒ द्विः सम्प॑द्यते द्वि॒पाद्यज॑मानः॒ प्रति॑ष्ठित्यै॒ यत्ति॑र॒श्चीन॑मति॒हरे॒दन॑भिविद्धं य॒ज्ञस्या॒भि वि॑ध्ये॒दग्रे॑ण॒ परि॑ हरति ती॒र्थेनै॒व परि॑ हरति॒ तत्पू॒ष्णे पर्य॑हर॒न्तत्~(४५)\ip

%2.6.8.5
पू॒षा प्राश्य॑ द॒तो॑\-ऽरुण॒त्तस्मा᳚त्पू॒षा प्र॑पि॒ष्टभा॑गो\-ऽद॒न्तको॒ हि तं दे॒वा अ॑ब्रुव॒न्वि वा अ॒यमा᳚र्ध्यप्राशित्रि॒यो वा अ॒यम॑भू॒दिति॒ तद्बृह॒स्पत॑ये॒ पर्य॑हर॒न्थ्सो॑\-ऽबिभे॒द्बृह॒स्पति॑रि॒त्थं वाव स्य आर्ति॒\-मारि॑\-ष्य॒\-तीति॒ स ए॒तं मन्त्र॑मपश्य॒थ्सूर्य॑स्य त्वा॒ चक्षु॑षा॒ प्रति॑ पश्या॒मीत्य॑ब्रवी॒न्न हि सूर्य॑स्य॒ चक्षुः॒~(४६)\ip

%2.6.8.6
किं च॒न हि॒नस्ति॒ सो॑\-ऽबिभेत्प्रतिगृ॒ह्णन्तं॑ मा हिꣳसिष्य॒तीति॑ दे॒वस्य॑ त्वा सवि॒तुः प्र॑स॒वे᳚\-ऽश्विनो᳚र्बा॒हु\-भ्यां᳚ पू॒ष्णो हस्ता᳚भ्यां॒ प्रति॑ गृह्णा॒मीत्य॑ब्रवीथ्सवि॒तृप्र॑सूत ए॒वैन॒द्ब्रह्म॑णा दे॒वता॑भिः॒ प्रत्य॑\-गृह्णा॒थ्सो॑\-ऽबिभेत्प्रा॒श्ञन्तं॑ मा हिꣳसिष्य॒तीत्य॒ग्नेस्त्वा॒\-ऽऽ\-\-स्ये॑न॒ प्राश्ञा॒मीत्य॑ब्रवी॒न्न ह्य॑ग्नेरा॒स्यं॑ किं च॒न हि॒नस्ति॒ सो॑\-ऽबिभे॒त्~(४७)\ip

%2.6.8.7
प्राशि॑तं मा हिꣳसिष्य॒तीति॑ ब्राह्म॒णस्यो॒दरे॒णेत्य॑ब्रवी॒न्न हि ब्रा᳚ह्म॒णस्यो॒दरं॒ किं च॒न हि॒नस्ति॒ बृह॒स्पते॒र्ब्रह्म॒णेति॒ स हि ब्रह्मि॒ष्ठो\-ऽप॒ वा ए॒तस्मा᳚त्प्रा॒णाः क्रा॑मन्ति॒ यः प्रा॑शि॒त्रं प्रा॒श्ञात्य॒द्भिर्मा᳚र्जयि॒त्वा प्रा॒णान्थ्सम्मृ॑शते॒\-ऽमृतं॒ वै प्रा॒णा अ॒मृत॒मापः॑ प्रा॒णाने॒व य॑थास्था॒नमुप॑ ह्वयते॥~(४८)\ip

{\anuvakamend[{प्रा॒श्ञी॒याद्धोता॑ य॒ज्ञं निर॑हर॒न्तच्चक्षु॑रा॒स्य॑ङ्किं च॒न हि॒नस्ति॒ सो॑\-ऽबिभे॒च्चतु॑श्चत्वारिꣳशच्च}]}

%2.6.9.1
अ॒ग्नीध॒ आ द॑धात्य॒ग्निमु॑खाने॒वर्तून्प्री॑णाति स॒मिध॒मा द॑धा॒त्युत्त॑रासा॒माहु॑तीनां॒ प्रति॑ष्ठित्या॒ अथो॑ स॒मिद्व॑त्ये॒व जु॑होति परि॒धीन्थ्सम्मा᳚र्ष्टि पु॒नात्ये॒वैना᳚न्थ्स॒कृथ्स॑कृ॒थ्सम्मा᳚र्ष्टि॒ परा॑ङिव॒ ह्ये॑तर्\mbox{}हि॑ य॒ज्ञश्च॒तुः सम्प॑द्यते॒ चतु॑ष्पादः प॒शवः॑ प॒शूने॒वाव॑ रुन्धे॒ ब्रह्म॒न्प्र स्था᳚स्याम॒ इत्या॒हात्र॒ वा ए॒तर्\mbox{}हि॑ य॒ज्ञः श्रि॒तो~-~(४९)\ip

%2.6.9.2
यत्र॑ ब्र॒ह्मा यत्रै॒व य॒ज्ञः श्रि॒तस्तत॑ ए॒वैन॒मा र॑भते॒ यद्धस्ते॑न प्र॒मीवे᳚द्वेप॒नः स्या॒द्यच्छी॒र्ष्णा शी॑र्\mbox{}षक्ति॒मान्थ्स्या॒द्यत्तू॒ष्णीमासी॒तासं॑ प्रत्तो य॒ज्ञः स्या॒त्प्र ति॒ष्ठेत्ये॒व ब्रू॑याद्वा॒चि वै य॒ज्ञः श्रि॒तो यत्रै॒व य॒ज्ञः श्रि॒तस्तत॑ ए॒वैन॒ꣳ॒ सम्प्र य॑च्छति॒ देव॑ सवितरे॒तत्ते॒ प्रा-~(५०)\ip

%2.6.9.3
ऽ\-ऽहेत्या॑ह॒ प्रसू᳚त्यै॒ बृह॒स्पति॑र्ब्र॒ह्मेत्या॑ह॒ स हि ब्रह्मि॑ष्ठः॒ स य॒ज्ञं पा॑हि॒ स य॒ज्ञप॑तिं पाहि॒ स मां पा॒हीत्या॑ह य॒ज्ञाय॒ यज॑मानाया॒\-ऽऽ\-त्मने॒ तेभ्य॑ ए॒वा\-ऽऽ\-शिष॒मा शा॒स्ते\-ऽना᳚र्त्या आ॒श्राव्या॑\-ऽ\-ऽ\-ह दे॒वान् य॒जेति॑ ब्रह्मवा॒दिनो॑ वदन्ती॒ष्टा दे॒वता॒ अथ॑ कत॒म ए॒ते दे॒वा इति॒ छन्दा॒ꣳ॒सीति॑ ब्रूयाद्गाय॒त्रीं त्रि॒ष्टुभं॒~(५१)\ip

%2.6.9.4
जग॑ती॒मित्यथो॒ खल्वा॑हुर्ब्राह्म॒णा वै छन्दा॒ꣳ॒सीति॒ ताने॒व तद्य॑जति दे॒वानां॒ वा इ॒ष्टा दे॒वता॒ आस॒न्नथा॒ग्निर्नोद॑ज्वल॒त्तं दे॒वा आहु॑तीभिरनूया॒जेष्वन्व॑विन्द॒न्॒ यद॑नूया॒जान् यज॑त्य॒ग्निमे॒व तथ्समि॑न्द्ध ए॒तदु॒र्वै नामा॑सु॒र आ॑सी॒थ्स ए॒तर्\mbox{}हि॑ य॒ज्ञस्या॒\-ऽऽ\-शिष॑मवृङ्क्त॒ यद्ब्रू॒यादे॒त-~(५२)\ip

%2.6.9.5
दु॑ द्यावापृथिवी भ॒द्रम॑भू॒दित्ये॒तदु॑मे॒वासु॒रं य॒ज्ञस्या॒\-ऽऽ\-शिषं॑ गमयेदि॒दं द्या॑वापृथिवी भ॒द्रम॑भू॒दित्ये॒व ब्रू॑या॒द्यज॑मानमे॒व य॒ज्ञस्या॒\-ऽऽ\-शिषं॑ गमय॒त्यार्ध्म॑ सूक्तवा॒कमु॒त न॑मोवा॒कमित्या॑\-हे॒दम॑रा॒थ्स्मेति॒ वावैतदा॒होप॑श्रितो दि॒वः पृ॑थि॒व्योरित्या॑ह॒ द्यावा॑पृथि॒व्योर्\mbox{}हि य॒ज्ञ उप॑श्रित॒ ओम॑न्वती ते॒\-ऽस्मिन् य॒ज्ञे य॑जमान॒ द्यावा॑\-पृथि॒वी~(५३)\ip

%2.6.9.6
स्ता॒मित्या॑हा॒\-ऽऽ\-शिष॑मे॒वैतामा शा᳚स्ते॒ यद्ब्रू॒याथ्सू॑पावसा॒ना च॑ स्वध्यवसा॒ना चेति॑ प्र॒मायु॑को॒ यज॑मानः स्याद्य॒दा हि प्र॒मीय॒ते\-ऽथे॒मामु॑पाव॒स्यति॑ सूपचर॒णा च॑ स्वधिचर॒णा चेत्ये॒व ब्रू॑या॒द्वरी॑यसीमे॒वास्मै॒ गव्यू॑ति॒मा शा᳚स्ते॒ न प्र॒मायु॑को भवति॒ तयो॑रा॒विद्य॒ग्निरि॒दꣳ ह॒विर॑जुष॒तेत्या॑ह॒ या अया᳚क्ष्म~(५४)\ip

%2.6.9.7
दे॒वता॒स्ता अ॑रीरधा॒मेति॒ वावैतदा॑ह॒ यन्न नि॑र्दि॒शेत्प्रति॑वेशं य॒ज्ञस्या॒\-ऽऽ\-शीर्ग॑च्छे॒दा शा᳚स्ते॒\-ऽयं यज॑मानो॒\-ऽसावित्या॑ह नि॒र्दिश्यै॒वैनꣳ॑ सुव॒र्गं लो॒कं ग॑मय॒त्यायु॒रा शा᳚स्ते सुप्रजा॒स्त्वमा शा᳚स्त॒ इत्या॑हा॒\-ऽऽ\-शिष॑मे॒वैतामा शा᳚स्ते सजातवन॒स्यामा शा᳚स्त॒ इत्या॑ह प्रा॒णा वै स॑जा॒ताः प्रा॒णाने॒व~(५५)\ip

%2.6.9.8
नान्तरे॑ति॒ तद॒ग्निर्दे॒वो दे॒वेभ्यो॒ वन॑ते व॒यम॒ग्नेर्मानु॑षा॒ इत्या॑हा॒ग्निर्दे॒वेभ्यो॑ वनु॒ते व॒यं म॑नु॒ष्ये᳚भ्य॒ इति॒ वावैतदा॑हे॒ह गति॑र्वा॒मस्ये॒दं च॒ नमो॑ दे॒वेभ्य॒ इत्या॑ह॒ याश्चै॒व दे॒वता॒ यज॑ति॒ याश्च॒ न ताभ्य॑ ए॒वोभयी᳚भ्यो॒ नम॑स्करोत्या॒त्मनो\-ऽना᳚र्त्यै॥~(५६)\ip

{\anuvakamend[{श्रि॒तस्ते॒ प्र त्रि॒ष्टुभ॑मे॒तद्द्यावा॑\-पृथि॒वी या अया᳚क्ष्म प्रा॒णाने॒व षट्च॑त्वारिꣳशच्च}]}

%2.6.10.1
दे॒वा वै य॒ज्ञस्य॑ स्वगाक॒र्तारं॒ नावि॑न्द॒न्ते शं॒युं बा॑र्\mbox{}हस्प॒त्य\-म॑ब्रुवन्नि॒मं नो॑ य॒ज्ञꣴ स्व॒गा कु॒र्विति॒ सो᳚\-ऽब्रवी॒द्वरं॑ वृणै॒ यदे॒वा\-ब्रा᳚ह्मणो॒क्तो\-ऽश्र॑द्दधानो॒ यजा॑तै॒ सा मे॑ य॒ज्ञस्या॒\-ऽऽ\-शीर॑स॒दिति॒ तस्मा॒द्यदब्रा᳚ह्मणो॒क्तो\-ऽश्र॑द्दधानो॒ यज॑ते शं॒युमे॒व तस्य॑ बार्\mbox{}हस्प॒त्यं य॒ज्ञस्या॒\-ऽऽ\-शीर्ग॑च्छ\-त्ये॒तन्ममेत्य॑ब्रवी॒त्किं मे᳚ प्र॒जाया॒~-~(५७)\ip

%2.6.10.2
इति॒ यो॑\-ऽपगु॒रातै॑ श॒तेन॑ यातया॒द्यो नि॒हन॑थ्स॒हस्रे॑ण यातया॒द्यो लोहि॑तं क॒रव॒द्याव॑तः प्र॒स्कद्य॑ पा॒ꣳ॒सून्थ्स॑ङ्गृ॒ह्णात् ताव॑तः संवथ्स॒रान्पि॑तृलो॒कं न प्र जा॑ना॒दिति॒ तस्मा᳚द्ब्राह्म॒णाय॒ नाप॑ गुरेत॒ न नि ह॑न्या॒न्न लोहि॑तं कुर्यादे॒ताव॑ता॒ हैन॑सा भवति॒ तच्छं॒ योरा वृ॑णीमह॒ इत्या॑ह य॒ज्ञमे॒व तथ्स्व॒गा क॑रोति॒ त-~(५८)\ip

%2.6.10.3
च्छं॒ योरा वृ॑णीमह॒ इत्या॑ह शं॒युमे॒व बा॑र्\mbox{}हस्प॒त्यं भा॑ग॒धेये॑न॒ सम॑र्धयति गा॒तुं य॒ज्ञाय॑ गा॒तुं य॒ज्ञप॑तय॒ इत्या॑हा॒\-ऽऽ\-शिष॑मे॒वैतामा शा᳚स्ते॒ सोमं॑ यजति॒ रेत॑ ए॒व तद्द॑धाति॒ त्वष्टा॑रं यजति॒ रेत॑ ए॒व हि॒तं त्वष्टा॑ रू॒पाणि॒ वि क॑रोति दे॒वानां॒ पत्नी᳚र्यजति मिथुन॒त्वाया॒ग्निं गृ॒हप॑तिं यजति॒ प्रति॑ष्ठित्यै जा॒मि वा ए॒तद्य॒ज्ञस्य॑ क्रियते॒~(५९)\ip

%2.6.10.4
यदाज्ये॑न प्रया॒जा इ॒ज्यन्त॒ आज्ये॑न पत्नीसंया॒जा ऋच॑म॒नूच्य॑ पत्नीसंया॒जाना॑मृ॒चा य॑ज॒त्यजा॑मित्वा॒याथो॑ मिथुन॒त्वाय॑ प॒ङ्क्तिप्रा॑यणो॒ वै य॒ज्ञः प॒ङ्क्त्यु॑दयनः॒ पञ्च॑ प्रया॒जा इ॑ज्यन्ते च॒त्वारः॑ पत्नीसंया॒जाः स॑मिष्टय॒जुः प॑ञ्च॒मं प॒ङ्क्तिमे॒वानु॑ प्र॒ यन्ति॑ प॒ङ्क्तिमनूद्य॑न्ति॥~(६०)\ip

{\anuvakamend[{प्र॒जायाः᳚ करोति॒ तत्क्रि॑यते॒ त्रय॑स्त्रिꣳशच्च}]}%॥10॥

%2.6.11.1
यु॒क्ष्वा हि दे॑व॒हूत॑मा॒ꣳ॒ अश्वाꣳ॑ अग्ने र॒थीरि॑व। नि होता॑ पू॒र्व्यः स॑दः। उ॒त नो॑ देव दे॒वाꣳ अच्छा॑ वोचो वि॒दुष्ट॑रः। श्रद्विश्वा॒ वार्या॑ कृधि। त्वꣳ ह॒ यद्य॑विष्ठ्य॒ सह॑सः सूनवाहुत। ऋ॒तावा॑ य॒ज्ञियो॒ भुवः॑। अ॒यम॒ग्निः स॑ह॒स्रिणो॒ वाज॑स्य श॒तिन॒स्पतिः॑। मू॒र्धा क॒वी र॑यी॒णाम्। तं ने॒मिमृ॒भवो॑ य॒था न॑मस्व॒ सहू॑तिभिः। नेदी॑यो य॒ज्ञ-~(६१)\ip

%2.6.11.2
म॑ङ्गिरः। तस्मै॑ नू॒नम॒भिद्य॑वे वा॒चा वि॑रूप॒ नित्य॑या। वृष्णे॑ चोदस्व सुष्टु॒तिम्। कमु॑ ष्विदस्य॒ सेन॑या॒\-ऽग्नेरपा॑कचक्षसः। प॒णिं गोषु॑ स्तरामहे। मा नो॑ दे॒वानां॒ विशः॑ प्रस्ना॒तीरि॑वो॒स्राः। कृ॒शं न हा॑सु॒रघ्नि॑याः। मा नः॑ समस्य दू॒ढ्यः॑ परि॑द्वेषसो अꣳह॒तिः। ऊ॒र्मिर्न नाव॒मा व॑धीत्। नम॑स्ते अग्न॒ ओज॑से गृ॒णन्ति॑ देव कृ॒ष्टयः॑। अमै॑-~(६२)\ip

%2.6.11.3
र॒मित्र॑मर्दय। कु॒विथ्सु नो॒ गवि॑ष्ट॒ये\-ऽग्ने॑ सं॒वेषि॑षो र॒यिम्। उरु॑कृदु॒रु ण॑स्कृधि। मा नो॑ अ॒स्मिन्म॑हाध॒ने परा॑ वर्ग्भार॒भृद्य॑था। सं॒वर्ग॒ꣳ॒ सꣳ र॒यिं ज॑य। अ॒न्यम॒स्मद्भि॒या इ॒यमग्ने॒ सिष॑क्तु दु॒च्छुना᳚। वर्धा॑ नो॒ अम॑व॒च्छवः॑। यस्याजु॑षन्नम॒स्विनः॒ शमी॒मदु॑र्मखस्य वा। तं घेद॒ग्निर्वृ॒धाव॑ति। पर॑स्या॒ अधि॑~(६३)\ip

%2.6.11.4
सं॒वतो\-ऽव॑राꣳ अ॒भ्या त॑र। यत्रा॒हमस्मि॒ ताꣳ अ॑व। वि॒द्मा हि ते॑ पु॒रा व॒यमग्ने॑ पि॒तुर्यथाऽव॑सः। अधा॑ ते सु॒म्नमी॑महे। य उ॒ग्र इ॑व शर्य॒हा ति॒ग्मशृ॑ङ्गो॒ न वꣳस॑गः। अग्ने॒ पुरो॑ रु॒रोजि॑थ। सखा॑यः॒ सं वः॑ स॒म्यञ्च॒मिष॒ꣴ॒ स्तोमं॑ चा॒ग्नये᳚। वर्\mbox{}षि॑ष्ठाय क्षिती॒नामू॒र्जो नप्त्रे॒ सह॑\-स्वते। सꣳस॒मिद्यु॑वसे वृष॒न्नग्ने॒ विश्वा᳚न्य॒र्य आ। इ॒डस्प॒दे समि॑ध्यसे॒ स नो॒ वसू॒न्या भ॑र। प्रजा॑पते॒ स वे॑द॒ सोमा॑पूषणे॒मौ दे॒वौ॥~(६४)\ip

{\anuvakamend[{य॒ज्ञममै॒रधि॑ वृष॒न्नेका॒न्नविꣳ॑श॒तिश्च॑}]}%॥11॥

%2.6.12.1
उ॒शन्त॑स्त्वा हवामह उ॒शन्तः॒ समि॑धीमहि। उ॒शन्नु॑श॒त आ व॑ह पि॒तॄन् ह॒विषे॒ अत्त॑वे। त्वꣳ सो॑म॒ प्रचि॑कितो मनी॒षा त्वꣳ रजि॑ष्ठ॒मनु॑ नेषि॒ पन्था᳚म्। तव॒ प्रणी॑ती पि॒तरो॑ न इन्दो दे॒वेषु॒ रत्न॑मभजन्त॒ धीराः᳚। त्वया॒ हि नः॑ पि॒तरः॑ सोम॒ पूर्वे॒ कर्मा॑णि च॒क्रुः प॑वमान॒ धीराः᳚। व॒न्वन्नवा॑तः परि॒धीꣳ रपो᳚र्णु वी॒रेभि॒रश्वै᳚र्म॒घवा॑ भवा~-~(६५)\ip

%2.6.12.2
नः। त्वꣳ सो॑म पि॒तृभिः॑ संविदा॒नो\-ऽनु॒ द्यावा॑\-पृथि॒वी आ त॑तन्थ। तस्मै॑ त इन्दो ह॒विषा॑ विधेम व॒यꣴ स्या॑म॒ पत॑यो रयी॒णाम्। अग्नि॑ष्वात्ताः पितर॒ एह ग॑च्छत॒ सदः॑सदः सदत सुप्रणीतयः। अ॒त्ता ह॒वीꣳषि॒ प्रय॑तानि ब॒र्॒\mbox{}हिष्यथा॑ र॒यिꣳ सर्व॑वीरं दधातन। बर्\mbox{}हि॑षदः पितर ऊ॒त्य॑र्वागि॒मा वो॑ ह॒व्या च॑कृमा जु॒षध्वम्᳚। त आ ग॒ताव॑सा॒ शन्त॑मे॒नाथा॒स्मभ्य॒ꣳ॒~(६६)\ip

%2.6.12.3
शं योर॑र॒पो द॑धात। आहं पि॒तॄन्थ्सु॑वि॒दत्राꣳ॑ अविथ्सि॒ नपा॑तं च वि॒क्रम॑णं च॒ विष्णोः᳚। ब॒र्॒\mbox{}हि॒षदो॒ ये स्व॒धया॑ सु॒तस्य॒ भज॑न्त पि॒त्वस्त इ॒हाग॑मिष्ठाः। उप॑हूताः पि॒तरो॑ बर्\mbox{}हि॒ष्ये॑षु नि॒धिषु॑ प्रि॒येषु॑। त आग॑मन्तु॒ त इ॒ह श्रु॑व॒न्त्वधि॑ ब्रुवन्तु॒ ते अ॑वन्त्व॒स्मान्। उदी॑रता॒मव॑र॒ उत्परा॑स॒ उन्म॑ध्य॒माः पि॒तरः॑ सो॒म्यासः॑। असुं॒~(६७)\ip

%2.6.12.4
य ई॒युर॑वृ॒का ऋ॑त॒ज्ञास्ते नो॑\-ऽवन्तु पि॒तरो॒ हवे॑षु। इ॒दं पि॒तृभ्यो॒ नमो॑ अस्त्व॒द्य ये पूर्वा॑सो॒ य उप॑रास ई॒युः। ये पार्थि॑वे॒ रज॒स्या निष॑त्ता॒ ये वा॑ नू॒नꣳ सु॑वृ॒जना॑सु वि॒क्षु। अधा॒ यथा॑ नः पि॒तरः॒ परा॑सः प्र॒त्नासो॑ अग्न ऋ॒तमा॑शुषा॒णाः। शुचीद॑य॒न्दीधि॑तिमुक्थ॒शासः॒ क्षामा॑ भि॒न्दन्तो॑ अरु॒णीरप॑ व्रन्न्। यद॑ग्ने~(६८)\ip

%2.6.12.5
कव्यवाहन पि॒तॄन् यक्ष्यृ॑ता॒वृधः॑। प्र च॑ ह॒व्यानि॑ वक्ष्यसि दे॒वेभ्य॑श्च पि॒तृभ्य॒ आ। त्वम॑ग्न ईडि॒तो जा॑तवे॒दो\-ऽवा᳚ड्ढ॒व्यानि॑ सुर॒भीणि॑ कृ॒त्वा। प्रादाः᳚ पि॒तृभ्यः॑ स्व॒धया॒ ते अ॑क्षन्न॒द्धि त्वं दे॑व॒ प्रय॑ता ह॒वीꣳषि॑। मात॑ली क॒व्यैर्य॒मो अङ्गि॑रोभि॒र्बृह॒स्पति॒र्॒\mbox{}ऋक्व॑भिर्वावृधा॒नः। याꣴश्च॑ दे॒वा वा॑वृ॒धुर्ये च॑ दे॒वान्थ्\-स्वाहा॒\-ऽन्ये स्व॒धया॒\-ऽन्ये म॑दन्ति।~(६९)\ip

%2.6.12.6
इ॒मं य॑म प्रस्त॒रमा हि सीदाङ्गि॑रोभिः पि॒तृभिः॑ संविदा॒नः। आ त्वा॒ मन्त्राः᳚ कविश॒स्ता व॑हन्त्वे॒ना रा॑जन् ह॒विषा॑ मादयस्व। अङ्गि॑रोभि॒रा ग॑हि य॒ज्ञिये॑भि॒र्यम॑ वैरू॒पैरि॒ह मा॑दयस्व। विव॑स्वन्तꣳ हुवे॒ यः पि॒ता ते॒\-ऽस्मिन् य॒ज्ञे ब॒र्॒\mbox{}हिष्या नि॒षद्य॑। अङ्गि॑रसो नः पि॒तरो॒ नव॑ग्वा॒ अथ॑र्वाणो॒ भृग॑वः सो॒म्यासः॑। तेषां᳚ व॒यꣳ सु॑म॒तौ य॒ज्ञिया॑ना॒मपि॑ भ॒द्रे सौ॑मन॒से स्या॑म॥~(७०)\ip

{\anuvakamend[{भ॒वा॒स्मभ्य॒मसुं॒ यद॑ग्ने मदन्ति सौमन॒स एक॑ञ्च}]}%॥12॥

\prashnaend{स॒मिध॒श्चक्षु॑षी प्र॒जा\-प॑ति॒राज्यं॑ दे॒वस्य॒ स्फ्यम्ब्र॑ह्मवा॒दिनो॒\-ऽद्भिर॒ग्नेस्त्रयो॒ मनुः॑ पृथि॒व्याः प॒शवो॒\-ऽग्नीधे॑ दे॒वा वै य॒ज्ञस्य॑ यु॒क्ष्वोशन्त॑स्त्वा॒ द्वाद॑श॥१२॥}{स॒मिधो॑ या॒ज्या॑ तस्मा॒न्नाभा॒गꣳ हि तमन्वित्या॑ह प्र॒जा वा आ॒हेत्या॑ह यु॒क्ष्वा हि स॑प्त॒तिः॥७०॥}{स॒मिधः॑ सौमन॒से स्या॑म॥}%%२-६
{हरिः॑ ॐ}{॥कृष्ण-यजुर्वेदीय-तैत्तिरीय-संहितायां द्वितीयकाण्डे षष्ठः प्रश्नः समाप्तः॥२-६॥}

\centerline{॥कृष्ण-यजुर्वेदीय-तैत्तिरीय-संहितायां द्वितीयकाण्डः समाप्तः॥२॥}

\hyperref[sec:startkanda2]{\closesection}
%%% END PRASHNA
