\chapt{काण्डम् ३}
\sect{तृतीयः प्रश्नः}\setcounter{anuvakam}{0}
\dnsub{तैत्तिरीयसंहितायां तृतीयकाण्डे तृतीयः प्रश्नः}
%3.3.1.1
अग्ने॑ तेजस्विन्तेज॒स्वी त्वं दे॒वेषु॑ भूया॒स्तेज॑स्वन्तं॒ मामायु॑ष्मन्तं॒ वर्च॑स्वन्तं मनु॒ष्ये॑षु कुरु दी॒क्षायै॑ च त्वा॒ तप॑सश्च॒ तेज॑से जुहोमि तेजो॒विद॑सि॒ तेजो॑ मा॒ मा हा॑सी॒न्मा\-ऽहं तेजो॑ हासिषं॒ मा मां तेजो॑ हासी॒दिन्द्रौ॑जस्विन्नोज॒स्वी त्वं दे॒वेषु॑ भूया॒ ओज॑स्वन्तं॒ मामायु॑ष्मन्तं॒ वर्च॑स्वन्तं मनु॒ष्ये॑षु कुरु॒ ब्रह्म॑णश्च त्वा क्ष॒त्रस्य॒ चौ-~(१)

%3.3.1.2
ज॑से जुहोम्योजो॒विद॒स्योजो॑ मा॒ मा हा॑सी॒न्मा\-ऽहमोजो॑ हासिषं॒ मा मामोजो॑ हासी॒थ्सूर्य॑ भ्राजस्विन्भ्राज॒स्वी त्वं दे॒वेषु॑ भूया॒ भ्राज॑स्वन्तं॒ मामायु॑ष्मन्तं॒ वर्च॑स्वन्तं मनु॒ष्ये॑षु कुरु वा॒योश्च॑ त्वा॒\-ऽपां च॒ भ्राज॑से जुहोमि सुव॒र्विद॑सि॒ सुव॑र्मा॒ मा हा॑सी॒न्मा\-ऽहꣳ सुव॑र्\mbox{}हासिषं॒ मा माꣳ सुव॑र्\mbox{}हासी॒न्मयि॑ मे॒धां मयि॑ प्र॒जां मय्य॒ग्निस्तेजो॑ दधातु॒ मयि॑ मे॒धां मयि॑ प्र॒जां मयीन्द्र॑ इन्द्रि॒यं द॑धातु॒ मयि॑ मे॒धां मयि॑ प्र॒जां मयि॒ सूर्यो॒ भ्राजो॑ दधातु॥~(२)

{\anuvakamend[{क्ष॒त्रस्य॑ च॒ मयि॒ त्रयो॑विꣳशतिश्च}]}%~(१)

%3.3.2.1
वा॒युर्\mbox{}हि॑ङ्क॒र्ता\-ऽग्निः प्र॑स्तो॒ता प्र॒जा\-प॑तिः॒ साम॒ बृह॒स्पति॑रुद्गा॒ता विश्वे॑ दे॒वा उ॑पगा॒तारो॑ म॒रुतः॑ प्रतिह॒र्तार॒ इन्द्रो॑ नि॒धनं॒ ते दे॒वाः प्रा॑ण॒भृतः॑ प्रा॒णं मयि॑ दधत्वे॒तद्वै सर्व॑मध्व॒र्युरु॑पाकु॒र्वन्नु॑द्गा॒तृभ्य॑ उ॒पा\-क॑रोति॒ ते दे॒वाः प्रा॑ण॒भृतः॑ प्रा॒णं मयि॑ दध॒त्वित्या॑है॒तदे॒व सर्व॑मा॒त्मन्ध॑त्त॒ इडा॑ देव॒हूर्मनु॑र्यज्ञ॒नीर्बृह॒स्पति॑रुक्थाम॒दानि॑ शꣳसिष॒द् विश्वे॑ दे॒वाः~(३)

%3.3.2.2
सू᳚क्त॒वाचः॒ पृथि॑वि मात॒र्मा मा॑ हिꣳसी॒र्मधु॑ मनिष्ये॒ मधु॑ जनिष्ये॒ मधु॑ वक्ष्यामि॒ मधु॑ वदिष्यामि॒ मधु॑मतीं दे॒वेभ्यो॒ वाच॑मुद्यासꣳ शुश्रू॒षेण्यां᳚ मनु॒ष्ये᳚भ्य॒स्तं मा॑ दे॒वा अ॑वन्तु शो॒भायै॑ पि॒तरो\-ऽनु॑ मदन्तु॥~(४)

{\anuvakamend[{श॒ꣳ॒सि॒ष॒द्विश्वे॑ दे॒वा अ॒ष्टाविꣳ॑शतिश्च}]}%~(२)

%3.3.3.1
वस॑वस्त्वा॒ प्र वृ॑हन्तु गाय॒त्रेण॒ छन्द॑सा॒\-ऽग्नेः प्रि॒यं पाथ॒ उपे॑हि रु॒द्रास्त्वा॒ प्र वृ॑हन्तु॒ त्रैष्टु॑भेन॒ छन्द॒सेन्द्र॑स्य प्रि॒यं पाथ॒ उपे᳚ह्यादि॒त्यास्त्वा॒ प्र वृ॑हन्तु॒ जाग॑तेन॒ छन्द॑सा॒ विश्वे॑षां दे॒वानां᳚ प्रि॒यं पाथ॒ उपे॑हि॒ मान्दा॑सु ते शुक्र शु॒क्रमा धू॑नोमि भ॒न्दना॑सु॒ कोत॑नासु॒ नूत॑नासु॒ रेशी॑षु॒ मेषी॑षु॒ वाशी॑षु विश्व॒भृथ्सु॒ माध्वी॑षु ककु॒हासु॒ शक्व॑रीषु~(५)

%3.3.3.2
शु॒क्रासु॑ ते शुक्र शु॒क्रमा धू॑नोमि शु॒क्रं ते॑ शु॒क्रेण॑ गृह्णा॒म्यह्नो॑ रू॒पेण॒ सूर्य॑स्य र॒श्मिभिः॑। आ\-ऽस्मि॑न्नु॒ग्रा अ॑चुच्यवुर्दि॒वो धारा॑ असश्चत। क॒कु॒हꣳ रू॒पं वृ॑ष॒भस्य॑ रोचते बृ॒हथ्सोमः॒ सोम॑स्य पुरो॒गाः शु॒क्रः शु॒क्रस्य॑ पुरो॒गाः। यत्ते॑ सो॒मादा᳚भ्यं॒ नाम॒ जागृ॑वि॒ तस्मै॑ ते सोम॒ सोमा॑य॒ स्वाहो॒शिक्त्वं दे॑व सोम गाय॒त्रेण॒ छन्द॑सा॒\-ऽग्नेः~(६)

%3.3.3.3
प्रि॒यं पाथो॒ अपी॑हि व॒शी त्वं दे॑व सोम॒ त्रैष्टु॑भेन॒ छन्द॒सेन्द्र॑स्य प्रि॒यं पाथो॒ अपी᳚ह्य॒स्मथ्स॑खा॒ त्वं दे॑व सोम॒ जाग॑तेन॒ छन्द॑सा॒ विश्वे॑षां दे॒वानां᳚ प्रि॒यं पाथो॒ अपी॒ह्या नः॑ प्रा॒ण ए॑तु परा॒वत॒ आन्तरि॑क्षाद्दि॒वस्परि॑। आयुः॑ पृथि॒व्या अध्य॒मृत॑मसि प्रा॒णाय॑ त्वा। इ॒न्द्रा॒ग्नी मे॒ वर्चः॑ कृणुतां॒ वर्चः॒ सोमो॒ बृह॒स्पतिः॑। वर्चो॑ मे॒ विश्वे॑ दे॒वा वर्चो॑ मे धत्तमश्विना। द॒ध॒न्वे वा॒ यदी॒मनु॒ वोच॒द्ब्रह्मा॑णि॒ वेरु॒ तत्। परि॒ विश्वा॑नि॒ काव्या॑ ने॒मिश्च॒क्रमि॑वाभवत्॥~(७)

{\anuvakamend[{शक्व॑रीष्व॒ग्नेर्बृह॒स्पतिः॒ पञ्च॑विꣳशतिश्च}]}%~(३)

%3.3.4.1
ए॒तद्वा अ॒पां ना॑म॒धेयं॒ गुह्यं॒ यदा॑धा॒वा मान्दा॑सु ते शुक्र शु॒क्रमा धू॑नो॒मीत्या॑हा॒पामे॒व ना॑म॒धेये॑न॒ गुह्ये॑न दि॒वो वृष्टि॒मव॑ रुन्धे शु॒क्रं ते॑ शु॒क्रेण॑ गृह्णा॒मीत्या॑है॒तद्वा अह्नो॑ रू॒पं यद्रात्रिः॒ सूर्य॑स्य र॒श्मयो॒ वृष्ट्या॑ ईश॒ते\-ऽह्न॑ ए॒व रू॒पेण॒ सूर्य॑स्य र॒श्मिभि॑र्दि॒वो वृष्टिं॑ च्यावय॒त्या\-ऽस्मि॑न्नु॒ग्रा~-~(८)

%3.3.4.2
अ॑चुच्यवु॒रित्या॑ह यथाय॒जुरे॒वैतत्क॑कु॒हꣳ रू॒पं वृ॑ष॒भस्य॑ रोचते बृ॒हदित्या॑है॒तद्वा अ॑स्य ककु॒हꣳ रू॒पं यद्वृष्टी॑ रू॒पेणै॒व वृष्टि॒मव॑ रुन्धे॒ यत्ते॑ सो॒मादा᳚भ्यं॒ नाम॒ जागृ॒वीत्या॑है॒ष ह॒ वै ह॒विषा॑ ह॒विर्य॑जति॒ यो\-ऽ\-दा᳚भ्यं गृही॒त्वा सोमा॑य जु॒होति॒ परा॒ वा ए॒तस्या\-ऽऽ\-युः॑ प्रा॒ण ए॑ति॒~(९)

%3.3.4.3
यो\-ऽꣳ॑शुं गृ॒ह्णात्या नः॑ प्रा॒ण ए॑तु परा॒वत॒ इत्या॒हा\-ऽऽ\-यु॑रे॒व प्रा॒णमा॒त्मन्ध॑त्ते॒\-ऽमृत॑मसि प्रा॒णाय॒ त्वेति॒ हिर॑ण्यम॒भि व्य॑नित्य॒मृतं॒ वै हिर॑ण्य॒मायुः॑ प्रा॒णो॑\-ऽमृते॑नै॒वा\-ऽऽ\-\-यु॑रा॒त्मन्ध॑त्ते श॒तमा॑नं भवति श॒तायुः॒ पुरु॑षः श॒तेन्द्रि॑य॒ आयु॑ष्ये॒वेन्द्रि॒ये प्रति॑ तिष्ठत्य॒प उप॑ स्पृशति भेष॒जं वा आपो॑ भेष॒जमे॒व कु॑रुते॥~(१०)

{\anuvakamend[{उ॒ग्रा ए॒त्याप॒स्त्रीणि॑ च}]}%~(४)

%3.3.5.1
वा॒युर॑सि प्रा॒णो नाम॑ सवि॒तुराधि॑पत्ये\-ऽपा॒नं मे॑ दा॒श्चक्षु॑रसि॒ श्रोत्रं॒ नाम॑ धा॒तुराधि॑पत्य॒ आयु॑र्मे दा रू॒पम॑सि॒ वर्णो॒ नाम॒ बृह॒स्पते॒राधि॑पत्ये प्र॒जां मे॑ दा ऋ॒तम॑सि स॒त्यं नामेन्द्र॒स्या\-ऽऽ\-धि॑पत्ये क्ष॒त्रं मे॑ दा भू॒तम॑सि॒ भव्यं॒ नाम॑ पितृ॒णामाधि॑पत्ये॒\-ऽपामोष॑धीनां॒ गर्भं॑ धा ऋ॒तस्य॑ त्वा॒ व्यो॑मन ऋ॒तस्य॑~(११)

%3.3.5.2
त्वा॒ विभू॑मन ऋ॒तस्य॑ त्वा॒ विध॑र्मण ऋ॒तस्य॑ त्वा स॒त्याय॒र्तस्य॑ त्वा॒ ज्योति॑षे प्र॒जा\-प॑तिर्वि॒राज॑मपश्य॒त्तया॑ भू॒तं च॒ भव्यं॑ चासृजत॒ तामृषि॑भ्यस्ति॒रो॑\-ऽदधा॒त्तां ज॒मद॑ग्नि॒स्तप॑सा\-ऽपश्य॒त्तया॒ वै स पृश्ञी॒न्कामा॑नसृजत॒ तत्पृश्ञी॑नां पृश्ञि॒त्वं यत्पृश्ञ॑यो गृ॒ह्यन्ते॒ पृश्ञी॑ने॒व तैः कामा॒न्॒ यज॑मा॒नो\-ऽव॑ रुन्धे वा॒युर॑सि प्रा॒णो~(१२)

%3.3.5.3
नामेत्या॑ह प्राणापा॒नावे॒वाव॑ रुन्धे॒ चक्षु॑रसि॒ श्रोत्रं॒ नामेत्या॒हा\-ऽऽ\-यु॑रे॒वाव॑ रुन्धे रू॒पम॑सि॒ वर्णो॒ नामेत्या॑ह प्र॒जामे॒वाव॑ रुन्ध ऋ॒तम॑सि स॒त्यं नामेत्या॑ह क्ष॒त्रमे॒वाव॑ रुन्धे भू॒तम॑सि॒ भव्यं॒ नामेत्या॑ह प॒शवो॒ वा अ॒पामोष॑धीनां॒ गर्भः॑ प॒शूने॒वा-~(१३)

%3.3.5.4
ऽव॑ रुन्ध ए॒ताव॒द्वै पुरु॑षं प॒रित॒स्तदे॒वाव॑ रुन्ध ऋ॒तस्य॑ त्वा॒ व्यो॑मन॒ इत्या॑हे॒यं वा ऋ॒तस्य॒ व्यो॑मे॒मामे॒वाभि ज॑यत्यृ॒तस्य॑ त्वा॒ विभू॑मन॒ इत्या॑हा॒न्तरि॑क्षं॒ वा ऋ॒तस्य॒ विभू॑मा॒न्तरि॑क्षमे॒वाभि ज॑यत्यृ॒तस्य॑ त्वा॒ विध॑र्मण॒ इत्या॑ह॒ द्यौर्वा ऋ॒तस्य॒ विध॑र्म॒ दिव॑मे॒वाभि ज॑यत्यृ॒तस्य॑~(१४)

%3.3.5.5
त्वा स॒त्यायेत्या॑ह॒ दिशो॒ वा ऋ॒तस्य॑ स॒त्यं दिश॑ ए॒वाभि ज॑यत्यृ॒तस्य॑ त्वा॒ ज्योति॑ष॒ इत्या॑ह सुव॒र्गो वै लो॒क ऋ॒तस्य॒ ज्योतिः॑ सुव॒र्गमे॒व लो॒कम॒भि ज॑यत्ये॒ताव॑न्तो॒ वै दे॑वलो॒कास्ताने॒वाभि ज॑यति॒ दश॒ सं प॑द्यन्ते॒ दशा᳚क्षरा वि॒राडन्नं॑ वि॒राड्वि॒राज्ये॒वान्नाद्ये॒ प्रति॑ तिष्ठति॥~(१५)

{\anuvakamend[{व्यो॑मन ऋ॒तस्य॑ प्रा॒णः प॒शूने॒व विध॑र्म॒ दिव॑मे॒वाभि ज॑यत्यृ॒तस्य॒ षट्च॑त्वारिꣳशच्च}]}%~(५)

%3.3.6.1
दे॒वा वै यद्य॒ज्ञेन॒ नावारु॑न्धत॒ तत्परै॒रवा॑रुन्धत॒ तत्परा॑णां पर॒त्वं यत्परे॑ गृ॒ह्यन्ते॒ यदे॒व य॒ज्ञेन॒ नाव॑रु॒न्धे तस्याव॑रुद्ध्यै॒ यं प्र॑थ॒मं गृ॒ह्णाती॒ममे॒व तेन॑ लो॒कम॒भि ज॑यति॒ यं द्वि॒तीय॑म॒न्तरि॑क्षं॒ तेन॒ यं तृ॒तीय॑म॒मुमे॒व तेन॑ लो॒कम॒भि ज॑यति॒ यदे॒ते गृ॒ह्यन्त॑ ए॒षां लो॒काना॑म॒भिजि॑त्या॒~-~(१६)

%3.3.6.2
उत्त॑रे॒ष्वहः॑स्व॒मुतो॒\-ऽर्वाञ्चो॑ गृह्यन्ते\-ऽभि॒जित्यै॒वेमाँल्लो॒कान्पुन॑रि॒मं लो॒कं प्र॒त्यव॑रोहन्ति॒ यत्पूर्वे॒ष्वहः॑स्वि॒तः परा᳚ञ्चो गृ॒ह्यन्ते॒ तस्मा॑दि॒तः परा᳚ञ्च इ॒मे लो॒का यदुत्त॑रे॒ष्वहः॑स्व॒मुतो॒\-ऽर्वाञ्चो॑ गृ॒ह्यन्ते॒ तस्मा॑द॒मुतो॒\-ऽर्वाञ्च॑ इ॒मे लो॒कास्तस्मा॒द\-या॑तयाम्नो लो॒कान्म॑नु॒ष्या॑ उप॑ जीवन्ति ब्रह्मवा॒दिनो॑ वदन्ति॒ कस्मा᳚थ्स॒त्याद॒द्भ्य ओष॑धयः॒ सम्भ॑व॒न्त्योष॑धयो~(१७)

%3.3.6.3
मनु॒ष्या॑णा॒मन्नं॑ प्र॒जा\-प॑तिं प्र॒जा अनु॒ प्र जा॑यन्त॒ इति॒ परा॒नन्विति॑ ब्रूया॒द्यद्गृ॒ह्णात्य॒द्भ्यस्त्वौष॑\-धीभ्यो गृह्णा॒मीति॒ तस्मा॑द॒द्भ्य ओष॑धयः॒ सम्भ॑वन्ति॒ यद्गृ॒ह्णात्योष॑धीभ्यस्त्वा प्र॒जाभ्यो॑ गृह्णा॒मीति॒ तस्मा॒दोष॑धयो मनु॒ष्या॑णा॒मन्नं॒ यद्गृ॒ह्णाति॑ प्र॒जाभ्य॑स्त्वा प्र॒जा\-प॑तये गृह्णा॒मीति॒ तस्मा᳚त्प्र॒जा\-प॑तिं प्र॒जा अनु॒ प्र जा॑यन्ते॥~(१८)

{\anuvakamend[{अ॒भिजि॑त्या॒ ओष॑धयो॒\-ऽष्टाच॑त्वारिꣳशच्च}]}%~(६)

%3.3.7.1
प्र॒जा\-प॑तिर्देवासु॒रान॑सृजत॒ तदनु॑ य॒ज्ञो॑\-ऽसृज्यत य॒ज्ञं छन्दाꣳ॑सि॒ ते विष्व॑ञ्चो॒ व्य॑क्राम॒न्थ्सो\-ऽसु॑रा॒ननु॑ य॒ज्ञो\-ऽपा᳚क्रामद्य॒ज्ञं छन्दाꣳ॑सि॒ ते दे॒वा अ॑मन्यन्ता॒मी वा इ॒दम॑भूव॒न्॒ यद्व॒यꣴ स्म इति॒ ते प्र॒जा\-प॑ति॒मुपा॑धाव॒न्थ्सो᳚\-ऽब्रवीत्प्र॒जा\-प॑ति॒श्छन्द॑सां वी॒र्य॑मा॒दाय॒ तद्वः॒ प्र दा᳚स्या॒मीति॒ स छन्द॑सां वी॒र्य॑-~(१९)

%3.3.7.2
मा॒दाय॒ तदे᳚भ्यः॒ प्राय॑च्छ॒त्तदनु॒ छन्दा॒ꣴ॒स्यपा᳚क्राम॒ञ्छन्दाꣳ॑सि य॒ज्ञस्ततो॑ दे॒वा अभ॑व॒न्परासु॑रा॒ य ए॒वं छन्द॑सां वी॒र्यं॑ वेदा\-ऽऽ\- श्रा॑व॒यास्तु॒ श्रौष॒ड्यज॒ ये यजा॑महे वषट्का॒रो भव॑त्या॒त्मना॒ परा᳚\-ऽस्य॒ भ्रातृ॑व्यो भवति ब्रह्मवा॒दिनो॑ वदन्ति॒ कस्मै॒ कम॑ध्व॒र्युरा श्रा॑वय॒तीति॒ छन्द॑सां वी॒र्या॑येति॑ ब्रूयादे॒तद्वै~(२०)

%3.3.7.3
छन्द॑सां वी॒र्य॑मा श्रा॑व॒यास्तु॒ श्रौष॒ड्यज॒ ये यजा॑महे वषट्का॒रो य ए॒वं वेद॒ सवी᳚र्यैरे॒व छन्दो॑भिरर्चति॒ यत्किं चार्च॑ति॒ यदिन्द्रो॑ वृ॒त्रमह॑न्नमे॒ध्यं तद्यद्यती॑न॒पाव॑पदमे॒ध्यं तदथ॒ कस्मा॑दै॒न्द्रो य॒ज्ञ आ सꣴस्था॑तो॒रित्या॑हु॒रिन्द्र॑स्य॒ वा ए॒षा य॒ज्ञिया॑ त॒नूर्यद्य॒ज्ञस्तामे॒व तद्य॑जन्ति॒ य ए॒वं वेदोपै॑नं य॒ज्ञो न॑मति॥~(२१)

{\anuvakamend[{स छन्द॑सां वी॒र्यं॑ वा ए॒व तद॒ष्टौ च॑}]}%~(७)

%3.3.8.1
आ॒यु॒र्दा अ॑ग्ने ह॒विषो॑ जुषा॒णो घृ॒तप्र॑तीको घृ॒तयो॑निरेधि। घृ॒तं पी॒त्वा मधु॒ चारु॒ गव्यं॑ पि॒तेव॑ पु॒त्रम॒भि र॑क्षतादि॒मम्। आ वृ॑श्च्यते॒ वा ए॒तद्यज॑मानो॒\-ऽग्निभ्यां॒ यदे॑नयोः शृत॒ङ्कृत्याथा॒न्यत्रा॑वभृ॒थम॒वैत्या॑यु॒र्दा अ॑ग्ने ह॒विषो॑ जुषा॒ण इत्य॑वभृ॒थम॑वै॒ष्यञ्जु॑हुया॒दाहु॑त्यै॒वैनौ॑ शमयति॒ नार्ति॒मार्च्छ॑ति॒ यज॑मानो॒ यत्कुसी॑द॒-~(२२)

%3.3.8.2
मप्र॑तीत्तं॒ मयि॒ येन॑ य॒मस्य॑ ब॒लिना॒ चरा॑मि। इ॒हैव सन्नि॒रव॑दये॒ तदे॒तत्तद॑ग्ने अनृ॒णो भ॑वामि। विश्व॑लोप विश्वदा॒वस्य॑ त्वा॒\-ऽऽ\-सञ्जु॑होम्य॒ग्धादेको॑\-ऽहु॒तादेकः॑ समस॒नादेकः॑। ते नः॑ कृण्वन्तु भेष॒जꣳ सदः॒ सहो॒ वरे᳚ण्यम्। अ॒यं नो॒ नभ॑सा पु॒रः स॒ꣴ॒स्फानो॑ अ॒भि र॑क्षतु। गृ॒हाणा॒मस॑मर्त्यै ब॒हवो॑ नो गृ॒हा अ॑सन्न्। स त्वं नो॑~(२३)

%3.3.8.3
नभसस्पत॒ ऊर्जं॑ नो धेहि भ॒द्रया᳚। पुन॑र्नो न॒ष्टमा कृ॑धि॒ पुन॑र्नो र॒यिमा कृ॑धि। देव॑ सꣴस्फान सहस्रपो॒षस्ये॑शिषे॒ स नो॑ रा॒स्वाज्या॑निꣳ रा॒यस्पोषꣳ॑ सु॒वीर्यꣳ॑ संवथ्स॒रीणाꣴ॑ स्व॒स्तिम्। अ॒ग्निर्वाव य॒म इ॒यं य॒मी कुसी॑दं॒ वा ए॒तद्य॒मस्य॒ यज॑मान॒ आ द॑त्ते॒ यदोष॑धीभि॒र्वेदिꣴ॑ स्तृ॒णाति॒ यदनु॑पौष्य प्रया॒याद्ग्री॑वब॒द्धमे॑न-~(२४)

%3.3.8.4
म॒मुष्मिँ॑ल्लो॒के ने॑नीयेर॒न्॒ यत्कुसी॑द॒मप्र॑तीत्तं॒ मयीत्युपौ॑षती॒हैव सन् य॒मं कुसी॑दं निरव॒दाया॑नृ॒णः सु॑व॒र्गं लो॒कमे॑ति॒ यदि॑ मि॒श्रमि॑व॒ चरे॑दञ्ज॒लिना॒ सक्तू᳚न्प्रदा॒व्ये॑ जुहुयादे॒ष वा अ॒ग्निर्वै᳚श्वान॒रो यत्प्र॑दा॒व्यः॑ स ए॒वैनꣴ॑ स्वदय॒त्यह्नां᳚ वि॒धान्या॑मेकाष्ट॒काया॑मपू॒पं चतुः॑शरावं प॒क्त्वा प्रा॒तरे॒तेन॒ कक्ष॒मुपौ॑षे॒द्यदि॒~(२५)

%3.3.8.5
दह॑ति पुण्य॒समं॑ भवति॒ यदि॒ न दह॑ति पाप॒सम॑मे॒तेन॑ ह स्म॒ वा ऋष॑यः पु॒रा वि॒ज्ञाने॑न दीर्घस॒त्रमुप॑ यन्ति॒ यो वा उ॑पद्र॒ष्टार॑मुपश्रो॒तार॑मनुख्या॒तारं॑ वि॒द्वान् यज॑ते॒ सम॒मुष्मिँ॑ल्लो॒क इ॑ष्टापू॒र्तेन॑ गच्छते॒\-ऽग्निर्वा उ॑पद्र॒ष्टा वा॒युरु॑पश्रो॒ता\-ऽऽ\-\-दि॒त्यो॑\-ऽनुख्या॒ता तान् य ए॒वं वि॒द्वान् यज॑ते॒ सम॒मुष्मिँ॑ल्लो॒क इ॑ष्टापू॒र्तेन॑ गच्छते॒\-ऽयं नो॒ नभ॑सा पु॒र~(२६)

%3.3.8.6
इत्या॑हा॒\-ऽग्निर्वै नभ॑सा पु॒रो᳚\-ऽग्निमे॒व तदा॑है॒तन्मे॑ गोपा॒येति॒ स त्वं नो॑ नभसस्पत॒ इत्या॑ह वा॒युर्वै नभ॑स॒स्पति॑र्वा॒युमे॒व तदा॑है॒तन्मे॑ गोपा॒येति॒ देव॑ सꣴस्फा॒नेत्या॑हा॒सौ वा आ॑दि॒त्यो दे॒वः स॒ꣴ॒स्फान॑ आदि॒त्यमे॒व तदा॑है॒तन्मे॑ गोपा॒येति॑॥~(२७)

{\anuvakamend[{कुसी॑द॒न्त्वन्न॑ एनमोषे॒द्यदि॑ पु॒र आ॑दि॒त्यमे॒व तदा॑है॒तन्मे॑ गोपा॒येति॑}]}%~(८)

%3.3.9.1
ए॒तं युवा॑नं॒ परि॑ वो ददामि॒ तेन॒ क्रीड॑न्तीश्चरत प्रि॒येण॑। मा नः॑ शाप्त ज॒नुषा॑ सुभागा रा॒यस्पोषे॑ण॒ समि॒षा म॑देम। नमो॑ महि॒म्न उ॒त चक्षु॑षे ते॒ मरु॑तां पित॒स्तद॒हं गृ॑णामि। अनु॑ मन्यस्व सु॒यजा॑ यजाम॒ जुष्टं॑ दे॒वाना॑मि॒दम॑स्तु ह॒व्यम्। दे॒वाना॑मे॒ष उ॑पना॒ह आ॑सीद॒पां गर्भ॒ ओष॑धीषु॒ न्य॑क्तः। सोम॑स्य द्र॒फ्सम॑वृणीत पू॒षा~(२८)

%3.3.9.2
बृ॒हन्नद्रि॑रभव॒त्तदे॑षाम्। पि॒ता व॒थ्सानां॒ पति॑रघ्नि॒याना॒मथो॑ पि॒ता म॑ह॒तां गर्ग॑राणाम्। व॒थ्सो ज॒रायु॑ प्रति॒धुक्पी॒यूष॑ आ॒मिक्षा॒ मस्तु॑ घृ॒तम॑स्य॒ रेतः॑। त्वां गावो॑\-ऽवृणत रा॒ज्याय॒ त्वाꣳ ह॑वन्त म॒रुतः॑ स्व॒र्काः। वर्ष्म॑न्क्ष॒त्रस्य॑ क॒कुभि॑ शिश्रिया॒णस्ततो॑ न उ॒ग्रो वि भ॑जा॒ वसू॑नि। व्यृ॑द्धेन॒ वा ए॒ष प॒शुना॑ यजते॒ यस्यै॒तानि॒ न क्रि॒यन्त॑ ए॒ष ह॒ त्वै समृ॑द्धेन यजते॒ यस्यै॒तानि॑ क्रि॒यन्ते᳚॥~(२९)

{\anuvakamend[{पू॒षा क्रि॒यन्त॑ ए॒षो᳚\-ऽष्टौ च॑}]}%~(९)

%3.3.10.1
सूर्यो॑ दे॒वो दि॑वि॒षद्भ्यो॑ धा॒ता क्ष॒त्राय॑ वा॒युः प्र॒जाभ्यः॑। बृह॒स्पति॑स्त्वा प्र॒जा\-प॑तये॒ ज्योति॑ष्मतीं जुहोतु। यस्या᳚स्ते॒ हरि॑तो॒ गर्भो\-ऽथो॒ योनि॑र्\mbox{}हिर॒ण्ययी᳚। अङ्गा॒न्यह्रु॑ता॒ यस्यै॒ तां दे॒वैः सम॑जीगमम्। आ व॑र्तन वर्तय॒ नि नि॑वर्तन वर्त॒येन्द्र॑ नर्दबुद। भूम्या॒श्चत॑स्रः प्र॒दिश॒स्ताभि॒रा व॑र्तया॒ पुनः॑। वि ते॑ भिनद्मि तक॒रीं वि योनिं॒ वि ग॑वी॒न्यौ᳚। वि~(३०)

%3.3.10.2
मा॒तरं॑ च पु॒त्रं च॒ वि गर्भं॑ च ज॒रायु॑ च। ब॒हिस्ते॑ अस्तु॒ बालिति॑। उ॒रु॒द्र॒फ्सो वि॒श्वरू॑प॒ इन्दुः॒ पव॑मानो॒ धीर॑ आनञ्ज॒ गर्भम्᳚। एक॑पदी द्वि॒पदी᳚ त्रि॒पदी॒ चतु॑ष्पदी॒ पञ्च॑पदी॒ षट्प॑दी स॒प्तप॑द्य॒ष्टाप॑दी॒ भुव॒नानु॑ प्रथता॒ꣴ॒ स्वाहा᳚। म॒ही द्यौः पृ॑थि॒वी च॑ न इ॒मं य॒ज्ञं मि॑मिक्षताम्। पि॒पृ॒तां नो॒ भरी॑मभिः॥~(३१)

{\anuvakamend[{ग॒वी॒न्यौ॑ वि चतु॑श्चत्वारिꣳशच्च}]}%॥10॥

%3.3.11.1
इ॒दं वा॑मा॒स्ये॑ ह॒विः प्रि॒यमि॑न्द्राबृहस्पती। उ॒क्थं मद॑श्च शस्यते। अ॒यं वां॒ परि॑ षिच्यते॒ सोम॑ इन्द्राबृहस्पती। चारु॒र्मदा॑य पी॒तये᳚। अ॒स्मे इ॑न्द्राबृहस्पती र॒यिं ध॑त्तꣳ शत॒ग्विनम्᳚। अश्वा॑वन्तꣳ सह॒स्रिणम्᳚। बृह॒स्पति॑र्नः॒ परि॑ पातु प॒श्चादु॒तोत्त॑रस्मा॒दध॑रादघा॒योः। इन्द्रः॑ पु॒रस्ता॑दु॒त म॑ध्य॒तो नः॒ सखा॒ सखि॑भ्यो॒ वरि॑वः कृणोतु। वि ते॒ विष्व॒ग्वात॑जूतासो अग्ने॒ भामा॑सः~(३२)

%3.3.11.2
शुचे॒ शुच॑यश्चरन्ति। तु॒वि॒म्र॒क्षासो॑ दि॒व्या नव॑ग्वा॒ वना॑ वनन्ति धृष॒ता रु॒जन्तः॑। त्वाम॑ग्ने॒ मानु॑षीरीडते॒ विशो॑ होत्रा॒विदं॒ विवि॑चिꣳ रत्न॒धात॑मम्। गुहा॒ सन्तꣳ॑ सुभग वि॒श्वद॑र्\mbox{}शतं तुविष्म॒णसꣳ॑ सु॒यजं॑ घृत॒श्रियम्᳚। धा॒ता द॑दातु नो र॒यिमीशा॑नो॒ जग॑त॒स्पतिः॑। स नः॑ पू॒र्णेन॑ वावनत्। धा॒ता प्र॒जाया॑ उ॒त रा॒य ई॑शे धा॒तेदं विश्वं॒ भुव॑नं जजान। धा॒ता पु॒त्रं यज॑मानाय॒ दाता॒~(३३)

%3.3.11.3
तस्मा॑ उ ह॒व्यं घृ॒तव॑द्विधेम। धा॒ता द॑दातु नो र॒यिं प्राचीं᳚ जी॒वातु॒मक्षि॑ताम्। व॒यं दे॒वस्य॑ धीमहि सुम॒तिꣳ स॒त्यरा॑धसः। धा॒ता द॑दातु दा॒शुषे॒ वसू॑नि प्र॒जाका॑माय मी॒ढुषे॑ दुरो॒णे। तस्मै॑ दे॒वा अ॒मृताः॒ सं व्य॑यन्तां॒ विश्वे॑ दे॒वासो॒ अदि॑तिः स॒जोषाः᳚। अनु॑ नो॒\-ऽद्यानु॑मतिर्य॒ज्ञं दे॒वेषु॑ मन्यताम्। अ॒ग्निश्च॑ हव्य॒वाह॑नो॒ भव॑तां दा॒शुषे॒ मयः॑। अन्विद॑नुमते॒ त्वं~(३४)

%3.3.11.4
मन्या॑सै॒ शं च॑ नः कृधि। क्रत्वे॒ दक्षा॑य नो हिनु॒ प्र ण॒ आयूꣳ॑षि तारिषः। अनु॑ मन्यतामनु॒मन्य॑माना प्र॒जाव॑न्तꣳ र॒यिमक्षी॑यमाणम्। तस्यै॑ व॒यꣳ हेड॑सि॒ मा\-ऽपि॑ भूम॒ सा नो॑ दे॒वी सु॒हवा॒ शर्म॑ यच्छतु। यस्या॑मि॒दं प्र॒दिशि॒ यद्वि॒रोच॒ते\-ऽनु॑मतिं॒ प्रति॑ भूषन्त्या॒यवः॑। यस्या॑ उ॒पस्थ॑ उ॒र्व॑न्तरि॑क्ष॒ꣳ॒ सा नो॑ दे॒वी सु॒हवा॒ शर्म॑ यच्छतु।~(३५)

%3.3.11.5
रा॒काम॒हꣳ सु॒हवाꣳ॑ सुष्टु॒ती हु॑वे शृ॒णोतु॑ नः सु॒भगा॒ बोध॑तु॒ त्मना᳚। सीव्य॒त्वपः॑ सू॒च्या\-ऽच्छि॑द्यमानया॒ ददा॑तु वी॒रꣳ श॒तदा॑यमु॒क्थ्यम्᳚। यास्ते॑ राके सुम॒तयः॑ सु॒पेश॑सो॒ याभि॒र्ददा॑सि दा॒शुषे॒ वसू॑नि। ताभि॑र्नो अ॒द्य सु॒मना॑ उ॒पाग॑हि सहस्रपो॒षꣳ सु॑भगे॒ ररा॑णा। सिनी॑वालि॒ या सु॑पा॒णिः। कु॒हूम॒हꣳ सु॒भगां᳚ विद्म॒नाप॑सम॒स्मिन् य॒ज्ञे सु॒हवां᳚ जोहवीमि। सा नो॑ ददातु॒ श्रव॑णं पितृ॒णां तस्या᳚स्ते देवि ह॒विषा॑ विधेम। कु॒हूर्दे॒वाना॑म॒मृत॑स्य॒ पत्नी॒ हव्या॑ नो अ॒स्य ह॒विष॑श्चिकेतु। सं दा॒शुषे॑ कि॒रतु॒ भूरि॑ वा॒मꣳ रा॒यस्पोषं॑ चिकि॒तुषे॑ दधातु॥~(३६)

{\anuvakamend[{भामा॑सो॒ दाता॒ त्वम॒न्तरि॑क्ष॒ꣳ॒ सा नो॑ दे॒वी सु॒हवा॒ शर्म॑ यच्छतु॒ श्रव॑णं॒ चतु॑र्विꣳशतिश्च}]}%॥11॥

\prashnaend{अग्ने॑ तेजस्विन्वा॒युर्वस॑वस्त्वै॒तद्वा अ॒पां वा॒युर॑सि प्रा॒णो नाम॑ दे॒वा वै यद्य॒ज्ञेन॒ न प्र॒जा\-प॑तिर्देवासु॒राना॑यु॒र्दा ए॒तं युवा॑न॒ꣳ॒ सूर्यो॑ दे॒व इ॒दं वा॒मेका॑\-दश॥११॥}{अग्ने॑ तेजस्विन्वा॒युर॑सि॒ छन्द॑सां वी॒र्यं॑ मा॒तरं॑ च॒ षट्त्रिꣳ॑शत्॥३६॥}{अग्ने॑ तेजस्विꣴश्चिकि॒तुषे॑ दधातु॥}%%३-३
{हरिः॑ ॐ}{॥कृष्ण-यजुर्वेदीय-तैत्तिरीय-संहितायां तृतीयकाण्डे तृतीयः प्रश्नः समाप्तः॥३-३॥}
%%% END PRASHNA
