\chapt{काण्डम् ५}
\sect{प्रथमः प्रश्नः}\setcounter{anuvakam}{0}
\dnsub{तैत्तिरीयसंहितायां पञ्चमकाण्डे प्रथमः प्रश्नः}
%5.1.1.1
सा॒वि॒त्राणि॑ जुहोति॒ प्रसू᳚त्यै चतुर्गृही॒तेन॑ जुहोति॒ चतु॑ष्पादः प॒शवः॑ प॒शूने॒वाव॑ रुन्धे॒ चत॑स्रो॒ दिशो॑ दि॒क्ष्वे॑व प्रति॑ तिष्ठति॒ छन्दाꣳ॑सि दे॒वेभ्यो\-ऽपा᳚क्राम॒न्न वो॑\-ऽभा॒गानि॑ ह॒व्यं व॑क्ष्याम॒ इति॒ तेभ्य॑ ए॒तच्च॑तुर्गृही॒तम॑धारयन् पुरोनुवा॒क्या॑यै या॒ज्या॑यै दे॒वता॑यै वषट्का॒राय॒ यच्च॑तुर्गृही॒तं जु॒होति॒ छन्दाꣴ॑स्ये॒व तत्प्री॑णाति॒ तान्य॑स्य प्री॒तानि॑ दे॒वेभ्यो॑ ह॒व्यं व॑हन्ति॒ यं का॒मये॑त~(१)

%5.1.1.2
पापी॑यान्थ्स्या॒दित्येकै॑कं॒ तस्य॑ जुहुया॒दाहु॑तीभिरे॒वैन॒मप॑ गृह्णाति॒ पापी॑यान्भवति॒ यं का॒मये॑त॒ वसी॑यान्थ्स्या॒दिति॒ सर्वा॑णि॒ तस्या॑नु॒द्रुत्य॑ जुहुया॒दाहु॑त्यै॒वैन॑म॒भि क्र॑मयति॒ वसी॑यान्भव॒त्यथो॑ य॒ज्ञस्यै॒वैषाभिक्रा᳚न्ति॒रेति॒ वा ए॒ष य॑ज्ञमु॒खादृद्ध्या॒ यो᳚\-ऽग्नेर्दे॒वता॑या॒ एत्य॒ष्टावे॒तानि॑ सावि॒त्राणि॑ भवन्त्य॒ष्टाक्ष॑रा गाय॒त्री गा॑य॒त्रः~(२)

%5.1.1.3
अ॒ग्निस्तेनै॒व य॑ज्ञमु॒खादृद्ध्या॑ अ॒ग्नेर्दे॒वता॑यै॒ नैत्य॒ष्टौ सा॑वि॒त्राणि॑ भव॒न्त्याहु॑तिर्नव॒मी त्रि॒वृत॑मे॒व य॑ज्ञमु॒खे वि या॑तयति॒ यदि॑ का॒मये॑त॒ छन्दाꣳ॑सि यज्ञयश॒सेना᳚र्पयेय॒मित्यृच॑मन्त॒मां कु॑र्या॒च्छन्दाꣴ॑स्ये॒व य॑ज्ञयश॒सेना᳚र्पयति॒ यदि॑ का॒मये॑त॒ यज॑मानं यज्ञयश॒सेना᳚र्पयेय॒मिति॒ यजु॑रन्त॒मं कु॑र्या॒द्यज॑मानमे॒व य॑ज्ञयश॒सेना᳚र्पयत्यृ॒चा स्तोम॒ꣳ॒ सम॑र्ध॒येति॑~(३)

%5.1.1.4
आ॒ह॒ समृ॑द्ध्यै च॒तुर्भि॒रभ्रि॒मा द॑त्ते च॒त्वारि॒ छन्दाꣳ॑सि॒ छन्दो॑भिरे॒व दे॒वस्य॑ त्वा सवि॒तुः प्र॑स॒व इत्या॑ह॒ प्रसू᳚त्या अ॒ग्निर्दे॒वेभ्यो॒ निला॑यत॒ स वेणुं॒ प्रावि॑श॒थ्स ए॒तामू॒तिमनु॒ सम॑चर॒द्यद्वेणोः᳚ सुषि॒रꣳ सु॑षि॒राभ्रि॑र्भवति सयोनि॒त्वाय॒ स यत्र॑य॒त्राव॑स॒त्तत्कृ॒ष्णम॑भवत्कल्मा॒षी भ॑वति रू॒पस॑मृद्ध्या उभयतः॒क्ष्णूर्भ॑वती॒तश्चा॒मुत॑श्चा॒र्कस्याव॑रुद्ध्यै व्याममा॒त्री भ॑वत्ये॒ताव॒द्वै पुरु॑षे वी॒र्यं॑ वी॒र्य॑सम्मि॒ता\-ऽप॑रिमिता भव॒त्यप॑रिमित॒स्याव॑रुद्ध्यै॒ यो वन॒स्पती॑नाम्फल॒ग्रहिः॒ स ए॑षां वी॒र्या॑वान्फल॒ग्रहि॒र्वेणु॑र्वैण॒वी भ॑वति वी॒र्य॑स्याव॑रुद्ध्यै॥~(४)

{\anuvakamend[{का॒मये॑त गाय॒त्रो᳚\-ऽर्ध॒येति॑ च स॒प्तविꣳ॑शतिश्च}]}%~(१)

%5.1.2.1
व्यृ॑द्धं॒ वा ए॒तद्य॒ज्ञस्य॒ यद॑य॒जुष्के॑ण क्रि॒यत॑ इ॒माम॑गृभ्णन्रश॒नामृ॒तस्येत्य॑श्वाभि॒धानी॒मा द॑त्ते॒ यजु॑ष्कृत्यै य॒ज्ञस्य॒ समृ॑द्ध्यै॒ प्रतू᳚र्तं वाजि॒न्ना द्र॒वेत्यश्व॑म॒भि द॑धाति रू॒पमे॒वास्यै॒तन्म॑हि॒मानं॒ व्याच॑ष्टे यु॒ञ्जाथा॒ꣳ॒ रास॑भं यु॒वमिति॑ गर्द॒भमस॑त्ये॒व ग॑र्द॒भं प्रति॑\-ष्ठापयति॒ तस्मा॒दश्वा᳚द्गर्द॒भो\-ऽस॑त्तरो॒ योगे॑योगे त॒वस्त॑र॒मित्या॑ह~(५)

%5.1.2.2
योगे॑योग ए॒वैनं॑ युङ्क्ते॒ वाजे॑वाजे हवामह॒ इत्या॒हान्नं॒ वै वाजो\-ऽन्न॑मे॒वाव॑ रुन्धे॒ सखा॑य॒ इन्द्र॑मू॒तय॒ इत्या॑हेन्द्रि॒यमे॒वाव॑ रुन्धे॒\-ऽग्निर्दे॒वेभ्यो॒ निला॑यत॒ तं प्र॒जा\-प॑ति॒रन्व॑विन्दत्प्राजाप॒त्यो\-ऽश्वो\-ऽश्वे॑न॒ सम्भ॑र॒त्यनु॑वित्त्यै पापवस्य॒सं वा ए॒तत्क्रि॑यते॒ यच्छ्रेय॑सा च॒ पापी॑यसा च समा॒नं कर्म॑ कु॒र्वन्ति॒ पापी॑यान्~(६)

%5.1.2.3
ह्यश्वा᳚द्गर्द॒भो\-ऽश्वं॒ पूर्वं॑ नयन्ति पापवस्य॒सस्य॒ व्यावृ॑त्त्यै॒ तस्मा॒च्छ्रेयाꣳ॑सं॒ पापी॑यान्प॒श्चादन्वे॑ति ब॒हुर्वै भव॑तो॒ भ्रातृ॑व्यो॒ भव॑तीव॒ खलु॒ वा ए॒ष यो᳚\-ऽग्निं चि॑नु॒ते व॒ज्र्यश्वः॑ प्र॒तूर्व॒न्नेह्य॑व॒क्राम॒न्नश॑स्ती॒रित्या॑ह॒ वज्रे॑णै॒व पा॒प्मान॒म्भ्रातृ॑व्य॒मव॑ क्रामति रु॒द्रस्य॒ गाण॑पत्या॒दित्या॑ह रौ॒द्रा वै प॒शवो॑ रु॒द्रादे॒व~(७)

%5.1.2.4
प॒शून्नि॒र्याच्या॒\-ऽऽ\-त्मने॒ कर्म॑ कुरुते पू॒ष्णा स॒युजा॑ स॒हेत्या॑ह पू॒षा वा अध्व॑नाꣳ सन्ने॒ता सम॑ष्ट्यै॒ पुरी॑षायतनो॒ वा ए॒ष यद॒ग्निरङ्गि॑रसो॒ वा ए॒तमग्रे॑ दे॒वता॑ना॒ꣳ॒ सम॑भरन्पृथि॒व्याः स॒धस्था॑द॒ग्निं पु॑री॒ष्य॑मङ्गिर॒\-स्वदच्छे॒हीत्या॑ह॒ साय॑तनमे॒वैनं॑ दे॒वता॑भिः॒ सम्भ॑रत्य॒ग्निं पु॑री॒ष्य॑मङ्गिर॒\-स्वदच्छे॑म॒ इत्या॑ह येन॑~(८)

%5.1.2.5
सं॒गच्छ॑ते॒ वाज॑मे॒वास्य॑ वृङ्क्ते प्र॒जा\-प॑तये प्रति॒प्रोच्या॒ग्निः स॒म्भृत्य॒ इत्या॑हुरि॒यं वै प्र॒जा\-प॑ति॒स्तस्या॑ ए॒तच्छ्रोत्रं॒ यद्व॒ल्मीको॒\-ऽग्निं पु॑री॒ष्य॑मङ्गिर॒\-स्वद्भ॑रिष्याम॒ इति॑ वल्मीकव॒पामुप॑ तिष्ठते सा॒क्षादे॒व प्र॒जा\-प॑तये प्रति॒प्रोच्या॒ग्निꣳ सम्भ॑रत्य॒ग्निं पु॑री॒ष्य॑मङ्गिर॒\-स्वद्भ॑राम॒ इत्या॑ह॒ येन॑ सं॒गच्छ॑ते॒ वाज॑मे॒वास्य॑ वृ॒ङ्क्ते\-ऽन्व॒ग्निरु॒षसा॒मग्रम्᳚~(९)

%5.1.2.6
अ॒ख्य॒दित्या॒हानु॑ख्यात्या आ॒गत्य॑ वा॒ज्यध्व॑न आ॒क्रम्य॑ वाजिन्पृथि॒वीमित्या॑हे॒च्छत्ये॒वैनं॒ पूर्व॑या वि॒न्दत्युत्त॑रया॒ द्वाभ्या॒मा क्र॑मयति॒ प्रति॑ष्ठित्या॒ अनु॑रूपाभ्या॒न्तस्मा॒दनु॑रूपाः प॒शवः॒ प्र जा॑यन्ते॒ द्यौस्ते॑ पृ॒ष्ठं पृ॑थि॒वी स॒धस्थ॒मित्या॑है॒भ्यो वा ए॒तं लो॒केभ्यः॑ प्र॒जा\-प॑तिः॒ समै॑रयद्रू॒पमे॒वास्यै॒तन्म॑हि॒मानं॒ व्याच॑ष्टे व॒ज्री वा ए॒ष यदश्वो॑ द॒द्भिर॒न्यतो॑दद्भ्यो॒ भूया॒ल्लोँम॑भिरुभ॒याद॑द्भ्यो॒ यं द्वि॒ष्यात्तम॑धस्प॒दं ध्या॑ये॒द्वज्रे॑णै॒वैनꣴ॑ स्तृणुते॥10॥

{\anuvakamend[{आ॒ह॒ पापी॑यान्रु॒द्रादे॒व येनाग्रं॑ व॒ज्री वै स॒प्तद॑श च}]}%~(२)

%5.1.3.1
उत्क्रा॒मोद॑क्रमी॒दिति॒ द्वाभ्या॒मुत्क्र॑मयति॒ प्रति॑ष्ठित्या॒ अनु॑रूपाभ्या॒न्तस्मा॒दनु॑रूपाः प॒शवः॒ प्र जा॑यन्ते॒\-ऽप उप॑ सृजति॒ यत्र॒ वा आप॑ उप॒गच्छ॑न्ति॒ तदोष॑धयः॒ प्रति॑ तिष्ठ॒न्त्योष॑धीः प्रति॒तिष्ठ॑न्तीः प॒शवो\-ऽनु॒ प्रति॑ तिष्ठन्ति प॒शून् य॒ज्ञो य॒ज्ञं यज॑मानो॒ यज॑मानं प्र॒जास्तस्मा॑द॒प उप॑ सृजति॒ प्रति॑ष्ठित्यै॒ यद॑ध्व॒र्युर॑न॒ग्नावाहु॑तिं जुहु॒याद॒न्धो᳚\-ऽध्व॒र्युः~(११)

%5.1.3.2
स्या॒द्रक्षाꣳ॑सि य॒ज्ञꣳ ह॑न्यु॒र्॒\mbox{}हिर॑ण्यमु॒पास्य॑ जुहोत्यग्नि॒वत्ये॒व जु॑होति॒ नान्धो᳚\-ऽध्व॒र्युर्भव॑ति॒ न य॒ज्ञꣳ रक्षाꣳ॑सि घ्नन्ति॒ जिघ॑र्म्य॒ग्निं मन॑सा घृ॒तेनेत्या॑ह॒ मन॑सा॒ हि पुरु॑षो य॒ज्ञम॑भि॒गच्छ॑ति प्रति॒क्ष्यन्तं॒ भुव॑नानि॒ विश्वेत्या॑ह॒ सर्व॒ꣴ॒ ह्ये॑ष प्र॒त्यङ्क्षेति॑ पृ॒थुं ति॑र॒श्चा वय॑सा बृ॒हन्त॒मित्या॒हाल्पो॒ ह्ये॑ष जा॒तो म॒हान्~(१२)

%5.1.3.3
भव॑ति॒ व्यचि॑ष्ठ॒मन्नꣳ॑ रभ॒सं विदा॑न॒मित्या॒हान्न॑मे॒वास्मै᳚ स्वदयति॒ सर्व॑मस्मै स्वदते॒ य ए॒वं वेदा त्वा॑ जिघर्मि॒ वच॑सा घृ॒तेनेत्या॑ह॒ तस्मा॒द्यत्पुरु॑षो॒ मन॑साभि॒गच्छ॑ति॒ तद्वा॒चा व॑दत्यर॒क्षसेत्या॑ह॒ रक्ष॑सा॒मप॑हत्यै॒ मर्य॑श्रीः स्पृह॒यद्व॑र्णो अ॒ग्निरित्या॒हाप॑चितिमे॒वास्मि॑न्दधा॒त्यप॑चितिमान्भवति॒ य ए॒वं~(१३)

%5.1.3.4
वेद॒ मन॑सा॒ त्वै तामाप्तु॑मर्\mbox{}हति॒ याम॑ध्व॒र्युर॑न॒ग्नावाहु॑तिं जु॒होति॒ मन॑स्वतीभ्यां जुहो॒त्याहु॑त्यो॒राप्त्यै॒ द्वाभ्यां॒ प्रति॑ष्ठित्यै यज्ञमु॒खेय॑ज्ञमुखे॒ वै क्रि॒यमा॑णे य॒ज्ञꣳ रक्षाꣳ॑सि जिघाꣳसन्त्ये॒तर्\mbox{}हि॒ खलु॒ वा ए॒तद्य॑ज्ञमु॒खं यर्\mbox{}ह्ये॑न॒दाहु॑तिरश्ञु॒ते परि॑ लिखति॒ रक्ष॑सा॒मप॑हत्यै ति॒सृभिः॒ परि॑ लिखति त्रि॒वृद्वा अ॒ग्निर्यावा॑ने॒वाग्निस्तस्मा॒द्रक्षा॒ꣴ॒स्यप॑ हन्ति~(१४)

%5.1.3.5
गा॒य॒त्रि॒या परि॑ लिखति॒ तेजो॒ वै गा॑य॒त्री तेज॑सै॒वैनं॒ परि॑ गृह्णाति त्रि॒ष्टुभा॒ परि॑ लिखतीन्द्रि॒यं वै त्रि॒ष्टुगि॑न्द्रि॒येणै॒वैनं॒ परि॑ गृह्णात्यनु॒ष्टुभा॒ परि॑ लिखत्यनु॒ष्टुफ्सर्वा॑णि॒ छन्दाꣳ॑सि परि॒भूः पर्या᳚प्त्यै मध्य॒तो॑\-ऽनु॒ष्टुभा॒ वाग्वा अ॑नु॒ष्टुप्तस्मा᳚न्मध्य॒तो वा॒चा व॑दामो गायत्रि॒या प्र॑थ॒मया॒ परि॑ लिख॒त्यथा॑नु॒ष्टुभाथ॑ त्रि॒ष्टुभा॒ तेजो॒ वै गा॑य॒त्री य॒ज्ञो॑\-ऽनु॒ष्टुगि॑न्द्रि॒यं त्रि॒ष्टुप्तेज॑सा चै॒वेन्द्रि॒येण॑ चोभ॒यतो॑ य॒ज्ञं परि॑ गृह्णाति॥~(१५)

{\anuvakamend[{अ॒न्धो᳚\-ऽध्व॒र्युर्म॒हान्भ॑वति त्रि॒ष्टुभा॒ तेजो॒ वै गा॑य॒त्री त्रयो॑दश च}]}%~(३)

%5.1.4.1
दे॒वस्य॑ त्वा सवि॒तुः प्र॑स॒व इति॑ खनति॒ प्रसू᳚त्या॒ अथो॑ धू॒ममे॒वैतेन॑ जनयति॒ ज्योति॑ष्मन्तं त्वाऽग्ने सु॒प्रती॑क॒मित्या॑ह॒ ज्योति॑रे॒वैतेन॑ जनयति॒ सो᳚\-ऽग्निर्जा॒तः प्र॒जाः शु॒चार्प॑य॒त्तं दे॒वा अ॑र्ध॒र्चेना॑शमयञ्छि॒वं प्र॒जाभ्यो\-ऽहिꣳ॑सन्त॒मित्या॑ह प्र॒जाभ्य॑ ए॒वैनꣳ॑ शमयति॒ द्वा\-भ्यां᳚ खनति॒ प्रति॑ष्ठित्या अ॒पां पृ॒ष्ठम॒सीति॑ पुष्करप॒र्णमा~(१६)

%5.1.4.2
ह॒र॒त्य॒पां वा ए॒तत्पृ॒ष्ठं यत्पु॑ष्करप॒र्णꣳ रू॒पेणै॒वैन॒दा ह॑रति पुष्करप॒र्णेन॒ सम्भ॑रति॒ योनि॒र्वा अ॒ग्नेः पु॑ष्करप॒र्णꣳ सयो॑निमे॒वाग्निꣳ सम्भ॑रति कृष्णाजि॒नेन॒ सम्भ॑रति य॒ज्ञो वै कृ॑ष्णाजि॒नं य॒ज्ञेनै॒व य॒ज्ञꣳ सम्भ॑रति॒ यद्ग्रा॒म्याणां᳚ पशू॒नां चर्म॑णा स॒म्भरे᳚द्ग्रा॒म्यान्प॒शूञ्छु॒चार्प॑येत्कृष्णाजि॒नेन॒ सम्भ॑रत्यार॒ण्याने॒व प॒शून्~(१७)

%5.1.4.3
शु॒चार्प॑यति॒ तस्मा᳚थ्स॒माव॑त्पशू॒नां प्र॒जाय॑मानानामार॒ण्याः प॒शवः॒ कनी॑याꣳसः शु॒चा ह्यृ॑ता लो॑म॒तः सम्भ॑र॒त्यतो॒ ह्य॑स्य॒ मेध्यं॑ कृष्णाजि॒नं च॑ पुष्करप॒र्णं च॒ सꣴ स्तृ॑णाती॒यं वै कृ॑ष्णाजि॒नम॒सौ पु॑ष्करप॒र्णमा॒भ्यामे॒वैन॑मुभ॒यतः॒ परि॑ गृह्णात्य॒ग्निर्दे॒वेभ्यो॒ निला॑यत॒ तमथ॒र्वान्व॑पश्य॒दथ॑र्वा त्वा प्रथ॒मो निर॑मन्थदग्न॒ इति॑~(१८)

%5.1.4.4
आ॒ह॒ य ए॒वैन॑म॒न्वप॑श्य॒त्तेनै॒वैन॒ꣳ॒ सम्भ॑रति॒ त्वाम॑ग्ने॒ पुष्क॑रा॒दधीत्या॑ह पुष्करप॒र्णे ह्ये॑न॒मुप॑श्रित॒मवि॑न्द॒त्तमु॑ त्वा द॒ध्यङ्ङृषि॒रित्या॑ह द॒ध्यङ्वा आ॑थर्व॒णस्ते॑ज॒स्व्या॑सी॒त्तेज॑ ए॒वास्मि॑न्दधाति॒ तमु॑ त्वा पा॒थ्यो वृषेत्या॑ह॒ पूर्व॑मे॒वोदि॒तमुत्त॑रेणा॒भि गृ॑णाति~(१९)

%5.1.4.5
च॒त॒सृभिः॒ सम्भ॑रति च॒त्वारि॒ छन्दाꣳ॑सि॒ छन्दो॑भिरे॒व गा॑य॒त्रीभि॑र्ब्राह्म॒णस्य॑ गाय॒त्रो हि ब्रा᳚ह्म॒णस्त्रि॒ष्टुग्भी॑ राज॒न्य॑स्य॒ त्रैष्टु॑भो॒ हि रा॑ज॒न्यो॑ यं का॒मये॑त॒ वसी॑यान्थ्स्या॒दित्यु॒भयी॑भि॒स्तस्य॒ सम्भ॑रे॒त्तेज॑श्चै॒वास्मा॑ इन्द्रि॒यं च॑ स॒मीची॑ दधात्यष्टा॒भिः सम्भ॑रत्य॒ष्टाक्ष॑रा गाय॒त्री गा॑य॒त्रो᳚\-ऽग्निर्यावा॑ने॒वाग्निस्तꣳ सम्भ॑रति॒ सीद॑ होत॒रित्या॑ह दे॒वता॑ ए॒वास्मै॒ सꣳ सा॑दयति॒ नि होतेति॑ मनु॒ष्या᳚न्थ्सꣳ सी॑द॒स्वेति॒ वयाꣳ॑सि॒ जनि॑ष्वा॒ हि जेन्यो॒ अग्रे॒ अह्ना॒मित्या॑ह देवमनु॒ष्याने॒वास्मै॒ सꣳस॑न्ना॒न्प्र ज॑नयति॥~(२०)

{\anuvakamend[{ऐव प॒शूनिति॑ गृणाति होत॒रिति॑ स॒प्तविꣳ॑शतिश्च}]}%~(४)

%5.1.5.1
क्रू॒रमि॑व॒ वा अ॑स्या ए॒तत्क॑रोति॒ यत्खन॑त्य॒प उप॑ सृज॒त्यापो॒ वै शा॒न्ताः शा॒न्ताभि॑रे॒वास्यै॒ शुचꣳ॑ शमयति॒ सं ते॑ वा॒युर्मा॑त॒रिश्वा॑ दधा॒त्वित्या॑ह प्रा॒णो वै वा॒युः प्रा॒णेनै॒वास्यै᳚ प्रा॒णꣳ सं द॑धाति॒ सं ते॑ वा॒युरित्या॑ह॒ तस्मा᳚द्वा॒युप्र॑च्युता दि॒वो वृ॑ष्टिरीर्ते॒ तस्मै॑ च देवि॒ वष॑डस्तु~(२१)

%5.1.5.2
तुभ्य॒मित्या॑ह॒ षड्वा ऋ॒तव॑ ऋ॒तुष्वे॒व वृष्टिं॑ दधाति॒ तस्मा॒थ्सर्वा॑नृ॒तून् व॑र्\mbox{}षति॒ यद्व॑षट्कु॒र्याद्रक्षाꣳ॑सि य॒ज्ञꣳ ह॑न्यु॒र्वडित्या॑ह प॒रोक्ष॑मे॒व वष॑ट्करोति॒ नास्य॑ या॒तया॑मा वषट्का॒रो भव॑ति॒ न य॒ज्ञꣳ रक्षाꣳ॑सि घ्नन्ति॒ सुजा॑तो॒ ज्योति॑षा स॒हेत्य॑नु॒ष्टुभोप॑ नह्यत्यनु॒ष्टुप्~(२२)

%5.1.5.3
सर्वा॑णि॒ छन्दाꣳ॑सि॒ छन्दाꣳ॑सि॒ खलु॒ वा अ॒ग्नेः प्रि॒या त॒नूः प्रि॒ययै॒वैनं॑ त॒नुवा॒ परि॑ दधाति॒ वेदु॑को॒ वासो॑ भवति॒ य ए॒वं वेद॑ वारु॒णो वा अ॒ग्निरुप॑नद्ध॒ उदु॑ तिष्ठ स्वध्वरो॒र्ध्व ऊ॒ षु ण॑ ऊ॒तय॒ इति॑ सावि॒त्रीभ्या॒मुत्ति॑ष्ठति सवि॒तृप्र॑सूत ए॒वास्यो॒र्ध्वां व॑रुणमे॒निमुथ्सृ॑जति॒ द्वाभ्यां॒ प्रति॑ष्ठित्यै॒ स जा॒तो गर्भो॑ असि~(२३)

%5.1.5.4
रोद॑स्यो॒रित्या॑हे॒मे वै रोद॑सी॒ तयो॑रे॒ष गर्भो॒ यद॒ग्निस्तस्मा॑दे॒वमा॒हाग्ने॒ चारु॒र्विभृ॑त॒ ओष॑धी॒ष्वित्या॑ह य॒दा ह्ये॑तं वि॒भर॒न्त्यथ॒ चारु॑तरो॒ भव॑ति॒ प्र मा॒तृभ्यो॒ अधि॒ कनि॑क्रदद्गा॒ इत्या॒हौष॑धयो॒ वा अ॑स्य मा॒तर॒स्ताभ्य॑ ए॒वैनं॒ प्र च्या॑वयति स्थि॒रो भ॑व वी॒ड्व॑ङ्ग॒ इति॑ गर्द॒भ आ सा॑दयति~(२४)

%5.1.5.5
सं न॑ह्यत्ये॒वैन॑मे॒तया᳚ स्थे॒म्ने ग॑र्द॒भेन॒ सम्भ॑रति॒ तस्मा᳚द्गर्द॒भः प॑शू॒नां भा॑रभा॒रित॑मो गर्द॒भेन॒ सम्भ॑रति॒ तस्मा᳚द्गर्द॒भो\-ऽप्य॑नाले॒शेत्य॒न्यान्प॒शून्मे᳚द्य॒त्यन्न॒ꣴ॒ ह्ये॑नेना॒र्कꣳ स॒म्भर॑न्ति गर्द॒भेन॒ सम्भ॑रति॒ तस्मा᳚द्गर्द॒भो द्वि॒रेताः॒ सन्कनि॑ष्ठं पशू॒नां प्र जा॑यते॒\-ऽग्निर्ह्य॑स्य॒ योनिं॑ नि॒र्दह॑ति प्र॒जासु॒ वा ए॒ष ए॒तर्\mbox{}ह्यारू॑ढः~(२५)

%5.1.5.6
स ई᳚श्व॒रः प्र॒जाः शु॒चा प्र॒दहः॑ शि॒वो भ॑व प्र॒जाभ्य॒ इत्या॑ह प्र॒जाभ्य॑ ए॒वैनꣳ॑ शमयति॒ मानु॑षीभ्य॒स्त्वम॑ङ्गिर॒ इत्या॑ह मान॒व्यो॑ हि प्र॒जा मा द्यावा॑पृथि॒वी अ॒भि शू॑शुचो॒ माऽन्तरि॑क्षं॒ मा वन॒स्पती॒नित्या॑है॒भ्य ए॒वैनं॑ लो॒केभ्यः॑ शमयति॒ प्रैतु॑ वा॒जी कनि॑क्रद॒दित्या॑ह वा॒जी ह्ये॑ष नान॑द॒द्रास॑भः॒ पत्वेति॑~(२६)

%5.1.5.7
आ॒ह॒ रास॑भ॒ इति॒ ह्ये॑तमृष॒यो\-ऽव॑द॒न्भर॑न्न॒ग्निं पु॑री॒ष्य॑मित्या॑हा॒ग्निꣴ ह्ये॑ष भर॑ति॒ मा पा॒द्यायु॑षः पु॒रेत्या॒हा\-ऽऽ\-यु॑रे॒वास्मि॑न्दधाति॒ तस्मा᳚द्गर्द॒भः सर्व॒मायु॑रेति॒ तस्मा᳚द्गर्द॒भे पु॒रा\-ऽऽ\-यु॑षः॒ प्रमी॑ते बिभ्यति॒ वृषा॒ग्निं वृष॑ण॒म्भर॒न्नित्या॑ह वृषा॒ ह्ये॑ष वृषा॒ग्निर॒पां गर्भम्᳚~(२७)

%5.1.5.8
स॒मु॒द्रिय॒मित्या॑हा॒पाꣳ ह्ये॑ष गर्भो॒ यद॒ग्निरग्न॒ आ या॑हि वी॒तय॒ इति॒ वा इ॒मौ लो॒कौ व्यै॑ता॒मग्न॒ आ या॑हि वी॒तय॒ इति॒ यदाहा॒नयो᳚र्लो॒कयो॒र्वीत्यै॒ प्रच्यु॑तो॒ वा ए॒ष आ॒यत॑ना॒दग॑तः प्रति॒ष्ठाꣳ स ए॒तर्\mbox{}ह्य॑ध्व॒र्युं च॒ यज॑मानं च ध्यायत्यृ॒तꣳ स॒त्यमित्या॑हे॒यं वा ऋ॒तम॒सौ~(२८)

%5.1.5.9
स॒त्यम॒नयो॑रे॒वैनं॒ प्रति॑\-ष्ठापयति॒ नार्ति॒मार्च्छ॑त्यध्व॒र्युर्न यज॑मानो॒ वरु॑णो॒ वा ए॒ष यज॑मानम॒भ्यैति॒ यद॒ग्निरुप॑नद्ध॒ ओष॑धयः॒ प्रति॑ गृह्णीता॒ग्निमे॒तमित्या॑ह॒ शान्त्यै॒ व्यस्य॒न्विश्वा॒ अम॑ती॒ररा॑ती॒रित्या॑ह॒ रक्ष॑सा॒मप॑हत्यै नि॒षीद॑न्नो॒ अप॑ दुर्म॒तिꣳ ह॑न॒दित्या॑ह॒ प्रति॑ष्ठित्या॒ ओष॑धयः॒ प्रति॑ मोदध्वम्॥~(२९)

%5.1.5.10
{\anuvakamend[{अ॒स्त्व॒नु॒ष्टुब॑सि सादय॒त्यारू॑ढः॒ पत्वेति॒ गर्भ॑म॒सौ मो॑दध्वं॒ द्विच॑त्वारिꣳशच्च}]}%~(५)

{{\anuvakamend[ए॒न॒मित्या॒हौष॑धयो॒ वा अ॒ग्नेर्भा॑ग॒धेयं॒ ताभि॑रे॒वैन॒ꣳ॒ सम॑र्धयति॒ पुष्पा॑वतीः सुपिप्प॒ला इत्या॑ह॒ तस्मा॒दोष॑धयः॒ फलं॑ गृह्णन्त्य॒यं वो॒ गर्भ॑ ऋ॒त्वियः॑ प्र॒त्नꣳ स॒धस्थ॒मास॑द॒दित्या॑ह॒ याभ्य॑ ए॒वैनं॑ प्रच्या॒वय॑ति॒ तास्वे॒वैनं॒ प्रति॑\-ष्ठापयति॒ द्वाभ्या॑मु॒पाव॑हरति॒ प्रति॑ष्ठित्यै~(~०)]}}

%5.1.6.1
वा॒रु॒णो वा अ॒ग्निरुप॑नद्धो॒ वि पाज॒सेति॒ वि स्रꣳ॑सयति सवि॒तृप्र॑सूत ए॒वास्य॒ विषू॑चीं वरुणमे॒निं वि सृ॑जत्य॒प उप॑ सृज॒त्यापो॒ वै शा॒न्ताः शा॒न्ताभि॑रे॒वास्य॒ शुचꣳ॑ शमयति ति॒सृभि॒रुप॑ सृजति त्रि॒वृद्वा अ॒ग्निर्यावा॑ने॒वाग्निस्तस्य॒ शुचꣳ॑ शमयति मि॒त्रः स॒ꣳ॒सृज्य॑ पृथि॒वीमित्या॑ह मि॒त्रो वै शि॒वो दे॒वाना॒न्तेनै॒व~(३१)

%5.1.6.2
ए॒न॒ꣳ॒ सꣳ सृ॑जति॒ शान्त्यै॒ यद्ग्रा॒म्याणां॒ पात्रा॑णां क॒पालैः᳚ सꣳसृ॒जेद्ग्रा॒म्याणि॒ पात्रा॑णि शु॒चार्प॑येदर्मकपा॒लैः सꣳ सृ॑जत्ये॒तानि॒ वा अ॑नुपजीवनी॒यानि॒ तान्ये॒व शु॒चार्प॑यति॒ शर्क॑राभिः॒ सꣳ सृ॑जति॒ धृत्या॒ अथो॑ शं॒त्वाया॑जलो॒मैः सꣳ सृ॑जत्ये॒षा वा अ॒ग्नेः प्रि॒या त॒नूर्यद॒जा प्रि॒ययै॒वैनं॑ त॒नुवा॒ सꣳ सृ॑ज॒त्यथो॒ तेज॑सा कृष्णाजि॒नस्य॒ लोम॑भिः॒ सम्~(३२)

%5.1.6.3
सृ॒ज॒ति॒ य॒ज्ञो वै कृ॑ष्णाजि॒नं य॒ज्ञेनै॒व य॒ज्ञꣳ सꣳ सृ॑जति रु॒द्राः स॒म्भृत्य॑ पृथि॒वीमित्या॑है॒ता वा ए॒तं दे॒वता॒ अग्रे॒ सम॑भरं॒ ताभि॑रे॒वैन॒ꣳ॒ सम्भ॑रति म॒खस्य॒ शिरो॒\-ऽसीत्या॑ह य॒ज्ञो वै म॒खस्तस्यै॒तच्छिरो॒ यदु॒खा तस्मा॑दे॒वमा॑ह य॒ज्ञस्य॑ प॒दे स्थ॒ इत्या॑ह य॒ज्ञस्य॒ ह्ये॑ते~(३३)

%5.1.6.4
प॒दे अथो॒ प्रति॑ष्ठित्यै॒ प्रान्याभि॒र्यच्छ॒त्यन्व॒न्यैर्म॑न्त्रयते मिथुन॒त्वाय॒ त्र्यु॑द्धिं करोति॒ त्रय॑ इ॒मे लो॒का ए॒षां लो॒काना॒माप्त्यै॒ छन्दो॑भिः करोति वी॒र्यं॑ वै छन्दाꣳ॑सि वी॒र्ये॑णै॒वैनां᳚ करोति॒ यजु॑षा॒ बिलं॑ करोति॒ व्यावृ॑त्त्या॒ इय॑तीं करोति प्र॒जा\-प॑तिना यज्ञमु॒खेन॒ सम्मि॑तां द्विस्त॒नां क॑रोति॒ द्यावा॑पृथि॒व्योर्दोहा॑य॒ चतुः॑ स्तनां करोति पशू॒नां दोहा॑या॒ष्टास्त॑नां करोति॒ छन्द॑सां॒ दोहा॑य॒ नवा᳚श्रिमभि॒चर॑तः कुर्यात्त्रि॒वृत॑मे॒व वज्रꣳ॑ स॒म्भृत्य॒ भ्रातृ॑व्याय॒ प्र ह॑रति॒ स्तृत्यै॑ कृ॒त्वाय॒ सा म॒हीमु॒खामिति॒ नि द॑धाति दे॒वता᳚स्वे॒वैनां॒ प्रति॑\-ष्ठापयति॥~(३४)

{\anuvakamend[{तेनै॒व लोम॑भिः॒ समे॒ते अ॑भि॒चर॑त॒ एक॑विꣳशतिश्च}]}%~(६)

%5.1.7.1
स॒प्तभि॑र्धूपयति स॒प्त वै शी॑र्\mbox{}ष॒ण्याः᳚ प्रा॒णाः शिर॑ ए॒तद्य॒ज्ञस्य॒ यदु॒खा शी॒र्॒\mbox{}षन्ने॒व य॒ज्ञस्य॑ प्रा॒णान्द॑धाति॒ तस्मा᳚थ्स॒प्त शी॒र्॒\mbox{}षन्प्रा॒णा अ॑श्वश॒केन॑ धूपयति प्राजाप॒त्यो वा अश्वः॑ सयोनि॒त्वायादि॑ति॒स्त्वेत्या॑हे॒यं वा अदि॑ति॒रदि॑त्यै॒वादि॑त्यां खनत्य॒स्या अक्रू॑रङ्काराय॒ न हि स्वः स्वꣳ हि॒नस्ति॑ दे॒वानां᳚ त्वा॒ पत्नी॒रित्या॑ह दे॒वाना᳚म्~(३५)

%5.1.7.2
वा ए॒तां पत्न॒यो\-ऽग्रे॑\-ऽकुर्व॒न्ताभि॑रे॒वैनां᳚ दधाति धि॒षणा॒स्त्वेत्या॑ह वि॒द्या वै धि॒षणा॑ वि॒द्याभि॑रे॒वैना॑म॒भीन्द्धे॒ ग्नास्त्वेत्या॑ह॒ छन्दाꣳ॑सि॒ वै ग्नाश्छन्दो॑भिरे॒वैनाꣴ॑ श्रपयति॒ वरू᳚त्रय॒स्त्वेत्या॑ह॒ होत्रा॒ वै वरू᳚त्रयो॒ होत्रा॑भिरे॒वैनां᳚ पचति॒ जन॑य॒स्त्वेत्या॑ह दे॒वानां॒ वै पत्नीः᳚~(३६)

%5.1.7.3
जन॑य॒स्ताभि॑रे॒वैनां᳚ पचति ष॒ड्भिः प॑चति॒ षड्वा ऋ॒तव॑ ऋ॒तुभि॑रे॒वैनां᳚ पचति॒ द्विः पच॒न्त्वित्या॑ह॒ तस्मा॒द्द्विः सं॑वथ्स॒रस्य॑ स॒स्यं प॑च्यते वारु॒ण्यु॑खाभीद्धा॑ मै॒त्रियोपै॑ति॒ शान्त्यै॑ दे॒वस्त्वा॑ सवि॒तोद्व॑प॒त्वित्या॑ह सवि॒तृप्र॑सूत ए॒वैनां॒ ब्रह्म॑णा दे॒वता॑भि॒रुद्व॑प॒त्यप॑द्यमाना पृथि॒व्याशा॒ दिश॒ आ पृ॑ण~(३७)

%5.1.7.4
इत्या॑ह॒ तस्मा॑द॒ग्निः सर्वा॒ दिशो\-ऽनु॒ वि भा॒त्युत्ति॑ष्ठ बृह॒ती भ॑वो॒र्ध्वा ति॑ष्ठ ध्रु॒वा त्वमित्या॑ह॒ प्रति॑ष्ठित्या असु॒र्यं॑ पात्र॒\-मना᳚च्छृण्ण॒मा च्छृ॑णत्ति देव॒त्राक॑रजक्षी॒रेणा च्छृ॑णत्ति पर॒मं वा ए॒तत्पयो॒ यद॑जक्षी॒रं प॑र॒मेणै॒वैनां॒ पय॒सा च्छृ॑णत्ति॒ यजु॑षा॒ व्यावृ॑त्त्यै॒ छन्दो॑भि॒रा च्छृ॑णत्ति॒ छन्दो॑भि॒र्वा ए॒षा क्रि॑यते॒ छन्दो॑भिरे॒व छन्दा॒ꣴ॒स्या च्छृ॑णत्ति॥~(३८)

{\anuvakamend[{आ॒ह॒ दे॒वानां॒ वै पत्नीः᳚ पृणै॒षा षट्च॑}]}%~(७)

%5.1.8.1
एक॑विꣳशत्या॒ माषैः᳚ पुरुषशी॒र्॒\mbox{}षमच्छै᳚त्यमे॒ध्या वै माषा॑ अमे॒ध्यं पु॑रुषशी॒र्॒\mbox{}षम॑मे॒ध्यैरे॒वास्या॑मे॒ध्यं नि॑रव॒दाय॒ मेध्यं॑ कृ॒त्वा ह॑र॒त्येक॑विꣳशतिर्भवन्त्येकवि॒ꣳ॒शो वै पुरु॑षः॒ पुरु॑ष॒स्याप्त्यै॒ व्यृ॑द्धं॒ वा ए॒तत्प्रा॒णैर॑मे॒ध्यं यत्पु॑रुषशी॒र्॒\mbox{}षꣳ स॑प्त॒धा वितृ॑ण्णां वल्मीकव॒पां प्रति॒ नि द॑धाति स॒प्त वै शी॑र्\mbox{}ष॒ण्याः᳚ प्रा॒णाः प्रा॒णैरे॒वैन॒थ्सम॑र्धयति मेध्य॒त्वाय॒ याव॑न्तः~(३९)

%5.1.8.2
वै मृ॒त्युब॑न्धव॒स्तेषां᳚ य॒म आधि॑पत्यं॒ परी॑याय यमगा॒थाभिः॒ परि॑ गायति य॒मादे॒वैन॑द्वृङ्क्ते ति॒सृभिः॒ परि॑ गायति॒ त्रय॑ इ॒मे लो॒का ए॒भ्य ए॒वैन॑ल्लो॒केभ्यो॑ वृङ्क्ते॒ तस्मा॒द्गाय॑ते॒ न देय॒ङ्गाथा॒ हि तद्वृ॒ङ्क्ते᳚\-ऽग्निभ्यः॑ प॒शूना ल॑भते॒ कामा॒ वा अ॒ग्नयः॒ कामा॑ने॒वाव॑ रुन्धे॒ यत्प॒शून्नालभे॒तान॑वरुद्धा अस्य~(४०)

%5.1.8.3
प॒शवः॑ स्यु॒र्यत्पर्य॑ग्निकृतानुथ्सृ॒जेद्य॑ज्ञवेश॒सं कु॑र्या॒द्यथ्सꣴ॑स्था॒पये᳚द्या॒तया॑मानि शी॒र्॒\mbox{}षाणि॑ स्यु॒र्यत्प॒शूना॒लभ॑ते॒ तेनै॒व प॒शूनव॑ रुन्धे॒ यत्पर्य॑ग्निकृतानुथ्सृ॒जति॑ शी॒र्ष्णामया॑तयामत्वाय प्राजाप॒त्येन॒ सꣴ स्था॑पयति य॒ज्ञो वै प्र॒जा\-प॑तिर्य॒ज्ञ ए॒व य॒ज्ञं प्रति॑\-ष्ठापयति प्र॒जा\-प॑तिः प्र॒जा अ॑सृजत॒ स रि॑रिचा॒नो॑\-ऽमन्यत॒ स ए॒ता आ॒प्रीर॑पश्य॒त्ताभि॒र्वै स मु॑ख॒तः~(४१)

%5.1.8.4
आ॒त्मान॒माप्री॑णीत॒ यदे॒ता आ॒प्रियो॒ भव॑न्ति य॒ज्ञो वै प्र॒जा\-प॑तिर्य॒ज्ञमे॒वैताभि॑र्मुख॒त आ प्री॑णा॒त्यप॑रिमितछन्दसो भव॒न्त्यप॑रिमितः प्र॒जा\-प॑तिः प्र॒जा\-प॑ते॒राप्त्या॑ ऊनातिरि॒क्ता मि॑थु॒नाः प्रजा᳚त्यै लोम॒शं वै नामै॒तच्छन्दः॑ प्र॒जा\-प॑तेः प॒शवो॑ लोम॒शाः प॒शूने॒वाव॑ रुन्धे॒ सर्वा॑णि॒ वा ए॒ता रू॒पाणि॒ सर्वा॑णि रू॒पाण्य॒ग्नौ चित्ये᳚ क्रियन्ते॒ तस्मा॑दे॒ता अ॒ग्नेश्चित्य॑स्य~(४२)

%5.1.8.5
भ॒व॒न्त्येक॑विꣳशतिꣳ सामिधे॒नीरन्वा॑ह॒ रुग्वा ए॑कवि॒ꣳ॒शो रुच॑मे॒व ग॑च्छ॒त्यथो᳚ प्रति॒ष्ठामे॒व प्र॑ति॒ष्ठा ह्ये॑कवि॒ꣳ॒शश्चतु॑र्विꣳशति॒मन्वा॑ह॒ चतु॑र्विꣳशतिरर्धमा॒साः सं॑वथ्स॒रः सं॑वथ्स॒रो᳚\-ऽग्निर्वै᳚श्वान॒रः सा॒क्षादे॒व वै᳚श्वान॒रमव॑ रुन्धे॒ परा॑ची॒रन्वा॑ह॒ परा॑ङिव॒ हि सु॑व॒र्गो लो॒कः समा᳚स्त्वाग्न ऋ॒तवो॑ वर्धय॒न्त्वित्या॑ह॒ समा॑भिरे॒वाग्निं व॑र्धयति~(४३)

%5.1.8.6
ऋ॒तुभिः॑ संवथ्स॒रं विश्वा॒ आ भा॑हि प्र॒दिशः॑ पृथि॒व्या इत्या॑ह॒ तस्मा॑द॒ग्निः सर्वा॒ दिशो\-ऽनु॒ वि भा॑ति॒ प्रत्यौ॑हताम॒श्विना॑ मृ॒त्युम॑स्मा॒दित्या॑ह मृ॒त्युमे॒वास्मा॒दप॑ नुद॒त्युद्व॒यं तम॑स॒स्परीत्या॑ह पा॒प्मा वै तमः॑ पा॒प्मान॑मे॒वास्मा॒दप॑ ह॒न्त्यग॑न्म॒ ज्योति॑रुत्त॒ममित्या॑हा॒सौ वा आ॑दि॒त्यो ज्योति॑रुत्त॒ममा॑दि॒त्यस्यै॒व सायु॑ज्यं गच्छति॒ न सं॑वथ्स॒रस्ति॑ष्ठति॒ नास्य॒ श्रीस्ति॑ष्ठति॒ यस्यै॒ताः क्रि॒यन्ते॒ ज्योति॑ष्मतीमुत्त॒मामन्वा॑ह॒ ज्योति॑रे॒वास्मा॑ उ॒परि॑ष्टाद्दधाति सुव॒र्गस्य॑ लो॒कस्यानु॑ख्यात्यै॥~(४४)

{\anuvakamend[{याव॑न्तो\-ऽस्य मुख॒तश्चित्य॑स्य वर्धयत्यादि॒त्यो᳚\-ऽष्टाविꣳ॑शतिश्च}]}%~(८)

%5.1.9.1
ष॒ड्भिर्दी᳚क्षयति॒ षड्वा ऋ॒तव॑ ऋ॒तुभि॑रे॒वैनं॑ दीक्षयति स॒प्तभि॑र्दीक्षयति स॒प्त छन्दाꣳ॑सि॒ छन्दो॑भिरे॒वैनं॑ दीक्षयति॒ विश्वे॑ दे॒वस्य॑ ने॒तुरित्य॑नु॒ष्टुभो᳚त्त॒मया॑ जुहोति॒ वाग्वा अ॑नु॒ष्टुप्तस्मा᳚त्प्रा॒णानां॒ वागु॑त्त॒मैक॑स्माद॒क्षरा॒दना᳚प्तं प्रथ॒मं प॒दं तस्मा॒द्यद्वा॒चो\-ऽना᳚प्तं॒ तन्म॑नु॒ष्या॑ उप॑ जीवन्ति पू॒र्णया॑ जुहोति पू॒र्ण इ॑व॒ हि प्र॒जा\-प॑तिः~(४५)

%5.1.9.2
प्र॒जा\-प॑ते॒राप्त्यै॒ न्यू॑नया जुहोति॒ न्यू॑ना॒द्धि प्र॒जा\-प॑तिः प्र॒जा असृ॑जत प्र॒जाना॒ꣳ॒ सृष्ट्यै॒ यद॒र्चिषि॑ प्रवृ॒ञ्ज्याद्भू॒तमव॑ रुन्धीत॒ यदङ्गा॑रेषु भवि॒ष्यदङ्गा॑रेषु॒ प्र वृ॑णक्ति भवि॒ष्यदे॒वाव॑ रुन्धे भवि॒ष्यद्धि भूयो॑ भू॒ताद्द्वाभ्यां॒ प्र वृ॑णक्ति द्वि॒पाद्यज॑मानः॒ प्रति॑ष्ठित्यै॒ ब्रह्म॑णा॒ वा ए॒षा यजु॑षा॒ सम्भृ॑ता॒ यदु॒खा सा यद्भिद्ये॒तार्ति॒मार्च्छे᳚त्~(४६)

%5.1.9.3
यज॑मानो ह॒न्येता᳚स्य य॒ज्ञो मित्रै॒तामु॒खां त॒पेत्या॑ह॒ ब्रह्म॒ वै मि॒त्रो ब्रह्म॑न्ने॒वैनां॒ प्रति॑\-ष्ठापयति॒ नार्ति॒मार्च्छ॑ति॒ यज॑मानो॒ नास्य॑ य॒ज्ञो ह॑न्यते॒ यदि॒ भिद्ये॑त॒ तैरे॒व क॒पालैः॒ सꣳ सृ॑जे॒थ्सैव ततः॒ प्राय॑श्चित्ति॒र्यो ग॒तश्रीः॒ स्यान्म॑थि॒त्वा तस्याव॑ दध्याद्भू॒तो वा ए॒ष स स्वां~(४७)

%5.1.9.4
दे॒वता॒मुपै॑ति॒ यो भूति॑कामः॒ स्याद्य उ॒खायै॑ स॒म्भवे॒थ्स ए॒व तस्य॑ स्या॒दतो॒ ह्ये॑ष स॒म्भव॑त्ये॒ष वै स्व॑य॒म्भूर्नाम॒ भव॑त्ये॒व यं का॒मये॑त॒ भ्रातृ॑व्यमस्मै जनयेय॒मित्य॒न्यत॒स्तस्या॒हृत्याव॑ दध्याथ्सा॒क्षादे॒वास्मै॒ भ्रातृ॑व्यं जनयत्यम्ब॒रीषा॒\-दन्न॑काम॒स्याव॑ दध्यादम्ब॒रीषे॒ वा अन्न॑म्भ्रियते॒ सयो᳚न्ये॒वान्नम्᳚~(४८)

%5.1.9.5
अव॑ रुन्धे॒ मुञ्जा॒नव॑ दधा॒त्यूर्ग्वै मुञ्जा॒ ऊर्ज॑मे॒वास्मा॒ अपि॑ दधात्य॒ग्निर्दे॒वेभ्यो॒ निला॑यत॒ स क्रु॑मु॒कं प्रावि॑शत् क्रुमु॒कमव॑ दधाति॒ यदे॒वास्य॒ तत्र॒ न्य॑क्तं॒ तदे॒वाव॑ रुन्ध॒ आज्ये॑न॒ सं यौ᳚त्ये॒तद्वा अ॒ग्नेः प्रि॒यं धाम॒ यदाज्यं॑ प्रि॒येणै॒वैनं॒ धाम्ना॒ सम॑र्धय॒त्यथो॒ तेज॑सा~(४९)

%5.1.9.6
वैकं॑कती॒मा द॑धाति॒ भा ए॒वाव॑ रुन्धे शमी॒मयी॒मा द॑धाति॒ शान्त्यै॒ सीद॒ त्वं मा॒तुर॒स्या उ॒पस्थ॒ इति॑ ति॒सृभि॑र्जा॒तमुप॑ तिष्ठते॒ त्रय॑ इ॒मे लो॒का ए॒ष्वे॑व लो॒केष्वा॒विदं॑ गच्छ॒त्यथो᳚ प्रा॒णाने॒वा\-ऽऽ\-त्मन्ध॑त्ते॥~(५०)

{\anuvakamend[{प्र॒जा\-प॑तिर्\mbox{}ऋच्छे॒थ्\-स्वामे॒वान्नं॒ तेज॑सा॒ चतु॑स्त्रिꣳशच्च}]}%~(९)

%5.1.10.1
न ह॑ स्म॒ वै पु॒राग्निरप॑रशुवृक्णं दहति॒ तद॑स्मै प्रयो॒ग ए॒वर्\mbox{}षि॑रस्वदय॒द्यद॑ग्ने॒ यानि॒ कानि॒ चेति॑ स॒मिध॒मा द॑धा॒त्यप॑रशुवृक्णमे॒वास्मै᳚ स्वदयति॒ सर्व॑मस्मै स्वदते॒ य ए॒वं वेदौदु॑म्बरी॒मा द॑धा॒त्यूर्ग्वा उ॑दु॒म्बर॒ ऊर्ज॑मे॒वास्मा॒ अपि॑ दधाति प्र॒जा\-प॑तिर॒ग्निम॑सृजत॒ तꣳ सृ॒ष्टꣳ रक्षाꣳ॑सि~(५१)

%5.1.10.2
अ॒जि॒घा॒ꣳ॒स॒न्थ्स ए॒तद्रा᳚क्षो॒घ्नम॑पश्य॒त्तेन॒ वै स रक्षा॒ꣴ॒स्यपा॑हत॒ यद्रा᳚क्षो॒घ्नं भव॑त्य॒ग्नेरे॒व तेन॑ जा॒ताद्रक्षा॒ꣴ॒स्यप॑ ह॒न्त्याश्व॑त्थी॒मा द॑धात्यश्व॒त्थो वै वन॒स्पती॑नाꣳ सपत्नसा॒हो विजि॑त्यै॒ वैक॑ङ्कती॒मा द॑धाति॒ भा ए॒वाव॑ रुन्धे शमी॒मयी॒मा द॑धाति॒ शान्त्यै॒ सꣳ॑शितं मे॒ ब्रह्मोदे॑षां बा॒हू अ॑तिर॒मित्यु॑त्त॒मे औदु॑म्बरी~(५२)

%5.1.10.3
वा॒च॒य॒ति॒ ब्रह्म॑णै॒व क्ष॒त्रꣳ सꣴ श्य॑ति क्ष॒त्रेण॒ ब्रह्म॒ तस्मा᳚द्ब्राह्म॒णो रा॑ज॒न्य॑वा॒नत्य॒न्यं ब्रा᳚ह्म॒णं तस्मा᳚द्राज॒न्यो᳚ ब्राह्म॒णवा॒नत्य॒न्यꣳ रा॑ज॒न्यं॑ मृ॒त्युर्वा ए॒ष यद॒ग्निर॒मृत॒ꣳ॒ हिर॑ण्यꣳ रु॒क्ममन्त॑रं॒ प्रति॑ मुञ्चते॒\-ऽमृत॑मे॒व मृ॒त्योर॒न्तर्ध॑त्त॒ एक॑विꣳशतिनिर्बाधो भव॒त्येक॑विꣳशति॒र्वै दे॑वलो॒का द्वाद॑श॒ मासाः॒ पञ्च॒र्तव॒स्त्रय॑ इ॒मे लो॒का अ॒सावा॑दि॒त्यः~(५३)

%5.1.10.4
ए॒क॒वि॒ꣳ॒श ए॒ताव॑न्तो॒ वै दे॑वलो॒कास्तेभ्य॑ ए॒व भ्रातृ॑व्यम॒न्तरे॑ति निर्बा॒धैर्वै दे॒वा असु॑रान्निर्बा॒धे॑\-ऽकुर्वत॒ तन्नि॑र्बा॒धानां᳚ निर्बाध॒त्वन्नि॑र्बा॒धी भ॑वति॒ भ्रातृ॑व्याने॒व नि॑र्बा॒धे कु॑रुते सावित्रि॒या प्रति॑ मुञ्चते॒ प्रसू᳚त्यै॒ नक्तो॒षासेत्युत्त॑रयाहोरा॒त्राभ्या॑\-मे॒वैन॒मुद्य॑च्छते दे॒वा अ॒ग्निं धा॑रयन्द्रविणो॒दा इत्या॑ह प्रा॒णा वै दे॒वा द्र॑विणो॒दा अ॑होरा॒त्राभ्या॑मे॒वैन॑मु॒द्यत्य॑~(५४)

%5.1.10.5
प्रा॒णैर्दा॑धा॒रासी॑नः॒ प्रति॑ मुञ्चते॒ तस्मा॒दासी॑नाः प्र॒जाः प्र जा॑यन्ते कृष्णाजि॒नमुत्त॑र॒न्तेजो॒ वै हिर॑ण्यं॒ ब्रह्म॑ कृष्णाजि॒नन्तेज॑सा चै॒वैनं॒ ब्रह्म॑णा चोभ॒यतः॒ परि॑ गृह्णाति॒ षडु॑द्यामꣳ शि॒क्यं॑ भवति॒ षड्वा ऋ॒तव॑ ऋ॒तुभि॑रे॒वैन॒मुद्य॑च्छते॒ यद्द्वाद॑शोद्यामꣳ संवथ्स॒रेणै॒व मौ॒ञ्जं भ॑व॒त्यूर्ग्वै मुञ्जा॑ ऊ॒र्जैवैन॒ꣳ॒ सम॑र्धयति सुप॒र्णो॑\-ऽसि ग॒रुत्मा॒नित्यवे᳚क्षते रू॒पमे॒वास्यै॒तन्म॑हि॒मानं॒ व्याच॑ष्टे॒ दिवं॑ गच्छ॒ सुवः॑ प॒तेत्या॑ह सुव॒र्गमे॒वैनं॑ लो॒कं ग॑मयति॥~(५)

{\anuvakamend[{रक्षा॒ꣴ॒स्यौदु॑म्बरी आदि॒त्य उ॒द्यत्य॒ स़ञ्चतु॑र्विꣳशतिश्च}]}%॥10॥

%5.1.11.1
समि॑द्धो अ॒ञ्जन्कृद॑रं मती॒नां घृ॒तम॑ग्ने॒ मधु॑म॒त्पिन्व॑मानः। वा॒जी वह॑न्वा॒जिनं॑ जातवेदो दे॒वानां᳚ वक्षि प्रि॒यमा स॒धस्थम्᳚। घृ॒तेना॒ञ्जन्थ्सम्प॒थो दे॑व॒याना᳚न्प्रजा॒नन्वा॒ज्यप्ये॑तु दे॒वान्। अनु॑ त्वा सप्ते प्र॒दिशः॑ सचन्ताꣴ स्व॒धाम॒स्मै यज॑मानाय धेहि। ईड्य॒श्चासि॒ वन्द्य॑श्च वाजिन्ना॒शुश्चासि॒ मेध्य॑श्च सप्ते। अ॒ग्निष्ट्वा᳚~(५६)

%5.1.11.2
दे॒वैर्वसु॑भिः स॒जोषाः᳚ प्री॒तं वह्निं॑ वहतु जा॒तवे॑दाः। स्ती॒र्णं ब॒र्॒\mbox{}हिः सु॒ष्टरी॑मा जुषा॒णोरु पृ॒थु प्रथ॑मानं पृथि॒व्याम्। दे॒वेभि॑र्यु॒क्तमदि॑तिः स॒जोषाः᳚ स्यो॒नं कृ॑ण्वा॒ना सु॑वि॒ते द॑धातु। ए॒ता उ॑ वः सु॒भगा॑ वि॒श्वरू॑पा॒ वि पक्षो॑भिः॒ श्रय॑माणा॒ उदातैः᳚। ऋ॒ष्वाः स॒तीः क॒वषः॒ शुम्भ॑माना॒ द्वारो॑ दे॒वीः सु॑प्राय॒णा भ॑वन्तु। अ॒न्त॒रा मि॒त्रावरु॑णा॒ चर॑न्ती॒ मुखं॑ य॒ज्ञाना॑म॒भि सं॑विदा॒ने। उ॒षासा॑ वाम्~(५७)

%5.1.11.3
सु॒हि॒र॒ण्ये सु॑शि॒ल्पे ऋ॒तस्य॒ योना॑वि॒ह सा॑दयामि। प्र॒थ॒मा वाꣳ॑ सर॒थिना॑ सु॒वर्णा॑ दे॒वौ पश्य॑न्तौ॒ भुव॑नानि॒ विश्वा᳚। अपि॑प्रयं॒ चोद॑ना वां॒ मिमा॑ना॒ होता॑रा॒ ज्योतिः॑ प्र॒दिशा॑ दि॒शन्ता᳚। आ॒दि॒त्यैर्नो॒ भार॑ती वष्टु य॒ज्ञꣳ सर॑स्वती स॒ह रु॒द्रैर्न॑ आवीत्। इडोप॑हूता॒ वसु॑भिः स॒जोषा॑ य॒ज्ञं नो॑ देवीर॒मृते॑षु धत्त। त्वष्टा॑ वी॒रं दे॒वका॑मं जजान॒ त्वष्टु॒रर्वा॑ जायत आ॒शुरश्वः॑।~(५८)

%5.1.11.4
त्वष्टे॒दं विश्वं॒ भुव॑नं जजान ब॒होः क॒र्तार॑मि॒ह य॑क्षि होतः। अश्वो॑ घृ॒तेन॒ त्मन्या॒ सम॑क्त॒ उप॑ दे॒वाꣳ ऋ॑तु॒शः पाथ॑ एतु। वन॒स्पति॑र्देवलो॒कं प्र॑जा॒नन्न॒ग्निना॑ ह॒व्या स्व॑दि॒तानि॑ वक्षत्। प्र॒जा\-प॑ते॒स्तप॑सा वावृधा॒नः स॒द्यो जा॒तो द॑धिषे य॒ज्ञम॑ग्ने। स्वाहा॑कृतेन ह॒विषा॑ पुरोगा या॒हि सा॒ध्या ह॒विर॑दन्तु दे॒वाः॥~(५९)

{\anuvakamend[{अ॒ग्निष्ट्वा॑ वा॒मश्वो॒ द्विच॑त्वारिꣳशच्च}]}%॥11॥

\prashnaend{सा॒वि॒त्राणि॒ व्यृद्ध॒मुत्क्रा॑म दे॒वस्य॑ खनति क्रू॒रं वा॑रु॒णः स॒प्तभि॒रेक॑विꣳशत्या ष॒ड्भिर्न ह॑ स्म॒ समि॑द्धो अ॒ञ्जन्नेका॑दश॥११॥}{सा॒वि॒त्राण्युत्क्रा॑म क्रू॒रं वा॑रु॒णः प॒शवः॑ स्यु॒र्न ह॑ स्म॒ नव॑पञ्चा॒शत्॥५९॥}{सा॒वि॒त्राणि॑ ह॒विर॑दन्तु दे॒वाः॥}%%५-१
{हरिः॑ ॐ}{॥कृष्ण-यजुर्वेदीय-तैत्तिरीय-संहितायां पञ्चम्काण्डे प्रथमः प्रश्नः समाप्तः॥५-१॥}
%%% END PRASHNA

\sect{द्वितीयः प्रश्नः}\setcounter{anuvakam}{0}
\dnsub{तैत्तिरीयसंहितायां पञ्चमकाण्डे द्वितीयः प्रश्नः}
%5.2.1.1
विष्णु॑मुखा॒ वै दे॒वाश्छन्दो॑भिरि॒माँल्लो॒कान॑नपज॒य्यम॒भ्य॑जय॒न्॒ यद्वि॑ष्णुक्र॒मान्क्रम॑ते॒ विष्णु॑रे॒व भू॒त्वा यज॑मान॒श्छन्दो॑भिरि॒माँल्लो॒कान॑नपज॒य्यम॒भि ज॑यति॒ विष्णोः॒ क्रमो᳚\-ऽस्यभिमाति॒हेत्या॑ह गाय॒त्री वै पृ॑थि॒वी त्रैष्ठु॑भम॒न्तरि॑क्षं॒ जाग॑ती॒ द्यौरानु॑ष्टुभी॒र्दिश॒श्छन्दो॑भिरे॒वेमाँल्लो॒कान् य॑थापू॒र्वम॒भि ज॑यति प्र॒जा\-प॑तिर॒ग्निम॑सृजत॒ सो᳚\-ऽस्माथ्सृ॒ष्टः~(१)

%5.2.1.2
परा॑ङै॒त्तमे॒तयान्वै॒दक्र॑न्द॒दिति॒ तया॒ वै सो᳚\-ऽग्नेः प्रि॒यं धामावा॑रुन्ध॒ यदे॒ताम॒न्वाहा॒ग्नेरे॒वैतया᳚ प्रि॒यं धामाव॑ रुन्ध ईश्व॒रो वा ए॒ष परा᳚ङ्प्र॒दघो॒ यो वि॑ष्णुक्र॒मान्क्रम॑ते चत॒सृभि॒रा व॑र्तते च॒त्वारि॒ छन्दाꣳ॑सि॒ छन्दाꣳ॑सि॒ खलु॒ वा अ॒ग्नेः प्रि॒या त॒नूः प्रि॒यामे॒वास्य॑ त॒नुव॑म॒भि~(२)

%5.2.1.3
प॒र्याव॑र्तते दक्षि॒णा प॒र्याव॑र्तते॒ स्वमे॒व वी॒र्य॑मनु॑ प॒र्याव॑र्तते॒ तस्मा॒द्दक्षि॒णो\-ऽर्ध॑ आ॒त्मनो॑ वी॒र्या॑वत्त॒रो\-ऽथो॑ आदि॒त्यस्यै॒वावृत॒मनु॑ प॒र्याव॑र्तते॒ शुनः॒शेप॒माजी॑गर्तिं॒ वरु॑णो\-ऽगृह्णा॒थ्स ए॒तां वा॑रु॒णीम॑पश्य॒त्तया॒ वै स आ॒त्मानं॑ वरुणपा॒शाद॑मुञ्च॒द्वरु॑णो॒ वा ए॒तं गृ॑ह्णाति॒ य उ॒खां प्र॑तिमु॒ञ्चत॒ उदु॑त्त॒मं व॑रुण॒ पाश॑म॒स्मदित्या॑हा॒\-ऽ॒\-ऽ॒त्मान॑मे॒वैतया᳚~(३)

%5.2.1.4
व॒रु॒ण॒पा॒शान्मु॑ञ्च॒त्या त्वा॑हार्\mbox{}ष॒मित्या॒हा ह्ये॑न॒ꣳ॒ हर॑ति ध्रु॒वस्ति॒ष्ठावि॑चाचलि॒रित्या॑ह॒ प्रति॑ष्ठित्यै॒ विश॑स्त्वा॒ सर्वा॑ वाञ्छ॒न्त्वित्या॑ह वि॒शैवैन॒ꣳ॒ सम॑र्धयत्य॒स्मिन्रा॒ष्ट्रमधि॑ श्र॒येत्या॑ह रा॒ष्ट्रमे॒वास्मि॑न्ध्रु॒वम॑क॒र्यं का॒मये॑त रा॒ष्ट्रꣴ स्या॒दिति॒ तं मन॑सा ध्यायेद्रा॒ष्ट्रमे॒व भ॑वति~(४)

%5.2.1.5
अग्रे॑ बृ॒हन्नु॒षसा॑मू॒र्ध्वो अ॑स्था॒दित्या॒हाग्र॑मे॒वैनꣳ॑ समा॒नानां᳚ करोति निर्जग्मि॒वान्तम॑स॒ इत्या॑ह॒ तम॑ ए॒वास्मा॒दप॑ हन्ति॒ ज्योति॒षागा॒दित्या॑ह॒ ज्योति॑रे॒वास्मि॑न्दधाति चत॒सृभिः॑ सादयति च॒त्वारि॒ छन्दाꣳ॑सि॒ छन्दो॑भिरे॒वाति॑छन्दसोत्त॒मया॒ वर्ष्म॒ वा ए॒षा छन्द॑सां॒ यदति॑च्छन्दा॒ वर्ष्मै॒वैनꣳ॑ समा॒नानां᳚ करोति॒ सद्व॑ती~(५)

%5.2.1.6
भ॒व॒ति॒ स॒त्त्वमे॒वैनं॑ गमयति वाथ्स॒प्रेणोप॑ तिष्ठत ए॒तेन॒ वै व॑थ्स॒प्रीर्भा॑लन्द॒नो᳚\-ऽग्नेः प्रि॒यं धामावा॑रुन्धा॒ग्नेरे॒वैतेन॑ प्रि॒यं धामाव॑ रुन्ध एकाद॒शं भ॑वत्येक॒धैव यज॑माने वी॒र्यं॑ दधाति॒ स्तोमे॑न॒ वै दे॒वा अ॒स्मिँल्लो॒क आ᳚र्ध्नुव॒ञ्छन्दो॑भिर॒मुष्मि॒न्थ्स्तोम॑स्येव॒ खलु॒ वा ए॒तद्रू॒पं यद्वा᳚थ्स॒प्रम्यद्वा᳚थ्स॒प्रेणो॑प॒तिष्ठ॑ते~(६)

%5.2.1.7
इ॒ममे॒व तेन॑ लो॒कम॒भि ज॑य॒ति यद्वि॑ष्णुक्र॒मान्क्रम॑ते॒\-ऽमुमे॒व तैर्लो॒कम॒भि ज॑यति पूर्वे॒द्युः प्र क्रा॑मत्युत्तरे॒द्युरुप॑ तिष्ठते॒ तस्मा॒द्योगे॒\-ऽन्यासां᳚ प्र॒जानां॒ मनः॒ क्षेमे॒\-ऽन्यासा॒न्तस्मा᳚द्यायाव॒रः क्षे॒म्यस्ये॑शे॒ तस्मा᳚द्यायाव॒रः क्षे॒म्यम॒ध्यव॑स्यति मु॒ष्टी क॑रोति॒ वाचं॑ यच्छति य॒ज्ञस्य॒ धृत्यै᳚॥~(७)

{\anuvakamend[{सृ॒ष्टो\-ऽभ्ये॑तया॑ भवति॒ सद्व॑त्युप॒तिष्ठ॑ते॒ द्विच॑त्वारिꣳशच्च}]}%~(१)

%5.2.2.1
अन्न॑प॒ते\-ऽन्न॑स्य नो दे॒हीत्या॑हा॒ग्निर्वा अन्न॑पतिः॒ स ए॒वास्मा॒ अन्नं॒ प्र य॑च्छत्यनमी॒वस्य॑ शु॒ष्मिण॒ इत्या॑हाय॒क्ष्मस्येति॒ वावैतदा॑ह॒ प्र प्र॑दा॒तारं॑ तारिष॒ ऊर्जं॑ नो धेहि द्वि॒पदे॒ चतु॑ष्पद॒ इत्या॑हा॒\-ऽऽ\-शिष॑मे॒वैतामा शा᳚स्त॒ उदु॑ त्वा॒ विश्वे॑ दे॒वा इत्या॑ह प्रा॒णा वै विश्वे॑ दे॒वाः~(८)

%5.2.2.2
प्रा॒णैरे॒वैन॒मुद्य॑च्छ॒ते\-ऽग्ने॒ भर॑न्तु॒ चित्ति॑भि॒रित्या॑ह॒ यस्मा॑ ए॒वैनं॑ चि॒त्तायो॒द्यच्छ॑ते॒ तेनै॒वैन॒ꣳ॒ सम॑र्धयति चत॒सृभि॒रा सा॑दयति च॒त्वारि॒ छन्दाꣳ॑सि॒ छन्दो॑भिरे॒वाति॑च्छन्दसोत्त॒मया॒ वर्ष्म॒ वा ए॒षा छन्द॑सां॒ यदति॑च्छन्दा॒ वर्ष्मै॒वैनꣳ॑ समा॒नानां᳚ करोति॒ सद्व॑ती भवति स॒त्त्वमे॒वैनं॑ गमयति॒ प्रेद॑ग्ने॒ ज्योति॑ष्मान्~(९)

%5.2.2.3
या॒हीत्या॑ह॒ ज्योति॑रे॒वास्मि॑न्दधाति त॒नुवा॒ वा ए॒ष हि॑नस्ति॒ यꣳ हि॒नस्ति॒ मा हिꣳ॑सीस्त॒नुवा᳚ प्र॒जा इत्या॑ह प्र॒जाभ्य॑ ए॒वैनꣳ॑ शमयति॒ रक्षाꣳ॑सि॒ वा ए॒तद्य॒ज्ञꣳ स॑चन्ते॒ यदन॑ उ॒थ्सर्ज॒त्यक्र॑न्द॒दित्यन्वा॑ह॒ रक्ष॑सा॒मप॑हत्या॒ अन॑सा वह॒न्त्यप॑चितिमे॒वास्मि॑न्दधाति॒ तस्मा॑दन॒स्वी च॑ र॒थी चाति॑थीना॒मप॑चिततमौ~(१०)

%5.2.2.4
अप॑चितिमान्भवति॒ य ए॒वं वेद॑ स॒मिधा॒\-ऽग्निं दु॑वस्य॒तेति॑ घृतानुषि॒क्तामव॑सिते स॒मिध॒मा द॑धाति॒ यथाति॑थय॒ आग॑ताय स॒र्पिष्व॑दाति॒थ्यं क्रि॒यते॑ ता॒दृगे॒व तद्गा॑यत्रि॒या ब्रा᳚ह्म॒णस्य॑ गाय॒त्रो हि ब्रा᳚ह्म॒णस्त्रि॒ष्टुभा॑ राज॒न्य॑स्य॒ त्रैष्टु॑भो॒ हि रा॑ज॒न्यो᳚\-ऽफ्सु भस्म॒ प्र वे॑शयत्य॒फ्सुयो॑नि॒र्वा अ॒ग्निः स्वामे॒वैनं॒ योनिं॑ गमयति ति॒सृभिः॒ प्र वे॑शयति त्रि॒वृद्वै~(११)

%5.2.2.5
अ॒ग्निर्यावा॑ने॒वाग्निस्तं प्र॑ति॒ष्ठां ग॑मयति॒ परा॒ वा ए॒षो᳚\-ऽग्निं व॑पति॒ यो᳚\-ऽफ्सु भस्म॑ प्रवे॒शय॑ति॒ ज्योति॑ष्मतीभ्या॒मव॑ दधाति॒ ज्योति॑रे॒वास्मि॑न्दधाति॒ द्वाभ्यां॒ प्रति॑ष्ठित्यै॒ परा॒ वा ए॒ष प्र॒जां प॒शून् व॑पति॒ यो᳚\-ऽफ्सु भस्म॑ प्रवे॒शय॑ति॒ पुन॑रू॒र्जा स॒ह र॒य्येति॒ पुन॑रु॒दैति॑ प्र॒जामे॒व प॒शूना॒त्मन्ध॑त्ते॒ पुन॑स्त्वादि॒त्याः~(१२)

%5.2.2.6
रु॒द्रा वस॑वः॒ समि॑न्धता॒मित्या॑है॒ता वा ए॒तं दे॒वता॒ अग्रे॒ समै᳚न्धत॒ ताभि॑रे॒वैन॒ꣳ॒ समि॑न्द्धे॒ बोधा॒ स बो॒धीत्युप॑ तिष्ठते बो॒धय॑त्ये॒वैन॒न्तस्मा᳚थ्सु॒प्त्वा प्र॒जाः प्र बु॑ध्यन्ते यथास्था॒नमुप॑ तिष्ठते॒ तस्मा᳚द्यथास्था॒नं प॒शवः॒ पुन॒रेत्योप॑ तिष्ठन्ते॥~(१३)

{\anuvakamend[{वै विश्वे॑ दे॒वा ज्योति॑ष्मा॒नप॑चिततमौ त्रि॒वृद्वा आ॑दि॒त्या द्विच॑त्वारिꣳशच्च}]}%~(२)

%5.2.3.1
याव॑ती॒ वै पृ॑थि॒वी तस्यै॑ य॒म आधि॑पत्यं॒ परी॑याय॒ यो वै य॒मं दे॑व॒यज॑नम॒स्या अनि॑र्याच्या॒ग्निं चि॑नु॒ते य॒मायै॑न॒ꣳ॒ स चि॑नु॒ते\-ऽपे॒तेत्य॒ध्यव॑साययति य॒ममे॒व दे॑व॒यज॑नम॒स्यै नि॒र्याच्या॒\-ऽऽ\-त्मने॒\-ऽग्निं चि॑नुत इष्व॒ग्रेण॒ वा अ॒स्या अना॑मृतमि॒च्छन्तो॒ नावि॑न्द॒न्ते दे॒वा ए॒तद्यजु॑रपश्य॒न्नपे॒तेति॒ यदे॒तेना᳚ध्यवसा॒यय॑ति~(१४)

%5.2.3.2
अना॑मृत ए॒वाग्निं चि॑नुत॒ उद्ध॑न्ति॒ यदे॒वास्या॑ अमे॒ध्यं तदप॑ हन्त्य॒पो\-ऽवो᳚क्षति॒ शान्त्यै॒ सिक॑ता॒ नि व॑पत्ये॒तद्वा अ॒ग्नेर्वै᳚श्वान॒रस्य॑ रू॒पꣳ रू॒पेणै॒व वै᳚श्वान॒रमव॑ रुन्ध॒ ऊषा॒न्नि व॑पति॒ पुष्टि॒र्वा ए॒षा प्र॒जन॑नं॒ यदूषाः॒ पुष्ट्या॑मे॒व प्र॒जन॑ने॒\-ऽग्निं चि॑नु॒ते\-ऽथो॑ सं॒ज्ञान॑ ए॒व सं॒ज्ञान॒ꣴ॒ ह्ये॑तत्~(१५)

%5.2.3.3
प॒शू॒नां यदूषा॒ द्यावा॑पृथि॒वी स॒हास्ता॒न्ते वि॑य॒ती अ॑ब्रूता॒मस्त्वे॒व नौ॑ स॒ह य॒ज्ञिय॒मिति॒ यद॒मुष्या॑ य॒ज्ञिय॒मासी॒त्तद॒स्याम॑दधा॒त्त ऊषा॑ अभव॒न्॒ यद॒स्या य॒ज्ञिय॒मासी॒त्तद॒मुष्या॑मदधा॒त्तद॒दश्च॒न्द्रम॑सि कृ॒ष्णमूषा᳚न्नि॒वप॑न्न॒दो ध्या॑ये॒द्द्यावा॑पृथि॒व्योरे॒व य॒ज्ञिये॒\-ऽग्निं चि॑नुते॒\-ऽयꣳ सो अ॒ग्निरिति॑ वि॒श्वामि॑त्रस्य~(१६)

%5.2.3.4
सू॒क्तं भ॑वत्ये॒तेन॒ वै वि॒श्वामि॑त्रो॒\-ऽग्नेः प्रि॒यं धामावा॑रुन्धा॒ग्नेरे॒वैतेन॑ प्रि॒यं धामाव॑ रुन्धे॒ छन्दो॑भि॒र्वै दे॒वाः सु॑व॒र्गं लो॒कमा॑य॒ञ्चत॑स्रः॒ प्राची॒रुप॑ दधाति च॒त्वारि॒ छन्दाꣳ॑सि॒ छन्दो॑भिरे॒व तद्यज॑मानः सुव॒र्गं लो॒कमे॑ति॒ तेषाꣳ॑ सुव॒र्गं लो॒कं य॒तां दिशः॒ सम॑व्लीयन्त॒ ते द्वे पु॒रस्ता᳚थ्स॒मीची॒ उपा॑दधत॒ द्वे~(१७)

%5.2.3.5
प॒श्चाथ्स॒मीची॒ ताभि॒र्वै ते दिशो॑\-ऽदृꣳह॒न्॒ यद्द्वे पु॒रस्ता᳚थ्स॒मीची॑ उप॒दधा॑ति॒ द्वे प॒श्चाथ्स॒मीची॑ दि॒शां विधृ॑त्या॒ अथो॑ प॒शवो॒ वै छन्दाꣳ॑सि पशूने॒वास्मै॑ स॒मीचो॑ दधात्य॒ष्टावुप॑ दधात्य॒ष्टाक्ष॑रा गाय॒त्री गा॑य॒त्रो᳚\-ऽग्निर्यावा॑ने॒वाग्निस्तं चि॑नुते॒\-ऽष्टावुप॑ दधात्य॒ष्टाक्ष॑रा गाय॒त्री गा॑य॒त्री सु॑व॒र्गं लो॒कमञ्ज॑सा वेद सुव॒र्गस्य॑ लो॒कस्य॑~(१८)

%5.2.3.6
प्रज्ञा᳚त्यै॒ त्रयो॑दश लोकं पृ॒णा उप॑ दधा॒त्येक॑विꣳशतिः॒ सम्प॑द्यन्ते प्रति॒ष्ठा वा ए॑कवि॒ꣳ॒शः प्र॑ति॒ष्ठा गार्\mbox{}ह॑पत्य एकवि॒ꣳ॒शस्यै॒व प्र॑ति॒ष्ठां गार्\mbox{}ह॑पत्य॒मनु॒ प्रति॑ तिष्ठति॒ प्रत्य॒ग्निं चि॑क्या॒नस्ति॑ष्ठति॒ य ए॒वं वेद॒ पञ्च॑चितीकं चिन्वीत प्रथ॒मं चि॑न्वा॒नः पाङ्क्तो॑ य॒ज्ञः पाङ्क्ताः᳚ प॒शवो॑ य॒ज्ञमे॒व प॒शूनव॑ रुन्धे॒ त्रिचि॑तीकं चिन्वीत द्वि॒तीयं॑ चिन्वा॒नस्त्रय॑ इ॒मे लो॒का ए॒ष्वे॑व लो॒केषु॑~(१९)

%5.2.3.7
प्रति॑ तिष्ठ॒त्येक॑चितीकं चिन्वीत तृ॒तीयं॑ चिन्वा॒न ए॑क॒धा वै सु॑व॒र्गो लो॒क ए॑क॒वृतै॒व सु॑व॒र्गं लो॒कमे॑ति॒ पुरी॑षेणा॒भ्यू॑हति॒ तस्मा᳚न्मा॒ꣳ॒सेनास्थि॑ छ॒न्नन्न दु॒श्चर्मा॑ भवति॒ य ए॒वं वेद॒ पञ्च॒ चित॑यो भवन्ति प॒ञ्चभिः॒ पुरी॑षैर॒भ्यू॑हति॒ दश॒ सं प॑द्यन्ते॒ दशा᳚क्षरा वि॒राडन्नं॑ वि॒राड्वि॒राज्ये॒वान्नाद्ये॒ प्रति॑ तिष्ठति॥~(२०)

{\anuvakamend[{अ॒द्ध्य॒व॒सा॒यय॑ति॒ ह्ये॑तद्वि॒श्वामि॑त्रस्यादधत॒ द्वे लो॒कस्य॑ लो॒केषु॑ स॒प्तच॑त्वारिꣳशच्च}]}%~(३)

%5.2.4.1
वि वा ए॒तौ द्वि॑षाते॒ यश्च॑ पु॒राग्निर्यश्चो॒खाया॒ꣳ॒ समि॑त॒मिति॑ चत॒सृभिः॒ सं नि व॑पति च॒त्वारि॒ छन्दाꣳ॑सि॒ छन्दाꣳ॑सि॒ खलु॒ वा अ॒ग्नेः प्रि॒या त॒नूः प्रि॒ययै॒वैनौ॑ त॒नुवा॒ सꣳ शा᳚स्ति॒ समि॑त॒मित्या॑ह॒ तस्मा॒द्ब्रह्म॑णा क्ष॒त्रꣳ समे॑ति॒ यथ्सं॒न्युप्य॑ वि॒हर॑ति॒ तस्मा॒द्ब्रह्म॑णा क्ष॒त्रं व्ये᳚त्यृ॒तुभिः॑~(२१)

%5.2.4.2
वा ए॒तं दी᳚क्षयन्ति॒ स ऋ॒तुभि॑रे॒व वि॒मुच्यो॑ मा॒तेव॑ पु॒त्रं पृ॑थि॒वी पु॑री॒ष्य॑मित्या॑ह॒र्तुभि॑रे॒वैनं॑ दीक्षयि॒त्वर्तुभि॒र्वि मु॑ञ्चति वैश्वान॒र्या शि॒क्य॑मा द॑त्ते स्व॒दय॑त्ये॒वैन॑न्नैर्\mbox{}ऋ॒तीः कृ॒ष्णास्ति॒स्रस्तुष॑पक्वा भवन्ति॒ निर्\mbox{}ऋ॑त्यै॒ वा ए॒तद्भा॑ग॒धेयं॒ यत्तुषा॒ निर्\mbox{}ऋ॑त्यै रू॒पं कृ॒ष्णꣳ रू॒पेणै॒व निर्\mbox{}ऋ॑तिं नि॒रव॑दयत इ॒मां दिशं॑ यन्त्ये॒षा~(२२)

%5.2.4.3
वै निर्\mbox{}ऋ॑त्यै॒ दिख्स्वाया॑मे॒व दि॒शि निर्\mbox{}ऋ॑तिं नि॒रव॑दयते॒ स्वकृ॑त॒ इरि॑ण॒ उप॑ दधाति प्रद॒रे वै॒तद्वै निर्\mbox{}ऋ॑त्या आ॒यत॑न॒ꣴ॒ स्व ए॒वायत॑ने॒ निर्\mbox{}ऋ॑तिं नि॒रव॑दयते शि॒क्य॑म॒भ्युप॑ दधाति नैर्\mbox{}ऋ॒तो वै पाशः॑ सा॒क्षादे॒वैनं॑ निर्\mbox{}ऋतिपा॒शान्मु॑ञ्चति ति॒स्र उप॑ दधाति त्रेधाविहि॒तो वै पुरु॑षो॒ यावा॑ने॒व पुरु॑ष॒स्तस्मा॒न्निर्\mbox{}ऋ॑ति॒मव॑ यजते॒ परा॑ची॒रुप॑~(२३)

%5.2.4.4
द॒धा॒ति॒ परा॑चीमे॒वास्मा॒न्निर्\mbox{}ऋ॑तिं॒ प्र णु॑द॒ते\-ऽप्र॑तीक्ष॒मा य॑न्ति॒ निर्\mbox{}ऋ॑त्या अ॒न्तर्\mbox{}हि॑त्यै मार्जयि॒त्वोप॑ तिष्ठन्ते मेध्य॒त्वाय॒ गार्\mbox{}ह॑पत्य॒मुप॑ तिष्ठन्ते निर्\mbox{}ऋतिलो॒क ए॒व च॑रि॒त्वा पू॒ता दे॑वलो॒कमु॒पाव॑र्तन्त॒ एक॒योप॑ तिष्ठन्त एक॒धैव यज॑माने वी॒र्यं॑ दधति नि॒वेश॑नः सं॒गम॑नो॒ वसू॑ना॒मित्या॑ह प्र॒जा वै प॒शवो॒ वसु॑ प्र॒जयै॒वैनं॑ प॒शुभिः॒ सम॑र्धयन्ति॥~(२४)

{\anuvakamend[{ऋ॒तुभि॑रे॒षा परा॑ची॒रुपा॒ष्टाच॑त्वारिꣳशच्च}]}%~(४)

%5.2.5.1
पु॒रु॒ष॒मा॒त्रेण॒ वि मि॑मीते य॒ज्ञेन॒ वै पुरु॑षः॒ सम्मि॑तो यज्ञप॒रुषै॒वैनं॒ वि मि॑मीते॒ यावा॒न्पुरु॑ष ऊ॒र्ध्वबा॑हु॒स्तावा᳚न्भव\-त्ये॒ताव॒द्वै पुरु॑षे वी॒र्यं॑ वी॒र्ये॑णै॒वैनं॒ वि मि॑मीते प॒क्षी भ॑वति॒ न ह्य॑प॒क्षः पति॑तु॒मर्\mbox{}ह॑त्यर॒त्निना॑ प॒क्षौ द्राघी॑याꣳसौ भवत॒स्तस्मा᳚त्प॒क्षप्र॑वयाꣳसि॒ वयाꣳ॑सि व्याममा॒त्रौ प॒क्षौ च॒ पुच्छं॑ च भवत्ये॒ताव॒द्वै पुरु॑षे वी॒र्यम्᳚~(२५)

%5.2.5.2
वी॒र्य॑सम्मितो॒ वेणु॑ना॒ वि मि॑मीत आग्ने॒यो वै वेणुः॑ सयोनि॒त्वाय॒ यजु॑षा युनक्ति॒ यजु॑षा कृषति॒ व्यावृ॑त्त्यै षड्ग॒वेन॑ कृषति॒ षड्वा ऋ॒तव॑ ऋ॒तुभि॑रे॒वैनं॑ कृषति॒ यद्द्वा॑दशग॒वेन॑ संवथ्स॒रेणै॒वेयं वा अ॒ग्नेर॑तिदा॒हाद॑बिभे॒थ्सैतद्द्वि॑गु॒णम॑पश्यत्कृ॒ष्टं चाकृ॑ष्टं च॒ ततो॒ वा इ॒मां नात्य॑दह॒द्यत्कृ॒ष्टं चाकृ॑ष्टं च~(२६)

%5.2.5.3
भव॑त्य॒स्या अन॑तिदाहाय द्विगु॒णं त्वा अ॒ग्निमुद्य॑न्तुमर्\mbox{}ह॒तीत्या॑हु॒र्यत्कृ॒ष्टं चाकृ॑ष्टं च॒ भव॑त्य॒ग्नेरुद्य॑त्या ए॒ताव॑न्तो॒ वै प॒शवो᳚ द्वि॒पाद॑श्च॒ चतु॑ष्पादश्च॒ तान् यत्प्राच॑ उथ्सृ॒जेद्रु॒द्रायापि॑ दध्या॒द्यद्द॑क्षि॒णा पि॒तृभ्यो॒ नि धु॑वे॒द्यत्प्र॒तीचो॒ रक्षाꣳ॑सि हन्यु॒रुदी॑च॒ उथ्सृ॑जत्ये॒षा वै दे॑वमनु॒ष्याणाꣳ॑ शा॒न्ता दिक्~(२७)

%5.2.5.4
तामे॒वैना॒ननूथ्सृ॑ज॒त्यथो॒ खल्वि॒मां दिश॒मुथ्सृ॑जत्य॒सौ वा आ॑दि॒त्यः प्रा॒णः प्रा॒णमे॒वैना॒ननूथ्सृ॑जति दक्षि॒णा प॒र्याव॑र्तन्ते॒ स्वमे॒व वी॒र्य॑मनु॑ प॒र्याव॑र्तन्ते॒ तस्मा॒द्दक्षि॒णो\-ऽर्ध॑ आ॒त्मनो॑ वी॒र्या॑वत्त॒रो\-ऽथो॑ आदि॒त्यस्यै॒वावृत॒मनु॑ प॒र्याव॑र्तन्ते॒ तस्मा॒त्परा᳚ञ्चः प॒शवो॒ वि ति॑ष्ठन्ते प्र॒त्यं च॒ आ व॑र्तन्ते ति॒स्रस्ति॑स्रः॒ सीताः᳚~(२८)

%5.2.5.5
कृ॒ष॒ति॒ त्रि॒वृत॑मे॒व य॑ज्ञमु॒खे वि या॑तय॒त्योष॑धीर्वपति॒ ब्रह्म॒णान्न॒मव॑ रुन्धे॒\-ऽर्के᳚\-ऽर्कश्ची॑यते चतुर्द॒शभि॑र्वपति स॒प्त ग्रा॒म्या ओष॑धयः स॒प्तार॒ण्या उ॒भयी॑षा॒मव॑रुद्ध्या॒ अन्न॑स्यान्नस्य वप॒त्यन्न॑स्यान्न॒स्याव॑रुद्ध्यै कृ॒ष्टे व॑पति कृ॒ष्टे ह्योष॑धयः प्रति॒तिष्ठ॑न्त्यनुसी॒तं व॑पति॒ प्रजा᳚त्यै द्वाद॒शसु॒ सीता॑सु वपति॒ द्वाद॑श॒ मासाः᳚ संवथ्स॒रः सं॑वथ्स॒रेणै॒वास्मा॒ अन्नं॑ पचति॒ यद॑ग्नि॒चित्~(२९)

%5.2.5.6
अन॑वरुद्धस्याश्ञी॒यादव॑रुद्धेन॒ व्यृ॑द्ध्येत॒ ये वन॒स्पती॑नाम्फल॒ग्रह॑य॒स्तानि॒ध्मे\-ऽपि॒ प्रोक्षे॒दन॑वरुद्ध॒स्याव॑रुद्ध्यै दि॒ग्भ्यो लो॒ष्टान्थ्सम॑स्यति दि॒शामे॒व वी॒र्य॑मव॒रुध्य॑ दि॒शां वी॒र्ये᳚\-ऽग्निं चि॑नुते॒ यं द्वि॒ष्याद्यत्र॒ स स्यात्तस्यै॑ दि॒शो लो॒ष्टमा ह॑रे॒दिष॒मूर्ज॑म॒हमि॒त आ द॑द॒ इतीष॑मे॒वोर्जं॒ तस्यै॑ दि॒शो\-ऽव॑ रुन्धे॒ क्षोधु॑को भवति॒ यस्तस्यां᳚ दि॒शि भव॑त्युत्तरवे॒दिमुप॑ वपत्युत्तरवे॒द्याꣳ ह्य॑ग्निश्ची॒यते\-ऽथो॑ प॒शवो॒ वा उ॑त्तरवे॒दिः प॒शूने॒वाव॑ रु॒न्धे\-ऽथो॑ यज्ञप॒रुषो\-ऽन॑न्तरित्यै॥~(३०)

{\anuvakamend[{च॒ भ॒व॒त्ये॒ताव॒द्वै पुरु॑षे वी॒र्यं॑ यत्कृ॒ष्टञ्चाकृ॑ष्टं च॒ दिख्सीता॑ अग्नि॒चिदव॒ पञ्च॑विꣳशतिश्च}]}%~(५)

%5.2.6.1
अग्ने॒ तव॒ श्रवो॒ वय॒ इति॒ सिक॑ता॒ नि व॑पत्ये॒तद्वा अ॒ग्नेर्वै᳚श्वान॒रस्य॑ सू॒क्तꣳ सू॒क्तेनै॒व वै᳚श्वान॒रमव॑ रुन्धे ष॒ड्भिर्नि व॑पति॒ षड्वा ऋ॒तवः॑ संवथ्स॒रः सं॑वथ्स॒रो᳚\-ऽग्निर्वै᳚श्वान॒रः सा॒क्षादे॒व वै᳚श्वान॒रमव॑ रुन्धे समु॒द्रं वै नामै॒तच्छन्दः॑ समु॒द्रमनु॑ प्र॒जाः प्र जा॑यन्ते॒ यदे॒तेन॒ सिक॑ता नि॒वप॑ति प्र॒जानां᳚ प्र॒जन॑ना॒येन्द्रः॑~(३१)

%5.2.6.2
वृ॒त्राय॒ वज्रं॒ प्राह॑र॒थ्स त्रे॒धा व्य॑भव॒थ्स्फ्यस्तृती॑य॒ꣳ॒ रथ॒स्तृती॑यं॒ यूप॒स्तृती॑यं॒ ये᳚\-ऽन्तःश॒रा अशी᳚र्यन्त॒ ताः शर्क॑रा अभव॒न्तच्छर्क॑राणाꣳ शर्कर॒त्वं वज्रो॒ वै शर्क॑राः प॒शुर॒ग्निर्यच्छर्क॑राभिर॒ग्निं प॑रिमि॒नोति॒ वज्रे॑णै॒वास्मै॑ प॒शून्परि॑ गृह्णाति॒ तस्मा॒द्वज्रे॑ण प॒शवः॒ परि॑गृहीता॒स्तस्मा॒थ्स्थेया॒नस्थे॑यसो॒ नोप॑ हरते त्रिस॒प्ताभिः॑~(३२)

%5.2.6.3
प॒शुका॑मस्य॒ परि॑ मिनुयाथ्स॒प्त वै शी॑र्\mbox{}ष॒ण्याः᳚ प्रा॒णाः प्रा॒णाः प॒शवः॑ प्रा॒णैरे॒वास्मै॑ प॒शूनव॑ रुन्धे त्रिण॒वाभि॒\-र्भ्रातृ॑व्यवतस्त्रि॒वृत॑मे॒व वज्रꣳ॑ स॒म्भृत्य॒ भ्रातृ॑व्याय॒ प्र ह॑रति॒ स्तृत्या॒ अप॑रिमिताभिः॒ परि॑ मिनुया॒दप॑रिमित॒स्याव॑रुद्ध्यै॒ यं का॒मये॑ताप॒शुः स्या॒दित्यप॑रिमित्य॒ तस्य॒ शर्क॑राः॒ सिक॑ता॒ व्यू॑हे॒दप॑रिगृहीत ए॒वास्य॑ विषू॒चीन॒ꣳ॒ रेतः॒ परा॒ सिञ्चत्यप॒शुरे॒व भ॑वति~(३३)

%5.2.6.4
यं का॒मये॑त पशु॒मान्थ्स्या॒दिति॑ परि॒मित्य॒ तस्य॒ शर्क॑राः॒ सिक॑ता॒ व्यू॑हे॒त्परि॑गृहीत ए॒वास्मै॑ समी॒चीन॒ꣳ॒ रेतः॑ सिञ्चति पशु॒माने॒व भ॑वति सौ॒म्या व्यू॑हति॒ सोमो॒ वै रे॑तो॒धा रेत॑ ए॒व तद्द॑धाति गायत्रि॒या ब्रा᳚ह्म॒णस्य॑ गाय॒त्रो हि ब्रा᳚ह्म॒णस्त्रि॒ष्टुभा॑ राज॒न्य॑स्य॒ त्रैष्टु॑भो॒ हि रा॑ज॒न्यः॑ शं॒युं बा॑र्\mbox{}हस्प॒त्यं मेधो॒ नोपा॑नम॒थ्सो᳚\-ऽग्निं प्रावि॑शत्~(३४)

%5.2.6.5
सो᳚\-ऽग्नेः कृष्णो॑ रू॒पं कृ॒त्वोदा॑यत॒ सो\-ऽश्वं॒ प्रावि॑श॒थ्सो\-ऽश्व॑स्यावान्तरश॒फो॑\-ऽभव॒द्यदश्व॑माक्र॒मय॑ति॒ य ए॒व मेधो\-ऽश्वं॒ प्रावि॑श॒त्तमे॒वाव॑ रुन्धे प्र॒जा\-प॑तिना॒ग्निश्चे॑त॒व्य॑ इत्या॑हुः प्राजाप॒त्यो\-ऽश्वो॒ यदश्व॑माक्र॒मय॑ति प्र॒जा\-प॑तिनै॒वाग्निं चि॑नुते पुष्करप॒र्णमुप॑ दधाति॒ योनि॒र्वा अ॒ग्नेः पु॑ष्करप॒र्णꣳ सयो॑निमे॒वाग्निं चि॑नुते॒\-ऽपां पृ॒ष्ठम॒सीत्युप॑ दधात्य॒पां वा ए॒तत्पृ॒ष्ठं यत्पु॑ष्करप॒र्णꣳ रू॒पेणै॒वैन॒दुप॑ दधाति॥~(३५)

{\anuvakamend[{इन्द्रः॑ प॒शुका॑मस्य भवत्यविश॒थ्सयो॑निं विꣳश॒तिश्च॑}]}%~(६)

%5.2.7.1
ब्रह्म॑ जज्ञा॒नमिति॑ रु॒क्ममुप॑ दधाति॒ ब्रह्म॑मुखा॒ वै प्र॒जा\-प॑तिः प्र॒जा अ॑सृजत॒ ब्रह्म॑मुखा ए॒व तत्प्र॒जा यज॑मानः सृजते॒ ब्रह्म॑ जज्ञा॒नमित्या॑ह॒ तस्मा᳚द्ब्राह्म॒णो मुख्यो॒ मुख्यो॑ भवति॒ य ए॒वं वेद॑ ब्रह्मवा॒दिनो॑ वदन्ति॒ न पृ॑थि॒व्यां नान्तरि॑क्षे॒ न दि॒व्य॑ग्निश्चे॑त॒व्य॑ इति॒ यत्पृ॑थि॒व्यां चि॑न्वी॒त पृ॑थि॒वीꣳ शु॒चार्प॑ये॒न्नौष॑धयो॒ न वन॒स्पत॑यः~(३६)

%5.2.7.2
प्र जा॑येर॒न्॒ यद॒न्तरि॑क्षे चिन्वी॒तान्तरि॑क्षꣳ शु॒चार्प॑ये॒न्न वयाꣳ॑सि॒ प्र जा॑येर॒न्॒ यद्दि॒वि चि॑न्वी॒त दिवꣳ॑ शु॒चार्प॑ये॒न्न प॒र्जन्यो॑ वर्\mbox{}षेद्रु॒क्ममुप॑ दधात्य॒मृतं॒ वै हिर॑ण्यम॒मृत॑ ए॒वाग्निं चि॑नुते॒ प्रजा᳚त्यै हिर॒ण्मयं॒ पुरु॑ष॒मुप॑ दधाति यजमानलो॒कस्य॒ विधृ॑त्यै॒ यदिष्ट॑काया॒ आतृ॑ण्णमनूपद॒ध्यात्प॑शू॒नां च॒ यज॑मानस्य च प्रा॒णमपि॑ दध्याद्दक्षिण॒तः~(३७)

%5.2.7.3
प्राञ्च॒मुप॑ दधाति दा॒धार॑ यजमानलो॒कन्न प॑शू॒नां च॒ यज॑मानस्य च प्रा॒णमपि॑ दधा॒त्यथो॒ खल्विष्ट॑काया॒ आतृ॑ण्ण॒मनूप॑ दधाति प्रा॒णाना॒मुथ्सृ॑ष्ट्यै द्र॒फ्सश्च॑स्क॒न्देत्य॒भि मृ॑शति॒ होत्रा᳚स्वे॒वैनं॒ प्रति॑\-ष्ठापयति॒ स्रुचा॒वुप॑ दधा॒त्याज्य॑स्य पू॒र्णां का᳚र्ष्मर्य॒मयीं᳚ द॒ध्नः पू॒र्णामौदु॑म्बरीमि॒यं वै का᳚र्ष्मर्य॒मय्य॒सावौदु॑म्बरी॒मे ए॒वोप॑ धत्ते~(३८)

%5.2.7.4
तू॒ष्णीमुप॑ दधाति॒ न हीमे यजु॒षाप्तु॒मर्\mbox{}ह॑ति॒ दक्षि॑णां कार्ष्मर्य॒मयी॒मुत्त॑रा॒मौदु॑म्बरी॒न्तस्मा॑द॒स्या अ॒सावुत्त॒राज्य॑स्य पू॒र्णां का᳚र्ष्मर्य॒मयीं॒ वज्रो॒ वा आज्यं॒ वज्रः॑ कार्ष्म॒र्यो॑ वज्रे॑णै॒व य॒ज्ञस्य॑ दक्षिण॒तो रक्षा॒ꣴ॒स्यप॑ हन्ति द॒ध्नः पू॒र्णामौदु॑म्बरीं प॒शवो॒ वै दध्यूर्गु॑दु॒म्बरः॑ प॒शुष्वे॒वोर्जं॑ दधाति पू॒र्णे उप॑ दधाति पू॒र्णे ए॒वैनम्᳚~(३९)

%5.2.7.5
अ॒मुष्मिँ॑ल्लो॒क उप॑ तिष्ठेते वि॒राज्य॒ग्निश्चे॑त॒व्य॑ इत्या॒॑हुः स्रुग्वै वि॒राड्यथ्स्रुचा॑वुप॒दधा॑ति वि॒राज्ये॒वाग्निं चि॑नुते यज्ञमु॒खेय॑ज्ञमुखे॒ वै क्रि॒यमा॑णे य॒ज्ञꣳ रक्षाꣳ॑सि जिघाꣳसन्ति यज्ञमु॒खꣳ रु॒क्मो यद्रु॒क्मं व्या॑घा॒रय॑ति यज्ञमु॒खादे॒व रक्षा॒ꣴ॒स्यप॑ हन्ति प॒ञ्चभि॒र्व्याघा॑रयति॒ पाङ्क्तो॑ य॒ज्ञो यावा॑ने॒व य॒ज्ञस्तस्मा॒द्रक्षा॒ꣴ॒स्यप॑ हन्त्यक्ष्ण॒या व्याघा॑रयति॒ तस्मा॑दक्ष्ण॒या प॒शवो\-ऽङ्गा॑नि॒ प्र ह॑रन्ति॒ प्रति॑ष्ठित्यै॥~(४०)

{\anuvakamend[{वन॒स्पत॑यो दक्षिण॒तो ध॑त्त एन॒न्तस्मा॑दक्ष्ण॒या पञ्च॑ च}]}%~(७)

%5.2.8.1
स्व॒य॒मा॒तृ॒ण्णामुप॑ दधाती॒यं वै स्व॑यमातृ॒ण्णेमामे॒वोप॑ ध॒त्ते\-ऽश्व॒मुप॑ घ्रापयति प्रा॒णमे॒वास्यां᳚ दधा॒त्यथो᳚ प्राजाप॒त्यो वा अश्वः॑ प्र॒जा\-प॑तिनै॒वाग्निं चि॑नुते प्रथ॒मेष्ट॑कोपधी॒यमा॑ना पशू॒नां च॒ यज॑मानस्य च प्रा॒णमपि॑ दधाति स्वयमातृ॒ण्णा भ॑वति प्रा॒णाना॒मुथ्सृ॑ष्ट्या॒ अथो॑ सुव॒र्गस्य॑ लो॒कस्यानु॑ख्यात्या अ॒ग्नाव॒ग्निश्चे॑त॒व्य॑ इत्या॑हुरे॒ष वै~(४१)

%5.2.8.2
अ॒ग्निर्वै᳚श्वान॒रो यद्ब्रा᳚ह्म॒णस्तस्मै᳚ प्रथ॒मामिष्ट॑कां॒ यजु॑ष्कृतां॒ प्र य॑च्छे॒त्तां ब्रा᳚ह्म॒णश्चोप॑ दध्याताम॒ग्नावे॒व तद॒ग्निं चि॑नुत ईश्व॒रो वा ए॒ष आर्ति॒मार्तो॒र्यो\-ऽवि॑द्वा॒निष्ट॑कामुप॒दधा॑ति॒ त्रीन् वरा᳚न्दद्या॒त्त्रयो॒ वै प्रा॒णाः प्रा॒णाना॒ꣴ॒ स्पृत्यै॒ द्वावे॒व देयौ॒ द्वौ हि प्रा॒णावेक॑ ए॒व देय॒ एको॒ हि प्रा॒णः प॒शुः~(४२)

%5.2.8.3
वा ए॒ष यद॒ग्निर्न खलु॒ वै प॒शव॒ आय॑वसे रमन्ते दूर्वेष्ट॒कामुप॑ दधाति पशू॒नां धृत्यै॒ द्वाभ्यां॒ प्रति॑ष्ठित्यै॒ काण्डा᳚त्काण्डात्प्र॒रोह॒न्तीत्या॑ह॒ काण्डे॑नकाण्डेन॒ ह्ये॑षा प्र॑ति॒तिष्ठ॑त्ये॒वा नो॑ दूर्वे॒ प्र त॑नु स॒हस्रे॑ण श॒तेन॒ चेत्या॑ह साह॒स्रः प्र॒जा\-प॑तिः प्र॒जा\-प॑ते॒राप्त्यै॑ देवल॒क्ष्मं वै त्र्या॑लिखि॒ता तामुत्त॑रलक्ष्माणं दे॒वा उपा॑दध॒ताध॑रलक्ष्माण॒मसु॑रा॒ यम्~(४३)

%5.2.8.4
का॒मये॑त॒ वसी॑यान्थ्स्या॒दित्युत्त॑रलक्ष्माणं॒ तस्योप॑ दध्या॒द्वसी॑याने॒व भ॑वति॒ यं का॒मये॑त॒ पापी॑यान्थ्स्या॒दित्यध॑र\-लक्ष्माणं॒ तस्योप॑ दध्यादसुरयो॒निमे॒वैन॒मनु॒ परा॑ भावयति॒ पापी॑यान्भवति त्र्यालिखि॒ता भ॑वती॒मे वै लो॒का\-स्त्र्या॑लिखि॒तैभ्य ए॒व लो॒केभ्यो॒ भ्रातृ॑व्यम॒न्तरे॒त्यङ्गि॑रसः सुव॒र्गं लो॒कं य॒तः पु॑रो॒डाशः॑ कू॒र्मो भू॒त्वानु॒ प्रास॑र्पत्~(४४)

%5.2.8.5
यत्कू॒र्ममु॑प॒दधा॑ति॒ यथा᳚ क्षेत्र॒विदञ्ज॑सा॒ नय॑त्ये॒वमे॒वैनं॑ कू॒र्मः सु॑व॒र्गं लो॒कमञ्ज॑सा नयति॒ मेधो॒ वा ए॒ष प॑शू॒नां यत्कू॒र्मो यत्कू॒र्ममु॑प॒दधा॑ति॒ स्वमे॒व मेधं॒ पश्य॑न्तः प॒शव॒ उप॑ तिष्ठन्ते श्मशा॒नं वा ए॒तत्क्रि॑यते॒ यन्मृ॒तानां᳚ पशू॒नाꣳ शी॒र्॒\mbox{}षाण्यु॑पधी॒यन्ते॒ यज्जीव॑न्तं कू॒र्ममु॑प॒दधा॑ति॒ तेनाश्म॑शानचिद्वास्त॒व्यो॑ वा ए॒ष यत्~(४५)

%5.2.8.6
कू॒र्मो मधु॒ वाता॑ ऋताय॒त इति॑ द॒ध्ना म॑धुमि॒श्रेणा॒भ्य॑नक्ति स्व॒दय॑त्ये॒वैनं॑ ग्रा॒म्यं वा ए॒तदन्नं॒ यद्दध्या॑र॒ण्यं मधु॒ यद्द॒ध्ना म॑धुमि॒श्रेणा᳚भ्य॒नक्त्यु॒भय॒स्याव॑रुद्ध्यै म॒ही द्यौः पृ॑थि॒वी च॑ न॒ इत्या॑हा॒भ्यामे॒वैन॑मुभ॒यतः॒ परि॑ गृह्णाति॒ प्राञ्च॒मुप॑ दधाति॒ सुव॒र्गस्य॑ लो॒कस्य॒ सम॑ष्ट्यै पु॒रस्ता᳚त्प्र॒त्यञ्च॒मुप॑ दधाति॒ तस्मा᳚त्~(४६)

%5.2.8.7
पु॒रस्ता᳚त्प्र॒त्यञ्चः॑ प॒शवो॒ मेध॒मुप॑ तिष्ठन्ते॒ यो वा अप॑नाभिम॒ग्निं चि॑नु॒ते यज॑मानस्य॒ नाभि॒मनु॒ प्र वि॑शति॒ स ए॑नमीश्व॒रो हिꣳसि॑तोरु॒लूख॑ल॒मुप॑ दधात्ये॒षा वा अ॒ग्नेर्नाभिः॒ सना॑भिमे॒वाग्निं चि॑नु॒ते\-ऽहिꣳ॑साया॒ औ॑दुम्बरं भव॒त्यूर्ग्वा उ॑दु॒म्बर॒ ऊर्ज॑मे॒वाव॑ रुन्धे मध्य॒त उप॑ दधाति मध्य॒त ए॒वास्मा॒ ऊर्जं॑ दधाति॒ तस्मा᳚न्मध्य॒त ऊ॒र्जा भु॑ञ्जत॒ इय॑द्भवति प्र॒जा\-प॑तिना यज्ञमु॒खेन॒ सम्मि॑त॒मव॑ ह॒न्त्यन्न॑मे॒वाक॑र्वैष्ण॒व्यर्चोप॑ दधाति॒ विष्णु॒र्वै य॒ज्ञो वै᳚ष्ण॒वा वन॒स्पत॑यो य॒ज्ञ ए॒व य॒ज्ञं प्रति॑\-ष्ठापयति॥~(४७)

{\anuvakamend[{ए॒ष वै प॒शुर्यम॑सर्पदे॒ष यत्तस्मा॒त्तस्मा᳚थ्स॒प्तविꣳ॑शतिश्च}]}%~(८)

%5.2.9.1
ए॒षां वा ए॒तल्लो॒कानां॒ ज्योतिः॒ सम्भृ॑तं॒ यदु॒खा यदु॒खामु॑प॒दधा᳚त्ये॒भ्य ए॒व लो॒केभ्यो॒ ज्योति॒रव॑ रुन्धे मध्य॒त उप॑ दधाति मध्य॒त ए॒वास्मै॒ ज्योति॑र्दधाति॒ तस्मा᳚न्मध्य॒तो ज्योति॒रुपा᳚स्महे॒ सिक॑ताभिः पूरयत्ये॒तद्वा अ॒ग्नेर्वै᳚श्वान॒रस्य॑ रू॒पꣳ रू॒पेणै॒व वै᳚श्वान॒रमव॑ रुन्धे॒ यं का॒मये॑त॒ क्षोधु॑कः स्या॒दित्यू॒नां तस्योप॑~(४८)

%5.2.9.2
द॒ध्या॒त्क्षोधु॑क ए॒व भ॑वति॒ यं का॒मये॒तानु॑पदस्य॒दन्न॑मद्या॒दिति॑ पू॒र्णां तस्योप॑ दध्या॒दनु॑पदस्यदे॒वान्न॑मत्ति स॒हस्रं॒ वै प्रति॒ पुरु॑षः पशू॒नां य॑च्छति स॒हस्र॑म॒न्ये प॒शवो॒ मध्ये॑ पुरुषशी॒र्॒\mbox{}षमुप॑ दधाति सवीर्य॒त्वायो॒खाया॒मपि॑ दधाति प्रति॒ष्ठामे॒वैन॑द्गमयति॒ व्यृ॑द्धं॒ वा ए॒तत्प्रा॒णैर॑मे॒ध्यं यत्पु॑रुषशी॒र्॒\mbox{}षम॒मृतं॒ खलु॒ वै प्रा॒णाः~(४९)

%5.2.9.3
अ॒मृत॒ꣳ॒ हिर॑ण्यं प्रा॒णेषु॑ हिरण्यश॒ल्कान्प्रत्य॑स्यति प्रति॒ष्ठामे॒वैन॑द्गमयि॒त्वा प्रा॒णैः सम॑र्धयति द॒ध्ना म॑धुमि॒श्रेण॑ पूरयति मध॒व्यो॑\-ऽसा॒नीति॑ शृतात॒ङ्क्ये॑न मेध्य॒त्वाय॑ ग्रा॒म्यं वा ए॒तदन्नं॒ यद्दध्या॑र॒ण्यं मधु॒ यद्द॒ध्ना म॑धुमि॒श्रेण॑ पू॒रय॑त्यु॒भय॒स्याव॑रुद्ध्यै पशुशी॒र्॒\mbox{}षाण्युप॑ दधाति प॒शवो॒ वै प॑शुशी॒र्॒\mbox{}षाणि॑ प॒शूने॒वाव॑ रुन्धे॒ यं का॒मये॑ताप॒शुः स्या॒दिति॑~(५०)

%5.2.9.4
वि॒षू॒चीना॑नि॒ तस्योप॑ दध्या॒द्विषू॑च ए॒वास्मा᳚त्प॒शून्द॑धात्यप॒शुरे॒व भ॑वति॒ यं का॒मये॑त पशु॒मान्थ्स्या॒दिति॑ समी॒चीना॑नि॒ तस्योप॑ दध्याथ्स॒मीच॑ ए॒वास्मै॑ प॒शून्द॑धाति पशु॒माने॒व भ॑वति पु॒रस्ता᳚त्प्रती॒चीन॒मश्व॒स्योप॑ दधाति प॒श्चात्प्रा॒चीन॑मृष॒भस्याप॑शवो॒ वा अ॒न्ये गो॑अ॒श्वेभ्यः॑ प॒शवो॑ गोअ॒श्वाने॒वास्मै॑ स॒मीचो॑ दधात्ये॒ताव॑न्तो॒ वै प॒शवः॑~(५१)

%5.2.9.5
द्वि॒पाद॑श्च॒ चतु॑ष्पादश्च॒ तान् वा ए॒तद॒ग्नौ प्र द॑धाति॒ यत्प॑शुशी॒र्॒\mbox{}षाण्यु॑प॒दधा᳚त्य॒मुमा॑र॒ण्यमनु॑ ते दिशा॒मीत्या॑ह ग्रा॒म्येभ्य॑ ए॒व प॒शुभ्य॑ आर॒ण्यान्प॒शूञ्छुच॒मनूथ्सृ॑जति॒ तस्मा᳚थ्स॒माव॑त्पशू॒नां प्र॒जाय॑मानानामार॒ण्याः प॒शवः॒ कनी॑याꣳसः शु॒चा ह्यृ॑ताः स॑र्पशी॒र्॒\mbox{}षमुप॑ दधाति॒ यैव स॒र्पे त्विषि॒स्तामे॒वाव॑ रुन्धे~(५२)

%5.2.9.6
यथ्स॑मी॒चीनं॑ पशुशी॒र्॒\mbox{}षैरु॑पद॒ध्याद्ग्रा॒म्यान्प॒शून्दꣳशु॑काः स्यु॒र्यद्वि॑षू॒चीन॑मार॒ण्यान् यजु॑रे॒व व॑दे॒दव॒ तां त्विषिꣳ॑ रुन्धे॒ या स॒र्पे न ग्रा॒म्यान्प॒शून् हि॒नस्ति॒ नार॒ण्यानथो॒ खलू॑प॒धेय॑मे॒व यदु॑प॒दधा॑ति॒ तेन॒ तां त्विषि॒मव॑ रुन्धे॒ या स॒र्पे यद्यजु॒र्वद॑ति॒ तेन॑ शा॒न्तम्॥~(५३)

{\anuvakamend[{ऊ॒नान्तस्योप॑ प्रा॒णाः स्या॒दिति॒ वै प॒शवो॑ रुन्धे॒ चतु॑श्चत्वारिꣳशच्च}]}%~(९)

%5.2.10.1
प॒शुर्वा ए॒ष यद॒ग्निर्योनिः॒ खलु॒ वा ए॒षा प॒शोर्वि क्रि॑यते॒ यत्प्रा॒चीन॑मैष्ट॒काद्यजुः॑ क्रि॒यते॒ रेतो॑\-ऽप॒स्या॑ अप॒स्या॑ उप॑ दधाति॒ योना॑वे॒व रेतो॑ दधाति॒ पञ्चोप॑ दधाति॒ पाङ्क्ताः᳚ प॒शवः॑ प॒शूने॒वास्मै॒ प्र ज॑नयति॒ पञ्च॑ दक्षिण॒तो वज्रो॒ वा अ॑प॒स्या॑ वज्रे॑णै॒व य॒ज्ञस्य॑ दक्षिण॒तो रक्षा॒ꣴ॒स्यप॑ हन्ति॒ पञ्च॑ प॒श्चात्~(५४)

%5.2.10.2
प्राची॒रुप॑ दधाति प॒श्चाद्वै प्रा॒चीन॒ꣳ॒ रेतो॑ धीयते प॒श्चादे॒वास्मै᳚ प्रा॒चीन॒ꣳ॒ रेतो॑ दधाति॒ पञ्च॑ पु॒रस्ता᳚त्प्र॒तीची॒रुप॑ दधाति॒ पञ्च॑ प॒श्चात्प्राची॒स्तस्मा᳚त्प्रा॒चीन॒ꣳ॒ रेतो॑ धीयते प्र॒तीचीः᳚ प्र॒जा जा॑यन्ते॒ पञ्चो᳚त्तर॒तश्छ॑न्द॒स्याः᳚ प॒शवो॒ वै छ॑न्द॒स्याः᳚ प॒शूने॒व प्रजा॑ता॒न्थ्स्वमा॒यत॑नम॒भि पर्यू॑हत इ॒यं वा अ॒ग्नेर॑तिदा॒हाद॑बिभे॒थ्सैताः~(५)

%5.2.10.3
अ॒प॒स्या॑ अपश्य॒त्ता उपा॑धत्त॒ ततो॒ वा इ॒मां नात्य॑दह॒द्यद॑प॒स्या॑ उप॒दधा᳚त्य॒स्या अन॑तिदाहायो॒वाच॑ हे॒यमद॒दिथ्स ब्रह्म॒णान्नं॒ यस्यै॒ता उ॑पधी॒यान्तै॒ य उ॑ चैना ए॒वं वेद॒दिति॑ प्राण॒भृत॒ उप॑ दधाति॒ रेत॑स्ये॒व प्रा॒णान्द॑धाति॒ तस्मा॒द्वद॑न्प्रा॒णन्पश्य॑ञ्छृ॒ण्वन्प॒शुर्जा॑यते॒\-ऽयं पु॒रः~(५६)

%5.2.10.4
भुव॒ इति॑ पु॒रस्ता॒दुप॑ दधाति प्रा॒णमे॒वैताभि॑र्दाधारा॒यं द॑क्षि॒णा वि॒श्वक॒र्मेति॑ दक्षिण॒तो मन॑ ए॒वैताभि॑र्दाधारा॒यं प॒श्चाद्वि॒श्वव्य॑चा॒ इति॑ प॒श्चाच्चक्षु॑रे॒वैताभि॑र्दाधारे॒दमु॑त्त॒राथ्सुव॒रित्यु॑त्तर॒तः श्रोत्र॑मे॒वैताभि॑र्दाधारे॒यमु॒परि॑ म॒तिरित्यु॒परि॑ष्टा॒द्वाच॑मे॒वैताभि॑र्दाधार॒ दश॑द॒शोप॑ दधाति सवीर्य॒त्वाया᳚क्ष्ण॒या~(५७)

%5.2.10.5
उप॑ दधाति॒ तस्मा॑दक्ष्ण॒या प॒शवो\-ऽङ्गा॑नि॒ प्र ह॑रन्ति॒ प्रति॑ष्ठित्यै॒ याः प्राची॒स्ताभि॒र्वसि॑ष्ठ आर्ध्नो॒द्या द॑क्षि॒णा ताभि॑र्भ॒रद्वा॑जो॒ याः प्र॒तीची॒स्ताभि॑र्वि॒श्वामि॑त्रो॒ या उदी॑ची॒स्ताभि॑र्ज॒मद॑ग्नि॒र्या ऊ॒र्ध्वास्ताभि॑र्वि॒श्वक॑र्मा॒ य ए॒वमे॒तासा॒मृद्धिं॒ वेद॒र्ध्नोत्ये॒व य आ॑सामे॒वं ब॒न्धुतां॒ वेद॒ बन्धु॑मान्भवति॒ य आ॑सामे॒वं कॢप्तिं॒ वेद॒ कल्प॑ते~(५८)

%5.2.10.6
अ॒स्मै॒ य आ॑सामे॒वमा॒यत॑नं॒ वेदा॒यत॑नवान्भवति॒ य आ॑सामे॒वं प्र॑ति॒ष्ठां वेद॒ प्रत्ये॒व ति॑ष्ठति प्राण॒भृत॑ उप॒धाय॑ सं॒यत॒ उप॑ दधाति प्रा॒णाने॒वास्मि॑न्धि॒त्वा सं॒यद्भिः॒ सं य॑च्छति॒ तथ्सं॒यताꣳ॑ संय॒त्त्वमथो᳚ प्रा॒ण ए॒वापा॒नं द॑धाति॒ तस्मा᳚त्प्राणापा॒नौ सं च॑रतो॒ विषू॑ची॒रुप॑ दधाति॒ तस्मा॒द्विष्व॑ञ्चौ प्राणापा॒नौ यद्वा अ॒ग्नेरसं॑ यतम्~(५९)

%5.2.10.7
असु॑वर्ग्यमस्य॒ तथ्सु॑व॒र्ग्यो᳚\-ऽग्निर्यथ्सं॒ यत॑ उप॒दधा॑ति॒ समे॒वैनं॑ यच्छति सुव॒र्ग्य॑मे॒वाक॒स्त्र्यवि॒र्वयः॑ कृ॒तमया॑ना॒मित्या॑ह॒ वयो॑भिरे॒वाया॒नव॑ रु॒न्धे\-ऽयै॒र्वयाꣳ॑सि स॒र्वतो॑ वायु॒मती᳚र्भवन्ति॒ तस्मा॑द॒यꣳ स॒र्वतः॑ पवते॥~(६०)

{\anuvakamend[{प॒श्चादे॒ताः पु॒रो᳚\-ऽक्ष्ण॒या कल्प॒ते\-ऽसं॑ यतं॒ पञ्च॑त्रिꣳशच्च}]}%॥10॥

%5.2.11.1
गा॒य॒त्री त्रि॒ष्टुब्जग॑त्यनु॒ष्टुक्प॒ङ्क्त्या॑ स॒ह। बृ॒ह॒त्यु॑ष्णिहा॑ क॒कुथ्सू॒चीभिः॑ शिम्यन्तु त्वा। द्वि॒पदा॒ या चतु॑ष्पदा त्रि॒पदा॒ या च॒ षट्प॑दा। सछ॑न्दा॒ या च॒ विच्छ॑न्दाः सू॒चीभिः॑ शिम्यन्तु त्वा। म॒हाना᳚म्नी रे॒वत॑यो॒ विश्वा॒ आशाः᳚ प्र॒सूव॑रीः। मेघ्या॑ वि॒द्युतो॒ वाचः॑ सू॒चीभिः॑ शिम्यन्तु त्वा। र॒ज॒ता हरि॑णीः॒ सीसा॒ युजो॑ युज्यन्ते॒ कर्म॑भिः। अश्व॑स्य वा॒जिन॑स्त्व॒चि सू॒चीभिः॑ शिम्यन्तु त्वा। नारीः᳚~(६१)

%5.2.11.2
ते॒ पत्न॑यो॒ लोम॒ वि चि॑न्वन्तु मनी॒षया᳚। दे॒वानां॒ पत्नी॒र्दिशः॑ सू॒चीभिः॑ शिम्यन्तु त्वा। कु॒विद॒ङ्ग यव॑मन्तो॒ यवं॑ चि॒द्यथा॒ दान्त्य॑नुपू॒र्वं वि॒यूय॑। इ॒हेहै॑षां कृणुत॒ भोज॑नानि॒ ये ब॒र्॒\mbox{}हिषो॒ नमो॑वृक्तिं॒ न ज॒ग्मुः॥~(६२)

{\anuvakamend[{नारी᳚स्त्रि॒ꣳ॒शच्च॑}]}%॥11॥

%5.2.12.1
कस्त्वा᳚ छ्यति॒ कस्त्वा॒ वि शा᳚स्ति॒ कस्ते॒ गात्रा॑णि शिम्यति। क उ॑ ते शमि॒ता क॒विः। ऋ॒तव॑स्त ऋतु॒धा परुः॑ शमि॒तारो॒ वि शा॑सतु। सं॒व॒थ्स॒रस्य॒ धाय॑सा॒ शिमी॑भिः शिम्यन्तु त्वा। दैव्या॑ अध्व॒र्यव॑स्त्वा॒ छ्यन्तु॒ वि च॑ शासतु। गात्रा॑णि पर्व॒शस्ते॒ शिमाः᳚ कृण्वन्तु॒ शिम्य॑न्तः। अ॒र्ध॒मा॒साः परूꣳ॑षि ते॒ मासा᳚श्छ्यन्तु॒ शिम्य॑न्तः। अ॒हो॒रा॒त्राणि॑ म॒रुतो॒ विलि॑ष्टं~(६३)

%5.2.12.2
सू॒द॒य॒न्तु॒ ते॒। पृ॒थि॒वी ते॒\-ऽन्तरि॑क्षेण वा॒युश्छि॒द्रं भि॑षज्यतु। द्यौस्ते॒ नक्ष॑त्रैः स॒ह रू॒पं कृ॑णोतु साधु॒या। शं ते॒ परे᳚भ्यो॒ गात्रे᳚भ्यः॒ शम॒स्त्वव॑रेभ्यः। शम॒स्थभ्यो॑ म॒ज्जभ्यः॒ शमु॑ ते त॒नुवे॑ भुवत्~(६४)

{\anuvakamend[{विलि॑ष्टन्त्रि॒ꣳ॒शच्च॑}]}%॥12॥

\prashnaend{विष्णु॑मुखा॒ अन्न॑पते॒ याव॑ती॒ वि वै पु॑रुषमा॒त्रेणाग्ने॒ तव॒ श्रवो॒ ब्रह्म॑ जज्ञा॒नꣴ स्व॑यमातृ॒ण्णामे॒षां वै प॒शुर्गा॑य॒त्री कस्त्वा॒ द्वाद॑श॥१२॥}{विष्णु॑मुखा॒ अप॑चितिमा॒न् वि वा ए॒तावग्ने॒ तव॑ स्वयमातृ॒ण्णां वि॑षू॒चीना॑नि गाय॒त्री चतुः॑ षष्टिः॥६४॥}{विष्णु॑मुखास्त॒नुवे॑ भुवत्॥}%%५-२
{हरिः॑ ॐ}{॥कृष्ण-यजुर्वेदीय-तैत्तिरीय-संहितायां पञ्चम्काण्डे द्वितीयः प्रश्नः समाप्तः॥५-२॥}
%%% END PRASHNA

\sect{तृतीयः प्रश्नः}\setcounter{anuvakam}{0}
\dnsub{तैत्तिरीयसंहितायां पञ्चमकाण्डे तृतीयः प्रश्नः}
%5.3.1.1
उ॒थ्स॒न्न॒य॒ज्ञो वा ए॒ष यद॒ग्निः किं वाहै॒तस्य॑ क्रि॒यते॒ किं वा॒ न यद्वै य॒ज्ञस्य॑ क्रि॒यमा॑णस्यान्त॒र्यन्ति॒ पूय॑ति॒ वा अ॑स्य॒ तदा᳚श्वि॒नीरुप॑ दधात्य॒श्विनौ॒ वै दे॒वानां᳚ भि॒षजौ॒ ताभ्या॑मे॒वास्मै॑ भेष॒जं क॑रोति॒ पञ्चोप॑ दधाति॒ पाङ्क्तो॑ य॒ज्ञो यावा॑ने॒व य॒ज्ञस्तस्मै॑ भेष॒जं क॑रोत्यृत॒व्या॑ उप॑ दधात्यृतू॒नां कॢप्त्यै᳚~(१)

%5.3.1.2
पञ्चोप॑ दधाति पञ्च॒ वा ऋ॒तवो॒ याव॑न्त ए॒वर्तव॒स्तान्क॑ल्पयति समा॒नप्र॑भृतयो भवन्ति समा॒नोद॑र्का॒स्तस्मा᳚थ्समा॒ना ऋ॒तव॒ एके॑न प॒देन॒ व्याव॑र्तन्ते॒ तस्मा॑दृ॒तवो॒ व्याव॑र्तन्ते प्राण॒भृत॒ उप॑ दधात्यृ॒तुष्वे॒व प्रा॒णान्द॑धाति॒ तस्मा᳚थ्समा॒नाः सन्त॑ ऋ॒तवो॒ न जी᳚र्य॒न्त्यथो॒ प्र ज॑नयत्ये॒वैना॑ने॒ष वै वा॒युर्यत्प्रा॒णो यदृ॑त॒व्या॑ उप॒धाय॑ प्राण॒भृतः॑~(२)

%5.3.1.3
उ॒प॒दधा॑ति॒ तस्मा॒थ्सर्वा॑नृ॒तूननु॑ वा॒युरा व॑रीवर्त्ति वृष्टि॒सनी॒रुप॑ दधाति॒ वृष्टि॑मे॒वाव॑ रुन्धे॒ यदे॑क॒धोप॑द॒ध्यादेक॑मृ॒तुं व॑र्\mbox{}षेदनुपरि॒हारꣳ॑ सादयति॒ तस्मा॒थ्सर्वा॑नृ॒तून् व॑र्\mbox{}षति॒ यत्प्रा॑ण॒भृत॑ उप॒धाय॑ वृष्टि॒सनी॑रुप॒दधा॑ति॒ तस्मा᳚द्वा॒युप्र॑च्युता दि॒वो वृ॑ष्टिरीर्ते प॒शवो॒ वै व॑य॒स्या॑ नाना॑मनसः॒ खलु॒ वै प॒शवो॒ नाना᳚व्रता॒स्ते॑\-ऽप ए॒वाभि सम॑नसः~(३)

%5.3.1.4
यं का॒मये॑ताप॒शुः स्या॒दिति॑ वय॒स्या᳚स्तस्यो॑प॒धाया॑प॒स्या॑ उप॑ दध्या॒दसं᳚ज्ञानमे॒वास्मै॑ प॒शुभिः॑ करोत्यप॒शुरे॒व भ॑वति॒ यं का॒मये॑त पशु॒मान्थ्स्या॒दित्य॑प॒स्या᳚स्तस्यो॑प॒धाय॑ वय॒स्या॑ उप॑ दध्याथ्सं॒ज्ञान॑मे॒वास्मै॑ प॒शुभिः॑ करोति पशु॒माने॒व भ॑वति॒ चत॑स्रः पु॒रस्ता॒दुप॑ दधाति॒ तस्मा᳚च्च॒त्वारि॒ चक्षु॑षो रू॒पाणि॒ द्वे शु॒क्ले द्वे कृ॒ष्णे~(४)

%5.3.1.5
मू॒र्ध॒न्वती᳚र्भवन्ति॒ तस्मा᳚त्पु॒रस्ता᳚न्मू॒र्धा पञ्च॒ दक्षि॑णाया॒ꣴ॒ श्रोण्या॒मुप॑ दधाति॒ पञ्चोत्त॑रस्यां॒ तस्मा᳚त्प॒श्चाद्वर्\mbox{}षी॑यान् पु॒रस्ता᳚त्प्रवणः प॒शुर्ब॒स्तो वय॒ इति॒ दक्षि॒णे\-ऽꣳस॒ उप॑ दधाति वृ॒ष्णिर्वय॒ इत्युत्त॒रे\-ऽꣳसा॑वे॒व प्रति॑ दधाति व्या॒घ्रो वय॒ इति॒ दक्षि॑णे प॒क्ष उप॑ दधाति सि॒ꣳ॒हो वय॒ इत्युत्त॑रे प॒क्षयो॑रे॒व वी॒र्यं॑ दधाति॒ पुरु॑षो॒ वय॒ इति॒ मध्ये॒ तस्मा॒त्पुरु॑षः पशू॒नामधि॑पतिः॥~(५)

{\anuvakamend[{कॢप्त्या॑ उप॒धाय॑ प्राण॒भृतः॒ सम॑नसः कृ॒ष्णे पुरु॑षो॒ वय॒ इति॒ पञ्च॑ च}]}%~(१)

%5.3.2.1
इन्द्रा᳚ग्नी॒ अव्य॑थमाना॒मिति॑ स्वयमातृ॒ण्णामुप॑ दधातीन्द्रा॒ग्निभ्यां॒ वा इ॒मौ लो॒कौ विधृ॑ताव॒नयो᳚र्लो॒कयो॒र्विधृ॑त्या॒ अधृ॑तेव॒ वा ए॒षा यन्म॑ध्य॒मा चिति॑र॒न्तरि॑क्षमिव॒ वा ए॒षेन्द्रा᳚ग्नी॒ इत्या॑हेन्द्रा॒ग्नी वै दे॒वाना॑मोजो॒भृता॒वोज॑सै॒वैना॑\-म॒न्तरि॑क्षे चिनुते॒ धृत्यै᳚ स्वयमातृ॒ण्णामुप॑ दधात्य॒न्तरि॑क्षं॒ वै स्व॑यमातृ॒ण्णान्तरि॑क्षमे॒वोप॑ ध॒त्ते\-ऽश्व॒मुप॑~(६)

%5.3.2.2
घ्रा॒प॒य॒ति॒ प्रा॒णमे॒वास्यां᳚ दधा॒त्यथो᳚ प्राजाप॒त्यो वा अश्वः॑ प्र॒जा\-प॑तिनै॒वाग्निं चि॑नुते स्वयमातृ॒ण्णा भ॑वति प्रा॒णाना॒मुथ्सृ॑ष्ट्या॒ अथो॑ सुव॒र्गस्य॑ लो॒कस्यानु॑ख्यात्यै दे॒वानां॒ वै सु॑व॒र्गं लो॒कं य॒तां दिशः॒ सम॑व्लीयन्त॒ त ए॒ता दिश्या॑ अपश्य॒न्ता उपा॑दधत॒ ताभि॒र्वै ते दिशो॑\-ऽदृꣳह॒न्॒यद्दिश्या॑ उप॒दधा॑ति दि॒शां विधृ॑त्यै॒ दश॑ प्राण॒भृतः॑ पु॒रस्ता॒दुप॑~(७)

%5.3.2.3
द॒धा॒ति॒ नव॒ वै पुरु॑षे प्रा॒णा नाभि॑र्दश॒मी प्रा॒णाने॒व पु॒रस्ता᳚द्धत्ते॒ तस्मा᳚त्पु॒रस्ता᳚त्प्रा॒णा ज्योति॑ष्मतीमुत्त॒मामुप॑ दधाति॒ तस्मा᳚त्प्रा॒णानां॒ वाग्ज्योति॑रुत्त॒मा दशोप॑ दधाति॒ दशा᳚क्षरा वि॒राड्वि॒राट्छन्द॑सां॒ ज्योति॒र्ज्योति॑रे॒व पु॒रस्ता᳚द्धत्ते॒ तस्मा᳚त्पु॒रस्ता॒ज्ज्योति॒रुपा᳚स्महे॒ छन्दाꣳ॑सि प॒शुष्वा॒जिम॑यु॒स्तान्बृ॑ह॒त्युद॑जय॒त्तस्मा॒द्बार्\mbox{}ह॑ताः~(८)

%5.3.2.4
प॒शव॑ उच्यन्ते॒ मा छन्द॒ इति॑ दक्षिण॒त उप॑ दधाति॒ तस्मा᳚द्दक्षि॒णावृ॑तो॒ मासाः᳚ पृथि॒वी छन्द॒ इति॑ प॒श्चात्प्रति॑ष्ठित्या अ॒ग्निर्दे॒वतेत्यु॑त्तर॒त ओजो॒ वा अ॒ग्निरोज॑ ए॒वोत्त॑र॒तो ध॑त्ते॒ तस्मा॑दुत्तरतोभिप्रया॒यी ज॑यति॒ षट्त्रिꣳ॑श॒थ्सं प॑द्यन्ते॒ षट्त्रिꣳ॑शदक्षरा बृह॒ती बार्\mbox{}ह॑ताः प॒शवो॑ बृह॒त्यैवास्मै॑ प॒शूनव॑ रुन्धे बृह॒ती छन्द॑सा॒ꣴ॒ स्वारा᳚ज्यं॒ परी॑याय॒ यस्यै॒ताः~(९)

%5.3.2.5
उ॒प॒धी॒यन्ते॒ गच्छ॑ति॒ स्वारा᳚ज्यꣳ स॒प्त वाल॑खिल्याः पु॒रस्ता॒दुप॑ दधाति स॒प्त प॒श्चाथ्स॒प्त वै शी॑र्\mbox{}ष॒ण्याः᳚ प्रा॒णा द्वाववा᳚ञ्चौ प्रा॒णानाꣳ॑ सवीर्य॒त्वाय॑ मू॒र्धासि॒ राडिति॑ पु॒रस्ता॒दुप॑ दधाति॒ यन्त्री॒ राडिति॑ प॒श्चात्प्रा॒णाने॒वास्मै॑ स॒मीचो॑ दधाति॥~(१०)

{\anuvakamend[{अश्व॒मुप॑ पु॒रस्ता॒दुप॒ बार्\mbox{}ह॑ता ए॒ताश्चतु॑स्त्रिꣳशच्च}]}%~(२)

%5.3.3.1
दे॒वा वै यद्य॒ज्ञे\-ऽकु॑र्वत॒ तदसु॑रा अकुर्वत॒ ते दे॒वा ए॒ता अ॑क्ष्णयास्तो॒मीया॑ अपश्य॒न्ता अ॒न्यथा॒नूच्या॒न्यथोपा॑दधत॒ तदसु॑रा॒ नान्ववा॑य॒न्ततो॑ दे॒वा अभ॑व॒न्परासु॑रा॒ यद॑क्ष्णयास्तो॒मीया॑ अ॒न्यथा॒नूच्या॒न्यथो॑प॒दधा॑ति॒ भ्रातृ॑व्याभिभूत्यै॒ भव॑त्या॒त्मना॒ परा᳚स्य॒ भ्रातृ॑व्यो भवत्या॒शुस्त्रि॒वृदिति॑ पु॒रस्ता॒दुप॑ दधाति य॒ज्ञमु॒खं वै त्रि॒वृत्~(११)

%5.3.3.2
य॒ज्ञ॒मु॒खमे॒व पु॒रस्ता॒द्वि या॑तयति॒ व्यो॑म सप्तद॒श इति॑ दक्षिण॒तो\-ऽन्नं॒ वै व्यो॑मान्नꣳ॑ सप्तद॒शो\-ऽन्न॑मे॒व द॑क्षिण॒तो ध॑त्ते॒ तस्मा॒द्दक्षि॑णे॒नान्न॑मद्यते ध॒रुण॑ एकवि॒ꣳ॒श इति॑ प॒श्चात्प्र॑ति॒ष्ठा वा ए॑कवि॒ꣳ॒शः प्रति॑ष्ठित्यै भा॒न्तः प॑ञ्चद॒श इत्यु॑त्तर॒त ओजो॒ वै भा॒न्त ओजः॑ पञ्चद॒श ओज॑ ए॒वोत्त॑र॒तो ध॑त्ते॒ तस्मा॑दुत्तरतोभिप्रया॒यी ज॑यति॒ प्रतू᳚र्तिरष्टाद॒श इति॑ पु॒रस्ता᳚त्~(१२)

%5.3.3.3
उप॑ दधाति॒ द्वौ त्रि॒वृता॑वभिपू॒र्वं य॑ज्ञमु॒खे वि या॑तयत्यभिव॒र्तः स॑वि॒ꣳ॒श इति॑ दक्षिण॒तो\-ऽन्नं॒ वा अ॑भिव॒र्तो\-ऽन्नꣳ॑ सवि॒ꣳ॒शो\-ऽन्न॑मे॒व द॑क्षिण॒तो ध॑त्ते॒ तस्मा॒द्दक्षि॑णे॒नान्न॑मद्यते॒ वर्चो᳚ द्वावि॒ꣳ॒श इति॑ प॒श्चाद्यद्विꣳ॑श॒तिर्द्वे तेन॑ वि॒राजौ॒ यद्द्वे प्र॑ति॒ष्ठा तेन॑ वि॒राजो॑रे॒वाभि॑पू॒र्वम॒न्नाद्ये॒ प्रति॑ तिष्ठति॒ तपो॑ नवद॒श इत्यु॑त्तर॒तस्तस्मा᳚थ्स॒व्यः~(१३)

%5.3.3.4
हस्त॑योस्तप॒स्वित॑रो॒ योनि॑श्चतुर्वि॒ꣳ॒श इति॑ पु॒रस्ता॒दुप॑ दधाति॒ चतु॑र्विꣳशत्यक्षरा गाय॒त्री गा॑य॒त्री य॑ज्ञमु॒खं य॑ज्ञमु॒खमे॒व पु॒रस्ता॒द्वि या॑तयति॒ गर्भाः᳚ पञ्चवि॒ꣳ॒श इति॑ दक्षिण॒तो\-ऽन्नं॒ वै गर्भा॒ अन्नं॑ पञ्चवि॒ꣳ॒शो\-ऽन्न॑मे॒व द॑क्षिण॒तो ध॑त्ते॒ तस्मा॒द्दक्षि॑णे॒नान्न॑मद्यत॒ ओज॑स्त्रिण॒व इति॑ प॒श्चादि॒मे वै लो॒कास्त्रि॑ण॒व ए॒ष्वे॑व लो॒केषु॒ प्रति॑ तिष्ठति स॒म्भर॑णस्त्रयोवि॒ꣳ॒श इति॑~(१४)

%5.3.3.5
उ॒त्त॒र॒तस्तस्मा᳚थ्स॒व्यो हस्त॑योः सम्भा॒र्य॑तरः॒ क्रतु॑रेकत्रि॒ꣳ॒श इति॑ पु॒रस्ता॒दुप॑ दधाति॒ वाग्वै क्रतु॑र्यज्ञमु॒खं वाग्य॑ज्ञमु॒खमे॒व पु॒रस्ता॒द्वि या॑तयति ब्र॒ध्नस्य॑ वि॒ष्टपं॑ चतुस्त्रि॒ꣳ॒श इति॑ दक्षिण॒तो॑\-ऽसौ वा आ॑दि॒त्यो ब्र॒ध्नस्य॑ वि॒ष्टपं॑ ब्रह्मवर्च॒समे॒व द॑क्षिण॒तो ध॑त्ते॒ तस्मा॒द्दक्षि॒णो\-ऽर्धो᳚ ब्रह्मवर्च॒सित॑रः प्रति॒ष्ठा त्र॑यस्त्रि॒ꣳ॒श इति॑ प॒श्चात्प्रति॑ष्ठित्यै॒ नाकः॑ षट्त्रि॒ꣳ॒श इत्यु॑त्तर॒तः सु॑व॒र्गो वै लो॒को नाकः॑ सुव॒र्गस्य॑ लो॒कस्य॒ सम॑ष्ट्यै॥~(१५)

{\anuvakamend[{वै त्रि॒वृदिति॑ पु॒रस्ता᳚थ्स॒व्यस्त्र॑योवि॒ꣳ॒श इति॑ सुव॒र्गो वै पञ्च॑ च}]}%~(३) आ॒शुर्व्यो॑म ध॒रुणो॑ भा॒न्तः प्रतू᳚र्तिरभिव॒र्तो वर्च॒स्तपो॒ योनि॒र्गर्भा॒ ओजः॑ स॒म्भर॑णः॒ क्रतु॑र्ब्र॒ध्नस्य॑ प्रति॒ष्ठा नाकः॒ षोड॑श॥

%5.3.4.1
अ॒ग्नेर्भा॒गो॑\-ऽसीति॑ पु॒रस्ता॒दुप॑ दधाति यज्ञमु॒खं वा अ॒ग्निर्य॑ज्ञमु॒खं दी॒क्षा य॑ज्ञमु॒खं ब्रह्म॑ यज्ञमु॒खं त्रि॒वृद्य॑ज्ञमु॒खमे॒व पु॒रस्ता॒द्वि या॑तयति नृ॒चक्ष॑सां भा॒गो॑\-ऽसीति॑ दक्षिण॒तः शु॑श्रु॒वाꣳसो॒ वै नृ॒चक्ष॒सो\-ऽन्नं॑ धा॒ता जा॒तायै॒वास्मा॒ अन्न॒मपि॑ दधाति॒ तस्मा᳚ज्जा॒तो\-ऽन्न॑मत्ति ज॒नित्रꣴ॑ स्पृ॒तꣳ स॑प्तद॒शः स्तोम॒ इत्या॒हान्नं॒ वै ज॒नित्रम्᳚~(१६)

%5.3.4.2
अन्नꣳ॑ सप्तद॒शो\-ऽन्न॑मे॒व द॑क्षिण॒तो ध॑त्ते॒ तस्मा॒द्दक्षि॑णे॒नान्न॑मद्यते मि॒त्रस्य॑ भा॒गो॑\-ऽसीति॑ प॒श्चात्प्रा॒णो वै मि॒त्रो॑\-ऽपा॒नो वरु॑णः प्राणापा॒नावे॒वास्मि॑न्दधाति दि॒वो वृ॒ष्टिर्वाताः᳚ स्पृ॒ता ए॑कवि॒ꣳ॒शः स्तोम॒ इत्या॑ह प्रति॒ष्ठा वा ए॑कवि॒ꣳ॒शः प्रति॑ष्ठित्या॒ इन्द्र॑स्य भा॒गो॑\-ऽसीत्यु॑त्तर॒त ओजो॒ वा इन्द्र॒ ओजो॒ विष्णु॒रोजः॑ क्ष॒त्रमोजः॑ पञ्चद॒शः~(१७)

%5.3.4.3
ओज॑ ए॒वोत्त॑र॒तो ध॑त्ते॒ तस्मा॑दुत्तरतोभिप्रया॒यी ज॑यति॒ वसू॑नां भा॒गो॑\-ऽसीति॑ पु॒रस्ता॒दुप॑ दधाति यज्ञमु॒खं वै वस॑वो ॑यज्ञमु॒खꣳ रु॒द्रा य॑ज्ञमु॒खं च॑तुर्वि॒ꣳ॒शो य॑ज्ञमु॒खमे॒व पु॒रस्ता॒द्वि या॑तयत्यादि॒त्यानां᳚ भा॒गो॑\-ऽसीति॑ दक्षिण॒तो\-ऽन्नं॒ वा आ॑दि॒त्या अन्नं॑ म॒रुतो\-ऽन्नं॒ गर्भा॒ अन्नं॑ पञ्चवि॒ꣳ॒शो\-ऽन्न॑मे॒व द॑क्षिण॒तो ध॑त्ते॒ तस्मा॒द्दक्षि॑णे॒नान्न॑मद्य॒ते\-ऽदि॑त्यै भा॒गः~(१८)

%5.3.4.4
अ॒सीति॑ प॒श्चात्प्र॑ति॒ष्ठा वा अदि॑तिः प्रति॒ष्ठा पू॒षा प्र॑ति॒ष्ठा त्रि॑ण॒वः प्रति॑ष्ठित्यै दे॒वस्य॑ सवि॒तुर्भा॒गो॑\-ऽसीत्यु॑त्तर॒तो ब्रह्म॒ वै दे॒वः स॑वि॒ता ब्रह्म॒ बृह॒स्पति॒र्ब्रह्म॑ चतुष्टो॒मो ब्र॑ह्मवर्च॒समे॒वोत्त॑र॒तो ध॑त्ते॒ तस्मा॒दुत्त॒रो\-ऽर्धो᳚ ब्रह्मवर्च॒सित॑रः सावि॒त्रव॑ती भवति॒ प्रसू᳚त्यै॒ तस्मा᳚द्ब्राह्म॒णाना॒मुदी॑ची स॒निः प्रसू॑ता ध॒र्त्रश्च॑तुष्टो॒म इति॑ पु॒रस्ता॒दुप॑ दधाति यज्ञमु॒खं वै ध॒र्त्रः~(१९)

%5.3.4.5
य॒ज्ञ॒मु॒खं च॑तुष्टो॒मो य॑ज्ञमु॒खमे॒व पु॒रस्ता॒द्वि या॑तयति॒ यावा॑नां भा॒गो॑\-ऽसीति॑ दक्षिण॒तो मासा॒ वै यावा॑ अर्धमा॒सा अया॑वा॒स्तस्मा᳚द्दक्षि॒णावृ॑तो॒ मासा॒ अन्नं॒ वै यावा॒ अन्नं॑ प्र॒जा अन्न॑मे॒व द॑क्षिण॒तो ध॑त्ते॒ तस्मा॒द्दक्षि॑णे॒नान्न॑मद्यत ऋभू॒णां भा॒गो॑\-ऽसीति॑ प॒श्चात् प्रति॑ष्ठित्यै विव॒र्तो᳚\-ऽष्टाचत्वारि॒ꣳ॒श इत्यु॑त्तर॒तो॑\-ऽनयो᳚र्लो॒कयोः᳚ सवीर्य॒त्वाय॒ तस्मा॑दि॒मौ लो॒कौ स॒माव॑द्वीर्यौ~(२०)

%5.3.4.6
यस्य॒ मुख्य॑वतीः पु॒रस्ता॑दुपधी॒यन्ते॒ मुख्य॑ ए॒व भ॑व॒त्यास्य॒ मुख्यो॑ जायते॒ यस्यान्न॑वतीर्दक्षिण॒तो\-ऽत्त्यन्न॒मास्या᳚न्ना॒दो जा॑यते॒ यस्य॑ प्रति॒ष्ठाव॑तीः प॒श्चात्प्रत्ये॒व ति॑ष्ठति॒ यस्यौज॑स्वतीरुत्तर॒त ओ॑ज॒स्व्ये॑व भ॑व॒त्यास्यौ॑ज॒स्वी जा॑यते॒\-ऽर्को वा ए॒ष यद॒ग्निस्तस्यै॒तदे॒व स्तो॒त्रमे॒तच्छ॒स्त्रं यदे॒षा वि॒धा~(२१)

%5.3.4.7
वि॒धी॒यते॒\-ऽर्क ए॒व तद॒र्क्य॑मनु॒ वि धी॑य॒ते\-ऽत्त्यन्न॒मास्या᳚न्ना॒दो जा॑यते॒ यस्यै॒षा वि॒धा वि॑धी॒यते॒ य उ॑ चैनामे॒वं वेद॒ सृष्टी॒रुप॑ दधाति यथासृ॒ष्टमे॒वाव॑ रुन्धे॒ न वा इ॒दं दिवा॒ न नक्त॑मासी॒दव्या॑वृत्त॒न्ते दे॒वा ए॒ता व्यु॑ष्टीरपश्य॒न्ता उपा॑दधत॒ ततो॒ वा इ॒दं व्यौ᳚च्छ॒द्यस्यै॒ता उ॑पधी॒यन्ते॒ व्ये॑वास्मा॑ उच्छ॒त्यथो॒ तम॑ ए॒वाप॑ हते॥~(२२)

{\anuvakamend[{वै ज॒नित्रं॑ पञ्चद॒शो\-ऽदि॑त्यै भा॒गो वै ध॒र्त्रः स॒माव॑द्वीर्यौ वि॒धा ततो॒ वा इ॒दं चतु॑र्दश च}]}%~(४) अ॒ग्नेर्नृ॒चक्ष॑साञ्ज॒नित्रं॑ मि॒त्रस्येन्द्र॑स्य॒ वसू॑नामादि॒त्याना॒मदि॑त्यै दे॒वस्य॑ सवि॒तुः सा॑वि॒त्रव॑ती ध॒र्त्रो यावा॑नामृभू॒णां वि॑व॒र्तश्चतु॑र्दश॥

%5.3.5.1
अग्ने॑ जा॒तान्प्र णु॑दा नः स॒पत्ना॒निति॑ पु॒रस्ता॒दुप॑ दधाति जा॒ताने॒व भ्रातृ॑व्या॒न्प्र णु॑दते॒ सह॑सा जा॒तानिति॑ प॒श्चाज्ज॑नि॒ष्यमा॑णाने॒व प्रति॑ नुदते चतुश्चत्वारि॒ꣳ॒शः स्तोम॒ इति॑ दक्षिण॒तो ब्र॑ह्मवर्च॒सं वै च॑तुश्चत्वारि॒ꣳ॒शो ब्र॑ह्मवर्च॒समे॒व द॑क्षिण॒तो ध॑त्ते॒ तस्मा॒द्दक्षि॒णो\-ऽर्धो᳚ ब्रह्मवर्च॒सित॑रः षोड॒शः स्तोम॒ इत्यु॑त्तर॒त ओजो॒ वै षो॑ड॒श ओज॑ ए॒वोत्त॑र॒तो ध॑त्ते॒ तस्मा᳚त्~(२३)

%5.3.5.2
उ॒त्त॒र॒तो॒भि॒प्र॒या॒यी ज॑यति॒ वज्रो॒ वै च॑तुश्चत्वारि॒ꣳ॒शो वज्रः॑ षोड॒शो यदे॒ते इष्ट॑के उप॒दधा॑ति जा॒ताꣴश्चै॒व ज॑नि॒ष्यमा॑णाꣴश्च॒ भ्रातृ॑व्यान्प्र॒णुद्य॒ वज्र॒मनु॒ प्र ह॑रति॒ स्तृत्यै॒ पुरी॑षवतीं॒ मध्य॒ उप॑ दधाति॒ पुरी॑षं॒ वै मध्य॑मा॒त्मनः॒ सात्मा॑नमे॒वाग्निं चि॑नुते॒ सात्मा॒मुष्मिँ॑ल्लो॒के भ॑वति॒ य ए॒वं वेदै॒ता वा अ॑सप॒त्ना नामेष्ट॑का॒ यस्यै॒ता उ॑पधी॒यन्ते᳚~(२४)

%5.3.5.3
नास्य॑ स॒पत्नो॑ भवति प॒शुर्वा ए॒ष यद॒ग्निर्वि॒राज॑ उत्त॒मायां॒ चित्या॒मुप॑ दधाति वि॒राज॑मे॒वोत्त॒मां प॒शुषु॑ दधाति॒ तस्मा᳚त्पशु॒मानु॑त्त॒मां वाचं॑ वदति॒ दश॑द॒शोप॑ दधाति सवीर्य॒\-त्वा\-या᳚क्ष्ण॒\-योप॑ दधाति॒ तस्मा॑दक्ष्ण॒या प॒शवो\-ऽङ्गा॑नि॒ प्र ह॑रन्ति॒ प्रति॑ष्ठित्यै॒ यानि॒ वै छन्दाꣳ॑सि सुव॒र्ग्या᳚ण्यास॒न्तैर्दे॒वाः सु॑व॒र्गं लो॒कमा॑य॒न्तेनर्\mbox{}ष॑यः~(२५)

%5.3.5.4
अ॒श्रा॒म्य॒न्ते तपो॑\-ऽतप्यन्त॒ तानि॒ तप॑सापश्य॒न्तेभ्य॑ ए॒ता इष्ट॑का॒ निर॑मिम॒तेव॒श्छन्दो॒ वरि॑व॒श्छन्द॒ इति॒ ता उपा॑दधत॒ ताभि॒र्वै ते सु॑व॒र्गं लो॒कमा॑य॒न्॒ यदे॒ता इष्ट॑का उप॒दधा॑ति॒ यान्ये॒व छन्दाꣳ॑सि सुव॒र्ग्या॑णि॒ तैरे॒व यज॑मानः सुव॒र्गं लो॒कमे॑ति य॒ज्ञेन॒ वै प्र॒जा\-प॑तिः प्र॒जा अ॑सृजत॒ ताः स्तोम॑भागैरे॒वासृ॑जत॒ यत्~(२६)

%5.3.5.5
स्तोम॑भागा उप॒दधा॑ति प्र॒जा ए॒व तद्यज॑मानः सृजते॒ बृह॒स्पति॒र्वा ए॒तद्य॒ज्ञस्य॒ तेजः॒ सम॑भर॒द्यथ्स्तोम॑भागा॒ यथ्स्तोम॑भागा उप॒दधा॑ति॒ सते॑जसमे॒वाग्निं चि॑नुते॒ बृह॒स्पति॒र्वा ए॒तां य॒ज्ञस्य॑ प्रति॒ष्ठाम॑पश्य॒द्यथ्स्तोम॑भागा॒ यथ्स्तोम॑भागा उप॒दधा॑ति य॒ज्ञस्य॒ प्रति॑ष्ठित्यै स॒प्तस॒प्तोप॑ दधाति सवीर्य॒त्वाय॑ ति॒स्रो मध्ये॒ प्रति॑ष्ठित्यै॥~(२७)

{\anuvakamend[{उ॒त्त॒र॒तो ध॑त्ते॒ तस्मा॑दुपधी॒यन्त॒ ऋष॑यो\-ऽसृजत॒ यत्त्रिच॑त्वारिꣳशच्च}]}%~(५)

%5.3.6.1
र॒श्मिरित्ये॒वा\-ऽऽ\-दि॒त्यम॑सृजत॒ प्रेति॒रिति॒ धर्म॒मन्वि॑ति॒रिति॒ दिवꣳ॑ सं॒धिरित्य॒न्तरि॑क्षं प्रति॒धिरिति॑ पृथि॒वीं वि॑ष्ट॒म्भ इति॒ वृष्टिं॑ प्र॒वेत्यह॑रनु॒वेति॒ रात्रि॑मु॒शिगिति॒ वसू᳚न्प्रके॒त इति॑ रु॒द्रान्थ्सु॑दी॒तिरित्या॑दि॒त्यानोज॒ इति॑ पि॒तॄꣴस्तन्तु॒रिति॑ प्र॒जाः पृ॑तना॒षाडिति॑ प॒शून्रे॒वदित्योष॑धीरभि॒जिद॑सि यु॒क्तग्रा॑वा~(२८)

%5.3.6.2
इन्द्रा॑य॒ त्वेन्द्रं॑ जि॒न्वेत्ये॒व द॑क्षिण॒तो वज्रं॒ पर्यौ॑हद॒भिजि॑त्यै॒ ताः प्र॒जा अप॑प्राणा असृजत॒ तास्वधि॑पतिर॒सीत्ये॒व प्रा॒णम॑दधाद्य॒न्तेत्य॑पा॒नꣳ स॒ꣳ॒सर्प॒ इति॒ चक्षु॑र्वयो॒धा इति॒ श्रोत्र॒न्ताः प्र॒जाः प्रा॑ण॒तीर॑पान॒तीः पश्य॑न्तीः शृण्व॒तीर्न मि॑थु॒नी अ॑भव॒न्तासु॑ त्रि॒वृद॒सीत्ये॒व मि॑थु॒नम॑दधा॒त्ताः प्र॒जा मि॑थु॒नी~(२९)

%5.3.6.3
भव॑न्ती॒र्न प्राजा॑यन्त॒ ताः सꣳ॑रो॒हो॑\-ऽसि नीरो॒हो॑\-ऽसीत्ये॒व प्राज॑नय॒त्ताः प्र॒जाः प्रजा॑ता॒ न प्रत्य॑तिष्ठ॒न्ता व॑सु॒को॑\-ऽसि॒ वेष॑श्रिरसि॒ वस्य॑ष्टिर॒सीत्ये॒वैषु लो॒केषु॒ प्रत्य॑स्थापय॒द्यदाह॑ वसु॒को॑\-ऽसि॒ वेष॑श्रिरसि॒ वस्य॑ष्टिर॒सीति॑ प्र॒जा ए॒व प्रजा॑ता ए॒षु लो॒केषु॒ प्रति॑\-ष्ठापयति॒ सात्मा॒न्तरि॑क्षꣳ रोहति॒ सप्रा॑णो॒\-ऽमुष्मिँ॑ल्लो॒के प्रति॑ तिष्ठ॒त्यव्य॑र्धुकः प्राणापा॒ना\-भ्यां᳚ भवति॒ य ए॒वं वेद॑॥~(३०)

{\anuvakamend[{यु॒क्तग्रा॑वा प्र॒जा मि॑थु॒न्य॑न्तरि॑क्षं॒ द्वाद॑श च}]}%~(६)

%5.3.7.1
ना॒क॒सद्भि॒र्वै दे॒वाः सु॑व॒र्गं लो॒कमा॑य॒न्तन्ना॑क॒सदां᳚ नाकस॒त्त्वं यन्ना॑क॒सद॑ उप॒दधा॑ति नाक॒सद्भि॑रे॒व तद्यज॑मानः सुव॒र्गं लो॒कमे॑ति सुव॒र्गो वै लो॒को नाको॒ यस्यै॒ता उ॑पधी॒यन्ते॒ नास्मा॒ अकं॑ भवति यजमानायत॒नं वै ना॑क॒सदो॒ यन्ना॑क॒सद॑ उप॒दधा᳚त्या॒यत॑नमे॒व तद्यज॑मानः कुरुते पृ॒ष्ठानां॒ वा ए॒तत्तेजः॒ सम्भृ॑तं॒ यन्ना॑क॒सदो॒ यन्ना॑क॒सदः॑~(३१)

%5.3.7.2
उ॒प॒दधा॑ति पृ॒ष्ठाना॑मे॒व तेजो\-ऽव॑ रुन्धे पञ्च॒चोडा॒ उप॑ दधात्यफ्स॒रस॑ ए॒वैन॑मे॒ता भू॒ता अ॒मुष्मिँ॑ल्लो॒क उप॑ शे॒रे\-ऽथो॑ तनू॒पानी॑रे॒वैता यज॑मानस्य॒ यं द्वि॒ष्यात्तमु॑प॒दध॑द्ध्यायेदे॒ताभ्य॑ ए॒वैनं॑ दे॒वता᳚भ्य॒ आ वृ॑श्चति ता॒जगार्ति॒मार्च्छ॒त्युत्त॑रा नाक॒सद्भ्य॒ उप॑ दधाति॒ यथा॑ जा॒यामा॒नीय॑ गृ॒हेषु॑ निषा॒दय॑ति ता॒दृगे॒व तत्~(३२)

%5.3.7.3
प॒श्चात्प्राची॑मुत्त॒मामुप॑ दधाति॒ तस्मा᳚त्प॒श्चात्प्राची॒ पत्न्यन्वा᳚स्ते स्वयमातृ॒ण्णां च॑ विक॒र्णीं चो᳚त्त॒मे उप॑ दधाति प्रा॒णो वै स्व॑यमातृ॒ण्णायु॑र्विक॒र्णी प्रा॒णं चै॒वायु॑श्च प्रा॒णाना॑मुत्त॒मौ ध॑त्ते॒ तस्मा᳚त्प्रा॒णश्चायु॑श्च प्रा॒णाना॑मुत्त॒मौ नान्यामुत्त॑रा॒मिष्ट॑का॒मुप॑ दध्या॒द्यद॒न्यामुत्त॑रा॒मिष्ट॑कामुपद॒ध्यात्प॑शू॒नाम्~(३३)

%5.3.7.4
च॒ यज॑मानस्य च प्रा॒णं चायु॒श्चापि॑ दध्या॒त्तस्मा॒न्नान्योत्त॒रेष्ट॑कोप॒धेया᳚ स्वयमातृ॒ण्णामुप॑ दधात्य॒सौ वै स्व॑यमातृ॒ण्णामूमे॒वोप॑ ध॒त्ते\-ऽश्व॒मुप॑ घ्रापयति प्रा॒णमे॒वास्यां᳚ दधा॒त्यथो᳚ प्राजाप॒त्यो वा अश्वः॑ प्र॒जा\-प॑तिनै॒वाग्निं चि॑नुते स्वयमातृ॒ण्णा भ॑वति प्रा॒णाना॒मुथ्सृ॑ष्ट्या॒ अथो॑ सुव॒र्गस्य॑ लो॒कस्यानु॑ख्यात्या ए॒षा वै दे॒वानां॒ विक्रा᳚न्ति॒र्यद्वि॑क॒र्णी यद्वि॑क॒र्णीमु॑प॒दधा॑ति दे॒वाना॑मे॒व विक्रा᳚न्ति॒मनु॒ वि क्र॑मत उत्तर॒त उप॑ दधाति॒ तस्मा॑दुत्तर॒तउ॑पचारो॒\-ऽग्निर्वा॑यु॒मती॑ भवति॒ समि॑द्ध्यै॥~(३४)

{\anuvakamend[{सम्भृ॑तं॒ यन्ना॑क॒सदो॒ यन्ना॑क॒सद॒स्तत्प॑शू॒नामे॒षां वै द्वाविꣳ॑शतिश्च}]}%~(७)

%5.3.8.1
छन्दा॒ꣴ॒स्युप॑ दधाति प॒शवो॒ वै छन्दाꣳ॑सि प॒शूने॒वाव॑ रुन्धे॒ छन्दाꣳ॑सि॒ वै दे॒वानां᳚ वा॒मं प॒शवो॑ वा॒ममे॒व प॒शूनव॑ रुन्ध ए॒ताꣳ ह॒ वै य॒ज्ञसे॑नश्चैत्रियाय॒णश्चितिं॑ वि॒दां च॑कार॒ तया॒ वै स प॒शूनवा॑रुन्ध॒ यदे॒तामु॑प॒दधा॑ति प॒शूने॒वाव॑ रुन्धे गाय॒त्रीः पु॒रस्ता॒दुप॑ दधाति॒ तेजो॒ वै गा॑य॒त्री तेज॑ ए॒व~(३५)

%5.3.8.2
मु॒ख॒तो ध॑त्ते मूर्ध॒न्वती᳚र्भवन्ति मू॒र्धान॑मे॒वैनꣳ॑ समा॒नानां᳚ करोति त्रि॒ष्टुभ॒ उप॑ दधातीन्द्रि॒यं वै त्रि॒ष्टुगि॑न्द्रि॒यमे॒व म॑ध्य॒तो ध॑त्ते॒ जग॑ती॒रुप॑ दधाति॒ जाग॑ता॒ वै प॒शवः॑ प॒शूने॒वाव॑ रुन्धे\-ऽनु॒ष्टुभ॒ उप॑ दधाति प्रा॒णा वा अ॑नु॒ष्टुप्प्रा॒णाना॒मुथ्सृ॑ष्ट्यै बृह॒तीरु॒ष्णिहाः᳚ प॒ङ्क्तीर॒क्षर॑पङ्क्ती॒रिति॒ विषु॑रूपाणि॒ छन्दा॒ꣴ॒स्युप॑ दधाति॒ विषु॑रूपा॒ वै प॒शवः॑ प॒शवः॑~(३६)

%5.3.8.3
छन्दाꣳ॑सि॒ विषु॑रूपाने॒व प॒शूनव॑ रुन्धे॒ विषु॑रूपमस्य गृ॒हे दृ॑श्यते॒ यस्यै॒ता उ॑पधी॒यन्ते॒ य उ॑ चैना ए॒वं वेदाति॑च्छन्दस॒मुप॑ दधा॒त्यति॑च्छन्दा॒ वै सर्वा॑णि॒ छन्दाꣳ॑सि॒ सर्वे॑भिरे॒वैनं॒ छन्दो॑भिश्चिनुते॒ वर्ष्म॒ वा ए॒षा छन्द॑सां॒ यदति॑च्छन्दा॒ यदति॑च्छन्दसमुप॒दधा॑ति॒ वर्ष्मै॒वैनꣳ॑ समा॒नानां᳚ करोति द्वि॒पदा॒ उप॑ दधाति द्वि॒पाद्यज॑मानः॒ प्रति॑ष्ठित्यै॥~(३७)

{\anuvakamend[{तेज॑ ए॒व प॒शवः॑ प॒शवो॒ यज॑मान॒ एक॑ञ्च}]}%~(८)

%5.3.9.1
सर्वा᳚भ्यो॒ वै दे॒वता᳚भ्यो॒\-ऽग्निश्ची॑यते॒ यथ्स॒युजो॒ नोप॑द॒ध्याद्दे॒वता॑ अस्या॒ग्निं वृ॑ञ्जीर॒न्॒ यथ्स॒युज॑ उप॒दधा᳚त्या॒त्मनै॒वैनꣳ॑ स॒युजं॑ चिनुते॒ नाग्निना॒ व्यृ॑ध्य॒ते\-ऽथो॒ यथा॒ पुरु॑षः॒ स्नाव॑भिः॒ सन्त॑त ए॒वमे॒वैताभि॑र॒ग्निः सन्त॑तो॒\-ऽग्निना॒ वै दे॒वाः सु॑व॒र्गं लो॒कमा॑य॒न्ता अ॒मूः कृ॑त्तिका अभव॒न्॒ यस्यै॒ता उ॑पधी॒यन्ते॑ सुव॒र्गमे॒व~(३८)

%5.3.9.2
लो॒कमे॑ति॒ गच्छ॑ति प्रका॒शं चि॒त्रमे॒व भ॑वति मण्डलेष्ट॒का उप॑ दधाती॒मे वै लो॒का म॑ण्डलेष्ट॒का इ॒मे खलु॒ वै लो॒का दे॑वपु॒रा दे॑वपु॒रा ए॒व प्र वि॑शति॒ नार्ति॒मार्च्छ॑त्य॒ग्निं चि॑क्या॒नो वि॒श्वज्यो॑तिष॒ उप॑ दधाती॒माने॒वैताभि॑र्लो॒कां ज्योति॑ष्मतः कुरु॒ते\-ऽथो᳚ प्रा॒णाने॒वैता यज॑मानस्य दाध्रत्ये॒ता वै दे॒वताः᳚ सुव॒र्ग्या᳚स्ता ए॒वान्वा॒रभ्य॑ सुव॒र्गं लो॒कमे॑ति॥~(३९)

{\anuvakamend[{सु॒व॒र्गमे॒व ता ए॒व च॒त्वारि॑ च}]}%~(९)

%5.3.10.1
वृ॒ष्टि॒सनी॒रुप॑ दधाति॒ वृष्टि॑मे॒वाव॑ रुन्धे॒ यदे॑क॒धोप॑द॒ध्यादेक॑मृ॒तुं व॑र्\mbox{}षेदनुपरि॒हारꣳ॑ सादयति॒ तस्मा॒थ्सर्वा॑नृ॒तून् व॑र्\mbox{}षति पुरोवात॒सनि॑र॒सीत्या॑है॒तद्वै वृष्ट्यै॑ रू॒पꣳ रू॒पेणै॒व वृष्टि॒मव॑ रुन्धे सं॒यानी॑भि॒र्वै दे॒वा इ॒माँल्लो॒कान्थ्सम॑यु॒स्तथ्सं॒यानी॑नाꣳ संयानि॒त्वं यथ्सं॒यानी॑रुप॒दधा॑ति॒ यथा॒फ्सु ना॒वा सं॒यात्ये॒वम्~(४०)

%5.3.10.2
ए॒वैताभि॒र्यज॑मान इ॒माँल्लो॒कान्थ्सं या॑ति प्ल॒वो वा ए॒षो᳚\-ऽग्नेर्यथ्सं॒यानी॒र्यथ्सं॒यानी॑रुप॒दधा॑ति प्ल॒वमे॒वैतम॒ग्नय॒ उप॑ दधात्यु॒त यस्यै॒तासूप॑हिता॒स्वापो॒\-ऽग्निꣳ हर॒न्त्यहृ॑त ए॒वास्या॒ग्निरा॑दित्येष्ट॒का उप॑ दधात्यादि॒त्या वा ए॒तं भूत्यै॒ प्रति॑ नुदन्ते॒ यो\-ऽलं॒ भूत्यै॒ सन्भूतिं॒ न प्रा॒प्नोत्या॑दि॒त्याः~(४१)

%5.3.10.3
ए॒वैनं॒ भूतिं॑ गमयन्त्य॒सौ वा ए॒तस्या॑दि॒त्यो रुच॒मा द॑त्ते॒ यो᳚\-ऽग्निं चि॒त्वा न रोच॑ते॒ यदा॑दित्येष्ट॒का उ॑प॒दधा᳚त्य॒सावे॒वास्मि॑न्नादि॒त्यो रुचं॑ दधाति॒ यथा॒सौ दे॒वाना॒ꣳ॒ रोच॑त ए॒वमे॒वैष म॑नु॒ष्या॑णाꣳ रोचते घृतेष्ट॒का उप॑ दधात्ये॒तद्वा अ॒ग्नेः प्रि॒यं धाम॒ यद् घृ॒तं प्रि॒येणै॒वैनं॒ धाम्ना॒ सम॑र्धयति~(४२)

%5.3.10.4
अथो॒ तेज॑सानुपरि॒हारꣳ॑ सादय॒त्यप॑रिवर्गमे॒वास्मि॒न्तेजो॑ दधाति प्र॒जा\-प॑तिर॒ग्निम॑चिनुत॒ स यश॑सा॒ व्या᳚र्ध्यत॒ स ए॒ता य॑शो॒दा अ॑पश्य॒त्ता उपा॑धत्त॒ ताभि॒र्वै स यश॑ आ॒त्मन्न॑धत्त॒ यद्य॑शो॒दा उ॑प॒दधा॑ति॒ यश॑ ए॒व ताभि॒र्यज॑मान आ॒त्मन्ध॑त्ते॒ पञ्चोप॑ दधाति॒ पाङ्क्तः॒ पुरु॑षो॒ यावा॑ने॒व पुरु॑ष॒स्तस्मि॒न्॒ यशो॑ दधाति॥~(४३)

{\anuvakamend[{ए॒वं प्रा॒प्नोत्या॑दि॒त्या अ॑र्धय॒त्येका॒न्नप॑ञ्चा॒शच्च॑}]}%॥10॥

%5.3.11.1
दे॒वा॒सु॒राः संय॑त्ता आस॒न्कनी॑याꣳसो दे॒वा आस॒न्भूया॒ꣳ॒सो\-ऽसु॑रा॒स्ते दे॒वा ए॒ता इष्ट॑का अपश्य॒न्ता उपा॑दधत भूय॒स्कृद॒सीत्ये॒व भूयाꣳ॑सो\-ऽभव॒न्वन॒स्पति॑भि॒रोष॑धीभिर्वरिव॒स्कृद॒सीती॒माम॑जय॒न्प्राच्य॒सीति॒ प्राचीं॒ दिश॑मजयन्नू॒र्ध्वासीत्य॒मूम॑जयन्नन्तरिक्ष॒सद॑स्य॒न्तरि॑क्षे सी॒देत्य॒न्तरि॑क्षमजय॒न्ततो॑ दे॒वा अभ॑वन्न्~(४४)

%5.3.11.2
परासु॑रा॒ यस्यै॒ता उ॑पधी॒यन्ते॒ भूया॑ने॒व भ॑वत्य॒भीमाँल्लो॒काञ्ज॑यति॒ भव॑त्या॒त्मना॒ परा᳚स्य॒ भ्रातृ॑व्यो भवत्यफ्सु॒षद॑सि श्येन॒सद॒सीत्या॑है॒तद्वा अ॒ग्ने रू॒पꣳ रू॒पेणै॒वाग्निमव॑ रुन्धे पृथि॒व्यास्त्वा॒ द्रवि॑णे सादया॒मीत्या॑हे॒माने॒वैताभि॑र्लो॒कान् द्रवि॑णावतः कुरुत आयु॒ष्या॑ उप॑ दधा॒त्यायु॑रे॒व~(४५)

%5.3.11.3
अ॒स्मि॒न्द॒धा॒त्यग्ने॒ यत्ते॒ पर॒ꣳ॒ हृन्नामेत्या॑है॒तद्वा अ॒ग्नेः प्रि॒यं धाम॑ प्रि॒यमे॒वास्य॒ धामोपा᳚प्नोति॒ तावेहि॒ सꣳ र॑भावहा॒ इत्या॑ह॒ व्ये॑वैने॑न॒ परि॑ धत्ते॒ पाञ्च॑जन्ये॒ष्वप्ये᳚ध्यग्न॒ इत्या॑है॒ष वा अ॒ग्निः पाञ्च॑जन्यो॒ यः पञ्च॑चितीक॒स्तस्मा॑दे॒वमा॑हर्त॒व्या॑ उप॑ दधात्ये॒तद्वा ऋ॑तू॒नां प्रि॒यं धाम॒ यदृ॑त॒व्या॑ ऋतू॒नामे॒व प्रि॒यं धामाव॑ रुन्धे सु॒मेक॒ इत्या॑ह संवथ्स॒रो वै सु॒मेकः॑ संवथ्स॒रस्यै॒व प्रि॒यं धामोपा᳚प्नोति॥~(४६)

{\anuvakamend[{अभ॑व॒न्नायु॑रे॒वर्त॒व्या॑ उप॒ षड्विꣳ॑शतिश्च}]}%॥11॥

%5.3.12.1
प्र॒जा\-प॑ते॒रक्ष्य॑श्वय॒त्तत्परा॑पत॒त्तदश्वो॑\-ऽभव॒द्यदश्व॑य॒त्तदश्व॑स्याश्व॒त्वन्तद्दे॒वा अ॑श्वमे॒धेनै॒व प्रत्य॑दधुरे॒ष वै प्र॒जा\-प॑ति॒ꣳ॒ सर्वं॑ करोति॒ यो᳚\-ऽश्वमे॒धेन॒ यज॑ते॒ सर्व॑ ए॒व भ॑वति॒ सर्व॑स्य॒ वा ए॒षा प्राय॑श्चित्तिः॒ सर्व॑स्य भेष॒जꣳ सर्वं॒ वा ए॒तेन॑ पा॒प्मानं॑ दे॒वा अ॑तर॒न्नपि॒ वा ए॒तेन॑ ब्रह्मह॒त्याम॑तर॒न्थ्सर्वं॑ पा॒प्मानम्᳚~(४७)

%5.3.12.2
त॒र॒ति॒ तर॑ति ब्रह्मह॒त्यां यो᳚\-ऽश्वमे॒धेन॒ यज॑ते॒ य उ॑ चैनमे॒वं वेदोत्त॑रं॒ वै तत्प्र॒जा\-प॑ते॒रक्ष्य॑श्वय॒त्तस्मा॒दश्व॑स्योत्तर॒तो\-ऽव॑ द्यन्ति दक्षिण॒तो᳚\-ऽन्येषां᳚ पशू॒नाम्वै॑त॒सः कटो॑ भवत्य॒फ्सुयो॑नि॒र्वा अश्वो᳚\-ऽफ्सु॒जो वे॑त॒सः स्व ए॒वैनं॒ योनौ॒ प्रति॑\-ष्ठापयति चतुष्टो॒मः स्तोमो॑ भवति स॒रड्ढ॒ वा अश्व॑स्य॒ सक्थ्यावृ॑ह॒त्तद्दे॒वाश्च॑तुष्टो॒मेनै॒व प्रत्य॑दधु॒र्यच्च॑तुष्टो॒मः स्तोमो॒ भव॒त्यश्व॑स्य सर्व॒त्वाय॑॥~(४८)

{\anuvakamend[{सर्वं॑ पा॒प्मान॑मवृह॒द्द्वाद॑श च}]}%॥12॥

\prashnaend{उ॒थ्स॒न्न॒य॒ज्ञ इन्द्रा᳚ग्नी दे॒वा वा अ॑क्षणयास्तो॒मीया॑ अ॒ग्नेर्भा॒गो᳚\-ऽस्यग्ने॑ जा॒तान्र॒श्मिरिति॑ नाक॒सद्भि॒श्छन्दाꣳ॑सि॒ सर्वा᳚भ्यो वृष्टि॒सनी᳚र्देवासु॒राः कनी॑याꣳसः प्र॒जा\-प॑ते॒रक्षि॒ द्वाद॑श॥१२॥}{उ॒थ्स॒न्न॒य॒ज्ञो दे॒वा वै यस्य॒ मुख्य॑वतीर्नाक॒सद्भि॑रे॒वैताभि॑र॒ष्टाच॑त्वारिꣳशत्॥४८॥}{उ॒थ्स॒न्न॒य॒ज्ञः स॑र्व॒त्वाय॑॥}%%५-३
{हरिः॑ ॐ}{॥कृष्ण-यजुर्वेदीय-तैत्तिरीय-संहितायां पञ्चम्काण्डे तृतीयः प्रश्नः समाप्तः॥५-३॥}
%%% END PRASHNA

\sect{चतुर्थः प्रश्नः}\setcounter{anuvakam}{0}
\dnsub{तैत्तिरीयसंहितायां पञ्चमकाण्डे चतुर्थः प्रश्नः}
%5.4.1.1
दे॒वा॒सु॒राः संय॑त्ता आस॒न्ते न व्य॑जयन्त॒ स ए॒ता इन्द्र॑स्त॒नूर॑पश्य॒त्ता उपा॑धत्त॒ ताभि॒र्वै स त॒नुव॑मिन्द्रि॒यं वी॑र्यमा॒त्मन्न॑धत्त॒ ततो॑ दे॒वा अभ॑व॒न्परासु॑रा॒ यदि॑न्द्रत॒नूरु॑प॒दधा॑ति त॒नुव॑मे॒व ताभि॑रिन्द्रि॒यं वी॒र्यं॑ यज॑मान आ॒त्मन्ध॒त्ते\-ऽथो॒ सेन्द्र॑मे॒वाग्निꣳ सत॑नुं चिनुते॒ भव॑त्या॒त्मना॒ परा᳚स्य॒ भ्रातृ॑व्यः~(१)

%5.4.1.2
भ॒व॒ति॒ य॒ज्ञो दे॒वेभ्यो\-ऽपा᳚क्राम॒त्तम॑व॒रुधं॒ नाश॑क्नुव॒न्त ए॒ता य॑ज्ञत॒नूर॑पश्य॒न्ता उपा॑दधत॒ ताभि॒र्वै ते य॒ज्ञमवा॑रुन्धत॒ यद्य॑ज्ञत॒नूरु॑प॒दधा॑ति य॒ज्ञमे॒व ताभि॒र्यज॑मा॒नो\-ऽव॑ रुन्धे॒ त्रय॑स्त्रिꣳशत॒मुप॑ दधाति॒ त्रय॑स्त्रिꣳश॒द्वै दे॒वता॑ दे॒वता॑ ए॒वाव॑ रु॒न्धे\-ऽथो॒ सात्मा॑नमे॒वाग्निꣳ सत॑नुं चिनुते॒ सात्मा॒मुष्मिँ॑ल्लो॒के~(२)

%5.4.1.3
भ॒व॒ति॒ य ए॒वं वेद॒ ज्योति॑ष्मती॒रुप॑ दधाति॒ ज्योति॑रे॒वास्मि॑न्दधात्ये॒ताभि॒र्वा अ॒ग्निश्चि॒तो ज्व॑लति॒ ताभि॑रे॒वैन॒ꣳ॒ समि॑न्द्ध उ॒भयो॑रस्मै लो॒कयो॒र्ज्योति॑र्भवति नक्षत्रेष्ट॒का उप॑ दधात्ये॒तानि॒ वै दि॒वो ज्योतीꣳ॑षि॒ तान्ये॒वाव॑ रुन्धे सु॒कृतां॒ वा ए॒तानि॒ ज्योतीꣳ॑षि॒ यन्नक्ष॑त्राणि॒ तान्ये॒वाप्नो॒त्यथो॑ अनूका॒शमे॒वैतानि॑~(३)

%5.4.1.4
ज्योतीꣳ॑षि कुरुते सुव॒र्गस्य॑ लो॒कस्यानु॑ख्यात्यै॒ यथ्सꣴस्पृ॑ष्टा उपद॒ध्याद्वृष्ट्यै॑ लो॒कमपि॑ दध्या॒दव॑र्\mbox{}षुकः प॒र्जन्यः॑ स्या॒दसꣴ॑स्पृष्टा॒ उप॑ दधाति॒ वृष्ट्या॑ ए॒व लो॒कं क॑रोति॒ वर्\mbox{}षु॑कः प॒र्जन्यो॑ भवति पु॒रस्ता॑द॒न्याः प्र॒तीची॒रुप॑ दधाति प॒श्चाद॒न्याः प्राची॒स्तस्मा᳚त्प्रा॒चीना॑नि च प्रती॒चीना॑नि च॒ नक्ष॑त्रा॒ण्या व॑र्तन्ते॥~(४)

{\anuvakamend[{भ्रातृ॑व्यो लो॒क ए॒वैतान्येक॑चत्वारिꣳशच्च}]}%~(१)

%5.4.2.1
ऋ॒त॒व्या॑ उप॑ दधात्यृतू॒नां कॢप्त्यै᳚ द्वं॒द्वमुप॑ दधाति॒ तस्मा᳚द्द्वं॒द्वमृ॒तवो\-ऽधृ॑तेव॒ वा ए॒षा यन्म॑ध्य॒मा चिति॑र॒न्तरि॑क्षमिव॒ वा ए॒षा द्वं॒द्वम॒न्यासु॒ चिती॒षूप॑ दधाति॒ चत॑स्रो॒ मध्ये॒ धृत्या॑ अन्तः॒श्लेष॑णं॒ वा ए॒ताश्चिती॑नां॒ यदृ॑त॒व्या॑ यदृ॑त॒व्या॑ उप॒दधा॑ति॒ चितीनां॒ विधृ॑त्या॒ अव॑का॒मनूप॑ दधात्ये॒षा वा अ॒ग्नेर्योनिः॒ सयो॑निम्~(५)

%5.4.2.2
ए॒वाग्निं चि॑नुत उ॒वाच॑ ह वि॒श्वामि॒त्रो\-ऽद॒दिथ्स ब्रह्म॒णान्नं॒ यस्यै॒ता उ॑पधी॒यान्तै॒ य उ॑ चैना ए॒वं वेद॒दिति॑ संवथ्स॒रो वा ए॒तं प्र॑ति॒ष्ठायै॑ नुदते॒ यो᳚\-ऽग्निं चि॒त्वा न प्र॑ति॒तिष्ठ॑ति॒ पञ्च॒ पूर्वा॒श्चित॑यो भव॒न्त्यथ॑ ष॒ष्ठीं चितिं॑ चिनुते॒ षड्वा ऋ॒तवः॑ संवथ्स॒र ऋ॒तुष्वे॒व सं॑वथ्स॒रे प्रति॑ तिष्ठत्ये॒ता वै~(६)

%5.4.2.3
अधि॑पत्नी॒र्नामेष्ट॑का॒ यस्यै॒ता उ॑पधी॒यन्ते\-ऽधि॑पतिरे॒व स॑मा॒नानां᳚ भवति॒ यं द्वि॒ष्यात्तमु॑प॒दध॑द्ध्यायेदे॒ताभ्य॑ ए॒वैनं॑ दे॒वता᳚भ्य॒ आ वृ॑श्चति ता॒जगार्ति॒मार्च्छ॒त्यङ्गि॑रसः सुव॒र्गं लो॒कं यन्तो॒ या य॒ज्ञस्य॒ निष्कृ॑ति॒रासी॒त्तामृषि॑भ्यः॒ प्रत्यौ॑ह॒न् तद्धिर॑ण्यमभव॒द्यद्धि॑रण्यश॒ल्कैः प्रो॒क्षति॑ य॒ज्ञस्य॒ निष्कृ॑त्या॒ अथो॑ भेष॒जमे॒वास्मै॑ करोति~(७)

%5.4.2.4
अथो॑ रू॒पेणै॒वैन॒ꣳ॒ सम॑र्धय॒त्यथो॒ हिर॑ण्यज्योतिषै॒व सु॑व॒र्गं लो॒कमे॑ति साह॒स्रव॑ता॒ प्रोक्ष॑ति साह॒स्रः प्र॒जा\-प॑तिः प्र॒जा\-प॑ते॒राप्त्या॑ इ॒मा मे॑ अग्न॒ इष्ट॑का धे॒नवः॑ स॒न्त्वित्या॑ह धे॒नूरे॒वैनाः᳚ कुरुते॒ ता ए॑नं काम॒दुघा॑ अ॒मुत्रा॒मुष्मिँ॑ल्लो॒क उप॑ तिष्ठन्ते॥~(८)

{\anuvakamend[{सयो॑निमे॒ता वै क॑रो॒त्येका॒न्नच॑त्वारि॒ꣳ॒शच्च॑}]}%~(२)

%5.4.3.1
रु॒द्रो वा ए॒ष यद॒ग्निः स ए॒तर्\mbox{}हि॑ जा॒तो यर्\mbox{}हि॒ सर्व॑श्चि॒तः स यथा॑ व॒थ्सो जा॒तः स्तनं॑ प्रे॒फ्सत्ये॒वं वा ए॒ष ए॒तर्\mbox{}हि॑ भाग॒धेयं॒ प्रेफ्स॑ति॒ तस्मै॒ यदाहु॑तिं॒ न जु॑हु॒याद॑ध्व॒र्युं च॒ यज॑मानं च ध्यायेच्छतरु॒द्रीयं॑ जुहोति भाग॒धेये॑नै॒वैनꣳ॑ शमयति॒ नार्ति॒मार्च्छ॑त्यध्व॒र्युर्न यज॑मानो॒ यद्ग्रा॒म्याणां᳚ पशू॒नाम्~(९)

%5.4.3.2
पय॑सा जुहु॒याद्ग्रा॒म्यान्प॒शूञ्छु॒चार्प॑ये॒द्यदा॑र॒ण्याना॑मार॒ण्याञ्ज॑र्तिलयवा॒ग्वा॑ वा जुहु॒याद्ग॑वीधुकयवा॒ग्वा॑ वा॒ न ग्रा॒म्यान्प॒शून् हि॒नस्ति॒ नार॒ण्यानथो॒ खल्वा॑हु॒रना॑हुति॒र्वै ज॒र्तिला᳚श्च ग॒वीधु॑का॒श्चेत्य॑जक्षी॒रेण॑ जुहोत्याग्ने॒यी वा ए॒षा यद॒जाहु॑त्यै॒व जु॑होति॒ न ग्रा॒म्यान्प॒शून् हि॒नस्ति॒ नार॒ण्यानङ्गि॑रसः सुव॒र्गं लो॒कं यन्तः॑~(१०)

%5.4.3.3
अ॒जायां᳚ घ॒र्मं प्रासि॑ञ्च॒न्थ्सा शोच॑न्ती प॒र्णं परा॑जिहीत॒ सो \-ऽर्को॑\-ऽभव॒त्तद॒र्कस्या᳚र्क॒त्वम॑र्कप॒र्णेन॑ जुहोति सयोनि॒त्वायोद॒ङ्तिष्ठ॑ञ्जुहोत्ये॒षा वै रु॒द्रस्य॒ दिख्स्वाया॑मे॒व दि॒शि रु॒द्रं नि॒रव॑दयते चर॒माया॒मिष्ट॑कायां जुहोत्यन्त॒त ए॒व रु॒द्रं नि॒रव॑दयते त्रेधाविभ॒क्तं जु॑होति॒ त्रय॑ इ॒मे लो॒का इ॒माने॒व लो॒कान्थ्स॒माव॑द्वीर्यान्करो॒तीय॒त्यग्रे॑ जुहोति~(११)

%5.4.3.4
अथेय॒त्यथेय॑ति॒ त्रय॑ इ॒मे लो॒का ए॒भ्य ए॒वैनं॑ लो॒केभ्यः॑ शमयति ति॒स्र उत्त॑रा॒ आहु॑तीर्जुहोति॒ षट्थ्सम्प॑द्य॒न्ते षड्वा ऋ॒तव॑ ऋ॒तुभि॑रे॒वैनꣳ॑ शमयति॒ यद॑नुपरि॒क्रामं॑ जुहु॒याद॑न्तरवचा॒रिणꣳ॑ रु॒द्रं कु॑र्या॒दथो॒ खल्वा॑हुः॒ कस्यां॒ वाह॑ दि॒शि रु॒द्रः कस्यां॒ वेत्य॑नुपरि॒क्राम॑मे॒व हो॑त॒व्य॑मप॑रिवर्गमे॒वैनꣳ॑ शमयति~(१२)

%5.4.3.5
ए॒ता वै दे॒वताः᳚ सुव॒र्ग्या॑ या उ॑त्त॒मास्ता यज॑मानं वाचयति॒ ताभि॑रे॒वैनꣳ॑ सुव॒र्गं लो॒कं ग॑मयति॒ यं द्वि॒ष्यात्तस्य॑ सञ्च॒रे प॑शू॒नां न्य॑स्ये॒द्यः प्र॑थ॒मः प॒शुर॑भि॒तिष्ठ॑ति॒ स आर्ति॒मार्च्छ॑ति॥~(१३)

{\anuvakamend[{प॒शू॒नां यन्तो\-ऽग्रे॑ जुहो॒त्यप॑रिवर्गमे॒वैनꣳ॑ शमयति त्रि॒ꣳ॒शच्च॑}]}%~(३)

%5.4.4.1
अश्म॒न्नूर्ज॒मिति॒ परि॑ षिञ्चति मा॒र्जय॑त्ये॒वैन॒मथो॑ त॒र्पय॑त्ये॒व स ए॑नं तृ॒प्तो\-ऽक्षु॑ध्य॒न्नशो॑चन्न॒मुष्मिँ॑ल्लो॒क उप॑ तिष्ठते॒ तृप्य॑ति प्र॒जया॑ प॒शुभि॒र्य ए॒वं वेद॒ तां न॒ इष॒मूर्जं॑ धत्त मरुतः सꣳररा॒णा इत्या॒हान्नं॒ वा ऊर्गन्नं॑ म॒रुतो\-ऽन्न॑मे॒वाव॑ रु॒न्धे\-ऽश्मꣴ॑स्ते॒ क्षुद॒मुं ते॒ शुक्~(१४)

%5.4.4.2
ऋ॒च्छ॒तु॒ यं द्वि॒ष्म इत्या॑ह॒ यमे॒व द्वेष्टि॒ तम॑स्य क्षु॒धा च॑ शु॒चा चा᳚र्पयति॒ त्रिः प॑रिषि॒ञ्चन्पर्ये॑ति त्रि॒वृद्वा अ॒ग्निर्यावा॑ने॒वाग्निस्तस्य॒ शुचꣳ॑ शमयति॒ त्रिः पुनः॒ पर्ये॑ति॒ षट्थ्सं प॑द्यन्ते॒ षड्वा ऋ॒तव॑ ऋ॒तुभि॑रे॒वास्य॒ शुचꣳ॑ शमयत्य॒पां वा ए॒तत्पुष्पं॒ यद्वे॑त॒सो॑\-ऽपाम्~(१५)

%5.4.4.3
शरो\-ऽव॑का वेतसशा॒खया॒ चाव॑काभिश्च॒ वि क॑र्\mbox{}ष॒त्यापो॒ वै शा॒न्ताः शा॒न्ताभि॑रे॒वास्य॒ शुचꣳ॑ शमयति॒ यो वा अ॒ग्निं चि॒तं प्र॑थ॒मः प॒शुर॑धि॒क्राम॑तीश्व॒रो वै तꣳ शु॒चा प्र॒दहो म॒ण्डूके॑न॒ वि क॑र्\mbox{}षत्ये॒ष वै प॑शू॒नाम॑नुपजीवनी॒यो न वा ए॒ष ग्रा॒म्येषु॑ प॒शुषु॑ हि॒तो नार॒ण्येषु॒ तमे॒व शु॒चार्प॑यत्यष्टा॒भिर्वि क॑र्\mbox{}षति~(१६)

%5.4.4.4
अ॒ष्टाक्ष॑रा गाय॒त्री गा॑य॒त्रो᳚\-ऽग्निर्यावा॑ने॒वाग्निस्तस्य॒ शुचꣳ॑ शमयति पाव॒कव॑तीभि॒रन्नं॒ वै पा॑व॒को\-ऽन्ने॑नै॒वास्य॒ शुचꣳ॑ शमयति मृ॒त्युर्वा ए॒ष यद॒ग्निर्ब्रह्म॑ण ए॒तद्रू॒पं यत्कृ॑ष्णाजि॒नं कार्\mbox{}ष्णी॑ उपा॒नहा॒वुप॑ मुञ्चते॒ ब्रह्म॑णै॒व मृ॒त्योर॒न्तर्ध॑त्ते॒\-ऽन्तर्मृ॒त्योर्ध॑त्ते॒\-ऽन्तर॒न्नाद्या॒दित्या॑हुर॒न्यामु॑पमु॒ञ्चते॒\-ऽन्यां नान्तः~(१७)

%5.4.4.5
ए॒व मृ॒त्योर्ध॒त्ते\-ऽवा॒न्नाद्यꣳ॑ रुन्धे॒ नम॑स्ते॒ हर॑से शो॒चिष॒ इत्या॑ह नम॒स्कृत्य॒ हि वसी॑याꣳसमुप॒चर॑न्त्य॒न्यं ते॑ अ॒स्मत्त॑पन्तु हे॒तय॒ इत्या॑ह॒ यमे॒व द्वेष्टि॒ तम॑स्य शु॒चार्प॑यति पाव॒को अ॒स्मभ्यꣳ॑ शि॒वो भ॒वेत्या॒हान्नं॒ वै पा॑व॒को\-ऽन्न॑मे॒वाव॑ रुन्धे॒ द्वाभ्या॒मधि॑ क्रामति॒ प्रति॑ष्ठित्या अप॒स्य॑वतीभ्या॒ꣳ॒ शान्त्यै᳚॥~(१८)

{\anuvakamend[{शुग्वे॑त॒सो॑\-ऽपाम॑ष्टा॒भिर्विक॑र्\mbox{}षति॒ नान्तरेका॒न्नप॑ञ्चा॒शच्च॑}]}%~(४)

%5.4.5.1
नृ॒षदे॒ वडिति॒ व्याघा॑रयति प॒ङ्क्त्याहु॑त्या यज्ञमु॒खमा र॑भते\-ऽक्ष्ण॒या व्याघा॑रयति॒ तस्मा॑दक्ष्ण॒या प॒शवो\-ऽङ्गा॑नि॒ प्र ह॑रन्ति॒ प्रति॑ष्ठित्यै॒ यद्व॑षट्कु॒र्याद्या॒तया॑मास्य वषट्का॒रः स्या॒द्यन्न व॑षट्कु॒र्याद्रक्षाꣳ॑सि य॒ज्ञꣳ ह॑न्यु॒र्वडित्या॑ह प॒रोक्ष॑मे॒व वष॑ट्करोति॒ नास्य॑ या॒तया॑मा वषट्का॒रो भव॑ति॒ न य॒ज्ञꣳ रक्षाꣳ॑सि घ्नन्ति हु॒तादो॒ वा अ॒न्ये दे॒वाः~(१९)

%5.4.5.2
अ॒हु॒तादो॒\-ऽन्ये तान॑ग्नि॒चिदे॒वोभया᳚न्प्रीणाति॒ ये दे॒वा दे॒वाना॒मिति॑ द॒ध्ना म॑धुमि॒श्रेणावो᳚क्षति हु॒ताद॑श्चै॒व दे॒वान॑हु॒ताद॑श्च॒ यज॑मानः प्रीणाति॒ ते यज॑मानं प्रीणन्ति द॒ध्नैव हु॒तादः॑ प्री॒णाति॒ मधु॑षाहु॒तादो᳚ ग्रा॒म्यं वा ए॒तदन्नं॒ यद्दध्या॑र॒ण्यं मधु॒ यद्द॒ध्ना म॑धुमि॒श्रेणा॒वोक्ष॑त्यु॒भय॒स्याव॑रुद्ध्यै ग्रुमु॒ष्टिनावो᳚क्षति प्राजाप॒त्यः~(२०)

%5.4.5.3
वै ग्रु॑मु॒ष्टिः स॑योनि॒त्वाय॒ द्वाभ्यां॒ प्रति॑ष्ठित्या अनुपरि॒चार॒मवो᳚क्ष॒त्यप॑रिवर्गमे॒वैना᳚न्प्रीणाति॒ वि वा ए॒ष प्रा॒णैः प्र॒जया॑ प॒शुभि॑र्\mbox{}ऋध्यते॒ यो᳚\-ऽग्निं चि॒न्वन्न॑धि॒क्राम॑ति प्राण॒दा अ॑पान॒दा इत्या॑ह प्रा॒णाने॒वा\-ऽऽ\-त्मन्ध॑त्ते वर्चो॒दा व॑रिवो॒दा इत्या॑ह प्र॒जा वै वर्चः॑ प॒शवो॒ वरि॑वः प्र॒जामे॒व प॒शूना॒त्मन्ध॑त्त॒ इन्द्रो॑ वृ॒त्रम॑ह॒न्तं वृ॒त्रः~(२१)

%5.4.5.4
ह॒तः षो॑ड॒शभि॑र्भो॒गैर॑सिना॒थ्स ए॒ताम॒ग्नये\-ऽनी॑कवत॒ आहु॑तिमपश्य॒त्ताम॑जुहो॒त्तस्या॒ग्निरनी॑कवा॒न्थ्स्वेन॑ भाग॒धेये॑न प्री॒तः षो॑डश॒धा वृ॒त्रस्य॑ भो॒गानप्य॑दहद्वैश्वकर्म॒णेन॑ पा॒प्मनो॒ निर॑मुच्यत॒ यद॒ग्नये\-ऽनी॑कवत॒ आहु॑तिं जु॒होत्य॒ग्निरे॒वास्यानी॑कवा॒न्थ्स्वेन॑ भाग॒धेये॑न प्री॒तः पा॒प्मान॒मपि॑ दहति वैश्वकर्म॒णेन॑ पा॒प्मनो॒ निर्मु॑च्यते॒ यं का॒मये॑त चि॒रं पा॒प्मनः॑~(२२)

%5.4.5.5
निर्मु॑च्ये॒तेत्येकै॑कं॒ तस्य॑ जुहुयाच्चि॒रमे॒व पा॒प्मनो॒ निर्मु॑च्यते॒ यं का॒मये॑त ता॒जक्पा॒प्मनो॒ निर्मु॑च्ये॒तेति॒ सर्वा॑णि॒ तस्या॑नु॒द्रुत्य॑ जुहुयात्ता॒जगे॒व पा॒प्मनो॒ निर्मु॑च्य॒ते\-ऽथो॒ खलु॒ नानै॒व सू॒क्ता\-भ्यां᳚ जुहोति॒ नानै॒व सू॒क्तयो᳚र्वी॒र्यं॑ दधा॒त्यथो॒ प्रति॑ष्ठित्यै॥~(२३)

{\anuvakamend[{दे॒वाः प्रा॑जाप॒त्यो वृ॒त्रश्चि॒रं पा॒प्मन॑श्चत्वारि॒ꣳ॒शच्च॑}]}%~(५)

%5.4.6.1
उदे॑नमुत्त॒रां न॒येति॑ स॒मिध॒ आ द॑धाति॒ यथा॒ जनं॑ य॒ते॑\-ऽव॒सं क॒रोति॑ ता॒दृगे॒व तत्ति॒स्र आ द॑धाति त्रि॒वृद्वा अ॒ग्निर्यावा॑ने॒वाग्निस्तस्मै॑ भाग॒धेयं॑ करो॒त्यौदु॑म्बरीर्भव॒न्त्यूर्ग्वा उ॑दु॒म्बर॒ ऊर्ज॑मे॒वास्मा॒ अपि॑ दधा॒त्युदु॑ त्वा॒ विश्वे॑ दे॒वा इत्या॑ह प्रा॒णा वै विश्वे॑ दे॒वाः प्रा॒णैः~(२४)

%5.4.6.2
ए॒वैन॒मुद्य॑च्छ॒ते\-ऽग्ने॒ भर॑न्तु॒ चित्ति॑भि॒रित्या॑ह॒ यस्मा॑ ए॒वैनं॑ चि॒त्तायो॒द्यच्छ॑ते॒ तेनै॒वैन॒ꣳ॒ सम॑र्धयति॒ पञ्च॒ दिशो॒ दैवी᳚र्य॒ज्ञम॑वन्तु दे॒वीरित्या॑ह॒ दिशो॒ ह्ये॑षो\-ऽनु॑ प्र॒च्यव॒ते\-ऽपाम॑तिं दुर्म॒तिं बाध॑माना॒ इत्या॑ह॒ रक्ष॑सा॒मप॑हत्यै रा॒यस्पोषे॑ य॒ज्ञप॑तिमा॒भज॑न्ती॒रित्या॑ह प॒शवो॒ वै रा॒यस्पोषः॑~(२५)

%5.4.6.3
प॒शूने॒वाव॑ रुन्धे ष॒ड्भिर्\mbox{}ह॑रति॒ षड्वा ऋ॒तव॑ ऋ॒तुभि॑रे॒वैनꣳ॑ हरति॒ द्वे प॑रि॒गृह्य॑वती भवतो॒ रक्ष॑सा॒मप॑हत्यै॒ सूर्य॑रश्मि॒र्॒\mbox{}हरि॑केशः पु॒रस्ता॒दित्या॑ह॒ प्रसू᳚त्यै॒ ततः॑ पाव॒का आ॒शिषो॑ नो जुषन्ता॒मित्या॒हान्नं॒ वै पा॑व॒को\-ऽन्न॑मे॒वाव॑ रुन्धे देवासु॒राः संय॑त्ता आस॒न्ते दे॒वा ए॒तदप्र॑तिरथमपश्य॒न्तेन॒ वै ते᳚\-ऽप्र॒ति~(२६)

%5.4.6.4
असु॑रानजय॒न्तदप्र॑तिरथस्याप्रतिरथ॒त्वं यदप्र॑तिरथं द्वि॒तीयो॒ होता॒न्वाहा᳚प्र॒त्ये॑व तेन॒ यज॑मानो॒ भ्रातृ॑व्याञ्जय॒त्यथो॒ अन॑भिजितमे॒वाभि ज॑यति दश॒र्चं भ॑वति॒ दशा᳚क्षरा वि॒राड्वि॒राजे॒मौ लो॒कौ विधृ॑ताव॒नयो᳚र्लो॒कयो॒र्विधृ॑त्या॒ अथो॒ दशा᳚क्षरा वि॒राडन्नं॑ वि॒राड्वि॒राज्ये॒वान्नाद्ये॒ प्रति॑ तिष्ठ॒त्यस॑दिव॒ वा अ॒न्तरि॑क्षम॒न्तरि॑क्षमि॒वाग्नी᳚ध्र॒माग्नी᳚ध्रे~(२७)

%5.4.6.5
अश्मा॑नं॒ नि द॑धाति स॒त्त्वाय॒ द्वाभ्यां॒ प्रति॑ष्ठित्यै वि॒मान॑ ए॒ष दि॒वो मध्य॑ आस्त॒ इत्या॑ह॒ व्ये॑वैतया॑ मिमीते॒ मध्ये॑ दि॒वो निहि॑तः पृश्ञि॒रश्मेत्या॒हान्नं॒ वै पृश्ञ्यन्न॑मे॒वाव॑ रुन्धे चत॒सृभि॒रा पुच्छा॑देति च॒त्वारि॒ छन्दाꣳ॑सि॒ छन्दो॑भिरे॒वेन्द्रं॒ विश्वा॑ अवीवृध॒न्नित्या॑ह॒ वृद्धि॑मे॒वोपाव॑र्तते॒ वाजा॑ना॒ꣳ॒ सत्प॑तिं॒ पतिम्᳚~(२८)

%5.4.6.6
इत्या॒हान्नं॒ वै वाजो\-ऽन्न॑मे॒वाव॑ रुन्धे सुम्न॒हूर्य॒ज्ञो दे॒वाꣳ आ च॑ वक्ष॒दित्या॑ह प्र॒जा वै प॒शवः॑ सु॒म्नं प्र॒जामे॒व प॒शूना॒त्मन्ध॑त्ते॒ यक्ष॑द॒ग्निर्दे॒वो दे॒वाꣳ आ च॑ वक्ष॒दित्या॑ह स्व॒गाकृ॑त्यै॒ वाज॑स्य मा प्रस॒वेनो᳚द्ग्रा॒भेणोद॑ग्रभी॒दित्या॑हा॒सौ वा आ॑दि॒त्य उ॒द्यन्नु॑द्ग्रा॒भ ए॒ष नि॒म्रोच॑न्निग्रा॒भो ब्रह्म॑णै॒वात्मान॑मुद्गृ॒ह्णाति॒ ब्रह्म॑णा॒ भ्रातृ॑व्यं॒ नि गृ॑ह्णाति॥~(२९)

{\anuvakamend[{प्रा॒णैः पोषो᳚\-ऽप्र॒त्याग्नी᳚ध्रे॒ पति॑मे॒ष दश॑ च}]}%~(६)

%5.4.7.1
प्राची॒मनु॑ प्र॒दिशं॒ प्रेहि॑ वि॒द्वानित्या॑ह देवलो॒कमे॒वैतयो॒पाव॑र्तते॒ क्रम॑ध्वम॒ग्निना॒ नाक॒मित्या॑हे॒माने॒वैतया॑ लो॒कान्क्र॑मते पृथि॒व्या अ॒हमुद॒न्तरि॑क्ष॒मारु॑ह॒मित्या॑हे॒माने॒वैतया॑ लो॒कान्थ्स॒मारो॑हति॒ सुव॒र्यन्तो॒ नापे᳚क्षन्त॒ इत्या॑ह सुव॒र्गमे॒वैतया॑ लो॒कमे॒त्यग्ने॒ प्रेहि॑~(३०)

%5.4.7.2
प्र॒थ॒मो दे॑वय॒तामित्या॑हो॒भये᳚ष्वे॒वैतया॑ देवमनु॒ष्येषु॒ चक्षु॑र्दधाति प॒ञ्चभि॒रधि॑ क्रामति॒ पाङ्क्तो॑ य॒ज्ञो यावा॑ने॒व य॒ज्ञस्तेन॑ स॒ह सु॑व॒र्गं लो॒कमे॑ति॒ नक्तो॒षासेति॑ पुरोनुवा॒क्या॑मन्वा॑ह॒ प्रत्त्या॒ अग्ने॑ सहस्रा॒क्षेत्या॑ह साह॒स्रः प्र॒जा\-प॑तिः प्र॒जा\-प॑ते॒राप्त्यै॒ तस्मै॑ ते विधेम॒ वाजा॑य॒ स्वाहेत्या॒हान्नं॒ वै वाजो\-ऽन्न॑मे॒वाव॑~(३१)

%5.4.7.3
रु॒न्धे॒ द॒ध्नः पू॒र्णामौदु॑म्बरीꣴ स्वयमातृ॒ण्णायां᳚ जुहो॒त्यूर्ग्वै दध्यूर्गु॑दु॒म्बरो॒\-ऽसौ स्व॑यमातृ॒ण्णामुष्या॑मे॒वोर्जं॑ दधाति॒ तस्मा॑द॒मुतो॒\-ऽर्वाची॒मूर्ज॒मुप॑ जीवामस्ति॒सृभिः॑ सादयति त्रि॒वृद्वा अ॒ग्निर्यावा॑ने॒वाग्निस्तं प्र॑ति॒ष्ठां ग॑मयति॒ प्रेद्धो॑ अग्ने दीदिहि पु॒रो न॒ इत्यौदुम्ब॑री॒मा द॑धात्ये॒षा वै सू॒र्मी कर्ण॑कावत्ये॒तया॑ ह स्म~(३२)

%5.4.7.4
वै दे॒वा असु॑राणाꣳ शतत॒र्॒\mbox{}हाꣴस्तृꣳ॑हन्ति॒ यदे॒तया॑ स॒मिध॑मा॒दधा॑ति॒ वज्र॑मे॒वैतच्छ॑त॒घ्नीं यज॑मानो॒ भ्रातृ॑व्याय॒ प्र ह॑रति॒ स्तृत्या॒ अछ॑म्बट्कारं वि॒धेम॑ ते पर॒मे जन्म॑न्नग्न॒ इति॒ वैक॑ङ्कती॒मा द॑धाति॒ भा ए॒वाव॑ रुन्धे॒ ताꣳ स॑वि॒तुर्वरे᳚ण्यस्य चि॒त्रामिति॑ शमी॒मयी॒ꣳ॒ शान्त्या॑ अ॒ग्निर्वा॑ ह॒ वा अ॑ग्नि॒चितं॑ दु॒हे᳚\-ऽग्नि॒चिद्वा॒ग्निं दु॑हे॒ ताम्~(३३)

%5.4.7.5
स॒वि॒तुर्वरे᳚ण्यस्य चि॒त्रामित्या॑है॒ष वा अ॒ग्नेर्दोह॒स्तम॑स्य॒ कण्व॑ ए॒व श्रा॑य॒सो॑\-ऽवे॒त्तेन॑ ह स्मैन॒ꣳ॒ स दु॑हे॒ यदे॒तया॑ स॒मिध॑मा॒दधा᳚त्यग्नि॒चिदे॒व तद॒ग्निं दु॑हे स॒प्त ते॑ अग्ने स॒मिधः॑ स॒प्त जि॒ह्वा इत्या॑ह स॒प्तैवास्य॒ साप्ता॑नि प्रीणाति पू॒र्णया॑ जुहोति पू॒र्ण इ॑व॒ हि प्र॒जा\-प॑तिः प्र॒जा\-प॑तेः~(३४)

%5.4.7.6
आप्त्यै॒ न्यू॑नया जुहोति॒ न्यू॑ना॒द्धि प्र॒जा\-प॑तिः प्र॒जा असृ॑जत प्र॒जाना॒ꣳ॒ सृष्ट्या॑ अ॒ग्निर्दे॒वेभ्यो॒ निला॑यत॒ स दिशो\-ऽनु॒ प्रावि॑श॒ज्जुह्व॒न्मन॑सा॒ दिशो᳚ ध्यायेद्दि॒ग्भ्य ए॒वैन॒मव॑ रुन्धे द॒ध्ना पु॒रस्ता᳚ज्जुहो॒त्याज्ये॑नो॒परि॑ष्टा॒त्तेज॑श्चै॒वास्मा॑ इन्द्रि॒यं च॑ स॒मीची॑ दधाति॒ द्वाद॑श\-कपालो वैश्वान॒रो भ॑वति॒ द्वाद॑श॒ मासाः᳚ संवथ्स॒रः सं॑वथ्स॒रो᳚\-ऽग्निर्वै᳚श्वान॒रः सा॒क्षात्~(३५)

%5.4.7.7
ए॒व वै᳚श्वान॒रमव॑ रुन्धे॒ यत्प्र॑याजानूया॒जान्कु॒र्याद्विक॑स्तिः॒ सा य॒ज्ञस्य॑ दर्विहो॒मं क॑रोति य॒ज्ञस्य॒ प्रति॑ष्ठित्यै रा॒ष्ट्रं वै वै᳚श्वान॒रो विण्म॒रुतो वैश्वान॒रꣳ हु॒त्वा मा॑रु॒ताञ्जु॑होति रा॒ष्ट्र ए॒व विश॒मनु॑ बध्नात्यु॒च्चैर्वै᳚श्वान॒रस्या श्रा॑वयत्युपा॒ꣳ॒शु मा॑रु॒ताञ्जु॑होति॒ तस्मा᳚द्रा॒ष्ट्रं विश॒मति॑ वदति मारु॒ता भ॑वन्ति म॒रुतो॒ वै दे॒वानां॒ विशो॑ देववि॒शेनै॒वास्मै॑ मनुष्यवि॒शमव॑ रुन्धे स॒प्त भ॑वन्ति स॒प्तग॑णा॒ वै म॒रुतो॑ गण॒श ए॒व विश॒मव॑ रुन्धे ग॒णेन॑ ग॒णम॑नु॒द्रुत्य॑ जुहोति॒ विश॑मे॒वास्मा॒ अनु॑वर्त्मानं करोति॥~(३६)

{\anuvakamend[{अग्ने॒ प्रेह्यव॑ स्म दुहे॒ तां प्र॒जा\-प॑तेः सा॒क्षान्म॑नुष्यवि॒शमेक॑विꣳशतिश्च}]}%~(७)

%5.4.8.1
वसो॒र्धारां᳚ जुहोति॒ वसो᳚र्मे॒ धारा॑स॒दिति॒ वा ए॒षा हू॑यते घृ॒तस्य॒ वा ए॑नमे॒षा धारा॒मुष्मिँ॑ल्लो॒के पिन्व॑मा॒नोप॑ तिष्ठत॒ आज्ये॑न जुहोति॒ तेजो॒ वा आज्यं॒ तेजो॒ वसो॒र्धारा॒ तेज॑सै॒वास्मै॒ तेजो\-ऽव॑ रु॒न्धे\-ऽथो॒ कामा॒ वै वसो॒र्धारा॒ कामा॑ने॒वाव॑ रुन्धे॒ यं का॒मये॑त प्रा॒णान॑स्या॒न्नाद्यं॒ वि~(३७)

%5.4.8.2
छि॒न्द्या॒मिति॑ वि॒ग्राहं॒ तस्य॑ जुहुयात्प्रा॒णाने॒वास्या॒न्नाद्यं॒ विच्छि॑नत्ति॒ यं का॒मये॑त प्रा॒णान॑स्या॒न्नाद्य॒ꣳ॒ सं त॑नुया॒मिति॒ सन्त॑तां॒ तस्य॑ जुहुयात्प्रा॒णाने॒वास्या॒न्नाद्य॒ꣳ॒ सं त॑नोति॒ द्वाद॑श द्वाद॒शानि॑ जुहोति॒ द्वाद॑श॒ मासाः᳚ संवथ्स॒रः संवथ्स॒रेणै॒वास्मा॒ अन्न॒मव॑ रु॒न्धे\-ऽन्नं॑ च॒ मे\-ऽक्षु॑च्च म॒ इत्या॑है॒तद्वै~(३८)

%5.4.8.3
अन्न॑स्य रू॒पꣳ रू॒पेणै॒वान्न॒मव॑ रुन्धे॒\-ऽग्निश्च॑ म॒ आप॑श्च म॒ इत्या॑है॒षा वा अन्न॑स्य॒ योनिः॒ सयो᳚न्ये॒वान्न॒मव॑ रुन्धे\-ऽर्धे॒न्द्राणि॑ जुहोति दे॒वता॑ ए॒वाव॑ रुन्धे॒ यथ्सर्वे॑षाम॒र्धमिन्द्रः॒ प्रति॒ तस्मा॒दिन्द्रो॑ दे॒वता॑नां भूयिष्ठ॒भाक्त॑म॒ इन्द्र॒मुत्त॑रमाहेन्द्रि॒यमे॒वास्मि॑न्नु॒परि॑ष्टाद्दधाति यज्ञायु॒धानि॑ जुहोति य॒ज्ञः~(३९)

%5.4.8.4
वै य॑ज्ञायु॒धानि॑ य॒ज्ञमे॒वाव॑ रु॒न्धे\-ऽथो॑ ए॒तद्वै य॒ज्ञस्य॑ रू॒पꣳ रू॒पेणै॒व य॒ज्ञमव॑ रुन्धे\-ऽवभृ॒थश्च॑ मे स्वगाका॒रश्च॑ म॒ इत्या॑ह स्व॒गाकृ॑त्या अ॒ग्निश्च॑ मे घ॒र्मश्च॑ म॒ इत्या॑है॒तद्वै ब्र॑ह्मवर्च॒सस्य॑ रू॒पꣳ रू॒पेणै॒व ब्र॑ह्मवर्च॒समव॑ रुन्ध॒ ऋक्च॑ मे॒ साम॑ च म॒ इत्या॑ह~(४०)

%5.4.8.5
ए॒तद्वै छन्द॑साꣳ रू॒पꣳ रू॒पेणै॒व छन्दा॒ꣴ॒स्यव॑ रुन्धे॒ गर्भा᳚श्च मे व॒थ्साश्च॑ म॒ इत्या॑है॒तद्वै प॑शू॒नाꣳ रू॒पꣳ रू॒पेणै॒व प॒शूनव॑ रुन्धे॒ कल्पा᳚ञ्जुहो॒त्यकॢ॑प्तस्य॒ कॢप्त्यै॑ युग्मदयु॒जे जु॑होति मिथुन॒त्वायो᳚त्त॒राव॑ती भवतो॒\-ऽभिक्रा᳚न्त्या॒ एका॑ च मे ति॒स्रश्च॑ म॒ इत्या॑ह देवछन्द॒सं वा एका॑ च ति॒स्रश्च॑~(४१)

%5.4.8.6
म॒नु॒ष्य॒छ॒न्द॒सं चत॑स्रश्चा॒ष्टौ च॑ देवछन्द॒सं चै॒व म॑नुष्यछन्द॒सं चाव॑ रुन्ध॒ आ त्रय॑स्त्रिꣳशतो जुहोति॒ त्रय॑स्त्रिꣳश॒द्वै दे॒वता॑ दे॒वता॑ ए॒वाव॑ रुन्ध॒ आष्टाच॑त्वारिꣳशतो जुहोत्य॒ष्टाच॑त्वारिꣳशदक्षरा॒ जग॑ती॒ जाग॑ताः प॒शवो॒ जग॑त्यै॒वास्मै॑ प॒शूनव॑ रुन्धे॒ वाज॑श्च प्रस॒वश्चेति॑ द्वाद॒शं जु॑होति॒ द्वाद॑श॒ मासाः᳚ संवथ्स॒रः सं॑वथ्स॒र ए॒व प्रति॑ तिष्ठति॥~(४२)

{\anuvakamend[{वि वै य॒ज्ञः साम॑ च म॒ इत्या॑ह च ति॒स्रश्चैका॒न्नप॑ञ्चा॒शच्च॑}]}%~(८)

%5.4.9.1
अ॒ग्निर्दे॒वेभ्यो\-ऽपा᳚क्रामद्भाग॒धेय॑मि॒च्छमा॑न॒स्तं दे॒वा अ॑ब्रुव॒न्नुप॑ न॒ आ व॑र्तस्व ह॒व्यं नो॑ व॒हेति॒ सो᳚\-ऽब्रवी॒द्वरं॑ वृणै॒ मह्य॑मे॒व वा॑जप्रस॒वीयं॑ जुहव॒न्निति॒ तस्मा॑द॒ग्नये॑ वाजप्रस॒वीयं॑ जुह्वति॒ यद्वा॑जप्रस॒वीयं॑ जु॒होत्य॒ग्निमे॒व तद्भा॑ग॒धेये॑न॒ सम॑र्धय॒त्यथो॑ अभिषे॒क ए॒वास्य॒ स च॑तुर्द॒शभि॑र्जुहोति स॒प्त ग्रा॒म्या ओष॑धयः स॒प्त~(४३)

%5.4.9.2
आ॒र॒ण्या उ॒भयी॑षा॒मव॑रुद्ध्या॒ अन्न॑स्यान्नस्य जुहो॒त्यन्न॑स्यान्न॒स्याव॑रुद्ध्या॒ औदु॑म्बरेण स्रु॒वेण॑ जुहो॒त्यूर्ग्वा उ॑दु॒म्बर॒ ऊर्गन्न॑मू॒र्जैवास्मा॒ ऊर्ज॒मन्न॒मव॑ रुन्धे॒\-ऽग्निर्वै दे॒वाना॑म॒भिषि॑क्तो\-ऽग्नि॒चिन्म॑नु॒ष्या॑णा॒न्तस्मा॑दग्नि॒चिद्वर्\mbox{}ष॑ति॒ न धा॑वे॒दव॑रुद्ध॒ꣴ॒ ह्य॑स्यान्न॒मन्न॑मिव॒ खलु॒ वै व॒र्॒\mbox{}षं यद्धावे॑द॒न्नाद्या᳚द्धावेदु॒पाव॑र्तेता॒न्नाद्य॑मे॒वाभि~(४४)

%5.4.9.3
उ॒पाव॑र्तते॒ नक्तो॒षासेति॑ कृ॒ष्णायै᳚ श्वे॒तव॑थ्सायै॒ पय॑सा जुहो॒त्यह्नै॒वास्मै॒ रात्रिं॒ प्र दा॑पयति॒ रात्रि॒याह॑रहोरा॒त्रे ए॒वास्मै॒ प्रत्ते॒ काम॑म॒न्नाद्यं॑ दुहाते राष्ट्र॒भृतो॑ जुहोति रा॒ष्ट्रमे॒वाव॑ रुन्धे ष॒ड्भिर्जु॑होति॒ षड्वा ऋ॒तव॑ ऋ॒तुष्वे॒व प्रति॑ तिष्ठति॒ भुव॑नस्य पत॒ इति॑ रथमु॒खे पञ्चाहु॑तीर्जुहोति॒ वज्रो॒ वै रथो॒ वज्रे॑णै॒व दिशः॑~(४५)

%5.4.9.4
अ॒भि ज॑यत्यग्नि॒चितꣳ॑ ह॒ वा अ॒मुष्मिँ॑ल्लो॒के वातो॒\-ऽभि प॑वते वातना॒मानि॑ जुहोत्य॒भ्ये॑वैन॑म॒मुष्मिँ॑ल्लो॒के वातः॑ पवते॒ त्रीणि॑ जुहोति॒ त्रय॑ इ॒मे लो॒का ए॒भ्य ए॒व लो॒केभ्यो॒ वात॒मव॑ रुन्धे समु॒द्रो॑\-ऽसि॒ नभ॑स्वा॒नित्या॑है॒तद्वै वात॑स्य रू॒पꣳ रू॒पेणै॒व वात॒मव॑ रुन्धे\-ऽञ्ज॒लिना॑ जुहोति॒ न ह्ये॑तेषा॑म॒न्यथाहु॑तिरव॒कल्प॑ते॥~(४६)

{\anuvakamend[{ओष॑धयः स॒प्ताभि दिशो॒\-ऽन्यथा॒ द्वे च॑}]}%~(९)

%5.4.10.1
सुव॒र्गाय॒ वै लो॒काय॑ देवर॒थो यु॑ज्यते यत्राकू॒ताय॑ मनुष्यर॒थ ए॒ष खलु॒ वै दे॑वर॒थो यद॒ग्निर॒ग्निं यु॑नज्मि॒ शव॑सा घृ॒तेनेत्या॑ह यु॒नक्त्ये॒वैन॒ꣳ॒ स ए॑नं यु॒क्तः सु॑व॒र्गं लो॒कम॒भि व॑हति॒ यथ्सर्वा॑भिः प॒ञ्चभि॑र्यु॒ञ्ज्याद्यु॒क्तो᳚\-ऽस्या॒ग्निः प्रच्यु॑तः स्या॒दप्र॑तिष्ठिता॒ आहु॑तयः॒ स्युरप्र॑तिष्ठिताः॒ स्तोमा॒ अप्र॑तिष्ठितान्यु॒क्थानि॑ ति॒सृभिः॑ प्रातःसव॒ने॑\-ऽभि मृ॑शति त्रि॒वृत्~(४७)

%5.4.10.2
वा अ॒ग्निर्यावा॑ने॒वाग्निस्तं यु॑नक्ति॒ यथान॑सि यु॒क्त आ॑धी॒यत॑ ए॒वमे॒व तत्प्रत्याहु॑तय॒स्तिष्ठ॑न्ति॒ प्रति॒ स्तोमाः॒ प्रत्यु॒क्थानि॑ यज्ञाय॒ज्ञिय॑स्य स्तो॒त्रे द्वाभ्या॑म॒भि मृ॑शत्ये॒तावा॒न् वै य॒ज्ञो यावा॑नग्निष्टो॒मो भू॒मा त्वा अ॒स्यात॑ ऊ॒र्ध्वः क्रि॑यते॒ यावा॑ने॒व य॒ज्ञस्तम॑न्त॒तो᳚\-ऽन्वारो॑हति॒ द्वाभ्यां॒ प्रति॑ष्ठित्या॒ एक॒याप्र॑स्तुतं॒ भव॒त्यथ॑~(४८)

%5.4.10.3
अ॒भि मृ॑श॒त्युपै॑न॒मुत्त॑रो य॒ज्ञो न॑म॒त्यथो॒ सन्त॑त्यै॒ प्र वा ए॒षो᳚\-ऽस्माल्लो॒काच्च्य॑वते॒ यो᳚\-ऽग्निं चि॑नु॒ते न वा ए॒तस्या॑निष्ट॒क आहु॑ति॒रव॑ कल्पते॒ यां वा ए॒षो॑\-ऽनिष्ट॒क आहु॑तिं जु॒होति॒ स्रव॑ति॒ वै सा ताꣴ स्रव॑न्तीं य॒ज्ञो\-ऽनु॒ परा॑ भवति य॒ज्ञं यज॑मानो॒ यत्पु॑नश्चि॒तिं चि॑नु॒त आहु॑तीनां॒ प्रति॑ष्ठित्यै॒ प्रत्याहु॑तय॒स्तिष्ठ॑न्ति~(४९)

%5.4.10.4
न य॒ज्ञः प॑रा॒भव॑ति॒ न यज॑मानो॒\-ऽष्टावुप॑ दधात्य॒ष्टाक्ष॑रा गाय॒त्री गा॑य॒त्रेणै॒वैनं॒ छन्द॑सा चिनुते॒ यदेका॑\-दश॒ त्रैष्टु॑भेन॒ यद्द्वाद॑श॒ जाग॑तेन॒ छन्दो॑भिरे॒वैनं॑ चिनुते नपा॒त्को वै नामै॒षो᳚\-ऽग्निर्यत्पु॑नश्चि॒तिर्य ए॒वं वि॒द्वान्पु॑नश्चि॒तिं चि॑नु॒त आ तृ॒तीया॒त्पुरु॑षा॒दन्न॑मत्ति॒ यथा॒ वै पु॑नरा॒धेय॑ ए॒वं पु॑नश्चि॒तिर्यो᳚\-ऽग्न्या॒धेये॑न॒ न~(५०)

%5.4.10.5
ऋ॒ध्नोति॒ स पु॑नरा॒धेय॒मा ध॑त्ते॒ यो᳚\-ऽग्निं चि॒त्वा नर्ध्नोति॒ स पु॑नश्चि॒तिं चि॑नुते॒ यत्पु॑नश्चि॒तिं चि॑नु॒त ऋद्ध्या॒ अथो॒ खल्वा॑हु॒र्न चे॑त॒व्येति॑ रु॒द्रो वा ए॒ष यद॒ग्निर्यथा᳚ व्या॒घ्रꣳ सु॒प्तं बो॒धय॑ति ता॒दृगे॒व तदथो॒ खल्वा॑हुश्चेत॒व्येति॒ यथा॒ वसी॑याꣳसं भाग॒धेये॑न बो॒धय॑ति ता॒दृगे॒व तन्मनु॑र॒ग्निम॑चिनुत॒ तेन॒ नार्ध्नो॒थ्स ए॒तां पु॑नश्चि॒तिम॑पश्य॒त्ताम॑चिनुत॒ तया॒ वै स आ᳚र्ध्नो॒द्यत्पु॑नश्चि॒तिं चि॑नु॒त ऋद्ध्यै᳚॥~(५१)

{\anuvakamend[{त्रि॒वृदथ॒ तिष्ठ॑न्त्यग्न्या॒धेये॑न॒ नाचि॑नुत स॒प्तद॑श च}]}%॥10॥

%5.4.11.1
छ॒न्द॒श्चितं॑ चिन्वीत प॒शुका॑मः प॒शवो॒ वै छन्दाꣳ॑सि पशु॒माने॒व भ॑वति श्येन॒चितं॑ चिन्वीत सुव॒र्गका॑मः श्ये॒नो वै वय॑सां॒ पति॑ष्ठः श्ये॒न ए॒व भू॒त्वा सु॑व॒र्गं लो॒कं प॑तति कङ्क॒चितं॑ चिन्वीत॒ यः का॒मये॑त शीर्\mbox{}ष॒ण्वान॒मुष्मिँ॑ल्लो॒के स्या॒मिति॑ शीर्\mbox{}ष॒ण्वाने॒वामुष्मिँ॑ल्लो॒के भ॑वत्यलज॒चितं॑ चिन्वीत॒ चतुः॑सीतं प्रति॒ष्ठाका॑म॒श्चत॑स्रो॒ दिशो॑ दि॒क्ष्वे॑व प्रति॑ तिष्ठति प्रउग॒चितं॑ चिन्वीत॒ भ्रातृ॑व्यवा॒न्प्र~(५२)

%5.4.11.2
ए॒व भ्रातृ॑व्यान्नुदत उभ॒यतः॑प्रउगं चिन्वीत॒ यः का॒मये॑त॒ प्र जा॒तान्भ्रातृ॑व्यान्नु॒देय॒ प्रति॑जनि॒ष्यमा॑णा॒निति॒ प्रैव जा॒तान्भ्रातृ॑व्यान्नु॒दते॒ प्रति॑ जनि॒ष्यमा॑णान्रथचक्र॒चितं॑ चिन्वीत॒ भ्रातृ॑व्य॒वान् वज्रो॒ वै रथो॒ वज्र॑मे॒व भ्रातृ॑व्येभ्यः॒ प्र ह॑रति द्रोण॒चितं॑ चिन्वी॒तान्न॑कामो॒ द्रोणे॒ वा अन्न॑म्भ्रियते॒ सयो᳚न्ये॒वान्न॒मव॑ रुन्धे समू॒ह्यं॑ चिन्वीत प॒शुका॑मः पशु॒माने॒व भ॑वति~(५३)

%5.4.11.3
प॒रि॒चा॒य्यं॑ चिन्वीत॒ ग्राम॑कामो ग्रा॒म्ये॑व भ॑वति श्मशान॒चितं॑ चिन्वीत॒ यः का॒मये॑त पितृलो॒क ऋ॑ध्नुया॒मिति॑ पितृलो॒क ए॒वर्ध्नो॑ति विश्वामित्रजमद॒ग्नी वसि॑ष्ठेनास्पर्धेता॒ꣳ॒ स ए॒ता ज॒मद॑ग्निर्विह॒व्या॑ अपश्य॒त्ता उपा॑धत्त॒ ताभि॒र्वै स वसि॑ष्ठस्येन्द्रि॒यं वी॒र्य॑मवृङ्क्त॒ यद्वि॑ह॒व्या॑ उप॒दधा॑तीन्द्रि॒यमे॒व ताभि॑र्वी॒र्यं॑ यज॑मानो॒ भ्रातृ॑व्यस्य वृङ्क्ते॒ होतु॒र्धिष्णि॑य॒ उप॑ दधाति यजमानायत॒नं वै~(५४)

%5.4.11.4
होता॒ स्व ए॒वास्मा॑ आ॒यत॑न इन्द्रि॒यं वी॒र्य॑मव॑ रुन्धे॒ द्वाद॒शोप॑ दधाति॒ द्वाद॑शाक्षरा॒ जग॑ती॒ जाग॑ताः प॒शवो॒ जग॑त्यै॒वास्मै॑ प॒शूनव॑ रुन्धे॒\-ऽष्टाव॑ष्टाव॒न्येषु॒ धिष्णि॑ये॒षूप॑ दधात्य॒ष्टाश॑फाः प॒शवः॑ प॒शूने॒वाव॑ रुन्धे॒ षण्मा᳚र्जा॒लीये॒ षड्वा ऋ॒तव॑ ऋ॒तवः॒ खलु॒ वै दे॒वाः पि॒तर॑ ऋ॒तूने॒व दे॒वान्पि॒तॄन्प्री॑णाति॥~(५)

{\anuvakamend[{प्र भ॑वति यजमानायत॒नं वा अ॒ष्टाच॑त्वारिꣳशच्च}]}%॥11॥

%5.4.12.1
पव॑स्व॒ वाज॑सातय॒ इत्य॑नु॒ष्टुक्प्र॑ति॒पद्भ॑वति ति॒स्रो॑\-ऽनु॒ष्टुभ॒श्चत॑स्रो गाय॒त्रियो॒ यत्ति॒स्रो॑\-ऽनु॒ष्टुभ॒स्तस्मा॒दः श्व॑स्त्रि॒भिस्तिष्ठꣴ॑ स्तिष्ठति॒ यच्चत॑स्रो गाय॒त्रिय॒स्तस्मा॒थ्सर्वाꣴ॑श्च॒तुरः॑ प॒दः प्र॑ति॒दध॒त्पला॑यते पर॒मा वा ए॒षा छन्द॑सां॒ यद॑नु॒ष्टुक्प॑र॒मश्च॑तुष्टो॒मः स्तोमा॑नां पर॒मस्त्रि॑रा॒त्रो य॒ज्ञानां᳚ पर॒मो\-ऽश्वः॑ पशू॒नां पर॑मेणै॒वैनं॑ पर॒मतां᳚ गमयत्येकवि॒ꣳ॒शमह॑र्भवति~(५६)

%5.4.12.2
यस्मि॒न्नश्व॑ आल॒भ्यते॒ द्वाद॑श॒ मासाः॒ पञ्च॒र्तव॒स्त्रय॑ इ॒मे लो॒का अ॒सावा॑दि॒त्य ए॑कवि॒ꣳ॒श ए॒ष प्र॒जा\-प॑तिः प्राजाप॒त्यो\-ऽश्व॒स्तमे॒व सा॒क्षादृ॑ध्नोति॒ शक्व॑रयः पृ॒ष्ठं भ॑वन्त्य॒न्यद॑न्य॒च्छन्दो॒\-ऽन्ये᳚न्ये॒ वा ए॒ते प॒शव॒ आ ल॑भ्यन्त उ॒तेव॑ ग्रा॒म्या उ॒तेवा॑र॒ण्या यच्छक्व॑रयः पृ॒ष्ठं भव॒न्त्यश्व॑स्य सर्व॒त्वाय॑ पार्थुर॒श्मं ब्र॑ह्मसा॒मं भ॑वति र॒श्मिना॒ वा अश्वः॑~(५७)

%5.4.12.3
य॒त ई᳚श्व॒रो वा अश्वो\-ऽय॒तो\-ऽप्र॑तिष्ठितः॒ परां᳚ परा॒वतं॒ गन्तो॒र्त्पा᳚र्थुर॒श्मं ब्र॑ह्मसा॒मं भव॒त्यश्व॑स्य॒ यत्यै॒ धृत्यै॒ सङ्कृ॑त्यच्छावाकसा॒मं भ॑वत्युथ्सन्नय॒ज्ञो वा ए॒ष यद॑श्वमे॒धः कस्तद्वे॒देत्या॑हु॒र्यदि॒ सर्वो॑ वा क्रि॒यते॒ न वा॒ सर्व॒ इति॒ यथ्सङ्कृ॑त्यच्छावाकसा॒मं भव॒त्यश्व॑स्य सर्व॒त्वाय॒ पर्या᳚प्त्या॒ अन॑न्तरायाय॒ सर्व॑स्तोमो\-ऽतिरा॒त्र उ॑त्त॒ममह॑र्भवति॒ सर्व॒स्याप्त्यै॒ सर्व॑स्य॒ जित्यै॒ सर्व॑मे॒व तेना᳚\-ऽऽ\-प्नोति॒ सर्वं॑ जयति॥~(५८)

{\anuvakamend[{अह॑र्भवति॒ वा अश्वो\-ऽह॑र्भवति॒ दश॑ च}]}%॥12॥

\prashnaend{दे॒वा॒सु॒राः तेनर्त॒व्या॑ रु॒द्रो\-ऽश्म॑न्नृ॒षदे॒ वडुदे॑नं॒ प्राची॒मिति॒ वसो॒र्धारा॑म॒ग्निर्दे॒वेभ्यः॑ सुव॒र्गाय॑ यत्राकू॒ताय॑ छन्द॒श्चितं॒ पव॑स्व॒ द्वाद॑श॥१२॥}{दे॒वा॒सु॒रा अ॒जायां॒ वै ग्रु॑मु॒ष्टिः प्र॑थ॒मो दे॑वय॒तामे॒तद्वै छन्द॑सामृ॒ध्नोत्य॒ष्टौ प॑ञ्चाशचत्॥५८॥}{दे॒वा॒सु॒राः सर्वं॑ जयति॥}%%५-४
{हरिः॑ ॐ}{॥कृष्ण-यजुर्वेदीय-तैत्तिरीय-संहितायां पञ्चम्काण्डे चतुर्थः प्रश्नः समाप्तः॥५-४॥}
%%% END PRASHNA

\sect{पञ्चमः प्रश्नः}\setcounter{anuvakam}{0}
\dnsub{तैत्तिरीयसंहितायां पञ्चमकाण्डे पञ्चमः प्रश्नः}
%5.5.1.1
यदेके॑न सꣴस्था॒पय॑ति य॒ज्ञस्य॒ सन्त॑त्या॒ अवि॑च्छेदायै॒न्द्राः प॒शवो॒ ये मु॑ष्क॒रा यदै॒न्द्राः सन्तो॒\-ऽग्निभ्य॑ आल॒भ्यन्ते॑ दे॒वता᳚भ्यः स॒मदं॑ दधात्याग्ने॒यीस्त्रि॒ष्टुभो॑ याज्यानुवा॒क्याः᳚ कुर्या॒द्यदा᳚ग्ने॒यीस्तेना᳚ग्ने॒या यत्त्रि॒ष्टुभ॒स्तेनै॒न्द्राः समृ॑द्ध्यै॒ न दे॒वता᳚भ्यः स॒मदं॑ दधाति वा॒यवे॑ नि॒युत्व॑ते तूप॒रमा ल॑भते॒ तेजो॒\-ऽग्नेर्वा॒युस्तेज॑स ए॒ष आ ल॑भ्यते॒ तस्मा᳚द्य॒द्रिय॑ङ्वा॒युः~(१)

%5.5.1.2
वाति॑ त॒द्रिय॑ङ्ङ॒ग्निर्द॑हति॒ स्वमे॒व तत्तेजो\-ऽन्वे॑ति॒ यन्न नि॒युत्व॑ते॒ स्यादुन्मा᳚द्ये॒द्यज॑मानो नि॒युत्व॑ते भवति॒ यज॑मान॒स्यानु॑न्मादाय वायु॒मती᳚ श्वे॒तव॑ती याज्यानुवा॒क्ये॑ भवतः सतेज॒स्त्वाय॑ हिरण्यग॒र्भः सम॑वर्त॒ताग्र॒ इत्या॑घा॒रमा घा॑रयति प्र॒जा\-प॑ति॒र्वै हि॑रण्यग॒र्भः प्र॒जा\-प॑तेरनुरूप॒त्वाय॒ सर्वा॑णि॒ वा ए॒ष रू॒पाणि॑ पशू॒नां प्रत्या ल॑भ्यते॒ यच्छ्म॑श्रु॒णस्तत्~(२)

%5.5.1.3
पुरु॑षाणाꣳ रू॒पं यत्तू॑प॒रस्तदश्वा॑नां॒ यद॒न्यतो॑द॒न्तद्गवां॒ यदव्या॑ इव श॒फास्तदवी॑नां॒ यद॒जस्तद॒जानां᳚ वा॒युर्वै प॑शू॒नां प्रि॒यं धाम॒ यद्वा॑य॒व्यो॑ भव॑त्ये॒तमे॒वैन॑म॒भि सं॑जाना॒नाः प॒शव॒ उप॑ तिष्ठन्ते वाय॒व्यः॑ का॒र्या(३)ः प्रा॑जाप॒त्या(३) इत्या॑हु॒र्यद्वा॑य॒व्यं॑ कु॒र्यात्प्र॒जा\-प॑तेरिया॒द्यत्प्रा॑जाप॒त्यं कु॒र्याद्वा॒योः~(३)

%5.5.1.4
इ॒या॒द्यद्वा॑य॒व्यः॑ प॒शुर्भव॑ति॒ तेन॑ वा॒योर्नैति॒ यत्प्रा॑जाप॒त्यः पु॑रो॒डाशो॒ भव॑ति॒ तेन॑ प्रा॒जाप॑ते॒र्नैति॒ यद्द्वाद॑श\-कपाल॒स्तेन॑ वैश्वान॒रान्नैत्या᳚ग्नावैष्ण॒वमेका॑\-दश\-कपालं॒ निर्व॑पति दीक्षि॒ष्यमा॑णो॒\-ऽग्निः सर्वा॑ दे॒वता॒ विष्णु॑र्य॒ज्ञो दे॒वता᳚श्चै॒व य॒ज्ञं चा र॑भते॒\-ऽग्निर॑व॒मो दे॒वता॑नां॒ विष्णुः॑ पर॒मो यदा᳚ग्नावैष्ण॒वमेका॑\-दश\-कपालं नि॒र्वप॑ति दे॒वताः᳚~(४)

%5.5.1.5
ए॒वोभ॒यतः॑ परि॒गृह्य॒ यज॑मा॒नो\-ऽव॑ रुन्धे पुरो॒डाशे॑न॒ वै दे॒वा अ॒मुष्मिँ॑ल्लो॒क आ᳚र्धुवं च॒रुणा॒स्मिन् यः का॒मये॑ता॒मुष्मिँ॑ल्लो॒क ऋ॑ध्नुया॒मिति॒ स पु॑रो॒डाशं॑ कुर्वीता॒मुष्मि॑न्ने॒व लो॒क ऋ॑ध्नोति॒ यद॒ष्टाक॑पाल॒स्तेना᳚ग्ने॒यो यत्त्रि॑कपा॒लस्तेन॑ वैष्ण॒वः समृ॑द्ध्यै॒ यः का॒मये॑ता॒स्मिँल्लो॒क ऋ॑ध्नुया॒मिति॒ स च॒रुं कु॑र्वीता॒ग्नेर्घृ॒तं विष्णो᳚स्तण्डु॒लास्तस्मा᳚त्~(५)

%5.5.1.6
च॒रुः का॒र्यो᳚\-ऽस्मिन्ने॒व लो॒क ऋ॑ध्नोत्यादि॒त्यो भ॑वती॒यं वा अदि॑तिर॒स्यामे॒व प्रति॑ तिष्ठ॒त्यथो॑ अ॒स्यामे॒वाधि॑ य॒ज्ञं त॑नुते॒ यो वै सं॑वथ्स॒रमुख्य॒मभृ॑त्वा॒ग्निं चि॑नु॒ते यथा॑ सा॒मि गर्भो॑\-ऽव॒पद्य॑ते ता॒दृगे॒व तदार्ति॒मार्च्छे᳚द्वैश्वान॒रं द्वाद॑श\-कपालं पु॒रस्ता॒न्निर्व॑पेथ्संवथ्स॒रो वा अ॒ग्निर्वै᳚श्वान॒रो यथा॑ संवथ्स॒रमा॒प्त्वा~(६)

%5.5.1.7
का॒ल आग॑ते वि॒जाय॑त ए॒वमे॒व सं॑वथ्स॒रमा॒प्त्वा का॒ल आग॑ते॒\-ऽग्निं चि॑नुते॒ नार्ति॒मार्च्छ॑त्ये॒षा वा अ॒ग्नेः प्रि॒या त॒नूर्यद्वै᳚श्वान॒रः प्रि॒यामे॒वास्य॑ त॒नुव॒मव॑ रुन्धे॒ त्रीण्ये॒तानि॑ ह॒वीꣳषि॑ भवन्ति॒ त्रय॑ इ॒मे लो॒का ए॒षां लो॒काना॒ꣳ॒ रोहा॑य॥~(७)

{\anuvakamend[{य॒द्रिय॑ङ्वा॒युर्यच्छ्म॑श्रु॒णस्तद्वा॒योर्नि॒र्वप॑ति दे॒वता॒स्तस्मा॑दा॒प्त्वाष्टात्रिꣳ॑शच्च}]}%~(१)

%5.5.2.1
प्र॒जा\-प॑तिः प्र॒जाः सृ॒ष्ट्वा प्रे॒णानु॒ प्रावि॑श॒त्ताभ्यः॒ पुनः॒ सम्भ॑वितुं॒ नाश॑क्नो॒थ्सो᳚\-ऽब्रवीदृ॒ध्नव॒दिथ्स यो मे॒तः पुनः॑ सञ्चि॒नव॒दिति॒ तं दे॒वाः सम॑चिन्व॒न्ततो॒ वै त आ᳚र्ध्नुव॒न्॒ यथ्स॒मचि॑न्व॒न्तच्चित्य॑स्य चित्य॒त्वं य ए॒वं वि॒द्वान॒ग्निं चि॑नु॒त ऋ॒ध्नोत्ये॒व कस्मै॒ कम॒ग्निश्ची॑यत॒ इत्या॑हुरग्नि॒वान्~(८)

%5.5.2.2
अ॒सा॒नीति॒ वा अ॒ग्निश्ची॑यते\-ऽग्नि॒वाने॒व भ॑वति॒ कस्मै॒ कम॒ग्निश्ची॑यत॒ इत्या॑हुर्दे॒वा मा॑ वेद॒न्निति॒ वा अ॒ग्निश्ची॑यते वि॒दुरे॑नं दे॒वाः कस्मै॒ कम॒ग्निश्ची॑यत॒ इत्या॑हुर्गृ॒ह्य॑सा॒नीति॒ वा अ॒ग्निश्ची॑यते गृ॒ह्ये॑व भ॑वति॒ कस्मै॒ कम॒ग्निश्ची॑यत॒ इत्या॑हुः पशु॒मान॑सा॒नीति॒ वा अ॒ग्निः~(९)

%5.5.2.3
ची॒य॒ते॒ प॒शु॒माने॒व भ॑वति॒ कस्मै॒ कम॒ग्निश्ची॑यत॒ इत्या॑हुः स॒प्त मा॒ पुरु॑षा॒ उप॑ जीवा॒निति॒ वा अ॒ग्निश्ची॑यते॒ त्रयः॒ प्राञ्च॒स्त्रयः॑ प्र॒त्यं च॑ आ॒त्मा स॑प्त॒म ए॒ताव॑न्त ए॒वैन॑म॒मुष्मिँ॑ल्लो॒क उप॑ जीवन्ति प्र॒जा\-प॑तिर॒ग्निम॑चिकीषत॒ तं पृ॑थि॒व्य॑ब्रवी॒न्न मय्य॒ग्निं चे᳚ष्य॒सेति॑ मा धक्ष्यति॒ सा त्वा॑तिद॒ह्यमा॑ना॒ वि ध॑विष्ये~(१०)

%5.5.2.4
स पापी॑यान्भविष्य॒सीति॒ सो᳚\-ऽब्रवी॒त्तथा॒ वा अ॒हं क॑रिष्यामि॒ यथा᳚ त्वा॒ नाति॑ध॒क्ष्यतीति॒ स इ॒माम॒भ्य॑मृशत् प्र॒जा\-प॑तिस्त्वा सादयतु॒ तया॑ दे॒वत॑याङ्गिर॒स्वद्ध्रु॒वा सी॒देती॒मामे॒वेष्ट॑कां कृ॒त्वोपा॑ध॒त्तान॑तिदाहाय॒ यत्प्रत्य॒ग्निं चि॑न्वी॒त तद॒भि मृ॑शेत्प्र॒जा\-प॑तिस्त्वा सादयतु॒ तया॑ दे॒वत॑याङ्गिर॒स्वद्ध्रु॒वा सी॑द~(११)

%5.5.2.5
इती॒मामे॒वेष्ट॑कां कृ॒त्वोप॑ ध॒त्ते\-ऽन॑तिदाहाय प्र॒जा\-प॑तिरकामयत॒ प्र जा॑ये॒येति॒ स ए॒तमुख्य॑मपश्य॒त्तꣳ सं॑वथ्स॒रम॑बिभ॒स्ततो॒ वै स प्राजा॑यत॒ तस्मा᳚थ्संवथ्स॒रं भा॒र्यः॑ प्रैव जा॑यते॒ तं वस॑वो\-ऽब्रुव॒न्प्र त्वम॑जनिष्ठा व॒यं प्र जा॑यामहा॒ इति॒ तं वसु॑भ्यः॒ प्राय॑च्छ॒त्तं त्रीण्यहा᳚न्यबिभरु॒स्तेन॑~(१२)

%5.5.2.6
त्रीणि॑ च श॒तान्यसृ॑जन्त॒ त्रय॑स्त्रिꣳशतं च॒ तस्मा᳚त्त्र्य॒हं भा॒र्यः॑ प्रैव जा॑यते॒ तान्रु॒द्रा अ॑ब्रुव॒न्प्र यू॒यम॑जनिढ्वं व॒यं प्र जा॑यामहा॒ इति॒ तꣳ रु॒द्रेभ्यः॒ प्राय॑च्छ॒न्तꣳ षडहा᳚न्यबिभरु॒स्तेन॒ त्रीणि॑ च श॒तान्यसृ॑जन्त॒ त्रय॑स्त्रिꣳशतं च॒ तस्मा᳚त्षड॒हं भा॒र्यः॑ प्रैव जा॑यते॒ ताना॑दि॒त्या अ॑ब्रुव॒न्प्र यू॒यम॑जनिढ्वं व॒यं ~(१३)

%5.5.2.7
प्र जा॑यामहा॒ इति॒ तमा॑दि॒त्येभ्यः॒ प्राय॑च्छ॒न्तं द्वाद॒शाहा᳚न्यबिभरु॒स्तेन॒ त्रीणि॑ च श॒तान्यसृ॑जन्त॒ त्रय॑स्त्रिꣳशतं च॒ तस्मा᳚द्द्वादशा॒हं भा॒र्यः॑ प्रैव जा॑यते॒ तेनै॒व ते स॒हस्र॑मसृजन्तो॒खाꣳ स॑हस्रत॒मीं य ए॒वमुख्यꣳ॑ साह॒स्रं वेद॒ प्र स॒हस्रं॑ प॒शूना᳚प्नोति॥~(१४)

{\anuvakamend[{अ॒ग्नि॒वान्प॑शु॒मान॑सा॒नीति॒ वा अ॒ग्निर्ध॑विष्ये मृशेत्प्र॒जा\-प॑तिस्त्वा सादयतु॒ तया॑ दे॒वत॑याङ्गिर॒स्वद्ध्रु॒वा सी॑द॒ तेन॒ ताना॑दि॒त्या अ॑ब्रुव॒न्प्र यू॒यम॑जनिढ्वं व॒यञ्च॑त्वारि॒ꣳ॒शच्च॑}]}%~(२)

%5.5.3.1
यजु॑षा॒ वा ए॒षा क्रि॑यते॒ यजु॑षा पच्यते॒ यजु॑षा॒ वि मु॑च्यते॒ यदु॒खा सा वा ए॒षैतर्\mbox{}हि॑ या॒तया᳚म्नी॒ सा न पुनः॑ प्र॒युज्येत्या॑हु॒रग्ने॑ यु॒क्ष्वा हि ये तव॑ यु॒क्ष्वा हि दे॑व॒हूत॑मा॒ꣳ॒ इत्यु॒खायां᳚ जुहोति॒ तेनै॒वैनां॒ पुनः॒ प्र यु॑ङ्क्ते॒ तेनाया॑तयाम्नी॒ यो वा अ॒ग्निं योग॒ आग॑ते यु॒नक्ति॑ यु॒ङ्क्ते यु॑ञ्जा॒नेष्वग्ने᳚~(१५)

%5.5.3.2
यु॒क्ष्वा हि ये तव॑ यु॒क्ष्वा हि दे॑व॒हूत॑मा॒ꣳ॒ इत्या॑है॒ष वा अ॒ग्नेर्योग॒स्तेनै॒वैनं॑ युनक्ति यु॒ङ्क्ते यु॑ञ्जा॒नेषु॑ ब्रह्मवा॒दिनो॑ वदन्ति न्य॑ङ्ङ॒ग्निश्चे॑त॒व्या(३) उ॑त्ता॒ना(३) इति॒ वय॑सां॒ वा ए॒ष प्र॑ति॒मया॑ चीयते॒ यद॒ग्निर्यन्न्य॑ञ्चं चिनु॒यात्पृ॑ष्टि॒त ए॑न॒माहु॑तय ऋच्छेयु॒र्यदु॑त्ता॒नं न पति॑तुꣳ शक्नुया॒दसु॑वर्ग्यो\-ऽस्य स्यात्प्रा॒चीन॑मुत्ता॒नम्~(१६)

%5.5.3.3
पु॒रु॒ष॒शी॒र्॒\mbox{}षमुप॑ दधाति मुख॒त ए॒वैन॒माहु॑तय ऋच्छन्ति॒ नोत्ता॒नं चि॑नुते सुव॒र्ग्यो᳚\-ऽस्य भवति सौ॒र्या जु॑होति॒ चक्षु॑रे॒वास्मि॒न्प्रति॑ दधाति॒ द्विर्जु॑होति॒ द्वे हि चक्षु॑षी समा॒न्या जु॑होति समा॒नꣳ हि चक्षुः॒ समृ॑द्ध्यै देवासु॒राः संय॑त्ता आस॒न्ते वा॒मं वसु॒ सं न्य॑दधत॒ तद्दे॒वा वा॑म॒भृता॑वृञ्जत॒ तद्वा॑म॒भृतो॑ वामभृ॒त्त्वं यद्वा॑म॒भृत॑मुप॒दधा॑ति वा॒ममे॒व तया॒ वसु॒ यज॑मानो॒ भ्रातृ॑व्यस्य वृङ्क्ते॒ हिर॑ण्यमूर्ध्नी भवति॒ ज्योति॒र्वै हिर॑ण्यं॒ ज्योति॑र्वा॒मं ज्योति॑षै॒वास्य॒ ज्योति॑र्वा॒मं वृ॑ङ्क्ते द्विय॒जुर्भ॑वति॒ प्रति॑ष्ठित्यै॥~(१७)

{\anuvakamend[{यु॒ञ्जा॒नेष्वग्ने᳚ प्रा॒चीन॑मुत्ता॒नं वा॑म॒भृत॒ञ्चतु॑र्विꣳशतिश्च}]}%~(३)

%5.5.4.1
आपो॒ वरु॑णस्य॒ पत्न॑य आस॒न्ता अ॒ग्निर॒भ्य॑ध्याय॒त्ताः सम॑भव॒त्तस्य॒ रेतः॒ परा॑पत॒त्तदि॒यम॑भव॒द्यद्द्वि॒तीयं॑ प॒राप॑त॒त्तद॒सा\-व॑भवदि॒यं वै वि॒राड॒सौ स्व॒राड्यद्वि॒राजा॑वुप॒दधा॑ती॒मे ए॒वोप॑ धत्ते॒ यद्वा अ॒सौ रेतः॑ सि॒ञ्चति॒ तद॒स्यां प्रति॑ तिष्ठति॒ तत्प्र जा॑यते॒ ता ओष॑धयः~(१८)

%5.5.4.2
वी॒रुधो॑ भवन्ति॒ ता अ॒ग्निर॑त्ति॒ य ए॒वं वेद॒ प्रैव जा॑यते\-ऽन्ना॒दो भ॑वति॒ यो रे॑त॒स्वी स्यात्प्र॑थ॒मायां॒ तस्य॒ चित्या॑मु॒भे उप॑ दध्यादि॒मे ए॒वास्मै॑ स॒मीची॒ रेतः॑ सिञ्चतो॒ यः सि॒क्तरे॑ताः॒ स्यात्प्र॑थ॒मायां॒ तस्य॒ चित्या॑म॒न्यामुप॑ दध्यादुत्त॒माया॑\-म॒न्याꣳ रेत॑ ए॒वास्य॑ सि॒क्तमा॒भ्यामु॑भ॒यतः॒ परि॑ गृह्णाति संवथ्स॒रं न कम्~(१९)

%5.5.4.3
च॒न प्र॒त्यव॑रोहे॒न्न हीमे कं च॒न प्र॑त्यव॒रोह॑त॒स्तदे॑नयोर्व्र॒तं यो वा अप॑शीर्\mbox{}षाणम॒ग्निं चि॑नु॒ते\-ऽप॑शीर्\mbox{}षा॒मुष्मिँ॑ल्लो॒के भ॑वति॒ यः सशी॑र्\mbox{}षाणं चिनु॒ते सशी॑र्\mbox{}षा॒मुष्मिँ॑ल्लो॒के भ॑वति॒ चित्तिं॑ जुहोमि॒ मन॑सा घृ॒तेन॒ यथा॑ दे॒वा इ॒हागम॑न्वी॒तिहो᳚त्रा ऋता॒वृधः॑ समु॒द्रस्य॑ व॒युन॑स्य॒ पत्म॑ञ्जु॒होमि॑ वि॒श्वक॑र्मणे॒ विश्वाहाम॑र्त्यꣳ ह॒विरिति॑ स्वयमातृ॒ण्णामु॑प॒धाय॑ जुहोति~(२०)

%5.5.4.4
ए॒तद्वा अ॒ग्नेः शिरः॒ सशी॑र्\mbox{}षाणमे॒वाग्निं चि॑नुते॒ सशी॑र्\mbox{}षा॒मुष्मिँ॑ल्लो॒के भ॑वति॒ य ए॒वं वेद॑ सुव॒र्गाय॒ वा ए॒ष लो॒काय॑ चीयते॒ यद॒ग्निस्तस्य॒ यदय॑थापूर्वं क्रि॒यते\-ऽसु॑वर्ग्यमस्य॒ तथ्सु॑व॒र्ग्यो᳚\-ऽग्निश्चिति॑मुप॒धाया॒भि मृ॑शे॒च्चित्ति॒मचि॑त्तिं चिनव॒द्वि वि॒द्वान्पृ॒ष्ठेव॑ वी॒ता वृ॑जि॒ना च॒ मर्ता᳚न्रा॒ये च॑ नः स्वप॒त्याय॑ देव॒ दितिं॑ च॒ रास्वादि॑तिमुरु॒ष्येति॑ यथापू॒र्वमे॒वैना॒मुप॑ धत्ते॒ प्राञ्च॑मेनं चिनुते सुव॒र्ग्यो᳚\-ऽस्य भवति॥~(२१)

{\anuvakamend[{ओष॑धयः॒ कञ्जु॑होति स्वप॒त्याया॒ष्टाद॑श च}]}%~(४)

%5.5.5.1
वि॒श्वक॑र्मा दि॒शां पतिः॒ स नः॑ प॒शून्पा॑तु॒ सो᳚\-ऽस्मान्पा॑तु॒ तस्मै॒ नमः॑ प्र॒जा\-प॑ती रु॒द्रो वरु॑णो॒\-ऽग्निर्दि॒शां पतिः॒ स नः॑ प॒शून्पा॑तु॒ सो᳚\-ऽस्मान्पा॑तु॒ तस्मै॒ नम॑ ए॒ता वै दे॒वता॑ ए॒तेषां᳚ पशू॒नामधि॑पतय॒स्ताभ्यो॒ वा ए॒ष आ वृ॑श्च्यते॒ यः प॑शुशी॒र्॒\mbox{}षाण्यु॑प॒दधा॑ति हिरण्येष्ट॒का उप॑ दधात्ये॒ताभ्य॑ ए॒व दे॒वता᳚भ्यो॒ नम॑स्करोति ब्रह्मवा॒दिनः॑~(२२)

%5.5.5.2
व॒द॒न्त्य॒ग्नौ ग्रा॒म्यान्प॒शून्प्र द॑धाति शु॒चार॒ण्यान॑र्पयति॒ किं तत॒ उच्छिꣳ॑ष॒तीति॒ यद्धि॑रण्येष्ट॒का उ॑प॒दधा᳚त्य॒मृतं॒ वै हिर॑ण्यम॒मृते॑नै॒व ग्रा॒म्येभ्यः॑ प॒शुभ्यो॑ भेष॒जं क॑रोति॒ नैनान्॑ हिनस्ति प्रा॒णो वै प्र॑थ॒मा स्व॑यमातृ॒ण्णा व्या॒नो द्वि॒तीया॑पा॒नस्तृ॒तीयानु॒ प्राण्या᳚त्प्रथ॒माꣴ स्व॑यमातृ॒ण्णामु॑प॒धाय॑ प्रा॒णेनै॒व प्रा॒णꣳ सम॑र्धयति॒ व्य॑न्यात्~(२३)

%5.5.5.3
द्वि॒तीया॑मुप॒धाय॑ व्या॒नेनै॒व व्या॒नꣳ सम॑र्धय॒त्यपा᳚न्यात्तृ॒तीया॑मुप॒धाया॑पा॒नेनै॒वापा॒नꣳ सम॑र्धय॒त्यथो᳚ प्रा॒णैरे॒वैन॒ꣳ॒ समि॑न्द्धे॒ भूर्भुवः॒ सुव॒रिति॑ स्वयमातृ॒ण्णा उप॑ दधाती॒मे वै लो॒काः स्व॑यमातृ॒ण्णा ए॒ताभिः॒ खलु॒ वै व्याहृ॑तीभिः प्र॒जा\-प॑तिः॒ प्राजा॑यत॒ यदे॒ताभि॒र्व्याहृ॑तीभिः स्वयमातृ॒ण्णा उ॑प॒दधा॑ती॒माने॒व लो॒कानु॑प॒धायै॒षु~(२४)

%5.5.5.4
लो॒केष्वधि॒ प्र जा॑यते प्रा॒णाय॑ व्या॒नाया॑पा॒नाय॑ वा॒चे त्वा॒ चक्षु॑षे त्वा॒ तया॑ दे॒वत॑याङ्गिर॒स्वद्ध्रु॒वा सी॑दा॒ग्निना॒ वै दे॒वाः सु॑व॒र्गं लो॒कम॑जिगाꣳस॒न्तेन॒ पति॑तुं॒ नाश॑क्नुव॒न्त ए॒ताश्चत॑स्रः स्वयमातृ॒ण्णा अ॑पश्य॒न्ता दि॒क्षूपा॑दधत॒ तेन॑ स॒र्वत॑श्चक्षुषा सुव॒र्गं लो॒कमा॑य॒न्॒यच्चत॑स्रः स्वयमातृ॒ण्णा दि॒क्षू॑प॒दधा॑ति स॒र्वत॑श्चक्षुषै॒व तद॒ग्निना॒ यज॑मानः सुव॒र्गं लो॒कमे॑ति॥~(२५)

{\anuvakamend[{ब्र॒ह्म॒वा॒दिनो॒ व्य॑न्यादे॒षु यज॑मान॒स्त्रीणि॑ च}]}%~(५)

%5.5.6.1
अग्न॒ आ या॑हि वी॒तय॒ इत्या॒हाह्व॑तै॒वैन॑म॒ग्निं दू॒तं वृ॑णीमह॒ इत्या॑ह हू॒त्वैवैनं॑ वृणीते॒\-ऽग्निना॒ग्निः समि॑ध्यत॒ इत्या॑ह॒ समि॑न्द्ध ए॒वैन॑म॒ग्निर्वृ॒त्राणि॑ जङ्घन॒दित्या॑ह॒ समि॑द्ध ए॒वास्मि॑न्निन्द्रि॒यं द॑धात्य॒ग्नेः स्तोम॑म्मनामह॒ इत्या॑ह मनु॒त ए॒वैन॑मे॒तानि॒ वा अह्नाꣳ॑ रू॒पाणि॑~(२६)

%5.5.6.2
अ॒न्व॒हमे॒वैनं॑ चिनु॒ते\-ऽवाह्नाꣳ॑ रू॒पाणि॑ रुन्धे ब्रह्मवा॒दिनो॑ वदन्ति॒ कस्मा᳚थ्स॒त्याद्या॒तया᳚म्नीर॒न्या इष्ट॑का॒ अया॑तयाम्नी लोकं पृ॒णेत्यै᳚न्द्रा॒ग्नी हि बा॑र्\mbox{}हस्प॒त्येति॑ ब्रूयादिन्द्रा॒ग्नी च॒ हि दे॒वानां॒ बृह॒स्पति॒श्चाया॑तयामानो\-ऽनुच॒रव॑ती भव॒त्यजा॑मित्वायानु॒ष्टुभानु॑ चरत्या॒त्मा वै लो॑कं पृ॒णा प्रा॒णो॑\-ऽनु॒ष्टुप्तस्मा᳚त्प्रा॒णः सर्वा॒ण्यङ्गा॒न्यनु॑ चरति॒ ता अ॑स्य॒ सूद॑दोहसः~(२७)

%5.5.6.3
इत्या॑ह॒ तस्मा॒त्परु॑षिपरुषि॒ रसः॒ सोमꣴ॑ श्रीणन्ति॒ पृश्ञ॑य॒ इत्या॒हान्नं॒ वै पृश्ञ्यन्न॑मे॒वाव॑ रुन्धे॒\-ऽर्को वा अ॒ग्निर॒र्को\-ऽन्न॒मन्न॑मे॒वाव॑ रुन्धे॒ जन्मं॑ दे॒वानां॒ विश॑स्त्रि॒ष्वा रो॑च॒ने दि॒व इत्या॑हे॒माने॒वास्मै॑ लो॒कां ज्योति॑ष्मतः करोति॒ यो वा इष्ट॑कानां प्रति॒ष्ठां वेद॒ प्रत्ये॒व ति॑ष्ठति॒ तया॑ दे॒वत॑याङ्गिर॒स्वद्ध्रु॒वा सी॒देत्या॑है॒षा वा इष्ट॑कानां प्रति॒ष्ठा य ए॒वं वेद॒ प्रत्ये॒व ति॑ष्ठति॥~(२८)

{\anuvakamend[{रू॒पाणि॒ सूद॑दोहस॒स्तया॒ षोड॑श च}]}%~(६)

%5.5.7.1
सु॒व॒र्गाय॒ वा ए॒ष लो॒काय॑ चीयते॒ यद॒ग्निर्वज्र॑ एकाद॒शिनी॒ यद॒ग्नावे॑काद॒शिनीं᳚ मिनु॒याद्वज्रे॑णैनꣳ सुव॒र्गाल्लो॒का\-द॒न्तर्द॑ध्या॒द्यन्न मि॑नु॒याथ्स्वरु॑भिः प॒शून्व्य॑र्धयेदेकयू॒पं मि॑नोति॒ नैनं॒ वज्रे॑ण सुव॒र्गाल्लो॒काद॑न्त॒र्दधा॑ति॒ न स्वरु॑भिः प॒शून्व्य॑र्धयति॒ वि वा ए॒ष इ॑न्द्रि॒येण॑ वी॒र्ये॑णर्ध्यते॒ यो᳚\-ऽग्निं चि॒न्वन्न॑धि॒क्राम॑त्यैन्द्रि॒या~(२९)

%5.5.7.2
ऋ॒चाक्रम॑णं॒ प्रतीष्ट॑का॒मुप॑ दध्या॒न्नेन्द्रि॒येण॑ वी॒र्ये॑ण॒ व्यृ॑ध्यते रु॒द्रो वा ए॒ष यद॒ग्निस्तस्य॑ ति॒स्रः श॑र॒व्याः᳚ प्र॒तीची॑ ति॒रश्च्य॒नूची॒ ताभ्यो॒ वा ए॒ष आ वृ॑श्च्यते॒ यो᳚\-ऽग्निं चि॑नु॒ते᳚\-ऽग्निं चि॒त्वा ति॑सृध॒न्वमया॑चितं ब्राह्म॒णाय॑ दद्या॒त्ताभ्य॑ ए॒व नम॑स्करो॒त्यथो॒ ताभ्य॑ ए॒वात्मानं॒ निष्क्री॑णीते॒ यत्ते॑ रुद्र पु॒रः~(३०)

%5.5.7.3
धनु॒स्तद्वातो॒ अनु॑ वातु ते॒ तस्मै॑ ते रुद्र संवथ्स॒रेण॒ नम॑स्करोमि॒ यत्ते॑ रुद्र दक्षि॒णा धनु॒स्तद्वातो॒ अनु॑ वातु ते॒ तस्मै॑ ते रुद्र परिवथ्स॒रेण॒ नम॑स्करोमि॒ यत्ते॑ रुद्र प॒श्चाद्धनु॒स्तद्वातो॒ अनु॑ वातु ते॒ तस्मै॑ ते रुद्रेदावथ्स॒रेण॒ नम॑स्करोमि॒ यत्ते॑ रुद्रोत्त॒राद्धनु॒स्तत्~(३१)

%5.5.7.4
वातो॒ अनु॑ वातु ते॒ तस्मै॑ ते रुद्रेदुवथ्स॒रेण॒ नम॑स्करोमि॒ यत्ते॑ रुद्रो॒परि॒ धनु॒स्तद्वातो॒ अनु॑ वातु ते॒ तस्मै॑ ते रुद्र वथ्स॒रेण॒ नम॑स्करोमि रु॒द्रो वा ए॒ष यद॒ग्निः स यथा᳚ व्या॒घ्रः क्रु॒द्धस्तिष्ठ॑त्ये॒वं वा ए॒ष ए॒तर्\mbox{}हि॒ सञ्चि॑तमे॒तैरुप॑ तिष्ठते नमस्का॒रैरे॒वैनꣳ॑ शमयति॒ ये᳚\-ऽग्नयः॑~(३२)

%5.5.7.5
पु॒री॒ष्याः᳚ प्रवि॑ष्टाः पृथि॒वीमनु॑। तेषां॒ त्वम॑स्युत्त॒मः प्र णो॑ जी॒वात॑वे सुव। आपं॑ त्वा\-ऽग्ने॒ मन॒सापं॑ त्वा\-ऽग्ने॒ तप॒सापं॑ त्वाऽग्ने दी॒क्षयापं॑ त्वाग्न उप॒सद्भि॒रापं॑ त्वाऽग्ने सु॒त्ययापं॑ त्वा\-ऽग्ने॒ दक्षि॑णाभि॒रापं॑ त्वाऽग्ने\-ऽवभृ॒थेनापं॑ त्वाऽग्ने व॒शयापं॑ त्वाऽग्ने स्वगाका॒रेणेत्या॑है॒षा वा अ॒ग्नेराप्ति॒स्तयै॒वैन॑माप्नोति॥~(३३)

{\anuvakamend[{ऐ॒न्द्रि॒या पु॒र उ॑त्त॒राद्धनु॒स्तद॒ग्नय॑ आहा॒ष्टौ च॑}]}%~(७)

%5.5.8.1
गा॒य॒त्रेण॑ पु॒रस्ता॒दुप॑ तिष्ठते प्रा॒णमे॒वास्मि॑न्दधाति बृहद्रथन्त॒रा\-भ्यां᳚ प॒क्षावोज॑ ए॒वास्मि॑न्दधात्यृतु॒स्थाय॑ज्ञाय॒ज्ञिये॑न॒ पुच्छ॑मृ॒तुष्वे॒व प्रति॑ तिष्ठति पृ॒ष्ठैरुप॑ तिष्ठते॒ तेजो॒ वै पृ॒ष्ठानि॒ तेज॑ ए॒वास्मि॑न्दधाति प्र॒जा\-प॑तिर॒ग्निम॑सृजत॒ सो᳚\-ऽस्माथ्सृ॒ष्टः परा॑ङै॒त्तं वा॑रव॒न्तीये॑नावारयत॒ तद्वा॑रव॒न्तीय॑स्य वारवन्तीय॒त्वꣴ श्यै॒तेन॑ श्ये॒ती अ॑कुरुत॒ तच्छ्यै॒तस्य॑ श्यैत॒त्वम्~(३४)

%5.5.8.2
यद्वा॑रव॒न्तीये॑नोप॒तिष्ठ॑ते वा॒रय॑त ए॒वैनꣴ॑ श्यै॒तेन॑ श्ये॒ती कु॑रुते प्र॒जा\-प॑ते॒र्॒\mbox{}हृद॑येनापिप॒क्षं प्रत्युप॑ तिष्ठते प्रे॒माण॑मे॒वास्य॑ गच्छति॒ प्राच्या᳚ त्वा दि॒शा सा॑दयामि गाय॒त्रेण॒ छन्द॑सा॒ग्निना॑ दे॒वत॑या॒ग्नेः शी॒र्ष्णाग्नेः शिर॒ उप॑ दधामि॒ दक्षि॑णया त्वा दि॒शा सा॑दयामि॒ त्रैष्टु॑भेन॒ छन्द॒सेन्द्रे॑ण दे॒वत॑या॒ग्नेः प॒क्षेणा॒ग्नेः प॒क्षमुप॑ दधामि प्र॒तीच्या᳚ त्वा दि॒शा सा॑दयामि~(३५)

%5.5.8.3
जाग॑तेन॒ छन्द॑सा सवि॒त्रा दे॒वत॑या॒ग्नेः पुच्छे॑ना॒ग्नेः पुच्छ॒मुप॑ दधा॒म्युदी᳚च्या त्वा दि॒शा सा॑दया॒म्यानु॑ष्टुभेन॒ छन्द॑सा मि॒त्रावरु॑णाभ्यां दे॒वत॑या॒ग्नेः प॒क्षेणा॒ग्नेः प॒क्षमुप॑ दधाम्यू॒र्ध्वया᳚ त्वा दि॒शा सा॑दयामि॒ पाङ्क्ते॑न॒ छन्द॑सा॒ बृह॒स्पति॑ना दे॒वत॑या॒ग्नेः पृ॒ष्ठेना॒ग्नेः पृ॒ष्ठमुप॑ दधामि॒ यो वा अपा᳚त्मानम॒ग्निं चि॑नु॒ते\-ऽपा᳚त्मा॒मुष्मिँ॑ल्लो॒के भ॑वति॒ यः सात्मा॑नं चिनु॒ते सात्मा॒मुष्मिँ॑ल्लो॒के भ॑वत्यात्मेष्ट॒का उप॑ दधात्ये॒ष वा अ॒ग्नेरा॒त्मा सात्मा॑नमे॒वाग्निं चि॑नुते॒ सात्मा॒मुष्मिँ॑ल्लो॒के भ॑वति॒ य ए॒वं वेद॑॥~(३६)

{\anuvakamend[{श्यै॒त॒त्वं प्र॒तीच्या᳚ त्वा दि॒शा सा॑दयामि॒ यः सात्मा॑नञ्चिनु॒ते द्वाविꣳ॑शतिश्च}]}%~(८)

%5.5.9.1
अग्न॑ उदधे॒ या त॒ इषु॑र्यु॒वा नाम॒ तया॑ नो मृड॒ तस्या᳚स्ते॒ नम॒स्तस्या᳚स्त॒ उप॒ जीव॑न्तो भूया॒स्माग्ने॑ दुध्र गह्य किꣳशिल वन्य॒ या त॒ इषु॑र्यु॒वा नाम॒ तया॑ नो मृड॒ तस्या᳚स्ते॒ नम॒स्तस्या᳚स्त॒ उप॒ जीव॑न्तो भूयास्म॒ पञ्च॒ वा ए॒ते᳚\-ऽग्नयो॒ यच्चित॑य उद॒धिरे॒व नाम॑ प्रथ॒मो दु॒ध्रः~(३७)

%5.5.9.2
द्वि॒तीयो॒ गह्य॑स्तृ॒तीयः॑ किꣳशि॒लश्च॑तु॒र्थो वन्यः॑ पञ्च॒मस्तेभ्यो॒ यदाहु॑ती॒र्न जु॑हु॒याद॑ध्व॒र्युं च॒ यज॑मानं च॒ प्र द॑हेयु॒र्यदे॒ता आहु॑तीर्जु॒होति॑ भाग॒धेये॑नै॒वैना᳚ञ्छमयति॒ नार्ति॒मार्च्छ॑त्यध्व॒र्युर्न यज॑मानो॒ वाङ्म॑ आ॒सन्न॒सोः प्रा॒णो᳚\-ऽक्ष्योश्चक्षुः॒ कर्ण॑योः॒ श्रोत्रं॑ बाहु॒वोर्बल॑मूरु॒वोरोजो\-ऽरि॑ष्टा॒ विश्वा॒न्यङ्गा॑नि त॒नूः~(३८)

%5.5.9.3
त॒नुवा॑ मे स॒ह नम॑स्ते अस्तु॒ मा मा॑ हिꣳसी॒रप॒ वा ए॒तस्मा᳚त्प्रा॒णाः क्रा॑मन्ति॒ यो᳚\-ऽग्निं चि॒न्वन्न॑धि॒क्राम॑ति॒ वाङ्म॑ आ॒सन्न॒सोः प्रा॒ण इत्या॑ह प्रा॒णाने॒वा\-ऽऽ\-त्मन्ध॑त्ते॒ यो रु॒द्रो अ॒ग्नौ यो अ॒फ्सु य ओष॑धीषु॒ यो रु॒द्रो विश्वा॒ भुव॑नावि॒वेश॒ तस्मै॑ रु॒द्राय॒ नमो॑ अ॒स्त्वाहु॑तिभागा॒ वा अ॒न्ये रु॒द्रा ह॒विर्भा॑गाः~(३९)

%5.5.9.4
अ॒न्ये श॑तरु॒द्रीयꣳ॑ हु॒त्वा गा॑वीधु॒कं च॒रुमे॒तेन॒ यजु॑षा चर॒माया॒मिष्ट॑कायां॒ नि द॑ध्याद्भाग॒धेये॑नै॒वैनꣳ॑ शमयति॒ तस्य॒ त्वै श॑तरु॒द्रीयꣳ॑ हु॒तमित्या॑हु॒र्यस्यै॒तद॒ग्नौ क्रि॒यत॒ इति॒ वस॑वस्त्वा रु॒द्रैः पु॒रस्ता᳚त्पान्तु पि॒तर॑स्त्वा य॒मरा॑जानः पि॒तृभि॑र्दक्षिण॒तः पा᳚न्त्वादि॒त्यास्त्वा॒ विश्वै᳚र्दे॒वैः प॒श्चात्पा᳚न्तु द्युता॒नस्त्वा॑ मारु॒तो म॒रुद्भि॑रुत्तर॒तः पा॑तु~(४०)

%5.5.9.5
दे॒वास्त्वेन्द्र॑ज्येष्ठा॒ वरु॑णराजानो॒\-ऽधस्ता᳚च्चो॒परि॑ष्टाच्च पान्तु॒ न वा ए॒तेन॑ पू॒तो न मेध्यो॒ न प्रोक्षि॑तो॒ यदे॑न॒मतः॑ प्रा॒चीनं॑ प्रो॒क्षति॒ यथ्सञ्चि॑त॒माज्ये॑न प्रो॒क्षति॒ तेन॑ पू॒तस्तेन॒ मेध्य॒स्तेन॒ प्रोक्षि॑तः॥~(४१)

{\anuvakamend[{दु॒ध्रस्त॒नूर्\mbox{}ह॒विर्भा॑गाः पातु॒ द्वात्रिꣳ॑शच्च}]}%~(९)

%5.5.10.1
स॒मीची॒ नामा॑सि॒ प्राची॒ दिक्तस्या᳚स्ते॒\-ऽग्निरधि॑पतिरसि॒तो र॑क्षि॒ता यश्चाधि॑पति॒र्यश्च॑ गो॒प्ता ताभ्यां॒ नम॒स्तौ नो॑ मृडयता॒न्ते यं द्वि॒ष्मो यश्च॑ नो॒ द्वेष्टि॒ तं वां॒ जम्भे॑ दधाम्योज॒स्विनी॒ नामा॑सि दक्षि॒णा दिक्तस्या᳚स्त॒ इन्द्रो\-ऽधि॑पतिः॒ पृदा॑कुः॒ प्राची॒ नामा॑सि प्र॒तीची॒ दिक्तस्या᳚स्ते~(४२)

%5.5.10.2
सोमो\-ऽधि॑पतिः स्व॒जो॑\-ऽव॒स्थावा॒ नामा॒स्युदी॑ची॒ दिक्तस्या᳚स्ते॒ वरु॒णो\-ऽधि॑पतिस्ति॒रश्च॑राजि॒रधि॑पत्नी॒ नामा॑सि बृह॒ती दिक्तस्या᳚स्ते॒ बृह॒स्पति॒रधि॑पतिः श्वि॒त्रो व॒शिनी॒ नामा॑सी॒यं दिक्तस्या᳚स्ते य॒मो\-ऽधि॑पतिः क॒ल्माष॑ग्रीवो रक्षि॒ता यश्चाधि॑पति॒र्यश्च॑ गो॒प्ता ताभ्यां॒ नम॒स्तौ नो॑ मृडयता॒न्ते यं द्वि॒ष्मो यश्च॑~(४३)

%5.5.10.3
नो॒ द्वेष्टि॒ तं वां॒ जम्भे॑ दधाम्ये॒ता वै दे॒वता॑ अ॒ग्निं चि॒तꣳ र॑क्षन्ति॒ ताभ्यो॒ यदाहु॑ती॒र्न जु॑हु॒याद॑ध्व॒र्युं च॒ यज॑मानं च ध्यायेयु॒र्यदे॒ता आहु॑तीर्जु॒होति॑ भाग॒धेये॑नै॒वैना᳚ञ्छमयति॒ नार्ति॒मार्च्छ॑त्यध्व॒र्युर्न यज॑मानो हे॒तयो॒ नाम॑ स्थ॒ तेषां᳚ वः पु॒रो गृ॒हा अ॒ग्निर्व॒ इष॑वः सलि॒लो निलि॒म्पा नाम॑~(४४)

%5.5.10.4
स्थ॒ तेषां᳚ वो दक्षि॒णा गृ॒हाः पि॒तरो॑ व॒ इष॑वः॒ सग॑रो व॒ज्रिणो॒ नाम॑ स्थ॒ तेषां᳚ वः प॒श्चाद्गृ॒हाः स्वप्नो॑ व॒ इष॑वो॒ गह्व॑रो\-ऽव॒स्थावा॑नो॒ नाम॑ स्थ॒ तेषां᳚ व उत्त॒राद्गृ॒हा आपो॑ व॒ इष॑वः समु॒द्रो\-ऽधि॑पतयो॒ नाम॑ स्थ॒ तेषां᳚ व उ॒परि॑ गृ॒हा व॒र्॒\mbox{}षं व॒ इष॒वो\-ऽव॑स्वान्क्र॒व्या नाम॑ स्थ॒ पार्थि॑वा॒स्तेषां᳚ व इ॒ह गृ॒हाः~(४५)

%5.5.10.5
अन्नं॑ व॒ इष॑वो निमि॒षो वा॑तना॒मन्तेभ्यो॑ वो॒ नम॒स्ते नो॑ मृडयत॒ ते यं द्वि॒ष्मो यश्च॑ नो॒ द्वेष्टि॒ तं वो॒ जम्भे॑ दधामि हु॒तादो॒ वा अ॒न्ये दे॒वा अ॑हु॒तादो॒\-ऽन्ये तान॑ग्नि॒चिदे॒वोभया᳚न्प्रीणाति द॒ध्ना म॑धुमि॒श्रेणै॒ता आहु॑तीर्जुहोति भाग॒धेये॑नै॒वैना᳚न्प्रीणा॒त्यथो॒ खल्वा॑हु॒रिष्ट॑का॒ वै दे॒वा अ॑हु॒ताद॒ इति॑~(४६)

%5.5.10.6
अ॒नु॒प॒रि॒क्रामं॑ जुहो॒त्यप॑रिवर्गमे॒वैना᳚न्प्रीणाती॒मꣴ स्तन॒मूर्ज॑स्वन्तं धया॒पां प्रप्या॑तमग्ने सरि॒रस्य॒ मध्ये᳚। उथ्सं॑ जुषस्व॒ मधु॑मन्तमूर्व समु॒द्रिय॒ꣳ॒ सद॑न॒मा वि॑शस्व। यो वा अ॒ग्निं प्र॒युज्य॒ न वि॑मु॒ञ्चति॒ यथाश्वो॑ यु॒क्तो\-ऽवि॑मुच्यमानः॒ क्षुध्य॑न्परा॒भव॑त्ये॒वम॑स्या॒ग्निः परा॑ भवति॒ तं प॑रा॒भव॑न्तं॒ यज॑मा॒नो\-ऽनु॒ परा॑ भवति॒ सो᳚\-ऽग्निं चि॒त्वा लू॒क्षः~(४७)

%5.5.10.7
भ॒व॒ती॒मꣴ स्तन॒मूर्ज॑स्वन्तं धया॒पामित्याज्य॑स्य पू॒र्णाꣴ स्रुचं॑ जुहोत्ये॒ष वा अ॒ग्नेर्वि॑मो॒को वि॒मुच्यै॒वास्मा॒ अन्न॒मपि॑ दधाति॒ तस्मा॑दाहु॒र्यश्चै॒वं वेद॒ यश्च॒ न सु॒धायꣳ॑ ह॒ वै वा॒जी सुहि॑तो दधा॒तीत्य॒ग्निर्वाव वा॒जी तमे॒व तत्प्री॑णाति॒ स ए॑नं प्री॒तः प्री॑णाति॒ वसी॑यान्भवति॥~(४८)

{\anuvakamend[{प्र॒तीची॒ दिक्तस्या᳚स्ते द्वि॒ष्मो यश्च॑ निलि॒म्पा नामे॒ह गृ॒हा इति॑ लू॒क्षो वसी॑यान्भवति}]}%॥10॥

%5.5.11.1
इन्द्रा॑य॒ राज्ञे॑ सूक॒रो वरु॑णाय॒ राज्ञे॒ कृष्णो॑ य॒माय॒ राज्ञ॒ ऋश्य॑ ऋष॒भाय॒ राज्ञे॑ गव॒यः शा᳚र्दू॒लाय॒ राज्ञे॑ गौ॒रः पु॑रुषरा॒जाय॑ म॒र्कटः॑ क्षिप्रश्ये॒नस्य॒ वर्ति॑का॒ नीलं॑गोः॒ क्रिमिः॒ सोम॑स्य॒ राज्ञः॑ कुलु॒ङ्गः सिन्धोः᳚ शिꣳशु॒मारो॑ हि॒मव॑तो ह॒स्ती॥~(४९)

{\anuvakamend[{इन्द्रा॑या॒ष्टाविꣳ॑शतिः}]}%॥11॥

%5.5.12.1
म॒युः प्रा॑जाप॒त्य ऊ॒लो हली᳚क्ष्णो वृषद॒ꣳ॒शस्ते धा॒तुः सर॑स्वत्यै॒ शारिः॑ श्ये॒ता पु॑रुष॒वाख्सर॑स्वते॒ शुकः॑ श्ये॒तः पु॑रुष॒वागा॑र॒ण्यो॑\-ऽजो न॑कु॒लः शका॒ ते पौ॒ष्णा वा॒चे क्रौ॒ञ्चः॥~(५०)

{\anuvakamend[{म॒युस्त्रयो॑विꣳशतिः}]}%॥12॥

%5.5.13.1
अ॒पां नप्त्रे॑ ज॒षो ना॒क्रो मक॑रः कुली॒कय॒स्ते\-ऽकू॑पारस्य वा॒चे पै᳚ङ्गरा॒जो भगा॑य कु॒षीत॑क आ॒ती वा॑ह॒सो दर्वि॑दा॒ ते वा॑य॒व्या॑ दि॒ग्भ्यश्च॑क्रवा॒कः॥~(५१)

{\anuvakamend[{अ॒पामेका॒न्नविꣳ॑शतिः}]}%॥13॥

%5.5.14.1
बला॑याजग॒र आ॒खुः सृ॑ज॒या श॒यण्ड॑क॒स्ते मै॒त्रा मृ॒त्यवे॑\-ऽसि॒तो म॒न्यवे᳚ स्व॒जः कु॑म्भी॒नसः॑ पुष्करसा॒दो लो॑हिता॒हिस्ते त्वा॒ष्ट्राः प्र॑ति॒श्रुत्का॑यै वाह॒सः॥~(५२)

{\anuvakamend[{}]}

%5.5.15.1
पु॒रु॒ष॒मृ॒गश्च॒न्द्रम॑से गो॒धा काल॑का दार्वाघा॒टस्ते वन॒स्पती॑नामे॒ण्यह्ने॒ कृष्णो॒ रात्रि॑यै पि॒कः क्ष्विङ्का॒ नील॑शीर्ष्णी॒ ते᳚\-ऽर्य॒म्णे धा॒तुः क॑त्क॒टः॥~(५३)

{\anuvakamend[{}]}

%5.5.16.1
सौ॒री ब॒लाकर्श्यो॑ म॒यूरः॑ श्ये॒नस्ते ग॑न्ध॒र्वाणां॒ वसू॑नां क॒पिञ्ज॑लो रु॒द्राणां᳚ तित्ति॒री रो॒हित्कु॑ण्डृ॒णाची॑ गो॒लत्ति॑का॒ ता अ॑फ्स॒रसा॒मर॑ण्याय सृम॒रः॥~(५४)

{\anuvakamend[{}]}

%5.5.17.1
पृ॒ष॒तो वै᳚श्वदे॒वः पि॒त्वो न्यङ्कुः॒ कश॒स्ते\-ऽनु॑मत्या अन्यवा॒पो᳚\-ऽर्धमा॒सानां᳚ मा॒सां क॒श्यपः॒ क्वयिः॑ कु॒टरु॑र्दात्यौ॒हस्ते सि॑नीवा॒ल्यै बृह॒स्पत॑ये शित्पु॒टः॥~(५)

{\anuvakamend[{}]}

%5.5.18.1
शका॑ भौ॒मी पा॒न्त्रः कशो॑ मान्थी॒लव॒स्ते पि॑तृ॒णामृ॑तू॒नां जह॑का संवथ्स॒राय॒ लोपा॑ क॒पोत॒ उलू॑कः श॒शस्ते नैर्॑\mbox{}॑ऋताः कृ॑क॒वाकुः॑ सावि॒त्रः॥~(५६)

{\anuvakamend[{बला॑य पुरुषमृ॒गः सौ॒री पृ॑ष॒तः शका॒ष्टाद॑शा॒ष्टाद॑श}]}%॥14-18॥

%5.5.19.1
रुरू॑ रौ॒द्रः कृ॑कला॒सः श॒कुनिः॒ पिप्प॑का॒ ते श॑र॒व्या॑यै हरि॒णो मा॑रु॒तो ब्रह्म॑णे शा॒र्गस्त॒रक्षुः॑ कृ॒ष्णः श्वा च॑तुर॒क्षो ग॑र्द॒भस्त इ॑तरज॒नाना॑म॒ग्नये॒ धूङ्क्ष्णा᳚॥~(५७)

{\anuvakamend[{रुरु॑र्विꣳश॒तिः}]}%॥19॥

%5.5.20.1
अ॒ल॒ज आ᳚न्तरि॒क्ष उ॒द्रो म॒द्गुः प्ल॒वस्ते॑\-ऽपामदि॑त्यै हꣳस॒साचि॑रिन्द्रा॒ण्यै कीर्\mbox{}शा॒ गृध्रः॑ शितिक॒क्षी वा᳚र्ध्राण॒सस्ते दि॒व्या द्या॑वापृथि॒व्या᳚ श्वा॒वित्॥~(५८)

{\anuvakamend[{}]}

%5.5.21.1
सु॒प॒र्णः पा᳚र्ज॒न्यो ह॒ꣳ॒सो वृको॑ वृषद॒ꣳ॒शस्त ऐ॒न्द्रा अ॒पामु॒द्रो᳚\-ऽर्य॒म्णे लो॑पा॒शः सि॒ꣳ॒हो न॑कु॒लो व्या॒घ्रस्ते म॑हे॒न्द्राय॒ कामा॑य॒ पर॑स्वान्॥~(५९)

{\anuvakamend[{अ॒ल॒जः सु॑प॒र्णो᳚\-ऽष्टाद॑शाष्टा॒द॑श}]}%॥21॥

%5.5.22.1
आ॒ग्ने॒यः कृ॒ष्णग्री॑वः सारस्व॒ती मे॒षी ब॒भ्रुः सौ॒म्यः पौ॒ष्णः श्या॒मः शि॑तिपृ॒ष्ठो बा॑र्\mbox{}हस्प॒त्यः शि॒ल्पो वै᳚श्वदे॒व ऐ॒न्द्रो॑\-ऽरु॒णो मा॑रु॒तः क॒ल्माष॑ ऐन्द्रा॒ग्नः सꣳ॑हि॒तो॑\-ऽधोरा॑मः सावि॒त्रो वा॑रु॒णः पेत्वः॑॥~(६०)

{\anuvakamend[{आ॒ग्ने॒यो द्वाविꣳ॑शतिः}]}%॥22॥

%5.5.23.1
अश्व॑स्तूप॒रो गो॑मृ॒गस्ते प्रा॑जाप॒त्या आ᳚ग्ने॒यौ कृ॒ष्णग्री॑वौ त्वा॒ष्ट्रौ लो॑मशस॒क्थौ शि॑तिपृ॒ष्ठौ बा॑र्\mbox{}हस्प॒त्यौ धा॒त्रे पृ॑षोद॒रः सौ॒र्यो ब॒लक्षः॒ पेत्वः॑॥~(६१)

{\anuvakamend[{अश्वः॒ षोड॑श}]}%॥23॥

%5.5.24.1
अ॒ग्नये\-ऽनी॑कवते॒ रोहि॑ताञ्जिरन॒ड्वान॒धोरा॑मौ सावि॒त्रौ पौ॒ष्णौ र॑ज॒तना॑भी वैश्वदे॒वौ पि॒शङ्गौ॑ तूप॒रौ मा॑रु॒तः क॒ल्माष॑ आग्ने॒यः कृ॒ष्णो॑\-ऽजः सा॑रस्व॒ती मे॒षी वा॑रु॒णः कृ॒ष्ण एक॑शितिपा॒त्पेत्वः॑~(६२)

{\anuvakamend[{अ॒ग्नयो\-ऽनी॑कवते॒ द्वाविꣳ॑शतिः}]}%॥24॥

\prashnaend{यदेके॑न प्र॒जा\-प॑तिः प्रे॒णानु॒ यजु॒षापो॑ वि॒श्वक॒र्माग्न॒ आ या॑हि सुव॒र्गाय॒ वज्रो॑ गाय॒त्रेणाग्न॑ उदधे स॒मीचीन्द्रा॑य म॒युर॒पां बला॑य पुरुषमृ॒गः सौ॒री पृ॑ष॒तः शका॒ रुरु॑रल॒जः सु॑प॒र्ण आ᳚ग्ने॒यो\-ऽश्वो॒\-ऽग्नये\-ऽनी॑कवते॒ चतु॑र्विꣳशतिः॥२४॥}{यदेके॑न॒ स पापी॑याने॒तद्वा अ॒ग्नेर्धनु॒स्तद्दे॒वास्त्वेन्द्र॑ज्येष्ठा अ॒पां नप्रे\-ऽश्व॑स्तूप॒रो द्विष॑ष्टिः॥६२॥}{यदेके॒नैक॑शितिपा॒त्पेत्वः॑॥}%%५-५
{हरिः॑ ॐ}{॥कृष्ण-यजुर्वेदीय-तैत्तिरीय-संहितायां पञ्चम्काण्डे पञ्चमः प्रश्नः समाप्तः॥५-५॥}
%%% END PRASHNA

\sect{षष्ठमः प्रश्नः}\setcounter{anuvakam}{0}
\dnsub{तैत्तिरीयसंहितायां पञ्चमकाण्डे षष्ठमः प्रश्नः}
%5.6.1.1
हिर॑ण्यवर्णाः॒ शुच॑यः पाव॒का यासु॑ जा॒तः क॒श्यपो॒ यास्विन्द्रः॑। अ॒ग्निं या गर्भं॑ दधि॒रे विरू॑पा॒स्ता न॒ आपः॒ शꣴ स्यो॒ना भ॑वन्तु। यासा॒ꣳ॒ राजा॒ वरु॑णो॒ याति॒ मध्ये॑ सत्यानृ॒ते अ॑व॒पश्य॒ञ्जना॑नाम्। म॒धु॒श्चुतः॒ शुच॑यो॒ याः पा॑व॒कास्ता न॒ आपः॒ शꣴ स्यो॒ना भ॑वन्तु। यासां᳚ दे॒वा दि॒वि कृ॒ण्वन्ति॑ भ॒क्षं या अ॒न्तरि॑क्षे बहु॒धा भव॑न्ति। याः पृ॑थि॒वीं पय॑सो॒न्दन्ति॑~(१)

%5.6.1.2
शु॒क्रास्ता न॒ आपः॒ शꣴ स्यो॒ना भ॑वन्तु। शि॒वेन॑ मा॒ चक्षु॑षा पश्यतापः शि॒वया॑ त॒नुवोप॑ स्पृशत॒ त्वचं॑ मे। सर्वाꣳ॑ अ॒ग्नीꣳ र॑फ्सु॒षदो॑ हुवे वो॒ मयि॒ वर्चो॒ बल॒मोजो॒ नि ध॑त्त। यद॒दः स॑म्प्रय॒तीरहा॒वन॑दता ह॒ते। तस्मा॒दा न॒द्यो॑ नाम॑ स्थ॒ ता वो॒ नामा॑नि सिन्धवः। यत्प्रेषि॑ता॒ वरु॑णेन॒ ताः शीभꣳ॑ स॒मव॑ल्गत।~(२)

%5.6.1.3
तदा᳚प्नो॒दिन्द्रो॑ वो य॒तीस्तस्मा॒दापो॒ अनु॑ स्थन। अ॒प॒का॒मꣴ स्यन्द॑माना॒ अवी॑वरत वो॒ हिकम्᳚। इन्द्रो॑ वः॒ शक्ति॑भिर्देवी॒स्तस्मा॒द्वार्णाम॑ वो हि॒तम्। एको॑ दे॒वो अप्य॑तिष्ठ॒थ्स्यन्द॑माना यथाव॒शम्। उदा॑निषुर्म॒हीरिति॒ तस्मा॑दुद॒कमु॑च्यते। आपो॑ भ॒द्रा घृ॒तमिदाप॑ आसुर॒ग्नी\-षोमौ॑ बिभ्र॒त्याप॒ इत्ताः। ती॒व्रो रसो॑ मधु॒पृचा᳚म्~(३)

%5.6.1.4
अ॒रं॒ग॒म आ मा᳚ प्रा॒णेन॑ स॒ह वर्च॑सा गन्न्। आदित्प॑श्याम्यु॒त वा॑ शृणो॒म्या मा॒ घोषो॑ गच्छति॒ वाङ्न॑ आसाम्। मन्ये॑ भेजा॒नो अ॒मृत॑स्य॒ तर्\mbox{}हि॒ हिर॑ण्यवर्णा॒ अतृ॑पं य॒दा वः॑। आपो॒ हि ष्ठा म॑यो॒भुव॒स्ता न॑ ऊ॒र्जे द॑धातन। म॒हे रणा॑य॒ चक्ष॑से। यो वः॑ शि॒वत॑मो॒ रस॒स्तस्य॑ भाजयते॒ह नः॑। उ॒श॒तीरि॑व मा॒तरः॑। तस्मा॒ अरं॑ गमाम वो॒ यस्य॒ क्षया॑य॒ जिन्व॑थ। आपो॑ ज॒नय॑था च नः। दि॒वि श्र॑यस्वा॒न्तरि॑क्षे यतस्व पृथि॒व्या सम्भ॑व ब्रह्मवर्च॒सम॑सि ब्रह्मवर्च॒साय॑ त्वा॥~(४)

{\anuvakamend[{उ॒न्दन्ति॑ स॒मव॑ल्गत मधु॒पृचां᳚ मा॒तरो॒ द्वाविꣳ॑शतिश्च}]}%~(१)

%5.6.2.1
अ॒पां ग्रहा᳚न्गृह्णात्ये॒तद्वाव रा॑ज॒सूयं॒ यदे॒ते ग्रहाः᳚ स॒वो᳚\-ऽग्निर्व॑रुणस॒वो रा॑ज॒सूय॑मग्निस॒वश्चित्य॒स्ताभ्या॑मे॒व सू॑य॒ते\-ऽथो॑ उ॒भावे॒व लो॒काव॒भि ज॑यति॒ यश्च॑ राज॒सूये॑नेजा॒नस्य॒ यश्चा᳚ग्नि॒चित॒ आपो॑ भव॒न्त्यापो॒ वा अ॒ग्नेर्भ्रातृ॑व्या॒ यद॒पो᳚\-ऽग्नेर॒धस्ता॑दुप॒दधा॑ति॒ भ्रातृ॑व्याभिभूत्यै॒ भव॑त्या॒त्मना॒ परा᳚स्य॒ भ्रातृ॑व्यो भवत्य॒मृतम्᳚~(५)

%5.6.2.2
वा आप॒स्तस्मा॑द॒द्भिरव॑तान्तम॒भि षि॑ञ्चन्ति॒ नार्ति॒मार्च्छ॑ति॒ सर्व॒मायु॑रेति॒ यस्यै॒ता उ॑पधी॒यन्ते॒ य उ॑ चैना ए॒वं वेदान्नं॒ वा आपः॑ प॒शव॒ आपो\-ऽन्नं॑ प॒शवो᳚\-ऽन्ना॒दः प॑शु॒मान्भ॑वति॒ यस्यै॒ता उ॑पधी॒यन्ते॒ य उ॑ चैना ए॒वं वेद॒ द्वाद॑श भवन्ति॒ द्वाद॑श॒ मासाः᳚ संवथ्स॒रः सं॑वथ्स॒रेणै॒वास्मै᳚~(६)

%5.6.2.3
अन्न॒मव॑ रुन्धे॒ पात्रा॑णि भवन्ति॒ पात्रे॒ वा अन्न॑मद्यते॒ सयो᳚न्ये॒वान्न॒मव॑ रुन्ध॒ आ द्वा॑द॒शात्पुरु॑षा॒दन्न॑म॒त्त्यथो॒ पात्रा॒न्न छि॑द्यते॒ यस्यै॒ता उ॑पधी॒यन्ते॒ य उ॑ चैना ए॒वं वेद॑ कु॒म्भाश्च॑ कु॒म्भीश्च॑ मिथु॒नानि॑ भवन्ति मिथु॒नस्य॒ प्रजा᳚त्यै॒ प्र प्र॒जया॑ प॒शुभि॑र्मिथु॒नैर्जा॑यते॒ यस्यै॒ता उ॑पधी॒यन्ते॒ य उ॑~(७)

%5.6.2.4
चै॒ना॒ ए॒वं वेद॒ शुग्वा अ॒ग्निः सो᳚\-ऽध्व॒र्युं यज॑मानं प्र॒जाः शु॒चार्प॑यति॒ यद॒प उ॑प॒दधा॑ति॒ शुच॑मे॒वास्य॑ शमयति॒ नार्ति॒मार्च्छ॑त्यध्व॒र्युर्न यज॑मानः॒ शाम्य॑न्ति प्र॒जा यत्रै॒ता उ॑पधी॒यन्ते॒\-ऽपां वा ए॒तानि॒ हृद॑यानि॒ यदे॒ता आपो॒ यदे॒ता अ॒प उ॑प॒दधा॑ति दि॒व्याभि॑रे॒वैनाः॒ सꣳ सृ॑जति॒ वर्\mbox{}षु॑कः प॒र्जन्यः॑~(८)

%5.6.2.5
भ॒व॒ति॒ यो वा ए॒तासा॑मा॒यत॑नं॒ कॢप्तिं॒ वेदा॒यत॑नवान्भवति॒ कल्प॑ते\-ऽस्मा अनुसी॒तमुप॑ दधात्ये॒तद्वा आ॑सामा॒यत॑नमे॒षा कॢप्ति॒र्य ए॒वं वेदा॒यत॑नवान्भवति॒ कल्प॑ते\-ऽस्मै द्व॒न्द्वम॒न्या उप॑ दधाति॒ चत॑स्रो॒ मध्ये॒ धृत्या॒ अन्नं॒ वा इष्ट॑का ए॒तत्खलु॒ वै सा॒क्षादन्नं॒ यदे॒ष च॒रुर्यदे॒तं च॒रुमु॑प॒दधा॑ति सा॒क्षात्~(९)

%5.6.2.6
ए॒वास्मा॒ अन्न॒मव॑ रुन्धे मध्य॒त उप॑ दधाति मध्य॒त ए॒वास्मा॒ अन्नं॑ दधाति॒ तस्मा᳚न्मध्य॒तो\-ऽन्न॑मद्यते बार्\mbox{}हस्प॒त्यो भ॑वति॒ ब्रह्म॒ वै दे॒वानां॒ बृह॒स्पति॒र्ब्रह्म॑णै॒वास्मा॒ अन्न॒मव॑ रुन्धे ब्रह्मवर्च॒सम॑सि ब्रह्मवर्च॒साय॒ त्वेत्या॑ह तेज॒स्वी ब्र॑ह्मवर्च॒सी भ॑वति॒ यस्यै॒ष उ॑पधी॒यते॒ य उ॑ चैनमे॒वं वेद॑॥~(१०)

{\anuvakamend[{अ॒मृत॑मस्मै जायते॒ यस्यै॒ता उ॑पधी॒यन्ते॒ य उ॑ प॒र्जन्य॑ उप॒दधा॑ति सा॒क्षाथ्स॒प्तच॑त्वारिꣳशच्च}]}%~(२)

%5.6.3.1
भू॒ते॒ष्ट॒का उप॑ दधा॒त्यत्रा᳚त्र॒ वै मृ॒त्युर्जा॑यते॒ यत्र॑यत्रै॒व मृ॒त्युर्जाय॑ते॒ तत॑ ए॒वैन॒मव॑ यजते॒ तस्मा॑दग्नि॒चिथ्सर्व॒मायु॑रेति॒ सर्वे॒ ह्य॑स्य मृ॒त्यवो\-ऽवे᳚ष्टा॒स्तस्मा॑दग्नि॒चिन्नाभिच॑रित॒वै प्र॒त्यगे॑नमभिचा॒रः स्तृ॑णुते सू॒यते॒ वा ए॒ष यो᳚\-ऽग्निं चि॑नु॒ते दे॑वसु॒वामे॒तानि॑ ह॒वीꣳषि॑ भवन्त्ये॒ताव॑न्तो॒ वै दे॒वानाꣳ॑ स॒वास्त ए॒व~(११)

%5.6.3.2
अ॒स्मै॒ स॒वान्प्र य॑च्छन्ति॒ त ए॑नꣳ सुवन्ते स॒वो᳚\-ऽग्निर्व॑रुणस॒वो रा॑ज॒सूयं॑ ब्रह्मस॒वश्चित्यो॑ दे॒वस्य॑ त्वा सवि॒तुः प्र॑स॒व इत्या॑ह सवि॒तृप्र॑सूत ए॒वैनं॒ ब्रह्म॑णा दे॒वता॑भिर॒भि षि॑ञ्च॒त्यन्न॑स्यान्नस्या॒भि षि॑ञ्च॒त्यन्न॑स्यान्न॒स्याव॑रुद्ध्यै पु॒रस्ता᳚त्प्र॒त्यञ्च॑म॒भि षि॑ञ्चति पु॒रस्ता॒द्धि प्र॑ती॒चीन॒मन्न॑म॒द्यते॑ शीर्\mbox{}ष॒तो॑\-ऽभि षि॑ञ्चति शीर्\mbox{}ष॒तो ह्यन्न॑म॒द्यत॒ आ मुखा॑द॒न्वव॑स्रावयति~(१२)

%5.6.3.3
मु॒ख॒त ए॒वास्मा॑ अ॒न्नाद्यं॑ दधात्य॒ग्नेस्त्वा॒ साम्रा᳚ज्येना॒भि षि॑ञ्चा॒मीत्या॑है॒ष वा अ॒ग्नेः स॒वस्तेनै॒वैन॑म॒भि षि॑ञ्चति॒ बृह॒स्पते᳚स्त्वा॒ साम्रा᳚ज्येना॒भि षि॑ञ्चा॒मीत्या॑ह॒ ब्रह्म॒ वै दे॒वानां॒ बृह॒स्पति॒र्ब्रह्म॑णै॒वैन॑म॒भि षि॑ञ्च॒तीन्द्र॑स्य त्वा॒ साम्रा᳚ज्येना॒भि षि॑ञ्चा॒मीत्या॑हेन्द्रि॒यमे॒वास्मि॑न्नु॒परि॑ष्टाद्दधात्ये॒तत्~(१३)

%5.6.3.4
वै रा॑ज॒सूय॑स्य रू॒पं य ए॒वं वि॒द्वान॒ग्निं चि॑नु॒त उ॒भावे॒व लो॒काव॒भि ज॑यति॒ यश्च॑ राज॒सूये॑नेजा॒नस्य॒ यश्चा᳚ग्नि॒चित॒ इन्द्र॑स्य सुषुवा॒णस्य॑ दश॒धेन्द्रि॒यं वी॒र्यं॑ परा॑पत॒त्तद्दे॒वाः सौ᳚त्राम॒ण्या सम॑भरन्थ्सू॒यते॒ वा ए॒ष यो᳚\-ऽग्निं चि॑नु॒ते᳚\-ऽग्निं चि॒त्वा सौ᳚त्राम॒ण्या य॑जेतेन्द्रि॒यमे॒व वी॒र्यꣳ॑ स॒म्भृत्या॒त्मन्ध॑त्ते॥~(१४)

{\anuvakamend[{त ए॒वान्वव॑स्रावयत्ये॒तद॒ष्टाच॑त्वारिꣳशच्च}]}%~(३)

%5.6.4.1
स॒जूरब्दो\-ऽया॑वभिः स॒जूरु॒षा अरु॑णीभिः स॒जूः सूर्य॒ एत॑शेन स॒जोषा॑व॒श्विना॒ दꣳसो॑भिः स॒जूर॒ग्निर्वै᳚श्वान॒र इडा॑भिर्घृ॒तेन॒ स्वाहा॑ संवथ्स॒रो वा अब्दो॒ मासा॒ अया॑वा उ॒षा अरु॑णी॒ सूर्य॒ एत॑श इ॒मे अ॒श्विना॑ संवथ्स॒रो᳚\-ऽग्निर्वै᳚श्वान॒रः प॒शव॒ इडा॑ प॒शवो॑ घृ॒तꣳ सं॑वथ्स॒रं प॒शवो\-ऽनु॒ प्र जा॑यन्ते संवथ्स॒रेणै॒वास्मै॑ प॒शून्प्र ज॑नयति दर्भस्त॒म्बे जु॑होति॒ यत्~(१५)

%5.6.4.2
वा अ॒स्या अ॒मृतं॒ यद्वी॒र्यं॑ तद्द॒र्भास्तस्मि॑ञ्जुहोति॒ प्रैव जा॑यते\-ऽन्ना॒दो भ॑वति॒ यस्यै॒वं जुह्व॑त्ये॒ता वै दे॒वता॑ अ॒ग्नेः पु॒रस्ता᳚द्भागा॒स्ता ए॒व प्री॑णा॒त्यथो॒ चक्षु॑रे॒वाग्नेः पु॒रस्ता॒त्प्रति॑ दधा॒त्यन॑न्धो भवति॒ य ए॒वं वेदापो॒ वा इ॒दमग्रे॑ सलि॒लमा॑सी॒थ्स प्र॒जा\-प॑तिः पुष्करप॒र्णे वातो॑ भू॒तो॑\-ऽलेलाय॒थ्सः~(१६)

%5.6.4.3
प्र॒ति॒ष्ठां नावि॑न्द॒त स ए॒तद॒पां कु॒लाय॑मपश्य॒त्तस्मि॑न्न॒ग्निम॑चिनुत॒ तदि॒यम॑भव॒त्ततो॒ वै स प्रत्य॑तिष्ठ॒द्यां पु॒रस्ता॑दु॒पा\-द॑धा॒त्तच्छिरो॑\-ऽभव॒थ्सा प्राची॒ दिग्यां द॑क्षिण॒त उ॒पाद॑धा॒थ्स दक्षि॑णः प॒क्षो॑\-ऽभव॒थ्सा द॑क्षि॒णा दिग्यां प॒श्चादु॒पा\-द॑धा॒त्तत्पुच्छ॑मभव॒थ्सा प्र॒तीची॒ दिग्यामु॑त्तर॒त उ॒पाद॑धात्~(१७)

%5.6.4.4
स उत्त॑रः प॒क्षो॑\-ऽभव॒थ्सोदी॑ची॒ दिग्यामु॒परि॑ष्टादु॒पाद॑धा॒त्तत्पृ॒ष्ठम॑भव॒थ्सोर्ध्वा दिगि॒यं वा अ॒ग्निः पञ्चे᳚ष्टक॒स्तस्मा॒द्यद॒स्यां खन॑न्त्य॒भीष्ट॑कां तृ॒न्दन्त्य॒भि शर्क॑रा॒ꣳ॒ सर्वा॒ वा इ॒यं वयो᳚भ्यो॒ नक्तं॑ दृ॒शे दी᳚प्यते॒ तस्मा॑दि॒मां वयाꣳ॑सि॒ नक्तं॒ नाध्या॑सते॒ य ए॒वं वि॒द्वान॒ग्निं चि॑नु॒ते प्रत्ये॒व~(१८)

%5.6.4.5
ति॒ष्ठ॒त्य॒भि दिशो॑ जयत्याग्ने॒यो वै ब्रा᳚ह्म॒णस्तस्मा᳚द्ब्राह्म॒णाय॒ सर्वा॑सु दि॒क्ष्वर्धु॑क॒ꣴ॒ स्वामे॒व तद्दिश॒मन्वे᳚त्य॒पां वा अ॒ग्निः कु॒लाय॒न्तस्मा॒दापो॒\-ऽग्निꣳ हारु॑काः॒ स्वामे॒व तद्योनिं॒ प्र वि॑शन्ति॥~(१९)

{\anuvakamend[{यद॑लेलाय॒थ्स उ॑त्तर॒त उ॒पाद॑धादे॒व द्वात्रिꣳ॑शच्च}]}%~(४)

%5.6.5.1
सं॒व॒थ्स॒रमुख्य॑म्भृ॒त्वा द्वि॒तीये॑ संवथ्स॒र आ᳚ग्ने॒यम॒ष्टा\-क॑पालं॒ निर्व॑पेदै॒न्द्रमेका॑\-दश\-कपालं वैश्वदे॒वं द्वाद॑श\-कपालं बार्\mbox{}हस्प॒त्यं च॒रुं वै᳚ष्ण॒वं त्रि॑कपा॒लं तृ॒तीये॑ संवथ्स॒रे॑\-ऽभि॒जिता॑ यजेत॒ यद॒ष्टाक॑पालो॒ भव॑त्य॒ष्टाक्ष॑रा गाय॒त्र्या᳚ग्ने॒यं गा॑य॒त्रं प्रा॑तःसव॒नं प्रा॑तःसव॒नमे॒व तेन॑ दाधार गाय॒त्रं छन्दो॒ यदेका॑\-दश\-कपालो॒ भव॒त्येका॑\-दशाक्षरा त्रि॒ष्टुगै॒न्द्रं त्रैष्टु॑भं॒ माध्य॑न्दिन॒ꣳ॒ सव॑नं॒ माध्य॑न्दिनमे॒व सव॑नं॒ तेन॑ दाधार त्रि॒ष्टुभम्᳚~(२०)

%5.6.5.2
छन्दो॒ यद्द्वाद॑श\-कपालो॒ भव॑ति॒ द्वाद॑शाक्षरा॒ जग॑ती वैश्वदे॒वं जाग॑तं तृतीयसव॒नन्तृ॑तीयसव॒नमे॒व तेन॑ दाधार॒ जग॑तीं॒ छन्दो॒ यद्बा॑र्\mbox{}हस्प॒त्यश्च॒रुर्भव॑ति॒ ब्रह्म॒ वै दे॒वानां॒ बृह॒स्पति॒र्ब्रह्मै॒व तेन॑ दाधार॒ यद्वै᳚ष्ण॒वस्त्रि॑कपा॒लो भव॑ति य॒ज्ञो वै विष्णु॑र्य॒ज्ञमे॒व तेन॑ दाधार॒ यत्तृ॒तीये॑ संवथ्स॒रे॑\-ऽभि॒जिता॒ यज॑ते॒\-ऽभिजि॑त्यै॒ यथ्सं॑वथ्स॒रमुख्यं॑ बि॒भर्ती॒ममे॒व~(२१)

%5.6.5.3
तेन॑ लो॒कꣴ स्पृ॑णोति॒ यद्द्वि॒तीये॑ संवथ्स॒रे᳚\-ऽग्निं चि॑नु॒ते᳚\-ऽन्तरि॑क्षमे॒व तेन॑ स्पृणोति॒ यत्तृ॒तीये॑ संवथ्स॒रे यज॑ते॒\-ऽमुमे॒व तेन॑ लो॒कꣴ स्पृ॑णोत्ये॒तं वै पर॑ आट्णा॒रः क॒क्षीवाꣳ॑ औशि॒जो वी॒तह॑व्यः श्राय॒सस्त्र॒सद॑स्युः पौरुकु॒थ्स्यः प्र॒जाका॑मा अचिन्वत॒ ततो॒ वै ते स॒हस्रꣳ॑सहस्रं पु॒त्रान॑विन्दन्त॒ प्रथ॑ते प्र॒जया॑ प॒शुभि॒स्तां मात्रा॑माप्नोति॒ यां ते\-ऽग॑च्छ॒न्॒ य ए॒वं वि॒द्वाने॒तम॒ग्निं चि॑नु॒ते॥~(२२)

{\anuvakamend[{दा॒धा॒र॒ त्रि॒ष्टुभ॑मि॒ममे॒वैवं च॒त्वारि॑ च}]}%~(५)

%5.6.6.1
प्र॒जा\-प॑तिर॒ग्निम॑चिनुत॒ स क्षु॒रप॑विर्भू॒त्वाति॑ष्ठ॒त्तं दे॒वा बिभ्य॑तो॒ नोपा॑य॒न्ते छन्दो॑भिरा॒त्मानं॑ छादयि॒त्वोपा॑य॒न्तच्छन्द॑सां छन्द॒स्त्वं ब्रह्म॒ वै छन्दाꣳ॑सि॒ ब्रह्म॑ण ए॒तद्रू॒पं यत्कृ॑ष्णाजि॒नङ्कार्ष्णी॑ उपा॒नहा॒वुप॑ मुञ्चते॒ छन्दो॑भिरे॒वात्मानं॑ छादयि॒त्वाग्निमुप॑ चरत्या॒त्मनो\-ऽहिꣳ॑सायै देवनि॒धिर्वा ए॒ष नि धी॑यते॒ यद॒ग्निः~(२३)

%5.6.6.2
अ॒न्ये वा॒ वै नि॒धिमगु॑प्तं वि॒न्दन्ति॒ न वा॒ प्रति॒ प्र जा॑नात्यु॒खामा क्रा॑मत्या॒त्मान॑मे॒वाधि॒पां कु॑रुते॒ गुप्त्या॒ अथो॒ खल्वा॑हु॒र्नाक्रम्येति॑ नैर्\mbox{}ऋ॒त्यु॑खा यदा॒क्रामे॒न्निर्\mbox{}ऋ॑त्या आ॒त्मान॒मपि॑ दध्या॒त्तस्मा॒न्नाक्रम्या॑ पुरुषशी॒र्॒\mbox{}षमुप॑ दधाति॒ गुप्त्या॒ अथो॒ यथा᳚ ब्रू॒यादे॒तन्मे॑ गोपा॒येति॑ ता॒दृगे॒व तत्~(२४)

%5.6.6.3
प्र॒जा\-प॑ति॒र्वा अथ॑र्वा॒ग्निरे॒व द॒ध्यङ्ङा॑थर्व॒णस्तस्येष्ट॑का अ॒स्थान्ये॒तꣳ ह॒ वाव तदृषि॑र॒भ्यनू॑वा॒चेन्द्रो॑ दधी॒चो अ॒स्थभि॒रिति॒ यदिष्ट॑काभिर॒ग्निं चि॒नोति॒ सात्मा॑नमे॒वाग्निं चि॑नुते॒ सात्मा॒मुष्मिँ॑ल्लो॒के भ॑वति॒ य ए॒वं वेद॒ शरी॑रं॒ वा ए॒तद॒ग्नेर्यच्चित्य॑ आ॒त्मा वै᳚श्वान॒रो यच्चि॒ते वै᳚श्वान॒रं जु॒होति॒ शरी॑रमे॒व स॒ꣴ॒स्कृत्य॑~(२५)

%5.6.6.4
अ॒भ्यारो॑हति॒ शरी॑रं॒ वा ए॒तद्यज॑मानः॒ सꣴस्कु॑रुते॒ यद॒ग्निं चि॑नु॒ते यच्चि॒ते वै᳚श्वान॒रं जु॒होति॒ शरी॑रमे॒व स॒ꣴ॒स्कृत्या॒त्मना॒भ्यारो॑हति॒ तस्मा॒त्तस्य॒ नाव॑ द्यन्ति॒ जीव॑न्ने॒व दे॒वानप्ये॑ति वैश्वान॒र्यर्चा पुरी॑ष॒मुप॑ दधाती॒यं वा अ॒ग्निर्वै᳚श्वान॒रस्तस्यै॒षा चिति॒र्यत्पुरी॑षम॒ग्निमे॒व वै᳚श्वान॒रं चि॑नुत ए॒षा वा अ॒ग्नेः प्रि॒या त॒नूर्यद्वै᳚श्वान॒रः प्रि॒यामे॒वास्य॑ त॒नुव॒मव॑ रुन्धे॥~(२६)

{\anuvakamend[{अ॒ग्निस्तथ्स॒ꣴ॒स्कृत्या॒ग्नेर्दश॑ च}]}%~(६)

%5.6.7.1
अ॒ग्नेर्वै दी॒क्षया॑ दे॒वा वि॒राज॑माप्नुवन्ति॒स्रो रात्री᳚र्दीक्षि॒तः स्या᳚त्त्रि॒पदा॑ वि॒राड्वि॒राज॑माप्नोति॒ षड्रात्री᳚र्दीक्षि॒तः स्या॒त् षड्वा ऋ॒तवः॑ संवथ्स॒रः सं॑वथ्स॒रो वि॒राड्वि॒राज॑माप्नोति॒ दश॒ रात्री᳚र्दीक्षि॒तः स्या॒द्दशा᳚क्षरा वि॒राड्वि॒राज॑माप्नोति॒ द्वाद॑श॒ रात्री᳚र्दीक्षि॒तः स्या॒द्द्वाद॑श॒ मासाः᳚ संवथ्स॒रः सं॑वथ्स॒रो वि॒राड्वि॒राज॑माप्नोति॒ त्रयो॑दश॒ रात्री᳚र्दीक्षि॒तः स्या॒त्त्रयो॑दश~(२७)

%5.6.7.2
मासाः᳚ संवथ्स॒रः सं॑वथ्स॒रो वि॒राड्वि॒राज॑माप्नोति॒ पञ्च॑दश॒ रात्री᳚र्दीक्षि॒तः स्या॒त्पञ्च॑दश॒ वा अ॑र्धमा॒सस्य॒ रात्र॑यो\-ऽर्धमास॒शः सं॑वथ्स॒र आ᳚प्यते संवथ्स॒रो वि॒राड्वि॒राज॑माप्नोति स॒प्तद॑श॒ रात्री᳚र्दीक्षि॒तः स्या॒द्द्वाद॑श॒ मासाः॒ पञ्च॒र्तवः॒ स सं॑वथ्स॒रः सं॑वथ्स॒रो वि॒राड्वि॒राज॑माप्नोति॒ चतु॑र्विꣳशति॒ꣳ॒ रात्री᳚र्दीक्षि॒तः स्या॒च्चतु॑र्विꣳशतिरर्धमा॒साः सं॑वथ्स॒रः सं॑वथ्स॒रो वि॒राड्वि॒राज॑माप्नोति त्रि॒ꣳ॒शत॒ꣳ॒ रात्री᳚र्दीक्षि॒तः स्या᳚त्~(२८)

%5.6.7.3
त्रि॒ꣳ॒शद॑क्षरा वि॒राड्वि॒राज॑माप्नोति॒ मासं॑ दीक्षि॒तः स्या॒द्यो मासः॒ स सं॑वथ्स॒रः सं॑वथ्स॒रो वि॒राड्वि॒राज॑माप्नोति च॒तुरो॑ मा॒सो दी᳚क्षि॒तः स्या᳚च्च॒तुरो॒ वा ए॒तं मा॒सो वस॑वो\-ऽबिभरु॒स्ते पृ॑थि॒वीमाज॑यन्गाय॒त्रीं छन्दो॒\-ऽष्टौ रु॒द्रास्ते᳚\-ऽन्तरि॑क्ष॒माज॑यन्त्रि॒ष्टुभं॒ छन्दो॒ द्वाद॑शादि॒त्यास्ते दिव॒माज॑य॒ञ्जग॑तीं॒ छन्द॒स्ततो॒ वै ते व्या॒वृत॑मगच्छ॒ञ्छ्रैष्ठ्यं॑ दे॒वानां॒ तस्मा॒द्द्वाद॑श मा॒सो भृ॒त्वाग्निं चि॑न्वीत॒ द्वाद॑श॒ मासाः᳚ संवथ्स॒रः सं॑वथ्स॒रो᳚\-ऽग्निश्चित्य॒स्तस्या॑होरा॒त्राणीष्ट॑का आ॒प्तेष्ट॑कमेनं चिनु॒ते\-ऽथो᳚ व्या॒वृत॑मे॒व ग॑च्छति॒ श्रैष्ठ्यꣳ॑ समा॒नाना᳚म्॥~(२९)

{\anuvakamend[{स्या॒त्त्रयो॑दश त्रि॒ꣳ॒शत॒ꣳ॒ रात्री᳚र्दीक्षि॒तः स्या॒द्वै ते᳚\-ऽष्टाविꣳ॑शतिश्च}]}%~(७)

%5.6.8.1
सु॒व॒र्गाय॒ वा ए॒ष लो॒काय॑ चीयते॒ यद॒ग्निस्तं यन्नान्वा॒रोहे᳚थ्सुव॒र्गाल्लो॒काद्यज॑मानो हीयेत पृथि॒वीमाक्र॑मिषं प्रा॒णो मा॒ मा हा॑सीद॒न्तरि॑क्ष॒माक्र॑मिषं प्र॒जा मा॒ मा हा॑सी॒द्दिव॒माक्र॑मिष॒ꣳ॒ सुव॑रग॒न्मेत्या॑है॒ष वा अ॒ग्नेर॑न्वारो॒हस्तेनै॒वैन॑\-म॒न्वारो॑हति सुव॒र्गस्य॑ लो॒कस्य॒ सम॑ष्ट्यै॒ यत्प॒क्षस॑म्मितां मिनु॒यात्~(३०)

%5.6.8.2
कनी॑याꣳसं यज्ञक्र॒तुमुपे॑या॒त्पापी॑यस्यस्या॒त्मनः॑ प्र॒जा स्या॒द्वेदि॑सम्मितां मिनोति॒ ज्यायाꣳ॑समे॒व य॑ज्ञक्र॒तुमुपै॑ति॒ नास्या॒त्मनः॒ पापी॑यसी प्र॒जा भ॑वति साह॒स्रं चि॑न्वीत प्रथ॒मं चि॑न्वा॒नः स॒हस्र॑सम्मितो॒ वा अ॒यं लो॒क इ॒ममे॒व लो॒कम॒भि ज॑यति॒ द्विषा॑हस्रं चिन्वीत द्वि॒तीयं॑ चिन्वा॒नो द्विषा॑हस्रं॒ वा अ॒न्तरि॑क्षम॒न्तरि॑क्षमे॒वाभि ज॑यति॒ त्रिषा॑हस्रं चिन्वीत तृ॒तीयं॑ चिन्वा॒नः~(३१)

%5.6.8.3
त्रिषा॑हस्रो॒ वा अ॒सौ लो॒को॑\-ऽमुमे॒व लो॒कम॒भि ज॑यति जानुद॒घ्नं चि॑न्वीत प्रथ॒मं चि॑न्वा॒नो गा॑यत्रि॒यैवेमं लो॒कम॒भ्यारो॑हति नाभिद॒घ्नं चि॑न्वीत द्वि॒तीयं॑ चिन्वा॒नस्त्रि॒ष्टुभै॒वान्तरि॑क्षम॒भ्यारो॑हति ग्रीवद॒घ्नं चि॑न्वीत तृ॒तीयं॑ चिन्वा॒नो जग॑त्यै॒वामुं लो॒कम॒भ्यारो॑हति॒ नाग्निं चि॒त्वा रा॒मामुपे॑यादयो॒नौ रेतो॑ धास्या॒मीति॒ न द्वि॒तीयं॑ चि॒त्वान्यस्य॒ स्त्रियम्᳚~(३२)

%5.6.8.4
उपे॑या॒न्न तृ॒तीयं॑ चि॒त्वा कां च॒नोपे॑या॒द्रेतो॒ वा ए॒तन्नि ध॑त्ते॒ यद॒ग्निं चि॑नु॒ते यदु॑पे॒याद्रेत॑सा॒ व्यृ॑ध्ये॒ताथो॒ खल्वा॑हुरप्रज॒स्यं तद्यन्नोपे॒यादिति॒ यद्रे॑तः॒सिचा॑वुप॒दधा॑ति॒ ते ए॒व यज॑मानस्य॒ रेतो॑ बिभृत॒स्तस्मा॒दुपे॑या॒द्रेत॒सो\-ऽस्क॑न्दाय॒ त्रीणि॒ वाव रेताꣳ॑सि पि॒ता पु॒त्रः पौत्रः॑~(३३)

%5.6.8.5
यद्द्वे रे॑तः॒सिचा॑वुपद॒ध्याद्रेतो᳚\-ऽस्य॒ विच्छि॑न्द्यात्ति॒स्र उप॑ दधाति॒ रेत॑सः॒ सन्त॑त्या इ॒यं वाव प्र॑थ॒मा रे॑तः॒सिग्वाग्वा इ॒यं तस्मा॒त्पश्य॑न्ती॒मां पश्य॑न्ति॒ वाचं॒ वद॑न्तीम॒न्तरि॑क्षं द्वि॒तीया᳚ प्रा॒णो वा अ॒न्तरि॑क्षं॒ तस्मा॒न्नान्तरि॑क्षं॒ पश्य॑न्ति॒ न प्रा॒णम॒सौ तृ॒तीया॒ चक्षु॒र्वा अ॒सौ तस्मा॒त्पश्य॑न्त्य॒मूं पश्य॑न्ति॒ चक्षु॒र्यजु॑षे॒मां च॑~(३४)

%5.6.8.6
अ॒मूं चोप॑ दधाति॒ मन॑सा मध्य॒मामे॒षां लो॒कानां॒ कॢप्त्या॒ अथो᳚ प्रा॒णाना॑मि॒ष्टो य॒ज्ञो भृगु॑भिराशी॒र्दा वसु॑भि॒स्तस्य॑ त इ॒ष्टस्य॑ वी॒तस्य॒ द्रवि॑णे॒ह भ॑क्षी॒येत्या॑ह स्तुतश॒स्त्रे ए॒वैतेन॑ दुहे पि॒ता मा॑त॒रिश्वाच्छि॑द्रा प॒दा धा॒ अच्छि॑द्रा उ॒शिजः॑ प॒दानु॑ तक्षुः॒ सोमो॑ विश्व॒विन्ने॒ता ने॑ष॒द्बृह॒स्पति॑रुक्थाम॒दानि॑ शꣳसिष॒दित्या॑है॒तद्वा अ॒ग्नेरु॒क्थन्तेनै॒वैन॒मनु॑ शꣳसति॥~(३५)

{\anuvakamend[{मि॒नु॒यात्तृ॒तीयं॑ चिन्वा॒नस्त्रियं॒ पौत्र॑श्च॒ वै स॒प्तद॑श च}]}%~(८)

%5.6.9.1
सू॒यते॒ वा ए॒षो᳚\-ऽग्नी॒नां य उ॒खायां᳚ भ्रि॒यते॒ यद॒धः सा॒दये॒द्गर्भाः᳚ प्र॒पादु॑काः स्यु॒रथो॒ यथा॑ स॒वात्प्र॑त्यव॒रोह॑ति ता॒दृगे॒व तदा॑स॒न्दी सा॑दयति॒ गर्भा॑णां॒ धृत्या॒ अप्र॑पादा॒याथो॑ स॒वमे॒वैनं॑ करोति॒ गर्भो॒ वा ए॒ष यदुख्यो॒ योनिः॑ शि॒क्यं॑ यच्छि॒क्या॑दु॒खां नि॒रूहे॒द्योने॒र्गर्भं॒ निर्\mbox{}ह॑ण्या॒थ्षडु॑द्यामꣳ शि॒क्यं॑ भवति षोढाविहि॒तो वै~(३६)

%5.6.9.2
पुरु॑ष आ॒त्मा च॒ शिर॑श्च च॒त्वार्यङ्गा᳚न्या॒त्मन्ने॒वैनं॑ बिभर्ति प्र॒जा\-प॑ति॒र्वा ए॒ष यद॒ग्निस्तस्यो॒खा चो॒लूख॑लं च॒ स्तनौ॒ ताव॑स्य प्र॒जा उप॑ जीवन्ति॒ यदु॒खां चो॒लूख॑लं चोप॒दधा॑ति॒ ताभ्या॑मे॒व यज॑मानो॒\-ऽमुष्मिँ॑ल्लो॒के᳚\-ऽग्निं दु॑हे संवथ्स॒रो वा ए॒ष यद॒ग्निस्तस्य॑ त्रेधाविहि॒ता इ॑ष्टकाः प्राजाप॒त्या वै᳚ष्ण॒वीः~(३७)

%5.6.9.3
वै॒श्व॒क॒र्म॒णीर॑होरा॒त्राण्ये॒वास्य॑ प्राजाप॒त्या यदुख्यं॑ बि॒भर्ति॑ प्राजाप॒त्या ए॒व तदुप॑ धत्ते॒ यथ्स॒मिध॑ आ॒दधा॑ति वैष्ण॒वा वै वन॒स्पत॑यो वैष्ण॒वीरे॒व तदुप॑ धत्ते॒ यदिष्ट॑काभिर॒ग्निं चि॒नोती॒यं वै वि॒श्वक॑र्मा वैश्वकर्म॒णीरे॒व तदुप॑ धत्ते तस्मा॑दाहुस्त्रि॒वृद॒ग्निरिति॒ तं वा ए॒तं यज॑मान ए॒व चि॑न्वीत॒ यद॑स्या॒न्यश्चि॑नु॒याद्यत्तं दक्षि॑णाभि॒र्न रा॒धये॑द॒ग्निम॑स्य वृञ्जीत॒ यो᳚\-ऽस्या॒ग्निं चि॑नु॒यात्तं दक्षि॑णाभी राधयेद॒ग्निमे॒व तथ्स्पृ॑णोति॥~(३८)

{\anuvakamend[{षो॒ढा॒वि॒हि॒तो वै वै᳚ष्ण॒वीर॒न्यो विꣳ॑श॒तिश्च॑}]}%~(९)

%5.6.10.1
प्र॒जा\-प॑तिर॒ग्निम॑चिनुत॒र्तुभिः॑ संवथ्स॒रं व॑स॒न्तेनै॒वास्य॑ पूर्वा॒र्धम॑चिनुत ग्री॒ष्मेण॒ दक्षि॑णं प॒क्षं व॒र्॒\mbox{}षाभिः॒ पुच्छꣳ॑ श॒रदोत्त॑रं प॒क्षꣳ हे॑म॒न्तेन॒ मध्यं॒ ब्रह्म॑णा॒ वा अ॑स्य॒ तत्पू᳚र्वा॒र्धम॑चिनुत क्ष॒त्रेण॒ दक्षि॑णं प॒क्षं प॒शुभिः॒ पुच्छं॑ वि॒शोत्त॑रं प॒क्षमा॒शया॒ मध्यं॒ य ए॒वं वि॒द्वान॒ग्निं चि॑नु॒त ऋ॒तुभि॑रे॒वैनं॑ चिनु॒ते\-ऽथो॑ ए॒तदे॒व सर्व॒मव॑~(३९)

%5.6.10.2
रु॒न्धे॒ शृ॒ण्वन्त्ये॑नम॒ग्निं चि॑क्या॒नमत्त्यन्न॒ꣳ॒ रोच॑त इ॒यं वाव प्र॑थ॒मा चिति॒रोष॑धयो॒ वन॒स्पत॑यः॒ पुरी॑षम॒न्तरि॑क्षं द्वि॒तीया॒ वयाꣳ॑सि॒ पुरी॑षम॒सौ तृ॒तीया॒ नक्ष॑त्राणि॒ पुरी॑षं य॒ज्ञश्च॑तु॒र्थी दक्षि॑णा॒ पुरी॑षं॒ यज॑मानः पञ्च॒मी प्र॒जा पुरी॑षं॒ यत्त्रिचि॑तीकं चिन्वी॒त य॒ज्ञं दक्षि॑णामा॒त्मानं॑ प्र॒जाम॒न्तरि॑या॒त्तस्मा॒त्पञ्च॑चितीकश्चेत॒व्य॑ ए॒तदे॒व सर्वꣴ॑ स्पृणोति॒ यत्ति॒स्रश्चित॑यः~(४०)

%5.6.10.3
त्रि॒वृद्ध्य॑ग्निर्यद्द्वे द्वि॒पाद्यज॑मानः॒ प्रति॑ष्ठित्यै॒ पञ्च॒ चित॑यो भवन्ति॒ पाङ्क्तः॒ पुरु॑ष आ॒त्मान॑मे॒व स्पृ॑णोति॒ पञ्च॒ चित॑यो भवन्ति प॒ञ्चभिः॒ पुरी॑षैर॒भ्यू॑हति॒ दश॒ सं प॑द्यन्ते॒ दशा᳚क्षरो॒ वै पुरु॑षो॒ यावा॑ने॒व पुरु॑ष॒स्तꣴ स्पृ॑णो॒त्यथो॒ दशा᳚क्षरा वि॒राडन्नं॑ वि॒राड्वि॒राज्ये॒वान्नाद्ये॒ प्रति॑ तिष्ठति संवथ्स॒रो वै ष॒ष्ठी चिति॑र्\mbox{}ऋ॒तवः॒ पुरी॑ष॒ꣳ॒ षट्चित॑यो भवन्ति॒ षट्पुरी॑षाणि॒ द्वाद॑श॒ सं प॑द्यन्ते॒ द्वाद॑श॒ मासाः᳚ संवथ्स॒रः सं॑वथ्स॒र ए॒व प्रति॑ तिष्ठति॥~(४१)

{\anuvakamend[{अव॒ चित॑यः॒ पुरी॑षं॒ पञ्च॑दश च}]}%॥10॥

%5.6.11.1
रोहि॑तो धू॒म्ररो॑हितः क॒र्कन्धु॑रोहित॒स्ते प्रा॑जाप॒त्या ब॒भ्रुर॑रु॒णब॑भ्रुः॒ शुक॑बभ्रु॒स्ते रौ॒द्राः श्येतः॑ श्येता॒क्षः श्येत॑ग्रीव॒स्ते पि॑तृदेव॒त्या᳚स्ति॒स्रः कृ॒ष्णा व॒शा वा॑रु॒ण्य॑स्ति॒स्रः श्वे॒ता व॒शाः सौ॒र्यो॑ मैत्राबार्\mbox{}हस्प॒त्या धू॒म्रल॑लामास्तूप॒राः॥~(४२)

{\anuvakamend[{}]}%॥11॥

%5.6.12.1
पृश्ञि॑स्तिर॒श्चीन॑पृश्ञिरू॒र्ध्वपृ॑श्ञि॒स्ते मा॑रु॒ताः फ॒ल्गूर्लो॑हितो॒र्णी ब॑ल॒क्षी ताः सा॑रस्व॒त्यः॑ पृष॑ती स्थू॒लपृ॑षती क्षु॒द्रपृ॑षती॒ ता वै᳚श्वदे॒व्य॑स्ति॒स्रः श्या॒मा व॒शाः पौ॒ष्णिय॑स्ति॒स्रो रोहि॑णीर्व॒शा मै॒त्रिय॑ ऐन्द्राबार्\mbox{}हस्प॒त्या अ॑रु॒णल॑लामास्तूप॒राः॥~(४३)

{\anuvakamend[{रोहि॑तः॒ पृश्ञिः॒ षड्विꣳ॑शतिः॒ षड्विꣳ॑शतिः}]}%॥12॥

%5.6.13.1
शि॒ति॒बा॒हुर॒न्यतः॑शितिबाहुः सम॒न्तशि॑तिबाहु॒स्त ऐ᳚न्द्रवाय॒वाः शि॑ति॒रन्ध्रो॒\-ऽन्यतः॑शितिरन्ध्रः सम॒न्तशि॑तिरन्ध्र॒स्ते मै᳚त्रावरु॒णाः शु॒द्धवा॑लः स॒र्वशु॑द्धवालो म॒णिवा॑ल॒स्त आ᳚श्वि॒नास्ति॒स्रः शि॒ल्पा व॒शा वै॑श्वदे॒व्य॑स्ति॒स्रः श्येनीः᳚ परमे॒ष्ठिने॑ सोमापौ॒ष्णाः श्या॒मल॑लामास्तूप॒राः॥~(४४)

{\anuvakamend[{}]}%॥13॥

%5.6.14.1
उ॒न्न॒त ऋ॑ष॒भो वा॑म॒नस्त ऐ᳚न्द्रावरु॒णाः शिति॑ककुच्छितिपृ॒ष्ठः शिति॑भस॒त्त ऐ᳚न्द्राबार्\mbox{}हस्प॒त्याः शिति॒पाच्छि॒त्योष्ठः॑ शिति॒भ्रुस्त ऐ᳚न्द्रावैष्ण॒वास्ति॒स्रः सि॒ध्मा व॒शा वै᳚श्वकर्म॒ण्य॑स्ति॒स्रो धा॒त्रे पृ॑षोद॒रा ऐ᳚न्द्रापौ॒ष्णाः श्येत॑ललामास्तूप॒राः॥~(४५)

{\anuvakamend[{शि॒ति॒बा॒हुरु॑न्न॒तः पञ्च॑विꣳशतिः॒ पञ्च॑विꣳशतिः}]}%॥14॥

%5.6.15.1
क॒र्णास्त्रयो॑ या॒माः सौ॒म्यास्त्रयः॑ श्विति॒ङ्गा अ॒ग्नये॒ यवि॑ष्ठाय॒ त्रयो॑ नकु॒लास्ति॒स्रो रोहि॑णी॒स्त्र्यव्य॒स्ता वसू॑नान्ति॒स्रो॑\-ऽरु॒णा दि॑त्यौ॒ह्य॑स्ता रु॒द्राणाꣳ॑ सोमै॒न्द्रा ब॒भ्रुल॑लामास्तूप॒राः॥~(४६)

{\anuvakamend[{क॒र्णास्त्रयो॑विꣳशतिः}]}%॥15॥

%5.6.16.1
शु॒ण्ठास्त्रयो॑ वैष्ण॒वा अ॑धीलोध॒कर्णा॒स्त्रयो॒ विष्ण॑व उरुक्र॒माय॑ लफ्सु॒दिन॒स्त्रयो॒ विष्ण॑व उरुगा॒याय॒ पञ्चा॑वीस्ति॒स्र आ॑दि॒त्याना᳚न्त्रिव॒थ्सास्ति॒स्रो\-ऽङ्गि॑रसामैन्द्रावैष्ण॒वा गौ॒रल॑लामास्तूप॒राः॥~(४७)

{\anuvakamend[{शु॒ण्ठा विꣳ॑श॒तिः}]}%॥16॥

%5.6.17.1
इन्द्रा॑य॒ राज्ञे॒ त्रयः॑ शितिपृ॒ष्ठा इन्द्रा॑याधिरा॒जाय॒ त्रयः॒ शिति॑ककुद॒ इन्द्रा॑य स्व॒राज्ञे॒ त्रयः॒ शिति॑भसदस्ति॒स्रस्तु॑र्यौ॒ह्यः॑ सा॒ध्याना᳚न्ति॒स्रः प॑ष्ठौ॒ह्यो॑ विश्वे॑षां दे॒वाना॑माग्ने॒न्द्राः कृ॒ष्णल॑लामास्तूप॒राः॥~(४८)

{\anuvakamend[{इन्द्रा॑य॒ राज्ञे॒ द्वाविꣳ॑शतिः}]}%॥17॥

%5.6.18.1
अदि॑त्यै॒ त्रयो॑ रोहितै॒ता इ॑न्द्रा॒ण्यै त्रयः॑ कृष्णै॒ताः कु॒ह्वै᳚ त्रयो॑\-ऽरुणै॒तास्ति॒स्रो धे॒नवो॑ रा॒कायै॒ त्रयो॑\-ऽन॒ड्वाहः॑ सिनीवा॒ल्या आ᳚ग्नावैष्ण॒वा रोहि॑तललामास्तूप॒राः॥~(४९)

{\anuvakamend[{अदि॑त्या अ॒ष्टाद॑श}]}%॥18॥

%5.6.19.1
सौ॒म्यास्त्रयः॑ पि॒शङ्गाः॒ सोमा॑य॒ राज्ञे॒ त्रयः॑ सा॒रङ्गाः᳚ पार्ज॒न्या नभो॑रूपास्ति॒स्रो॑\-ऽजा म॒ल्॒\mbox{}हा इ॑न्द्रा॒ण्यै ति॒स्रो मे॒ष्य॑ आदि॒त्या द्या॑वापृथि॒व्या॑ मा॒लङ्गा᳚स्तूप॒राः॥~(५०)

{\anuvakamend[{सौ॒म्या एका॒न्नविꣳ॑शतिः}]}%॥19॥

%5.6.20.1
वा॒रु॒णास्त्रयः॑ कृ॒ष्णल॑लामा॒ वरु॑णाय॒ राज्ञे॒ त्रयो॒ रोहि॑तोललामा॒ वरु॑णाय रि॒शाद॑से॒ त्रयो॑\-ऽरु॒णल॑लामाः शि॒ल्पास्त्रयो॑ वैश्वदे॒वास्त्रयः॒ पृश्ञ॑यः सर्वदेव॒त्या॑ ऐन्द्रासू॒राः श्येत॑ललामास्तूप॒राः॥~(५१)

{\anuvakamend[{वा॒रु॒णा विꣳ॑श॒तिः}]}%॥20॥

%5.6.21.1
सोमा॑य स्व॒राज्ञे॑\-ऽनोवा॒हाव॑न॒ड्वाहा॑विन्द्रा॒ग्निभ्या॑मोजो॒दाभ्या॒मुष्टा॑राविन्द्रा॒ग्नि\-भ्यां᳚ बल॒दाभ्याꣳ॑ सीरवा॒हाववी॒ द्वे धे॒नू भौ॒मी दि॒ग्भ्यो वड॑बे॒ द्वे धे॒नू भौ॒मी वै॑रा॒जी पु॑रु॒षी द्वे धे॒नू भौ॒मी वा॒यव॑ आरोहणवा॒हाव॑न॒ड्वाहौ॑ वारु॒णी कृ॒ष्णे व॒शे अ॑रा॒ड्यौ॑ दि॒व्यावृ॑ष॒भौ प॑रिम॒रौ॥~(५२)

{\anuvakamend[{सोमा॑य स्व॒राज्ञे॒ चतु॑स्त्रिꣳशत्}]}%॥21॥

%5.6.22.1
एका॑\-दश प्रा॒तर्ग॒व्याः प॒शव॒ आ ल॑भ्यन्ते छग॒लः क॒ल्माषः॑ किकिदी॒विर्वि॑दी॒गय॒स्ते त्वा॒ष्ट्राः सौ॒रीर्नव॑ श्वे॒ता व॒शा अ॑नूब॒न्ध्या॑ भवन्त्याग्ने॒य ऐ᳚न्द्रा॒ग्न आ᳚श्वि॒नस्ते वि॑शालयू॒प आ ल॑भ्यन्ते॥~(५३)

{\anuvakamend[{एका॑\-दश॒ पञ्च॑विꣳशतिः}]}%॥22॥

%5.6.23.1
पि॒शङ्गा॒स्त्रयो॑ वास॒न्ताः सा॒रङ्गा॒स्त्रयो॒ ग्रैष्माः॒ पृष॑न्त॒स्त्रयो॒ वार्\mbox{}षि॑काः॒ पृश्ञ॑य॒स्त्रयः॑ शार॒दाः पृ॑श्ञिस॒क्थास्त्रयो॒ हैम॑न्तिका अवलि॒प्तास्त्रयः॑ शैशि॒राः सं॑वथ्स॒राय॒ निव॑क्षसः~(५४)

{\anuvakamend[{पि॒शङ्गा॑ विꣳश॒तिः}]}%॥23॥

\prashnaend{हिर॑ण्यवर्णा अ॒पां ग्रहा᳚न्भूतेष्ट॒काः स॒जूः सं॑वथ्स॒रं प्र॒जा\-प॑तिः॒ स क्षु॒रप॑विर॒ग्नेर्वै दी॒क्षया॑ सुव॒र्गाय॒ तं यन्न सू॒यते᳚ प्र॒जा\-प॑तिर्\mbox{}ऋ॒तुभी॒ रोहि॑तः॒ पृश्ञिः॑ शितिबा॒हुरु॑न्न॒तः क॒र्णाः शु॒ण्ठा इन्द्रा॒यादि॑त्यै सौ॒म्या वा॑रु॒णाः सोमा॒यैका॑\-दश पि॒शङ्गा॒स्त्रयो॑विꣳशतिः॥२३॥}{हिर॑ण्यवर्णा भूतेष्ट॒काश्छन्दो॒ यत्कनी॑याꣳसन्त्रि॒वृद्ध्य॑ग्निर्वा॑रु॒णाश्चतुः॑पञ्चाशत्॥५४॥}{हिर॑ण्यवर्णा॒ निव॑क्षसः॥}%%५-६
{हरिः॑ ॐ}{॥कृष्ण-यजुर्वेदीय-तैत्तिरीय-संहितायां पञ्चम्काण्डे षष्ठः प्रश्नः समाप्तः॥५-६॥}
%%% END PRASHNA

\sect{सप्तमः प्रश्नः}\setcounter{anuvakam}{0}
\dnsub{तैत्तिरीयसंहितायां पञ्चमकाण्डे सप्तमः प्रश्नः}
%5.7.1.1
यो वा अय॑थादेवतम॒ग्निं चि॑नु॒त आ दे॒वता᳚भ्यो वृश्च्यते॒ पापी॑यान्भवति॒ यो य॑थादेव॒तं न दे॒वता᳚भ्य॒ आ वृ॑श्च्यते॒ वसी॑यान्भवत्याग्ने॒य्या गा॑यत्रि॒या प्र॑थ॒मां चिति॑म॒भि मृ॑शेत्त्रि॒ष्टुभा᳚ द्वि॒तीयां॒ जग॑त्या तृ॒तीया॑मनु॒ष्टुभा॑ चतु॒र्थीं प॒ङ्क्त्या प॑ञ्च॒मीं य॑थादेव॒तमे॒वाग्निं चि॑नुते॒ न दे॒वता᳚भ्य॒ आ वृ॑श्च्यते॒ वसी॑यान्भव॒तीडा॑यै॒ वा ए॒षा विभ॑क्तिः प॒शव॒ इडा॑ प॒शुभि॑रेनम्~(१)

%5.7.1.2
चि॒नु॒ते॒ यो वै प्र॒जा\-प॑तये प्रति॒प्रोच्या॒ग्निं चि॒नोति॒ नार्ति॒मार्च्छ॒त्यश्वा॑व॒भित॑स्तिष्ठेतां कृ॒ष्ण उ॑त्तर॒तः श्वे॒तो दक्षि॑ण॒\-स्तावा॒लभ्येष्ट॑का॒ उप॑ दध्यादे॒तद्वै प्र॒जा\-प॑ते रू॒पं प्रा॑जाप॒त्यो\-ऽश्वः॑ सा॒क्षादे॒व प्र॒जा\-प॑तये प्रति॒प्रोच्या॒ग्निं चि॑नोति॒ नार्ति॒मार्च्छ॑त्ये॒तद्वा अह्नो॑ रू॒पं यच्छ्वे॒तो\-ऽश्वो॒ रात्रि॑यै कृ॒ष्ण ए॒तदह्नः॑~(२)

%5.7.1.3
रू॒पं यदिष्ट॑का॒ रात्रि॑यै॒ पुरी॑ष॒मिष्ट॑का उपधा॒स्यञ्छ्वे॒तमश्व॑म॒भि मृ॑शे॒त्पुरी॑षमुपधा॒स्यन्कृ॒ष्णम॑होरा॒त्राभ्या॑मे॒वैनं॑ चिनुते हिरण्यपा॒त्रं मधोः᳚ पू॒र्णं द॑दाति मध॒व्यो॑\-ऽसा॒नीति॑ सौ॒र्या चि॒त्रव॒त्यावे᳚क्षते चि॒त्रमे॒व भ॑वति म॒ध्यन्दि॒ने\-ऽश्व॒मव॑ घ्रापयत्य॒सौ वा आ॑दि॒त्य इन्द्र॑ ए॒ष प्र॒जा\-प॑तिः प्राजाप॒त्यो\-ऽश्व॒स्तमे॒व सा॒क्षादृ॑ध्नोति॥~(३)

{\anuvakamend[{ए॒नमे॒तदह्नो॒\-ऽष्टाच॑त्वारिꣳशच्च}]}%~(१)

%5.7.2.1
त्वाम॑ग्ने वृष॒भं चेकि॑तानं॒ पुन॒र्युवा॑नञ्ज॒नय॑न्नु॒पागा᳚म्। अ॒स्थू॒रि णो॒ गार्\mbox{}ह॑पत्यानि सन्तु ति॒ग्मेन॑ नो॒ ब्रह्म॑णा॒ सꣳ शि॑शाधि। प॒शवो॒ वा ए॒ते यदिष्ट॑का॒श्चित्यां᳚चित्यामृष॒भमुप॑ दधाति मिथु॒नमे॒वास्य॒ तद्य॒ज्ञे क॑रोति प्र॒जन॑नाय॒ तस्मा᳚द्यू॒थेयू॑थ ऋष॒भः। सं॒व॒थ्स॒रस्य॑ प्रति॒मां यां त्वा॑ रात्र्यु॒पास॑ते। प्र॒जाꣳ सु॒वीरां᳚ कृ॒त्वा विश्व॒मायु॒र्व्य॑श्ञवत्। प्रा॒जा॒प॒त्याम्~(४)

%5.7.2.2
ए॒तामुप॑ दधाती॒यं वावैषैका᳚ष्ट॒का यदे॒वैका᳚ष्ट॒काया॒मन्नं॑ क्रि॒यते॒ तदे॒वैतयाव॑ रुन्ध ए॒षा वै प्र॒जा\-प॑तेः काम॒दुघा॒ तयै॒व यज॑मानो॒\-ऽमुष्मिँ॑ल्लो॒के᳚\-ऽग्निं दु॑हे॒ येन॑ दे॒वा ज्योति॑षो॒र्ध्वा उ॒दाय॒न्॒ येना॑दि॒त्या वस॑वो॒ येन॑ रु॒द्राः। येनाङ्गि॑रसो महि॒मान॑मान॒शुस्तेनै॑तु॒ यज॑मानः स्व॒स्ति। सु॒व॒र्गाय॒ वा ए॒ष लो॒काय॑~(५)

%5.7.2.3
ची॒य॒ते॒ यद॒ग्निर्येन॑ दे॒वा ज्योति॑षो॒र्ध्वा उ॒दाय॒न्नित्युख्य॒ꣳ॒ समि॑न्द्ध॒ इष्ट॑का ए॒वैता उप॑ धत्ते वानस्प॒त्याः सु॑व॒र्गस्य॑ लो॒कस्य॒ सम॑ष्ट्यै श॒तायु॑धाय श॒तवी᳚र्याय श॒तोत॑ये\-ऽभिमाति॒षाहे᳚। श॒तं यो नः॑ श॒रदो॒ अजी॑ता॒निन्द्रो॑ नेष॒दति॑ दुरि॒तानि॒ विश्वा᳚। ये च॒त्वारः॑ प॒थयो॑ देव॒याना॑ अन्त॒रा द्यावा॑पृथि॒वी वि॒यन्ति॑। तेषां॒ यो अज्या॑नि॒मजी॑तिमा॒ वहा॒त्तस्मै॑ नो देवाः~(६)

%5.7.2.4
परि॑ दत्ते॒ह सर्वे᳚। ग्री॒ष्मो हे॑म॒न्त उ॒त नो॑ वस॒न्तः श॒रद्व॒र्॒\mbox{}षाः सु॑वि॒तं नो॑ अस्तु। तेषा॑मृतू॒नाꣳ श॒तशा॑रदानां निवा॒त ए॑षा॒मभ॑ये स्याम। इ॒दु॒व॒थ्स॒राय॑ परिवथ्स॒राय॑ संवथ्स॒राय॑ कृणुता बृ॒हन्नमः॑। तेषां᳚ व॒यꣳ सु॑म॒तौ य॒ज्ञिया॑नां॒ ज्योगजी॑ता॒ अह॑ताः स्याम। भ॒द्रान्नः॒ श्रेयः॒ सम॑नैष्ट देवा॒स्त्वया॑व॒सेन॒ सम॑शीमहि त्वा। स नो॑ मयो॒भूः पि॑तो~(७)

%5.7.2.5
आ वि॑शस्व॒ शं तो॒काय॑ त॒नुवे᳚ स्यो॒नः। अज्या॑नीरे॒ता उप॑ दधात्ये॒ता वै दे॒वता॒ अप॑राजिता॒स्ता ए॒व प्र वि॑शति॒ नैव जी॑यते ब्रह्मवा॒दिनो॑ वदन्ति॒ यद॑र्धमा॒सा मासा॑ ऋ॒तवः॑ संवथ्स॒र ओष॑धीः॒ पच॒न्त्यथ॒ कस्मा॑द॒न्याभ्यो॑ दे॒वता᳚भ्य आग्रय॒णं निरु॑प्यत॒ इत्ये॒ता हि तद्दे॒वता॑ उ॒दज॑य॒न्॒ यदृ॒तुभ्यो॑ नि॒र्वपे᳚द्दे॒वता᳚भ्यः स॒मदं॑ दध्यादाग्रय॒णं नि॒रुप्यै॒ता आहु॑तीर्जुहोत्यर्धमा॒साने॒व मासा॑नृ॒तून्थ्सं॑वथ्स॒रं प्री॑णाति॒ न दे॒वता᳚भ्यः स॒मदं॑ दधाति भ॒द्रान्नः॒ श्रेयः॒ सम॑नैष्ट देवा॒ इत्या॑ह हु॒ताद्या॑य॒ यज॑मान॒स्याप॑राभावाय॥~(८)

{\anuvakamend[{प्रा॒जा॒प॒त्यां लो॒काय॑ देवाः पितो दध्यादाग्रय॒णं पञ्च॑विꣳशतिश्च}]}%~(२)

%5.7.3.1
इन्द्र॑स्य॒ वज्रो॑\-ऽसि॒ वार्त्र॑घ्नस्तनू॒पा नः॑ प्रतिस्प॒शः। यो नः॑ पु॒रस्ता᳚द्दक्षिण॒तः प॒श्चादु॑त्तर॒तो॑\-ऽघा॒युर॑भि॒दास॑त्ये॒तꣳ सो\-ऽश्मा॑नमृच्छतु। दे॒वा॒सु॒राः संय॑त्ता आस॒न्ते\-ऽसु॑रा दि॒ग्भ्य आबा॑धन्त॒ तां दे॒वा इष्वा॑ च॒ वज्रे॑ण॒ चापा॑नुदन्त॒ यद्व॒ज्रिणी॑रुप॒दधा॒तीष्वा॑ चै॒व तद्वज्रे॑ण च॒ यज॑मानो॒ भ्रातृ॑व्या॒नप॑ नुदते दि॒क्षूप॑~(९)

%5.7.3.2
द॒धा॒ति॒ दे॒व॒पु॒रा ए॒वैतास्त॑नू॒पानीः॒ पर्यू॑ह॒ते\-ऽग्ना॑विष्णू स॒जोष॑से॒मा व॑र्धन्तु वां॒ गिरः॑। द्यु॒म्नैर्वाजे॑भि॒रा ग॑तम्। ब्र॒ह्म॒वा॒दिनो॑ वदन्ति॒ यन्न दे॒वता॑यै॒ जुह्व॒त्यथ॑ किन्देव॒त्या॑ वसो॒र्धारेत्य॒ग्निर्वसु॒स्तस्यै॒षा धारा॒ विष्णु॒र्वसु॒स्तस्यै॒षा धारा᳚ग्नावैष्ण॒व्यर्चा वसो॒र्धारां᳚ जुहोति भाग॒धेये॑नै॒वैनौ॒ सम॑र्धय॒त्यथो॑ ए॒ताम्~(१०)

%5.7.3.3
ए॒वाहु॑तिमा॒यत॑नवतीं करोति॒ यत्का॑म एनां जु॒होति॒ तदे॒वाव॑ रुन्धे रु॒द्रो वा ए॒ष यद॒ग्निस्तस्यै॒ते त॒नुवौ॑ घो॒रान्या शि॒वान्या यच्छ॑तरु॒द्रीयं॑ जु॒होति॒ यैवास्य॑ घो॒रा त॒नूस्तां तेन॑ शमयति॒ यद्वसो॒र्धारां᳚ जु॒होति॒ यैवास्य॑ शि॒वा त॒नूस्तां तेन॑ प्रीणाति॒ यो वै वसो॒र्धारा॑यै~(११)

%5.7.3.4
प्र॒ति॒ष्ठां वेद॒ प्रत्ये॒व ति॑ष्ठति॒ यदाज्य॑मु॒च्छिष्ये॑त॒ तस्मि॑न्ब्रह्मौद॒नं प॑चे॒त्तं ब्रा᳚ह्म॒णाश्च॒त्वारः॒ प्राश्ञी॑युरे॒ष वा अ॒ग्निर्वै᳚श्वान॒रो यद्ब्रा᳚ह्म॒ण ए॒षा खलु॒ वा अ॒ग्नेः प्रि॒या त॒नूर्यद्वै᳚श्वान॒रः प्रि॒याया॑मे॒वैनां᳚ त॒नुवां॒ प्रति॑\-ष्ठापयति॒ चत॑स्रो धे॒नूर्द॑द्या॒त्ताभि॑रे॒व यज॑मानो॒\-ऽमुष्मिँ॑ल्लो॒के᳚\-ऽग्निं दु॑हे॥~(१२)

{\anuvakamend[{उपै॒तान्धारा॑यै॒ षट्च॑त्वारिꣳशच्च}]}%~(३)

%5.7.4.1
चित्ति॑ञ्जुहोमि॒ मन॑सा घृ॒तेनेत्या॒हादा᳚भ्या॒ वै नामै॒षाहु॑तिर्वैश्वकर्म॒णी नैनं॑ चिक्या॒नं भ्रातृ॑व्यो दभ्नो॒त्यथो॑ दे॒वता॑ ए॒वाव॑ रु॒न्धे\-ऽग्ने॒ तम॒द्येति॑ प॒ङ्क्त्या जु॑होति प॒ङ्क्त्याहु॑त्या यज्ञमु॒खमार॑भते स॒प्त ते॑ अग्ने स॒मिधः॑ स॒प्त जि॒ह्वा इत्या॑ह॒ होत्रा॑ ए॒वाव॑ रुन्धे॒\-ऽग्निर्दे॒वेभ्यो\-ऽपा᳚क्रामद्भाग॒धेयम्᳚~(१३)

%5.7.4.2
इ॒च्छमा॑न॒स्तस्मा॑ ए॒तद्भा॑ग॒धेयं॒ प्राय॑च्छन्ने॒तद्वा अ॒ग्नेर॑ग्निहो॒त्रमे॒तर्\mbox{}हि॒ खलु॒ वा ए॒ष जा॒तो यर्\mbox{}हि॒ सर्व॑श्चि॒तो जा॒तायै॒वास्मा॒ अन्न॒मपि॑ दधाति॒ स ए॑नं प्री॒तः प्री॑णाति॒ वसी॑यान्भवति ब्रह्मवा॒दिनो॑ वदन्ति॒ यदे॒ष गार्\mbox{}ह॑पत्यश्ची॒यते\-ऽथ॒ क्वा᳚स्याहव॒नीय॒ इत्य॒सावा॑दि॒त्य इति॑ ब्रूयादे॒तस्मि॒न् हि सर्वा᳚भ्यो दे॒वता᳚भ्यो॒ जुह्व॑ति~(१४)

%5.7.4.3
य ए॒वं वि॒द्वान॒ग्निं चि॑नु॒ते सा॒क्षादे॒व दे॒वता॑ ऋध्नो॒त्यग्ने॑ यशस्वि॒न्॒ यश॑से॒मम॑र्प॒येन्द्रा॑वती॒मप॑चितीमि॒हा व॑ह। अ॒यं मू॒र्धा प॑रमे॒ष्ठी सु॒वर्चाः᳚ समा॒नाना॑मुत्त॒मश्लो॑को अस्तु। भ॒द्रं पश्य॑न्त॒ उप॑ सेदु॒रग्रे॒ तपो॑ दी॒क्षामृष॑यः सुव॒र्विदः॑। ततः॑ क्ष॒त्रं बल॒मोज॑श्च जा॒तं तद॒स्मै दे॒वा अ॒भि सं न॑मन्तु। धा॒ता वि॑धा॒ता प॑र॒मा~(१५)

%5.7.4.4
उ॒त स॒न्दृक्प्र॒जा\-प॑तिः परमे॒ष्ठी वि॒राजा᳚। स्तोमा॒श्छन्दाꣳ॑सि नि॒विदो॑ म आहुरे॒तस्मै॑ रा॒ष्ट्रम॒भि सं न॑माम। अ॒भ्याव॑र्तध्व॒मुप॒ मेत॑ सा॒कम॒यꣳ शा॒स्ताधि॑पतिर्वो अस्तु। अ॒स्य वि॒ज्ञान॒मनु॒ सꣳ र॑भध्वमि॒मं प॒श्चादनु॑ जीवाथ॒ सर्वे᳚। रा॒ष्ट्रभृत॑ ए॒ता उप॑ दधात्ये॒षा वा अ॒ग्नेश्चिती॑ राष्ट्र॒भृत्तयै॒वास्मि॑न्रा॒ष्ट्रं द॑धाति रा॒ष्ट्रमे॒व भ॑वति॒ नास्मा᳚द्रा॒ष्ट्रं भ्रꣳ॑शते॥~(१६)

{\anuvakamend[{भा॒ग॒धेय॒ञ्जुह्व॑ति पर॒मा रा॒ष्ट्रं द॑धाति स॒प्त च॑}]}%~(४)

%5.7.5.1
यथा॒ वै पु॒त्रो जा॒तो म्रि॒यत॑ ए॒वं वा ए॒ष म्रि॑यते॒ यस्या॒ग्निरुख्य॑ उ॒द्वाय॑ति॒ यन्नि॑र्म॒न्थ्यं॑ कु॒र्याद्विच्छि॑न्द्या॒द्भ्रातृ॑व्यमस्मै जनये॒थ्स ए॒व पुनः॑ प॒रीध्यः॒ स्वादे॒वैनं॒ योने᳚र्जनयति॒ नास्मै॒ भ्रातृ॑व्यं जनयति॒ तमो॒ वा ए॒तं गृ॑ह्णाति॒ यस्या॒ग्निरुख्य॑ उ॒द्वाय॑ति मृ॒त्युस्तमः॑ कृ॒ष्णं वासः॑ कृ॒ष्णा धे॒नुर्दक्षि॑णा॒ तम॑सा~(१७)

%5.7.5.2
ए॒व तमो॑ मृ॒त्युमप॑ हते॒ हिर॑ण्यं ददाति॒ ज्योति॒र्वै हिर॑ण्यं॒ ज्योति॑षै॒व तमो\-ऽप॑ ह॒ते\-ऽथो॒ तेजो॒ वै हिर॑ण्य॒न्तेज॑ ए॒वा\-ऽऽ\-त्मन्ध॑त्ते॒ सुव॒र्न घ॒र्मः स्वाहा॒ सुव॒र्नार्कः स्वाहा॒ सुव॒र्न शु॒क्रः स्वाहा॒ सुव॒र्न ज्योतिः॒ स्वाहा॒ सुव॒र्न सूर्यः॒ स्वाहा॒र्को वा ए॒ष यद॒ग्निर॒सावा॑दि॒त्यः~(१८)

%5.7.5.3
अ॒श्व॒मे॒धो यदे॒ता आहु॑तीर्जु॒होत्य॑र्काश्वमे॒धयो॑रे॒व ज्योतीꣳ॑षि॒ सं द॑धात्ये॒ष ह॒ त्वा अ॑र्काश्वमे॒धी यस्यै॒तद॒ग्नौ क्रि॒यत॒ आपो॒ वा इ॒दमग्रे॑ सलि॒लमा॑सी॒थ्स ए॒तां प्र॒जा\-प॑तिः प्रथ॒मां चिति॑मपश्य॒त्तामुपा॑धत्त॒ तदि॒यम॑भव॒त्तं वि॒श्वक॑र्माब्रवी॒दुप॒ त्वाया॒नीति॒ नेह लो॒को᳚\-ऽस्तीति॑~(१९)

%5.7.5.4
अ॒ब्र॒वी॒थ्स ए॒तां द्वि॒तीयां॒ चिति॑मपश्य॒त्तामुपा॑धत्त॒ तद॒न्तरि॑क्षमभव॒थ्स य॒ज्ञः प्र॒जा\-प॑तिमब्रवी॒दुप॒ त्वाया॒नीति॒ नेह लो॒को᳚\-ऽस्तीत्य॑ब्रवी॒थ्स वि॒श्वक॑र्माणमब्रवी॒दुप॒ त्वाया॒नीति॒ केन॑ मो॒पैष्य॒सीति॒ दिश्या॑भि॒रित्य॑ब्रवी॒त्तन्दिश्या॑भिरु॒पैत्ता उपा॑धत्त॒ ता दिशः॑~(२०)

%5.7.5.5
अ॒भ॒व॒न्थ्स प॑रमे॒ष्ठी प्र॒जा\-प॑तिमब्रवी॒दुप॒ त्वाया॒नीति॒ नेह लो॒को᳚\-ऽस्तीत्य॑ब्रवी॒थ्स वि॒श्वक॑र्माणं च य॒ज्ञं चा᳚ब्रवी॒दुप॑ वा॒माया॒नीति॒ नेह लो॒को᳚\-ऽस्तीत्य॑ब्रूता॒ꣳ॒ स ए॒तां तृ॒तीयां॒ चिति॑मपश्य॒त्तामुपा॑धत्त॒ तद॒साव॑भव॒थ्स आ॑दि॒त्यः प्र॒जा\-प॑तिमब्रवी॒दुप॑ त्वा~(२१)

%5.7.5.6
आ॒या॒नीति॒ नेह लो॒को᳚\-ऽस्तीत्य॑ब्रवी॒थ्स वि॒श्वक॑र्माणं च य॒ज्ञं चा᳚ब्रवी॒दुप॑ वा॒माया॒नीति॒ नेह लो॒को᳚\-ऽस्तीत्य॑ब्रूता॒ꣳ॒ स प॑रमे॒ष्ठिन॑मब्रवी॒दुप॒ त्वाया॒नीति॒ केन॑ मो॒पैष्य॒सीति॑ लोकं पृ॒णयेत्य॑ब्रवी॒त्तं लो॑कं पृ॒णयो॒पैत्तस्मा॒दया॑तयाम्नी लोकं पृ॒णा\-ऽया॑तयामा॒ ह्य॑सौ~(२२)

%5.7.5.7
आ॒दि॒त्यस्तानृष॑यो\-ऽब्रुव॒न्नुप॑ व॒ आया॒मेति॒ केन॑ न उ॒पैष्य॒थेति॑ भू॒म्नेत्य॑ब्रुव॒न्तां द्वाभ्यां॒ चिती᳚भ्यामु॒पाय॒न्थ्स पञ्च॑चितीकः॒ सम॑पद्यत॒ य ए॒वं वि॒द्वान॒ग्निं चि॑नु॒ते भूया॑ने॒व भ॑वत्य॒भीमाँल्लो॒काञ्ज॑यति वि॒दुरे॑नं दे॒वा अथो॑ ए॒तासा॑मे॒व दे॒वता॑ना॒ꣳ॒ सायु॑ज्यं गच्छति॥~(२३)

{\anuvakamend[{तम॑सा\-ऽऽ\-दि॒त्यो᳚\-ऽस्तीति॒ दिश॑ आदि॒त्यः प्र॒जा\-प॑तिमब्रवी॒दुप॑ त्वा॒\-ऽसौ पञ्च॑चत्वारिꣳशच्च}]}%~(६)

%5.7.6.1
वयो॒ वा अ॒ग्निर्यद॑ग्नि॒चित्प॒क्षिणो᳚\-ऽश्ञी॒यात्तमे॒वाग्निम॑द्या॒दार्ति॒मार्च्छे᳚थ्संवथ्स॒रं व्र॒तं च॑रेथ्संवथ्स॒रꣳ हि व्र॒तं नाति॑ प॒शुर्वा ए॒ष यद॒ग्निर्\mbox{}हि॒नस्ति॒ खलु॒ वै तं प॒शुर्य ए॑नं पु॒रस्ता᳚त्प्र॒त्यञ्च॑मुप॒चर॑ति॒ तस्मा᳚त्प॒श्चात्प्राङु॑प॒चर्य॑ आ॒त्मनो\-ऽहिꣳ॑सायै॒ तेजो॑\-ऽसि॒ तेजो॑ मे यच्छ पृथि॒वीं य॑च्छ~(२४)

%5.7.6.2
पृ॒थि॒व्यै मा॑ पाहि॒ ज्योति॑रसि॒ ज्योति॑र्मे यच्छा॒न्तरि॑क्षं यच्छा॒न्तरि॑क्षान्मा पाहि॒ सुव॑रसि॒ सुव॑र्मे यच्छ॒ दिवं॑ यच्छ दि॒वो मा॑ पा॒हीत्या॑है॒ताभि॒र्वा इ॒मे लो॒का विधृ॑ता॒ यदे॒ता उ॑प॒दधा᳚त्ये॒षां लो॒कानां॒ विधृ॑त्यै स्वयमातृ॒ण्णा उ॑प॒धाय॑ हिरण्येष्ट॒का उप॑ दधाती॒मे वै लो॒काः स्व॑यमातृ॒ण्णा ज्योति॒र्॒\mbox{}हिर॑ण्यं॒ यथ्स्व॑यमातृ॒ण्णा उ॑प॒धाय॑~(२५)

%5.7.6.3
हि॒र॒ण्ये॒ष्ट॒का उ॑प॒दधा॑ती॒माने॒वैताभि॑र्लो॒कां ज्योति॑ष्मतः कुरु॒ते\-ऽथो॑ ए॒ताभि॑रे॒वास्मा॑ इ॒मे लो॒काः प्र भा᳚न्ति॒ यास्ते॑ अग्ने॒ सूर्ये॒ रुच॑ उद्य॒तो दिव॑मात॒न्वन्ति॑ र॒श्मिभिः॑। ताभिः॒ सर्वा॑भी रु॒चे जना॑य नस्कृधि। या वो॑ देवाः॒ सूर्ये॒ रुचो॒ गोष्वश्वे॑षु॒ या रुचः॑। इन्द्रा᳚ग्नी॒ ताभिः॒ सर्वा॑भी॒ रुचं॑ नो धत्त बृहस्पते। रुचं॑ नो धेहि~(२६)

%5.7.6.4
ब्रा॒ह्म॒णेषु॒ रुच॒ꣳ॒ राज॑सु नस्कृधि। रुचं॑ वि॒श्ये॑षु शू॒द्रेषु॒ मयि॑ धेहि रु॒चा रुचम्᳚। द्वे॒धा वा अ॒ग्निं चि॑क्या॒नस्य॒ यश॑ इन्द्रि॒यं ग॑च्छत्य॒ग्निं वा॑ चि॒तमी॑जा॒नं वा॒ यदे॒ता आहु॑तीर्जु॒होत्या॒त्मन्ने॒व यश॑ इन्द्रि॒यं ध॑त्त ईश्व॒रो वा ए॒ष आर्ति॒मार्तो॒र्यो᳚\-ऽग्निं चि॒न्वन्न॑धि॒क्राम॑ति॒ तत्त्वा॑ यामि॒ ब्रह्म॑णा॒ वन्द॑मान॒ इति॑ वारु॒ण्यर्चा~(२७)

%5.7.6.5
जु॒हु॒या॒च्छान्ति॑रे॒वैषाग्नेर्गुप्ति॑रा॒त्मनो॑ ह॒विष्कृ॑तो॒ वा ए॒ष यो᳚\-ऽग्निं चि॑नु॒ते यथा॒ वै ह॒विः स्कन्द॑त्ये॒वं वा ए॒ष स्क॑न्दति॒ यो᳚\-ऽग्निं चि॒त्वा स्त्रिय॑मु॒पैति॑ मैत्रावरु॒ण्यामिक्ष॑या यजेत मैत्रावरु॒णता॑मे॒वोपै᳚त्या॒त्मनो\-ऽस्क॑न्दाय॒ यो वा अ॒ग्निमृ॑तु॒स्थां वेद॒र्तुर्\mbox{}ऋ॑तुरस्मै॒ कल्प॑मान एति॒ प्रत्ये॒व ति॑ष्ठति संवथ्स॒रो वा अ॒ग्निः~(२८)

%5.7.6.6
ऋ॒तु॒स्थास्तस्य॑ वस॑न्तः॒ शिरो᳚ ग्री॒ष्मो दक्षि॑णः प॒क्षो व॒र्॒\mbox{}षाः पुच्छꣳ॑ श॒रदुत्त॑रः प॒क्षो हे॑म॒न्तो मध्यं॑ पूर्वप॒क्षाश्चित॑यो\-ऽपरप॒क्षाः पुरी॑षमहोरा॒त्राणीष्ट॑का ए॒ष वा अ॒ग्निर्\mbox{}ऋ॑तु॒स्था य ए॒वं वेद॒र्तुर्\mbox{}ऋ॑तुरस्मै॒ कल्प॑मान एति॒ प्रत्ये॒व ति॑ष्ठति प्र॒जा\-प॑ति॒र्वा ए॒तं ज्यैष्ठ्य॑कामो॒ न्य॑धत्त॒ ततो॒ वै स ज्यैष्ठ्य॑मगच्छ॒द्य ए॒वं वि॒द्वान॒ग्निं चि॑नु॒ते ज्यैष्ठ्य॑मे॒व ग॑च्छति॥~(२९)

{\anuvakamend[{पृ॒थि॒वीं य॑च्छ॒ यथ्स्व॑यमातृ॒ण्णा उ॑प॒धाय॑ धेह्यृ॒चाग्निश्चि॑नु॒ते त्रीणि॑ च}]}%~(७)

%5.7.7.1
यदाकू॑ताथ्स॒मसु॑स्रोद्धृ॒दो वा॒ मन॑सो वा॒ सम्भृ॑तं॒ चक्षु॑षो वा। तमनु॒ प्रेहि॑ सुकृ॒तस्य॑ लो॒कं यत्रर्\mbox{}ष॑यः प्रथम॒जा ये पु॑रा॒णाः। ए॒तꣳ स॑धस्थ॒ परि॑ ते ददामि॒ यमा॒वहा᳚च्छेव॒धिं जा॒तवे॑दाः। अ॒न्वा॒ग॒न्ता य॒ज्ञप॑तिर्वो॒ अत्र॒ तꣴ स्म॑ जानीत पर॒मे व्यो॑मन्न्। जा॒नी॒तादे॑नं पर॒मे व्यो॑म॒न्देवाः᳚ सधस्था वि॒द रू॒पम॑स्य। यदा॒गच्छा᳚त्~(३०)

%5.7.7.2
प॒थिभि॑र्देव॒यानै॑रिष्टापू॒र्ते कृ॑णुतादा॒विर॑स्मै। सम्प्र च्य॑वध्व॒मनु॒ सम्प्र या॒ताग्ने॑ प॒थो दे॑व॒याना᳚न्कृणुध्वम्। अ॒स्मिन्थ्स॒धस्थे॒ अध्युत्त॑रस्मि॒न्विश्वे॑ देवा॒ यज॑मानश्च सीदत। प्र॒स्त॒रेण॑ परि॒धिना᳚ स्रु॒चा वेद्या॑ च ब॒र्॒\mbox{}हिषा᳚। ऋ॒चेमं य॒ज्ञं नो॑ वह॒ सुव॑र्दे॒वेषु॒ गन्त॑वे। यदि॒ष्टं यत्प॑रा॒दानं॒ यद्द॒त्तं या च॒ दक्षि॑णा। तत्~(३१)

%5.7.7.3
अ॒ग्निर्वै᳚श्वकर्म॒णः सुव॑र्दे॒वेषु॑ नो दधत्। येना॑ स॒हस्रं॒ वह॑सि॒ येना᳚ग्ने सर्ववेद॒सम्। तेने॒मं य॒ज्ञं नो॑ वह॒ सुव॑र्दे॒वेषु॒ गन्त॑वे। येना᳚ग्ने॒ दक्षि॑णा यु॒क्ता य॒ज्ञं वह॑न्त्यृ॒त्विजः॑। तेने॒मं य॒ज्ञं नो॑ वह॒ सुव॑र्दे॒वेषु॒ गन्त॑वे। येना᳚ग्ने सु॒कृतः॑ प॒था मधो॒र्धारा᳚ व्यान॒शुः। तेने॒मं य॒ज्ञं नो॑ वह॒ सुव॑र्दे॒वेषु॒ गन्त॑वे। यत्र॒ धारा॒ अन॑पेता॒ मधो᳚र्घृ॒तस्य॑ च॒ याः। तद॒ग्निर्वै᳚श्वकर्म॒णः सुव॑र्दे॒वेषु॑ नो दधत्॥~(३२)

{\anuvakamend[{आ॒गच्छा॒त्तद्व्या॑न॒शुस्तेने॒मं य॒ज्ञं नो॑ वह॒ सुव॑र्दे॒वेषु॒ गन्त॑वे॒ चतु॑र्दश च}]}%~(७)

%5.7.8.1
यास्ते॑ अग्ने स॒मिधो॒ यानि॒ धाम॒ या जि॒ह्वा जा॑तवेदो॒ यो अ॒र्चिः। ये ते॑ अग्ने मे॒डयो॒ य इन्द॑व॒स्तेभि॑रा॒त्मानं॑ चिनुहि प्रजा॒नन्न्। उ॒थ्स॒न्न॒य॒ज्ञो वा ए॒ष यद॒ग्निः किं वाहै॒तस्य॑ क्रि॒यते॒ किं वा॒ न यद्वा अ॑ध्व॒र्युर॒ग्नेश्चि॒न्वन्न॑न्त॒रेत्या॒त्मनो॒ वै तद॒न्तरे॑ति॒ यास्ते॑ अग्ने स॒मिधो॒ यानि॑~(३३)

%5.7.8.2
धामेत्या॑है॒षा वा अ॒ग्नेः स्व॑यञ्चि॒तिर॒ग्निरे॒व तद॒ग्निं चि॑नोति॒ नाध्व॒र्युरा॒त्मनो॒\-ऽन्तरे॑ति॒ चत॑स्र॒ आशाः॒ प्र च॑रन्त्व॒ग्नय॑ इ॒मं नो॑ य॒ज्ञं न॑यतु प्रजा॒नन्न्। घृ॒तं पिन्व॑न्न॒जरꣳ॑ सु॒वीरं॒ ब्रह्म॑ स॒मिद्भ॑व॒त्याहु॑तीनाम्। सु॒व॒र्गाय॒ वा ए॒ष लो॒कायोप॑ धीयते॒ यत्कू॒र्मश्चत॑स्र॒ आशाः॒ प्र च॑रन्त्व॒ग्नय॒ इत्या॑ह~(३४)

%5.7.8.3
दिश॑ ए॒वैतेन॒ प्र जा॑नाती॒मं नो॑ य॒ज्ञं न॑यतु प्रजा॒नन्नित्या॑ह सुव॒र्गस्य॑ लो॒कस्या॒भ᳚नी॑त्यै॒ ब्रह्म॑ स॒मिद्भ॑व॒त्याहु॑तीना॒मित्या॑ह॒ ब्रह्म॑णा॒ वै दे॒वाः सु॑व॒र्गं लो॒कमा॑य॒न्॒ यद्ब्रह्म॑ण्वत्योप॒दधा॑ति॒ ब्रह्म॑णै॒व तद्यज॑मानः सुव॒र्गं लो॒कमे॑ति प्र॒जा\-प॑ति॒र्वा ए॒ष यद॒ग्निस्तस्य॑ प्र॒जाः प॒शव॒श्छन्दाꣳ॑सि रू॒पꣳ सर्वा॒न् वर्णा॒निष्ट॑कानां कुर्याद्रू॒पेणै॒व प्र॒जां प॒शूञ्छन्दा॒ꣴ॒स्यव॑ रु॒न्धे\-ऽथो᳚ प्र॒जाभ्य॑ ए॒वैनं॑ प॒शुभ्य॒श्छन्दो᳚भ्यो\-ऽव॒रुद्ध्य॑ चिनुते॥~(३५)

{\anuvakamend[{यान्य॒ग्नय॒ इत्या॒हेष्ट॑काना॒ꣳ॒ षोड॑श च}]}%~(८)

%5.7.9.1
मयि॑ गृह्णा॒म्यग्रे॑ अ॒ग्निꣳ रा॒यस्पोषा॑य सुप्रजा॒स्त्वाय॑ सु॒वीर्या॑य। मयि॑ प्र॒जां मयि॒ वर्चो॑ दधा॒म्यरि॑ष्टाः स्याम त॒नुवा॑ सु॒वीराः᳚। यो नो॑ अ॒ग्निः पि॑तरो हृ॒थ्स्व॑न्तरम॑र्त्यो॒ मर्त्याꣳ॑ आवि॒वेश॑। तमा॒त्मन्परि॑ गृह्णीमहे व॒यं मा सो अ॒स्माꣳ अ॑व॒हाय॒ परा॑ गात्। यद॑ध्व॒र्युरा॒त्मन्न॒ग्निमगृ॑हीत्वा॒ग्निं चि॑नु॒याद्यो᳚\-ऽस्य॒ स्वो᳚\-ऽग्निस्तमपि॑~(३६)

%5.7.9.2
यज॑मानाय चिनुयाद॒ग्निं खलु॒ वै प॒शवो\-ऽनूप॑ तिष्ठन्ते\-ऽप॒क्रामु॑का अस्मात्प॒शवः॑ स्यु॒र्मयि॑ गृह्णाम्यग्रे॑ अ॒ग्निमित्या॑हा॒त्मन्ने॒व स्वम॒ग्निं दा॑धार॒ नास्मा᳚त्प॒शवो\-ऽप॑ क्रामन्ति ब्रह्मवा॒दिनो॑ वदन्ति॒ यन्मृच्चाप॑श्चा॒ग्नेर॑ना॒द्यमथ॒ कस्मा᳚न्मृ॒दा चा॒द्भिश्चा॒ग्निश्ची॑यत॒ इति॒ यद॒द्भिः सं॒यौति॑~(३७)

%5.7.9.3
आपो॒ वै सर्वा॑ दे॒वता॑ दे॒वता॑भिरे॒वैन॒ꣳ॒ सꣳ सृ॑जति॒ यन्मृ॒दा चि॒नोती॒यं वा अ॒ग्निर्वै᳚श्वान॒रो᳚\-ऽग्निनै॒व तद॒ग्निं चि॑नोति ब्रह्मवा॒दिनो॑ वदन्ति॒ यन्मृ॒दा चा॒द्भिश्चा॒ग्निश्ची॒यते\-ऽथ॒ कस्मा॑द॒ग्निरु॑च्यत॒ इति॒ यच्छन्दो॑भिश्चि॒नोत्य॒ग्नयो॒ वै छन्दाꣳ॑सि॒ तस्मा॑द॒ग्निरु॑च्य॒ते\-ऽथो॑ इ॒यं वा अ॒ग्निर्वै᳚श्वान॒रो यत्~(३८)

%5.7.9.4
मृ॒दा चि॒नोति॒ तस्मा॑द॒ग्निरु॑च्यते हिरण्येष्ट॒का उप॑ दधाति॒ ज्योति॒र्वै हिर॑ण्यं॒ ज्योति॑रे॒वास्मि॑न्दधा॒त्यथो॒ तेजो॒ वै हिर॑ण्यं॒ तेज॑ ए॒वा\-ऽऽ\-त्मन्ध॑त्ते॒ यो वा अ॒ग्निꣳ स॒र्वतो॑मुखं चिनु॒ते सर्वा॑सु प्र॒जास्वन्न॑मत्ति॒ सर्वा॒ दिशो॒\-ऽभि ज॑यति गाय॒त्रीं पु॒रस्ता॒दुप॑ दधाति त्रिष्टुभं॑ दक्षिण॒तो जग॑तीं प॒श्चाद॑नु॒ष्टुभ॑मुत्तर॒तः प॒ङ्क्तिं मध्य॑ ए॒ष वा अ॒ग्निः स॒र्वतो॑मुख॒स्तं य ए॒वं वि॒द्वाꣴश्चि॑नु॒ते सर्वा॑सु प्र॒जास्वन्न॑मत्ति॒ सर्वा॒ दिशो॒\-ऽभि ज॑य॒त्यथो॑ दि॒श्ये॑व दिशं॒ प्र व॑यति॒ तस्मा᳚द्दि॒शि दिक्प्रोता᳚॥~(३९)

{\anuvakamend[{अपि॑ सं॒ यौति॑ वैश्वान॒रो यदे॒ष वै पञ्च॑विꣳशतिश्च}]}%~(९)

%5.7.10.1
प्र॒जा\-प॑तिर॒ग्निम॑सृजत॒ सो᳚\-ऽस्माथ्सृ॒ष्टः प्राङ्प्राद्र॑व॒त्तस्मा॒ अश्वं॒ प्रत्या᳚स्य॒थ्स द॑क्षि॒णाव॑र्तत॒ तस्मै॑ वृ॒ष्णिं प्रत्या᳚स्य॒थ्स प्र॒त्यङ्ङाव॑र्तत॒ तस्मा॑ ऋष॒भं प्रत्या᳚स्य॒थ्स उद॒ङ्ङाव॑र्तत॒ तस्मै॑ ब॒स्तं प्रत्या᳚स्य॒थ्स ऊ॒र्ध्वो᳚\-ऽद्रव॒त्तस्मै॒ पुरु॑षं॒ प्रत्या᳚स्य॒त् यत्प॑शुशी॒र्॒\mbox{}षाण्यु॑प॒दधा॑ति स॒र्वत॑ ए॒वैनम्᳚~(४०)

%5.7.10.2
अ॒व॒रुध्य॑ चिनुत ए॒ता वै प्रा॑ण॒भृत॒श्चक्षु॑ष्मती॒रिष्ट॑का॒ यत्प॑शुशी॒र्॒\mbox{}षाणि॒ यत्प॑शुशी॒र्॒\mbox{}षाण्यु॑प॒दधा॑ति॒ ताभि॑रे॒व यज॑मानो॒\-ऽमुष्मिँ॑ल्लो॒के प्राणि॒त्यथो॒ ताभि॑रे॒वास्मा॑ इ॒मे लो॒काः प्र भा᳚न्ति मृ॒दाभि॒लिप्योप॑ दधाति मेध्य॒त्वाय॑ प॒शुर्वा ए॒ष यद॒ग्निरन्नं॑ प॒शव॑ ए॒ष खलु॒ वा अ॒ग्निर्यत्प॑शुशी॒र्॒\mbox{}षाणि॒ यं का॒मये॑त॒ कनी॑यो॒\-ऽस्यान्नम्᳚~(४१)

%5.7.10.3
स्या॒दिति॑ सन्त॒रां तस्य॑ पशुशी॒र्॒\mbox{}षाण्युप॑ दध्या॒त्कनी॑य ए॒वास्यान्न॑म्भवति॒ यं का॒मये॑त स॒माव॑द॒स्यान्नꣴ॑ स्या॒दिति॑ मध्य॒तस्तस्योप॑ दध्याथ्स॒माव॑दे॒वास्यान्न॑म्भवति॒ यं का॒मये॑त॒ भूयो॒\-ऽस्यान्नꣴ॑ स्या॒दित्यन्ते॑षु॒ तस्य॑ व्यु॒दूह्योप॑ दध्यादन्त॒त ए॒वास्मा॒ अन्न॒मव॑ रुन्धे॒ भूयो॒\-ऽस्यान्न॑म्भवति॥~(४२)

{\anuvakamend[{ए॒न॒म॒स्यान्न॒म्भूयो॒स्यान्न॑म्भवति}]}%॥10॥

%5.7.11.1
स्ते॒गान्दꣴष्ट्रा᳚भ्यां म॒ण्डूका॒ञ्जम्भ्ये॑भि॒राद॑कां खा॒देनोर्जꣳ॑ सꣳसू॒देनार॑ण्यं॒ जाम्बी॑लेन॒ मृद॑म्ब॒र्स्वे॑भिः॒ शर्क॑राभि॒रव॑का॒मव॑काभिः॒ शर्क॑रामुथ्सा॒देन॑ जि॒ह्वाम॑वक्र॒न्देन॒ तालु॒ꣳ॒ सर॑स्वतीं जिह्वा॒ग्रेण॑॥~(४३)

{\anuvakamend[{स्ते॒गान्द्वाविꣳ॑शतिः}]}%॥11॥

%5.7.12.1
वाज॒ꣳ॒ हनू᳚भ्याम॒प आ॒स्ये॑नादि॒त्याञ्छ्मश्रु॑भिरुपया॒ममध॑रे॒णोष्ठे॑न॒ सदुत्त॑रे॒णान्त॑रेणानूका॒शं प्र॑का॒शेन॒ बाह्यꣴ॑ स्तनयि॒त्नुं नि॑र्बा॒धेन॑ सूर्या॒ग्नी चक्षु॑र्भ्यां वि॒द्युतौ॑ क॒नान॑काभ्याम॒शनिं॑ म॒स्तिष्के॑ण॒ बलं॑ म॒ज्जभिः॑॥~(४४)

{\anuvakamend[{वाजं॒ पञ्च॑विꣳशतिः}]}%॥12॥

%5.7.13.1
कू॒र्माञ्छ॒फैर॒च्छला॑भिः क॒पिञ्ज॑ला॒न्थ्साम॒ कुष्ठि॑काभिर्ज॒वं जङ्घा॑भिरग॒दं जानु॑भ्यां वी॒र्यं॑ कु॒हा\-भ्यां᳚ भ॒यं प्र॑चा॒लाभ्यां॒ गुहो॑पप॒क्षाभ्या॑म॒श्विना॒वꣳसा᳚भ्या॒मदि॑तिꣳ शी॒र्ष्णा निर्\mbox{}ऋ॑तिं॒ निर्जा᳚ल्मकेन शी॒र्ष्णा॥~(४५)

{\anuvakamend[{कू॒र्मान्त्रयो॑विꣳशतिः}]}%॥13॥

%5.7.14.1
योक्त्रं॒ गृध्रा॑भिर्यु॒गमान॑तेन चि॒त्तं मन्या॑भिः सङ्क्रो॒शान्प्रा॒णैः प्र॑का॒शेन॒ त्वचं॑ पराका॒शेनान्त॑रां म॒शका॒न्केशै॒रिन्द्र॒ꣴ॒ स्वप॑सा॒ वहे॑न॒ बृह॒स्पतिꣳ॑ शकुनिसा॒देन॒ रथ॑मु॒ष्णिहा॑भिः॥~(४६)

{\anuvakamend[{योक्त्र॒मेक॑विꣳशतिः}]}%॥14॥

%5.7.15.1
मि॒त्रावरु॑णौ॒ श्रोणी᳚भ्यामिन्द्रा॒ग्नी शि॑ख॒ण्डाभ्या॒मिन्द्रा॒बृह॒स्पती॑ ऊ॒रुभ्या॒मिन्द्रा॒विष्णू॑ अष्ठी॒वद्भ्याꣳ॑ सवि॒तारं॒ पुच्छे॑न गन्ध॒र्वाञ्छेपे॑नाफ्स॒रसो॑ मु॒ष्काभ्यां॒ पव॑मानं पा॒युना॑ प॒वित्रं॒ पोत्रा᳚भ्यामा॒क्रम॑णꣴ स्थू॒रा\-भ्यां᳚ प्रति॒क्रम॑णं॒ कुष्ठा᳚भ्याम्॥~(४७)

{\anuvakamend[{}]}%॥15॥

%5.7.16.1
इन्द्र॑स्य क्रो॒डो\-ऽदि॑त्यै पाज॒स्य॑न्दि॒शां ज॒त्रवो॑ जी॒मूता᳚न्हृदयौप॒शाभ्या॑म॒न्तरि॑क्षं पुरि॒तता॒ नभ॑ उद॒र्ये॑णेन्द्रा॒णीं प्ली॒ह्ना व॒ल्मीका᳚न्क्लो॒म्ना गि॒रीन्प्ला॒शिभिः॑ समु॒द्रमु॒दरे॑ण वैश्वान॒रं भस्म॑ना॥~(४८)

{\anuvakamend[{मि॒त्रावरु॑णा॒विन्द्र॑स्य॒ द्वाविꣳ॑शति॒र्द्वाविꣳ॑शतिः}]}%॥16॥

%5.7.17.1
पू॒ष्णो व॑नि॒ष्ठुर॑न्धा॒हेः स्थू॑रगु॒दा स॒र्पान्गुदा॑भिर्\mbox{}ऋ॒तून्पृ॒ष्टीभि॒र्दिवं॑ पृ॒ष्ठेन॒ वसू॑नां प्रथ॒मा कीक॑सा रु॒द्राणां᳚ द्वि॒तीया॑दि॒त्यानां᳚ तृ॒तीयाङ्गि॑रसां चतु॒र्थी सा॒ध्यानां᳚ पञ्च॒मी विश्वे॑षां दे॒वानाꣳ॑ ष॒ष्ठी॥~(४९)

{\anuvakamend[{पू॒ष्णश्चतु॑र्विꣳशतिः}]}%॥17॥

%5.7.18.1
ओजो᳚ ग्री॒वाभि॒र्निर्\mbox{}ऋ॑तिम॒स्थभि॒रिन्द्र॒ꣴ॒ स्वप॑सा॒ वहे॑न रु॒द्रस्य॑ विच॒लः स्क॒न्धो॑\-ऽहोरा॒त्रयो᳚र्द्वि॒तीयो᳚\-ऽर्धमा॒सानां᳚ तृ॒तीयो॑ मा॒सां च॑तु॒र्थ ऋ॑तू॒नां प॑ञ्च॒मः सं॑वथ्स॒रस्य॑ ष॒ष्ठः॥~(५०)

{\anuvakamend[{ओजो॑ विꣳश॒तिः}]}%॥18॥

%5.7.19.1
आ॒न॒न्दं न॒न्दथु॑ना॒ कामं॑ प्रत्या॒सा\-भ्यां᳚ भ॒यꣳ शि॑ती॒म\-भ्यां᳚ प्र॒शिषं॑ प्रशा॒साभ्याꣳ॑ सूर्याचन्द्र॒मसौ॒ वृक्या᳚भ्याꣴ श्यामशब॒लौ मत॑स्नाभ्या॒व्व्युँ॑ष्टिꣳ रू॒पेण॒ निम्रु॑क्ति॒मरू॑पेण॥~(५१)

{\anuvakamend[{आ॒न॒न्दꣳ षोड॑श}]}%॥19॥

%5.7.20.1
अह॑र्मा॒ꣳ॒सेन॒ रात्रिं॒ पीव॑सा॒पो यू॒षेण॑ घृ॒तꣳ रसे॑न॒ श्यां वस॑या दू॒षीका॑भिर्\mbox{}ह्रा॒दुनि॒मश्रु॑भिः॒ पृष्वा॒न्दिवꣳ॑ रू॒पेण॒ नक्ष॑त्राणि॒ प्रति॑रूपेण पृथि॒वीं चर्म॑णा छ॒वीं छ॒व्यो॑पाकृ॑ताय॒ स्वाहाल॑ब्धाय॒ स्वाहा॑ हु॒ताय॒ स्वाहा᳚॥~(५२)

{\anuvakamend[{अह॑र॒ष्टाविꣳ॑शतिः}]}%॥20॥

%5.7.21.1
अ॒ग्नेः प॑क्ष॒तिः सर॑स्वत्यै॒ निप॑क्षतिः॒ सोम॑स्य तृ॒तीया॒पां च॑तु॒र्थ्योष॑धीनां पञ्च॒मी सं॑वथ्स॒रस्य॑ ष॒ष्ठी म॒रुताꣳ॑ सप्त॒मी बृह॒स्पते॑रष्ट॒मी मि॒त्रस्य॑ नव॒मी वरु॑णस्य दश॒मीन्द्र॑स्यैकाद॒शी विश्वे॑षां दे॒वानां᳚ द्वाद॒शी द्यावा॑पृथि॒व्योः पा॒र्श्वं य॒मस्य॑ पाटू॒रः॥~(५३)

{\anuvakamend[{अ॒ग्नेरेका॒न्नत्रि॒ꣳ॒शत्}]}%॥21॥

%5.7.22.1
वा॒योः प॑क्ष॒तिः सर॑स्वतो॒ निप॑क्षतिश्च॒न्द्रम॑सस्तृ॒तीया॒ नक्ष॑त्राणां चतु॒र्थी स॑वि॒तुः प॑ञ्च॒मी रु॒द्रस्य॑ ष॒ष्ठी स॒र्पाणाꣳ॑ सप्त॒म्य॑र्य॒म्णो᳚\-ऽष्ट॒मी त्वष्टु॑र्नव॒मी धा॒तुर्द॑श॒मीन्द्रा॒ण्या ए॑काद॒श्यदि॑त्यै द्वाद॒शी द्यावा॑पृथि॒व्योः पा॒र्श्वं य॒म्यै॑ पाटू॒रः॥~(५४)

{\anuvakamend[{वा॒योर॒ष्टाविꣳ॑शतिः}]}%॥22॥

%5.7.23.1
पन्था॑मनू॒वृग्भ्या॒ꣳ॒ सन्त॑तिꣴ स्नाव॒न्या᳚भ्या॒ꣳ॒ शुका᳚न्पि॒त्तेन॑ हरि॒माणं॑ य॒क्ना हली᳚क्ष्णान्पापवा॒तेन॑ कू॒श्माञ्छक॑भिः शव॒र्तानूव॑ध्येन॒ शुनो॑ वि॒शस॑नेन स॒र्पाल्लोँ॑हितग॒न्धेन॒ वयाꣳ॑सि पक्वग॒न्धेन॑ पि॒पीलि॑काः प्रशा॒देन॑॥~(५)

{\anuvakamend[{पन्था॒न्द्वाविꣳ॑शतिः}]}%॥23॥

%5.7.24.1
क्रमै॒रत्य॑क्रमीद्वा॒जी विश्वै᳚र्दे॒वैर्य॒ज्ञियैः᳚ संविदा॒नः। स नो॑ नय सुकृ॒तस्य॑ लो॒कं तस्य॑ ते व॒यꣴ स्व॒धया॑ मदेम॥~(५६)

{\anuvakamend[{क्रमै॑र॒ष्टाद॑श}]}%॥24॥

%5.7.25.1
द्यौस्ते॑ पृ॒ष्ठं पृ॑थि॒वी स॒धस्थ॑मा॒त्माऽन्तरि॑क्षꣳ समु॒द्रो योनिः॒ सूर्य॑स्ते॒ चक्षु॒र्वातः॑ प्रा॒णश्च॒न्द्रमाः॒ श्रोत्रं॒ मासा᳚श्चार्धमा॒साश्च॒ पर्वा᳚ण्यृ॒तवोङ्गा॑नि संवथ्स॒रो म॑हि॒मा॥~(५७)

{\anuvakamend[{द्यौः पञ्च॑विꣳशतिः}]}%॥25॥

%5.7.26.1
अ॒ग्निः प॒शुरा॑सी॒त्तेना॑यजन्त॒ स ए॒तं लो॒कम॑जय॒द्यस्मि॑न्न॒ग्निः स ते॑ लो॒कस्तं जे᳚ष्य॒स्यथाव॑ जिघ्र वा॒युः प॒शुरा॑सी॒त्तेना॑यजन्त॒ स ए॒तं लो॒कम॑जय॒द्यस्मि॑न्वा॒युः स ते॑ लो॒कस्तस्मा᳚त्त्वा॒न्तरे᳚ष्यामि॒ यदि॒ नाव॒जिघ्र॑स्यादि॒त्यः प॒शुरा॑सी॒त्तेना॑यजन्त॒ स ए॒तं लो॒कम॑जय॒द्यस्मि॑न्नादि॒त्यः स ते॑ लो॒कस्तं जे᳚ष्यसि॒ यद्य॑व॒जिघ्र॑सि॥~(५८)

{\anuvakamend[{यस्मि॑न्न॒ष्टौ च॑}]}%॥26॥

\prashnaend{प्रा॒चीन॑वꣳशं॒ याव॑न्त ऋख्सा॒मे वाग्वै दे॒वेभ्यो॑ दे॒वा वै दे॑व॒यज॑नङ्क॒द्रूश्च॒ तद्धिर॑ण्य॒ꣳ॒ षट्प॒दानि॑ ब्रह्मवा॒दिनो॑ वि॒चित्यो॒ यत्क॒लया॑ ते वारु॒णो वै क्री॒तः सोम॒ एका॑\-दश॥११॥}{प्रा॒चीन॑वꣳश॒ꣴ॒ स्वाहेत्या॑ह॒ ये᳚\-ऽन्तः श॒रा ह्ये॑ष सं तप॑सा च॒ यत्क॑र्णगृही॒तेति॑ लोम॒तो वा॑रु॒णः षट्थ्स॑प्ततिः॥७६॥}{प्रा॒चीन॑वꣳशं॒ परि॑चरति॥}%%५-७
{हरिः॑ ॐ}{॥कृष्ण-यजुर्वेदीय-तैत्तिरीय-संहितायां पञ्चम्काण्डे सप्तमः प्रश्नः समाप्तः॥५-७॥}
%%% END PRASHNA

%%% END KANDAM
