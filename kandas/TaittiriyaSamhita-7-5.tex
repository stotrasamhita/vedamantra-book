\chapt{काण्डम् ७}
\sect{पञ्चमः प्रश्नः}\setcounter{anuvakam}{0}
\dnsub{तैत्तिरीयसंहितायां सप्तमकाण्डे पञ्चमः प्रश्नः}
%7.5.1.1
गावो॒ वा ए॒तथ्स॒त्रमा॑सताशृ॒ङ्गाः स॒तीः शृङ्गा॑णि नो जायन्ता॒ इति॒ कामे॑न॑ तासां॒ दश॒ मासा॒ निष॑ण्णा॒ आस॒न्नथ॒ शृङ्गा᳚ण्यजायन्त॒ ता उद॑तिष्ठ॒न्नरा॒थ्स्मेत्यथ॒ यासां॒ नाजा॑यन्त॒ ताः सं॑वथ्स॒रमा॒प्त्वोद॑तिष्ठ॒न्नरा॒थ्स्मेति॒ यासां॒ चाजा॑यन्त॒ यासां᳚ च॒ न ता उ॒भयी॒रुद॑तिष्ठ॒न्नरा॒थ्स्मेति॑ गोस॒त्रं वै~(१)

%7.5.1.2
सं॒व॒थ्स॒रो य ए॒वं वि॒द्वाꣳसः॑ संवथ्स॒रमु॑प॒यन्त्यृ॑ध्नु॒वन्त्ये॒व तस्मा᳚त्तूप॒रा वार्\mbox{}षि॑कौ॒ मासौ॒ पर्त्वा॑ चरति स॒त्राभि॑जित॒ꣴ॒ ह्य॑स्यै॒ तस्मा᳚थ्संवथ्सर॒सदो॒ यत्किं च॑ गृ॒हे क्रि॒यते॒ तदा॒प्तमव॑रुद्धम॒भिजि॑तं क्रियते समु॒द्रं वा ए॒ते प्र प्ल॑वन्ते॒ ये सं॑वथ्स॒रमु॑प॒यन्ति॒ यो वै स॑मु॒द्रस्य॒ पारं॒ न पश्य॑ति॒ न वै स तत॒ उदे॑ति संवथ्स॒रः~(२)

%7.5.1.3
वै स॑मु॒द्रस्तस्यै॒तत्पा॒रं यद॑तिरा॒त्रौ य ए॒वं वि॒द्वाꣳसः॑ संवथ्स॒रमु॑प॒यन्त्यना᳚र्ता ए॒वोदृचं॑ गच्छन्ती॒यं वै पूर्वो॑\-ऽतिरा॒त्रो॑\-ऽ\-सावुत्त॑रो॒ मनः॒ पूर्वो॒ वागुत्त॑रः प्रा॒णः पूर्वो॑\-ऽपा॒न उत्त॑रः प्र॒रोध॑नं॒ पूर्व॑ उ॒दय॑न॒मुत्त॑रो॒ ज्योति॑ष्टोमो वैश्वान॒रो॑\-ऽतिरा॒त्रो भ॑वति॒ ज्योति॑रे॒व पु॒रस्ता᳚द्दधते सुव॒र्गस्य॑ लो॒कस्यानु॑ख्यात्यै चतुर्वि॒ꣳ॒शः प्रा॑य॒णीयो॑ भवति चतु॑र्विꣳशतिरर्धमा॒साः~(३)

%7.5.1.4
सं॒व॒थ्स॒रः प्र॒यन्त॑ ए॒व सं॑वथ्स॒रे प्रति॑ तिष्ठन्ति॒ तस्य॒ त्रीणि॑ च श॒तानि॑ ष॒ष्टिश्च॑ स्तो॒त्रीया॒स्ताव॑तीः संवथ्स॒रस्य॒ रात्र॑य उ॒भे ए॒व सं॑वथ्स॒रस्य॑ रू॒पे आ᳚प्नुवन्ति॒ ते सꣴस्थि॑त्या॒ अरि॑ष्ट्या॒ उत्त॑रै॒रहो॑भिश्चरन्ति षड॒हा भ॑वन्ति॒ षड्वा ऋ॒तवः॑ संवथ्स॒र ऋ॒तुष्वे॒व सं॑वथ्स॒रे प्रति॑ तिष्ठन्ति॒ गौश्चायु॑श्च मध्य॒तः स्तोमौ॑ भवतः संवथ्स॒रस्यै॒व तन्मि॑थु॒नं म॑ध्य॒तः~(४)

%7.5.1.5
द॒ध॒ति॒ प्र॒जन॑नाय॒ ज्योति॑र॒भितो॑ भवति वि॒मोच॑नमे॒व तच्छन्दाꣴ॑स्ये॒व तद्वि॒मोकं॑ य॒न्त्यथो॑ उभ॒यतो᳚ज्योतिषै॒व ष॑ड॒हेन॑ सुव॒र्गं लो॒कं य॑न्ति ब्रह्मवा॒दिनो॑ वद॒न्त्यास॑ते॒ केन॑ य॒न्तीति॑ देव॒याने॑न प॒थेति॑ ब्रूया॒च्छन्दाꣳ॑सि॒ वै दे॑व॒यानः॒ पन्था॑ गाय॒त्री त्रि॒ष्टुब्जग॑ती॒ ज्योति॒र्वै गा॑य॒त्री गौस्त्रि॒ष्टुगायु॒र्जग॑ती यदे॒ते स्तोमा॒ भव॑न्ति देव॒याने॑नै॒व~(५)

%7.5.1.6
तत्प॒था य॑न्ति समा॒नꣳ साम॑ भवति देवलो॒को वै साम॑ देवलो॒कादे॒व न य॑न्त्य॒न्याअ॑न्या॒ ऋचो॑ भवन्ति मनुष्यलो॒को वा ऋचो॑ मनुष्यलो॒कादे॒वान्यम॑न्यं देवलो॒कम॑भ्या॒रोह॑न्तो यन्त्यभिव॒र्तो ब्र॑ह्मसा॒मं भ॑वति सुव॒र्गस्य॑ लो॒कस्या॒भिवृ॑त्त्या अभि॒जिद्भ॑वति सुव॒र्गस्य॑ लो॒कस्या॒भिजि॑त्यै विश्व॒जिद्भ॑वति॒ विश्व॑स्य॒ जित्यै॑ मा॒सिमा॑सि पृ॒ष्ठान्युप॑ यन्ति मा॒सिमा᳚स्यतिग्रा॒ह्या॑ गृह्यन्ते मा॒सिमा᳚स्ये॒व वी॒र्यं॑ दधति मा॒सां प्रति॑ष्ठित्या उ॒परि॑ष्टान्मा॒सां पृ॒ष्ठान्युप॑ यन्ति॒ तस्मा॑दु॒परि॑ष्टा॒दोष॑धयः॒ फलं॑ गृह्णन्ति॥~(६)

{\anuvakamend[{गो॒स॒त्रं वा ए॑ति संवथ्स॒रो᳚\-ऽर्धमा॒सा मि॑थु॒नं म॑ध्य॒तो दे॑व॒याने॑नै॒व वी॒र्य॑न्त्रयो॑दश च}]}%~(१)

%7.5.2.1
गावो॒ वा ए॒तथ्स॒त्रमा॑सताशृ॒ङ्गाः स॒तीः शृ॑ङ्गाणि॒ सिषा॑सन्ती॒स्तासां॒ दश॒ मासा॒ निष॑ण्णा॒ आस॒न्नथ॒ शृङ्गा᳚ण्यजायन्त॒ ता अ॑ब्रुव॒न्नरा॒थ्स्मोत्ति॑ष्ठा॒माव॒ तं काम॑मरुथ्स्महि॒ येन॒ कामे॑न॒ न्यष॑दा॒मेति॒ तासा॑मु॒ त्वा अ॑ब्रुवन्न॒र्धा वा॒ याव॑ती॒र्वासा॑महा ए॒वेमौ द्वा॑द॒शौ मासौ॑ संवथ्स॒रꣳ स॒म्पाद्योत्ति॑ष्ठा॒मेति॒ तासा᳚म्~(७)

%7.5.2.2
द्वा॒द॒शे मा॒सि शृङ्गा॑णि॒ प्राव॑र्तन्त श्र॒द्धया॒ वाश्र॑द्धया वा॒ ता इ॒मा यास्तू॑प॒रा उ॒भय्यो॒ वाव ता आ᳚र्ध्नुव॒न्॒ याश्च॒ शृङ्गा॒ण्यस॑न्व॒न्॒ याश्चोर्ज॑म॒वारु॑न्धत॒र्ध्नोति॑ द॒शसु॑ मा॒सू᳚त्तिष्ठ॑न्नृ॒ध्नोति॑ द्वाद॒शसु॒ य ए॒वं वेद॑ प॒देन॒ खलु॒ वा ए॒ते य॑न्ति वि॒न्दति॒ खलु॒ वै प॒देन॒ यन्तद्वा ए॒तदृ॒द्धमय॑न॒न्तस्मा॑दे॒तद्गो॒सनि॑॥~(८)

{\anuvakamend[{ति॒ष्ठा॒मेति॒ तासा॒न्तस्मा॒द्द्वे च॑}]}%~(२)

%7.5.3.1
प्र॒थ॒मे मा॒सि पृ॒ष्ठान्युप॑ यन्ति मध्य॒म उप॑ यन्त्युत्त॒म उप॑ यन्ति॒ तदा॑हु॒र्यां वै त्रिरेक॒स्याह्न॑ उप॒सीद॑न्ति द॒ह्रं वै साप॑राभ्यां॒ दोहा᳚भ्यां दु॒हे\-ऽथ॒ कुतः॒ सा धो᳚क्ष्यते॒ यां द्वाद॑श॒ कृत्व॑ उप॒सीद॒न्तीति॑ संवथ्स॒रꣳ स॒म्पाद्यो᳚त्त॒मे मा॒सि स॒कृत्पृ॒ष्ठान्युपे॑यु॒स्तद्यज॑माना य॒ज्ञं प॒शूनव॑ रुन्धते समु॒द्रं वै~(९)

%7.5.3.2
ए॒ते॑\-ऽनवा॒रम॑पा॒रं प्र प्ल॑वन्ते॒ ये सं॑वथ्स॒रमु॑प॒यन्ति॒ यद्बृ॑हद्रथन्त॒रे अ॒न्वर्जे॑यु॒र्यथा॒ मध्ये॑ समु॒द्रस्य॑ प्ल॒वम॒न्वर्जे॑युस्ता॒दृक्त\-दनु॑थ्सर्गं बृहद्रथन्त॒राभ्या॑मि॒त्वा प्र॑ति॒ष्ठां ग॑च्छन्ति॒ सर्वे᳚भ्यो॒ वै कामे᳚भ्यः स॒न्धिर्दु॑हे॒ तद्यज॑मानाः॒ सर्वा॒न्कामा॒नव॑ रुन्धते॥~(१०)

{\anuvakamend[{स॒मु॒द्रं वै चतु॑स्त्रिꣳशच्च}]}%~(३)

%7.5.4.1
स॒मा॒न्य॑ ऋचो॑ भवन्ति मनुष्यलो॒को वा ऋचो॑ मनुष्यलो॒कादे॒व न य॑न्त्य॒न्यद॑न्य॒थ्साम॑ भवति देवलो॒को वै साम॑ देवलो॒कादे॒वान्यम॑न्यं मनुष्यलो॒कं प्र॑त्यव॒रोह॑न्तो यन्ति॒ जग॑ती॒मग्र॒ उप॑ यन्ति॒ जग॑तीं॒ वै छन्दाꣳ॑सि प्र॒त्यव॑रोहन्त्याग्रय॒णं ग्रहा॑ बृ॒हत्पृ॒ष्ठानि॑ त्रयस्त्रि॒ꣳ॒शꣴ स्तोमा॒स्तस्मा॒ज्ज्यायाꣳ॑सं॒ कनी॑यान्प्र॒त्यव॑रोहति वैश्वकर्म॒णो गृ॑ह्यते॒ विश्वा᳚न्ये॒व तेन॒ कर्मा॑णि॒ यज॑माना॒ अव॑ रुन्धत आदि॒त्यः~(११)

%7.5.4.2
गृ॒ह्य॒त॒ इ॒यं वा अदि॑तिर॒स्यामे॒व प्रति॑ तिष्ठन्त्य॒न्यो᳚न्यो गृह्येते मिथुन॒त्वाय॒ प्रजा᳚त्या अवान्त॒रं वै द॑शरा॒त्रेण॑ प्र॒जा\-प॑तिः प्र॒जा अ॑सृजत॒ यद्द॑शरा॒त्रो भव॑ति प्र॒जा ए॒व तद्यज॑मानाः सृजन्त ए॒ताꣳ ह॒ वा उ॑द॒ङ्कः शौ᳚ल्बाय॒नः स॒त्रस्यर्द्धि॑मुवाच॒ यद्द॑शरा॒त्रो यद्द॑शरा॒त्रो भव॑ति स॒त्रस्यर्द्ध्या॒ अथो॒ यदे॒व पूर्वे॒ष्वहः॑सु॒ विलो॑म क्रि॒यते॒ तस्यै॒वैषा शान्तिः॑॥~(१२)

{\anuvakamend[{आ॒दि॒त्यस्तस्यै॒व द्वे च॑}]}%~(४)

%7.5.5.1
यदि॒ सोमौ॒ सꣳसु॑तौ॒ स्यातां᳚ मह॒ति रात्रि॑यै प्रातरनुवा॒कमु॒पाकु॑र्या॒त्पूर्वो॒ वाचं॒ पूर्वो॑ दे॒वताः॒ पूर्व॒श्छन्दाꣳ॑सि वृङ्क्ते॒ वृष॑ण्वतीं प्रति॒पदं॑ कुर्यात्प्रातःसव॒नादे॒वैषा॒मिन्द्रं॑ वृ॒ङ्क्ते\-ऽथो॒ खल्वा॑हुः सवनमु॒खेस॑वनमुखे का॒र्येति॑ सवनमु॒खाथ्स॑वनमुखादे॒वैषा॒मिन्द्रं॑ वृङ्क्ते संवे॒शायो॑पवे॒शाय॑ गायत्रि॒यास्त्रि॒ष्टुभो॒ जग॑त्या अनु॒ष्टुभः॑ प॒ङ्क्त्या अ॒भिभू᳚त्यै॒ स्वाहा॒ छन्दाꣳ॑सि॒ वै सं॑वे॒श उ॑पवे॒शश्छन्दो॑भिरे॒वैषा᳚म्~(१३)

%7.5.5.2
छन्दाꣳ॑सि वृङ्क्ते सज॒नीय॒ꣳ॒ शस्यं॑ विह॒व्यꣳ॑ शस्य॑म॒गस्त्य॑स्य कयाशु॒भीय॒ꣳ॒ शस्य॑मे॒ताव॒द्वा अ॑स्ति॒ याव॑दे॒तद्याव॑दे॒वास्ति॒ तदे॑षां वृङ्क्ते॒ यदि॑ प्रातःसव॒ने क॒लशो॒ दीर्ये॑त वैष्ण॒वीषु॑ शिपिवि॒ष्टव॑तीषु स्तुवीर॒न्॒ यद्वै य॒ज्ञस्या॑ति॒रिच्य॑ते॒ विष्णुं॒ तच्छि॑पिवि॒ष्टम॒भ्यति॑ रिच्यते॒ तद्विष्णुः॑ शिविपि॒ष्टो\-ऽति॑रिक्त ए॒वाति॑रिक्तं दधा॒त्यथो॒ अति॑रिक्तेनै॒वाति॑रिक्तमा॒प्त्वाव॑ रुन्धते॒ यदि॑ म॒ध्यन्दि॑ने॒ दीर्ये॑त वषट्का॒रनि॑धन॒ꣳ॒ साम॑ कुर्युर्वषट्का॒रो वै य॒ज्ञस्य॑ प्रति॒ष्ठा प्र॑ति॒ष्ठामे॒वैन॑द्गमयन्ति॒ यदि॑ तृतीयसव॒न ए॒तदे॒व॥~(१४)


{\anuvakamend[{छन्दो॑भिरे॒वैषा॒मवैका॒न्नविꣳ॑श॒तिश्च॑}]}%~(५)

%7.5.6.1
ष॒ड॒हैर्मासा᳚न्थ्स॒म्पाद्याह॒रुथ्सृ॑जन्ति षड॒हैर्\mbox{}हि मासा᳚न्थ्स॒म्पश्य॑न्त्यर्धमा॒सैर्मासा᳚न्थ्स॒म्पाद्याह॒रुथ्सृ॑जन्त्यर्धमा॒सैर्\mbox{}हि मासा᳚न्थ्स॒म्पश्य॑न्त्यमावा॒स्य॑या॒ मासा᳚न्थ्स॒म्पाद्याह॒रुथ्सृ॑जन्त्यमावा॒स्य॑या॒ हि मासा᳚न्थ्स॒म्पश्य॑न्ति पौर्णमा॒स्या मासा᳚न्थ्स॒म्पाद्याह॒रुथ्सृ॑जन्ति पौर्णमा॒स्या हि मासा᳚न्थ्स॒म्पश्य॑न्ति॒ यो वै पू॒र्ण आ॑सि॒ञ्चति॒ परा॒ स सि॑ञ्चति॒ यः पू॒र्णादु॒दच॑ति~(१५)

%7.5.6.2
प्रा॒णम॑स्मि॒न्थ्स द॑धाति॒ यत्पौ᳚र्णमा॒स्या मासा᳚न्थ्स॒म्पाद्याह॑रुथ्सृ॒जन्ति॑ संवथ्स॒रायै॒व तत्प्रा॒णं द॑धति॒ तदनु॑ स॒त्रिणः॒ प्राण॑न्ति॒ यदह॒र्नोथ्सृ॒जेयु॒र्यथा॒ दृति॒रुप॑नद्धो वि॒पत॑त्ये॒वꣳ सं॑वथ्स॒रो वि प॑ते॒दार्ति॒मार्च्छे॑यु॒र्यत्पौ᳚र्णमा॒स्या मासा᳚न्थ्स॒म्पाद्याह॑रुथ्सृ॒जन्ति॑ संवथ्स॒रायै॒व तदु॑दा॒नं द॑धति॒ तदनु॑ स॒त्रिण॒ उत्~(१६)

%7.5.6.3
अ॒न॒न्ति॒ नार्ति॒मार्च्छ॑न्ति पू॒र्णमा॑से॒ वै दे॒वानाꣳ॑ सु॒तो यत्पौ᳚र्णमा॒स्या मासा᳚न्थ्स॒म्पाद्याह॑रुथ्सृ॒जन्ति॑ दे॒वाना॑मे॒व तद्य॒ज्ञेन॑ य॒ज्ञं प्र॒त्यव॑रोहन्ति॒ वि वा ए॒तद्य॒ज्ञं छि॑न्दन्ति॒ यत्ष॑ड॒हस॑न्तत॒ꣳ॒ सन्त॒मथाह॑रुथ्सृ॒जन्ति॑ प्राजाप॒त्यं प॒शुमाल॑भन्ते प्र॒जा\-प॑तिः॒ सर्वा॑ दे॒वता॑ दे॒वता॑भिरे॒व य॒ज्ञꣳ सं त॑न्वन्ति॒ यन्ति॒ वा ए॒ते सव॑ना॒द्ये\-ऽहः॑~(१७)

%7.5.6.4
उ॒थ्सृ॒जन्ति॑ तु॒रीयं॒ खलु॒ वा ए॒तथ्सव॑नं॒ यथ्सा᳚न्ना॒य्यं यथ्सा᳚न्ना॒य्यं भव॑ति॒ तेनै॒व सव॑ना॒न्न य॑न्ति समुप॒हूय॑ भक्षयन्त्ये॒तथ्सो॑मपीथा॒ ह्ये॑तर्\mbox{}हि॑ यथायत॒नं वा ए॒तेषाꣳ॑ सवन॒भाजो॑ दे॒वता॑ गच्छन्ति॒ ये\-ऽह॑रुथ्सृ॒जन्त्य॑नुसव॒नं पु॑रो॒डाशा॒न्निर्व॑पन्ति यथायत॒नादे॒व स॑वन॒भाजो॑ दे॒वता॒ अव॑ रुन्धते॒\-ऽष्टाक॑पालान्प्रातःसव॒न एका॑\-दश\-कपाला॒\-न्माध्य॑न्दिने॒ सव॑ने॒ द्वाद॑श\-कपालाꣴस्तृतीयसव॒ने छन्दाꣴ॑स्ये॒वाप्त्वाव॑ रुन्धते वैश्वदे॒वं च॒रुं तृ॑तीयसव॒ने निर्व॑पन्ति वैश्वदे॒वं वै तृ॑तीयसव॒नन्तेनै॒व तृ॑तीयसव॒नान्न य॑न्ति॥~(१८)

{\anuvakamend[{उ॒दच॒त्युद्ये\-ऽह॑रा॒प्त्वा पञ्च॑दश च}]}%~(६)

%7.5.7.1
उ॒थ्सृज्या~(३) न्नोथ्सृज्या~(३) मिति॑ मीमाꣳसन्ते ब्रह्मवा॒दिन॒स्तद्वा॑हुरु॒थ्सृज्य॑मे॒वेत्य॑मावा॒स्या॑यां च पौर्णमा॒स्यां चो॒थ्सृज्य॒मित्या॑हुरे॒ते हि य॒ज्ञं वह॑त॒ इति॒ ते त्वाव नोथ्सृज्ये॒ इत्या॑हु॒र्ये अ॑वान्त॒रं य॒ज्ञं भे॒जाते॒ इति॒ या प्र॑थ॒मा व्य॑ष्टका॒ तस्या॑मु॒थ्सृज्य॒मित्या॑हुरे॒ष वै मा॒सो वि॑श॒र इति॒ नादि॑ष्टम्~(१९)

%7.5.7.2
उथ्सृ॑जेयु॒र्यदादि॑ष्टमुथ्सृ॒जेयु॑र्या॒दृशे॒ पुनः॑ पर्याप्ला॒वे मध्ये॑ षड॒हस्य॑ स॒म्पद्ये॑त षड॒हैर्मासा᳚न्थ्स॒म्पाद्य॒ यथ्स॑प्त॒ममह॒\-स्तस्मि॒न्नुथ्सृ॑ज्येयु॒स्तद॒ग्नये॒ वसु॑मते पुरो॒डाश॑\-म॒ष्टा\-क॑पालं॒ निर्व॑पेयुरै॒न्द्रं दधीन्द्रा॑य म॒रुत्व॑ते पुरो॒डाश॒मेका॑\-दश\-कपालं वैश्वदे॒वं द्वाद॑श\-कपालम॒ग्नेर्वै वसु॑मतः प्रातःसव॒नं यद॒ग्नये॒ वसु॑मते पुरो॒डाश॑\-म॒ष्टाक॑पालं नि॒र्वप॑न्ति दे॒वता॑मे॒व तद्भा॒गिनीं᳚ कु॒र्वन्ति॑~(२०)

%7.5.7.3
सव॑नमष्टा॒भिरुप॑ यन्ति॒ यदैन्द्रं दधि॒ भव॒तीन्द्र॑मे॒व तद्भा॑ग॒धेया॒न्न च्या॑वय॒न्तीन्द्र॑स्य॒ वै म॒रुत्व॑तो॒ माध्य॑न्दिन॒ꣳ॒ सव॑नं॒ यदिन्द्रा॑य म॒रुत्व॑ते पुरो॒डाश॒मेका॑\-दश\-कपालं नि॒र्वप॑न्ति दे॒वता॑मे॒व तद्भा॒गिनीं᳚ कु॒र्वन्ति॒ सव॑नमेकाद॒शभि॒रुप॑ यन्ति॒ विश्वे॑षां॒ वै दे॒वाना॑मृभु॒मतां᳚ तृतीयसव॒नं यद्वै᳚श्वदे॒वं द्वाद॑श\-कपालं नि॒र्वप॑न्ति दे॒वता॑ ए॒व तद्भा॒गिनीः᳚ कु॒र्वन्ति॒ सव॑नं द्वाद॒शभिः॑~(२१)

%7.5.7.4
उप॑ यन्ति प्राजाप॒त्यं प॒शुमा ल॑भन्ते य॒ज्ञो वै प्र॒जा\-प॑तिर्य॒ज्ञस्यान॑नुसर्गायाभिव॒र्त इ॒तः षण्मा॒सो ब्र॑ह्मसा॒मं भ॑वति॒ ब्रह्म॒ वा अ॑भिव॒र्तो ब्रह्म॑णै॒व तथ्सु॑व॒र्गं लो॒कम॑भिव॒र्तय॑न्तो यन्ति प्रतिकू॒लमि॑व॒ हीतः सु॑व॒र्गो लो॒क इन्द्र॒ क्रतुं॑ न॒ आ भ॑र पि॒ता पु॒त्रेभ्यो॒ यथा᳚। शिक्षा॑ नो अ॒स्मिन्पु॑रुहूत॒ याम॑नि जी॒वा ज्योति॑रशीम॒हीत्य॒मुत॑ आय॒ताꣳ षण्मा॒सो ब्र॑ह्मसा॒मं भ॑वत्य॒यं वै लो॒को ज्योतिः॑ प्र॒जा ज्योति॑रि॒ममे॒व तल्लो॒कं पश्य॑न्तो\-ऽभि॒वद॑न्त॒ आ य॑न्ति॥~(२२)

{\anuvakamend[{नादि॑ष्टङ्कु॒र्वन्ति॑ द्वाद॒शभि॒रिति॑ विꣳश॒तिश्च॑}]}%~(७)

%7.5.8.1
दे॒वानां॒ वा अन्तं॑ ज॒ग्मुषा॑मिन्द्रि॒यं वी॒र्य॑मपा᳚क्राम॒त्तत्क्रो॒शेनावा॑ रुन्धत॒ तत्क्रो॒शस्य॑ क्रोश॒त्वं यत्क्रो॒शेन॒ चात्वा॑ल॒स्यान्ते᳚ स्तु॒वन्ति॑ य॒ज्ञस्यै॒वान्तं॑ ग॒त्वेन्द्रि॒यं वी॒र्य॑मव॑ रुन्धते स॒त्रस्यर्द्ध्या॑हव॒नीय॒स्यान्ते᳚ स्तुवन्त्य॒ग्निमे॒वोप॑द्र॒ष्टारं॑ कृ॒त्वर्द्धि॒मुप॑ यन्ति प्र॒जा\-प॑ते॒र्॒\mbox{}हृद॑येन हवि॒र्धाने॒\-ऽन्तः स्तु॑वन्ति प्रे॒माण॑मे॒वास्य॑ गच्छन्ति श्लो॒केन॑ पु॒रस्ता॒थ्सद॑सः~(२३)

%7.5.8.2
स्तु॒व॒न्त्यनु॑श्लोकेन प॒श्चाद्य॒ज्ञस्यै॒वान्तं॑ ग॒त्वा श्लो॑क॒भाजो॑ भवन्ति न॒वभि॑रध्व॒र्युरुद्गा॑यति॒ नव॒ वै पुरु॑षे प्रा॒णाः प्रा॒णाने॒व यज॑मानेषु दधाति॒ सर्वा॑ ऐ॒न्द्रियो॑ भवन्ति प्रा॒णेष्वे॒वेन्द्रि॒यं द॑ध॒त्यप्र॑तिहृताभि॒रुद्गा॑यति॒ तस्मा॒त्पुरु॑षः॒ सर्वा᳚ण्य॒न्यानि॑ शी॒र्ष्णो\-ऽङ्गा॑नि॒ प्रत्य॑चति॒ शिर॑ ए॒व न पञ्च॑द॒शꣳ र॑थन्त॒रं भ॑वतीन्द्रि॒यमे॒वाव॑ रुन्धते सप्तद॒शम्~(२४)

%7.5.8.3
बृ॒हद॒न्नाद्य॒स्याव॑रुद्ध्या॒ अथो॒ प्रैव तेन॑ जायन्त एकवि॒ꣳ॒शं भ॒द्रं द्वि॒पदा॑सु॒ प्रति॑ष्ठित्यै॒ पत्न॑य॒ उप॑ गायन्ति मिथुन॒त्वाय॒ प्रजा᳚त्यै प्र॒जा\-प॑तिः प्र॒जा अ॑सृजत॒ सो॑\-ऽकामयता॒साम॒हꣳ रा॒ज्यं परी॑या॒मिति॒ तासाꣳ॑ राज॒नेनै॒व रा॒ज्यं पर्यै॒त्तद्रा॑ज॒नस्य॑ राजन॒त्वं यद्रा॑ज॒नं भव॑ति प्र॒जाना॑मे॒व तद्यज॑माना रा॒ज्यं परि॑ यन्ति पञ्चवि॒ꣳ॒शं भ॑वति प्र॒जा\-प॑तेः~(२५)

%7.5.8.4
आप्त्यै॑ प॒ञ्चभि॒स्तिष्ठ॑न्तः स्तुवन्ति देवलो॒कमे॒वाभि ज॑यन्ति प॒ञ्चभि॒रासी॑ना मनुष्यलो॒कमे॒वाभि ज॑यन्ति॒ दश॒ सं प॑द्यन्ते॒ दशा᳚क्षरा वि॒राडन्नं॑ वि॒राजै॒वान्नाद्य॒मव॑ रुन्धते पञ्च॒धा वि॑नि॒षद्य॑ स्तुवन्ति॒ पञ्च॒ दिशो॑ दि॒क्ष्वे॑व प्रति॑ तिष्ठ॒न्त्येकै॑क॒यास्तु॑तया स॒माय॑न्ति दि॒ग्भ्य ए॒वान्नाद्य॒ꣳ॒ सम्भ॑रन्ति॒ ताभि॑रुद्गा॒तोद्गा॑यति दि॒ग्भ्य ए॒वान्नाद्यम्᳚~(२६)

%7.5.8.5
स॒म्भृत्य॒ तेज॑ आ॒त्मन्द॑धते॒ तस्मा॒देकः॑ प्रा॒णः सर्वा॒ण्यङ्गा᳚न्यव॒त्यथो॒ यथा॑ सुप॒र्ण उ॑त्पति॒ष्यञ्छिर॑ उत्त॒मं कु॑रु॒त ए॒वमे॒व तद्यज॑मानाः प्र॒जाना॑मुत्त॒मा भ॑वन्त्यास॒न्दीमु॑द्गा॒ता रो॑हति॒ साम्रा᳚ज्यमे॒व ग॑च्छन्ति प्ले॒ङ्खꣳ होता॒ नाक॑स्यै॒व पृ॒ष्ठꣳ रो॑हन्ति कू॒र्चाव॑ध्व॒र्युर्ब्र॒ध्नस्यै॒व वि॒ष्टपं॑ गच्छन्त्ये॒ताव॑न्तो॒ वै दे॑वलो॒कास्तेष्वे॒व य॑थापू॒र्वं प्रति॑ तिष्ठ॒न्त्यथो॑ आ॒क्रम॑णमे॒व तथ्सेतुं॒ यज॑मानाः कुर्वते सुव॒र्गस्य॑ लो॒कस्य॒ सम॑ष्ट्यै॥~(२७)

{\anuvakamend[{सद॑सः सप्तद॒शं प्र॒जा\-प॑तेर्गायति दि॒ग्भ्य ए॒वान्नाद्यं॒ प्रत्येका॑\-दश च}]}%~(८)

%7.5.9.1
अ॒र्क्ये॑ण॒ वै स॑हस्र॒शः प्र॒जा\-प॑तिः प्र॒जा अ॑सृजत॒ ताभ्य॒ इला᳚न्दे॒नेरां॒ लूता॒मवा॑रुन्ध॒ यद॒र्क्य॑म्भव॑ति प्र॒जा ए॒व तद्यज॑मानाः सृजन्त॒ इला᳚न्दं भवति प्र॒जाभ्य॑ ए॒व सृ॒ष्टाभ्य॒ इरां॒ लूता॒मव॑ रुन्धते॒ तस्मा॒द्याꣳ समाꣳ॑ स॒त्रꣳ समृ॑द्धं॒ क्षोधु॑का॒स्ताꣳ समां᳚ प्र॒जा इष॒ꣴ॒ ह्या॑सा॒मूर्ज॑मा॒दद॑ते॒ याꣳ समां॒ व्यृ॑द्ध॒मक्षो॑धुका॒स्ताꣳ समां᳚ प्र॒जाः~(२८)

%7.5.9.2
न ह्या॑सा॒मिष॒मूर्ज॑मा॒दद॑त उत्क्रो॒दं कु॑र्वते॒ यथा॑ ब॒न्धान्मु॑मुचा॒ना उ॑त्क्रो॒दं कु॒र्वत॑ ए॒वमे॒व तद्यज॑माना देवब॒न्धान्मु॑मुचा॒ना उ॑त्क्रो॒दं कु॑र्वत॒ इष॒मूर्ज॑मा॒त्मन्दधा॑ना वा॒णः श॒तत॑न्तुर्भवति श॒तायुः॒ पुरु॑षः श॒तेन्द्रि॑य॒ आयु॑ष्ये॒वेन्द्रि॒ये प्रति॑ तिष्ठन्त्या॒जिं धा॑व॒न्त्यन॑भिजितस्या॒भिजि॑त्यै दुन्दु॒भीन्थ्स॒माघ्न॑न्ति पर॒मा वा ए॒षा वाग्या दु॑न्दु॒भौ प॑र॒मामे॒व~(२९)

%7.5.9.3
वाच॒मव॑ रुन्धते भूमिदुन्दु॒भिमा घ्न॑न्ति॒ यैवेमां वाक्प्रवि॑ष्टा॒ तामे॒वाव॑ रुन्ध॒ते\-ऽथो॑ इ॒मामे॒व ज॑यन्ति॒ सर्वा॒ वाचो॑ वदन्ति॒ सर्वा॑सां वा॒चामव॑रुद्ध्या आ॒र्द्रे चर्म॒न्व्याय॑च्छेते इन्द्रि॒यस्याव॑रुद्ध्या॒ आन्यः क्रोश॑ति॒ प्रान्यः शꣳ॑सति॒ य आ॒क्रोश॑ति पु॒नात्ये॒वैना॒न्थ्स स यः प्र॒शꣳस॑ति पू॒तेष्वे॒वान्नाद्यं॑ दधा॒त्यृषि॑कृतं च~(३०)

%7.5.9.4
वा ए॒ते दे॒वकृ॑तं च॒ पूर्वै॒र्मासै॒रव॑ रुन्धते॒ यद्भू॑ते॒च्छदा॒ꣳ॒ सामा॑नि॒ भव॑न्त्यु॒भय॒स्याव॑रुद्ध्यै॒ यन्ति॒ वा ए॒ते मि॑थु॒नाद्ये सं॑वथ्स॒रमु॑प॒यन्त्य॑न्तर्वे॒दि मि॑थु॒नौ सम्भ॑वत॒स्तेनै॒व मि॑थु॒नान्न य॑न्ति॥~(३१)

{\anuvakamend[{व्यृ॑द्ध॒मक्षो॑धुका॒स्ताꣳ समां᳚ प्र॒जाः प॑र॒मामे॒व च॑ त्रि॒ꣳ॒शच्च॑}]}%~(९)

%7.5.10.1
चर्माव॑ भिन्दन्ति पा॒प्मान॑मे॒वैषा॒मव॑ भिन्दन्ति॒ माप॑ राथ्सी॒र्माति॑ व्याथ्सी॒रित्या॑ह सम्प्र॒त्ये॑वैषां᳚ पा॒प्मान॒मव॑ भिन्दन्त्युदकु॒म्भान॑धिनि॒धाय॑ दा॒स्यो॑ मार्जा॒लीयं॒ परि॑ नृत्यन्ति प॒दो नि॑घ्न॒तीरि॒दं म॑धुं॒ गाय॑न्त्यो॒ मधु॒ वै दे॒वानां᳚ पर॒मम॒न्नाद्यं॑ पर॒ममे॒वान्नाद्य॒मव॑ रुन्धते प॒दो नि घ्न॑न्ति मही॒यामे॒वैषु॑ दधति॥~(३२)

{\anuvakamend[{चर्मैका॒न्नप॑ञ्चा॒शत्}]}%॥10॥

%7.5.11.1
पृ॒थि॒व्यै स्वाहा॒न्तरि॑क्षाय॒ स्वाहा॑ दि॒वे स्वाहा॑ सम्प्लोष्य॒ते स्वाहा॑ स॒म्प्लव॑मानाय॒ स्वाहा॒ सम्प्लु॑ताय॒ स्वाहा॑ मेघायिष्य॒ते स्वाहा॑ मेघाय॒ते स्वाहा॑ मेघि॒ताय॒ स्वाहा॑ मे॒घाय॒ स्वाहा॑ नीहा॒राय॒ स्वाहा॑ नि॒हाका॑यै॒ स्वाहा᳚ प्रास॒चाय॒ स्वाहा᳚ प्रच॒लाका॑यै॒ स्वाहा॑ विद्योतिष्य॒ते स्वाहा॑ वि॒द्योत॑मानाय॒ स्वाहा॑ संवि॒द्योत॑मानाय॒ स्वाहा᳚ स्तनयिष्य॒ते स्वाहा᳚ स्त॒नय॑ते॒ स्वाहो॒ग्रꣴ स्त॒नय॑ते॒ स्वाहा॑ वर्\mbox{}षिष्य॒ते स्वाहा॒ वर्\mbox{}ष॑ते॒ स्वाहा॑भि॒वर्\mbox{}ष॑ते॒ स्वाहा॑ परि॒वर्\mbox{}ष॑ते॒ स्वाहा॑ सं॒वर्\mbox{}ष॑ते~(३३)

%7.5.11.2
स्वाहा॑नु॒वर्\mbox{}ष॑ते॒ स्वाहा॑ शीकायिष्य॒ते स्वाहा॑ शीकाय॒ते स्वाहा॑ शीकि॒ताय॒ स्वाहा᳚ प्रोषिष्य॒ते स्वाहा᳚ प्रुष्ण॒ते स्वाहा॑ परिप्रुष्ण॒ते स्वाहो᳚द्ग्रहीष्य॒ते स्वाहो᳚द्गृह्ण॒ते स्वाहोद्गृ॑हीताय॒ स्वाहा॑ विप्लोष्य॒ते स्वाहा॑ वि॒प्लव॑मानाय॒ स्वाहा॒ विप्लु॑ताय॒ स्वाहा॑तफ्स्य॒ते स्वाहा॒तप॑ते॒ स्वाहो॒ग्रमा॒तप॑ते॒ स्वाह॒र्ग्भ्यः स्वाहा॒ यजु॑र्भ्यः॒ स्वाहा॒ साम॑भ्यः॒ स्वाहाङ्गि॑रोभ्यः॒ स्वाहा॒ वेदे᳚भ्यः॒ स्वाहा॒ गाथा᳚भ्यः॒ स्वाहा॑ नाराश॒ꣳ॒सीभ्यः॒ स्वाहा॒ रैभी᳚भ्यः॒ स्वाहा॒ सर्व॑स्मै॒ स्वाहा᳚॥~(३४)

{\anuvakamend[{सं॒ वर्\mbox{}ष॑ते॒ रैभी᳚भ्यः॒ स्वाहा॒ द्वे च॑}]}%॥11॥

%7.5.12.1
द॒त्वते॒ स्वाहा॑\-ऽद॒न्तका॑य॒ स्वाहा᳚ प्रा॒णिने॒ स्वाहा᳚\-ऽप्रा॒णाय॒ स्वाहा॒ मुख॑वते॒ स्वाहा॑\-ऽमु॒खाय॒ स्वाहा॒ नासि॑कवते॒ स्वाहा॑\-ऽनासि॒काय॒ स्वाहा᳚\-ऽक्ष॒ण्वते॒ स्वाहा॑\-ऽन॒क्षिका॑य॒ स्वाहा॑ क॒र्णिने॒ स्वाहा॑\-ऽक॒र्णका॑य॒ स्वाहा॑ शीर्\mbox{}ष॒ण्वते॒ स्वाहा॑\-ऽ\-शी॒र्॒\mbox{}षका॑य॒ स्वाहा॑ प॒द्वते॒ स्वाहा॑\-ऽपा॒दका॑य॒ स्वाहा᳚ प्राण॒ते स्वाहा\-ऽप्रा॑णते॒ स्वाहा॒ वद॑ते॒ स्वाहा\-ऽव॑दते॒ स्वाहा॒ पश्य॑ते॒ स्वाहा\-ऽप॑श्यते॒ स्वाहा॑ शृण्व॒ते स्वाहा\-ऽशृ॑ण्वते॒ स्वाहा॑ मन॒स्विने॒ स्वाहा᳚~(३५)

%7.5.12.2
अ॒म॒नसे॒ स्वाहा॑ रेत॒स्विने॒ स्वाहा॑\-ऽरे॒तस्का॑य॒ स्वाहा᳚ प्र॒जाभ्यः॒ स्वाहा᳚ प्र॒जन॑नाय॒ स्वाहा॒ लोम॑वते॒ स्वाहा॑\-ऽलो॒मका॑य॒ स्वाहा᳚ त्व॒चे स्वाहा॒\-ऽत्वक्का॑य॒ स्वाहा॒ चर्म॑ण्वते॒ स्वाहा॑\-ऽच॒र्मका॑य॒ स्वाहा॒ लोहि॑तवते॒ स्वाहा॑\-ऽलोहि॒ताय॒ स्वाहा॑ माꣳस॒न्वते॒ स्वाहा॑\-ऽमा॒ꣳ॒सका॑य॒ स्वाहा॒ स्नाव॑भ्यः॒ स्वाहा᳚\-ऽस्ना॒वका॑य॒ स्वाहा᳚\-ऽस्थ॒न्वते॒ स्वाहा॑\-ऽन॒स्थिका॑य॒ स्वाहा॑ मज्ज॒न्वते॒ स्वाहा॑\-ऽम॒ज्जका॑य॒ स्वाहा॒\-ऽङ्गिने॒ स्वाहा॑\-ऽन॒ङ्गाय॒ स्वाहा॒\-ऽऽ\-त्मने॒ स्वाहा\-ऽना᳚त्मने॒ स्वाहा॒ सर्व॑स्मै॒ स्वाहा᳚॥~(३६)

{\anuvakamend[{म॒न॒स्विने॒ स्वाहा\-ऽना᳚त्मने॒ स्वाहा॒ द्वे च॑}]}%॥12॥

%7.5.13.1
कस्त्वा॑ युनक्ति॒ स त्वा॑ युनक्तु॒ विष्णु॑स्त्वा युनक्त्व॒स्य य॒ज्ञस्यर्द्ध्यै॒ मह्य॒ꣳ॒ सन्न॑त्या अ॒मुष्मै॒ कामा॒यायु॑षे त्वा प्रा॒णाय॑ त्वा\-ऽपा॒नाय॑ त्वा व्या॒नाय॑ त्वा॒ व्यु॑ष्ट्यै त्वा र॒य्यै त्वा॒ राध॑से त्वा॒ घोषा॑य त्वा॒ पोषा॑य त्वाराद् घो॒षाय॑ त्वा॒ प्रच्यु॑त्यै त्वा॥~(३७)

{\anuvakamend[{कस्त्वा॒\-ऽष्टात्रिꣳ॑शत्}]}%॥13॥

%7.5.14.1
अ॒ग्नये॑ गाय॒त्राय॑ त्रि॒वृते॒ राथ॑न्तराय वास॒न्ताया॒ष्टाक॑पाल॒ इन्द्रा॑य॒ त्रैष्टु॑भाय पञ्चद॒शाय॒ बार्\mbox{}ह॑ताय॒ ग्रैष्मा॒यैका॑\-दश\-कपालो॒ विश्वे᳚भ्यो दे॒वेभ्यो॒ जाग॑तेभ्यः सप्तद॒शेभ्यो॑ वैरू॒पेभ्यो॒ वार्\mbox{}षि॑केभ्यो॒ द्वाद॑श\-कपालो मि॒त्रावरु॑णाभ्या॒मानु॑ष्टुभाभ्यामेक\-वि॒ꣳ॒शा\-भ्यां᳚ वैरा॒जाभ्याꣳ॑ शार॒दा\-भ्यां᳚ पय॒स्या॑ बृह॒स्पत॑ये॒ पाङ्क्ता॑य त्रिण॒वाय॑ शाक्व॒राय॒ हैम॑न्तिकाय च॒रुः स॑वि॒त्र आ॑तिच्छन्द॒साय॑ त्रयस्त्रि॒ꣳ॒शाय॑ रैव॒ताय॑ शैशि॒राय॒ द्वाद॑श\-कपा॒लो\-ऽदि॑त्यै॒ विष्णु॑पत्न्यै च॒रुर॒ग्नये॑ वैश्वान॒राय॒ द्वाद॑श\-कपा॒लो\-ऽनु॑मत्यै च॒रुः का॒य एक॑कपालः॥~(३८)

{\anuvakamend[{अ॒ग्नये\-ऽदि॑त्या॒ अनु॑मत्यै स॒प्तच॑त्वारिꣳशत्}]}%॥14॥

%7.5.15.1
यो वा अ॒ग्नाव॒ग्निः प्र॑ह्रि॒यते॒ यश्च॒ सोमो॒ राजा॒ तयो॑रे॒ष आ॑ति॒थ्यं यद॑ग्नीषो॒मीयो\-ऽथै॒ष रु॒द्रो यश्ची॒यते॒ यथ्सञ्चि॑ते॒\-ऽग्नावे॒तानि॑ ह॒वीꣳषि॒ न नि॒र्वपे॑दे॒ष ए॒व रु॒द्रो\-ऽशा᳚न्त उपो॒त्थाय॑ प्र॒जां प॒शून् यज॑मानस्या॒भि म॑न्येत॒ यथ्सञ्चि॑ते॒\-ऽग्नावे॒तानि॑ ह॒वीꣳषि॑ नि॒र्वप॑ति भाग॒धेये॑नै॒वैनꣳ॑ शमयति॒ नास्य॑ रु॒द्रो\-ऽशा᳚न्तः~(३९)

%7.5.15.2
उ॒पो॒त्थाय॑ प्र॒जां प॒शून॒भि म॑न्यते॒ दश॑ ह॒वीꣳषि॑ भवन्ति॒ नव॒ वै पुरु॑षे प्रा॒णा नाभि॑र्दश॒मी प्रा॒णाने॒व यज॑माने दधा॒त्यथो॒ दशा᳚क्षरा वि॒राडन्नं॑ वि॒राज्ये॒वान्नाद्ये॒ प्रति॑ तिष्ठत्यृ॒तुभि॒र्वा ए॒ष छन्दो॑भिः॒ स्तोमैः᳚ पृ॒ष्ठैश्चे॑त॒व्य॑ इत्या॑हु॒र्यदे॒तानि॑ ह॒वीꣳषि॑ नि॒र्वप॑त्यृ॒तुभि॑रे॒वैनं॒ छन्दो॑भिः॒ स्तोमैः᳚ पृ॒ष्ठैश्चि॑नुते॒ दिशः॑ सुषुवा॒णेन॑~(४०)

%7.5.15.3
अ॒भि॒जित्या॒ इत्या॑हु॒र्यदे॒तानि॑ ह॒वीꣳषि॑ नि॒र्वप॑ति दि॒शाम॒भिजि॑त्या ए॒तया॒ वा इन्द्रं॑ दे॒वा अ॑याजय॒न्तस्मा॑दिन्द्रस॒व ए॒तया॒ मनुं॑ मनु॒ष्या᳚स्तस्मा᳚न्मनुस॒वो यथेन्द्रो॑ दे॒वानां॒ यथा॒ मनु॑र्मनु॒ष्या॑णामे॒वं भ॑वति॒ य ए॒वं वि॒द्वाने॒तयेष्ट्या॒ यज॑ते॒ दिग्व॑तीः पुरोनुवा॒क्या॑ भवन्ति॒ सर्वा॑सां दि॒शाम॒भिजि॑त्यै॥~(४१)

{\anuvakamend[{अशा᳚न्तः सुषुवा॒णेनैक॑चत्वारिꣳशच्च}]}%॥15॥

%7.5.16.1
यः प्रा॑ण॒तो नि॑मिष॒तो म॑हि॒त्वैक॒ इद्राजा॒ जग॑तो ब॒भूव॑। य ईशे॑ अ॒स्य द्वि॒पद॒श्चतु॑ष्पदः॒ कस्मै॑ दे॒वाय॑ ह॒विषा॑ विधेम। उ॒प॒या॒मगृ॑हीतो\-ऽसि प्र॒जा\-प॑तये त्वा॒ जुष्टं॑ गृह्णामि॒ तस्य॑ ते॒ द्यौर्म॑हि॒मा नक्ष॑त्राणि रू॒पमा॑दि॒त्यस्ते॒ तेज॒स्तस्मै᳚ त्वा महि॒म्ने प्र॒जा\-प॑तये॒ स्वाहा᳚॥~(४२)

{\anuvakamend[{यः प्रा॑ण॒तो द्यौरा॑दि॒त्यो᳚\-ऽष्टात्रिꣳ॑शत्}]}%॥16॥

%7.5.17.1
य आ᳚त्म॒दा ब॑ल॒दा यस्य॒ विश्व॑ उ॒पास॑ते प्र॒शिषं॒ यस्य॑ दे॒वाः। यस्य॑ छा॒यामृतं॒ यस्य॑ मृ॒त्युः कस्मै॑ दे॒वाय॑ ह॒विषा॑ विधेम। उ॒प॒या॒मगृ॑हीतो\-ऽसि प्र॒जा\-प॑तये त्वा॒ जुष्टं॑ गृह्णामि॒ तस्य॑ ते पृथि॒वी म॑हि॒मौष॑धयो॒ वन॒स्पत॑यो रू॒पम॒ग्निस्ते॒ तेज॒स्तस्मै᳚ त्वा महि॒म्ने प्र॒जा\-प॑तये॒ स्वाहा᳚॥~(४३)

{\anuvakamend[{य आ᳚त्म॒दाः पृ॑थि॒व्य॑ग्निरेका॒न्नच॑त्वारि॒ꣳ॒शत्}]}%॥17॥

%7.5.18.1
आ ब्रह्म॑न्ब्राह्म॒णो ब्र॑ह्मवर्च॒सी जा॑यता॒मा\-ऽस्मिन्रा॒ष्ट्रे रा॑ज॒न्य॑ इष॒व्यः॑ शूरो॑ महार॒थो जा॑यता॒न्दोग्ध्री॑ धे॒नुर्वोढा॑\-ऽ\-न॒ड्वाना॒शुः सप्तिः॒ पुर॑न्धि॒र्योषा॑ जि॒ष्णू र॑थे॒ष्ठाः स॒भेयो॒ युवा\-ऽस्य यज॑मानस्य वी॒रो जा॑यतान्निका॒मेनि॑कामे नः प॒र्जन्यो॑ वर्\mbox{}षतु फ॒लिन्यो॑ न॒ ओष॑धयः पच्यन्तां योगक्षे॒मो नः॑ कल्पताम्॥~(४४)

{\anuvakamend[{आ ब्रह्म॒न्नेक॑चत्वारिꣳशत्}]}%॥18॥

%7.5.19.1
आक्रान्॑ वा॒जी पृ॑थि॒वीम॒ग्निं युज॑मकृत वा॒ज्यर्वाक्रान्॑ वा॒ज्य॑न्तरि॑क्षं वा॒युं युज॑मकृत वा॒ज्यर्वा॒ द्यां वा॒ज्या\-ऽक्रꣴ॑स्त॒ सूर्यं॒ युज॑मकृत वा॒ज्यर्वा॒ग्निस्ते॑ वाजि॒न् युङ्ङनु॒ त्वा र॑भे स्व॒स्ति मा॒ सं पा॑रय वा॒युस्ते॑ वाजि॒न् युङ्ङनु॒ त्वा र॑भे स्व॒स्ति मा॒ सम्~(४५)

%7.5.19.2
पा॒र॒यादि॒त्यस्ते॑ वाजि॒न् युङ्ङनु॒ त्वा र॑भे स्व॒स्ति मा॒ सं पा॑रय प्राण॒धृग॑सि प्रा॒णं मे॑ दृꣳह व्यान॒धृग॑सि व्या॒नं मे॑ दृꣳहापान॒धृग॑स्यपा॒नं मे॑ दृꣳह॒ चक्षु॑रसि॒ चक्षु॒र्मयि॑ धेहि॒ श्रोत्र॑मसि॒ श्रोत्रं॒ मयि॑ धे॒ह्यायु॑र॒स्यायु॒र्मयि॑ धेहि॥~(४६)

{\anuvakamend[{वा॒युस्ते॑ वाजि॒न् युङ्ङनु॒ त्वा र॑भे स्व॒स्ति मा॒ सन्त्रिच॑त्वारिꣳशच्च}]}%॥19॥

%7.5.20.1
जज्ञि॒ बीजं॒ वर्\mbox{}ष्टा॑ प॒र्जन्यः॒ पक्ता॑ स॒स्यꣳ सु॑पिप्प॒ला ओष॑धयः स्वधिचर॒णेयꣳ सू॑पसद॒नो᳚\-ऽग्निः स्व॑ध्य॒क्षम॒न्तरि॑क्षꣳ सुपा॒वः पव॑मानः सूपस्था॒ना द्यौः शि॒वम॒सौ तप॑न् यथापू॒र्वम॑होरा॒त्रे प॑ञ्चद॒शिनो᳚\-ऽर्धमा॒सास्त्रि॒ꣳ॒शिनो॒ मासाः᳚ कॢ॒प्ता ऋ॒तवः॑ शा॒न्तः सं॑वथ्स॒रः॥~(४७)

{\anuvakamend[{जज्ञि॒ बीज॒मेक॑त्रिꣳशत्}]}%॥20॥

%7.5.21.1
आ॒ग्ने॒यो᳚\-ऽष्टाक॑पालः सौ॒म्यश्च॒रुः सा॑वि॒त्रो᳚\-ऽष्टाक॑पालः पौ॒ष्णश्च॒रू रौ॒द्रश्च॒रुर॒ग्नये॑ वैश्वान॒राय॒ द्वाद॑श\-कपालो मृगाख॒रे यदि॒ नागच्छे॑द॒ग्नये\-ऽꣳ॑हो॒मुचे॒\-ऽष्टाक॑पालः सौ॒र्यं पयो॑ वाय॒व्य॑ आज्य॑भागः॥~(४८)

{\anuvakamend[{आ॒ग्ने॒यश्चतु॑र्विꣳशतिः}]}%॥21॥

%7.5.22.1
अ॒ग्नये\-ऽꣳ॑हो॒मुचे॒\-ऽष्टाक॑पाल॒ इन्द्रा॑याꣳहो॒मुच॒ एका॑\-दश\-कपालो मि॒त्रावरु॑णाभ्यामागो॒मुग्\-भ्यां᳚ पय॒स्या॑ वायोसावि॒त्र आ॑गो॒मुग्\-भ्यां᳚ च॒रुर॒श्विभ्या॑मागो॒मुग्\-भ्यां᳚ धा॒ना म॒रुद्भ्य॑ एनो॒मुग्भ्यः॑ स॒प्तक॑पालो॒ विश्वे᳚भ्यो दे॒वेभ्य॑ एनो॒मुग्भ्यो॒ द्वाद॑श\-कपा॒लो\-ऽनु॑मत्यै च॒रुर॒ग्नये॑ वैश्वान॒राय॒ द्वाद॑श\-कपालो॒ द्यावा॑\-पृथि॒वीभ्या॑मꣳहो॒मुग्\-भ्यां᳚ द्विकपा॒लः॥~(४९)

{\anuvakamend[{अ॒ग्नये\-ऽꣳ॑हो॒मुचे᳚ त्रि॒ꣳ॒शत्}]}%॥22॥

%7.5.23.1
अ॒ग्नये॒ सम॑नमत्पृथि॒व्यै सम॑नम॒द्यथा॒ग्निः पृ॑थि॒व्या स॒मन॑मदे॒वं मह्यं॑ भ॒द्राः सन्न॑तयः॒ सं न॑मन्तु वा॒यवे॒ सम॑नमद॒न्तरि॑क्षाय॒ सम॑नम॒द्यथा॑ वा॒युर॒न्तरि॑क्षेण॒ सूर्या॑य॒ सम॑नमद्दि॒वे सम॑नम॒द्यथा॒ सूर्यो॑ दि॒वा च॒न्द्रम॑से॒ सम॑नम॒न्नक्ष॑त्रेभ्यः॒ सम॑नम॒द्यथा॑ च॒न्द्रमा॒ नक्ष॑त्रै॒र्वरु॑णाय॒ सम॑नमद॒द्भ्यः सम॑नम॒द्यथा᳚~(५०)

%7.5.23.2
वरु॑णो॒\-ऽद्भिः साम्ने॒ सम॑नमदृ॒चे सम॑नम॒द्यथा॒ साम॒र्चा ब्रह्म॑णे॒ सम॑नमत्क्ष॒त्राय॒ सम॑नम॒द्यथा॒ ब्रह्म॑ क्ष॒त्रेण॒ राज्ञे॒ सम॑नमद्वि॒शे सम॑नम॒द्यथा॒ राजा॑ वि॒शा रथा॑य॒ सम॑नम॒दश्वे᳚भ्यः॒ सम॑नम॒द्यथा॒ रथो\-ऽश्वैः᳚ प्र॒जा\-प॑तये॒ सम॑नमद्भू॒तेभ्यः॒ सम॑नम॒द्यथा᳚ प्र॒जा\-प॑तिर्भू॒तैः स॒मन॑मदे॒वं मह्यं॑ भ॒द्राः सन्न॑तयः॒ सं न॑मन्तु॥~(५१)

{\anuvakamend[{अ॒द्भ्यः सम॑नम॒द्यथा॒ मह्यं॑ च॒त्वारि॑ च}]}%॥23॥

%7.5.24.1
ये ते॒ पन्था॑नः सवितः पू॒र्व्यासो॑\-ऽरे॒णवो॒ वित॑ता अ॒न्तरि॑क्षे। तेभि॑र्नो अ॒द्य प॒थिभिः॑ सु॒गेभी॒ रक्षा॑ च नो॒ अधि॑ च देव ब्रूहि। नमो॒\-ऽग्नये॑ पृथिवि॒क्षिते॑ लोक॒स्पृते॑ लो॒कम॒स्मै यज॑मानाय देहि॒ नमो॑ वा॒यवे᳚\-ऽन्तरिक्ष॒क्षिते॑ लोक॒स्पृते॑ लो॒कम॒स्मै यज॑मानाय देहि॒ नमः॒ सूर्या॑य दिवि॒क्षिते॑ लोक॒स्पृते॑ लो॒कम॒स्मै यज॑मानाय देहि॥~(५२)

{\anuvakamend[{ये ते॒ चतु॑श्चत्वारिꣳशत्}]}%॥24॥

%7.5.25.1
यो वा अश्व॑स्य॒ मेध्य॑स्य॒ शिरो॒ वेद॑ शीर्\mbox{}ष॒ण्वान्मेध्यो॑ भवत्यु॒षा वा अश्व॑स्य॒ मेध्य॑स्य॒ शिरः॒ सूर्य॒श्चक्षु॒र्वातः॑ प्रा॒णश्च॒न्द्रमाः॒ श्रोत्र॒न्दिशः॒ पादा॑ अवान्तरदि॒शाः पर्\mbox{}श॑वो\-ऽहोरा॒त्रे नि॑मे॒षो᳚\-ऽर्धमा॒साः पर्वा॑णि॒ मासाः᳚ स॒न्धाना᳚न्यृ॒तवो\-ऽङ्गा॑नि संवथ्स॒र आ॒त्मा र॒श्मयः॒ केशा॒ नक्ष॑त्राणि रू॒पन्तार॑का अ॒स्थानि॒ नभो॑ मा॒ꣳ॒सान्योष॑धयो॒ लोमा॑नि॒ वन॒स्पत॑यो॒ वाला॑ अ॒ग्निर्मुखं॑ वैश्वान॒रो व्यात्तम्᳚~(५३)

%7.5.25.2
स॒मु॒द्र उ॒दर॑म॒न्तरि॑क्षं पा॒युर्द्यावा॑\-पृथि॒वी आ॒ण्डौ ग्रावा॒ शेपः॒ सोमो॒ रेतो॒ यज्ज॑ञ्ज॒भ्यते॒ तद्वि द्यो॑तते॒ यद्वि॑धूनु॒ते तथ्स्त॑नयति॒ यन्मेह॑ति॒ तद्व॑र्\mbox{}षति॒ वागे॒वास्य॒ वागह॒र्वा अश्व॑स्य॒ जाय॑मानस्य महि॒मा पु॒रस्ता᳚ज्जायते॒ रात्रि॑रेनं महि॒मा प॒श्चादनु॑ जायत ए॒तौ वै म॑हि॒माना॒वश्व॑म॒भितः॒ सम्ब॑भूवतु॒र्॒\mbox{}हयो॑ दे॒वान॑वह॒दर्वासु॑रान् वा॒जी ग॑न्ध॒र्वानश्वो॑ मनु॒ष्या᳚न्थ्समु॒द्रो वा अश्व॑स्य॒ योनिः॑ समु॒द्रो बन्धुः॑॥~(५४)

{\anuvakamend[{व्यात्त॑मवह॒द्द्वाद॑श च}]}%॥25॥

\prashnaend{गावो॒ गावः॒ सिषा॑सन्तीः प्रथ॒मे मा॒सि स॑मा॒न्यो॑ यदि॒ सोमौ॑ षड॒हैरु॒थ्सृज्या(३)ं दे॒वाना॑म॒र्क्ये॑ण॒ चर्माव॑ पृथि॒व्यै द॒त्वते॒ कस्त्वा॒ग्नये॒ यो वै यः प्रा॑ण॒तो य आ᳚त्म॒दा आ ब्रह्म॒न्नाक्रा॒ञ्जज्ञि॒ बीज॑माग्ने॒यो᳚\-ऽष्टाक॑पालो॒\-ऽग्नये\-ऽꣳ॑हो॒मुचे॒\-ऽ\-ष्टाक॑पालो॒\-ऽग्नये॒ सम॑नम॒द्ये ते॒ पन्था॑नो॒ यो वा अश्व॑स्य॒ मेध्य॑स्य॒ शिरः॒ प़ञ्च॑विꣳशतिः॥२५॥}{गावः॑ समा॒न्यः॑ सव॑नमष्टा॒भिर्वा ए॒ते दे॒वकृ॑तञ्चाभि॒जित्या॒ इत्या॑हु॒र्वरु॑णो॒\-ऽद्भिः साम्ने॒ चतुः॑पञ्चा॒शत्॥५४॥}{गावो॒ योनिः॑ समु॒द्रो बन्धुः॑॥}%%७-५
{हरिः॑ ॐ}{॥कृष्ण-यजुर्वेदीय-तैत्तिरीय-संहितायां सप्तमकाण्डे पञ्चमः प्रश्नः समाप्तः॥७-५॥}
%%% END PRASHNA
