% !TeX program = XeLaTeX
% !TeX root = ../vedamantrabook.tex

\renewcommand{\dot}{·\allowbreak}
\definecolor{fgcolour}{HTML}{013484}
\definecolor{bgcolour}{HTML}{DEE6EF}
\pagecolor{bgcolour}
\color{fgcolour}


\chapt{शिवपूजा यावदुक्त-महान्यास-पूर्विका}

[आचम्य।] शुक्लाम्बर … शान्तये। [प्राणानायम्य।] ममोपात्त + श्रीसाम्बपरमेश्वरपूजां करिष्ये।

ॐ ग॒णानां᳚ त्वा ग॒णप॑तिꣳ हवामहे क॒विं क॑वी॒ना\dot{}मु॑प॒मश्र॑वस्तमम्। ज्ये॒ष्ठ॒राजं॒ ब्रह्म॑णां ब्रह्मणस्पत॒ आ नः॑ शृ॒ण्व\dot{}न्नू॒तिभिः॑ सीद॒ साद॑नम्॥

ॐ महागणपतये॒ नमः॥

%{\small \closesection}
\sect{प्रथमः न्यासः – शिखादिः अस्त्रान्तः}

या ते॑ रुद्र शि॒वा त॒नूरघो॒राऽपा॑पकाशिनी।\\
तया॑ नस्त॒नुवा॒ शन्त॑मया॒ गिरि॑शन्ता॒भिचा॑कशीहि॥\\
शिखायै नमः॥

अ॒स्मिन् म॑ह॒त्य॑र्ण॒वे᳚ऽन्तरि॑क्षे भ॒वा अधि॑।\\
तेषाꣳ॑ सहस्रयोज॒नेऽव॒धन्वा॑नि तन्मसि॥\\
शिरसे नमः॥

स॒हस्रा॑णि सहस्र॒शो ये रु॒द्रा अधि॒ भूम्या᳚म्।\\
तेषाꣳ॑ सहस्रयोज॒नेऽव॒धन्वा॑नि तन्मसि॥\\
ललाटाय नमः॥

ह॒ꣳ॒सः शु॑चि॒षद्वसु॑रन्तरिक्ष॒सद्धोता॑ वेदि॒षदति॑थिर्दुरोण॒सत्।\\
नृ॒षद्व॑र॒सदृ॑त॒सद्व्यो॑म॒सद॒ब्जा गो॒जा ऋ॑त॒जा अ॑द्रि॒जा ऋ॒तं बृ॒हत्॥\\
भ्रुवोर्मध्याय नमः॥

त्र्य॑म्बकं यजामहे सुग॒न्धिं पु॑ष्टि॒वर्ध॑नम्।\\
उ॒र्वा॒रु॒कमि॑व॒ बन्ध॑नान्मृ॒त्योर्मु॑क्षीय॒ माऽमृता᳚त्॥\\
नेत्राभ्यां नमः॥

नमः॒ स्रुत्या॑य च॒ पथ्या॑य च॒ (नमः॑ का॒ट्या॑य च नी॒प्या॑य च॒ %\\
नमः॒ सूद्या॑य च सर॒स्या॑य च॒ नमो॑ ना॒द्याय॑ च वैश॒न्ताय॑ च)।\\
कर्णाभ्यां नमः॥

मा न॑स्तो॒के तन॑ये॒ मा न॒ आयु॑षि॒ मा नो॒ गोषु॒ मा नो॒ अश्वे॑षु रीरिषः।\\
वी॒रान्मा नो॑ रुद्र भामि॒तोऽव॑धीर्‌ह॒विष्म॑न्तो॒ नम॑सा विधेम ते॥\\
नासिकाभ्यां नमः॥

अ॒व॒तत्य॒ धनु॒स्त्वꣳ सह॑स्राक्ष॒ शते॑षुधे॥\\
नि॒शीर्य॑ श॒ल्यानां॒ मुखा॑ शि॒वो नः॑ सु॒मना॑ भव।\\
मुखाय नमः॥

नील॑ग्रीवाः शिति॒कण्ठाः᳚ श॒र्वा अ॒धः, क्ष॑माच॒राः।\\
तेषाꣳ॑ सहस्रयोज॒नेऽव॒धन्वा॑नि तन्मसि॥\\
नील॑ग्रीवाः शिति॒कण्ठा॒ दिवꣳ॑ रु॒द्रा उप॑श्रिताः।\\
तेषाꣳ॑ सहस्रयोज॒नेऽव॒धन्वा॑नि तन्मसि॥\\
कण्ठाय नमः॥

नम॑स्ते अ॒स्त्वायु॑धा॒याना॑तताय धृ॒ष्णवे᳚।\\
उ॒भाभ्या॑मु॒त ते॒ नमो॑ बा॒हुभ्यां॒ तव॒ धन्व॑ने॥\\
बाहुभ्यां नमः॥

या ते॑ हे॒तिर्मी॑ढुष्टम॒ हस्ते॑ ब॒भूव॑ ते॒ धनुः॑।\\
तया॒ऽस्मान् वि॒श्वत॒स्त्वम॑य॒क्ष्मया॒ परि॑ब्भुज॥\\
उपबाहुभ्यां नमः॥

परि॑ णो रु॒द्रस्य॑ हे॒तिर्वृ॑णक्तु॒ परि॑ त्वे॒षस्य॑ दुर्म॒तिर॑घा॒योः।\\
अव॑ स्थि॒रा म॒घव॑द्भ्यस्तनुष्व॒ मीढ्व॑स्तो॒काय॒ तन॑याय मृडय॥\\
मणिबन्धाभ्यां नमः॥

ये ती॒र्थानि॑ प्र॒चर॑न्ति सृ॒काव॑न्तो निष॒ङ्गिणः॑।\\
तेषाꣳ॑ सहस्रयोज॒नेऽव॒धन्वा॑नि तन्मसि॥\\
हस्ताभ्यां नमः॥

स॒द्योजा॒तं प्र॑पद्या॒मि॒ स॒द्योजा॒ताय॒ वै नमो॒ नमः॑।\\
भ॒वे भ॑वे॒ नाति॑ भवे भवस्व॒ माम्। भ॒वोद्भ॑वाय॒ नमः॑॥\\
अङ्गुष्ठाभ्यां नमः॥

वा॒म॒दे॒वाय॒ नमो᳚ ज्ये॒ष्ठाय॒ नमः॑ श्रे॒ष्ठाय॒ नमो॑ रु॒द्राय॒ नमः॒ काला॑य॒ नमः॒ कल॑विकरणाय॒ नमो॒ बल॑विकरणाय॒ नमो॒ बला॑य॒ नमो॒ बल॑प्रमथनाय॒ नमः॒ सर्व॑भूतदमनाय॒ नमो॑ म॒नोन्म॑नाय॒ नमः॑॥\\
तर्जनीभ्यां नमः॥

अ॒घोरे᳚भ्योऽथ॒ घोरे᳚भ्यो॒ घोर॒घोर॑तरेभ्यः।\\
सर्वे᳚भ्यः सर्व॒शर्वे᳚भ्यो॒ नम॑स्ते अस्तु रु॒द्ररू॑पेभ्यः॥\\
मध्यमाभ्यां नमः॥

तत्पुरु॑षाय वि॒द्महे॑ महादे॒वाय॑ धीमहि।\\
तन्नो॑ रुद्रः प्रचो॒दया᳚त्॥\\
अनामिकाभ्यां नमः॥

ईशानः सर्व॑विद्या॒ना॒मीश्वरः सर्व॑भूता॒नां॒ ब्रह्माधि॑पति॒र्ब्रह्म॒णो\-ऽधि॑पति॒र्ब्रह्मा॑ शि॒वो मे॑ अस्तु सदाशि॒वोम्॥\\
कनिष्ठिकाभ्यां नमः॥

नमो॑ वः किरि॒केभ्यो॑ दे॒वाना॒ꣳ॒ हृद॑येभ्यः॥\\
हृदयाय नमः॥

नमो॑ ग॒णेभ्यो॑ ग॒णप॑तिभ्यश्च वो॒ नमः॑॥\\
पृष्ठाय नमः॥

नमो॒ हिर॑ण्यबाहवे सेना॒न्ये॑ दि॒शां च॒ पत॑ये॒ नमः॑॥\\
पार्श्वाभ्यां नमः॥

विज्यं॒ धनुः॑ कप॒र्दिनो॒ विश॑ल्यो॒ बाण॑वाꣳ उ॒त।\\
अने॑शन्न॒स्येष॑व आ॒भुर॑स्य निष॒ङ्गथिः॑॥\\
जठराय नमः॥

हि॒र॒ण्य॒ग॒र्भः सम॑वर्त॒ताग्रे॑ भू॒तस्य॑ जा॒तः पति॒रेक॑ आसीत्।\\
सदा॑धार पृथि॒वीं द्यामु॒तेमां कस्मै॑ दे॒वाय॑ ह॒विषा॑ विधेम॥\\
नाभ्यै नमः॥

मीढु॑ष्टम॒ शिव॑तम शि॒वो नः॑ सु॒मना॑ भव।\\
प॒र॒मे वृ॒क्ष आयु॑धं नि॒धाय॒ कृत्तिं॒ वसा॑न॒ आ च॑र॒ पिना॑कं॒ बिभ्र॒दा ग॑हि॥\\
कट्यै नमः॥

ये भू॒ताना॒मधि॑पतयो विशि॒खासः॑ कप॒र्दिनः॑। \\
तेषाꣳ॑ सहस्रयोज॒नेऽव॒धन्वा॑नि तन्मसि॥\\
गुह्याय नमः॥

स शि॒रा जा॒तवे॑दा अ॒क्षरं॑ पर॒मं प॒दम्।\\
वेदा॑ना॒ꣳ॒ शिर॑सि मा॒ता॒ आ॒युष्मन्तं॑ करोतु॒ माम्॥\\
अपानाय नमः॥

मा नो॑ म॒हान्त॑मु॒त मा नो॑ अर्भ॒कं मा न॒ उक्ष॑न्तमु॒त मा न॑ उक्षि॒तम्।\\
मा नो॑ वधीः पि॒तरं॒ मोत मा॒तरं॑ प्रि॒या मा न॑स्त॒नुवो॑ रुद्र रीरिषः॥\\
ऊरुभ्यां नमः॥

ए॒ष ते॑ रुद्रभा॒गस्तं जु॑षस्व॒ तेना॑व॒सेन॑ प॒रो मूज॑व॒तोऽती॒\-ह्यव॑ततधन्वा॒ पिना॑कहस्तः॒ कृत्ति॑वासाः॥\\
जानुभ्यां नमः॥

स॒ꣳ॒सृ॒ष्ट॒जिथ्सो॑म॒पा बा॑हुश॒ध्यू᳚र्ध्वध॑न्वा॒ प्रति॑हिताभि॒रस्ता᳚।\\
बृह॑स्पते॒ परि॑दीया॒ रथे॑न रक्षो॒हाऽमित्राꣳ॑ अप॒बाध॑मानः॥\\
जङ्घाभ्यां नमः॥

विश्वं॑ भू॒तं भुव॑नं चि॒त्रं ब॑हु॒धा जा॒तं जाय॑मानं च॒ यत्।\\
सर्वो॒ ह्ये॑ष रु॒द्रस्तस्मै॑ रु॒द्राय॒ नमो॑ अस्तु॥\\
गुल्फाभ्यां नमः॥

ये प॒थां प॑थि॒रक्ष॑य ऐलबृ॒दा य॒व्युधः॑।\\
तेषाꣳ॑ सहस्रयोज॒नेऽव॒धन्वा॑नि तन्मसि॥\\
पादाभ्यां नमः॥

अध्य॑वोचदधिव॒क्ता प्र॑थ॒मो दैव्यो॑ भि॒षक्।\\
अहीꣴ॑श्च॒ सर्वा᳚ञ्ज॒म्भय॒-न्थ्सर्वा᳚श्च यातुधा॒न्यः॑॥\\
कवचाय हुम्॥

नमो॑ बि॒ल्मिने॑ च कव॒चिने॑ च॒ (नमः॑ श्रु॒ताय॑ च श्रुतसे॒नाय॑ च)॥\\
उपकवचाय हुम्॥

नमो॑ अस्तु॒ नील॑ग्रीवाय सहस्रा॒क्षाय॑ मी॒ढुषे᳚। \\
अथो॒ ये अ॑स्य॒ सत्वा॑नो॒ऽहं तेभ्यो॑ऽकरं॒ नमः॑॥\\
तृतीयनेत्राय वौषट्॥

प्र मु॑ञ्च॒ धन्व॑न॒स्त्वमु॒भयो॒रार्त्नि॑यो॒र्ज्याम्।\\
याश्च॑ ते॒ हस्त॒ इष॑वः॒ परा॒ ता भ॑गवो वप॥\\
अस्त्राय फट्॥

य ए॒ताव॑न्तश्च॒ भूयाꣳ॑सश्च॒ दिशो॑ रु॒द्रा वि॑तस्थि॒रे।\\
तेषाꣳ॑ सहस्रयोज॒नेऽव॒धन्वा॑नि तन्मसि॥\\

{[इति दिग्बन्धः॥]}

{\small \closesection}

\sect{द्वितीयः न्यासः – मूर्धादिः पादान्तः}
ॐ मूर्ध्ने नमः। नं नासिकाभ्यां नमः। मों ललाटाय नमः। भं मुखाय नमः। गं कण्ठाय नमः। वं हृदयाय नमः। तें दक्षिणहस्ताय नमः। रुं वामहस्ताय नमः। द्रां नाभ्यै नमः। यं पादाभ्यां नमः।
% ॐ नमो मूर्ध्नि। नं नमो नासिकाग्रे। मों नमो ललाटे। भं नमो मुखमध्ये। गं नमः कण्ठे। वं नमो हृदि। तें नमो दक्षिणहस्ते। रुं नमो वामहस्ते। द्रां नमो नाभौ। यं नमः पादयोः।

\sect{तृतीयः न्यासः – पादादिः मूर्धान्तः}
स॒द्योजा॒तं प्र॑पद्या॒मि॒ स॒द्योजा॒ताय॒ वै नमो॒ नमः॑।\\
भ॒वे भ॑वे॒ नाति॑ भवे भवस्व॒ माम्। भ॒वोद्भ॑वाय॒ नमः॑॥\\
पादाभ्यां नमः॥

वा॒म॒दे॒वाय॒ नमो᳚ ज्ये॒ष्ठाय॒ नमः॑ श्रे॒ष्ठाय॒ नमो॑ रु॒द्राय॒ नमः॒ काला॑य॒ नमः॒ कल॑विकरणाय॒ नमो॒ बल॑विकरणाय॒ नमो॒ बला॑य॒ नमो॒ बल॑प्रमथनाय॒ नमः॒ सर्व॑भूतदमनाय॒ नमो॑ म॒नोन्म॑नाय॒ नमः॑॥\\
ऊरुभ्यां नमः॥

अ॒घोरे᳚भ्योऽथ॒ घोरे᳚भ्यो॒ घोर॒घोर॑तरेभ्यः।\\
सर्वे᳚भ्यः सर्व॒शर्वे᳚भ्यो॒ नम॑स्ते अस्तु रु॒द्ररू॑पेभ्यः॥\\
हृदयाय नमः॥

तत्पुरु॑षाय वि॒द्महे॑ महादे॒वाय॑ धीमहि।\\
तन्नो॑ रुद्रः प्रचो॒दया᳚त्॥\\
मुखाय नमः॥

ईशानः सर्व॑विद्या॒ना॒मीश्वरः सर्व॑भूता॒नां॒ ब्रह्माधि॑पति॒र्ब्रह्म॒णो\-ऽधि॑पति॒र्ब्रह्मा॑ शि॒वो मे॑ अस्तु सदाशि॒वोम्॥\\
मूर्ध्ने नमः॥\\

{\small \closesection}

\sect{हंसगायत्री}
हंसः सोऽहं सोऽहं हंसः॥

{\small \closesection}

\sect{सम्पुटन्यासः}
\newcounter{dik}
\newcommand{\samputa}[7]{\refstepcounter{dik}
ॐ भूर्भुवः॒ सुव॒रोम्। #1 #2।\\
#3।\\#4॥\\%
#1 #2 भूर्भुवः॒ सुवः॑।\\
#5 #7 नमः। #6 #7 नमः॥\\
\hfill॥\devanumber{\arabic{dik}}॥}

\samputa{ॐ}{लं}%
{त्रा॒तार॒मिन्द्र॑मवि॒तार॒मिन्द्र॒ꣳ॒ हवे॑हवे सु॒हव॒ꣳ॒ शूर॒मिन्द्रम्᳚}%
{हु॒वे नु श॒क्रं पु॑रुहू॒तमिन्द्रꣴ॑ स्व॒स्ति नो॑ म॒घवा॑ धा॒त्विन्द्रः॑}%
{प्राच्याम्}{ललाटे}{इन्द्राय}

\samputa{नं}{रं}%
{त्वन्नो॑ अग्ने॒ वरु॑णस्य वि॒द्वान् दे॒वस्य॒ हेडोऽव॑ यासिसीष्ठाः}%
{यजि॑ष्ठो॒ वह्नि॑तमः॒ शोशु॑चानो॒ विश्वा॒ द्वेषाꣳ॑सि॒ प्रमु॑मुग्ध्य॒स्मत्}%
{दक्षिणपूर्वस्याम्}{नेत्रयोः}{अग्नये}

\samputa{मों}{हं}%
{सु॒गं नः॒ पन्था॒मभ॑यं कृणोतु। यस्मि॒न्नक्ष॑त्रे य॒म एति॒ राजा᳚}%
{यस्मि॑न्नेनम॒भ्यषि॑ञ्चन्त दे॒वाः। तद॑स्य चि॒त्रꣳ ह॒विषा॑ यजाम}%
{दक्षिणस्यां}{कर्णयोः}{यमाय}

\samputa{भं}{षं}%
{असु॑न्वन्त॒मय॑जमानमिच्छ स्ते॒नस्ये॒त्यां तस्क॑र॒स्यान्वे॑षि}%
{अ॒न्यम॒स्मदि॑च्छ॒ सा त॑ इ॒त्या नमो॑ देवि निर्‌ऋते॒ तुभ्य॑मस्तु}%
{दक्षिणापरस्यां}{मुखे}{निर्‌ऋतये}

\samputa{गं}{वं}%
{तत्त्वा॑ यामि॒ ब्रह्म॑णा॒ वन्द॑मान॒स्तदा शा᳚स्ते॒ यज॑मानो ह॒विर्भिः॑}%
{अहे॑डमानो वरुणे॒ह बो॒ध्युरु॑शꣳस॒ मा न॒ आयुः॒ प्रमो॑षीः}%
{प्रतीच्यां}{बाह्वोः}{वरुणाय}

\samputa{वं}{यं}%
{आ नो॑ नि॒युद्भिः॑ श॒तिनी॑भिरध्व॒रम्। स॒ह॒स्रिणी॑भि॒रुप॑\-याहि य॒ज्ञम्}%
{वायो॑ अ॒स्मिन् ह॒विषि॑ मादयस्व। यू॒यं पा॑त स्व॒स्तिभिः॒ सदा॑ नः}%
{उत्तरापरस्यां}{नासिकायां}{वायवे}

\samputa{तें}{सं}%
{व॒यꣳ सो॑म व्र॒ते तव॑। मन॑स्त॒नूषु॒ बिभ्र॑तः}%
{प्र॒जाव॑न्तो अशीमहि}%
{उत्तरस्यां}{हृदये}{सोमाय}

\samputa{रुं}{शं}%
{तमीशा᳚नं॒ जग॑तस्त॒स्थुष॒स्पतिं᳚ धि॒यं॒ जि॒न्वमव॑से हूमहे व॒यम्}%
{पू॒षा नो॒ यथा॒ वेद॑सा॒मस॑द् वृधे र॑क्षि॒ता पा॒युरद॑ब्धः स्व॒स्तये᳚}%
{उत्तरपूर्वस्यां}{नाभौ}{ईशानाय}

\samputa{द्रां}{खं}%
{अ॒स्मे रु॒द्रा मे॒हना॒ पर्व॑तासो वृत्र॒हत्ये॒ भर॑हूतौ स॒जोषाः᳚}%
{यः शंस॑ते स्तुव॒ते धायि॑ प॒ज्र इन्द्र॑ज्येष्ठा अ॒स्माँ अ॑वन्तु दे॒वाः}%
{ऊर्ध्वायां}{मूर्ध्नि}{ब्रह्मणे}

\samputa{यं}{ह्रीं}%
{स्यो॒ना पृ॑थिवि॒ भवा॑ऽनृक्ष॒रा नि॒वेश॑नी}%
{यच्छा॑नः॒ शर्म॑ स॒प्रथाः᳚}%
{अधरायां}{पादयोः}{विष्णवे}

{\small \closesection}
%\newpage
\sect{रौद्रीकरणम्}
\newcounter{rau}
\newcommand{\raudri}[2]{\refstepcounter{rau}%
ॐ #1 रौद्रे॒णानी॑केन पा॒हि मा᳚ऽग्ने पिपृ॒हि मा॒ मा मा॑ हिꣳसीः॥\\
#2 रुद्राय नमः॥\devanumber{\arabic{rau}}॥\\[1ex]}
\raudri{वि॒भूर॑सि प्र॒वाह॑णो॒}{शिखायां}
\raudri{वह्नि॑रसि हव्य॒वाह॑नो॒}{शिरसि}
\raudri{श्वा॒त्रो॑ऽसि॒ प्रचे॑ता॒}{मूर्ध्नि}
\raudri{तु॒थो॑ऽसि वि॒श्ववे॑दा॒}{ललाटे}
\raudri{उ॒शिग॑सि क॒वी}{भ्रुवि}
\raudri{अङ्घा॑रिरसि॒ बम्भा॑री॒}{मुखे}
\raudri{अ॒व॒स्युर॑सि॒ दुव॑स्वा॒न्}{कण्ठे}
\raudri{शु॒न्ध्यूर॑सि मार्जा॒लीयो॒}{बाह्वोः}
\raudri{स॒म्राड॑सि कृ॒शानू॒}{उरसि}
\raudri{प॒रि॒षद्यो॑ऽसि॒ पव॑मानो॒}{हृदये}
\raudri{प्र॒तक्वा॑ऽसि॒ नभ॑स्वा॒न्}{नाभौ}
\raudri{अस॑म्मृष्टोऽसि हव्य॒सूदो॒}{कटौ}
\raudri{ऋ॒तधा॑माऽसि॒ सुव॑र्ज्योती॒}{ऊर्वोः}
\raudri{ब्रह्म॑ज्योतिरसि॒ सुव॑र्धामा॒}{जानुनोः}
\raudri{अ॒जो᳚ऽस्येक॑पा॒द्}{जङ्घयोः}
\raudri{अहि॑रसि बु॒ध्नियो॒}{पादयोः}
(त्वगस्थिगतैः सर्वपापैः प्रमुच्यते। सर्वभूतेष्वपराजितो भवति।)\\
यक्ष-गन्धर्व-भूत-प्रेत-पिशाच-ब्रह्मराक्षस-यमदूत-शाकिनी-डाकिनी-सर्प-श्वापद-तस्कराद्युपद्रवाद्युपघाताः (रुद्राः) सर्वे ज्वलन्तं पश्यन्तु। मां रक्षन्तु॥\\
{\small\closesection}

%\vfill
%\newpage
\sect{चतुर्थः न्यासः – गुह्यादिः मस्तकान्तः}

मनो॒ ज्योति॑र्जुषता॒माज्यं॒ विच्छि॑न्नं य॒ज्ञꣳ समि॒मं द॑धातु।\\
या इ॒ष्टा उ॒षसो॑ नि॒म्रुच॑श्च॒ ताः सन्द॑धामि ह॒विषा॑ घृ॒तेन॑॥\\
गुह्याय नमः॥१॥

अबो᳚ध्य॒ग्निः स॒मिधा॒ जना॑नां॒ प्रति॑ धे॒नुमि॑वाय॒तीमु॒षासम्᳚।\\
य॒ह्वा इ॑व॒ प्रव॒यामु॒ज्जिहा॑नाः॒ प्रभा॒नवः॑ सिस्रते॒ नाक॒मच्छ॑॥\\
नाभ्यै नमः॥२॥

अ॒ग्निर्मू॒र्धा दि॒वः क॒कुत्पतिः॑ पृथि॒व्या अ॒यम्।\\
अ॒पाꣳ रेताꣳ॑सि जिन्वति॥\\
हृदयाय नमः॥३॥

मू॒र्धानं॑ दि॒वो अ॑र॒तिं पृ॑थि॒व्या वै᳚श्वान॒रमृ॒ताय॑ जा॒तम॒ग्निम्।\\
क॒विꣳ स॒म्राज॒मति॑थिं॒ जना॑नामा॒सन्ना पात्रं॑ जनयन्त दे॒वाः॥\\
कण्ठाय नमः॥४॥

मर्मा॑णि ते॒ वर्म॑भिश्छादयामि॒ सोम॑स्त्वा॒ राजा॒ऽमृते॑ना॒भिव॑स्ताम्।\\
उ॒रोर्वरी॑यो॒ वरि॑वस्ते अस्तु॒ जय॑न्तं॒ त्वामनु॑ मदन्तु दे॒वाः॥\\
मुखाय नमः॥५॥

जा॒तवे॑दा॒ यदि॑ वा पाव॒कोऽसि॑। वै॒श्वा॒न॒रो यदि॑ वा वैद्यु॒तोऽसि॑।\\
शं प्र॒जाभ्यो॒ यज॑मानाय लो॒कम्। ऊर्जं॒ पुष्टिं॒ दद॑द॒भ्याव॑वृत्स्व॥\\
शिरसे नमः॥६॥

{\small \closesection}

\newcounter{ssk}
\newcommand{\ssankalpa}[1]{\refstepcounter{ssk}
#1तन्मे॒ मनः॑ शि॒वस॑ङ्क॒ल्पम॑स्तु॥\devanumber{\arabic{ssk}}॥}
\newcommand{\ssankalpaalign}[2]{
\setcounter{shlokacount}{\value{ssk}}
\twolineshloka{#1}{#2 तन्मे॒ मनः॑ शि॒वस॑ङ्क॒ल्पम॑स्तु}
\refstepcounter{ssk}
\pagebreak[0]}

\sect{पञ्चमे न्यासे शिवसङ्कल्पः}
\begin{center}
\ssankalpa{यज्जाग्र॑तो दू॒रमु॒दैति॒ दैवं॒ तदु॑ सु॒प्तस्य॒ तथै॒वैति॑। दू॒र॒ङ्ग॒मं ज्योति॑षां॒ ज्योति॒रेकं॒}

\ssankalpa{येन॒ कर्मा᳚ण्य॒पसो॑ मनी॒षिणो॑ य॒ज्ञे कृ॒ण्वन्ति॑ वि॒दथे॑षु॒ धीराः᳚। यद॑पू॒र्वं य॒क्षम॒न्तः प्र॒जानां॒}

\ssankalpa{यत् प्र॒ज्ञान॑मु॒त चेतो॒ धृति॑श्च॒ यज्ज्योति॑र॒न्तर॒मृतं॑ प्र॒जासु॑। यस्मा॒न्न ऋ॒ते किं च॒न कर्म॑ क्रि॒यते॒}

\ssankalpa{येने॒दं भू॒तं भुव॑नं भवि॒ष्यत् परि॑गृहीतम॒मृते॑न॒ सर्वम्᳚}
{येन॑ य॒ज्ञस्ता॒यते॑ स॒प्तहो॑ता॒}

\ssankalpa{यस्मि॒नृचः॒ साम॒ यजूꣳ॑षि यस्मि॒न् प्रति॑ष्ठिता रथना॒भावि॑वा॒राः। यस्मिꣴ॑श्चि॒त्तꣳ सर्व॒मोतं॑ प्र॒जानां॒}

\ssankalpa{सु॒षा॒र॒थिरश्वा॑निव॒ यन्म॑नु॒ष्या᳚न्नेनी॒यते॒ऽभीशु॑भिर्वा॒जिन॑ इव। हृत्प्रति॑ष्ठं॒ यद॑जिरं॒ जवि॑ष्ठं॒}

हृदयाय नमः॥
\end{center}
{\small \closesection}

\sect{पञ्चमे न्यासे पुरुषसूक्तम्}
स॒हस्र॑शीर्‌षा॒ पुरु॑षः। … यत्र॒ पूर्वे॑ सा॒ध्याः सन्ति॑ दे॒वाः॥

शिरसे स्वाहा॥

{\small \closesection}

\sect{पञ्चमे न्यासे उत्तरनारायणम्}

अ॒द्भ्यः सम्भू॑तः पृथि॒व्यै रसा᳚च्च। … सर्वं॑ मनिषाण॥

शिखायै वषट्॥

{\small \closesection}
\sect{पञ्चमे न्यासे अप्रतिरथम्}

आ॒शुः शिशा॑नो वृष॒भो न यु॒ध्मो घ॑नाघ॒नः क्षोभ॑णश्चर्‌षणी॒नाम्। … इन्द्रो॑ न॒स्तत्र॑ वृत्र॒हा वि॑श्वा॒हा शर्म॑ यच्छतु॥

कवचाय हुम्॥

{\small \closesection}

\sect{पञ्चमे न्यासे शतरुद्रीयस्य रूपम्}
त्वम॑ग्ने रु॒द्रो असु॑रो म॒हो दि॒वः। त्वꣳ शर्धो॒ मारु॑तं पृ॒क्ष ई॑शिषे।
त्वं वातै॑ररु॒णैर्या॑सि शङ्ग॒यः। त्वं पू॒षा वि॑ध॒तः पा॑सि॒ नु त्मना᳚।
देवा॑ दे॒वेषु॑ श्रयध्वम्। प्रथ॑मा द्वि॒तीये॑षु श्रयध्वम्।
द्विती॑यास्तृ॒तीये॑षु श्रयध्वम्। … ए॒क॒त्रि॒ꣳ॒शा द्वा᳚त्रि॒ꣳ॒शेषु॑ श्रयध्वम्।
द्वा॒त्रि॒ꣳ॒शास्त्र॑यस्त्रि॒ꣳ॒शेषु॑ श्रयध्वम्। देवा᳚स्त्रिरेकादशा॒स्त्रिस्त्र॑यस्त्रिꣳशाः।
उत्त॑रे भवत। उत्त॑रवर्त्मान॒ उत्त॑रसत्वानः।
यत्का॑म इ॒दं जु॒होमि॑। तन्मे॒ समृ॑ध्यताम्।
व॒यꣴ स्या॑म॒ पत॑यो रयी॒णाम्। भूर्भुवः॒ स्वः॑ स्वाहा᳚॥\\
अस्त्राय फट्॥\\

%{\small \closesection}
%\clearpage

\sect{पञ्चाङ्ग-जपः}
ह॒ꣳ॒सः शु॑चि॒षद्वसु॑रन्तरिक्ष॒सद्धोता॑ वेदि॒षदति॑थिर्दुरोण॒सत्।\\
नृ॒षद्व॑र॒सदृ॑त॒सद्व्यो॑म॒सद॒ब्जा गो॒जा ऋ॑त॒जा अ॑द्रि॒जा ऋ॒तं बृ॒हत्॥

प्रतद्विष्णुः॑ स्तवते वी॒र्या॑य। मृ॒गो न भी॒मः कु॑च॒रो गि॑रि॒ष्ठाः।\\
यस्यो॒रुषु॑ त्रि॒षु वि॒क्रम॑णेषु। अधि॑क्षि॒यन्ति॒ भुव॑नानि॒ विश्वा᳚॥

त्र्य॑म्बकं यजामहे सुग॒न्धिं पु॑ष्टि॒वर्ध॑नम्।\\
उ॒र्वा॒रु॒कमि॑व॒ बन्ध॑नान्मृ॒त्योर्मु॑क्षीय॒ माऽमृता᳚त्॥

तथ्स॑वि॒तुर्वृ॑णीमहे। व॒यं दे॒वस्य॒ भोज॑नम्‌।\\
श्रेष्ठꣳ॑ सर्व॒धात॑मम्‌। तुरं॒ भग॑स्य धीमहि॥

विष्णु॒र्योनिं॑ कल्पयतु॒ त्वष्टा॑ रू॒पाणि॑ पिꣳशतु।\\
आसि॑ञ्चतु प्र॒जाप॑तिर्धा॒ता गर्भं॑ दधातु ते॥

{\small \closesection}

\sect{अष्टाङ्ग-नमस्काराः}
\resetShloka

\twolineshloka*
{हि॒र॒ण्य॒ग॒र्भः सम॑वर्त॒ताग्रे॑ भू॒तस्य॑ जा॒तः पति॒रेक॑ आसीत्}
{सदा॑धार पृथि॒वीं द्यामु॒तेमां कस्मै॑ दे॒वाय॑ ह॒विषा॑ विधेम}
उरसा नमः॥

\twolineshloka*
{यः प्रा॑ण॒तो नि॑मिष॒तो म॑हि॒त्वैक॒ इद्राजा॒ जग॑तो ब॒भूव॑}
{य ईशे॑ अ॒स्य द्वि॒पद॒श्चतु॑ष्पदः॒ कस्मै॑ दे॒वाय॑ ह॒विषा॑ विधेम}
शिरसा नमः॥

\twolineshloka
{ब्रह्म॑ जज्ञा॒नं प्र॑थ॒मं पु॒रस्ता॒द् विसी॑म॒तः सु॒रुचो॑ वे॒न आ॑वः}
{स बु॒ध्निया॑ उप॒मा अ॑स्य वि॒ष्ठाः स॒तश्च॒ योनि॒मस॑तश्च॒ विवः॑}
दृष्ट्या नमः॥

म॒ही द्यौः पृ॑थि॒वी च॑ न इ॒मं य॒ज्ञं मि॑मिक्षताम्। पि॒पृ॒तान्नो॒ भरी॑मभिः॥\\
मनसा नमः॥

\twolineshloka*
{उप॑श्वासय पृथि॒वीमु॒त द्यां पु॑रु॒त्रा ते॑ मनुतां॒ विष्ठि॑तं॒ जग॑त्}
{स दु॑न्दुभे स॒जूरिन्द्रे॑ण दे॒वैर्दू॒राद्दवी॑यो॒ अप॑सेध॒ शत्रून्॑}
वचसा नमः॥

अग्ने॒ नय॑ सु॒पथा॑ रा॒ये अ॒स्मान् विश्वा॑नि देव व॒युना॑नि वि॒द्वान्।
यु॒यो॒ध्य॑स्मज्जु॑हुरा॒णमेनो॒ भूयि॑ष्ठां ते॒ नम॑ उक्तिं विधेम॥\\
पद्भ्यां नमः॥

या ते॑ अग्ने॒ रुद्रि॑या त॒नूस्तया॑ नः पाहि॒ तस्या᳚स्ते॒ स्वाहा᳚॥\\
कराभ्यां नमः॥

इ॒मं य॑मप्रस्त॒रमाहि सीदाऽङ्गि॑रोभिः पि॒तृभिः॑ संविदा॒नः। आत्वा॒ मन्त्राः᳚ कविश॒स्ता व॑हन्त्वे॒ना रा॑जन् ह॒विषा॑ मादयस्व॥\\
कर्णाभ्यां नमः॥
{\small \closesection}
%\pagebreak[4]

\sect{ध्यानम्}
अथा·त्मानं (शिवा·त्मानं) श्री-रुद्र-रूपं ध्यायेत्॥

शुद्ध-स्फटिक-सङ्काशं त्रि-नेत्रं पञ्च-वक्त्रकम्।\\
गङ्गा-धरं दश-भुजं सर्वा·भरण-भूषितम्॥

नील-ग्रीवं शशाङ्का·ङ्कं नाग-यज्ञो·पवीतिनम्।\\
व्याघ्र-चर्मो·त्तरीयं च वरेण्य·मभय-प्रदम्॥

कमण्ड·ल्वक्षसूत्राभ्या·मन्वितं शूल-/पाणिनम्।\\
ज्वलन्तं पिङ्गल-जटा-शिखा-मध्यो·द-धारिणम्॥

अमृतेना·प्लुतं हृष्ट·मुमा-देहा·र्ध-धारिणम्।\\
दिव्य-सिंहा·सनासीनं दिव्य-भोग-समन्वितम्॥

दिग्-देवता-समायुक्तं सुरा·सुर-नमस्कृतम्।\\
नित्यं च शाश्वतं शुद्धं ध्रुव·मक्षर·मव्ययम्।\\
सर्व-व्यापिन·मीशानं रुद्रं वै विश्व-रूपिणम्॥

[एवं ध्यात्वा द्विजः सम्यक् ततो यजन·मारभेत्॥]

\sect{षष्ठे न्यासे देवता-स्थापनम्}

[शिवस्य दक्षिण-प्रत्यग्-देशे देवा·भिमुखः स्थित्वा आत्मनि देवताः स्थापयेत्।]

प्रजनने ब्रह्मा तिष्ठतु। पादयो\dot{}र्विष्णु\dot{}स्तिष्ठतु। हस्तयो\dot{}र्हर\dot{}स्तिष्ठतु। बाह्वो\dot{}रिन्द्र\dot{}स्तिष्ठतु। जठरेऽग्नि\dot{}स्तिष्ठतु। हृदये शिव\dot{}स्तिष्ठतु। कण्ठे वसव\dot{}स्तिष्ठन्तु। वक्त्रे सरस्वती तिष्ठतु। नासिकयो\dot{}र्वायु\dot{}स्तिष्ठतु। नयनयो\dot{}श्चन्द्रा·दित्यौ तिष्ठताम्। कर्णयो\dot{}रश्विनौ तिष्ठताम्। ललाटे रुद्रा\dot{}स्तिष्ठन्तु। मू·र्ध्न्यादित्या\dot{}स्तिष्ठन्तु। शिरसि महादेव\dot{}स्तिष्ठतु। शिखायां वामदेव\dot{}स्तिष्ठतु। पृष्ठे पिनाकी तिष्ठतु। पुरतः शूली तिष्ठतु। पार्श्वयोः शिवा-शङ्करौ तिष्ठताम्। सर्वतो वायु\dot{}स्तिष्ठतु। ततो बहिः सर्वतोऽग्नि\dot{}र्ज्वाला-माला-परिवृत\dot{}स्तिष्ठतु। सर्वे\dot{}ष्वङ्गेषु सर्वा देवता यथा-स्थानं तिष्ठन्तु। मां रक्षन्तु॥

\sect{षष्ठे न्यासे सम्मर्शनम्}

अ॒ग्निर्मे॑ वा॒चि श्रि॒तः। वाग्घृद॑ये। हृद॑यं॒ मयि॑। अ॒हम॒मृते᳚। अ॒मृतं॒ ब्रह्म॑णि। [जिह्वाम्]

वा॒युर्मे᳚ प्रा॒णे श्रि॒तः। प्रा॒णो हृद॑ये। हृद॑यं॒ मयि॑। अ॒हम॒मृते᳚। अ॒मृतं॒ ब्रह्म॑णि। [नासिकाम्]

सूर्यो॑ मे॒ चक्षु॑षि श्रि॒तः। चक्षु॒र्‌॒हृद॑ये। हृद॑यं॒ मयि॑। अ॒हम॒मृते᳚। अ॒मृतं॒ ब्रह्म॑णि। [नेत्रे]

च॒न्द्रमा॑ मे॒ मन॑सि श्रि॒तः। मनो॒ हृद॑ये। हृद॑यं॒ मयि॑। अ॒हम॒मृते᳚। अ॒मृतं॒ ब्रह्म॑णि। [वक्षः]

दिशो॑ मे॒ श्रोत्रे᳚ श्रि॒ताः। श्रोत्र॒ꣳ॒ हृद॑ये। हृद॑यं॒ मयि॑। अ॒हम॒मृते᳚। अ॒मृतं॒ ब्रह्म॑णि। [श्रोत्रे]

आपो॑ मे॒ रेत॑सि श्रि॒ताः। रेतो॒ हृद॑ये। हृद॑यं॒ मयि॑। अ॒हम॒मृते᳚। अ॒मृतं॒ ब्रह्म॑णि। [गुह्यम्]

पृ॒थि॒वी मे॒ शरी॑रे श्रि॒ता। शरी॑र॒ꣳ॒ हृद॑ये। हृद॑यं॒ मयि॑। अ॒हम॒मृते᳚। अ॒मृतं॒ ब्रह्म॑णि। [शरीरम्]

ओ॒ष॒धि॒व॒न॒स्प॒तयो॑ मे॒ लोम॑सु श्रि॒ताः। लोमा॑नि॒ हृद॑ये। हृद॑यं॒ मयि॑। अ॒हम॒मृते᳚। अ॒मृतं॒ ब्रह्म॑णि। [लोमानि]

इन्द्रो॑ मे॒ बले᳚ श्रि॒तः। बल॒ꣳ॒ हृद॑ये। हृद॑यं॒ मयि॑। अ॒हम॒मृते᳚। अ॒मृतं॒ ब्रह्म॑णि। [बाहू]

प॒र्जन्यो॑ मे मू॒र्ध्नि श्रि॒तः। मू॒र्धा हृद॑ये। हृद॑यं॒ मयि॑। अ॒हम॒मृते᳚। अ॒मृतं॒ ब्रह्म॑णि। [शिरः]

ईशा॑नो मे म॒न्यौ श्रि॒तः। म॒न्युर्‌हृद॑ये। हृद॑यं॒ मयि॑। अ॒हम॒मृते᳚। अ॒मृतं॒ ब्रह्म॑णि। [हृदयम्]

आ॒त्मा म॑ आ॒त्मनि॑ श्रि॒तः। आ॒त्मा हृद॑ये। हृद॑यं॒ मयि॑। अ॒हम॒मृते᳚। अ॒मृतं॒ ब्रह्म॑णि। [हृदयम्]

पुन॑र्म आ॒त्मा पुन॒रायु॒रागा᳚त्। पुनः॑ प्रा॒णः पुन॒राकू॑त॒मागा᳚त्। वै॒श्वा॒न॒रो र॒श्मिभि॑र्वावृधा॒नः। अ॒न्तस्ति॑ष्ठत्व॒मृत॑स्य गो॒पाः॥ [सर्वाण्यङ्गानि]
{\small \closesection}

\sect{आत्मपूजा}

\twolineshloka*
{आराधितो मनुष्यै·स्त्वं सिद्धै·र्देवासुरादिभिः}
{आराधयामि भक्त्या त्वां मां गृहाण महेश्वर}

(त्र्यम्बकं यजामहे इति च।)

[मनसा आत्मनि श्रीरुद्रम् आवाह्य मनसा गन्ध-अक्षत-पत्र-पुष्प-धूप-दीप-नैवेद्य-ताम्बूलैः अर्चयेत्।]

\sect{शिवपूजा - पूर्वाङ्गानि}

आ त्वा॑ वहन्तु॒ हर॑यः॒ सचे॑तसः श्वे॒तैरश्वैः᳚ स॒ह के॑तु॒मद्भिः॑।\\
वाता॑जितै॒·र्बल॑वद्भि॒·र्मनो॑जवै॒·राया॑हि शी॒घ्रं मम॑ ह॒व्याय॑ श॒र्वोम्॥\\
श्रीरुद्रम् आवाहयामि।

स॒द्यो जा॒तं प्र॑पद्यामि। – स्वागतं करोमि।

स॒द्योजा॒ताय॒ वै नमो॒ नमः॑। – आसनं समर्पयामि।

भ॒वे भ॑वे॒ नाति॑ भवे भवस्व॒ माम्। – पाद्यं समर्पयामि।

भ॒वोद्भ॑वाय॒ नमः॑। – अर्घ्यं समर्पयामि।

(आचमनीयं समर्पयामि।)

आपो॒ हि ष्ठा + ज॒नय॑था च नः। [स्नपनम् आरभेत]

भवं देवं तर्पयामि। शर्वं देवं तर्पयामि। ईशानं देवं तर्पयामि। पशुपतिं देवं तर्पयामि। रुद्रं देवं तर्पयामि। उग्रं देवं तर्पयामि। भीमं देवं तर्पयामि। महान्तं देवं तर्पयामि।

भवस्य देवस्य पत्नीं तर्पयामि। शर्वस्य देवस्य पत्नीं तर्पयामि। ईशानस्य देवस्य पत्नीं तर्पयामि। पशुपतेर्देवस्य पत्नीं तर्पयामि। रुद्रस्य देवस्य पत्नीं तर्पयामि। उग्रस्य देवस्य पत्नीं तर्पयामि। भीमस्य देवस्य पत्नीं तर्पयामि। महतो देवस्य पत्नीं तर्पयामि।

\sect{शिवपूजा - अभिषेकः}

अस्य श्री-रुद्राध्याय-प्रश्न-महामन्त्रस्य। अघोर ऋषिः। अनुष्टुप् छन्दः। सङ्कर्षण-मूर्ति-स्वरूपो योऽसावादित्यः परम-पुरुषः स एष रुद्रो देवता॥

नमः शिवाये·ति बीजम्। शिवतराये·ति शक्तिः। महादेवाये·ति कीलकम्। श्री-साम्ब-सदाशिव-प्रसाद-सिद्ध्यर्थे जपे विनियोगः॥

\centerline{॥करन्यास-अङ्गन्यासौ॥}
अग्निहोत्रात्मने, दर्शपूर्णमासात्मने, चातुर्मास्यात्मने, निरूढपशुबन्धात्मने, ज्योतिष्टोमात्मने, सर्वक्रत्वात्मने॥

भूर्भुवः सुवरोम् इति दिग्बन्धः।\\

\sect{ध्यानम्}
\setlength{\shlokaspaceskip}{2pt}
\fourlineindentedshloka*
{आपाताल-नभः-स्थलान्त-भुवन-ब्रह्माण्ड·माविस्फुरत्-}
{ज्योतिः-स्फाटिक-लिङ्ग-मौलि-विलसत्-पूर्णेन्दु-वान्ता·मृतैः}
{अस्तोका·प्लुत·मेक·मीश·मनिशं रुद्रा·नुवाकान् जपन्}
{ध्याये·दीप्सित-सिद्धये ध्रुव-पदं विप्रोऽभिषिञ्चे·च्छिवम्}

\fourlineindentedshloka*
{ब्रह्माण्ड-व्याप्त-देहा भसित-हिमरुचा भासमाना भुजङ्गैः}
{कण्ठे-कालाः कपर्दा·कलित-शशि-कला·श्चण्ड-कोदण्ड-हस्ताः}
{त्र्यक्षा रुद्राक्ष-भूषाः प्रणत-भय-हराः शाम्भवा मूर्ति-भेदाः}
{रुद्राः श्री-रुद्र-सूक्त-प्रकटित-विभवा नः प्रयच्छन्तु सौख्यम्}

नम॑स्ते रुद्र + मुखा॑ कृधि।

स॒हस्रा॑णि सहस्र॒शो + तन्मसि।\\
अ॒स्मिन् म॑ह॒ति + तन्मसि।\\
नील॑ग्रीवाः … श॒र्वा + तन्मसि।\\
नील॑ग्रीवाः॒ … दिवं॒ + तन्मसि।\\
ये वृ॒क्षेषु॒ + तन्मसि।\\
ये भू॒तानां॒ + तन्मसि।\\
ये अन्ने॑षु॒ + तन्मसि।\\
ये प॒थां + तन्मसि।\\
ये ती॒र्थानि॒ + तन्मसि।\\
य ए॒ताव॑न्तश्च + तन्मसि।

नमो॑ रु॒द्रेभ्यो॒ ये पृ॑थि॒व्यां येषा॒मन्न॒मिष॑व॒स्तेभ्यो॒ … दधामि।\\
नमो॑ रु॒द्रेभ्यो॒ ये᳚ऽन्तरि॑क्षे॒ येषां॒ वात॒ इष॑व॒स्तेभ्यो॒ … दधामि।\\
नमो॑ रु॒द्रेभ्यो॒ ये दि॒वि येषां᳚ व॒र्षमिष॑व॒स्तेभ्यो॒ … दधामि।

त्र्य॑म्बकं … सदाशि॒वोम्।

अग्ना॑विष्णू … अधि॑पतिश्च।

इडा॑ देव॒हूः … पि॒तरोऽनु॑ मदन्तु।

\sect{शिवपूजा - उत्तराङ्गानि}

वा॒म॒दे॒वाय॒ नमः॑ – वस्त्रं समर्पयामि।

ज्ये॒ष्ठाय॒ नमः॑। – उत्तरीयं समर्पयामि।

श्रे॒ष्ठाय॒ नमः॑। – यज्ञोपवीतं समर्पयामि।

रु॒द्राय॒ नमः॑। – गन्धं सर्पयामि।

काला॑य॒ नमः॑। – अक्षतान् समर्पयामि।

कल॑विकरणाय॒ नमः॑। – पुष्पं समर्पामि। [अत्र नामावलिं पठेत्।]

बल॑विकरणाय॒ नमः॑। – धूपम् आघ्रापयामि।

बला॑य॒ नमः॑। – दीपं दर्शयामि।

बल॑प्रमथनाय॒ नमः॑। – नैवेद्यं निवेदयामि।

सर्व॑भूतदमनाय॒ नमः॑। – ताम्बूलं समर्पयामि।

म॒नोन्म॑नाय॒ नमः॑। – नीराजनं समर्पयामि।

अ॒घोरे᳚भ्योऽथ॒ घोरे᳚भ्यो॒ घोर॒घोर॑तरेभ्यः। सर्वे᳚भ्यः सर्व॒शर्वे᳚भ्यो॒ नम॑स्ते अस्तु रु॒द्ररू॑पेभ्यः॥ – प्रदक्षिणनमस्कारान् समर्पयामि।

तत्पुरु॑षाय वि॒द्महे॑ महादे॒वाय॑ धीमहि। तन्नो॑ रुद्रः प्रचो॒दया᳚त्॥ [दशावरं जपेत्]

ईशानः सर्व॑विद्या॒ना॒मीश्वरः सर्व॑भूता॒नां॒ ब्रह्माधि॑पति॒\dot{}र्ब्रह्म॒णोऽधि॑पति॒\dot{}र्ब्रह्मा॑ शि॒वो मे॑ अस्तु सदाशि॒वोम्॥ – प्रार्थनाः समर्पयामि।

\centerline{॥ॐ शान्तिः॒ शान्तिः॒ शान्तिः॑॥}

{\small \closesection}
