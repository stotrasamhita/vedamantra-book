\chapt{काण्डम् ५}
\sect{प्रथमः प्रश्नः}\setcounter{anuvakam}{0}
\dnsub{तैत्तिरीयसंहितायां पञ्चमकाण्डे प्रथमः प्रश्नः}
%5.1.1.1
सा॒वि॒त्राणि॑ जुहोति॒ प्रसू᳚त्यै चतुर्गृही॒तेन॑ जुहोति॒ चतु॑ष्पादः प॒शवः॑ प॒शूने॒वाव॑ रुन्धे॒ चत॑स्रो॒ दिशो॑ दि॒क्ष्वे॑व प्रति॑ तिष्ठति॒ छन्दाꣳ॑सि दे॒वेभ्यो\-ऽपा᳚क्राम॒न्न वो॑\-ऽभा॒गानि॑ ह॒व्यं व॑क्ष्याम॒ इति॒ तेभ्य॑ ए॒तच्च॑तुर्गृही॒तम॑धारयन् पुरोनुवा॒क्या॑यै या॒ज्या॑यै दे॒वता॑यै वषट्का॒राय॒ यच्च॑तुर्गृही॒तं जु॒होति॒ छन्दाꣴ॑स्ये॒व तत्प्री॑णाति॒ तान्य॑स्य प्री॒तानि॑ दे॒वेभ्यो॑ ह॒व्यं व॑हन्ति॒ यं का॒मये॑त~(१)

%5.1.1.2
पापी॑यान्थ्स्या॒दित्येकै॑कं॒ तस्य॑ जुहुया॒दाहु॑तीभिरे॒वैन॒मप॑ गृह्णाति॒ पापी॑यान्भवति॒ यं का॒मये॑त॒ वसी॑यान्थ्स्या॒दिति॒ सर्वा॑णि॒ तस्या॑नु॒द्रुत्य॑ जुहुया॒दाहु॑त्यै॒वैन॑म॒भि क्र॑मयति॒ वसी॑यान्भव॒त्यथो॑ य॒ज्ञस्यै॒वैषाभिक्रा᳚न्ति॒रेति॒ वा ए॒ष य॑ज्ञमु॒खादृद्ध्या॒ यो᳚\-ऽग्नेर्दे॒वता॑या॒ एत्य॒ष्टावे॒तानि॑ सावि॒त्राणि॑ भवन्त्य॒ष्टाक्ष॑रा गाय॒त्री गा॑य॒त्रः~(२)

%5.1.1.3
अ॒ग्निस्तेनै॒व य॑ज्ञमु॒खादृद्ध्या॑ अ॒ग्नेर्दे॒वता॑यै॒ नैत्य॒ष्टौ सा॑वि॒त्राणि॑ भव॒न्त्याहु॑तिर्नव॒मी त्रि॒वृत॑मे॒व य॑ज्ञमु॒खे वि या॑तयति॒ यदि॑ का॒मये॑त॒ छन्दाꣳ॑सि यज्ञयश॒सेना᳚र्पयेय॒मित्यृच॑मन्त॒मां कु॑र्या॒च्छन्दाꣴ॑स्ये॒व य॑ज्ञयश॒सेना᳚र्पयति॒ यदि॑ का॒मये॑त॒ यज॑मानं यज्ञयश॒सेना᳚र्पयेय॒मिति॒ यजु॑रन्त॒मं कु॑र्या॒द्यज॑मानमे॒व य॑ज्ञयश॒सेना᳚र्पयत्यृ॒चा स्तोम॒ꣳ॒ सम॑र्ध॒येति॑~(३)

%5.1.1.4
आ॒ह॒ समृ॑द्ध्यै च॒तुर्भि॒रभ्रि॒मा द॑त्ते च॒त्वारि॒ छन्दाꣳ॑सि॒ छन्दो॑भिरे॒व दे॒वस्य॑ त्वा सवि॒तुः प्र॑स॒व इत्या॑ह॒ प्रसू᳚त्या अ॒ग्निर्दे॒वेभ्यो॒ निला॑यत॒ स वेणुं॒ प्रावि॑श॒थ्स ए॒तामू॒तिमनु॒ सम॑चर॒द्यद्वेणोः᳚ सुषि॒रꣳ सु॑षि॒राभ्रि॑र्भवति सयोनि॒त्वाय॒ स यत्र॑य॒त्राव॑स॒त्तत्कृ॒ष्णम॑भवत्कल्मा॒षी भ॑वति रू॒पस॑मृद्ध्या उभयतः॒क्ष्णूर्भ॑वती॒तश्चा॒मुत॑श्चा॒र्कस्याव॑रुद्ध्यै व्याममा॒त्री भ॑वत्ये॒ताव॒द्वै पुरु॑षे वी॒र्यं॑ वी॒र्य॑सम्मि॒ता\-ऽप॑रिमिता भव॒त्यप॑रिमित॒स्याव॑रुद्ध्यै॒ यो वन॒स्पती॑नाम्फल॒ग्रहिः॒ स ए॑षां वी॒र्या॑वान्फल॒ग्रहि॒र्वेणु॑र्वैण॒वी भ॑वति वी॒र्य॑स्याव॑रुद्ध्यै॥~(४)

{\anuvakamend[{का॒मये॑त गाय॒त्रो᳚\-ऽर्ध॒येति॑ च स॒प्तविꣳ॑शतिश्च}]}%~(१)

%5.1.2.1
व्यृ॑द्धं॒ वा ए॒तद्य॒ज्ञस्य॒ यद॑य॒जुष्के॑ण क्रि॒यत॑ इ॒माम॑गृभ्णन्रश॒नामृ॒तस्येत्य॑श्वाभि॒धानी॒मा द॑त्ते॒ यजु॑ष्कृत्यै य॒ज्ञस्य॒ समृ॑द्ध्यै॒ प्रतू᳚र्तं वाजि॒न्ना द्र॒वेत्यश्व॑म॒भि द॑धाति रू॒पमे॒वास्यै॒तन्म॑हि॒मानं॒ व्याच॑ष्टे यु॒ञ्जाथा॒ꣳ॒ रास॑भं यु॒वमिति॑ गर्द॒भमस॑त्ये॒व ग॑र्द॒भं प्रति॑\-ष्ठापयति॒ तस्मा॒दश्वा᳚द्गर्द॒भो\-ऽस॑त्तरो॒ योगे॑योगे त॒वस्त॑र॒मित्या॑ह~(५)

%5.1.2.2
योगे॑योग ए॒वैनं॑ युङ्क्ते॒ वाजे॑वाजे हवामह॒ इत्या॒हान्नं॒ वै वाजो\-ऽन्न॑मे॒वाव॑ रुन्धे॒ सखा॑य॒ इन्द्र॑मू॒तय॒ इत्या॑हेन्द्रि॒यमे॒वाव॑ रुन्धे॒\-ऽग्निर्दे॒वेभ्यो॒ निला॑यत॒ तं प्र॒जा\-प॑ति॒रन्व॑विन्दत्प्राजाप॒त्यो\-ऽश्वो\-ऽश्वे॑न॒ सम्भ॑र॒त्यनु॑वित्त्यै पापवस्य॒सं वा ए॒तत्क्रि॑यते॒ यच्छ्रेय॑सा च॒ पापी॑यसा च समा॒नं कर्म॑ कु॒र्वन्ति॒ पापी॑यान्~(६)

%5.1.2.3
ह्यश्वा᳚द्गर्द॒भो\-ऽश्वं॒ पूर्वं॑ नयन्ति पापवस्य॒सस्य॒ व्यावृ॑त्त्यै॒ तस्मा॒च्छ्रेयाꣳ॑सं॒ पापी॑यान्प॒श्चादन्वे॑ति ब॒हुर्वै भव॑तो॒ भ्रातृ॑व्यो॒ भव॑तीव॒ खलु॒ वा ए॒ष यो᳚\-ऽग्निं चि॑नु॒ते व॒ज्र्यश्वः॑ प्र॒तूर्व॒न्नेह्य॑व॒क्राम॒न्नश॑स्ती॒रित्या॑ह॒ वज्रे॑णै॒व पा॒प्मान॒म्भ्रातृ॑व्य॒मव॑ क्रामति रु॒द्रस्य॒ गाण॑पत्या॒दित्या॑ह रौ॒द्रा वै प॒शवो॑ रु॒द्रादे॒व~(७)

%5.1.2.4
प॒शून्नि॒र्याच्या॒\-ऽऽ\-त्मने॒ कर्म॑ कुरुते पू॒ष्णा स॒युजा॑ स॒हेत्या॑ह पू॒षा वा अध्व॑नाꣳ सन्ने॒ता सम॑ष्ट्यै॒ पुरी॑षायतनो॒ वा ए॒ष यद॒ग्निरङ्गि॑रसो॒ वा ए॒तमग्रे॑ दे॒वता॑ना॒ꣳ॒ सम॑भरन्पृथि॒व्याः स॒धस्था॑द॒ग्निं पु॑री॒ष्य॑मङ्गिर॒\-स्वदच्छे॒हीत्या॑ह॒ साय॑तनमे॒वैनं॑ दे॒वता॑भिः॒ सम्भ॑रत्य॒ग्निं पु॑री॒ष्य॑मङ्गिर॒\-स्वदच्छे॑म॒ इत्या॑ह येन॑~(८)

%5.1.2.5
स॒ङ्गच्छ॑ते॒ वाज॑मे॒वास्य॑ वृङ्क्ते प्र॒जा\-प॑तये प्रति॒प्रोच्या॒ग्निः स॒म्भृत्य॒ इत्या॑हुरि॒यं वै प्र॒जा\-प॑ति॒स्तस्या॑ ए॒तच्छ्रोत्रं॒ यद्व॒ल्मीको॒\-ऽग्निं पु॑री॒ष्य॑मङ्गिर॒\-स्वद्भ॑रिष्याम॒ इति॑ वल्मीकव॒पामुप॑ तिष्ठते सा॒क्षादे॒व प्र॒जा\-प॑तये प्रति॒प्रोच्या॒ग्निꣳ सम्भ॑रत्य॒ग्निं पु॑री॒ष्य॑मङ्गिर॒\-स्वद्भ॑राम॒ इत्या॑ह॒ येन॑ स॒ङ्गच्छ॑ते॒ वाज॑मे॒वास्य॑ वृ॒ङ्क्ते\-ऽन्व॒ग्निरु॒षसा॒मग्रम्᳚~(९)

%5.1.2.6
अ॒ख्य॒दित्या॒हानु॑ख्यात्या आ॒गत्य॑ वा॒ज्यध्व॑न आ॒क्रम्य॑ वाजिन्पृथि॒वीमित्या॑हे॒च्छत्ये॒वैनं॒ पूर्व॑या वि॒न्दत्युत्त॑रया॒ द्वाभ्या॒मा क्र॑मयति॒ प्रति॑ष्ठित्या॒ अनु॑रूपाभ्या॒न्तस्मा॒दनु॑रूपाः प॒शवः॒ प्र जा॑यन्ते॒ द्यौस्ते॑ पृ॒ष्ठं पृ॑थि॒वी स॒धस्थ॒मित्या॑है॒भ्यो वा ए॒तं लो॒केभ्यः॑ प्र॒जा\-प॑तिः॒ समै॑रयद्रू॒पमे॒वास्यै॒तन्म॑हि॒मानं॒ व्याच॑ष्टे व॒ज्री वा ए॒ष यदश्वो॑ द॒द्भिर॒न्यतो॑दद्भ्यो॒ भूया॒ल्लोँम॑भिरुभ॒याद॑द्भ्यो॒ यं द्वि॒ष्यात्तम॑धस्प॒दं ध्या॑ये॒द्वज्रे॑णै॒वैनꣴ॑ स्तृणुते॥10॥

{\anuvakamend[{आ॒ह॒ पापी॑यान्रु॒द्रादे॒व येनाग्रं॑ व॒ज्री वै स॒प्तद॑श च}]}%~(२)

%5.1.3.1
उत्क्रा॒मोद॑क्रमी॒दिति॒ द्वाभ्या॒मुत्क्र॑मयति॒ प्रति॑ष्ठित्या॒ अनु॑रूपाभ्या॒न्तस्मा॒दनु॑रूपाः प॒शवः॒ प्र जा॑यन्ते॒\-ऽप उप॑ सृजति॒ यत्र॒ वा आप॑ उप॒गच्छ॑न्ति॒ तदोष॑धयः॒ प्रति॑ तिष्ठ॒न्त्योष॑धीः प्रति॒तिष्ठ॑न्तीः प॒शवो\-ऽनु॒ प्रति॑ तिष्ठन्ति प॒शून् य॒ज्ञो य॒ज्ञं यज॑मानो॒ यज॑मानं प्र॒जास्तस्मा॑द॒प उप॑ सृजति॒ प्रति॑ष्ठित्यै॒ यद॑ध्व॒र्युर॑न॒ग्नावाहु॑तिं जुहु॒याद॒न्धो᳚\-ऽध्व॒र्युः~(११)

%5.1.3.2
स्या॒द्रक्षाꣳ॑सि य॒ज्ञꣳ ह॑न्यु॒र्॒\mbox{}हिर॑ण्यमु॒पास्य॑ जुहोत्यग्नि॒वत्ये॒व जु॑होति॒ नान्धो᳚\-ऽध्व॒र्युर्भव॑ति॒ न य॒ज्ञꣳ रक्षाꣳ॑सि घ्नन्ति॒ जिघ॑र्म्य॒ग्निं मन॑सा घृ॒तेनेत्या॑ह॒ मन॑सा॒ हि पुरु॑षो य॒ज्ञम॑भि॒गच्छ॑ति प्रति॒क्ष्यन्तं॒ भुव॑नानि॒ विश्वेत्या॑ह॒ सर्व॒ꣴ॒ ह्ये॑ष प्र॒त्यङ्क्षेति॑ पृ॒थुं ति॑र॒श्चा वय॑सा बृ॒हन्त॒मित्या॒हाल्पो॒ ह्ये॑ष जा॒तो म॒हान्~(१२)

%5.1.3.3
भव॑ति॒ व्यचि॑ष्ठ॒मन्नꣳ॑ रभ॒सं विदा॑न॒मित्या॒हान्न॑मे॒वास्मै᳚ स्वदयति॒ सर्व॑मस्मै स्वदते॒ य ए॒वं वेदा त्वा॑ जिघर्मि॒ वच॑सा घृ॒तेनेत्या॑ह॒ तस्मा॒द्यत्पुरु॑षो॒ मन॑साभि॒गच्छ॑ति॒ तद्वा॒चा व॑दत्यर॒क्षसेत्या॑ह॒ रक्ष॑सा॒मप॑हत्यै॒ मर्य॑श्रीः स्पृह॒यद्व॑र्णो अ॒ग्निरित्या॒हाप॑चितिमे॒वास्मि॑न्दधा॒त्यप॑चितिमान्भवति॒ य ए॒वं~(१३)

%5.1.3.4
वेद॒ मन॑सा॒ त्वै तामाप्तु॑मर्\mbox{}हति॒ याम॑ध्व॒र्युर॑न॒ग्नावाहु॑तिं जु॒होति॒ मन॑स्वतीभ्यां जुहो॒त्याहु॑त्यो॒राप्त्यै॒ द्वाभ्यां॒ प्रति॑ष्ठित्यै यज्ञमु॒खेय॑ज्ञमुखे॒ वै क्रि॒यमा॑णे य॒ज्ञꣳ रक्षाꣳ॑सि जिघाꣳसन्त्ये॒तर्\mbox{}हि॒ खलु॒ वा ए॒तद्य॑ज्ञमु॒खं यर्\mbox{}ह्ये॑न॒दाहु॑तिरश्ञु॒ते परि॑ लिखति॒ रक्ष॑सा॒मप॑हत्यै ति॒सृभिः॒ परि॑ लिखति त्रि॒वृद्वा अ॒ग्निर्यावा॑ने॒वाग्निस्तस्मा॒द्रक्षा॒ꣴ॒स्यप॑ हन्ति~(१४)

%5.1.3.5
गा॒य॒त्रि॒या परि॑ लिखति॒ तेजो॒ वै गा॑य॒त्री तेज॑सै॒वैनं॒ परि॑ गृह्णाति त्रि॒ष्टुभा॒ परि॑ लिखतीन्द्रि॒यं वै त्रि॒ष्टुगि॑न्द्रि॒येणै॒वैनं॒ परि॑ गृह्णात्यनु॒ष्टुभा॒ परि॑ लिखत्यनु॒ष्टुफ्सर्वा॑णि॒ छन्दाꣳ॑सि परि॒भूः पर्या᳚प्त्यै मध्य॒तो॑\-ऽनु॒ष्टुभा॒ वाग्वा अ॑नु॒ष्टुप्तस्मा᳚न्मध्य॒तो वा॒चा व॑दामो गायत्रि॒या प्र॑थ॒मया॒ परि॑ लिख॒त्यथा॑नु॒ष्टुभाथ॑ त्रि॒ष्टुभा॒ तेजो॒ वै गा॑य॒त्री य॒ज्ञो॑\-ऽनु॒ष्टुगि॑न्द्रि॒यं त्रि॒ष्टुप्तेज॑सा चै॒वेन्द्रि॒येण॑ चोभ॒यतो॑ य॒ज्ञं परि॑ गृह्णाति॥~(१५)

{\anuvakamend[{अ॒न्धो᳚\-ऽध्व॒र्युर्म॒हान्भ॑वति त्रि॒ष्टुभा॒ तेजो॒ वै गा॑य॒त्री त्रयो॑दश च}]}%~(३)

%5.1.4.1
दे॒वस्य॑ त्वा सवि॒तुः प्र॑स॒व इति॑ खनति॒ प्रसू᳚त्या॒ अथो॑ धू॒ममे॒वैतेन॑ जनयति॒ ज्योति॑ष्मन्तं त्वाऽग्ने सु॒प्रती॑क॒मित्या॑ह॒ ज्योति॑रे॒वैतेन॑ जनयति॒ सो᳚\-ऽग्निर्जा॒तः प्र॒जाः शु॒चार्प॑य॒त्तं दे॒वा अ॑र्ध॒र्चेना॑शमयञ्छि॒वं प्र॒जाभ्यो\-ऽहिꣳ॑सन्त॒मित्या॑ह प्र॒जाभ्य॑ ए॒वैनꣳ॑ शमयति॒ द्वा\-भ्यां᳚ खनति॒ प्रति॑ष्ठित्या अ॒पां पृ॒ष्ठम॒सीति॑ पुष्करप॒र्णमा~(१६)

%5.1.4.2
ह॒र॒त्य॒पां वा ए॒तत्पृ॒ष्ठं यत्पु॑ष्करप॒र्णꣳ रू॒पेणै॒वैन॒दा ह॑रति पुष्करप॒र्णेन॒ सम्भ॑रति॒ योनि॒र्वा अ॒ग्नेः पु॑ष्करप॒र्णꣳ सयो॑निमे॒वाग्निꣳ सम्भ॑रति कृष्णाजि॒नेन॒ सम्भ॑रति य॒ज्ञो वै कृ॑ष्णाजि॒नं य॒ज्ञेनै॒व य॒ज्ञꣳ सम्भ॑रति॒ यद्ग्रा॒म्याणां᳚ पशू॒नां चर्म॑णा स॒म्भरे᳚द्ग्रा॒म्यान्प॒शूञ्छु॒चार्प॑येत्कृष्णाजि॒नेन॒ सम्भ॑रत्यार॒ण्याने॒व प॒शून्~(१७)

%5.1.4.3
शु॒चार्प॑यति॒ तस्मा᳚थ्स॒माव॑त्पशू॒नां प्र॒जाय॑मानानामार॒ण्याः प॒शवः॒ कनी॑याꣳसः शु॒चा ह्यृ॑ता लो॑म॒तः सम्भ॑र॒त्यतो॒ ह्य॑स्य॒ मेध्यं॑ कृष्णाजि॒नं च॑ पुष्करप॒र्णं च॒ सꣴ स्तृ॑णाती॒यं वै कृ॑ष्णाजि॒नम॒सौ पु॑ष्करप॒र्णमा॒भ्यामे॒वैन॑मुभ॒यतः॒ परि॑ गृह्णात्य॒ग्निर्दे॒वेभ्यो॒ निला॑यत॒ तमथ॒र्वान्व॑पश्य॒दथ॑र्वा त्वा प्रथ॒मो निर॑मन्थदग्न॒ इति॑~(१८)

%5.1.4.4
आ॒ह॒ य ए॒वैन॑म॒न्वप॑श्य॒त्तेनै॒वैन॒ꣳ॒ सम्भ॑रति॒ त्वाम॑ग्ने॒ पुष्क॑रा॒दधीत्या॑ह पुष्करप॒र्णे ह्ये॑न॒मुप॑श्रित॒मवि॑न्द॒त्तमु॑ त्वा द॒ध्यङ्ङृषि॒रित्या॑ह द॒ध्यङ्वा आ॑थर्व॒णस्ते॑ज॒स्व्या॑सी॒त्तेज॑ ए॒वास्मि॑न्दधाति॒ तमु॑ त्वा पा॒थ्यो वृषेत्या॑ह॒ पूर्व॑मे॒वोदि॒तमुत्त॑रेणा॒भि गृ॑णाति~(१९)

%5.1.4.5
च॒त॒सृभिः॒ सम्भ॑रति च॒त्वारि॒ छन्दाꣳ॑सि॒ छन्दो॑भिरे॒व गा॑य॒त्रीभि॑र्ब्राह्म॒णस्य॑ गाय॒त्रो हि ब्रा᳚ह्म॒णस्त्रि॒ष्टुग्भी॑ राज॒न्य॑स्य॒ त्रैष्टु॑भो॒ हि रा॑ज॒न्यो॑ यं का॒मये॑त॒ वसी॑यान्थ्स्या॒दित्यु॒भयी॑भि॒स्तस्य॒ सम्भ॑रे॒त्तेज॑श्चै॒वास्मा॑ इन्द्रि॒यं च॑ स॒मीची॑ दधात्यष्टा॒भिः सम्भ॑रत्य॒ष्टाक्ष॑रा गाय॒त्री गा॑य॒त्रो᳚\-ऽग्निर्यावा॑ने॒वाग्निस्तꣳ सम्भ॑रति॒ सीद॑ होत॒रित्या॑ह दे॒वता॑ ए॒वास्मै॒ सꣳ सा॑दयति॒ नि होतेति॑ मनु॒ष्या᳚न्थ्सꣳ सी॑द॒स्वेति॒ वयाꣳ॑सि॒ जनि॑ष्वा॒ हि जेन्यो॒ अग्रे॒ अह्ना॒मित्या॑ह देवमनु॒ष्याने॒वास्मै॒ सꣳस॑न्ना॒न्प्र ज॑नयति॥~(२०)

{\anuvakamend[{ऐव प॒शूनिति॑ गृणाति होत॒रिति॑ स॒प्तविꣳ॑शतिश्च}]}%~(४)

%5.1.5.1
क्रू॒रमि॑व॒ वा अ॑स्या ए॒तत्क॑रोति॒ यत्खन॑त्य॒प उप॑ सृज॒त्यापो॒ वै शा॒न्ताः शा॒न्ताभि॑रे॒वास्यै॒ शुचꣳ॑ शमयति॒ सं ते॑ वा॒युर्मा॑त॒रिश्वा॑ दधा॒त्वित्या॑ह प्रा॒णो वै वा॒युः प्रा॒णेनै॒वास्यै᳚ प्रा॒णꣳ सं द॑धाति॒ सं ते॑ वा॒युरित्या॑ह॒ तस्मा᳚द्वा॒युप्र॑च्युता दि॒वो वृ॑ष्टिरीर्ते॒ तस्मै॑ च देवि॒ वष॑डस्तु~(२१)

%5.1.5.2
तुभ्य॒मित्या॑ह॒ षड्वा ऋ॒तव॑ ऋ॒तुष्वे॒व वृष्टिं॑ दधाति॒ तस्मा॒थ्सर्वा॑नृ॒तून् व॑र्\mbox{}षति॒ यद्व॑षट्कु॒र्याद्रक्षाꣳ॑सि य॒ज्ञꣳ ह॑न्यु॒र्वडित्या॑ह प॒रोक्ष॑मे॒व वष॑ट्करोति॒ नास्य॑ या॒तया॑मा वषट्का॒रो भव॑ति॒ न य॒ज्ञꣳ रक्षाꣳ॑सि घ्नन्ति॒ सुजा॑तो॒ ज्योति॑षा स॒हेत्य॑नु॒ष्टुभोप॑ नह्यत्यनु॒ष्टुप्~(२२)

%5.1.5.3
सर्वा॑णि॒ छन्दाꣳ॑सि॒ छन्दाꣳ॑सि॒ खलु॒ वा अ॒ग्नेः प्रि॒या त॒नूः प्रि॒ययै॒वैनं॑ त॒नुवा॒ परि॑ दधाति॒ वेदु॑को॒ वासो॑ भवति॒ य ए॒वं वेद॑ वारु॒णो वा अ॒ग्निरुप॑नद्ध॒ उदु॑ तिष्ठ स्वध्वरो॒र्ध्व ऊ॒ षु ण॑ ऊ॒तय॒ इति॑ सावि॒त्रीभ्या॒मुत्ति॑ष्ठति सवि॒तृप्र॑सूत ए॒वास्यो॒र्ध्वां व॑रुणमे॒निमुथ्सृ॑जति॒ द्वाभ्यां॒ प्रति॑ष्ठित्यै॒ स जा॒तो गर्भो॑ असि~(२३)

%5.1.5.4
रोद॑स्यो॒रित्या॑हे॒मे वै रोद॑सी॒ तयो॑रे॒ष गर्भो॒ यद॒ग्निस्तस्मा॑दे॒वमा॒हाग्ने॒ चारु॒र्विभृ॑त॒ ओष॑धी॒ष्वित्या॑ह य॒दा ह्ये॑तं वि॒भर॒न्त्यथ॒ चारु॑तरो॒ भव॑ति॒ प्र मा॒तृभ्यो॒ अधि॒ कनि॑क्रदद्गा॒ इत्या॒हौष॑धयो॒ वा अ॑स्य मा॒तर॒स्ताभ्य॑ ए॒वैनं॒ प्र च्या॑वयति स्थि॒रो भ॑व वी॒ड्व॑ङ्ग॒ इति॑ गर्द॒भ आ सा॑दयति~(२४)

%5.1.5.5
सं न॑ह्यत्ये॒वैन॑मे॒तया᳚ स्थे॒म्ने ग॑र्द॒भेन॒ सम्भ॑रति॒ तस्मा᳚द्गर्द॒भः प॑शू॒नां भा॑रभा॒रित॑मो गर्द॒भेन॒ सम्भ॑रति॒ तस्मा᳚द्गर्द॒भो\-ऽप्य॑नाले॒शेत्य॒न्यान्प॒शून्मे᳚द्य॒त्यन्न॒ꣴ॒ ह्ये॑नेना॒र्कꣳ स॒म्भर॑न्ति गर्द॒भेन॒ सम्भ॑रति॒ तस्मा᳚द्गर्द॒भो द्वि॒रेताः॒ सन्कनि॑ष्ठं पशू॒नां प्र जा॑यते॒\-ऽग्निर्ह्य॑स्य॒ योनिं॑ नि॒र्दह॑ति प्र॒जासु॒ वा ए॒ष ए॒तर्\mbox{}ह्यारू॑ढः~(२५)

%5.1.5.6
स ई᳚श्व॒रः प्र॒जाः शु॒चा प्र॒दहः॑ शि॒वो भ॑व प्र॒जाभ्य॒ इत्या॑ह प्र॒जाभ्य॑ ए॒वैनꣳ॑ शमयति॒ मानु॑षीभ्य॒स्त्वम॑ङ्गिर॒ इत्या॑ह मान॒व्यो॑ हि प्र॒जा मा द्यावा॑पृथि॒वी अ॒भि शू॑शुचो॒ माऽन्तरि॑क्षं॒ मा वन॒स्पती॒नित्या॑है॒भ्य ए॒वैनं॑ लो॒केभ्यः॑ शमयति॒ प्रैतु॑ वा॒जी कनि॑क्रद॒दित्या॑ह वा॒जी ह्ये॑ष नान॑द॒द्रास॑भः॒ पत्वेति॑~(२६)

%5.1.5.7
आ॒ह॒ रास॑भ॒ इति॒ ह्ये॑तमृष॒यो\-ऽव॑द॒न्भर॑न्न॒ग्निं पु॑री॒ष्य॑मित्या॑हा॒ग्निꣴ ह्ये॑ष भर॑ति॒ मा पा॒द्यायु॑षः पु॒रेत्या॒हा\-ऽऽ\-यु॑रे॒वास्मि॑न्दधाति॒ तस्मा᳚द्गर्द॒भः सर्व॒मायु॑रेति॒ तस्मा᳚द्गर्द॒भे पु॒रा\-ऽऽ\-यु॑षः॒ प्रमी॑ते बिभ्यति॒ वृषा॒ग्निं वृष॑ण॒म्भर॒न्नित्या॑ह वृषा॒ ह्ये॑ष वृषा॒ग्निर॒पां गर्भम्᳚~(२७)

%5.1.5.8
स॒मु॒द्रिय॒मित्या॑हा॒पाꣳ ह्ये॑ष गर्भो॒ यद॒ग्निरग्न॒ आ या॑हि वी॒तय॒ इति॒ वा इ॒मौ लो॒कौ व्यै॑ता॒मग्न॒ आ या॑हि वी॒तय॒ इति॒ यदाहा॒नयो᳚र्लो॒कयो॒र्वीत्यै॒ प्रच्यु॑तो॒ वा ए॒ष आ॒यत॑ना॒दग॑तः प्रति॒ष्ठाꣳ स ए॒तर्\mbox{}ह्य॑ध्व॒र्युं च॒ यज॑मानं च ध्यायत्यृ॒तꣳ स॒त्यमित्या॑हे॒यं वा ऋ॒तम॒सौ~(२८)

%5.1.5.9
स॒त्यम॒नयो॑रे॒वैनं॒ प्रति॑\-ष्ठापयति॒ नार्ति॒मार्च्छ॑त्यध्व॒र्युर्न यज॑मानो॒ वरु॑णो॒ वा ए॒ष यज॑मानम॒भ्यैति॒ यद॒ग्निरुप॑नद्ध॒ ओष॑धयः॒ प्रति॑ गृह्णीता॒ग्निमे॒तमित्या॑ह॒ शान्त्यै॒ व्यस्य॒न्विश्वा॒ अम॑ती॒ररा॑ती॒रित्या॑ह॒ रक्ष॑सा॒मप॑हत्यै नि॒षीद॑न्नो॒ अप॑ दुर्म॒तिꣳ ह॑न॒दित्या॑ह॒ प्रति॑ष्ठित्या॒ ओष॑धयः॒ प्रति॑ मोदध्वम्॥~(२९)

%5.1.5.10
{\anuvakamend[{अ॒स्त्व॒नु॒ष्टुब॑सि सादय॒त्यारू॑ढः॒ पत्वेति॒ गर्भ॑म॒सौ मो॑दध्वं॒ द्विच॑त्वारिꣳशच्च}]}%~(५)

{{\anuvakamend[ए॒न॒मित्या॒हौष॑धयो॒ वा अ॒ग्नेर्भा॑ग॒धेयं॒ ताभि॑रे॒वैन॒ꣳ॒ सम॑र्धयति॒ पुष्पा॑वतीः सुपिप्प॒ला इत्या॑ह॒ तस्मा॒दोष॑धयः॒ फलं॑ गृह्णन्त्य॒यं वो॒ गर्भ॑ ऋ॒त्वियः॑ प्र॒त्नꣳ स॒धस्थ॒मास॑द॒दित्या॑ह॒ याभ्य॑ ए॒वैनं॑ प्रच्या॒वय॑ति॒ तास्वे॒वैनं॒ प्रति॑\-ष्ठापयति॒ द्वाभ्या॑मु॒पाव॑हरति॒ प्रति॑ष्ठित्यै~(~०)]}}

%5.1.6.1
वा॒रु॒णो वा अ॒ग्निरुप॑नद्धो॒ वि पाज॒सेति॒ वि स्रꣳ॑सयति सवि॒तृप्र॑सूत ए॒वास्य॒ विषू॑चीं वरुणमे॒निं वि सृ॑जत्य॒प उप॑ सृज॒त्यापो॒ वै शा॒न्ताः शा॒न्ताभि॑रे॒वास्य॒ शुचꣳ॑ शमयति ति॒सृभि॒रुप॑ सृजति त्रि॒वृद्वा अ॒ग्निर्यावा॑ने॒वाग्निस्तस्य॒ शुचꣳ॑ शमयति मि॒त्रः स॒ꣳ॒सृज्य॑ पृथि॒वीमित्या॑ह मि॒त्रो वै शि॒वो दे॒वाना॒न्तेनै॒व~(३१)

%5.1.6.2
ए॒न॒ꣳ॒ सꣳ सृ॑जति॒ शान्त्यै॒ यद्ग्रा॒म्याणां॒ पात्रा॑णां क॒पालैः᳚ सꣳसृ॒जेद्ग्रा॒म्याणि॒ पात्रा॑णि शु॒चार्प॑येदर्मकपा॒लैः सꣳ सृ॑जत्ये॒तानि॒ वा अ॑नुपजीवनी॒यानि॒ तान्ये॒व शु॒चार्प॑यति॒ शर्क॑राभिः॒ सꣳ सृ॑जति॒ धृत्या॒ अथो॑ शं॒त्वाया॑जलो॒मैः सꣳ सृ॑जत्ये॒षा वा अ॒ग्नेः प्रि॒या त॒नूर्यद॒जा प्रि॒ययै॒वैनं॑ त॒नुवा॒ सꣳ सृ॑ज॒त्यथो॒ तेज॑सा कृष्णाजि॒नस्य॒ लोम॑भिः॒ सम्~(३२)

%5.1.6.3
सृ॒ज॒ति॒ य॒ज्ञो वै कृ॑ष्णाजि॒नं य॒ज्ञेनै॒व य॒ज्ञꣳ सꣳ सृ॑जति रु॒द्राः स॒म्भृत्य॑ पृथि॒वीमित्या॑है॒ता वा ए॒तं दे॒वता॒ अग्रे॒ सम॑भरं॒ ताभि॑रे॒वैन॒ꣳ॒ सम्भ॑रति म॒खस्य॒ शिरो॒\-ऽसीत्या॑ह य॒ज्ञो वै म॒खस्तस्यै॒तच्छिरो॒ यदु॒खा तस्मा॑दे॒वमा॑ह य॒ज्ञस्य॑ प॒दे स्थ॒ इत्या॑ह य॒ज्ञस्य॒ ह्ये॑ते~(३३)

%5.1.6.4
प॒दे अथो॒ प्रति॑ष्ठित्यै॒ प्रान्याभि॒र्यच्छ॒त्यन्व॒न्यैर्म॑न्त्रयते मिथुन॒त्वाय॒ त्र्यु॑द्धिं करोति॒ त्रय॑ इ॒मे लो॒का ए॒षां लो॒काना॒माप्त्यै॒ छन्दो॑भिः करोति वी॒र्यं॑ वै छन्दाꣳ॑सि वी॒र्ये॑णै॒वैनां᳚ करोति॒ यजु॑षा॒ बिलं॑ करोति॒ व्यावृ॑त्त्या॒ इय॑तीं करोति प्र॒जा\-प॑तिना यज्ञमु॒खेन॒ सम्मि॑तां द्विस्त॒नां क॑रोति॒ द्यावा॑पृथि॒व्योर्दोहा॑य॒ चतुः॑ स्तनां करोति पशू॒नां दोहा॑या॒ष्टास्त॑नां करोति॒ छन्द॑सां॒ दोहा॑य॒ नवा᳚श्रिमभि॒चर॑तः कुर्यात्त्रि॒वृत॑मे॒व वज्रꣳ॑ स॒म्भृत्य॒ भ्रातृ॑व्याय॒ प्र ह॑रति॒ स्तृत्यै॑ कृ॒त्वाय॒ सा म॒हीमु॒खामिति॒ नि द॑धाति दे॒वता᳚स्वे॒वैनां॒ प्रति॑\-ष्ठापयति॥~(३४)

{\anuvakamend[{तेनै॒व लोम॑भिः॒ समे॒ते अ॑भि॒चर॑त॒ एक॑विꣳशतिश्च}]}%~(६)

%5.1.7.1
स॒प्तभि॑र्धूपयति स॒प्त वै शी॑र्\mbox{}ष॒ण्याः᳚ प्रा॒णाः शिर॑ ए॒तद्य॒ज्ञस्य॒ यदु॒खा शी॒र्॒\mbox{}षन्ने॒व य॒ज्ञस्य॑ प्रा॒णान्द॑धाति॒ तस्मा᳚थ्स॒प्त शी॒र्॒\mbox{}षन्प्रा॒णा अ॑श्वश॒केन॑ धूपयति प्राजाप॒त्यो वा अश्वः॑ सयोनि॒त्वायादि॑ति॒स्त्वेत्या॑हे॒यं वा अदि॑ति॒रदि॑त्यै॒वादि॑त्यां खनत्य॒स्या अक्रू॑रङ्काराय॒ न हि स्वः स्वꣳ हि॒नस्ति॑ दे॒वानां᳚ त्वा॒ पत्नी॒रित्या॑ह दे॒वाना᳚म्~(३५)

%5.1.7.2
वा ए॒तां पत्न॒यो\-ऽग्रे॑\-ऽकुर्व॒न्ताभि॑रे॒वैनां᳚ दधाति धि॒षणा॒स्त्वेत्या॑ह वि॒द्या वै धि॒षणा॑ वि॒द्याभि॑रे॒वैना॑म॒भीन्द्धे॒ ग्नास्त्वेत्या॑ह॒ छन्दाꣳ॑सि॒ वै ग्नाश्छन्दो॑भिरे॒वैनाꣴ॑ श्रपयति॒ वरू᳚त्रय॒स्त्वेत्या॑ह॒ होत्रा॒ वै वरू᳚त्रयो॒ होत्रा॑भिरे॒वैनां᳚ पचति॒ जन॑य॒स्त्वेत्या॑ह दे॒वानां॒ वै पत्नीः᳚~(३६)

%5.1.7.3
जन॑य॒स्ताभि॑रे॒वैनां᳚ पचति ष॒ड्भिः प॑चति॒ षड्वा ऋ॒तव॑ ऋ॒तुभि॑रे॒वैनां᳚ पचति॒ द्विः पच॒न्त्वित्या॑ह॒ तस्मा॒द्द्विः सं॑वथ्स॒रस्य॑ स॒स्यं प॑च्यते वारु॒ण्यु॑खाभीद्धा॑ मै॒त्रियोपै॑ति॒ शान्त्यै॑ दे॒वस्त्वा॑ सवि॒तोद्व॑प॒त्वित्या॑ह सवि॒तृप्र॑सूत ए॒वैनां॒ ब्रह्म॑णा दे॒वता॑भि॒रुद्व॑प॒त्यप॑द्यमाना पृथि॒व्याशा॒ दिश॒ आ पृ॑ण~(३७)

%5.1.7.4
इत्या॑ह॒ तस्मा॑द॒ग्निः सर्वा॒ दिशो\-ऽनु॒ वि भा॒त्युत्ति॑ष्ठ बृह॒ती भ॑वो॒र्ध्वा ति॑ष्ठ ध्रु॒वा त्वमित्या॑ह॒ प्रति॑ष्ठित्या असु॒र्यं॑ पात्र॒\-मना᳚च्छृण्ण॒मा च्छृ॑णत्ति देव॒त्राक॑रजक्षी॒रेणा च्छृ॑णत्ति पर॒मं वा ए॒तत्पयो॒ यद॑जक्षी॒रं प॑र॒मेणै॒वैनां॒ पय॒सा च्छृ॑णत्ति॒ यजु॑षा॒ व्यावृ॑त्त्यै॒ छन्दो॑भि॒रा च्छृ॑णत्ति॒ छन्दो॑भि॒र्वा ए॒षा क्रि॑यते॒ छन्दो॑भिरे॒व छन्दा॒ꣴ॒स्या च्छृ॑णत्ति॥~(३८)

{\anuvakamend[{आ॒ह॒ दे॒वानां॒ वै पत्नीः᳚ पृणै॒षा षट्च॑}]}%~(७)

%5.1.8.1
एक॑विꣳशत्या॒ माषैः᳚ पुरुषशी॒र्॒\mbox{}षमच्छै᳚त्यमे॒ध्या वै माषा॑ अमे॒ध्यं पु॑रुषशी॒र्॒\mbox{}षम॑मे॒ध्यैरे॒वास्या॑मे॒ध्यं नि॑रव॒दाय॒ मेध्यं॑ कृ॒त्वा ह॑र॒त्येक॑विꣳशतिर्भवन्त्येकवि॒ꣳ॒शो वै पुरु॑षः॒ पुरु॑ष॒स्याप्त्यै॒ व्यृ॑द्धं॒ वा ए॒तत्प्रा॒णैर॑मे॒ध्यं यत्पु॑रुषशी॒र्॒\mbox{}षꣳ स॑प्त॒धा वितृ॑ण्णां वल्मीकव॒पां प्रति॒ नि द॑धाति स॒प्त वै शी॑र्\mbox{}ष॒ण्याः᳚ प्रा॒णाः प्रा॒णैरे॒वैन॒थ्सम॑र्धयति मेध्य॒त्वाय॒ याव॑न्तः~(३९)

%5.1.8.2
वै मृ॒त्युब॑न्धव॒स्तेषां᳚ य॒म आधि॑पत्यं॒ परी॑याय यमगा॒थाभिः॒ परि॑ गायति य॒मादे॒वैन॑द्वृङ्क्ते ति॒सृभिः॒ परि॑ गायति॒ त्रय॑ इ॒मे लो॒का ए॒भ्य ए॒वैन॑ल्लो॒केभ्यो॑ वृङ्क्ते॒ तस्मा॒द्गाय॑ते॒ न देय॒ङ्गाथा॒ हि तद्वृ॒ङ्क्ते᳚\-ऽग्निभ्यः॑ प॒शूना ल॑भते॒ कामा॒ वा अ॒ग्नयः॒ कामा॑ने॒वाव॑ रुन्धे॒ यत्प॒शून्नालभे॒तान॑वरुद्धा अस्य~(४०)

%5.1.8.3
प॒शवः॑ स्यु॒र्यत्पर्य॑ग्निकृतानुथ्सृ॒जेद्य॑ज्ञवेश॒सं कु॑र्या॒द्यथ्सꣴ॑स्था॒पये᳚द्या॒तया॑मानि शी॒र्॒\mbox{}षाणि॑ स्यु॒र्यत्प॒शूना॒लभ॑ते॒ तेनै॒व प॒शूनव॑ रुन्धे॒ यत्पर्य॑ग्निकृतानुथ्सृ॒जति॑ शी॒र्ष्णामया॑तयामत्वाय प्राजाप॒त्येन॒ सꣴ स्था॑पयति य॒ज्ञो वै प्र॒जा\-प॑तिर्य॒ज्ञ ए॒व य॒ज्ञं प्रति॑\-ष्ठापयति प्र॒जा\-प॑तिः प्र॒जा अ॑सृजत॒ स रि॑रिचा॒नो॑\-ऽमन्यत॒ स ए॒ता आ॒प्रीर॑पश्य॒त्ताभि॒र्वै स मु॑ख॒तः~(४१)

%5.1.8.4
आ॒त्मान॒माप्री॑णीत॒ यदे॒ता आ॒प्रियो॒ भव॑न्ति य॒ज्ञो वै प्र॒जा\-प॑तिर्य॒ज्ञमे॒वैताभि॑र्मुख॒त आ प्री॑णा॒त्यप॑रिमितछन्दसो भव॒न्त्यप॑रिमितः प्र॒जा\-प॑तिः प्र॒जा\-प॑ते॒राप्त्या॑ ऊनातिरि॒क्ता मि॑थु॒नाः प्रजा᳚त्यै लोम॒शं वै नामै॒तच्छन्दः॑ प्र॒जा\-प॑तेः प॒शवो॑ लोम॒शाः प॒शूने॒वाव॑ रुन्धे॒ सर्वा॑णि॒ वा ए॒ता रू॒पाणि॒ सर्वा॑णि रू॒पाण्य॒ग्नौ चित्ये᳚ क्रियन्ते॒ तस्मा॑दे॒ता अ॒ग्नेश्चित्य॑स्य~(४२)

%5.1.8.5
भ॒व॒न्त्येक॑विꣳशतिꣳ सामिधे॒नीरन्वा॑ह॒ रुग्वा ए॑कवि॒ꣳ॒शो रुच॑मे॒व ग॑च्छ॒त्यथो᳚ प्रति॒ष्ठामे॒व प्र॑ति॒ष्ठा ह्ये॑कवि॒ꣳ॒शश्चतु॑र्विꣳशति॒मन्वा॑ह॒ चतु॑र्विꣳशतिरर्धमा॒साः सं॑वथ्स॒रः सं॑वथ्स॒रो᳚\-ऽग्निर्वै᳚श्वान॒रः सा॒क्षादे॒व वै᳚श्वान॒रमव॑ रुन्धे॒ परा॑ची॒रन्वा॑ह॒ परा॑ङिव॒ हि सु॑व॒र्गो लो॒कः समा᳚स्त्वाग्न ऋ॒तवो॑ वर्धय॒न्त्वित्या॑ह॒ समा॑भिरे॒वाग्निं व॑र्धयति~(४३)

%5.1.8.6
ऋ॒तुभिः॑ संवथ्स॒रं विश्वा॒ आ भा॑हि प्र॒दिशः॑ पृथि॒व्या इत्या॑ह॒ तस्मा॑द॒ग्निः सर्वा॒ दिशो\-ऽनु॒ वि भा॑ति॒ प्रत्यौ॑हताम॒श्विना॑ मृ॒त्युम॑स्मा॒दित्या॑ह मृ॒त्युमे॒वास्मा॒दप॑ नुद॒त्युद्व॒यं तम॑स॒स्परीत्या॑ह पा॒प्मा वै तमः॑ पा॒प्मान॑मे॒वास्मा॒दप॑ ह॒न्त्यग॑न्म॒ ज्योति॑रुत्त॒ममित्या॑हा॒सौ वा आ॑दि॒त्यो ज्योति॑रुत्त॒ममा॑दि॒त्यस्यै॒व सायु॑ज्यं गच्छति॒ न सं॑वथ्स॒रस्ति॑ष्ठति॒ नास्य॒ श्रीस्ति॑ष्ठति॒ यस्यै॒ताः क्रि॒यन्ते॒ ज्योति॑ष्मतीमुत्त॒मामन्वा॑ह॒ ज्योति॑रे॒वास्मा॑ उ॒परि॑ष्टाद्दधाति सुव॒र्गस्य॑ लो॒कस्यानु॑ख्यात्यै॥~(४४)

{\anuvakamend[{याव॑न्तो\-ऽस्य मुख॒तश्चित्य॑स्य वर्धयत्यादि॒त्यो᳚\-ऽष्टाविꣳ॑शतिश्च}]}%~(८)

%5.1.9.1
ष॒ड्भिर्दी᳚क्षयति॒ षड्वा ऋ॒तव॑ ऋ॒तुभि॑रे॒वैनं॑ दीक्षयति स॒प्तभि॑र्दीक्षयति स॒प्त छन्दाꣳ॑सि॒ छन्दो॑भिरे॒वैनं॑ दीक्षयति॒ विश्वे॑ दे॒वस्य॑ ने॒तुरित्य॑नु॒ष्टुभो᳚त्त॒मया॑ जुहोति॒ वाग्वा अ॑नु॒ष्टुप्तस्मा᳚त्प्रा॒णानां॒ वागु॑त्त॒मैक॑स्माद॒क्षरा॒दना᳚प्तं प्रथ॒मं प॒दं तस्मा॒द्यद्वा॒चो\-ऽना᳚प्तं॒ तन्म॑नु॒ष्या॑ उप॑ जीवन्ति पू॒र्णया॑ जुहोति पू॒र्ण इ॑व॒ हि प्र॒जा\-प॑तिः~(४५)

%5.1.9.2
प्र॒जा\-प॑ते॒राप्त्यै॒ न्यू॑नया जुहोति॒ न्यू॑ना॒द्धि प्र॒जा\-प॑तिः प्र॒जा असृ॑जत प्र॒जाना॒ꣳ॒ सृष्ट्यै॒ यद॒र्चिषि॑ प्रवृ॒ञ्ज्याद्भू॒तमव॑ रुन्धीत॒ यदङ्गा॑रेषु भवि॒ष्यदङ्गा॑रेषु॒ प्र वृ॑णक्ति भवि॒ष्यदे॒वाव॑ रुन्धे भवि॒ष्यद्धि भूयो॑ भू॒ताद्द्वाभ्यां॒ प्र वृ॑णक्ति द्वि॒पाद्यज॑मानः॒ प्रति॑ष्ठित्यै॒ ब्रह्म॑णा॒ वा ए॒षा यजु॑षा॒ सम्भृ॑ता॒ यदु॒खा सा यद्भिद्ये॒तार्ति॒मार्च्छे᳚त्~(४६)

%5.1.9.3
यज॑मानो ह॒न्येता᳚स्य य॒ज्ञो मित्रै॒तामु॒खां त॒पेत्या॑ह॒ ब्रह्म॒ वै मि॒त्रो ब्रह्म॑न्ने॒वैनां॒ प्रति॑\-ष्ठापयति॒ नार्ति॒मार्च्छ॑ति॒ यज॑मानो॒ नास्य॑ य॒ज्ञो ह॑न्यते॒ यदि॒ भिद्ये॑त॒ तैरे॒व क॒पालैः॒ सꣳ सृ॑जे॒थ्सैव ततः॒ प्राय॑श्चित्ति॒र्यो ग॒तश्रीः॒ स्यान्म॑थि॒त्वा तस्याव॑ दध्याद्भू॒तो वा ए॒ष स स्वां~(४७)

%5.1.9.4
दे॒वता॒मुपै॑ति॒ यो भूति॑कामः॒ स्याद्य उ॒खायै॑ स॒म्भवे॒थ्स ए॒व तस्य॑ स्या॒दतो॒ ह्ये॑ष स॒म्भव॑त्ये॒ष वै स्व॑य॒म्भूर्नाम॒ भव॑त्ये॒व यं का॒मये॑त॒ भ्रातृ॑व्यमस्मै जनयेय॒मित्य॒न्यत॒स्तस्या॒हृत्याव॑ दध्याथ्सा॒क्षादे॒वास्मै॒ भ्रातृ॑व्यं जनयत्यम्ब॒रीषा॒\-दन्न॑काम॒स्याव॑ दध्यादम्ब॒रीषे॒ वा अन्न॑म्भ्रियते॒ सयो᳚न्ये॒वान्नम्᳚~(४८)

%5.1.9.5
अव॑ रुन्धे॒ मुञ्जा॒नव॑ दधा॒त्यूर्ग्वै मुञ्जा॒ ऊर्ज॑मे॒वास्मा॒ अपि॑ दधात्य॒ग्निर्दे॒वेभ्यो॒ निला॑यत॒ स क्रु॑मु॒कं प्रावि॑शत् क्रुमु॒कमव॑ दधाति॒ यदे॒वास्य॒ तत्र॒ न्य॑क्तं॒ तदे॒वाव॑ रुन्ध॒ आज्ये॑न॒ सं यौ᳚त्ये॒तद्वा अ॒ग्नेः प्रि॒यं धाम॒ यदाज्यं॑ प्रि॒येणै॒वैनं॒ धाम्ना॒ सम॑र्धय॒त्यथो॒ तेज॑सा~(४९)

%5.1.9.6
वैकं॑कती॒मा द॑धाति॒ भा ए॒वाव॑ रुन्धे शमी॒मयी॒मा द॑धाति॒ शान्त्यै॒ सीद॒ त्वं मा॒तुर॒स्या उ॒पस्थ॒ इति॑ ति॒सृभि॑र्जा॒तमुप॑ तिष्ठते॒ त्रय॑ इ॒मे लो॒का ए॒ष्वे॑व लो॒केष्वा॒विदं॑ गच्छ॒त्यथो᳚ प्रा॒णाने॒वा\-ऽऽ\-त्मन्ध॑त्ते॥~(५०)

{\anuvakamend[{प्र॒जा\-प॑तिर्\mbox{}ऋच्छे॒थ्\-स्वामे॒वान्नं॒ तेज॑सा॒ चतु॑स्त्रिꣳशच्च}]}%~(९)

%5.1.10.1
न ह॑ स्म॒ वै पु॒राग्निरप॑रशुवृक्णं दहति॒ तद॑स्मै प्रयो॒ग ए॒वर्\mbox{}षि॑रस्वदय॒द्यद॑ग्ने॒ यानि॒ कानि॒ चेति॑ स॒मिध॒मा द॑धा॒त्यप॑रशुवृक्णमे॒वास्मै᳚ स्वदयति॒ सर्व॑मस्मै स्वदते॒ य ए॒वं वेदौदु॑म्बरी॒मा द॑धा॒त्यूर्ग्वा उ॑दु॒म्बर॒ ऊर्ज॑मे॒वास्मा॒ अपि॑ दधाति प्र॒जा\-प॑तिर॒ग्निम॑सृजत॒ तꣳ सृ॒ष्टꣳ रक्षाꣳ॑सि~(५१)

%5.1.10.2
अ॒जि॒घा॒ꣳ॒स॒न्थ्स ए॒तद्रा᳚क्षो॒घ्नम॑पश्य॒त्तेन॒ वै स रक्षा॒ꣴ॒स्यपा॑हत॒ यद्रा᳚क्षो॒घ्नं भव॑त्य॒ग्नेरे॒व तेन॑ जा॒ताद्रक्षा॒ꣴ॒स्यप॑ ह॒न्त्याश्व॑त्थी॒मा द॑धात्यश्व॒त्थो वै वन॒स्पती॑नाꣳ सपत्नसा॒हो विजि॑त्यै॒ वैक॑ङ्कती॒मा द॑धाति॒ भा ए॒वाव॑ रुन्धे शमी॒मयी॒मा द॑धाति॒ शान्त्यै॒ सꣳ॑शितं मे॒ ब्रह्मोदे॑षां बा॒हू अ॑तिर॒मित्यु॑त्त॒मे औदु॑म्बरी~(५२)

%5.1.10.3
वा॒च॒य॒ति॒ ब्रह्म॑णै॒व क्ष॒त्रꣳ सꣴ श्य॑ति क्ष॒त्रेण॒ ब्रह्म॒ तस्मा᳚द्ब्राह्म॒णो रा॑ज॒न्य॑वा॒नत्य॒न्यं ब्रा᳚ह्म॒णं तस्मा᳚द्राज॒न्यो᳚ ब्राह्म॒णवा॒नत्य॒न्यꣳ रा॑ज॒न्यं॑ मृ॒त्युर्वा ए॒ष यद॒ग्निर॒मृत॒ꣳ॒ हिर॑ण्यꣳ रु॒क्ममन्त॑रं॒ प्रति॑ मुञ्चते॒\-ऽमृत॑मे॒व मृ॒त्योर॒न्तर्ध॑त्त॒ एक॑विꣳशतिनिर्बाधो भव॒त्येक॑विꣳशति॒र्वै दे॑वलो॒का द्वाद॑श॒ मासाः॒ पञ्च॒र्तव॒स्त्रय॑ इ॒मे लो॒का अ॒सावा॑दि॒त्यः~(५३)

%5.1.10.4
ए॒क॒वि॒ꣳ॒श ए॒ताव॑न्तो॒ वै दे॑वलो॒कास्तेभ्य॑ ए॒व भ्रातृ॑व्यम॒न्तरे॑ति निर्बा॒धैर्वै दे॒वा असु॑रान्निर्बा॒धे॑\-ऽकुर्वत॒ तन्नि॑र्बा॒धानां᳚ निर्बाध॒त्वन्नि॑र्बा॒धी भ॑वति॒ भ्रातृ॑व्याने॒व नि॑र्बा॒धे कु॑रुते सावित्रि॒या प्रति॑ मुञ्चते॒ प्रसू᳚त्यै॒ नक्तो॒षासेत्युत्त॑रयाहोरा॒त्राभ्या॑\-मे॒वैन॒मुद्य॑च्छते दे॒वा अ॒ग्निं धा॑रयन्द्रविणो॒दा इत्या॑ह प्रा॒णा वै दे॒वा द्र॑विणो॒दा अ॑होरा॒त्राभ्या॑मे॒वैन॑मु॒द्यत्य॑~(५४)

%5.1.10.5
प्रा॒णैर्दा॑धा॒रासी॑नः॒ प्रति॑ मुञ्चते॒ तस्मा॒दासी॑नाः प्र॒जाः प्र जा॑यन्ते कृष्णाजि॒नमुत्त॑र॒न्तेजो॒ वै हिर॑ण्यं॒ ब्रह्म॑ कृष्णाजि॒नन्तेज॑सा चै॒वैनं॒ ब्रह्म॑णा चोभ॒यतः॒ परि॑ गृह्णाति॒ षडु॑द्यामꣳ शि॒क्यं॑ भवति॒ षड्वा ऋ॒तव॑ ऋ॒तुभि॑रे॒वैन॒मुद्य॑च्छते॒ यद्द्वाद॑शोद्यामꣳ संवथ्स॒रेणै॒व मौ॒ञ्जं भ॑व॒त्यूर्ग्वै मुञ्जा॑ ऊ॒र्जैवैन॒ꣳ॒ सम॑र्धयति सुप॒र्णो॑\-ऽसि ग॒रुत्मा॒नित्यवे᳚क्षते रू॒पमे॒वास्यै॒तन्म॑हि॒मानं॒ व्याच॑ष्टे॒ दिवं॑ गच्छ॒ सुवः॑ प॒तेत्या॑ह सुव॒र्गमे॒वैनं॑ लो॒कं ग॑मयति॥~(५)

{\anuvakamend[{रक्षा॒ꣴ॒स्यौदु॑म्बरी आदि॒त्य उ॒द्यत्य॒ स़ञ्चतु॑र्विꣳशतिश्च}]}%॥10॥

%5.1.11.1
समि॑द्धो अ॒ञ्जन्कृद॑रं मती॒नां घृ॒तम॑ग्ने॒ मधु॑म॒त्पिन्व॑मानः। वा॒जी वह॑न्वा॒जिनं॑ जातवेदो दे॒वानां᳚ वक्षि प्रि॒यमा स॒धस्थम्᳚। घृ॒तेना॒ञ्जन्थ्सम्प॒थो दे॑व॒याना᳚न्प्रजा॒नन्वा॒ज्यप्ये॑तु दे॒वान्। अनु॑ त्वा सप्ते प्र॒दिशः॑ सचन्ताꣴ स्व॒धाम॒स्मै यज॑मानाय धेहि। ईड्य॒श्चासि॒ वन्द्य॑श्च वाजिन्ना॒शुश्चासि॒ मेध्य॑श्च सप्ते। अ॒ग्निष्ट्वा᳚~(५६)

%5.1.11.2
दे॒वैर्वसु॑भिः स॒जोषाः᳚ प्री॒तं वह्निं॑ वहतु जा॒तवे॑दाः। स्ती॒र्णं ब॒र्॒\mbox{}हिः सु॒ष्टरी॑मा जुषा॒णोरु पृ॒थु प्रथ॑मानं पृथि॒व्याम्। दे॒वेभि॑र्यु॒क्तमदि॑तिः स॒जोषाः᳚ स्यो॒नं कृ॑ण्वा॒ना सु॑वि॒ते द॑धातु। ए॒ता उ॑ वः सु॒भगा॑ वि॒श्वरू॑पा॒ वि पक्षो॑भिः॒ श्रय॑माणा॒ उदातैः᳚। ऋ॒ष्वाः स॒तीः क॒वषः॒ शुम्भ॑माना॒ द्वारो॑ दे॒वीः सु॑प्राय॒णा भ॑वन्तु। अ॒न्त॒रा मि॒त्रावरु॑णा॒ चर॑न्ती॒ मुखं॑ य॒ज्ञाना॑म॒भि सं॑विदा॒ने। उ॒षासा॑ वाम्~(५७)

%5.1.11.3
सु॒हि॒र॒ण्ये सु॑शि॒ल्पे ऋ॒तस्य॒ योना॑वि॒ह सा॑दयामि। प्र॒थ॒मा वाꣳ॑ सर॒थिना॑ सु॒वर्णा॑ दे॒वौ पश्य॑न्तौ॒ भुव॑नानि॒ विश्वा᳚। अपि॑प्रयं॒ चोद॑ना वां॒ मिमा॑ना॒ होता॑रा॒ ज्योतिः॑ प्र॒दिशा॑ दि॒शन्ता᳚। आ॒दि॒त्यैर्नो॒ भार॑ती वष्टु य॒ज्ञꣳ सर॑स्वती स॒ह रु॒द्रैर्न॑ आवीत्। इडोप॑हूता॒ वसु॑भिः स॒जोषा॑ य॒ज्ञं नो॑ देवीर॒मृते॑षु धत्त। त्वष्टा॑ वी॒रं दे॒वका॑मं जजान॒ त्वष्टु॒रर्वा॑ जायत आ॒शुरश्वः॑।~(५८)

%5.1.11.4
त्वष्टे॒दं विश्वं॒ भुव॑नं जजान ब॒होः क॒र्तार॑मि॒ह य॑क्षि होतः। अश्वो॑ घृ॒तेन॒ त्मन्या॒ सम॑क्त॒ उप॑ दे॒वाꣳ ऋ॑तु॒शः पाथ॑ एतु। वन॒स्पति॑र्देवलो॒कं प्र॑जा॒नन्न॒ग्निना॑ ह॒व्या स्व॑दि॒तानि॑ वक्षत्। प्र॒जा\-प॑ते॒स्तप॑सा वावृधा॒नः स॒द्यो जा॒तो द॑धिषे य॒ज्ञम॑ग्ने। स्वाहा॑कृतेन ह॒विषा॑ पुरोगा या॒हि सा॒ध्या ह॒विर॑दन्तु दे॒वाः॥~(५९)

{\anuvakamend[{अ॒ग्निष्ट्वा॑ वा॒मश्वो॒ द्विच॑त्वारिꣳशच्च}]}%॥11॥

\prashnaend{सा॒वि॒त्राणि॒ व्यृद्ध॒मुत्क्रा॑म दे॒वस्य॑ खनति क्रू॒रं वा॑रु॒णः स॒प्तभि॒रेक॑विꣳशत्या ष॒ड्भिर्न ह॑ स्म॒ समि॑द्धो अ॒ञ्जन्नेका॑दश॥११॥}{सा॒वि॒त्राण्युत्क्रा॑म क्रू॒रं वा॑रु॒णः प॒शवः॑ स्यु॒र्न ह॑ स्म॒ नव॑पञ्चा॒शत्॥५९॥}{सा॒वि॒त्राणि॑ ह॒विर॑दन्तु दे॒वाः॥}%%५-१
{हरिः॑ ॐ}{॥कृष्ण-यजुर्वेदीय-तैत्तिरीय-संहितायां पञ्चम्काण्डे प्रथमः प्रश्नः समाप्तः॥५-१॥}
%%% END PRASHNA
