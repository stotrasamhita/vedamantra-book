% !TeX program = XeLaTeX
% !TeX root = ../vedamantrabook.tex
\chapt{लघुन्यासः}

\sect{श्री रुद्रध्यानम्}
अथाऽऽत्मानं शिवात्मानं श्री रुद्र रूपं ध्यायेत्॥

शुद्धस्फटिकसङ्काशं त्रिनेत्रं पञ्चवक्त्रकम्।\\
गङ्गाधरं दशभुजं सर्वाभरणभूषितम्॥

नीलग्रीवं शशाङ्काङ्कं नागयज्ञोपवीतिनम्।\\
व्याघ्रचर्मोत्तरीयं च वरेण्यमभयप्रदम्॥

कमण्डल्वक्षसूत्राणां धारिणं शूलपाणिनम्।\\
ज्वलन्तं पिङ्गलजटाशिखामुद्योतधारिणम्॥

वृषस्कन्धसमारूढम् उमादेहार्धधारिणम्।\\
अमृते  नाप्लुतं शान्तं दिव्यभोगसमन्वितम्॥

दिग्देवता समायुक्तं सुरासुरनमस्कृतम्।\\
नित्यं च शाश्वतं शुद्धं ध्रुवमक्षरमव्ययम्॥

सर्वव्यापिनमीशानं रुद्रं वै विश्वरूपिणम्।\\
एवं ध्यात्वा द्विजः सम्यक् ततो यजनमारभेत्॥

अथातो रुद्र स्नानार्चनाभिषेकविधिं व्याख्यास्यामः। आदित एव तीर्थे स्नात्वा उदेत्य शुचिः
प्रयतो ब्रह्मचारी शुक्लवासा ईशानस्य प्रतिकृतिं कृत्वा तस्य दक्षिणप्रत्यग्देशे देवाभिमुखः स्थित्वा आत्मनि देवताः स्थापयेत्॥

\sect{देवता-स्थापनम्}
प्रजनने ब्रह्मा तिष्ठतु। पादयोर्विष्णुस्तिष्ठतु। 
हस्तयोर्हरस्तिष्ठतु। बाह्वोरिन्द्रस्तिष्ठतु। 
जठरे अग्निस्तिष्ठतु। हृदये शिवस्तिष्ठतु। 
कण्ठे वसवस्तिष्ठन्तु। वक्त्रे सरस्वती तिष्ठतु। 
नासिकयोर्\-वायुस्तिष्ठतु। नयनयोश्चन्द्रादित्यौ तिष्ठेताम्। 
कर्णयोरश्विनौ तिष्ठेताम्। ललाटे रुद्रास्तिष्ठन्तु। 
मूर्ध्न्यादित्यास्तिष्ठन्तु। शिरसि महादेवस्तिष्ठतु। 
शिखायां वामदेवस्तिष्ठतु। पृष्ठे पिनाकी तिष्ठतु। 
पुरतः शूली तिष्ठतु। पार्श्वयोः शिवाशङ्करौ तिष्ठेताम्। 
सर्वतो वायुस्तिष्ठतु। ततो बहिः सर्वतोऽग्निर्ज्वालामाला-परिवृतस्तिष्ठतु।
सर्वेष्वङ्गेषु सर्वा देवता यथास्थानं तिष्ठन्तु। मां रक्षन्तु।
\lbrack सर्वान् महाजनान् सकुटुम्बं रक्षन्तु॥\rbrack

अ॒ग्निर्मे॑ वा॒चि श्रि॒तः।   वाग्घृद॑ये।   हृद॑यं॒ मयि॑।   अ॒हम॒मृते᳚।   अ॒मृतं॒ ब्रह्म॑णि। (जिह्वा)

 वा॒युर्मे᳚ प्रा॒णे श्रि॒तः।   प्रा॒णो हृद॑ये।   हृद॑यं॒ मयि॑।   अ॒हम॒मृते᳚।   अ॒मृतं॒ ब्रह्म॑णि। (नासिका)

   सूर्यो॑ मे॒ चक्षु॑षि श्रि॒तः।   चक्षु॒र्॒‌हृद॑ये।   हृद॑यं॒ मयि॑।   अ॒हम॒मृते᳚।   अ॒मृतं॒ ब्रह्म॑णि। (नेत्रे)

   च॒न्द्रमा॑ मे॒ मन॑सि श्रि॒तः।   मनो॒ हृद॑ये।   हृद॑यं॒ मयि॑।   अ॒हम॒मृते᳚।   अ॒मृतं॒ ब्रह्म॑णि। (वक्षः)

   दिशो॑ मे॒ श्रोत्रे᳚ श्रि॒ताः।   श्रोत्र॒ꣳ॒ हृद॑ये।   हृद॑यं॒ मयि॑।   अ॒हम॒मृते᳚।   अ॒मृतं॒ ब्रह्म॑णि। (श्रोत्रे)

   आपो॑ मे॒ रेत॑सि श्रि॒ताः।   रेतो॒ हृद॑ये।   हृद॑यं॒ मयि॑।   अ॒हम॒मृते᳚।   अ॒मृतं॒ ब्रह्म॑णि। (गुह्यम्)

   पृ॒थि॒वी मे॒ शरी॑रे श्रि॒ता।   शरी॑र॒ꣳ॒ हृद॑ये।   हृद॑यं॒ मयि॑।   अ॒हम॒मृते᳚।   अ॒मृतं॒ ब्रह्म॑णि। (शरीरम्)

   ओ॒ष॒धि॒व॒न॒स्प॒तयो॑ मे॒ लोम॑सु श्रि॒ताः।   लोमा॑नि॒ हृद॑ये।   हृद॑यं॒ मयि॑।   अ॒हम॒मृते᳚।   अ॒मृतं॒ ब्रह्म॑णि। (लोमानि)

   इन्द्रो॑ मे॒ बले᳚ श्रि॒तः।   बल॒ꣳ॒ हृद॑ये।   हृद॑यं॒ मयि॑।   अ॒हम॒मृते᳚।   अ॒मृतं॒ ब्रह्म॑णि। (बाहू)

   प॒र्जन्यो॑ मे मू॒र्ध्नि श्रि॒तः।   मू॒र्धा हृद॑ये।   हृद॑यं॒ मयि॑।   अ॒हम॒मृते᳚।   अ॒मृतं॒ ब्रह्म॑णि। (शिरः)

   ईशा॑नो मे म॒न्यौ श्रि॒तः।   म॒न्युर्‌हृद॑ये।   हृद॑यं॒ मयि॑।    अ॒हम॒मृते᳚।   अ॒मृतं॒ ब्रह्म॑णि। (हृदयम्)

   आ॒त्मा म॑ आ॒त्मनि॑ श्रि॒तः।   आ॒त्मा हृद॑ये।   हृद॑यं॒ मयि॑।   अ॒हम॒मृते᳚।   अ॒मृतं॒ ब्रह्म॑णि।
(हृदयम्)

   पुन॑र्म आ॒त्मा पुन॒रायु॒रागा᳚त्।   पुनः॑ प्रा॒णः पुन॒राकू॑त॒मागा᳚त्।   वै॒श्वा॒न॒रो र॒श्मिभि॑र्वावृधा॒नः।   अ॒न्तस्ति॑ष्ठत्व॒मृत॑स्य गो॒पाः॥ (सर्वाण्यङ्गानि संस्पृश्य स्थापनं कृत्वा मानसैराराधयेत्॥)
{\closesection}

\sect{श्रीरुद्रजपः}

अस्य श्री रुद्राध्याय-प्रश्न-महामन्त्रस्य। अघोर ऋषिः।\\
अनुष्टुप् छन्दः। सङ्कर्षणमूर्तिस्वरूपो योऽसावादित्यः परमपुरुषः स एष रुद्रो देवता॥

नमः शिवायेति बीजम्। शिवतरायेति शक्तिः।\\
महादेवायेति कीलकम्।\\
श्री साम्बसदाशिवप्रसादसिद्ध्यर्थे जपे विनियोगः॥\\


\centerline{॥करन्यासः॥}
ॐ अग्निहोत्रात्मने अङ्गुष्ठाभ्यां नमः।\\
दर्शपूर्णमासात्मने तर्जनीभ्यां नमः।\\
चातुर्मास्यात्मने मध्यमाभ्यां नमः।\\
निरूढपशुबन्धात्मने अनामिकाभ्यां नमः।\\
ज्योतिष्टोमात्मने कनिष्ठिकाभ्यां नमः।\\
सर्वक्रत्वात्मने करतलकरपृष्ठाभ्यां  नमः।\\
%\pagebreak[4]


\centerline{॥अङ्गन्यासः॥}
अग्निहोत्रात्मने हॄदयाय नमः।\\
दर्शपूर्णमासात्मने शिरसे स्वाहा।\\
चातुर्मास्यात्मने शिखायै वषट्।\\
निरूढपशुबन्धात्मने कवचाय हुं।\\
ज्योतिष्टोमात्मने नेत्रत्रयाय वौषट्।\\
सर्वक्रत्वात्मने अस्त्राय फट्।\\
भूर्भुवस्सुवरोमिति दिग्बन्धः।\\

\sect{ध्यानम्}
\setlength{\shlokaspaceskip}{2pt}
\fourlineindentedshloka*
{आपाताल-नभः-स्थलान्त-भुवन-ब्रह्माण्डमाविस्फुरत्}
{ज्योतिः स्फाटिक-लिङ्ग-मौलि-विलसत्-पूर्णेन्दु-वान्तामृतैः}
{अस्तोकाप्लुतमेकमीशमनिशं रुद्रानुवाकान् जपन्}
{ध्यायेदीप्सितसिद्धये ध्रुवपदं विप्रोऽभिषिञ्चेच्छिवम्}

\fourlineindentedshloka*
{ब्रह्माण्ड-व्याप्त-देहा भसित-हिमरुचा भासमाना भुजङ्गैः}
{कण्ठे कालाः कपर्दा-कलित-शशि-कलाश्चण्ड-कोदण्ड-हस्ताः}
{त्र्यक्षा रुद्राक्षमालाः प्रकटित-विभवाः शाम्भवा मूर्तिभेदाः}
{रुद्राः श्रीरुद्रसूक्त-प्रकटित-विभवा नः प्रयच्छन्तु सौख्यम्}

\centerline{॥पञ्चपूजा॥}

लं पृथिव्यात्मने गन्धं समर्पयामि।\\
हं आकाशात्मने पूष्पैः पूजयामि।\\
यं वाय्वात्मने धूपमाघ्रापयामि।\\
रं अग्न्यात्मने दीपं दर्शयामि।\\
वं अमृतात्मने अमृतं महानैवेद्यं निवेदयामि।\\
सं सर्वात्मने सर्वोपचारपूजां समर्पयामि।

ॐ ग॒णानां᳚ त्वा ग॒णप॑तिꣳ हवामहे क॒विं क॑वी॒नामु॑प॒\-मश्र॑वस्तमम्। \\
ज्ये॒ष्ठ॒राजं॒ ब्रह्म॑णां ब्रह्मणस्पत॒ आ नः॑ शृ॒ण्वन्नू॒तिभिः॑ सीद॒ साद॑नम्॥ \\
ॐ महागणपतये॒ नमः॥ 


शं च॑ मे॒ मय॑श्च मे प्रि॒यं च॑ मेऽनुका॒मश्च॑ मे॒ काम॑श्च मे सौमन॒सश्च॑ मे भ॒द्रं च॑ मे॒ श्रेय॑श्च मे॒ वस्य॑श्च मे॒ यश॑श्च मे॒ भग॑श्च मे॒ द्रवि॑णं च मे य॒न्ता च॑ मे ध॒र्ता च॑ मे॒ क्षेम॑श्च मे॒ धृति॑श्च मे॒ विश्वं॑ च मे॒ मह॑श्च मे सं॒विच्च॑ मे॒ ज्ञात्रं॑ च मे॒ सूश्च॑ मे प्र॒सूश्च॑ मे॒ सीरं॑ च मे ल॒यश्च॑ म ऋ॒तं च॑ मे॒ऽमृतं॑ च मेऽय॒क्ष्मं च॒ मेऽना॑मयच्च मे जी॒वातु॑श्च मे दीर्घायु॒त्वं च॑ मेऽनमि॒त्रं च॒ मेऽभ॑यं च मे सु॒गं च॑ मे॒ शय॑नं च मे सू॒षा च॑ मे सु॒दिनं॑ च मे॥३॥ 
\centerline{॥ॐ शान्तिः॒ शान्तिः॒ शान्तिः॑॥}
{\closesection}