\sect{द्वितीयः प्रश्नः}
\setcounter{anuvakam}{0}
\dnsub{तैत्तिरीयब्राह्मणे प्रथमाष्टके द्वितीयः प्रपाठकः}

%1.2.1.1
उ॒द्ध॒न्यमा॑नम॒स्या अ॑मे॒ध्यम्। अप॑ पा॒प्मानं॒ यज॑मानस्य हन्तु। शि॒वा न॑ सन्तु प्र॒दिश॒श्चत॑स्रः। शं नो॑ मा॒ता पृ॑थि॒वी तोक॑साता। शं नो॑ दे॒वीर॒भिष्ट॑ये। आपो॑ भवन्तु पी॒तये। शंयोर॒भि स्र॑वन्तु नः। वै॒श्वा॒न॒रस्य॑ रू॒पम्। पृ॒थि॒व्यां प॑रि॒स्रसा। स्यो॒नमा वि॑शन्तु नः॥१॥

%1.2.1.2
यदि॒दं दि॒वो यद॒दः पृ॑थि॒व्याः। सं॒ज॒ज्ञा॒ने रोद॑सी सम्बभू॒वतु॑। ऊषान्कृ॒ष्णम॑वतु कृ॒ष्णमूषा। इ॒होभयोर्य॒ज्ञिय॒माग॑मिष्ठाः। ऊ॒तीः कु॑र्वा॒णो यत्पृ॑थि॒वीमच॑रः। गु॒हा॒कार॑माखुरू॒पं प्र॒तीत्य॑। तत्ते॒ न्य॑क्तमि॒ह स॒म्भर॑न्तः। श॒तं जी॑वेम श॒रद॒ सवी॑राः। ऊर्जं॑ पृथि॒व्या रस॑मा॒भर॑न्तः। श॒तं जी॑वेम श॒रद॑ पुरू॒चीः॥२॥

%1.2.1.3
व॒म्रीभि॒रनु॑वित्तं॒ गुहा॑सु। श्रोत्रं॑ त उ॒र्व्यब॑धिरा भवामः। प्र॒जाप॑तिसृष्टानां प्र॒जानाम्। क्षु॒धोऽप॑हत्यै सुवि॒तं नो॑ अस्तु। उप॒ प्रभि॑न्न॒मिष॒मूर्जं॑ प्र॒जाभ्य॑। सूदं॑ गृ॒हेभ्यो॒ रस॒माभ॑रामि। यस्य॑ रू॒पं बिभ्र॑दि॒मामवि॑न्दत्। गुहा॒ प्रवि॑ष्टा सरि॒रस्य॒ मध्ये। तस्ये॒दं विह॑तमा॒भर॑न्तः। अछ॑म्बट्कारम॒स्यां वि॑धेम॥३॥

%1.2.1.4
यत्प॒र्यप॑श्यत्सरि॒रस्य॒ मध्ये। उ॒र्वीमप॑श्य॒ज्जग॑तः प्रति॒ष्ठाम्। तत्पुष्क॑रस्या॒यत॑ना॒द्धि जा॒तम्। प॒र्णं पृ॑थि॒व्याः प्रथ॑न हरामि। याभि॒रदृह॒ज्जग॑तः प्रति॒ष्ठाम्। उ॒र्वीमि॒मां वि॑श्वज॒नस्य॑ भ॒र्त्रीम्। ता न॑ शि॒वाः शर्क॑राः सन्तु॒ सर्वा। अ॒ग्ने रेत॑श्च॒न्द्र हिर॑ण्यम्। अ॒द्भ्यः सम्भू॑तम॒मृतं॑ प्र॒जासु॑। तत्स॒म्भर॑न्नुत्तर॒तो नि॒धाय॑॥४॥

%1.2.1.5
अ॒ति॒प्र॒यच्छं॒ दुरि॑तिं तरेयम्। अश्वो॑ रू॒पं कृ॒त्वा यद॑श्व॒त्थेऽति॑ष्ठः। सं॒व॒त्स॒रं दे॒वेभ्यो॑ नि॒लाय॑। तत्ते॒ न्य॑क्तमि॒ह स॒म्भर॑न्तः। श॒तं जी॑वेम श॒रद॒ सवी॑राः। ऊ॒र्जः पृ॑थि॒व्या अध्युत्थि॑तोऽसि। वन॑स्पते श॒तव॑ल्‌शो॒ विरो॑ह। त्वया॑ व॒यमिष॒मूर्जं॒ मद॑न्तः। रा॒यस्पोषे॑ण॒ समि॒षा म॑देम। गा॒य॒त्रि॒या ह्रि॒यमा॑णस्य॒ यत्ते॥५॥

%1.2.1.6
प॒र्णमप॑तत्तृ॒तीय॑स्यै दि॒वोऽधि॑। सो॑ऽयं प॒र्णः सो॑मप॒र्णाद्धि जा॒तः। ततो॑ हरामि सोमपी॒थस्याव॑रुद्ध्यै। दे॒वानां ब्रह्मवा॒दं वद॑तां॒ यत्। उ॒पाशृ॑णोः सु॒श्रवा॒ वै श्रु॒तो॑ऽसि। ततो॒ मामावि॑शतु ब्रह्मवर्च॒सम्। तत्स॒म्भर॒ स्तदव॑रुन्धीय सा॒क्षात्। यया॑ ते सृ॒ष्टस्या॒ग्नेः। हे॒तिमश॑मयत्प्र॒जाप॑तिः। तामि॒मामप्र॑दाहाय॥६॥

%1.2.1.7
श॒मी शान्त्यै॑ हराम्य॒हम्। यत्ते॑ सृ॒ष्टस्य॑ य॒तः। विक॑ङ्कतं॒ भा आर्च्छज्जातवेदः। तया॑ भा॒सा सम्मि॑तः। उ॒रुं नो॑ लो॒कमनु॒ प्रभा॑हि। यत्ते॑ ता॒न्तस्य॒ हृद॑य॒माच्छि॑न्दञ्जातवेदः। म॒रुतो॒ऽद्भिस्त॑मयि॒त्वा। ए॒तत्ते॒ तद॑श॒नेः सम्भ॑रामि। सात्मा॑ अग्ने॒ सहृ॑दयो भवे॒ह। चित्रि॑यादश्व॒त्थात्सम्भृ॑ता बृह॒त्य॑॥७॥

%1.2.1.8
शरी॑रम॒भि सस्कृ॑ताः स्थ। प्र॒जाप॑तिना यज्ञमु॒खेन॒ सम्मि॑ताः। ति॒स्रस्त्रि॒वृद्भि॑र्मिथु॒नाः प्रजात्यै। अ॒श्व॒त्थाद्ध॑व्य\-वा॒हाद्धि जा॒ताम्। अ॒ग्नेस्त॒नूं य॒ज्ञिया॒ सम्भ॑रामि। शा॒न्तयो॑नि शमीग॒र्भम्। अ॒ग्नये॒ प्रज॑नयि॒तवे। यो अ॑श्व॒त्थः श॑मीग॒र्भः। आ॒रु॒रोह॒ त्वे सचा। तं ते॑ हरामि॒ ब्रह्म॑णा॥८॥

%1.2.1.9
य॒ज्ञियै के॒तुभि॑ स॒ह। यं त्वा॑ स॒मभ॑रञ्जातवेदः। य॒था॒श॒री॒रं भू॒तेषु॒ न्य॑क्तम्। स सम्भृ॑तः सीद शि॒वः प्र॒जाभ्य॑। उ॒रुं नो॑ लो॒कमनु॑नेषि वि॒द्वान्। प्रवे॒धसे॑ क॒वये॒ मेध्या॑य। वचो॑ व॒न्दारु॑ वृष॒भाय॒ वृष्णे। यतो॑ भ॒यमभ॑यं॒ तन्नो॑ अस्तु। अव॑ दे॒वान् य॑जे॒हेड्यान्॑। स॒मिधा॒ऽग्निं दु॑वस्यत॥९॥

%1.2.1.10
घृ॒तैर्बो॑धय॒ताति॑थिम्। आऽस्मि॑न् ह॒व्या जु॑होतन। उप॑ त्वाऽग्ने ह॒विष्म॑तीः। घृ॒ताचीर्यन्तु हर्यत। जु॒षस्व॑ स॒मिधो॒ मम॑। तं त्वा॑ स॒मिद्भि॑रङ्गिरः। घृ॒तेन॑ वर्धयामसि। बृ॒हच्छो॑चा यविष्ठ्य। स॒मि॒ध्यमा॑नः प्रथ॒मो नु धर्म॑। सम॒क्तुभि॑रज्यते वि॒श्ववा॑रः॥१०॥

%1.2.1.11
शो॒चिष्के॑शो घृ॒तनि॑र्णिक्पाव॒कः। सु॒य॒ज्ञो अ॒ग्निर्य॒जथा॑य दे॒वान्। घृ॒तप्र॑तीको घृ॒तयो॑निर॒ग्निः। घृ॒तैः समि॑द्धो घृ॒तम॒स्यान्नम्। घृ॒त॒प्रुष॑स्त्वा स॒रितो॑ वहन्ति। घृ॒तं पिबन्त्सु॒यजा॑ यक्षि दे॒वान्। आ॒यु॒र्दा अ॑ग्ने ह॒विषो॑ जुषा॒णः। घृ॒तप्र॑तीको घृ॒तयो॑निरेधि। घृ॒तं पी॒त्वा मधु॒ चारु॒ गव्यम्। पि॒तेव॑ पु॒त्रम॒भिर॑क्षतादि॒मम्॥११॥

%1.2.1.12
त्वाम॑ग्ने समिधा॒नं य॑विष्ठ। दे॒वा दू॒तं च॑क्रिरे हव्य॒वाहम्। उ॒रु॒ज्रय॑सं घृ॒तयो॑नि॒माहु॑तम्। त्वे॒षं चक्षु॑र्दधिरे चोद॒यन्व॑ति। त्वाम॑ग्ने प्र॒दिव॒ आहु॑तं घृ॒तेन॑। सु॒म्ना॒यव॑ सुष॒मिधा॒ समी॑धिरे। स वा॑वृधा॒न ओष॑धीभिरुक्षि॒तः। उ॒रु ज्रयासि॒ पार्थि॑वा॒ विति॑ष्ठसे। घृ॒तप्र॑तीकं व ऋ॒तस्य॑ धूर्॒षदम्। अ॒ग्निं मि॒त्रं न स॑मिधा॒न ऋ॑ञ्जते॥१२॥

%1.2.1.13
इन्धा॑नो अ॒क्रो वि॒दथे॑षु॒ दीद्य॑त्। शु॒क्रव॑र्णा॒मुदु॑ नो यसते॒ धियम्। प्र॒जा अ॑ग्ने॒ संवा॑सय। आशाश्च प॒शुभि॑ स॒ह। रा॒ष्ट्राण्य॑स्मा॒ आधे॑हि। यान्यासन्त्सवि॒तुः स॒वे। म॒ही वि॒श्पत्नी॒ सद॑ने ऋ॒तस्य॑। अ॒र्वाची॒ एतं॑ धरुणे रयी॒णाम्। अ॒न्तर्व॑त्नी॒ जन्यं॑ जा॒तवे॑दसम्। अ॒ध्व॒राणां जनयथः पुरो॒गाम्॥१३॥

%1.2.1.14
आरो॑हतं द॒शत॒ शक्व॑री॒र्मम॑। ऋ॒तेनाग्न॒ आयु॑षा॒ वर्च॑सा स॒ह। ज्योग्जीव॑न्त॒ उत्त॑रामुत्तरा॒ समाम्। दर्\mbox{}श॑म॒हं पू॒र्णमा॑सं य॒ज्ञं यथा॒ यजै। ऋत्वि॑यवती स्थो अ॒ग्निरे॑तसौ। गर्भं॑ दधाथां॒ ते वा॑म॒हं द॑दे। तत्स॒त्यं यद्वी॒रं बि॑भृथः। वी॒रं ज॑नयि॒ष्यथ॑। ते मत्प्रा॒तः प्रज॑निष्येथे। ते मा॒ प्रजा॑ते॒ प्रज॑नयि॒ष्यथ॑॥१४॥

%1.2.1.15
प्र॒जया॑ प॒शुभि॑र्ब्रह्मवर्च॒सेन॑ सुव॒र्गे लो॒के। अनृ॑तात्स॒त्यमुपै॑मि। मा॒नु॒षाद्देव्य॒मुपै॑मि। दैवीं॒ वाचं॑ यच्छामि। शल्कै॑र॒ग्निमि॑न्धा॒नः। उ॒भौ लो॒कौ स॑नेम॒हम्। उ॒भयोर्लो॒कयोर्॑ ऋ॒ध्वा। अति॑ मृ॒त्युं त॑राम्य॒हम्। जात॑वेदो॒ भुव॑नस्य॒ रेत॑। इ॒ह सि॑ञ्च॒ तप॑सो॒ यज्ज॑नि॒ष्यते॥१५॥

%1.2.1.16
अ॒ग्निम॑श्व॒त्थादधि॑ हव्य॒वाहम्। श॒मी॒ग॒र्भाज्ज॒नय॒न्॒ यो म॑यो॒भूः। अ॒यं ते॒ योनि॑र्‌ऋ॒त्विय॑। यतो॑ जा॒तो अरो॑चथाः। तं जा॒नन्न॑ग्न॒ आरो॑ह। अथा॑ नो वर्धया र॒यिम्। अपे॑त॒ वीत॒ वि च॑ सर्प॒तात॑। येऽत्र॒ स्थ पु॑रा॒णा ये च॒ नूत॑नाः। अदा॑दि॒दं य॒मो॑ऽव॒सानं॑ पृथि॒व्याः। अक्र॑न्नि॒मं पि॒तरो॑ लो॒कम॑स्मै॥१६॥

%1.2.1.17
अ॒ग्नेर्भस्मास्य॒ग्नेः पुरी॑षमसि। सं॒ज्ञान॑मसि काम॒धर॑णम्। मयि॑ ते काम॒धर॑णं भूयात्। संव॑ सृजामि॒ हृद॑यानि। ससृ॑ष्टं॒ मनो॑ अस्तु वः। संसृ॑ष्टः प्रा॒णो अ॑स्तु वः। सं या व॑ प्रि॒यास्त॒नुव॑। सं प्रि॒या हृद॑यानि वः। आ॒त्मा वो॑ अस्तु॒ संप्रि॑यः। संप्रि॑यास्त॒नुवो॒ मम॑॥१७॥

%1.2.1.18
कल्पे॑तां॒ द्यावा॑पृथि॒वी। कल्प॑न्ता॒माप॒ ओष॑धीः। कल्प॑न्ताम॒ग्नय॒ पृथ॑क्। मम॒ ज्यैष्ठ्या॑य॒ सव्र॑ताः। येऽग्नय॒ सम॑नसः। अ॒न्त॒रा द्यावा॑पृथि॒वी। वास॑न्तिकावृ॒तू अ॒भि कल्प॑मानाः। इन्द्र॑मिव दे॒वा अ॒भि सं वि॑शन्तु। दि॒वस्त्वा॑ वी॒र्ये॑ण। पृ॒थि॒व्यै म॑हि॒म्ना॥१८॥

%1.2.1.19
अ॒न्तरि॑क्षस्य॒ पोषे॑ण। स॒र्वप॑शु॒माद॑धे। अजी॑जनन्न॒मृतं॒ मर्त्या॑सः। अ॒स्रे॒माणं॑ त॒रणिं॑ वी॒डुज॑म्भम्। दश॒ स्वसा॑रो अ॒ग्रुव॑ समी॒चीः। पुमासं जा॒तम॒भि सर॑भन्ताम्। प्र॒जाप॑तेस्त्वा प्रा॒णेनाभि॒ प्राणि॑मि। पू॒ष्णः पोषे॑ण॒ मह्यम्। दी॒र्घा॒यु॒त्वाय॑ श॒तशा॑रदाय। श॒त श॒रद्भ्य॒ आयु॑षे॒ वर्च॑से॥१९॥

%1.2.1.20
जी॒वात्वै पुण्या॑य। अ॒हं त्वद॑स्मि॒ मद॑सि॒ त्वमे॒तत्। ममा॑सि॒ योनि॒स्तव॒ योनि॑रस्मि। ममै॒व सन्वह॑ ह॒व्यान्य॑ग्ने। पु॒त्रः पि॒त्रे लो॑क॒कृज्जा॑तवेदः। प्रा॒णे त्वा॒ऽमृत॒माद॑धामि। अ॒न्ना॒दम॒न्नाद्या॑य। गो॒प्तारं॒ गुप्त्यै। सु॒गा॒र्॒ह॒प॒त्यो वि॒दह॒न्नरा॑तीः। उ॒षस॒ श्रेय॑सीः श्रेयसी॒र्दध॑त्॥२०॥

%1.2.1.21
अग्ने॑ स॒पत्ना अप॒ बाध॑मानः। रा॒यस्पोष॒मिष॒मूर्ज॑म॒स्मासु॑ धेहि। इ॒मा उ॒ मामुप॑तिष्ठन्तु॒ राय॑। आ॒भिः प्र॒जाभि॑रि॒ह संव॑सेय। इ॒हो इडा॑ तिष्ठतु विश्वरू॒पी। मध्ये॒ वसोर्दीदिहि जातवेदः। ओज॑से॒ बला॑य॒ त्वोद्य॑च्छे। वृष॑णे॒ शुष्मा॒यायु॑षे॒ वर्च॑से। स॒प॒त्न॒तूर॑सि वृत्र॒तूः। यस्ते॑ दे॒वेषु॑ महि॒मा सु॑व॒र्गः॥२१॥

%1.2.1.22
यस्त॑ आ॒त्मा प॒शुषु॒ प्रवि॑ष्टः। पुष्टि॒र्या ते॑ मनु॒ष्ये॑षु पप्र॒थे। तया॑ नो अग्ने जु॒षमा॑ण॒ एहि॑। दि॒वः पृ॑थि॒व्याः पर्य॒न्तिरि॑क्षात्। वातात्प॒शुभ्यो॒ अध्योष॑धीभ्यः। यत्र॑ यत्र जातवेदः सम्ब॒भूथ॑। ततो॑ नो अग्ने जु॒षमा॑ण॒ एहि॑। प्राची॒मनु॑ प्र॒दिशं॒ प्रेहि॑ वि॒द्वान्। अ॒ग्नेर॑ग्ने पु॒रोअ॑ग्निर्भवे॒ह। विश्वा॒ आशा॒ दीद्या॑नो॒ वि भा॑हि॥२२॥

%1.2.1.23
ऊर्जं॑ नो धेहि द्वि॒पदे॒ चतु॑ष्पदे। अन्व॒ग्निरु॒षसा॒मग्र॑मख्यत्। अन्वहा॑नि प्रथ॒मो जा॒तवे॑दाः। अनु॒ सूर्य॑स्य पुरु॒त्रा च॑ र॒श्मीन्। अनु॒ द्यावा॑पृथि॒वी आत॑तान। विक्र॑मस्व म॒हा अ॑सि। वे॒दि॒षन्मानु॑षेभ्यः। त्रि॒षु लो॒केषु॑ जागृहि। यदि॒दं दि॒वो यद॒दः पृ॑थि॒व्याः। सं॒वि॒दा॒ने रोद॑सी सं बभू॒वतु॑॥२३॥

%1.2.1.24
तयो पृ॒ष्ठे सी॑दतु जा॒तवे॑दाः। श॒म्भूः प्र॒जाभ्य॑स्त॒नुवे स्यो॒नः। प्रा॒णं त्वा॒ऽमृत॒ आ द॑धामि। अ॒न्ना॒दम॒न्नाद्या॑य। गो॒प्तारं॒ गुप्त्यै। यत्ते॑ शुक्र शु॒क्रं वर्च॑ शु॒क्रा त॒नूः। शु॒क्रं ज्योति॒रज॑स्रम्। तेन॑ मे दीदिहि॒ तेन॒ त्वाऽऽद॑धे। अ॒ग्निनाऽग्ने॒ ब्रह्म॑णा। आ॒न॒शे व्या॑नशे॒ सर्व॒मायु॒र्व्या॑नशे॥२४॥

%1.2.1.25
नर्य॑ प्र॒जां मे॑ गोपाय। अ॒मृ॒त॒त्वाय॑ जी॒वसे। जा॒तां ज॑नि॒ष्यमा॑णां च। अ॒मृते॑ स॒त्ये प्रति॑ष्ठिताम्। अथ॑र्व पि॒तुं मे॑ गोपाय। रस॒मन्न॑मि॒हायु॑षे। अद॑ब्धा॒योऽशी॑ततनो। अवि॑षन्नः पि॒तुं कृ॑णु। शस्य॑ प॒शून्मे॑ गोपाय। द्वि॒पादो॒ ये चतु॑ष्पदः॥२५॥

%1.2.1.26
अ॒ष्टाश॑फाश्च॒ य इ॒हाग्ने। ये चैक॑शफा आशु॒गाः। सप्र॑थ स॒भां मे॑ गोपाय। ये च॒ सभ्या सभा॒सद॑। तानि॑न्द्रि॒याव॑तः कुरु। सर्व॒मायु॒रुपा॑सताम्। अहे॑ बुध्निय॒ मन्त्रं॑ मे गोपाय। यमृष॑यस्त्रैवि॒दा वि॒दुः। ऋच॒ सामा॑नि॒ यजूषि। सा हि श्रीर॒मृता॑ स॒ताम्॥२६॥

%1.2.1.27
चतु॑ शिखण्डा युव॒तिः सु॒पेशा। घृ॒तप्र॑तीका॒ भुव॑नस्य॒ मध्ये। म॒र्मृ॒ज्यमा॑ना मह॒ते सौभ॑गाय। मह्यं॑ धुक्ष्व॒ यज॑मानाय॒ कामान्॑। इ॒हैव सन्तत्र॑ स॒तो वो॑ अग्नयः। प्रा॒णेन॑ वा॒चा मन॑सा बिभर्मि। ति॒रो मा॒ सन्त॒मायु॒र्मा प्रहा॑सीत्। ज्योति॑षा वो वैश्वान॒रेणोप॑तिष्ठे। प॒ञ्च॒धाऽग्नीन्व्य॑क्रामत्। वि॒राट्त्सृ॒ष्टा प्र॒जाप॑तेः। ऊ॒र्ध्वाऽऽरो॑हद्रोहि॒णी। योनि॑र॒ग्नेः प्रति॑ष्ठितिः॥२७॥\anuvakamend[वि॒श॒न्तु॒ न॒ पु॒रू॒चीर्वि॑धेम नि॒धाय॒ यत्तेऽप्र॑दाहाय बृह॒त्यो ब्रह्म॑णा दुवस्यत वि॒श्ववा॑र इ॒ममृ॑ञ्जते पुरो॒गां प्रज॑नयि॒ष्यथो॑ जनि॒ष्यतेऽस्मै॒ मम॑ महि॒म्ना वर्च॑से॒ दध॑त्सुव॒र्गो भा॑हि सम्बभू॒वतु॒रायु॒र्व्या॑नशे॒ चतु॑ष्सदः स॒तां प्र॒जाप॑ते॒र्द्वे च॑]

%1.2.2.1
नवै॒तान्यहा॑नि भवन्ति। नव॒ वै सु॑व॒र्गा लो॒काः। यदे॒तान्यहान्युप॒यन्ति॑। न॒वस्वे॒व तत्सु॑व॒र्गेषु॑ लो॒केषु॑ स॒त्रिण॑ प्रति॒तिष्ठ॑न्तो यन्ति। अ॒ग्नि॒ष्टो॒माः पर॑ सामानः का॒र्या॑ इत्या॑हुः। अ॒ग्नि॒ष्टो॒मसं॑मितः सुव॒र्गो लो॒क इति॑। द्वाद॑शाग्निष्टो॒मस्य॑ स्तो॒त्राणि॑। द्वाद॑श॒ मासा संवत्स॒रः। तत्तन्न सूर्क्ष्यम्। उ॒क्थ्या॑ ए॒व स॑प्तद॒शाः पर॑ सामानः का॒र्या॥२८॥

%1.2.2.2
प॒शवो॒ वा उ॒क्थानि॑। प॒शू॒नामव॑रुद्ध्यै। वि॒श्व॒जि॒द॒भि॒जिता॑\-वग्निष्टो॒मौ। उ॒क्थ्या सप्तद॒शाः पर॑ समानः। ते सस्तु॑ता वि॒राज॑म॒भि सम्प॑द्यन्ते। द्वे चर्चा॒वति॑रिच्येते। एक॑या॒ गौरति॑रिक्तः। एक॒याऽऽयु॑रू॒नः। सु॒व॒र्गो वै लो॒को ज्योति॑। ऊर्ग्वि॒राट्॥२९॥

%1.2.2.3
सु॒व॒र्गमे॒व तेन॑ लो॒कम॒भि ज॑यन्ति। यत्पर॒ राथ॑न्तरम्। तत्प्र॑थ॒मेऽह॑न्का॒र्यम्। बृ॒हद्द्वि॒तीये। वै॒रू॒पं तृ॒तीये। वै॒रा॒जं च॑तु॒र्थे। शा॒क्व॒रं प॑ञ्च॒मे। रै॒व॒त ष॒ष्ठे। तदु॑ पृ॒ष्ठेभ्यो॒ नय॑न्ति। स॒न्तन॑य ए॒ते ग्रहा॑ गृह्यन्ते॥३०॥

%1.2.2.4
अ॒ति॒ग्रा॒ह्या पर॑ सामसु। इ॒माने॒वैतैर्लो॒कान्त्संत॑न्वन्ति। मि॒थु॒ना ए॒ते ग्रहा॑ गृह्यन्ते। अ॒ति॒ग्रा॒ह्या पर॑ सामसु। मि॒थु॒नमे॒व तैर्यज॑माना॒ अव॑रुन्धते। बृ॒हत्पृ॒ष्ठं भ॑वति। बृ॒हद्वै सु॑व॒र्गो लो॒कः। बृ॒ह॒तैव सु॑व॒र्गं लो॒कं य॑न्ति। त्र॒य॒स्त्रि॒शिनाम॒ साम॑। माध्य॑न्दिने॒ पव॑माने भवति॥३१॥

%1.2.2.5
त्रय॑स्त्रिश॒द्वै दे॒वता। दे॒वता॑ ए॒वाव॑रुन्धते। ये वा इ॒तः पराञ्च संवत्स॒रमु॑प॒यन्ति॑। न है॑नं॒ ते स्व॒स्ति सम॑श्ञुवते। अथ॒ ये॑ऽमुतो॒ऽर्वाञ्च॑मुप॒यन्ति॑। ते है॑न स्व॒स्ति सम॑श्ञुवते। ए॒तद्वा अ॒मुतो॒ऽर्वाञ्च॒मुप॑यन्ति। यदे॒वम्। यो ह॒ खलु॒ वाव प्र॒जाप॑तिः। स उ॑वे॒वेन्द्र॑। तदु॑ दे॒वेभ्यो॒ नय॑न्ति॥३२॥\anuvakamend[का॒र्या॑ वि॒राड्गृ॑ह्यन्ते॒ पव॑माने भव॒तीन्द्र॒ एकं॑ च]

%1.2.3.1
संत॑ति॒र्वा ए॒ते ग्रहा। यत्पर॑ सामानः। वि॒षू॒वान्दि॑वाकी॒र्त्यम्। यथा॒ शाला॑यै॒ पक्ष॑सी। ए॒व सं॑वत्स॒रस्य॒ पक्ष॑सी। यदे॒तेन गृ॒ह्येर\sn{}। विषू॑ची संवत्स॒रस्य॒ पक्ष॑सी॒ व्यव॑स्रसेयाताम्। आर्ति॒मार्च्छे॑युः। यदे॒ते गृ॒ह्यन्ते। यथा॒ शाला॑यै॒ प॑क्षसी मध्य॒मं व॒शम॒भि स॑मा॒यच्छ॑ति॥३३॥

%1.2.3.2
ए॒व सं॑वत्स॒रस्य॒ पक्ष॑सी दिवाकी॒र्त्य॑म॒भि सं त॑न्वन्ति। नार्ति॒मार्च्छ॑न्ति। ए॒क॒वि॒शमह॑र्भवति। शु॒क्राग्रा॒ ग्रहा॑ गृह्यन्ते। प्रत्युत्त॑ब्ध्यै सय॒त्वाय॑। सौ॒र्य॑ ए॒तदह॑ प॒शुराल॑भ्यते। सौ॒र्यो॑ऽतिग्रा॒ह्यो॑ गृह्यते। अह॑रे॒व रू॒पेण॒ सम॑र्धयन्ति। अथो॒ अह्न॑ ए॒वैष ब॒लिर्ह्रि॑यते। स॒प्तैतदह॑रतिग्रा॒ह्या॑ गृह्यन्ते॥३४॥

%1.2.3.3
स॒प्त वै शी॑र्\mbox{}ष॒ण्या प्रा॒णाः। अ॒सावा॑दि॒त्यः शिर॑ प्र॒जानाम्। शी॒र्॒षन्ने॒व प्र॒जानां प्रा॒णान्द॑धाति। तस्मात्स॒प्त शी॒र्॒षन्प्रा॒णाः। इन्द्रो॑ वृ॒त्र ह॒त्वा। असु॑रान्परा॒भाव्य॑। स इ॒माल्लोँ॒कान॒भ्य॑जयत्। तस्या॒सौ लो॒कोऽन॑भिजित आसीत्। तं वि॒श्वक॑र्मा भू॒त्वाऽभ्य॑जयत्। यद्वैश्वकर्म॒णो गृ॒ह्यते॥३५॥

%1.2.3.4
सु॒व॒र्गस्य॑ लो॒कस्या॒भिजि॑त्यै। प्र वा ए॒तेऽस्माल्लो॒काच्च्य॑वन्ते। ये वैश्वकर्म॒णं गृ॒ह्णते। आ॒दि॒त्यः श्वो गृ॑ह्यते। इ॒यं वा अदि॑तिः। अ॒स्यामे॒व प्रति॑ तिष्ठन्ति। अ॒न्योन्यो गृह्येते। विश्वान्ये॒वान्येन॒ कर्मा॑णि कुर्वा॒णा य॑न्ति। अ॒स्याम॒न्येन॒ प्रति॑ तिष्ठन्ति। तावाऽप॑रा॒र्धात्सं॑वत्स॒रस्या॒न्योन्यो गृह्येते। तावु॒भौ स॒ह म॑हाव्र॒ते गृ॑ह्येते। य॒ज्ञस्यै॒वान्तं॑ ग॒त्वा। उ॒भयोर्लो॒कयो॒ प्रति॑तिष्ठन्ति। अ॒र्क्य॑मु॒क्थं भ॑वति। अ॒न्नाद्य॒स्याव॑रुध्यै॥३६॥\anuvakamend[स॒मा॒यच्छ॑त्यतिग्रा॒ह्या॑ गृह्यन्ते गृ॒ह्यते॑ संवत्स॒रस्या॒न्योन्यो गृह्येते॒ पञ्च॑ च]

%1.2.4.1
ए॒क॒वि॒श ए॒ष भ॑वति। ए॒तेन॒ वै दे॒वा ए॑कवि॒शेन॑। आ॒दि॒त्यमि॒त उ॑त्त॒म सु॑व॒र्गं लो॒कमारो॑हयन्। स वा ए॒ष इ॒त ए॑कवि॒शः। तस्य॒ दशा॒वस्ता॒दहा॑नि। दश॑ प॒रस्तात्। स वा ए॒ष वि॒राज्यु॑भ॒यत॒ प्रति॑ष्ठितः। वि॒राजि॒ हि वा ए॒ष उ॑भ॒यत॒ प्रति॑ष्ठितः। तस्मा॑दन्त॒रेमौ लो॒कौ यन्। सर्वे॑षु सुव॒र्गेषु॑ लो॒केष्व॑भि॒तप॑न्नेति॥३७॥

%1.2.4.2
दे॒वा वा आ॑दि॒त्यस्य॑ सुव॒र्गस्य॑ लो॒कस्य॑। परा॑चोऽतिपा॒दाद॑बिभयुः। तं छन्दो॑भिरदृह॒न्धृत्यै। दे॒वा वा आ॑दि॒त्यस्य॑ सुव॒र्गस्य॑ लो॒कस्य॑। अवा॑चोऽवपा॒दाद॑बिभयुः। तं प॒ञ्चभी॑ र॒श्मिभि॒रुद॑वयन्। तस्मा॑देकवि॒शेऽह॒न्पञ्च॑ दिवाकी॒र्त्या॑नि क्रियन्ते। र॒श्मयो॒ वै दि॑वाकी॒र्त्या॑नि। ये गा॑य॒त्रे। ते गा॑य॒त्रीषूत्त॑रयो॒ पव॑मानयोः॥३८॥

%1.2.4.3
म॒हादि॑वाकीर्त्य॒ होतु॑ पृ॒ष्ठम्। वि॒क॒र्णं ब्र॑ह्मसा॒मम्। भा॒सोऽग्निष्टो॒मः। अथै॒तानि॒ परा॑णि। परै॒र्वै दे॒वा आ॑दि॒त्य सु॑व॒र्गं लो॒कम॑पारयन्। यदपा॑रयन्। तत्परा॑णां पर॒त्वम्। पा॒रय॑न्त्येनं॒ परा॑णि। य ए॒वं वेद॑। अथै॒तानि॒ स्परा॑णि। स्परै॒र्वै दे॒वा आ॑दि॒त्य सु॑व॒र्गं लो॒कम॑स्पारयन्। यदस्पा॑रयन्। तत्स्परा॑णा स्पर॒त्वम्। स्पा॒रय॑न्त्यैन॒ स्परा॑णि। य ए॒वं वेद॑॥३९॥\anuvakamend[ए॒ति॒ पव॑मानयो॒ स्परा॑णि॒ पञ्च॑ च]

%1.2.5.1
अप्र॑तिष्ठां॒ वा ए॒ते ग॑च्छन्ति। येषा संवत्स॒रेऽना॒प्तेऽथ॑। ए॒का॒द॒शिन्या॒प्यते। वै॒ष्ण॒वं वा॑म॒नमा ल॑भन्ते। य॒ज्ञो वै विष्णु॑। य॒ज्ञमे॒वाल॑भन्ते॒ प्रति॑ष्ठित्यै। ऐ॒न्द्रा॒ग्नमाल॑भन्ते। इ॒न्द्रा॒ग्नी वै दे॒वाना॒मया॑तयामानौ। ये ए॒व दे॒वते॒ अया॑तयाम्नी। ते ए॒वाल॑भन्ते॥४०॥

%1.2.5.2
वै॒श्व॒दे॒वमाल॑भन्ते। दे॒वता॑ ए॒वाव॑रुन्धते। द्या॒वा॒पृ॒थिव्यां धे॒नुमाल॑भन्ते। द्यावा॑पृथि॒व्योरे॒व प्रति॑ तिष्ठन्ति। वा॒य॒व्यं॑ व॒त्समाल॑भन्ते। वा॒युरे॒वैभ्यो॑ यथाऽऽयत॒नाद्दे॒वता॒ अव॑रुन्धे। आ॒दि॒त्यामविं॑ व॒शामाल॑भन्ते। इ॒यं वा अदि॑तिः। अ॒स्यामे॒व प्रति॑ तिष्ठन्ति। मै॒त्रा॒व॒रु॒णीमाल॑भन्ते॥४१॥

%1.2.5.3
मि॒त्रेणै॒व य॒ज्ञस्य॒ स्वि॑ष्ट शमयन्ति। वरु॑णेन॒ दुरि॑ष्टम्। प्रा॒जा॒प॒त्यं तू॑प॒रं म॑हाव्र॒त आल॑भन्ते। प्रा॒जा॒प॒त्यो॑ऽतिग्रा॒ह्यो॑ गृह्यते। अह॑रे॒व रू॒पेण॒ सम॑र्धयन्ति। अथो॒ अह्न॑ ए॒वैष ब॒लिर्ह्रि॑यते। आ॒ग्ने॒यमा ल॑भन्ते॒ प्रति॒ प्रज्ञात्यै। अ॒ज॒पे॒त्वान् वा ए॒ते पूर्वै॒र्मासै॒रव॑ रुन्धते। यदे॒ते ग॒व्याः प॒शव॑ आल॒भ्यन्ते। उ॒भये॑षां पशू॒नामव॑रुद्ध्यै॥४२॥

%1.2.5.4
यदति॑रिक्तामेकाद॒शिनी॑मा॒लभे॑रन्। अप्रि॑यं॒ भ्रातृ॑व्यम॒भ्यति॑\-रिच्येत। यद्द्वौ द्वौ॑ प॒शू स॒मस्ये॑युः। कनी॑य॒ आयु॑ कुर्वीरन्। यदे॒ते ब्राह्म॑णवन्तः प॒शव॑ आल॒भ्यन्ते। नाप्रि॑यं॒ भ्रातृ॑व्यम॒भ्य॑ति॒रिच्य॑ते। न कनी॑य॒ आयु॑ कुर्वते॥४३॥\anuvakamend[ते ए॒वाल॑भन्ते मैत्रावरु॒णीमाल॑भ॒न्तेऽव॑रुद्ध्यै स॒प्त च॑]

%1.2.6.1
प्र॒जाप॑तिः प्र॒जाः सृ॒ष्ट्वा वृ॒त्तो॑ऽशयत्। तं दे॒वा भू॒ताना॒ रसं॒ तेज॑ स॒म्भृत्य॑। तेनै॑नमभिषज्यन्। म॒हान॑वव॒र्तीति॑। तन्म॑हाव्र॒तस्य॑ महाव्रत॒त्वम्। म॒हद्व्र॒तमिति॑। तन्म॑हाव्र॒तस्य॑ महाव्रत॒त्वम्। म॒ह॒तो व्र॒तमिति॑। तन्म॑हाव्र॒तस्य॑ महाव्रत॒त्वम्। प॒ञ्च॒वि॒शः स्तोमो॑ भवति॥४४॥

%1.2.6.2
चतु॑र्विशत्यर्धमासः संवत्स॒रः। यद्वा ए॒तस्मिन्त्संवत्स॒रेऽधि॒ प्राजा॑यत। तदन्नं॑ पञ्चवि॒शम॑भवत्। म॒ध्य॒तः क्रि॑यते। म॒ध्य॒तो ह्यन्न॑मशि॒तं धि॒नोति॑। अथो॑ मध्य॒त ए॒व प्र॒जाना॒मूर्ग्धी॑यते। अथ॒ यद्वा इ॒दम॑न्त॒तः क्रि॒यते। तस्मा॑दुद॒न्ते प्र॒जाः समे॑धन्ते। अ॒न्त॒तः क्रि॑यते प्र॒जन॑नायै॒व। त्रि॒वृच्छिरो॑ भवति॥४५॥

%1.2.6.3
त्रे॒धा॒वि॒हि॒त हि शिर॑। लोम॑ छ॒वीरस्थि॑। परा॑चा स्तुवन्ति। तस्मा॒त्तत्स॒दृगे॒व। न मेद्य॒तोऽनु॑ मेद्यति। न कृश्य॒तोऽनु॑ कृश्यति। प॒ञ्च॒द॒शोऽन्यः प॒क्षो भ॑वति। स॒प्त॒द॒शोऽन्यः। तस्मा॒द्वयास्यन्यत॒रम॒र्धम॒भि प॒र्याव॑र्तन्ते। अ॒न्य॒त॒रतो॒ हि तद्गरी॑यः क्रि॒यते॥४६॥

%1.2.6.4
प॒ञ्च॒वि॒श आ॒त्मा भ॑वति। तस्मान्मध्य॒तः प॒शवो॒ वरि॑ष्ठाः। ए॒क॒वि॒शं पुच्छम्। द्वि॒पदा॑सु स्तुवन्ति॒ प्रति॑ष्ठित्यै। सर्वे॑ण स॒ह स्तु॑वन्ति। सर्वे॑ण॒ ह्यात्मनाऽऽत्म॒न्वी। स॒होत्पत॑न्ति। एकै॑का॒मुच्छिषन्ति। आ॒त्मन्न् ह्यङ्गा॑नि ब॒द्धानि॑। न वा ए॒तेन॒ सर्व॒ पुरु॑षः॥४७॥

%1.2.6.5
यदि॒तइ॑तो॒ लोमा॑नि द॒तो न॒खान्। प॒रि॒माद॑ क्रियन्ते। तान्ये॒व तेन॒ प्रत्यु॑प्यन्ते। औदु॑म्बर॒स्तल्पो॑ भवति। ऊर्ग्वा अन्न॑मुदु॒म्बर॑। ऊ॒र्ज ए॒वान्नाद्य॒स्याव॑रुध्यै। यस्य॑ तल्प॒सद्य॒मन॑भिजित॒ स्यात्। स दे॒वाना॒ साम्य॑क्षे। त॒ल्प॒सद्य॑म॒भिज॑या॒नीति॒ तल्प॑मा॒रुह्योद्गा॑येत्। त॒ल्प॒सद्य॑मे॒वाभि ज॑यति॥४८॥

%1.2.6.6
यस्य॑ तल्प॒सद्य॑म॒भिजि॑त॒ स्यात्। स दे॒वाना॒ साम्य॑क्षे। त॒ल्प॒सद्यं॒ मा परा॑जे॒षीति॒ तल्प॑मा॒रुह्योद्गा॑येत्। न त॑ल्प॒सद्यं॒ परा॑जयते। प्ले॒ङ्खे शसति। महो॒ वै प्ले॒ङ्खः। मह॑स ए॒वान्नाद्य॒स्याव॑रुद्ध्यै। दे॒वा॒सु॒राः संय॑त्ता आसन्। त आ॑दि॒त्ये व्याय॑च्छन्त। तं दे॒वाः सम॑जयन्॥४९॥

%1.2.6.7
ब्रा॒ह्म॒णश्च॑ शू॒द्रश्च॑ चर्मक॒र्ते व्याय॑च्छेते। दैव्यो॒ वै वर्णो ब्राह्म॒णः। अ॒सु॒र्य॑ शू॒द्रः। इ॒मे॑ऽरात्सुरि॒मे सु॑भू॒तम॑क्र॒न्नित्य॑न्यत॒रो ब्रू॑यात्। इ॒म उ॑द्वासीका॒रिण॑ इ॒मे दु॑र्भू॒तम॑क्र॒न्नित्य॑न्यत॒रः। यदे॒वैषा सुकृ॒तं या राद्धि॑। तद॑न्यत॒रो॑ऽभि श्री॑णाति। यदे॒वैषां दुष्कृ॒तं याऽराद्धिः। तद॑न्यत॒रोऽप॑ हन्ति। ब्रा॒ह्म॒णः सं ज॑यति। अ॒मुमे॒वादि॒त्यं भ्रातृ॑व्यस्य॒ संवि॑न्दन्ते॥५०॥\anuvakamend[भ॒व॒ति॒ भ॒व॒ति॒ क्रि॒यते॒ पुरु॑षो जयत्यजयञ्जय॒त्येकं॑ च]




\prashnaend{उ॒द्ध॒न्यमा॑नं॒ नवै॒तानि॒ सन्त॑तिरेकवि॒श ए॒षोऽप्र॑तिष्ठां प्र॒जाप॑तिर्वृ॒त्तष्षट्॥६॥}{उ॒द्ध॒न्यमा॑न शो॒चिष्के॒शोऽग्ने॑ स॒पत्ना॑नतिग्रा॒ह्या॑ वैश्वदे॒वमाल॑भन्ते पञ्चा॒शत्॥५०॥}{उद्ध॒न्यमा॑न॒ संवि॑न्दन्ते॥}{हरि॑ ओम्॥}{इति श्रीकृष्णयजुर्वेदीयतैत्तिरीयब्राह्मणे प्रथमाष्टके द्वितीयः प्रपाठकः समाप्तः॥}
\clearpage
