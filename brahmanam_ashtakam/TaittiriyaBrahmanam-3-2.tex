\sect{द्वितीयः प्रश्नः}
\setcounter{anuvakam}{0}
\dnsub{तैत्तिरीयब्राह्मणे तृतीयाष्टके द्वितीयः प्रपाठकः}

%3.2.1.1
तृ॒तीय॑स्यामि॒तो दि॒वि सोम॑ आसीत्। तङ्गा॑य॒त्र्याऽह॑रत्। तस्य॑ प॒र्णम॑च्छिद्यत। तत्प॒र्णो॑ऽभवत्। तत्प॒र्णस्य॑ पर्ण॒त्वम्। ब्रह्म॒ वै प॒र्णः। यत्प॑र्णशा॒खया॑ व॒त्सान॑पाक॒रोति॑। ब्रह्म॑णै॒वैना॑न॒पाक॑रोति। गा॒य॒त्रो वै प॒र्णः। गा॒य॒त्राः प॒शव॑॥१॥

%3.2.1.2
तस्मा॒त्रीणि॑त्रीणि प॒र्णस्य॑ पला॒शानि॑। त्रि॒पदा॑ गाय॒त्री। यत्प॑र्णशा॒खया॒ गाः प्रा॒र्पय॑ति। स्वयै॒वैना॑ दे॒वत॑या॒ प्रार्प॑यति। यङ्का॒मये॑ताप॒शुः स्या॒दिति॑। अ॒प॒र्णान्तस्मै॒ शुष्काग्रा॒माह॑रेत्। अ॒प॒शुरे॒व भ॑वति। यङ्का॒मये॑त पशु॒मान्त्स्या॒दिति॑। ब॒हु॒प॒र्णान्तस्मै॑ बहुशा॒खामाह॑रेत्। प॒शु॒मन्त॑मे॒वैनं॑ करोति॥२॥

%3.2.1.3
यत्प्राची॑मा॒ हरेत्। दे॒व॒लो॒कम॒भि ज॑येत्। यदुदी॑चीं मनुष्यलो॒कम्। प्राची॒मुदी॑ची॒मा ह॑रति। उ॒भयोर्लो॒कयो॑र॒भिजि॑त्यै। इ॒षे त्वो॒र्जे त्वेत्या॑ह। इष॑मे॒वोर्जं॒ यज॑माने दधाति। वा॒यव॒ स्थेत्या॑ह। वा॒युर्वा अ॒न्तरि॑क्ष॒स्याध्य॑क्षाः। अ॒न्त॒रि॒क्ष॒दे॒व॒त्या खलु॒ वै प॒शव॑॥३॥

%3.2.1.4
वा॒यव॑ ए॒वैना॒न्परि॑ ददाति। प्र वा ए॑नाने॒तदा क॑रोति। यदाह॑। वा॒यव॒ स्थेत्यु॑पा॒यव॒ स्थेत्या॑ह। यज॑मानायै॒व प॒शूनुप॑ ह्वयते। दे॒वो व॑ सवि॒ता प्रार्प॑य॒त्वित्या॑ह॒ प्रसूत्यै। श्रेष्ठ॑तमाय॒ कर्म॑ण॒ इत्या॑ह। य॒ज्ञो हि श्रेष्ठ॑तम॒ङ्कर्म॑। तस्मा॑दे॒वमा॑ह। आप्या॑यध्वमघ्निया देवभा॒गमित्या॑ह॥४॥

%3.2.1.5
व॒त्सेभ्य॑श्च॒ वा ए॒ताः पु॒रा म॑नु॒ष्येभ्य॒श्चाप्या॑यन्त। दे॒वेभ्य॑ ए॒वैना॒ इन्द्रा॒याप्या॑ययति। ऊर्ज॑स्वती॒ पय॑स्वती॒रित्या॑ह। ऊर्ज॒ हि पय॑ स॒म्भर॑न्ति। प्र॒जाव॑तीरनमी॒वा अ॑य॒क्ष्मा इत्या॑ह॒ प्रजात्यै। मा व॑ स्ते॒न ई॑शत॒ माऽघशस॒ इत्या॑ह॒ गुप्त्यै। रु॒द्रस्य॑ हे॒तिः परि॑ वो वृण॒क्त्वित्या॑ह। रु॒द्रादे॒वैनास्त्रायते। ध्रु॒वा अ॒स्मिन्गोप॑तौ स्यात ब॒ह्वीरित्या॑ह। ध्रु॒वा ए॒वास्मि॑न्ब॒ह्वीः क॑रोति॥५॥

%3.2.1.6
यज॑मानस्य प॒शून्पा॒हीत्या॑ह। प॒शू॒नाङ्गो॑पी॒थाय॑। तन्मात्सा॒यं प॒शव॒ उप॑स॒माव॑र्तन्ते। अन॑धः सादयति। गर्भा॑णा॒न्धृत्या॒ अप्र॑पादाय। तस्मा॒द्गर्भा प्र॒जाना॒मप्र॑पादुकाः। उ॒परी॑व॒ निद॑धाति। उ॒परी॑व॒ हि सु॑व॒र्गो लो॒कः। सु॒व॒र्गस्य॑ लो॒कस्य॒ सम॑ष्ट्यै॥६॥\anuvakamend[प॒शव॑ करोति प॒शवो॑ देवभा॒गमित्या॑ह करोति॒ नव॑ च]

%3.2.2.1
दे॒वस्य॑ त्वा सवि॒तुः प्र॑स॒व इत्य॑श्वप॒र्॒शुमाद॑त्ते॒ प्रसूत्यै। अ॒श्विनोर्बा॒हुभ्या॒मित्या॑ह। अ॒श्विनौ॒ हि दे॒वाना॑मध्व॒र्यू आस्ताम्। पू॒ष्णो हस्ताभ्या॒मित्या॑ह॒ यत्यै। यो वा ओष॑धीः पर्व॒शो वेद॑। नैना॒ स हि॑नस्ति। प्र॒जाप॑ति॒र्वा ओष॑धीः पर्व॒शो वे॑द। स ए॑ना॒ न हि॑नस्ति। अ॒श्व॒प॒र्श्वा ब॒र्॒हिरच्छै॑ति। प्रा॒जा॒प॒त्यो वा अश्व॑ सयोनि॒त्वाय॑॥७॥

%3.2.2.2
ओष॑धीना॒महिसायै। य॒ज्ञस्य॑ घो॒षद॒सीत्या॑ह। यज॑मान ए॒व र॒यिन्द॑धाति। प्रत्यु॑ष्ट॒ रक्ष॒ प्रत्यु॑ष्टा॒ अरा॑तय॒ इत्या॑ह। रक्ष॑सा॒मप॑हत्यै। प्रेयम॑गाद्धि॒षणा॑ ब॒र्॒हिरच्छेत्या॑ह। वि॒द्या वै धि॒षणा। वि॒द्ययै॒वैन॒दच्छै॑ति। मनु॑ना कृ॒ता स्व॒धया॒ वित॒ष्टेत्या॑ह। मा॒न॒वी हि पर्\mbox{}शु॑ स्व॒धाकृ॑ता॥८॥

%3.2.2.3
त आव॑हन्ति क॒वय॑ पु॒रस्ता॒दित्या॑ह। शु॒श्रु॒वासो॒ वै क॒वय॑। य॒ज्ञः पु॒रस्तात्। मु॒ख॒त ए॒व य॒ज्ञमा र॑भते। अथो॒ यदे॒तदु॒क्त्वा यत॒ कुत॑श्चा॒ हर॑ति। तत्प्राच्या॑ ए॒व दि॒शो भ॑वति। दे॒वेभ्यो॒ जुष्ट॑मि॒ह ब॒र्॒हिरा॒सद॒ इत्या॑ह। ब॒र्\mbox{}हिष॒ समृ॑द्ध्यै। कर्म॒णोऽन॑पराधाय। दे॒वानां परिषू॒तम॒सीत्या॑ह॥९॥

%3.2.2.4
यद्वा इ॒दङ्किं च॑। तद्दे॒वानां परिषू॒तम्। अथो॒ यथा॒ वस्य॑से प्रति॒प्रोच्याहे॒दङ्क॑रिष्या॒मीति॑। ए॒वमे॒व तद॑ध्व॒र्युर्दे॒वेभ्य॑ प्रति॒प्रोच्य॑ ब॒र्॒हिर्दा॑ति। आ॒त्मनोऽहिसायै। याव॑तः स्त॒म्बान्प॑रिदि॒शेत्। यत्तेषा॑मुच्छि॒ष्यात्। अति॒ तद्य॒ज्ञस्य॑ रेचयेत्। एक स्त॒म्बं परि॑दिशेत्। त सर्व॑न्दायात्॥१०॥

%3.2.2.5
य॒ज्ञस्यान॑तिरेकाय। व॒र्॒षवृ॑द्धम॒सीत्या॑ह। व॒र्॒षवृ॑द्धा॒ वा ओष॑धयः। देव॑बर्\mbox{}हि॒रित्या॑ह। दे॒वेभ्य॑ ए॒वैन॑त्करोति। मा त्वा॒ऽन्वङ्मा ति॒र्यगित्या॒हाहिसायै। पर्व॑ ते राध्यास॒मित्या॒हर्ध्यै। आ॒च्छे॒त्ता ते॒ मा रि॑ष॒मित्या॑ह। नास्या॒त्मनो॑ मीयते। य ए॒वं वेद॑॥११॥

%3.2.2.6
देव॑बर्\mbox{}हिः श॒तव॑ल्\mbox{}शं॒ विरो॒हेत्या॑ह। प्र॒जा वै ब॒र्॒हिः। प्र॒जानां प्र॒जन॑नाय। स॒हस्र॑वल्‌शा॒ वि व॒य रु॑हे॒मेत्या॑ह। आ॒शिष॑मे॒वैतामा शास्ते। पृ॒थि॒व्याः सं॒पृच॑ पा॒हीत्या॑ह॒ प्रति॑ष्ठित्यै। अयु॑ङ्गायुङ्गान्मु॒ष्टील्लुँ॑नोति। मि॒थु॒न॒त्वाय॒ प्रजात्यै। सु॒स॒म्भृता त्वा॒ सम्भ॑रा॒मीत्या॑ह। ब्रह्म॑णै॒वैन॒त्सम्भ॑रति॥१२॥

%3.2.2.7
अदि॑त्यै॒ रास्ना॒ऽसीत्या॑ह। इ॒यं वा अदि॑तिः। अ॒स्या ए॒वैन॒द्रास्नां करोति। इ॒न्द्रा॒ण्यै स॒न्नह॑न॒मित्या॑ह। इ॒न्द्रा॒णी वा अग्रे॑ दे॒वता॑ना॒ सम॑नह्यत। साऽऽर्ध्नोत्। ऋद्ध्यै॒ सन्न॑ह्यति। प्र॒जा वै ब॒र्॒हिः। प्र॒जाना॒मप॑रावापाय। तस्मा॒त्स्नाव॑सन्तताः प्र॒जा जा॑यन्ते॥१३॥

%3.2.2.8
पू॒षा ते ग्र॒न्थिङ्ग्र॑थ्ना॒त्वित्या॑ह। पुष्टि॑मे॒व यज॑माने दधाति। स ते॒ मास्था॒दित्या॒हाहिसायै। प॒श्चात्प्राञ्च॒मुप॑गूहति। प॒श्चाद्वै प्रा॒चीन॒ रेतो॑ धीयते। प॒श्चादे॒वास्मै प्रा॒चीन॒ रेतो॑ दधाति। इन्द्र॑स्य त्वा बा॒हुभ्या॒मुद्य॑च्छ॒ इत्या॑ह। इ॒न्द्रि॒यमे॒व यज॑माने दधाति। बृह॒स्पतेर्मू॒र्ध्ना ह॑रा॒मीत्या॑ह। ब्रह्म॒ वै दे॒वानां॒ बृह॒स्पति॑॥१४॥

%3.2.2.9
ब्रह्म॑णै॒वैन॑द्धरति। उ॒र्व॑न्तरि॑क्ष॒मन्वि॒हीत्या॑ह॒ गत्यै। दे॒व॒ङ्ग॒मम॒सीत्या॑ह। दे॒वाने॒वैन॑द्गमयति। अन॑धः सादयति। गर्भा॑णा॒न्धृत्या॒ अप्र॑पादाय। तस्मा॒द्गर्भा प्र॒जाना॒मप्र॑पादुकाः। उ॒परी॑व॒ नि द॑धाति। उ॒परी॑व॒ हि सु॑व॒र्गो लो॒कः। सु॒व॒र्गस्य॑ लो॒कस्य॒ सम॑ष्ट्यै॥१५॥\anuvakamend[स॒यो॒नि॒त्वाय॑ स्व॒धाकृ॑ता॒ऽसीत्या॑ह दाया॒द्वेद॑ भरति जायन्ते॒ बृह॒स्पति॒ सम॑ष्ट्यै]

%3.2.3.1
पू॒र्वे॒द्युरि॒ध्माब॒र्॒हिः क॑रोति। य॒ज्ञमे॒वारभ्य॑ गृही॒त्वोप॑वसति। प्र॒जाप॑तिर्य॒ज्ञम॑सृजत। तस्यो॒खे अ॑स्रसेताम्। य॒ज्ञो वै प्र॒जाप॑तिः। यत्सान्नाय्यो॒खे भव॑तः। य॒ज्ञस्यै॒व तदु॒खे उप॑दधा॒त्यप्र॑स्रसाय। शुन्ध॑ध्व॒न्दैव्या॑य॒ कर्म॑णे देवय॒ज्याया॒ इत्या॑ह। दे॒व॒य॒ज्याया॑ ए॒वैना॑नि शुन्धति। मा॒त॒रिश्व॑नो घ॒र्मो॑ऽसीत्या॑ह॥१६॥

%3.2.3.2
अ॒न्तरि॑क्षं॒ वै मा॑त॒रिश्व॑नो घ॒र्मः। ए॒षां लो॒कानां॒ विधृ॑त्यै। द्यौर॑सि पृथि॒व्य॑सीत्या॑ह। दि॒वश्च॒ ह्ये॑षा पृ॑थि॒व्याश्च॒ सम्भृ॑ता। यदु॒खा। तस्मा॑दे॒वमा॑ह। वि॒श्वधा॑या असि पर॒मेण॒ धाम्नेत्या॑ह। वृष्टि॒र्वै वि॒श्वधा॑याः। वृष्टि॑मे॒वाव॑रुन्धे। दृह॑स्व॒ मा ह्वा॒रित्या॑ह॒ धृत्यै॥१७॥

%3.2.3.3
वसू॑नां प॒वित्र॑म॒सीत्या॑ह। प्रा॒णा वै वस॑वः। तेषां॒ वा ए॒तद्भा॑ग॒धेयम्। यत्प॒वित्रम्। तेभ्य॑ ए॒वैन॑त्करोति। श॒तधा॑र स॒हस्र॑धार॒मित्या॑ह। प्रा॒णेष्वे॒वायु॑र्दधाति सर्व॒त्वाय॑। त्रि॒वृत्प॑लाशशा॒खायान्दर्भ॒मयं॑ भवति। त्रि॒वृद्वै प्रा॒णः। त्रि॒वृत॑मे॒व प्रा॒णं म॑ध्य॒तो यज॑माने दधाति॥१८॥

%3.2.3.4
सौ॒म्यः प॒र्णः स॑योनि॒त्वाय॑। सा॒क्षात्प॒वित्र॑न्द॒र्भाः। प्राख्सा॒यमधि॒नि द॑धाति। तत्प्रा॑णापा॒नयो॑ रू॒पम्। ति॒र्यक्प्रा॒तः। तद्दर्श॑स्य रू॒पम्। दा॒र्श्य ह्ये॑तदह॑। अन्नं॒ वै च॒न्द्रमा। अन्नं॑ प्रा॒णाः। उ॒भय॑मे॒वोपै॒त्यजा॑मित्वाय॥१९॥

%3.2.3.5
तस्मा॑द॒य स॒र्वत॑ पवते। हु॒तः स्तो॒को हु॒तो द्र॒प्स इत्या॑ह॒ प्रति॑ष्ठित्यै। ह॒विषोऽस्क॑न्दाय। न हि हु॒त स्वाहा॑कृत॒ स्कन्द॑ति। दि॒वि नाको॒ नामा॒ग्निः। तस्य॑ वि॒प्रुषो॑ भाग॒धेयम्। अ॒ग्नये॑ बृह॒ते नाका॒येत्या॑ह। नाक॑मे॒वाग्निं भा॑ग॒धेये॑न॒ सम॑र्धयति। स्वाहा॒ द्यावा॑पृथि॒वीभ्या॒मित्या॑ह। द्यावा॑पृथि॒व्योरे॒वैन॒त्प्रति॑ष्ठापयति॥२०॥

%3.2.3.6
प॒वित्र॑व॒त्यान॑यति। अ॒पाञ्चै॒वौष॑धीनां च॒ रस॒ ससृ॑जति। अथो॒ ओष॑धीष्वे॒व प॒शून्प्रति॑ष्ठापयति। अ॒न्वा॒रभ्य॒ वाचं॑ यच्छति। य॒ज्ञस्य॒ धृत्यै। धा॒रय॑न्नास्ते। धा॒रय॑न्त इव॒ हि दु॒हन्ति॑। काम॑धुक्ष॒ इत्या॒हातृ॒तीय॑स्यै। त्रय॑ इ॒मे लो॒काः। इ॒माने॒व लो॒कान्‌यज॑मानो दुहे॥२१॥

%3.2.3.7
अ॒मूमिति॒ नाम॑ गृह्णाति। भ॒द्रमे॒वासा॒ङ्कर्मा॒ विष्क॑रोति। सा वि॒श्वायु॒ सा वि॒श्वव्य॑चा॒ सा वि॒श्वक॒र्मेत्या॑ह। इ॒यं वै वि॒श्वायु॑। अ॒न्तरि॑क्षं वि॒श्वव्य॑चाः। अ॒सौ वि॒श्वक॑र्मा। इ॒माने॒वैताभि॑र्लो॒कान्‌ य॑थापू॒र्वन्दु॑हे। अथो॒ यथा प्रदा॒त्रे पुण्य॑मा॒शास्ते। ए॒वमे॒वैना॑ ए॒तदुप॑स्तौति। तस्मा॒त्प्रादा॒दित्यु॒न्नीय॒ वन्द॑माना उपस्तु॒वन्त॑ प॒शून्दु॑हन्ति॥२२॥

%3.2.3.8
ब॒हु दु॒ग्धीन्द्रा॑य दे॒वेभ्यो॑ ह॒विरिति॒ वाचं॒ विसृ॑जते। य॒था॒दे॒व॒तमे॒व प्रसौ॑ति। दैव्य॑स्य च मानु॒षस्य॑ च॒ व्यावृ॑त्यै। त्रिरा॑ह। त्रिष॑त्या॒ हि दे॒वाः। अवा॑चं य॒मोऽन॑न्वार॒भ्योत्त॑राः। अप॑रिमितमे॒वाव॑ रुन्धे। न दा॑रुपा॒त्रेण॑ दुह्यात्। अ॒ग्नि॒वद्वै दा॑रुपा॒त्रम्। यद्दा॑रुपा॒त्रेण॑ दु॒ह्यात्॥२३॥

%3.2.3.9
या॒तयाम्ना ह॒विषा॑ यजेत। अथो॒ खल्वा॑हुः। पु॒रो॒डाश॑मुखानि॒ वै ह॒वीषि॑। नेत इ॑तः पुरो॒डाश ह॒विषो॒ यामो॒ऽस्तीति॑। काम॑मे॒व दा॑रुपा॒त्रेण॑ दुह्यात्। शू॒द्र ए॒व न दु॑ह्यात्। अस॑तो॒ वा ए॒ष सम्भू॑तः। यच्छू॒द्रः। अह॑विरे॒व तदित्या॑हुः। यच्छू॒द्रो दोग्धीति॑॥२४॥

%3.2.3.10
अ॒ग्नि॒हो॒त्रमे॒व न दु॑ह्याच्छू॒द्रः। तद्धि नोत्पु॒नन्ति॑। य॒दा खलु॒ वै प॒वित्र॑म॒त्येति॑। अथ॒ तद्ध॒विरिति॑। संपृ॑च्यध्वमृतावरी॒रित्या॑ह। अ॒पाञ्चै॒वौष॑धीनां च॒ रस॒ स सृ॑जति। तस्मा॑द॒पाञ्चौष॑धीनां च॒ रस॒मुप॑जीवामः। म॒न्द्रा धन॑स्य सा॒तय॒ इत्या॑ह। पुष्टि॑मे॒व यज॑माने दधाति। सोमे॑न॒ त्वात॑न॒च्मीन्द्रा॑य॒ दधीत्या॑ह॥२५॥

%3.2.3.11
सोम॑मे॒वैन॑त्करोति। यो वै सोमं॑ भक्षयि॒त्वा। सं॒व॒त्स॒र सोम॒न्न पिब॑ति। पु॒न॒र्भक्ष्योऽस्य सोमपी॒थो भ॑वति। सोम॒ खलु॒ वै सान्ना॒य्यम्। य ए॒वं वि॒द्वान्त्सान्ना॒य्यं पिब॑ति। अ॒पु॒न॒र्भक्ष्योऽस्य सामपी॒थो भ॑वति। न मृ॒न्मये॒नापि॑ दध्यात्। यन्मृ॒न्मयो॑नापिद॒ध्यात्। पि॒तृ॒दे॒व॒त्य स्यात्॥२६॥

%3.2.3.12
अ॒य॒स्पा॒त्रेण॑ वा दारुपा॒त्रेण॒ वाऽपि॑ दधाति। तद्धि सदे॑वम्। उ॒द॒न्वद्भ॑वति। आपो॒ वै र॑क्षो॒घ्नीः। रक्ष॑सा॒मप॑हत्यै। अद॑स्तमसि॒ विष्ण॑वे॒ त्वे॒त्या॑ह। य॒ज्ञो वै विष्णु॑। य॒ज्ञायै॒वैन॒दद॑स्तं करोति। विष्णो॑ ह॒व्य र॑क्ष॒स्वेत्या॑ह॒ गुप्त्यै। अन॑धः सादयति। गर्भा॑णा॒न्धृत्या॒ अप्र॑पादाय। तस्मा॒द्गर्भा प्र॒जाना॒मप्र॑पादुकाः। उ॒परी॑व॒ निद॑धाति। उ॒परी॑व॒ हि सु॑व॒र्गो लो॒कः। सु॒व॒र्गस्य॑ लो॒कस्य॒ सम॑ष्ट्यै॥२७॥\anuvakamend[अ॒सीत्या॑ह॒ धृत्यै॒ यज॑माने दधा॒त्यजा॑मित्वाय स्थापयति दुहे दुहन्ति दु॒ह्याद्दोग्धीति॒ दधीत्या॑ह स्यात्सादयति॒ पञ्च॑ च]

%3.2.4.1
कर्म॑णे वान्दे॒वेभ्य॑ शकेय॒मित्या॑ह॒ शक्त्यै। य॒ज्ञस्य॒ वै सन्त॑ति॒मनु॑ प्र॒जाः प॒शवो॒ यज॑मानस्य॒ सन्ता॑यन्ते। य॒ज्ञस्य॒ विच्छि॑त्ति॒मनु॑ प्र॒जाः प॒शवो॒ यज॑मानस्य॒ विच्छि॑द्यन्ते। य॒ज्ञस्य॒ सन्त॑तिरसि य॒ज्ञस्य॑ त्वा॒ सन्त॑त्यै स्तृणामि॒ सन्त॑त्यै त्वा य॒ज्ञस्येत्याह॑व॒नीया॒त्सन्त॑नोति। यज॑मानस्य प्र॒जायै॑ पशू॒ना सन्त॑त्यै। अ॒पः प्रण॑यति। श्र॒द्धा वा आप॑। श्र॒द्धामे॒वारभ्य॑ प्र॒णीय॒ प्रच॑रति। अ॒पः प्रण॑यति। य॒ज्ञो वा आप॑॥२८॥

%3.2.4.2
य॒ज्ञमे॒वारभ्य॑ प्र॒णीय॒ प्रच॑रति। अ॒पः प्रण॑यति। वज्रो॒ वा आप॑। वज्र॑मे॒व भ्रातृ॑व्येभ्यः प्र॒हृत्य॑ प्र॒णीय॒ प्रच॑रति। अ॒पः प्रण॑यति। आपो॒ वै र॑क्षो॒घ्नीः। रक्ष॑सा॒मप॑हत्यै। अ॒पः प्रण॑यति। आपो॒ वै दे॒वानां प्रि॒यन्धाम॑। दे॒वाना॑मे॒व प्रि॒यन्धाम॑ प्र॒णीय॒ प्रच॑रति॥२९॥

%3.2.4.3
अ॒पः प्रण॑यति। आपो॒ वै सर्वा॑ दे॒वता। दे॒वता॑ ए॒वारभ्य॑ प्र॒णीय॒ प्रच॑रति। वेषा॑य॒ त्वेत्या॑ह। वेषा॑य॒ ह्ये॑नदाद॒त्ते। प्रत्यु॑ष्ट॒ रक्ष॒ प्रत्यु॑ष्टा॒ अरा॑तय॒ इत्या॑ह। रक्ष॑सा॒मप॑हत्यै। धूर॒सीत्या॑ह। ए॒ष वै धुर्यो॒ऽग्निः। तं यदनु॑पस्पृश्याती॒यात्॥३०॥

%3.2.4.4
अ॒ध्व॒र्युं च॒ यज॑मानं च॒ प्रद॑हेत्। उ॒प॒स्पृश्यात्ये॑ति। अ॒ध्व॒र्योश्च॒ यज॑मानस्य॒ चाप्र॑दाहाय। धूर्व॒ तय्योँस्मान्धूर्व॑ति॒ तन्धूर्व॒ यं व॒यन्धूर्वा॑म॒ इत्या॑ह। द्वौ वाव पुरु॑षौ। यञ्चै॒व धूर्व॑ति। यश़्चै॑न॒न्धूर्व॑ति। तावु॒भौ शु॒चाऽर्प॑यति। त्वन्दे॒वाना॑मसि॒ सस्नि॑तमं॒ पप्रि॑तमं॒ जुष्ट॑तमं॒ वह्नि॑तमन्देव॒हूत॑म॒मित्या॑ह। य॒था॒य॒जुरे॒वैतत्॥३१॥

%3.2.4.5
अह्रु॑तमसि हवि॒र्धान॒मित्या॒हानार्त्यै। दृह॑स्व॒ मा ह्वा॒रित्या॑ह॒ धृत्यै। मि॒त्रस्य॑ त्वा॒ चक्षु॑षा॒ प्रेक्ष॒ इत्या॑ह मित्र॒त्वाय॑। मा भेर्मा संवि॑क्था॒ मा त्वा॑ हिसिष॒मित्या॒हाहिसायै। यद्वै किं च॒ वातो॒ नाभि॒ वाति॑। तत्सर्वं॑ वरुणदेव॒त्यम्। उ॒रु वाता॒येत्या॑ह। अवा॑रुणमे॒वैन॑त्करोति। दे॒वस्य॑ त्वा सवि॒तुः प्र॑स॒व इत्या॑ह॒ प्रसूत्यै। अ॒श्विनोर्बा॒हुभ्या॒मित्या॑ह॥३२॥

%3.2.4.6
अ॒श्विनौ॒ हि दे॒वाना॑मध्व॒र्यू आस्ताम्। पू॒ष्णो हस्ताभ्या॒मित्या॑ह॒ यत्यै। अ॒ग्नये॒ जुष्टं॒ निर्व॑पा॒मीत्या॑ह। अ॒ग्नय॑ ए॒वैनां॒ जुष्टं॒ निर्व॑पति। त्रिर्यजु॑षा। त्रय॑ इ॒मे लो॒काः। ए॒षां लो॒काना॒माप्त्यै। तू॒ष्णीञ्च॑तु॒र्थम्। अप॑रिमितमे॒वाव॑रुन्धे। स ए॒वमे॒वानु॑पू॒र्व ह॒वीषि॒ निर्व॑पति॥३३॥

%3.2.4.7
इ॒दन्दे॒वाना॑मि॒दमु॑ नः स॒हेत्या॑ह॒ व्यावृ॑त्यै। स्फा॒त्यै त्वा॒ नारात्या॒ इत्या॑ह॒ गुप्त्यै। तम॑सीव॒ वा ए॒षोऽन्तश्च॑रति। यः प॑री॒णहि॑। सुव॑र॒भि वि ख्ये॑षं वैश्वान॒रञ्ज्योति॒रित्या॑ह। सुव॑रे॒वाभि वि प॑श्यति वैश्वान॒रञ्ज्योति॑। द्यावा॑पृथि॒वी ह॒विषि॑ गृही॒त उद॑वेपेताम्। दृह॑न्ता॒न्दुर्या॒ द्यावा॑पृथि॒व्योरित्या॑ह। गृ॒हाणा॒न्द्यावा॑पृथि॒व्योर्धृत्यै। उ॒र्व॑न्तरि॑क्ष॒मन्वि॒हीत्या॑ह॒ गत्यै। अदि॑त्यास्त्वो॒पस्थे॑ सादया॒मीत्या॑ह। इ॒यं वा अदि॑तिः। अ॒स्या ए॒वैन॑दु॒पस्थे॑ सादयति। अग्ने॑ ह॒व्य र॑क्ष॒स्वेत्या॑ह॒ गुप्त्यै॥३४॥\anuvakamend[य॒ज्ञो वा आपो॒ धाम॑ प्र॒णीय॒ प्रच॑रत्यती॒यादे॒तद्बा॒हुभ्या॒मित्या॑ह ह॒वीषि॒ निर्व॑पति॒ गत्यै॑ च॒त्वारि॑ च]

%3.2.5.1
इन्द्रो॑ वृ॒त्रम॑हन्। सो॑ऽपः। अ॒भ्य॑म्रियत। तासां॒ यन्मेध्यं॑ य॒ज्ञिय॒ सदे॑व॒मासीत्। तदपोद॑क्रामत्। ते द॒र्भा अ॑भवन्। यद्द॒र्भैर॒प उ॑त्पु॒नाति॑। या ए॒व मेध्या॑ य॒ज्ञिया॒ सदे॑वा॒ आप॑। ताभि॑रे॒वैना॒ उत्पु॑नाति। द्वाभ्या॒मुत्पु॑नाति॥३५॥

%3.2.5.2
द्वि॒पाद्यज॑मान॒ प्रति॑ष्ठित्यै। दे॒वो व॑ सवि॒तोत्पु॑ना॒त्वित्या॑ह। स॒वि॒तृप्र॑सूत ए॒वैना॒ उत्पु॑नाति। अच्छि॑द्रेण प॒वित्रे॒णेत्या॑ह। अ॒सौ वा आ॑दि॒त्योऽच्छि॑द्रं प॒वित्रम्। तेनै॒वैना॒ उत्पु॑नाति। वसो॒ सूर्य॑स्य र॒श्मिभि॒रित्या॑ह। प्रा॒णा वा आप॑। प्रा॒णा वस॑वः। प्रा॒णा र॒श्मय॑॥३६॥

%3.2.5.3
प्रा॒णैरे॒व प्रा॒णान्त्सं पृ॑णक्ति। सा॒वि॒त्रि॒यर्चा। स॒वि॒तृप्र॑सूतं मे॒ कर्मा॑स॒दिति॑। स॒वि॒तृप्र॑सूतमे॒वास्य॒ कर्म॑ भवति। प॒च्छो गा॑यत्रि॒या त्रि॑ष्षमृद्ध॒त्वाय॑। आपो॑ देवीरग्रेपुवो अग्रेगुव॒ इत्या॑ह। रू॒पमे॒वासा॑मे॒तन्म॑हि॒मानं॒ व्याच॑ष्टे। अग्र॑ इ॒मं य॒ज्ञं न॑य॒ताग्रे॑ य॒ज्ञप॑ति॒मित्या॑ह। अग्र॑ ए॒व य॒ज्ञं न॑यन्ति। अग्रे॑ य॒ज्ञप॑तिम्॥३७॥

%3.2.5.4
यु॒ष्मानिन्द्रो॑ऽवृणीत वृत्र॒तूर्ये॑ यू॒यमिन्द्र॑मवृणीध्वं वृत्र॒तूर्य॒ इत्या॑ह। वृ॒त्र ह॑ हनि॒ष्यन्निन्द्र॒ आपो॑ वव्रे। आपो॒ हेन्द्रं॑ वव्रिरे। सं॒ज्ञामे॒वासा॑मे॒तत्सामा॑नं॒ व्याच॑ष्टे। प्रोक्षि॑ता॒ स्थेत्या॑ह। तेनाप॒ प्रोक्षि॑ताः। अ॒ग्नये॑ वो॒ जुष्टं॒ प्रोक्षाम्य॒ग्नीषोमाभ्या॒मित्या॑ह। य॒था॒दे॒व॒तमे॒वैना॒न्प्रोक्ष॑ति। त्रिः प्रोक्ष॑ति। त्र्या॑वृ॒द्धि य॒ज्ञः॥३८॥

%3.2.5.5
अथो॒ रक्ष॑सा॒मप॑हत्यै। शुन्ध॑ध्वं॒ दैव्या॑य॒ कर्म॑णे देवय॒ज्याया॒ इत्या॑ह। दे॒व॒य॒ज्याया॑ ए॒वैना॑नि शुन्धति। त्रिः प्रोक्ष॑ति। त्र्या॑वृ॒द्धि य॒ज्ञः। अथो॑ मेध्य॒त्वाय॑। अव॑धूत॒ रक्षोऽव॑धूता॒ आरा॑तय॒ इत्या॑ह। रक्ष॑सा॒मप॑हत्यै। अदि॑त्या॒स्त्वग॒सीत्या॑ह। इ॒यं वा अदि॑तिः॥३९॥

%3.2.5.6
अ॒स्या ए॒वैन॒त्त्वचं॑ करोति। प्रति॑ त्वा पृथि॒वी वे॒त्त्वित्या॑ह॒ प्रति॑ष्ठित्यै। पु॒रस्तात्प्रती॒चीन॑ग्रीव॒मुत्त॑रलो॒मोप॑स्तृणाति मेध्य॒त्वाय॑। तस्मात्पु॒रस्तात्प्र॒त्यञ्च॑ प॒शवो॒ मेध॒मुप॑तिष्ठन्ते। तस्मात्प्र॒जा मृ॒गं ग्राहु॑काः। य॒ज्ञो दे॒वेभ्यो॒ निला॑यत। कृष्णो॑ रू॒पं कृ॒त्वा। यत्कृ॑ष्णाजि॒ने ह॒विर॑ध्यव॒हन्ति॑। य॒ज्ञादे॒व तद्य॒ज्ञं प्रयु॑ङ्क्ते। ह॒विषोऽस्क॑न्दाय॥४०॥

%3.2.5.7
अ॒धि॒षव॑णमसि वानस्प॒त्यमित्या॑ह। अ॒धि॒षव॑णमे॒वैन॑त्करोति। प्रति॒ त्वाऽदि॑त्या॒स्त्वग्वे॒त्त्वित्या॑ह सय॒त्वाय॑। अ॒ग्नेस्त॒नूर॒सीत्या॑ह। अ॒ग्नेर्वा ए॒षा त॒नूः। यदोष॑धयः। वा॒चो वि॒सर्ज॑न॒मित्या॑ह। य॒दा हि प्र॒जा ओष॑धीनाम॒श्ञन्ति॑। अथ॒ वाचं॒ विसृ॑जन्ते। दे॒ववी॑तये त्वा गृह्णा॒मीत्या॑ह॥४१॥

%3.2.5.8
दे॒वता॑भिरे॒वैन॒त्सम॑र्धयति। अद्रि॑रसि वानस्प॒त्य इत्या॑ह। ग्रावा॑णमे॒वैन॑त्करोति। स इ॒दं दे॒वेभ्यो॑ ह॒व्य सु॒शमि॑ शमि॒ष्वेत्या॑ह॒ शान्त्यै। हवि॑ष्कृ॒देहीत्या॑ह। य ए॒व दे॒वाना हवि॒ष्कृत॑। तान्‌ ह्व॑यति। त्रिर्ह्व॑यति। त्रिष॑त्या॒ हि दे॒वाः। इष॒माव॒दोर्ज॒माव॒देत्या॑ह॥४२॥

%3.2.5.9
इष॑मे॒वोर्जं॒ यज॑माने दधाति। द्यु॒मद्व॑दत व॒य सं॑घा॒तं जे॒ष्मेत्या॑ह॒ भ्रातृ॑व्याभिभूत्यै। मनो श्र॒द्धादे॑वस्य॒ यज॑मानस्यासुर॒घ्नी वाक्। य॒ज्ञा॒यु॒धेषु॒ प्रवि॑ष्टाऽऽसीत्। तेऽसु॑रा॒ याव॑न्तो यज्ञायु॒धाना॑मु॒द्वद॑तामु॒पाशृ॑ण्वन्। ते परा॑भवन्। तस्मा॒त्स्वानां॒ मध्ये॑ऽव॒साय॑ यजेत। याव॑न्तोऽस्य॒ भ्रातृ॑व्या यज्ञायु॒धाना॑मु॒द्वद॑तामुपशृ॒ण्वन्ति॑। ते परा॑ भवन्ति। उ॒च्चैः स॒माह॑न्त॒ वा आ॑ह॒ विजि॑त्यै॥४३॥

%3.2.5.10
वृ॒ङ्क्त ए॑षामिन्द्रि॒यं वी॒र्यम्। श्रेष्ठ॑ एषां भवति। व॒र्\mbox{}॒षवृ॑द्धमसि॒ प्रति॑ त्वा व॒र्\mbox{}॒षवृ॑द्धं वे॒त्त्वित्या॑ह। व॒र्\mbox{}॒षवृ॑द्धा॒ वा ओष॑धयः। व॒र्\mbox{}॒षवृ॑द्धा इ॒षीका॒ समृ॑द्ध्यै। य॒ज्ञ रक्षा॒स्यनु॒ प्रावि॑शन्। तान्य॒स्ना प॒शुभ्यो॑ नि॒रवा॑दयन्त। तुषै॒रोष॑धीभ्यः। परा॑पूत॒ रक्ष॒ परा॑पूता॒ अरा॑तय॒ इत्या॑ह। रक्ष॑सा॒मप॑हत्यै॥४४॥

%3.2.5.11
रक्ष॑सां भा॒गो॑ऽसीत्या॑ह। तुषै॑रे॒व रक्षासि नि॒रव॑दयते। अ॒प उप॑स्पृशति मेध्य॒त्वाय॑। वा॒युर्वो॒ विवि॑न॒क्त्वित्या॑ह। प॒वित्रं॒ वै वा॒युः। पु॒नात्ये॒वैनान्॑। अ॒न्तरि॑क्षादिव॒ वा ए॒ते प्रस्क॑न्दन्ति। ये शूर्पात्। दे॒वो व॑ सवि॒ता हिर॑ण्यपाणि॒ प्रति॑गृह्णा॒त्वित्या॑ह॒ प्रति॑ष्ठित्यै। ह॒विषोऽस्क॑न्दाय। त्रिष्फ॒लीक॑र्त॒वा आ॑ह। त्र्या॑वृ॒द्धि य॒ज्ञः। अथो॑ मेध्य॒त्वाय॑॥४५॥\anuvakamend[द्वाभ्या॒मुत्पु॑नाति र॒श्मयो॑ नय॒न्त्यग्रे॑ य॒ज्ञप॑तिं य॒ज्ञोऽदि॑ति॒रस्क॑न्दाय गृह्णा॒मीत्या॑ह व॒देत्या॑ह॒ विजि॑त्या॒ अप॑हत्या॒ अस्क॑न्दाय॒ त्रीणि॑ च]

%3.2.6.1
अव॑धूत॒ रक्षोऽव॑धूता॒ अरा॑तय॒ इत्या॑ह। रक्ष॑सा॒मप॑हत्यै। अदि॑त्या॒स्त्वग॒सीत्या॑ह। इ॒यं वा अदि॑तिः। अ॒स्या ए॒वैन॒त्त्वचं॑ करोति। प्रति॑ त्वा पृथि॒वी वे॒त्त्वित्या॑ह॒ प्रति॑ष्ठित्यै। पु॒रस्तात्प्रती॒चीन॑ग्रीव॒मुत्त॑रलो॒मोप॑स्तृणाति मेध्य॒त्वाय॑। तस्मात्पु॒रस्तात्प्र॒त्यञ्च॑ प॒शवो॒ मेध॒मुप॑तिष्ठन्ते। तस्मात्प्र॒जा मृ॒गं ग्राहु॑काः। य॒ज्ञो दे॒वेभ्यो॒ निला॑यत॥४६॥

%3.2.6.2
कृष्णो॑ रू॒पं कृ॒त्वा। यत्कृ॑ष्णाजि॒ने ह॒विर॑धिपि॒नष्टि॑। य॒ज्ञादे॒व तद्य॒ज्ञं प्रयु॑ङ्क्ते। ह॒विषोऽस्क॑न्दाय। द्यावा॑पृथि॒वी स॒हास्ताम्। ते श॑म्यामा॒त्रमेक॒मह॒र्व्यैता शम्यामा॒त्रमेक॒मह॑। दि॒वः स्क॑म्भ॒निर॑सि॒ प्रति॒ त्वाऽदि॑त्या॒स्त्वग्वे॒त्त्वित्या॑ह। द्यावा॑पृथि॒व्योर्वीत्यै। धि॒षणा॑ऽसि पर्व॒त्या प्रति॑ त्वा दि॒वः स्क॑म्भ॒निर्वे॒त्त्वित्या॑ह। द्यावा॑पृथि॒व्योर्विधृ॑त्यै॥४७॥

%3.2.6.3
धि॒षणा॑ऽसि पार्वते॒यी प्रति॑ त्वा पर्व॒तिर्वे॒त्त्वित्या॑ह। द्यावा॑पृथि॒व्योर्धृत्यै। दे॒वस्य॑ त्वा सवि॒तुः प्र॑स॒व इत्या॑ह॒ प्रसूत्यै। अ॒श्विनोर्बा॒हुभ्या॒मित्या॑ह। अ॒श्वि॒नौ हि दे॒वाना॑मध्व॒र्यू आस्ताम्। पू॒ष्णो हस्ताभ्या॒मित्या॑ह॒ यत्त्ये। अधि॑वपा॒मीत्या॑ह। य॒था॒दे॒व॒तमे॒वैना॒नधि॑ वपति। धा॒न्य॑मसि धिनु॒हि दे॒वानित्या॑ह। ए॒तस्य॒ यजु॑षो वी॒र्ये॑ण॥४८॥

%3.2.6.4
याव॒देका॑ दे॒वता॑ का॒मय॑ते॒ याव॒देका। ताव॒दाहु॑तिः प्रथते। न हि तदस्ति॑। यत्ताव॑दे॒व स्यात्। याव॑ज्जु॒होति॑। प्रा॒णाय॑ त्वाऽपा॒नाय॒ त्वेत्या॑ह। प्रा॒णाने॒व यज॑माने दधाति। दी॒र्घामनु॒ प्रसि॑ति॒मायु॑षे धा॒मित्या॑ह। आयु॑रे॒वास्मि॑न्दधाति। अ॒न्तरि॑क्षादिव॒ वा ए॒तानि॒ प्रस्क॑न्दन्ति। यानि॑ दृ॒षद॑। दे॒वो व॑ सवि॒ता हिर॑ण्यपाणि॒ प्रति॑गृह्णा॒त्वित्या॑ह॒ प्रति॑ष्ठित्यै। ह॒विषोऽस्क॑न्दाय। असं॑वपन्ती पिषा॒णूनि॑ कुरुता॒दित्या॑ह मेध्य॒त्वाय॑॥४९॥\anuvakamend[निला॑यत॒ विधृ॑त्यै वी॒र्ये॑ण स्कन्दन्ति च॒त्वारि॑ च]

%3.2.7.1
धृष्टि॑रसि॒ ब्रह्म॑ य॒च्छेत्या॑ह॒ धृत्यै। अपाग्ने॒ऽग्निमा॒मादं॑ जहि॒ निष्क्र॒व्याद से॒धा दे॑व॒यजं॑ व॒हेत्या॑ह। य ए॒वामात्क्र॒व्यात्। तम॑प॒हत्य॑। मेध्ये॒ऽग्नौ क॒पाल॒मुप॑दधाति। निर्द॑ग्ध॒ रक्षो॒ निर्द॑ग्धा॒ अरा॑तय॒ इत्या॑ह। रक्षास्ये॒व निर्द॑हति। अ॒ग्नि॒वत्युप॑दधाति। अ॒स्मिन्ने॒व लो॒के ज्योति॑र्धत्ते। अङ्गा॑र॒मधि॑ वर्तयति॥५०॥

%3.2.7.2
अ॒न्तरि॑क्ष ए॒व ज्योति॑र्धत्ते। आ॒दि॒त्यमे॒वामुष्मि॑ल्लोँ॒के ज्योति॑र्धत्ते। ज्योति॑ष्मन्तोऽस्मा इ॒मे लो॒का भ॑वन्ति। य ए॒वं वेद॑। ध्रु॒वम॑सि पृथि॒वीं दृ॒हेत्या॑ह। पृ॒थि॒वीमे॒वैतेन॑ दृ॒हति। ध॒र्त्रम॑स्य॒न्तरि॑क्षं दृ॒हेत्या॑ह। अ॒न्तरि॑क्षमे॒वैतेन॑ दृहति। ध॒रुण॑मसि॒ दिवं॑ दृ॒हेत्या॑ह। दिव॑मे॒वैतेन॑ दृहति॥५१॥

%3.2.7.3
धर्मा॑सि॒ दिशो॑ दृ॒हेत्या॑ह। दिश॑ ए॒वैतेन॑ दृहति। इ॒माने॒वैतैर्लो॒कान्दृहति। दृह॑न्तेऽस्मा इ॒मे लो॒काः प्र॒जया॑ प॒शुभि॑। य ए॒वं वेद॑। त्रीण्यग्रे॑ क॒पाला॒न्युप॑दधाति। त्रय॑ इ॒मे लो॒काः। ए॒षां लो॒काना॒माप्त्यै। एक॒मग्रे॑ क॒पाल॒मुप॑ दधाति। एकं॒ वा अग्रे॑ क॒पालं॒ पुरु॑षस्य स॒म्भव॑ति॥५२॥

%3.2.7.4
अथ॒ द्वे। अथ॒ त्रीणि॑। अथ॑ च॒त्वारि॑। अथा॒ष्टौ। तस्मा॑द॒ष्टाक॑पालं॒ पुरु॑षस्य॒ शिर॑। यदे॒वं क॒पालान्युप॒दधा॑ति। य॒ज्ञो वै प्र॒जाप॑तिः। य॒ज्ञमे॒व प्र॒जाप॑ति॒ सस्क॑रोति। आ॒त्मान॑मे॒व तत्सस्क॑रोति। त सस्कृ॑तमा॒त्मानम्॥५३॥

%3.2.7.5
अ॒मुष्मि॑ल्लोँ॒केऽनु॒ परै॑ति। यद॒ष्टावु॑प॒दधा॑ति। गा॒य॒त्रि॒या तत्सम्मि॑तम्। यन्नव॑। त्रि॒वृता॒ तत्। यद्दश॑। वि॒राजा॒ तत्। यदेका॑दश। त्रि॒ष्टुभा॒ तत्। यद्द्वाद॑श॥५४॥

%3.2.7.6
जग॑त्या॒ तत्। छन्द॑ सम्मितानि॒ स उ॑प॒दध॑त्क॒पाला॑नि। इ॒माल्लोँ॒कान॑नुपू॒र्वन्दिशो॒ विधृ॑त्यै दृहति। अथायु॑ प्रा॒णान्प्र॒जां प॒शून् यज॑माने दधाति। स॒जा॒तान॑स्मा अ॒भितो॑ बहु॒लान्क॑रोति। चित॒ स्थेत्या॑ह। य॒था॒य॒जुरे॒वैतत्। भृगू॑णा॒मङ्गि॑रसा॒न्तप॑सा तप्यध्व॒मित्या॑ह। दे॒वता॑नामे॒वैना॑नि॒ तप॑सा तपति। तानि॒ तत॒ सस्थि॑ते। यानि॑ घ॒र्मे क॒पालान्युपचि॒न्वन्ति॑ वे॒धस॒ इति॒ चतु॑ष्पदय॒र्चा वि मु॑ञ्चति। चतु॑ष्पादः प॒शव॑। प॒शुष्वे॒वोपरि॑ष्टा॒त्प्रति॑ तिष्ठति॥५५॥\anuvakamend[व॒र्त॒य॒ति॒ दिव॑मे॒वैतेन॑ दृहति स॒म्भव॑ति॒ त सस्कृ॑तमा॒त्मान॒न्द्वाद॑श॒ सस्थि॑ते॒ त्रीणि॑ च]

%3.2.8.1
दे॒वस्य॑ त्वा सवि॒तुः प्र॑स॒व इत्या॑ह॒ प्रसूत्यै। अ॒श्विनोर्बा॒हुभ्या॒मित्या॑ह। अ॒श्विनौ॒ हि दे॒वाना॑मध्व॒र्यू आस्ताम्। पू॒ष्णो हस्ताभ्या॒मित्या॑ह॒ यत्यै। सं व॑पा॒मीत्या॑ह। य॒था॒दे॒व॒तमे॒वैना॑नि॒ संव॑पति। समापो॑ अ॒द्भिर॑ग्मत॒ समोष॑धयो॒ रसे॒नेत्या॑ह। आपो॒ वा ओष॑धीर्जिन्वन्ति। ओष॑धयो॒ऽपो जि॑न्वन्ति। अ॒न्या वा ए॒तासा॑म॒न्या जि॑न्वन्ति॥५६॥

%3.2.8.2
तस्मा॑दे॒वमा॑ह। स रे॒वती॒र्जग॑तीभि॒र्मधु॑मती॒र्मधु॑मतीभिः सृज्यध्व॒मित्या॑ह। आपो॒ वै रे॒वती। प॒शवो॒ जग॑तीः। ओष॑धयो॒ मधु॑मतीः। आप॒ ओष॑धीः प॒शून्। ताने॒वास्मा॑ एक॒धा स॒सृज्य॑। मधु॑मतः करोति। अ॒द्भ्यः परि॒ प्रजा॑ताः स्थ॒ सम॒द्भिः पृ॑च्यध्व॒मिति॑ प॒र्याप्ला॑वयति। यथा॒ सुवृ॑ष्ट इ॒माम॑नुवि॒सृत्य॑॥५७॥

%3.2.8.3
आप॒ ओष॑धीर्म॒हय॑न्ति। ता॒दृगे॒व तत्। जन॑यत्यै त्वा॒ संयौ॒मीत्या॑ह। प्र॒जा ए॒वैतेन॑ दाधार। अ॒ग्नये त्वा॒ऽग्नीषोमाभ्या॒मित्या॑ह॒ व्यावृ॑त्त्यै। म॒खस्य॒ शिरो॒ऽसीत्या॑ह। य॒ज्ञो वै म॒खः। तस्यै॒तच्छिर॑। यत्पु॑रो॒डाश॑। तस्मा॑दे॒वमा॑ह॥५८॥

%3.2.8.4
घ॒र्मो॑ऽसि वि॒श्वायु॒रित्या॑ह। विश्व॑मे॒वायु॒र्यज॑माने दधाति। उ॒रु प्र॑थस्वो॒रु ते॑ य॒ज्ञप॑तिः प्रथता॒मित्या॑ह। यज॑मानमे॒व प्र॒जया॑ प॒शुभि॑ प्रथयति। त्वचं॑ गृह्णी॒ष्वेत्या॑ह। सर्व॑मे॒वैन॒ सत॑नुं करोति। अथा॒प आ॒नीय॒ परि॑मार्ष्टि। मा॒स ए॒व तत्त्वचं॑ दधाति। तस्मात्त्व॒चा मा॒सं छ॒न्नम्। घ॒र्मो वा ए॒षोऽशान्तः ॥५९॥

%3.2.8.5
अ॒र्ध॒मा॒सेऽर्धमासे॒ प्रवृ॑ज्यते। यत्पु॑रो॒डाश॑। स ईश्व॒रो यज॑मान शु॒चा प्र॒दह॑। पर्य॑ग्नि करोति। प॒शुमे॒वैन॑मकः। शान्त्या॒ अप्र॑दाहाय। त्रिः पर्य॑ग्नि करोति। त्र्या॑वृ॒द्धि य॒ज्ञः। अथो॒ रक्ष॑सा॒मप॑हत्यै। अ॒न्तरि॑त॒ रक्षो॒ऽन्तरि॑ता॒ अरा॑तय॒ इत्या॑ह॥६०॥

%3.2.8.6
रक्ष॑साम॒न्तर्\mbox{}हि॑त्यै। पु॒रो॒डाशं॒ वा अधि॑श्रित॒ रक्षास्यजिघासन्। दि॒वि नाको॒ नामा॒ग्नी र॑क्षो॒हा। स ए॒वास्मा॒द्रक्षा॒स्यपा॑हन्। दे॒वस्त्वा॑ सवि॒ता श्र॑पय॒त्वित्या॑ह। स॒वि॒तृप्र॑सूत ए॒वैन श्रपयति। वर्\mbox{}षि॑ष्ठे॒ अधि॒ नाक॒ इत्या॑ह। रक्ष॑सा॒मप॑हत्यै। अ॒ग्निस्ते॑ त॒नुवं॒ माऽति॑धा॒गित्या॒हाऽन॑तिदाहाय। अग्ने॑ ह॒व्य र॑क्ष॒स्वेत्या॑ह॒ गुप्त्यै॥६१॥

%3.2.8.7
अवि॑दहन्तः श्रपय॒तेति॒ वाचं॒ विसृ॑जते। य॒ज्ञमे॒व ह॒वीष्य॑भिव्या॒हृत्य॒ प्रत॑नुते। पु॒रो॒रुच॒मवि॑दाहाय॒ शृत्त्यै॑ करोति। म॒स्तिष्को॒ वै पु॑रो॒डाश॑। तं यन्नाभि॑ वा॒सयेत्। आ॒विर्म॒स्तिष्क॑ स्यात्। अ॒भिवा॑सयति। तस्मा॒द्गुहा॑ म॒स्तिष्क॑। भस्म॑ना॒ऽभिवा॑सयति। तस्मान्मा॒सेनास्थि॑ छ॒न्नम्॥६२॥

%3.2.8.8
वे॒देना॒भिवा॑सयति। तस्मा॒त्केशै॒ शिर॑श्छ॒न्नम्। अख॑लतिभावुको भवति। य ए॒वं वेद॑। प॒शोर्वै प्र॑ति॒मा पु॑रो॒डाश॑। स नाय॒जुष्क॑मभि॒वास्य॑। वृथे॑व स्यात्। ई॒श्व॒रा यज॑मानस्य प॒शव॒ प्रमे॑तोः। सं ब्रह्म॑णा पृच्य॒स्वेत्या॑ह। प्रा॒णा वै ब्रह्म॑॥६३॥

%3.2.8.9
प्रा॒णाः प॒शव॑। प्रा॒णैरे॒व प॒शून्त्संपृ॑णक्ति। न प्र॒मायु॑का भवन्ति। यज॑मानो॒ वै पु॑रो॒डाश॑। प्र॒जा प॒शव॒ पुरी॑षम्। यदे॒वम॑भिघा॒रय॑ति। यज॑मानमे॒व प्र॒जया॑ प॒शुभि॒ सम॑र्धयति। दे॒वा वै ह॒विर्भृ॒त्वाऽब्रु॑वन्। कस्मि॑न्नि॒दं म्र॑क्ष्यामह॒ इति॑। सोऽग्निर॑ब्रवीत्॥६४॥

%3.2.8.10
मयि॑ त॒नूः सं निध॑ध्वम्। अ॒हं व॒स्तं ज॑नयिष्यामि। यस्मि॑न्म्र॒क्ष्यध्व॒ इति॑। ते दे॒वा अ॒ग्नौ त॒नूः संन्य॑दधत। तस्मा॑दाहुः। अ॒ग्निः सर्वा॑ दे॒वता॒ इति॑। सोऽङ्गा॑रेणा॒पः। अ॒भ्य॑पातयत्। तत॑ एक॒तो॑ऽजायत। स द्वि॒तीय॑म॒भ्य॑पातयत्॥६५॥

%3.2.8.11
ततो द्वि॒तो॑ऽजायत। स तृ॒तीय॑म॒भ्य॑पातयत्। तत॑स्त्रि॒तो॑ऽजायत। यद॒द्भ्योऽजा॑यन्त। तदा॒प्याना॑माप्य॒त्वम्। यदा॒त्मभ्योऽजा॑यन्त। तदा॒त्म्याना॑मात्म्य॒त्वम्। ते दे॒वा आ॒प्येष्व॑मृजत। आ॒प्या अ॑मृजत॒ सूर्याभ्युदिते। सूर्याभ्युदित॒ सूर्या॑भिनिम्रुक्ते॥६६॥

%3.2.8.12
सूर्या॑भिनिम्रुक्तः कुन॒खिनि॑। कु॒न॒खी श्या॒वद॑ति। श्या॒वद॑न्नग्रदिधि॒षौ। अ॒ग्र॒दि॒धि॒षुः प॑रिवि॒त्ते। प॒रि॒वि॒त्तो वी॑र॒हणि॑। वी॒र॒हा ब्र॑ह्म॒हणि॑। तद्ब्र॑ह्म॒हणं॒ नात्य॑च्यवत। अ॒न्त॒र्वे॒दि निन॑य॒त्यव॑रुध्यै। उल्मु॑केना॒भि गृ॑ह्णाति शृत॒त्वाय॑। शृ॒तका॑मा इव॒ हि दे॒वाः॥६७॥\anuvakamend[अ॒न्या जि॑न्वन्त्यनु वि॒सृत्यै॒वमा॒हाशान्त आह॒ गुप्त्यै॑ छ॒न्नं ब्रह्माब्रवीद्द्वि॒तीय॑म॒भ्य॑पातय॒त्सूर्या॑भिनिम्रुक्ते दे॒वाः]

%3.2.9.1
दे॒वस्य॑ त्वा सवि॒तुः प्र॑स॒व इति॒ स्प्यमाद॑त्ते॒ प्रसूत्यै। अ॒श्विनोर्बा॒हुभ्या॒मित्या॑ह। अ॒श्विनौ॒ हि दे॒वाना॑मध्व॒र्यू आस्ताम्। पू॒ष्णो हस्ताभ्या॒मित्या॑ह॒ यत्यै। आद॑द॒ इन्द्र॑स्य बा॒हुर॑सि॒ दक्षि॑ण॒ इत्या॑ह। इ॒न्द्रि॒यमे॒व यज॑माने दधाति। स॒हस्र॑भृष्टिः श॒तते॑जा॒ इत्या॑ह। रू॒पमे॒वास्यै॒तन्म॑हि॒मानं॒ व्याच॑ष्टे। वा॒युर॑सि ति॒ग्मते॑जा॒ इत्या॑ह। तेजो॒ वै वा॒युः॥६८॥

%3.2.9.2
तेज॑ ए॒वास्मि॑न्दधाति। वि॒षाद्वै नामा॑सु॒र आ॑सीत्। सो॑ऽबिभेत्। य॒ज्ञेन॑ मा दे॒वा अ॒भिभ॑विष्य॒न्तीति॑। स पृ॑थि॒वीम॒भ्य॑वमीत्। सा मे॒ध्याऽभ॑वत्। अथो॒ यदिन्द्रो॑ वृ॒त्रमह\sn{}। तस्य॒ लोहि॑तं पृथि॒वीमनु॒ व्य॑धावत्। सा मे॒ध्याऽभ॑वत्। पृथि॑वि देवयज॒नीत्या॑ह॥६९॥

%3.2.9.3
मेध्या॑मे॒वैनां देव॒यज॑नीं करोति। ओष॑ध्यास्ते॒ मूलं॒ मा हिसिष॒मित्या॑ह। ओष॑धीना॒महिसायै। व्र॒जं ग॑च्छ गो॒स्थान॒मित्या॑ह। छन्दासि॒ वै व्र॒जो गो॒स्थान॑। छन्दास्ये॒वास्मै व्र॒जं गो॒स्थानं॑ करोति। वर्\mbox{}ष॑तु ते॒ द्यौरित्या॑ह। वृष्टि॒र्वै द्यौः। वृष्टि॑मे॒वाव॑ रुन्धे। ब॒धा॒न दे॑व सवितः पर॒मस्यां परा॒वतीत्या॑ह॥७०॥

%3.2.9.4
द्वौ वाव पुरु॑षौ। यं चै॒व द्वेष्टि॑। यश्चै॑नं॒ द्वेष्टि॑। तावु॒भौ ब॑ध्नाति पर॒मस्यां परा॒वति॑ श॒तेन॒ पाशै। योऽस्मान्द्वेष्टि॒ यं च॑ व॒यं द्वि॒ष्मस्तमतो॒ मा मौ॒गित्या॒हानि॑म्रुक्त्ये। अ॒ररु॒र्वै नामा॑सु॒र आ॑सीत्। स पृ॑थि॒व्यामुप॑म्लुप्तोऽशयत्। तं दे॒वा अप॑हतो॒ऽररु॑ पृथि॒व्या इति॑ पृथि॒व्या अपाघ्नन्। भ्रातृ॑व्यो॒ वा अ॒ररु॑। अप॑हतो॒ऽररु॑ पृथि॒व्या इति॒ यदाह॑॥७१॥

%3.2.9.5
भ्रातृ॑व्यमे॒व पृ॑थि॒व्या अप॑हन्ति। ते॑ऽमन्यन्त। दिवं॒ वा अ॒यमि॒तः प॑तिष्य॒तीति॑। तम॒ररु॑स्ते॒ दिवं॒ माऽस्का॒निति॑ दि॒वः पर्य॑बाधन्त। भ्रातृ॑व्यो॒ वा अ॒ररु॑। अ॒ररु॑स्ते॒ दिवं॒ मा स्का॒निति॒ यदाह॑। भ्रातृ॑व्यमे॒व दि॒वः परि॑बाधते। स्त॒म्ब॒य॒जुर्‌ह॑रति। पृ॒थि॒व्या ए॒व भ्रातृ॑व्य॒मप॑हन्ति। द्वि॒तीय हरति॥७२॥

%3.2.9.6
अ॒न्तरि॑क्षादे॒वैन॒मप॑हन्ति। तृ॒तीय हरति। दि॒व ए॒वैन॒मप॑हन्ति। तू॒ष्णीं च॑तु॒र्थ ह॑रति। अप॑रिमितादे॒वैन॒मप॑हन्ति। असु॑राणां॒ वा इ॒यमग्र॑ आसीत्। याव॒दासी॑नः परा॒पश्य॑ति। ताव॑द्दे॒वानाम्। ते दे॒वा अ॑ब्रुवन्। अस्त्वे॒व नो॒ऽस्यामपीति॑॥७३॥

%3.2.9.7
क्य॑न्नो दास्य॒थेति॑। याव॑त्स्व॒यं प॑रिगृह्णी॒थेति॑। ते वस॑व॒स्त्वेति॑ दक्षिण॒तः पर्य॑गृह्णन्। रु॒द्रास्त्वेति॑ प॒श्चात्। आ॒दि॒त्यास्त्वेत्यु॑त्तर॒तः। तेऽग्निना॒ प्राञ्चो॑ऽजयन्। वसु॑भिर्दक्षि॒णा। रु॒द्रैः प्र॒त्यञ्च॑। आ॒दि॒त्यैरुद॑ञ्चः। यस्यै॒वं वि॒दुषो॒ वेदिं॑ परिगृ॒ह्णन्ति॑॥७४॥

%3.2.9.8
भव॑त्या॒त्मना। पराऽस्य॒ भ्रातृ॑व्यो भवति। दे॒वस्य॑ सवि॒तुः स॒व इत्या॑ह॒ प्रसूत्यै। कर्म॑ कृण्वन्ति वे॒धस॒ इत्या॑ह। इ॒षि॒त हि कर्म॑ क्रि॒यते। पृ॒थि॒व्यै मेध्यं॑ चामे॒ध्यं च॒ व्युद॑क्रामताम्। प्रा॒चीन॑मुदी॒चीनं॒ मेध्यम्। प्र॒ती॒चीनं॑ दक्षि॒णाऽमे॒ध्यम्। प्राची॒मुदी॑चीं प्रव॒णां क॑रोति। मेध्या॑मे॒वैनां देव॒यज॑नीं करोति॥७५॥

%3.2.9.9
प्राञ्चौ॑ वेद्य॒सावुन्न॑यति। आ॒ह॒व॒नीय॑स्य॒ परि॑गृहीत्यै। प्र॒तीची॒ श्रोणी। गार्ह॑पत्यस्य॒ परि॑गृहीत्यै। अथो॑ मिथुन॒त्वाय॑। उद्ध॑न्ति। यदे॒वास्या॑ अमे॒ध्यम्। तदप॑हन्ति। उद्ध॑न्ति। तस्मा॒दोष॑धय॒ परा॑भवन्ति॥७६॥

%3.2.9.10
मूलं॑ छिनत्ति। भ्रातृ॑व्यस्यै॒व मूलं॑ छिनत्ति। मूलं॒ वा अ॑ति॒तिष्ठ॒द्रक्षा॒स्यनूत्पि॑पते। यद्धस्ते॑न छि॒न्द्यात्। कु॒न॒खिनी प्र॒जाः स्यु॑। स्प्येन॑ छिनत्ति। वज्रो॒ वै स्प्यः। वज्रे॑णै॒व य॒ज्ञाद्रक्षा॒स्यप॑हन्ति। पि॒तृ॒दे॒व॒त्याऽति॑खाता। इय॑तीं खनति॥७७॥

%3.2.9.11
प्र॒जाप॑तिना यज्ञमु॒खेन॒ सम्मि॑ताम्। वेदि॑र्दे॒वेभ्यो॒ निला॑यत। तां च॑तुरङ्गु॒लेऽन्व॑विन्दन्। तस्माच्चतुरङ्गु॒लं खेया। च॒तु॒र॒ङ्गु॒लं ख॑नति। च॒तु॒र॒ङ्गु॒ले ह्योष॑धयः प्रति॒तिष्ठ॑न्ति। आ प्र॑ति॒ष्ठायै॑ खनति। यज॑मानमे॒व प्र॑ति॒ष्ठां ग॑मयति। द॒क्षि॒ण॒तो वर्\mbox{}षी॑यसीं करोति। दे॒व॒यज॑नस्यै॒व रू॒पम॑कः॥७८॥

%3.2.9.12
पुरी॑षवतीं करोति। प्र॒जा वै प॒शव॒ पुरी॑षम्। प्र॒जयै॒वैनं॑ प॒शुभि॒ पुरी॑षवन्तं करोति। उत्त॑रं परिग्रा॒हं परि॑गृह्णाति। ए॒ताव॑ती॒ वै पृ॑थि॒वी। याव॑ती॒ वेदि॑। तस्या॑ ए॒तावत॑ ए॒व भ्रातृ॑व्यं नि॒र्भज्य॑। आ॒त्मन॒ उत्त॑रं परिग्रा॒हं परि॑गृह्णाति। ऋ॒तम॑स्यृत॒सद॑नमस्यृत॒श्रीर॒सीत्या॑ह। य॒था॒य॒जुरे॒वैतत्॥७९॥

%3.2.9.13
क्रू॒रमि॑व॒ वा ए॒तत्क॑रोति। यद्वेदिं॑ क॒रोति॑। धा अ॑सि स्व॒धा अ॒सीति॑ योयुप्यते॒ शान्त्यै। उ॒र्वी चासि॒ वस्वी॑ चा॒सीत्या॑ह। उ॒र्वीमे॒वैनां॒ वस्वीं करोति। पु॒रा क्रू॒रस्य॑ वि॒सृपो॑ विरप्शि॒न्नित्या॑ह मेध्य॒त्वाय॑। उ॒दा॒दाय॑ पृथि॒वीं जी॒रदा॑नु॒र्यामैर॑यञ्च॒न्द्रम॑सि स्व॒धाभि॒रित्या॑ह। यदे॒वास्या॑ अमे॒ध्यम्। तद॑प॒हत्य॑। मेध्यां देव॒यज॑नीं कृ॒त्वा॥८०॥

%3.2.9.14
यद॒दश्च॒न्द्रम॑सि॒ मेध्यम्। तद॒स्यामेर॑यति। तां धीरा॑सो अनु॒दृश्य॑ यजन्त॒ इत्या॒हानु॑ख्यात्यै। प्रोक्ष॑णी॒रा सा॑दय। इ॒ध्माब॒र्\mbox{}हिरुप॑सादय। स्रु॒वं च॒ स्रुच॑श्च॒ संमृ॑ड्ढि। पत्नी॒ संन॑ह्य। आज्ये॑नो॒देहीत्या॑हानुपू॒र्वता॑यै। प्रोक्ष॑णी॒रा सा॑दयति। आपो॒ वै र॑क्षो॒घ्नीः॥८१॥

%3.2.9.15
रक्ष॑सा॒मप॑हत्यै। स्प्यस्य॒ वर्त्मन्त्सादयति। य॒ज्ञस्य॒ संत॑त्यै। उ॒वाच॒ हासि॑तो दैव॒लः। ए॒ताव॑ती॒र्वा अ॒मुष्मि॑ल्लोँ॒क आप॑ आसन्। याव॑ती॒ प्रोक्ष॑णी॒रिति॑। तस्माद्ब॒ह्वीरा॒साद्या। स्प्यमु॒दस्य\sn{}। यं द्वि॒ष्यात्तं ध्या॑येत्। शु॒चैवैन॑मर्पयति॥८२॥\anuvakamend[वै वा॒युरा॑ह परा॒वतीत्या॒हाह॑ द्वि॒तीय हर॒तीति॑ परिगृ॒ह्णन्ति॑ देव॒यज॑नीं करोति भवन्ति खनत्यकरे॒तत्कृ॒त्वा र॑क्षो॒घ्नीर॑र्पयति]

%3.2.10.1
वज्रो॒ वै स्प्यः। यद॒न्वञ्चं॑ धा॒रयेत्। वज्रेऽध्व॒र्युः क्ष॑ण्वीत। पु॒रस्तात्ति॒र्यञ्चं॑ धारयति। वज्रो॒ वै स्प्यः। वज्रे॑णै॒व य॒ज्ञस्य॑ दक्षिण॒तो रक्षा॒स्यप॑हन्ति। अ॒ग्निभ्यां॒ प्राच॑श्च प्र॒तीच॑श्च। स्प्येनोदी॑चश्चाध॒राच॑श्च। स्प्येन॒ वा ए॒ष वज्रे॑णा॒स्यै पा॒प्मानं॒ भ्रातृ॑व्यमप॒हत्य॑। उ॒त्क॒रेऽधि॒ प्रवृ॑श्चति॥८३॥

%3.2.10.2
यथो॑प॒धाय॑ वृ॒श्चन्त्ये॒वम्। हस्ता॒वव॑ नेनिक्ते। आ॒त्मान॑मे॒व प॑वयते। स्प्यं प्रक्षा॑लयति मेध्य॒त्वाय॑। अथो॑ पा॒प्मन॑ ए॒व भ्रातृ॑व्यस्य न्य॒ङ्गं छि॑नत्ति। इ॒ध्माब॒र्\mbox{}हिरुप॑सादयति॒ युक्त्यै। य॒ज्ञस्य॑ मिथुन॒त्वाय॑। अथो॑ पुरो॒रुच॑मे॒वैतां द॑धाति। उत्त॑रस्य॒ कर्म॒णोऽनु॑ख्यात्यै। न पु॒रस्तात्प्र॒त्यगुप॑सादयेत्॥८४॥

%3.2.10.3
यत्पु॒रस्तात्प्र॒त्यगु॑पसा॒दयेत्। अ॒न्यत्रा॑हुतिप॒थादि॒ध्मं प्रति॑पादयेत्। प्र॒जा वै ब॒र्॒हिः। अप॑राध्नुयाद्ब॒र्॒हिषा प्र॒जानां प्र॒जन॑नम्। प॒श्चात्प्रागुप॑सादयति। आ॒हु॒ति॒प॒थेने॒ध्मं प्रति॑पादयति। सं॒प्र॒त्ये॑व ब॒र्॒हिषा प्र॒जानां प्र॒जन॑न॒मुपै॑ति। दक्षि॑णमि॒ध्मम्। उत्त॑रं ब॒र्॒हिः। आ॒त्मा वा इ॒ध्मः। प्र॒जा ब॒र्॒हिः। प्र॒जा ह्यात्मन॒ उत्त॑रतरा ती॒र्थे। ततो॒ मेध॑मुप॒नीय॑। य॒था॒दे॒व॒तमे॒वैन॒त्प्रति॑ष्ठापयति। प्रति॑तिष्ठति प्र॒जया॑ प॒शुभि॒र्यज॑मानः॥८५॥\anuvakamend[वृ॒श्च॒ति॒ सा॒द॒ये॒दि॒ध्मः पञ्च॑ च]




\prashnaend{तृ॒तीय॑स्यां दे॒वस्याश्वप॒र्\mbox{}शुं यो वै पूर्वे॒द्युः कर्म॑णे वा॒मिन्द्रो॑ वृ॒त्रम॑ह॒न्त्सो॑ऽपोऽव॑धूत॒न्धृष्टि॑र्दे॒वस्येत्या॑ह॒ सं व॑पामि दे॒वस्य॒ स्प्यमा द॑दे॒ वज्रो॒ वै स्प्यो दश॑॥१०॥}{तृ॒तीय॑स्यां य॒ज्ञस्यान॑तिरेकाय प॒वित्र॑वत्यध्व॒र्युं चा॑धि॒षव॑णमस्य॒न्तरि॑क्ष ए॒व रक्ष॑साम॒न्तर्\mbox{}हि॑त्यै॒ द्वौ वाव पुरु॑षौ॒ यद॒दश्च॒न्द्रम॑सि॒ मेध्यं॒ पञ्चाशी॑तिः॥८५॥}{तृ॒तीय॑स्यां॒ यज॑मानः॥}{हरि॑ ओम्॥}{इति श्रीकृष्णयजुर्वेदीयतैत्तिरीयब्राह्मणे तृतीयाष्टके द्वितीयः प्रपाठकः समाप्तः॥}
\clearpage
