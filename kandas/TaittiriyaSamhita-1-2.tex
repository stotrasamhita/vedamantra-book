\chapt{काण्डम् १}
\sect{द्वितीयः प्रश्नः}\setcounter{anuvakam}{0}
\dnsub{तैत्तिरीयसंहितायां प्रथमकाण्डे द्वितीयः प्रश्नः}
%1.2.1.1
आप॑ उन्दन्तु जी॒वसे॑ दीर्घायु॒त्वाय॒ वर्च॑स॒ ओष॑धे॒ त्राय॑स्वैन॒ꣴ॒ स्वधि॑ते॒ मैनꣳ॑ हिꣳसीर्देव॒श्रूरे॒तानि॒ प्र व॑पे स्व॒स्त्युत्त॑राण्यशी॒या\-ऽऽ\-पो॑ अ॒स्मान्मा॒तरः॑ शुन्धन्तु घृ॒तेन॑ नो घृत॒पुवः॑ पुनन्तु॒ विश्व॑म॒स्मत्प्र व॑हन्तु रि॒प्रमुदा᳚भ्यः॒ शुचि॒रा पू॒त ए॑मि॒ सोम॑स्य त॒नूर॑सि त॒नुवं॑ मे पाहि मही॒नां पयो॑\-ऽसि वर्चो॒धा अ॑सि॒ वर्चो॒~(१)

%1.2.1.2
मयि॑ धेहि वृ॒त्रस्य॑ क॒नीनि॑का\-ऽसि चक्षु॒ष्पा अ॑सि॒ चक्षु॑र्मे पाहि चि॒त्पति॑स्त्वा पुनातु वा॒क्पति॑स्त्वा पुनातु दे॒वस्त्वा॑ सवि॒ता पु॑ना॒त्वच्छि॑द्रेण प॒वित्रे॑ण॒ वसोः॒ सूर्य॑स्य र॒श्मिभि॒स्तस्य॑ ते पवित्रपते प॒वित्रे॑ण॒ यस्मै॒ कं पु॒ने तच्छ॑केय॒मा वो॑ देवास ईमहे॒ सत्य॑धर्माणो अध्व॒रे यद्वो॑ देवास आगु॒रे यज्ञि॑यासो॒ हवा॑मह॒ इन्द्रा᳚ग्नी॒ द्यावा॑पृथिवी॒ आप॑ ओषधी॒स्त्वं दी॒क्षाणा॒मधि॑\-पतिरसी॒ह मा॒ सन्तं॑ पाहि॥~(२)

{\anuvakamend[{वर्च॑ ओषधीर॒ष्टौ च॑}]}%~(१)

%1.2.2.1
आकू᳚त्यै प्र॒युजे॒\-ऽग्नये॒ स्वाहा॑ मे॒धायै॒ मन॑से॒\-ऽग्नये॒ स्वाहा॑ दी॒क्षायै॒ तप॑से॒\-ऽग्नये॒ स्वाहा॒ सर॑स्वत्यै पू॒ष्णे᳚\-ऽग्नये॒ स्वाहा\-ऽऽ\-पो॑ देवीर्बृहतीर्विश्वशम्भुवो॒ द्यावा॑\-पृथि॒वी उ॒र्व॑न्तरि॑क्षं॒ बृह॒स्पति॑र्नो ह॒विषा॑ वृधातु॒ स्वाहा॒ विश्वे॑ दे॒वस्य॑ ने॒तुर्मर्तो॑\-ऽवृणीत स॒ख्यं विश्वे॑ रा॒य इ॑षुध्यसि द्यु॒म्नं वृ॑णीत पु॒ष्यसे॒ स्वाह॑र्ख्सा॒मयोः॒ शिल्पे᳚ स्थ॒स्ते वा॒मा र॑भे॒ ते मा॑~(३)

%1.2.2.2
पात॒मा\-ऽस्य य॒ज्ञस्यो॒दृच॑ इ॒मां धिय॒ꣳ॒ शिक्ष॑माणस्य देव॒ क्रतुं॒ दक्षं॑ वरुण॒ सꣳशि॑शाधि॒ यया\-ऽति॒ विश्वा॑ दुरि॒ता तरे॑म सु॒तर्मा॑ण॒मधि॒ नावꣳ॑ रुहे॒मोर्ग॑स्याङ्गिर॒स्यूर्ण॑म्रदा॒ ऊर्जं॑ मे यच्छ पा॒हि मा॒ मा मा॑ हिꣳसी॒र्विष्णोः॒ शर्मा॑सि॒ शर्म॒ यज॑मानस्य॒ शर्म॑ मे यच्छ॒ नक्ष॑त्राणां मा\-ऽतीका॒शात् पा॒हीन्द्र॑स्य॒ योनि॑रसि॒~(४)

%1.2.2.3
मा मा॑ हिꣳसीः कृ॒ष्यै त्वा॑ सुस॒स्यायै॑ सुपिप्प॒लाभ्य॒स्त्वौष॑\-धीभ्यः सूप॒स्था दे॒वो वन॒स्पति॑रू॒र्ध्वो मा॑ पा॒ह्योदृचः॒ स्वाहा॑ य॒ज्ञं मन॑सा॒ स्वाहा॒ द्यावा॑\-पृथि॒वीभ्या॒ꣴ॒ स्वाहो॒रोर॒न्तरि॑क्षा॒थ्\-स्वाहा॑ य॒ज्ञं वाता॒दा र॑भे॥~(५)

{\anuvakamend[{मा॒ योनि॑रसि त्रि॒ꣳ॒शच्च॑}]}%~(२)

%1.2.3.1
दैवीं॒ धियं॑ मनामहे सुमृडी॒काम॒भिष्ट॑ये वर्चो॒धां य॒ज्ञवा॑हसꣳ सुपा॒रा नो॑ अस॒द्वशे᳚। ये दे॒वा मनो॑जाता मनो॒युजः॑ सु॒दक्षा॒ दक्ष॑पितार॒स्ते नः॑ पान्तु॒ ते नो॑\-ऽवन्तु॒ तेभ्यो॒ नम॒स्तेभ्यः॒ स्वाहा\-ऽग्ने॒ त्वꣳ सु जा॑गृहि व॒यꣳ सु म॑न्दिषीमहि गोपा॒य नः॑ स्व॒स्तये᳚ प्र॒बुधे॑ नः॒ पुन॑र्ददः। त्वम॑ग्ने व्रत॒पा अ॑सि दे॒व आ मर्त्ये॒ष्वा। त्वं~(६)

%1.2.3.2
य॒ज्ञेष्वीड्यः॑॥ विश्वे॑ दे॒वा अ॒भि मा मा\-ऽव॑वृत्रन् पू॒षा स॒न्या सोमो॒ राध॑सा दे॒वः स॑वि॒ता वसो᳚र्वसु॒दावा॒ रास्वेय॑थ्सो॒मा\-ऽऽ\-भूयो॑ भर॒ मा पृ॒णन्पू॒र्त्या वि रा॑धि॒ मा\-ऽहमायु॑षा च॒न्द्रम॑सि॒ मम॒ भोगा॑य भव॒ वस्त्र॑मसि॒ मम॒ भोगा॑य भवो॒स्रा\-ऽसि॒ मम॒ भोगा॑य भव॒ हयो॑\-ऽसि॒ मम॒ भोगा॑य भव॒~(७)

%1.2.3.3
छागो॑\-ऽसि॒ मम॒ भोगा॑य भव मे॒षो॑\-ऽसि॒ मम॒ भोगा॑य भव वा॒यवे᳚ त्वा॒ वरु॑णाय त्वा॒ निर्\mbox{}ऋ॑त्यै त्वा रु॒द्राय॑ त्वा॒ देवी॑रापो अपां नपा॒द्य ऊ॒र्मिर्\mbox{}ह॑वि॒ष्य॑ इन्द्रि॒यावा᳚न्म॒दिन्त॑म॒स्तं वो॒ मा\-ऽव॑क्रमिष॒मच्छि॑न्नं॒ तन्तुं॑ पृथि॒व्या अनु॑ गेषं भ॒द्राद॒भि श्रेयः॒ प्रेहि॒ बृह॒स्पतिः॑ पुरए॒ता ते॑ अ॒स्त्वथे॒मव॑ स्य॒ वर॒ आ पृ॑थि॒व्या आ॒रे शत्रू᳚न् कृणुहि॒ सर्व॑वीर॒ एदम॑गन्म देव॒यज॑नं पृथि॒व्या विश्वे॑ दे॒वा यदजु॑षन्त॒ पूर्व॑ ऋख्सा॒माभ्यां॒ यजु॑षा स॒न्तर॑न्तो रा॒यस्पोषे॑ण॒ समि॒षा म॑देम॥~(८)

{\anuvakamend[{आ त्वꣳ हयो॑\-ऽसि॒ मम॒ भोगा॑य भव स्य॒ पञ्च॑विꣳशतिश्च}]}%~(३)

%1.2.4.1
इ॒यं ते॑ शुक्र त॒नूरि॒दं वर्च॒स्तया॒ सं भ॑व॒ भ्राजं॑ गच्छ॒ जूर॑सि धृ॒ता मन॑सा॒ जुष्टा॒ विष्ण॑वे॒ तस्या᳚स्ते स॒त्यस॑वसः प्रस॒वे वा॒चो य॒न्त्रम॑शीय॒ स्वाहा॑ शु॒क्रम॑स्य॒मृत॑मसि वैश्वदे॒वꣳ ह॒विः सूर्य॑स्य॒ चक्षु॒रा\-ऽरु॑हम॒ग्नेर॒क्ष्णः क॒नीनि॑कां॒ यदेत॑शेभि॒रीय॑से॒ भ्राज॑मानो विप॒श्चिता॒ चिद॑सि म॒ना\-ऽसि॒ धीर॑सि॒ दक्षि॑णा-~(९)

%1.2.4.2
ऽसि य॒ज्ञिया॑\-ऽसि क्ष॒त्रिया॒\-ऽस्यदि॑तिरस्युभ॒यतः॑शीर्ष्णी॒ सा नः॒ सुप्रा॑ची॒ सुप्र॑तीची॒ सं भ॑व मि॒त्रस्त्वा॑ प॒दि ब॑ध्नातु पू॒षा\-ऽध्व॑नः पा॒त्विन्द्रा॒याध्य॑क्षा॒यानु॑ त्वा मा॒ता म॑न्यता॒मनु॑ पि॒ता\-ऽनु॒ भ्राता॒ सग॒र्भ्यो\-ऽनु॒ सखा॒ सयू᳚थ्यः॒ सा दे॑वि दे॒वमच्छे॒हीन्द्रा॑य॒ सोमꣳ॑ रु॒द्रस्त्वा\-ऽऽ\-व॑र्तयतु मि॒त्रस्य॑ प॒था स्व॒स्ति सोम॑सखा॒ पुन॒रेहि॑ स॒ह र॒य्या॥~(१०)

{\anuvakamend[{दक्षि॑णा॒ सोम॑सखा॒ पञ्च॑ च}]}%~(४)

%1.2.5.1
वस्व्य॑सि रु॒द्रा\-ऽस्यदि॑तिरस्यादि॒त्या\-ऽसि॑ शु॒क्रा\-ऽसि॑ च॒न्द्रा\-ऽसि॒ बृह॒स्पति॑स्त्वा सु॒म्ने र॑ण्वतु रु॒द्रो वसु॑भि॒रा चि॑केतु पृथि॒व्यास्त्वा॑ मू॒र्धन्ना जि॑घर्मि देव॒यज॑न॒ इडा॑याः प॒दे घृ॒तव॑ति॒ स्वाहा॒ परि॑लिखित॒ꣳ॒ रक्षः॒ परि॑लिखिता॒ अरा॑तय इ॒दम॒हꣳ रक्ष॑सो ग्री॒वा अपि॑ कृन्तामि॒ यो᳚\-ऽस्मान् द्वेष्टि॒ यं च॑ व॒यं द्वि॒ष्म इ॒दम॑स्य ग्री॒वा~(११)

%1.2.5.2
अपि॑ कृन्ताम्य॒स्मे राय॒स्त्वे राय॒स्तोते॒ रायः॒ सं दे॑वि दे॒व्योर्वश्या॑ पश्यस्व॒ त्वष्टी॑मती ते सपेय सु॒रेता॒ रेतो॒ दधा॑ना वी॒रं वि॑देय॒ तव॑ स॒न्दृशि॒ मा\-ऽहꣳ रा॒यस्पोषे॑ण॒ वि यो॑षम्॥~(१२)

{\anuvakamend[{अ॒स्य॒ ग्री॒वा एका॒न्नत्रि॒ꣳ॒शच्च॑}]}%~(५)

%1.2.6.1
अ॒ꣳ॒शुना॑ ते अ॒ꣳ॒शुः पृ॑च्यतां॒ परु॑षा॒ परु॑र्ग॒न्धस्ते॒ काम॑मवतु॒ मदा॑य॒ रसो॒ अच्यु॑तो॒\-ऽमात्यो॑\-ऽसि शु॒क्रस्ते॒ ग्रहो॒\-ऽभि त्यं दे॒वꣳ स॑वि॒तार॑मू॒ण्योः᳚ क॒विक्र॑तु॒मर्चा॑मि स॒त्यस॑वसꣳ रत्न॒धाम॒भि प्रि॒यं म॒तिमू॒र्ध्वा यस्या॒मति॒र्भा अदि॑द्युत॒थ्सवी॑मनि॒ हिर॑ण्यपाणिरमिमीत सु॒क्रतुः॑ कृ॒पा सुवः॑। प्र॒जाभ्य॑स्त्वा प्रा॒णाय॑ त्वा व्या॒नाय॑ त्वा प्र॒जास्त्वमनु॒ प्राणि॑हि प्र॒जास्त्वामनु॒ प्राण॑न्तु॥~(१३)

{\anuvakamend[{अनु॑ स॒प्त च॑}]}%~(६)

%1.2.7.1
सोमं॑ ते क्रीणा॒म्यूर्ज॑स्वन्तं॒ पय॑स्वन्तं वी॒र्या॑वन्तमभिमाति॒\-षाहꣳ॑ शु॒क्रं ते॑ शु॒क्रेण॑ क्रीणामि च॒न्द्रं च॒न्द्रेणा॒मृत॑म॒मृते॑न स॒म्यत्ते॒ गोर॒स्मे च॒न्द्राणि॒ तप॑सस्त॒नूर॑सि प्र॒जा\-प॑ते॒र्वर्ण॒स्तस्या᳚स्ते सहस्रपो॒षं पुष्य॑न्त्याश्चर॒मेण॑ प॒शुना᳚ क्रीणाम्य॒स्मे ते॒ बन्धु॒र्मयि॑ ते॒ रायः॑ श्रयन्ताम॒स्मे ज्योतिः॑ सोमविक्र॒यिणि॒ तमो॑ मि॒त्रो न॒ एहि॒ सुमि॑त्रधा॒ इन्द्र॑स्यो॒रु मा वि॑श॒ दक्षि॑णमु॒शन्नु॒शन्तꣴ॑ स्यो॒नः स्यो॒नꣴ स्वान॒ भ्राजाङ्घा॑रे॒ बम्भा॑रे॒ हस्त॒ सुह॑स्त॒ कृशा॑नवे॒ते वः॑ सोम॒क्रय॑णा॒स्तान्र॑क्षध्वं॒ मा वो॑ दभन्न्॥~(१४)

{\anuvakamend[{ऊ॒रुं द्वाविꣳ॑शतिश्व}]}%~(७)

%1.2.8.1
उदायु॑षा स्वा॒युषोदोष॑धीना॒ꣳ॒ रसे॒नोत्प॒र्जन्य॑स्य॒ शुष्मे॒णोद॑स्था\-म॒मृता॒ꣳ॒ अनु॑। उ॒र्व॑न्तरि॑क्ष॒मन्वि॒ह्यदि॑त्याः॒ सदो॒\-ऽस्यदि॑त्याः॒ सद॒ आसी॒\-दास्त॑भ्ना॒द्द्यामृ॑ष॒भो अ॒न्तरि॑क्ष॒ममि॑मीत वरि॒माणं॑ पृथि॒व्या आसी॑द॒द्विश्वा॒ भुव॑नानि स॒म्राड्विश्वेत्तानि॒ वरु॑णस्य व्र॒तानि॒ वने॑षु॒ व्य॑न्तरि॑क्षं ततान॒ वाज॒मर्व॑थ्सु॒ पयो॑ अघ्नि॒यासु॑ हृ॒थ्सु~(१५)

%1.2.8.2
क्रतुं॒ वरु॑णो वि॒क्ष्व॑ग्निं दि॒वि सूर्य॑मदधा॒थ्सोम॒मद्रा॒वुदु॒त्यं जा॒तवे॑दसं दे॒वं व॑हन्ति के॒तवः॑। दृ॒शे विश्वा॑य॒ सूर्यम्᳚॥ उस्रा॒वेतं॑ धूर्\mbox{}षाहावन॒श्रू अवी॑रहणौ ब्रह्म॒चोद॑नौ॒ वरु॑णस्य॒ स्कम्भ॑नमसि॒ वरु॑णस्य स्कम्भ॒सर्ज॑नमसि॒ प्रत्य॑स्तो॒ वरु॑णस्य॒ पाशः॑॥~(१६)

{\anuvakamend[{हृ॒थ्सु पञ्च॑त्रिꣳशच्च}]}%~(८)

%1.2.9.1
प्रच्य॑वस्व भुवस्पते॒ विश्वा᳚न्य॒भि धामा॑नि॒ मा त्वा॑ परिप॒री वि॑द॒न्मा त्वा॑ परिप॒न्थिनो॑ विद॒न्मा त्वा॒ वृका॑ अघा॒यवो॒ मा ग॑न्ध॒र्वो वि॒श्वाव॑सु॒रा द॑घच्छ्ये॒नो भू॒त्वा परा॑ पत॒ यज॑मानस्य नो गृ॒हे दे॒वैः सꣴ॑स्कृ॒तं यज॑मानस्य स्व॒स्त्यय॑न्य॒स्यपि॒ पन्था॑मगस्महि स्वस्ति॒गाम॑ने॒हसं॒ येन॒ विश्वाः॒ परि॒ द्विषो॑ वृ॒णक्ति॑ वि॒न्दते॒ वसु॒ नमो॑ मि॒त्रस्य॒ वरु॑णस्य॒ चक्ष॑से म॒हो दे॒वाय॒ तदृ॒तꣳ स॑पर्यत दूरे॒\-दृशे॑ दे॒व\-जा॑ताय के॒तवे॑ दि॒वस्पु॒त्राय॒ सूर्या॑य शꣳसत॒ वरु॑णस्य॒ स्कम्भ॑\-नमसि॒ वरु॑णस्य स्कम्भ॒\-सर्ज॑नम॒स्युन्मु॑क्तो॒ वरु॑णस्य॒ पाशः॑॥~(१७)

{\anuvakamend[{मि॒त्रस्य॒ त्रयो॑विꣳशतिश्च}]}%~(९)

%1.2.10.1
अ॒ग्नेरा॑ति॒थ्यम॑सि॒ विष्ण॑वे त्वा॒ सोम॑स्या\-ऽऽ\-ति॒थ्यम॑सि॒ विष्ण॑वे॒ त्वा\-ऽति॑थेराति॒थ्यम॑सि॒ विष्ण॑वे त्वा॒\-ऽग्नये᳚ त्वा रायस्पोष॒दाव्न्ने॒ विष्ण॑वे त्वा श्ये॒नाय॑ त्वा सोम॒भृते॒ विष्ण॑वे त्वा॒ या ते॒ धामा॑नि ह॒विषा॒ यज॑न्ति॒ ता ते॒ विश्वा॑ परि॒भूर॑स्तु य॒ज्ञं ग॑य॒स्फानः॑ प्र॒तर॑णः सु॒वीरो\-ऽवी॑रहा॒ प्र च॑रा सोम॒ दुर्या॒नदि॑त्याः॒ सदो॒\-ऽस्यदि॑त्याः॒ सद॒ आ~(१८)

%1.2.10.2
सी॑द॒ वरु॑णो\-ऽसि धृ॒तव्र॑तो वारु॒णम॑सि शं॒योर्दे॒वानाꣳ॑ स॒ख्यान्मा दे॒वाना॑म॒प\-स॑श्छिथ्स्म॒\-ह्याप॑तये त्वा गृह्णामि॒ परि॑पतये त्वा गृह्णामि॒ तनू॒नप्त्रे᳚ त्वा गृह्णामि शाक्व॒राय॑ त्वा गृह्णामि॒ शक्म॒न्नोजि॑ष्ठाय त्वा गृह्णा॒म्यना॑धृष्टम\-स्यनाधृ॒ष्यं दे॒वाना॒मोजो॑\-ऽभिशस्ति॒पा अ॑नभिशस्ते॒\-ऽन्यमनु॑ मे दी॒क्षां दी॒क्षाप॑तिर्मन्यता॒मनु॒ तप॒स्तप॑स्पति॒रञ्ज॑सा स॒त्यमुप॑ गेषꣳ सुवि॒ते मा॑ धाः॥~(१९)

{\anuvakamend[{आ मैकं॑ च}]}%॥10॥

%1.2.11.1
अ॒ꣳ॒शुरꣳ॑शुस्ते देव सो॒मा\-ऽऽ\-प्या॑यता॒मिन्द्रा॑यैकधन॒विद॒ आ तुभ्य॒मिन्द्रः॑ प्यायता॒मा त्वमिन्द्रा॑य प्याय॒स्वा\-ऽऽ\- प्या॑यय॒ सखी᳚न्थ्स॒न्या मे॒धया᳚ स्व॒स्ति ते॑ देव सोम सु॒त्याम॑शी॒येष्टा॒ रायः॒ प्रेषे भगा॑य॒र्तमृ॑तवा॒दिभ्यो॒ नमो॑ दि॒वे नमः॑ पृथि॒व्या अग्ने᳚ व्रतपते॒ त्वं व्र॒तानां᳚ व्र॒तप॑तिरसि॒ या मम॑ त॒नूरे॒षा सा त्वयि॒~(२०)

%1.2.11.2
या तव॑ त॒नूरि॒यꣳ सा मयि॑ स॒ह नौ᳚ व्रतपते व्र॒तिनो᳚र्व्र॒तानि॒ या ते॑ अग्ने॒ रुद्रि॑या त॒नूस्तया॑ नः पाहि॒ तस्या᳚स्ते॒ स्वाहा॒ या ते॑ अग्ने\-ऽयाश॒या र॑जाश॒या ह॑राश॒या त॒नूर्वर्\mbox{}षि॑ष्ठा गह्वरे॒ष्ठोग्रं वचो॒ अपा॑वधीं त्वे॒षं वचो॒ अपा॑वधी॒ꣴ॒ स्वाहा᳚॥~(२१)

{\anuvakamend[{त्वयि॑ चत्वारि॒ꣳ॒शच्च॑}]}%॥11॥

%1.2.12.1
वि॒त्ताय॑नी मे\-ऽसि ति॒क्ताय॑नी मे॒\-ऽस्यव॑तान्मा नाथि॒तमव॑तान्मा व्यथि॒तं वि॒देर॒ग्निर्नभो॒ नामाग्ने॑ अङ्गिरो॒ यो᳚\-ऽस्यां पृ॑थि॒व्यामस्या\-ऽऽ\-यु॑षा॒ नाम्नेहि॒ यत्ते\-ऽना॑धृष्टं॒ नाम॑ य॒ज्ञियं॒ तेन॒ त्वा\-ऽऽ\-द॒धे\-ऽग्ने॑ अङ्गिरो॒ यो द्वि॒तीय॑स्यां तृ॒तीय॑स्यां पृथि॒व्यामस्या\-ऽऽ\-यु॑षा॒ नाम्नेहि॒ यत्ते\-ऽना॑धृष्टं॒ नाम॑~(२२)

%1.2.12.2
य॒ज्ञियं॒ तेन॒ त्वा\-ऽऽ\-द॑धे सि॒ꣳ॒हीर॑सि महि॒षीर॑स्यु॒रु प्र॑थस्वो॒रु ते॑ य॒ज्ञप॑तिः प्रथतां ध्रु॒वा\-ऽसि॑ दे॒वेभ्यः॑ शुन्धस्व दे॒वेभ्यः॑ शुम्भस्वेन्द्रघो॒षस्त्वा॒ वसु॑भिः पु॒रस्ता᳚त्पातु॒ मनो॑जवास्त्वा पि॒तृभि॑र्दक्षिण॒तः पा॑तु॒ प्रचे॑तास्त्वा रु॒द्रैः प॒श्चात्पा॑तु वि॒श्वक॑र्मा त्वा\-ऽऽ\-दि॒त्यैरु॑त्तर॒तः पा॑तु सि॒ꣳ॒हीर॑सि सपत्नसा॒ही स्वाहा॑ सि॒ꣳ॒हीर॑सि सुप्रजा॒वनिः॒ स्वाहा॑ सि॒ꣳ॒ही-~(२३)

%1.2.12.3
-र॑सि रायस्पोष॒वनिः॒ स्वाहा॑ सि॒ꣳ॒हीर॑स्यादित्य॒वनिः॒ स्वाहा॑ सि॒ꣳ॒हीर॒स्या व॑ह दे॒वान्दे॑वय॒ते यज॑मानाय॒ स्वाहा॑ भू॒तेभ्य॑स्त्वा वि॒श्वायु॑रसि पृथि॒वीं दृꣳ॑ह ध्रुव॒क्षिद॑स्य॒न्तरि॑क्षं दृꣳहाच्युत॒क्षिद॑सि॒ दिवं॑ दृꣳहा॒ग्नेर्भस्मा᳚स्य॒ग्नेः पुरी॑षमसि॥~(२४)

{\anuvakamend[{नाम॑ सुप्रजा॒वनिः॒ स्वाहा॑ सि॒ꣳ॒हीः पञ्च॑त्रिꣳशच्च}]}%॥12॥

%1.2.13.1
यु॒ञ्जते॒ मन॑ उ॒त यु॑ञ्जते॒ धियो॒ विप्रा॒ विप्र॑स्य बृह॒तो वि॑प॒श्चितः॑। वि होत्रा॑ दधे वयुना॒विदेक॒ इन्म॒ही दे॒वस्य॑ सवि॒तुः परि॑ष्टुतिः॥ सु॒वाग्दे॑व॒ दुर्या॒ꣳ॒ आ व॑द देव॒श्रुतौ॑ दे॒वेष्वा घो॑षेथा॒मा नो॑ वी॒रो जा॑यतां कर्म॒ण्यो॑ यꣳ सर्वे॑\-ऽनु॒जीवा॑म॒ यो ब॑हू॒नामस॑द्व॒शी। इ॒दं विष्णु॒र्वि च॑क्रमे त्रे॒धा नि द॑धे प॒दम्। समू॑ढमस्य~(२५)

%1.2.13.2
पाꣳसु॒र इरा॑वती धेनु॒मती॒ हि भू॒तꣳ सू॑यव॒सिनी॒ मन॑वे यश॒स्ये᳚। व्य॑स्कभ्ना॒द्रोद॑सी॒ विष्णु॑रे॒ते दा॒धार॑ पृथि॒वीम॒भितो॑ म॒यूखैः᳚॥ प्राची॒ प्रेत॑मध्व॒रं क॒ल्पय॑न्ती ऊ॒र्ध्वं य॒ज्ञं न॑यतं॒ मा जी᳚ह्वरत॒मत्र॑ रमेथां॒ वर्ष्म॑न्पृथि॒व्या दि॒वो वा॑ विष्णवु॒त वा॑ पृथि॒व्या म॒हो वा॑ विष्णवु॒त वा॒\-ऽन्तरि॑क्षा॒द्धस्तौ॑ पृणस्व ब॒हुभि॑र्वस॒व्यै॑रा प्र य॑च्छ॒~(२६)

%1.2.13.3
दक्षि॑णा॒दोत स॒व्यात्। विष्णो॒र्नुकं॑ वी॒र्या॑णि॒ प्र वो॑चं॒ यः पार्थि॑वानि विम॒मे रजाꣳ॑सि॒ यो अस्क॑भाय॒दुत्त॑रꣳ स॒धस्थं॑ विचक्रमा॒णस्त्रे॒धोरु॑गा॒यो विष्णो॑ र॒राट॑मसि॒ विष्णोः᳚ पृ॒ष्ठम॑सि॒ विष्णोः॒ श्ञप्त्रे᳚ स्थो॒ विष्णोः॒ स्यूर॑सि॒ विष्णो᳚र्ध्रु॒वम॑सि वैष्ण॒वम॑सि॒ विष्ण॑वे त्वा॥~(२७)

{\anuvakamend[{अ॒स्य॒ य॒च्छैका॒न्नच॑त्वारि॒ꣳ॒शच्च॑}]}%॥13॥

%1.2.14.1
कृ॒णु॒ष्व पाजः॒ प्रसि॑तिं॒ न पृ॒थ्वीं या॒हि राजे॒वाम॑वा॒ꣳ॒ इभे॑न। तृ॒ष्वीमनु॒ प्रसि॑तिं द्रूणा॒नो\-ऽस्ता॑सि॒ विध्य॑ र॒क्षस॒स्तपि॑ष्ठैः॥ तव॑ भ्र॒मास॑ आशु॒या प॑त॒न्त्यनु॑ स्पृश धृष॒ता शोशु॑चानः। तपूꣴ॑ष्यग्ने जु॒ह्वा॑ पत॒ङ्गानस॑न्दितो॒ वि सृ॑ज॒ विष्व॑गु॒ल्काः॥ प्रति॒ स्पशो॒ वि सृ॑ज॒ तूर्णि॑तमो॒ भवा॑ पा॒युर्वि॒शो अ॒स्या अद॑ब्धः। यो नो॑ दू॒रे अ॒घशꣳ॑सो॒~(२८)

%1.2.14.2
यो अन्त्यग्ने॒ माकि॑ष्टे॒ व्यथि॒रा द॑धर्षीत्। उद॑ग्ने तिष्ठ॒ प्रत्या\-ऽऽ\-त॑नुष्व॒ न्य॑मित्राꣳ॑ ओषतात्तिग्महेते। यो नो॒ अरा॑तिꣳ समिधान च॒क्रे नी॒चा तं ध॑क्ष्यत॒सं न शुष्कम्᳚॥ ऊ॒र्ध्वो भ॑व॒ प्रति॑ वि॒ध्याध्य॒स्मदा॒विष्कृ॑णुष्व॒ दैव्या᳚न्यग्ने। अव॑ स्थि॒रा त॑नुहि यातु॒जूनां᳚ जा॒मिमजा॑मिं॒ प्र मृ॑णीहि॒ शत्रून्॑॥ स ते॑~(२९)

%1.2.14.3
जानाति सुम॒तिं य॑विष्ठ॒ य ईव॑ते॒ ब्रह्म॑णे गा॒तुमैर॑त्। विश्वा᳚न्यस्मै सु॒दिना॑नि रा॒यो द्यु॒म्नान्य॒र्यो वि दुरो॑ अ॒भि द्यौ᳚त्॥ सेद॑ग्ने अस्तु सु॒भगः॑ सु॒दानु॒र्यस्त्वा॒ नित्ये॑न ह॒विषा॒ य उ॒क्थैः। पिप्री॑षति॒ स्व आयु॑षि दुरो॒णे विश्वेद॑स्मै सु॒दिना॒ सा\-ऽस॑दि॒ष्टिः॥ अर्चा॑मि ते सुम॒तिं घोष्य॒र्वाख्सं ते॑ वा॒वाता॑ जरता-~(३०)

%1.2.14.4
मि॒यङ्गीः। स्वश्वा᳚स्त्वा सु॒रथा॑ मर्जयेमा॒स्मे क्ष॒त्राणि॑ धारये॒रनु॒ द्यून्॥ इ॒ह त्वा॒ भूर्या च॑रे॒दुप॒ त्मन्दोषा॑वस्तर्दीदि॒वाꣳ\-स॒मनु॒ द्यून्। कीड॑न्तस्त्वा सु॒मन॑सः सपेमा॒भि द्यु॒म्ना त॑स्थि॒वाꣳसो॒ जना॑नाम्॥ यस्त्वा॒ स्वश्वः॑ सुहिर॒ण्यो अ॑ग्न उप॒याति॒ वसु॑मता॒ रथे॑न। तस्य॑ त्रा॒ता भ॑वसि॒ तस्य॒ सखा॒ यस्त॑ आति॒थ्यमा॑नु॒षग्जुजो॑षत्॥ म॒हो रु॑जामि~(३१)

%1.2.14.5
ब॒न्धुता॒ वचो॑भि॒स्तन्मा॑ पि॒तुर्गोत॑मा॒दन्वि॑याय॥ त्वं नो॑ अ॒स्य वच॑सश्चिकिद्धि॒ होत॑र्यविष्ठ सुक्रतो॒ दमू॑नाः॥ अस्व॑प्नजस्त॒रण॑यः सु॒शेवा॒ अत॑न्द्रासो\-ऽवृ॒का अश्र॑मिष्ठाः। ते पा॒यवः॑ स॒ध्रिय॑ञ्चो नि॒षद्या\-ऽग्ने॒ तव॑ नः पान्त्वमूर॥ ये पा॒यवो॑ मामते॒यं ते॑ अग्ने॒ पश्य॑न्तो अ॒न्धं दु॑रि॒तादर॑क्षन्। र॒रक्ष॒ तान्थ्सु॒कृतो॑ वि॒श्ववे॑दा॒ दिफ्स॑न्त॒ इद्रि॒पवो॒ ना ह॑~(३२)

%1.2.14.6
देभुः॥ त्वया॑ व॒यꣳ स॑ध॒न्य॑स्त्वोता॒स्तव॒ प्रणी᳚त्यश्याम॒ वाजान्॑। उ॒भा शꣳसा॑ सूदय सत्यताते\-ऽनुष्ठु॒या कृ॑णुह्यह्रयाण॥ अ॒या ते॑ अग्ने स॒मिधा॑ विधेम॒ प्रति॒ स्तोमꣳ॑ श॒स्यमा॑नं गृभाय। दहा॒शसो॑ र॒क्षसः॑ पा॒ह्य॑स्मान्द्रु॒हो नि॒दो\-ऽमि॑त्रमहो अव॒द्यात्॥ र॒क्षो॒हणं॑ वा॒जिन॒मा\-ऽऽ\-जि॑घर्मि मि॒त्रं प्रथि॑ष्ठ॒मुप॑ यामि॒ शर्म॑। शिशा॑नो अ॒ग्निः क्रतु॑भिः॒ समि॑द्धः॒ स नो॒ दिवा॒~(३३)

%1.2.14.7
स रि॒षः पा॑तु॒ नक्तम्᳚॥ वि ज्योति॑षा बृह॒ता भा᳚त्य॒ग्निरा॒विर्विश्वा॑नि कृणुते महि॒त्वा। प्रादे॑वीर्मा॒याः स॑हते दु॒रेवाः॒ शिशी॑ते॒ शृङ्गे॒ रक्ष॑से वि॒निक्षे᳚॥ उ॒त स्वा॒नासो॑ दि॒विष॑न्त्व॒ग्नेस्ति॒ग्मायु॑धा॒ रक्ष॑से॒ हन्त॒वा उ॑। मदे॑ चिदस्य॒ प्ररु॑जन्ति॒ भामा॒ न व॑रन्ते परि॒बाधो॒ अदे॑वीः॥~(३४)

{\anuvakamend[{अ॒घशꣳ॑सः॒ स ते॑ जरताꣳ रुजामि ह॒ दिवैक॑चत्वारिꣳशच्च}]}

\prashnaend{आप॑ उन्द॒न्त्वाकू᳚त्यै॒ दैवी॑मि॒यं ते॒ वस्व्य॑स्य॒ꣳ॒शुना॑ ते॒ सोम॑न्त॒ उदायु॑षा॒ प्र च्य॑वस्वा॒ग्नेरा॑ति॒थ्यम॒ꣳ॒शुरꣳ॑ शुर्वि॒त्ताय॑नी मेसि यु॒ञ्जते॑ कृणु॒ष्व पाज॒श्चतु॑र्दश॥१४॥}{आपो॒ वस्व्य॑सि॒ या तवे॒यङ्गीश्चतु॑स्त्रिꣳशत्॥३४॥}{आप॑ उन्द॒न्त्वदे॑वीः॥}%%१-२
{हरिः॑ ॐ}{॥कृष्ण-यजुर्वेदीय-तैत्तिरीय-संहितायां प्रथमकाण्डे द्वितीयः प्रश्नः समाप्तः॥१-२॥}
%%% END PRASHNA
