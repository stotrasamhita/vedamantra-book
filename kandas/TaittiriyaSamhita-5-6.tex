\chapt{काण्डम् ५}
\sect{षष्ठमः प्रश्नः}\setcounter{anuvakam}{0}
\dnsub{तैत्तिरीयसंहितायां पञ्चमकाण्डे षष्ठमः प्रश्नः}
%5.6.1.1
हिर॑ण्यवर्णाः॒ शुच॑यः पाव॒का यासु॑ जा॒तः क॒श्यपो॒ यास्विन्द्रः॑। अ॒ग्निं या गर्भं॑ दधि॒रे विरू॑पा॒स्ता न॒ आपः॒ शꣴ स्यो॒ना भ॑वन्तु। यासा॒ꣳ॒ राजा॒ वरु॑णो॒ याति॒ मध्ये॑ सत्यानृ॒ते अ॑व॒पश्य॒ञ्जना॑नाम्। म॒धु॒श्चुतः॒ शुच॑यो॒ याः पा॑व॒कास्ता न॒ आपः॒ शꣴ स्यो॒ना भ॑वन्तु। यासां᳚ दे॒वा दि॒वि कृ॒ण्वन्ति॑ भ॒क्षं या अ॒न्तरि॑क्षे बहु॒धा भव॑न्ति। याः पृ॑थि॒वीं पय॑सो॒न्दन्ति॑~(१)\ip

%5.6.1.2
शु॒क्रास्ता न॒ आपः॒ शꣴ स्यो॒ना भ॑वन्तु। शि॒वेन॑ मा॒ चक्षु॑षा पश्यतापः शि॒वया॑ त॒नुवोप॑ स्पृशत॒ त्वचं॑ मे। सर्वाꣳ॑ अ॒ग्नीꣳ र॑फ्सु॒षदो॑ हुवे वो॒ मयि॒ वर्चो॒ बल॒मोजो॒ नि ध॑त्त। यद॒दः स॑म्प्रय॒तीरहा॒वन॑दता ह॒ते। तस्मा॒दा न॒द्यो॑ नाम॑ स्थ॒ ता वो॒ नामा॑नि सिन्धवः। यत्प्रेषि॑ता॒ वरु॑णेन॒ ताः शीभꣳ॑ स॒मव॑ल्गत।~(२)\ip

%5.6.1.3
तदा᳚प्नो॒दिन्द्रो॑ वो य॒तीस्तस्मा॒दापो॒ अनु॑ स्थन। अ॒प॒का॒मꣴ स्यन्द॑माना॒ अवी॑वरत वो॒ हिकम्᳚। इन्द्रो॑ वः॒ शक्ति॑भिर्देवी॒स्तस्मा॒द्वार्णाम॑ वो हि॒तम्। एको॑ दे॒वो अप्य॑तिष्ठ॒थ्स्यन्द॑माना यथाव॒शम्। उदा॑निषुर्म॒हीरिति॒ तस्मा॑दुद॒कमु॑च्यते। आपो॑ भ॒द्रा घृ॒तमिदाप॑ आसुर॒ग्नी\-षोमौ॑ बिभ्र॒त्याप॒ इत्ताः। ती॒व्रो रसो॑ मधु॒पृचा᳚म्~(३)\ip

%5.6.1.4
अ॒र॒ङ्ग॒म आ मा᳚ प्रा॒णेन॑ स॒ह वर्च॑सा गन्न्। आदित्प॑श्याम्यु॒त वा॑ शृणो॒म्या मा॒ घोषो॑ गच्छति॒ वाङ्न॑ आसाम्। मन्ये॑ भेजा॒नो अ॒मृत॑स्य॒ तर्\mbox{}हि॒ हिर॑ण्यवर्णा॒ अतृ॑पं य॒दा वः॑। आपो॒ हि ष्ठा म॑यो॒भुव॒स्ता न॑ ऊ॒र्जे द॑धातन। म॒हे रणा॑य॒ चक्ष॑से। यो वः॑ शि॒वत॑मो॒ रस॒स्तस्य॑ भाजयते॒ह नः॑। उ॒श॒तीरि॑व मा॒तरः॑। तस्मा॒ अरं॑ गमाम वो॒ यस्य॒ क्षया॑य॒ जिन्व॑थ। आपो॑ ज॒नय॑था च नः। दि॒वि श्र॑यस्वा॒न्तरि॑क्षे यतस्व पृथि॒व्या सम्भ॑व ब्रह्मवर्च॒सम॑सि ब्रह्मवर्च॒साय॑ त्वा॥~(४)\ip

{\anuvakamend[{उ॒न्दन्ति॑ स॒मव॑ल्गत मधु॒पृचां᳚ मा॒तरो॒ द्वाविꣳ॑शतिश्च}]}

%5.6.2.1
अ॒पां ग्रहा᳚न्गृह्णात्ये॒तद्वाव रा॑ज॒सूयं॒ यदे॒ते ग्रहाः᳚ स॒वो᳚\-ऽग्निर्व॑रुणस॒वो रा॑ज॒सूय॑मग्निस॒वश्चित्य॒स्ताभ्या॑मे॒व सू॑य॒ते\-ऽथो॑ उ॒भावे॒व लो॒काव॒भि ज॑यति॒ यश्च॑ राज॒सूये॑नेजा॒नस्य॒ यश्चा᳚ग्नि॒चित॒ आपो॑ भव॒न्त्यापो॒ वा अ॒ग्नेर्भ्रातृ॑व्या॒ यद॒पो᳚\-ऽग्नेर॒धस्ता॑दुप॒दधा॑ति॒ भ्रातृ॑व्याभिभूत्यै॒ भव॑त्या॒त्मना॒ परा᳚स्य॒ भ्रातृ॑व्यो भवत्य॒मृतम्᳚~(५)\ip

%5.6.2.2
वा आप॒स्तस्मा॑द॒द्भिरव॑तान्तम॒भि षि॑ञ्चन्ति॒ नार्ति॒मार्च्छ॑ति॒ सर्व॒मायु॑रेति॒ यस्यै॒ता उ॑पधी॒यन्ते॒ य उ॑ चैना ए॒वं वेदान्नं॒ वा आपः॑ प॒शव॒ आपो\-ऽन्नं॑ प॒शवो᳚\-ऽन्ना॒दः प॑शु॒मान्भ॑वति॒ यस्यै॒ता उ॑पधी॒यन्ते॒ य उ॑ चैना ए॒वं वेद॒ द्वाद॑श भवन्ति॒ द्वाद॑श॒ मासाः᳚ संवथ्स॒रः सं॑वथ्स॒रेणै॒वास्मै᳚~(६)\ip

%5.6.2.3
अन्न॒मव॑ रुन्धे॒ पात्रा॑णि भवन्ति॒ पात्रे॒ वा अन्न॑मद्यते॒ सयो᳚न्ये॒वान्न॒मव॑ रुन्ध॒ आ द्वा॑द॒शात्पुरु॑षा॒दन्न॑म॒त्त्यथो॒ पात्रा॒न्न छि॑द्यते॒ यस्यै॒ता उ॑पधी॒यन्ते॒ य उ॑ चैना ए॒वं वेद॑ कु॒म्भाश्च॑ कु॒म्भीश्च॑ मिथु॒नानि॑ भवन्ति मिथु॒नस्य॒ प्रजा᳚त्यै॒ प्र प्र॒जया॑ प॒शुभि॑र्मिथु॒नैर्जा॑यते॒ यस्यै॒ता उ॑पधी॒यन्ते॒ य उ॑~(७)\ip

%5.6.2.4
चै॒ना॒ ए॒वं वेद॒ शुग्वा अ॒ग्निः सो᳚\-ऽध्व॒र्युं यज॑मानं प्र॒जाः शु॒चार्प॑यति॒ यद॒प उ॑प॒दधा॑ति॒ शुच॑मे॒वास्य॑ शमयति॒ नार्ति॒मार्च्छ॑त्यध्व॒र्युर्न यज॑मानः॒ शाम्य॑न्ति प्र॒जा यत्रै॒ता उ॑पधी॒यन्ते॒\-ऽपां वा ए॒तानि॒ हृद॑यानि॒ यदे॒ता आपो॒ यदे॒ता अ॒प उ॑प॒दधा॑ति दि॒व्याभि॑रे॒वैनाः॒ सꣳ सृ॑जति॒ वर्\mbox{}षु॑कः प॒र्जन्यः॑~(८)\ip

%5.6.2.5
भ॒व॒ति॒ यो वा ए॒तासा॑मा॒यत॑नं॒ कॢप्तिं॒ वेदा॒यत॑नवान्भवति॒ कल्प॑ते\-ऽस्मा अनुसी॒तमुप॑ दधात्ये॒तद्वा आ॑सामा॒यत॑नमे॒षा कॢप्ति॒र्य ए॒वं वेदा॒यत॑नवान्भवति॒ कल्प॑ते\-ऽस्मै द्व॒न्द्वम॒न्या उप॑ दधाति॒ चत॑स्रो॒ मध्ये॒ धृत्या॒ अन्नं॒ वा इष्ट॑का ए॒तत्खलु॒ वै सा॒क्षादन्नं॒ यदे॒ष च॒रुर्यदे॒तं च॒रुमु॑प॒दधा॑ति सा॒क्षात्~(९)\ip

%5.6.2.6
ए॒वास्मा॒ अन्न॒मव॑ रुन्धे मध्य॒त उप॑ दधाति मध्य॒त ए॒वास्मा॒ अन्नं॑ दधाति॒ तस्मा᳚न्मध्य॒तो\-ऽन्न॑मद्यते बार्\mbox{}हस्प॒त्यो भ॑वति॒ ब्रह्म॒ वै दे॒वानां॒ बृह॒स्पति॒र्ब्रह्म॑णै॒वास्मा॒ अन्न॒मव॑ रुन्धे ब्रह्मवर्च॒सम॑सि ब्रह्मवर्च॒साय॒ त्वेत्या॑ह तेज॒स्वी ब्र॑ह्मवर्च॒सी भ॑वति॒ यस्यै॒ष उ॑पधी॒यते॒ य उ॑ चैनमे॒वं वेद॑॥~(१०)\ip

{\anuvakamend[{अ॒मृत॑मस्मै जायते॒ यस्यै॒ता उ॑पधी॒यन्ते॒ य उ॑ प॒र्जन्य॑ उप॒दधा॑ति सा॒क्षाथ्स॒प्तच॑त्वारिꣳशच्च}]}

%5.6.3.1
भू॒ते॒ष्ट॒का उप॑ दधा॒त्यत्रा᳚त्र॒ वै मृ॒त्युर्जा॑यते॒ यत्र॑यत्रै॒व मृ॒त्युर्जाय॑ते॒ तत॑ ए॒वैन॒मव॑ यजते॒ तस्मा॑दग्नि॒चिथ्सर्व॒मायु॑रेति॒ सर्वे॒ ह्य॑स्य मृ॒त्यवो\-ऽवे᳚ष्टा॒स्तस्मा॑दग्नि॒चिन्नाभिच॑रित॒वै प्र॒त्यगे॑नमभिचा॒रः स्तृ॑णुते सू॒यते॒ वा ए॒ष यो᳚\-ऽग्निं चि॑नु॒ते दे॑वसु॒वामे॒तानि॑ ह॒वीꣳषि॑ भवन्त्ये॒ताव॑न्तो॒ वै दे॒वानाꣳ॑ स॒वास्त ए॒व~(११)\ip

%5.6.3.2
अ॒स्मै॒ स॒वान्प्र य॑च्छन्ति॒ त ए॑नꣳ सुवन्ते स॒वो᳚\-ऽग्निर्व॑रुणस॒वो रा॑ज॒सूयं॑ ब्रह्मस॒वश्चित्यो॑ दे॒वस्य॑ त्वा सवि॒तुः प्र॑स॒व इत्या॑ह सवि॒तृप्र॑सूत ए॒वैनं॒ ब्रह्म॑णा दे॒वता॑भिर॒भि षि॑ञ्च॒त्यन्न॑स्यान्नस्या॒भि षि॑ञ्च॒त्यन्न॑स्यान्न॒स्याव॑रुद्ध्यै पु॒रस्ता᳚त्प्र॒त्यञ्च॑म॒भि षि॑ञ्चति पु॒रस्ता॒द्धि प्र॑ती॒चीन॒मन्न॑म॒द्यते॑ शीर्\mbox{}ष॒तो॑\-ऽभि षि॑ञ्चति शीर्\mbox{}ष॒तो ह्यन्न॑म॒द्यत॒ आ मुखा॑द॒न्वव॑स्रावयति~(१२)\ip

%5.6.3.3
मु॒ख॒त ए॒वास्मा॑ अ॒न्नाद्यं॑ दधात्य॒ग्नेस्त्वा॒ साम्रा᳚ज्येना॒भि षि॑ञ्चा॒मीत्या॑है॒ष वा अ॒ग्नेः स॒वस्तेनै॒वैन॑म॒भि षि॑ञ्चति॒ बृह॒स्पते᳚स्त्वा॒ साम्रा᳚ज्येना॒भि षि॑ञ्चा॒मीत्या॑ह॒ ब्रह्म॒ वै दे॒वानां॒ बृह॒स्पति॒र्ब्रह्म॑णै॒वैन॑म॒भि षि॑ञ्च॒तीन्द्र॑स्य त्वा॒ साम्रा᳚ज्येना॒भि षि॑ञ्चा॒मीत्या॑हेन्द्रि॒यमे॒वास्मि॑न्नु॒परि॑ष्टाद्दधात्ये॒तत्~(१३)\ip

%5.6.3.4
वै रा॑ज॒सूय॑स्य रू॒पं य ए॒वं वि॒द्वान॒ग्निं चि॑नु॒त उ॒भावे॒व लो॒काव॒भि ज॑यति॒ यश्च॑ राज॒सूये॑नेजा॒नस्य॒ यश्चा᳚ग्नि॒चित॒ इन्द्र॑स्य सुषुवा॒णस्य॑ दश॒धेन्द्रि॒यं वी॒र्यं॑ परा॑पत॒त्तद्दे॒वाः सौ᳚त्राम॒ण्या सम॑भरन्थ्सू॒यते॒ वा ए॒ष यो᳚\-ऽग्निं चि॑नु॒ते᳚\-ऽग्निं चि॒त्वा सौ᳚त्राम॒ण्या य॑जेतेन्द्रि॒यमे॒व वी॒र्यꣳ॑ स॒म्भृत्या॒त्मन्ध॑त्ते॥~(१४)\ip

{\anuvakamend[{त ए॒वान्वव॑स्रावयत्ये॒तद॒ष्टाच॑त्वारिꣳशच्च}]}

%5.6.4.1
स॒जूरब्दो\-ऽया॑वभिः स॒जूरु॒षा अरु॑णीभिः स॒जूः सूर्य॒ एत॑शेन स॒जोषा॑व॒श्विना॒ दꣳसो॑भिः स॒जूर॒ग्निर्वै᳚श्वान॒र इडा॑भिर्घृ॒तेन॒ स्वाहा॑ संवथ्स॒रो वा अब्दो॒ मासा॒ अया॑वा उ॒षा अरु॑णी॒ सूर्य॒ एत॑श इ॒मे अ॒श्विना॑ संवथ्स॒रो᳚\-ऽग्निर्वै᳚श्वान॒रः प॒शव॒ इडा॑ प॒शवो॑ घृ॒तꣳ सं॑वथ्स॒रं प॒शवो\-ऽनु॒ प्र जा॑यन्ते संवथ्स॒रेणै॒वास्मै॑ प॒शून्प्र ज॑नयति दर्भस्त॒म्बे जु॑होति॒ यत्~(१५)\ip

%5.6.4.2
वा अ॒स्या अ॒मृतं॒ यद्वी॒र्यं॑ तद्द॒र्भास्तस्मि॑ञ्जुहोति॒ प्रैव जा॑यते\-ऽन्ना॒दो भ॑वति॒ यस्यै॒वं जुह्व॑त्ये॒ता वै दे॒वता॑ अ॒ग्नेः पु॒रस्ता᳚द्भागा॒स्ता ए॒व प्री॑णा॒त्यथो॒ चक्षु॑रे॒वाग्नेः पु॒रस्ता॒त्प्रति॑ दधा॒त्यन॑न्धो भवति॒ य ए॒वं वेदापो॒ वा इ॒दमग्रे॑ सलि॒लमा॑सी॒थ्स प्र॒जा\-प॑तिः पुष्करप॒र्णे वातो॑ भू॒तो॑\-ऽलेलाय॒थ्सः~(१६)\ip

%5.6.4.3
प्र॒ति॒ष्ठां नावि॑न्द॒त स ए॒तद॒पां कु॒लाय॑मपश्य॒त्तस्मि॑न्न॒ग्निम॑चिनुत॒ तदि॒यम॑भव॒त्ततो॒ वै स प्रत्य॑तिष्ठ॒द्यां पु॒रस्ता॑दु॒पा\-द॑धा॒त्तच्छिरो॑\-ऽभव॒थ्सा प्राची॒ दिग्यां द॑क्षिण॒त उ॒पाद॑धा॒थ्स दक्षि॑णः प॒क्षो॑\-ऽभव॒थ्सा द॑क्षि॒णा दिग्यां प॒श्चादु॒पा\-द॑धा॒त्तत्पुच्छ॑मभव॒थ्सा प्र॒तीची॒ दिग्यामु॑त्तर॒त उ॒पाद॑धात्~(१७)\ip

%5.6.4.4
स उत्त॑रः प॒क्षो॑\-ऽभव॒थ्सोदी॑ची॒ दिग्यामु॒परि॑ष्टादु॒पाद॑धा॒त्तत्पृ॒ष्ठम॑भव॒थ्सोर्ध्वा दिगि॒यं वा अ॒ग्निः पञ्चे᳚ष्टक॒स्तस्मा॒द्यद॒स्यां खन॑न्त्य॒भीष्ट॑कां तृ॒न्दन्त्य॒भि शर्क॑रा॒ꣳ॒ सर्वा॒ वा इ॒यं वयो᳚भ्यो॒ नक्तं॑ दृ॒शे दी᳚प्यते॒ तस्मा॑दि॒मां वयाꣳ॑सि॒ नक्तं॒ नाध्या॑सते॒ य ए॒वं वि॒द्वान॒ग्निं चि॑नु॒ते प्रत्ये॒व~(१८)\ip

%5.6.4.5
ति॒ष्ठ॒त्य॒भि दिशो॑ जयत्याग्ने॒यो वै ब्रा᳚ह्म॒णस्तस्मा᳚द्ब्राह्म॒णाय॒ सर्वा॑सु दि॒क्ष्वर्धु॑क॒ꣴ॒ स्वामे॒व तद्दिश॒मन्वे᳚त्य॒पां वा अ॒ग्निः कु॒लाय॒न्तस्मा॒दापो॒\-ऽग्निꣳ हारु॑काः॒ स्वामे॒व तद्योनिं॒ प्र वि॑शन्ति॥~(१९)\ip

{\anuvakamend[{यद॑लेलाय॒थ्स उ॑त्तर॒त उ॒पाद॑धादे॒व द्वात्रिꣳ॑शच्च}]}

%5.6.5.1
सं॒व॒थ्स॒रमुख्य॑म्भृ॒त्वा द्वि॒तीये॑ संवथ्स॒र आ᳚ग्ने॒यम॒ष्टा\-क॑पालं॒ निर्व॑पेदै॒न्द्रमेका॑\-दश\-कपालं वैश्वदे॒वं द्वाद॑श\-कपालं बार्\mbox{}हस्प॒त्यं च॒रुं वै᳚ष्ण॒वं त्रि॑कपा॒लं तृ॒तीये॑ संवथ्स॒रे॑\-ऽभि॒जिता॑ यजेत॒ यद॒ष्टाक॑पालो॒ भव॑त्य॒ष्टाक्ष॑रा गाय॒त्र्या᳚ग्ने॒यं गा॑य॒त्रं प्रा॑तःसव॒नं प्रा॑तःसव॒नमे॒व तेन॑ दाधार गाय॒त्रं छन्दो॒ यदेका॑\-दश\-कपालो॒ भव॒त्येका॑\-दशाक्षरा त्रि॒ष्टुगै॒न्द्रं त्रैष्टु॑भं॒ माध्य॑न्दिन॒ꣳ॒ सव॑नं॒ माध्य॑न्दिनमे॒व सव॑नं॒ तेन॑ दाधार त्रि॒ष्टुभम्᳚~(२०)\ip

%5.6.5.2
छन्दो॒ यद्द्वाद॑श\-कपालो॒ भव॑ति॒ द्वाद॑शाक्षरा॒ जग॑ती वैश्वदे॒वं जाग॑तं तृतीयसव॒नन्तृ॑तीयसव॒नमे॒व तेन॑ दाधार॒ जग॑तीं॒ छन्दो॒ यद्बा॑र्\mbox{}हस्प॒त्यश्च॒रुर्भव॑ति॒ ब्रह्म॒ वै दे॒वानां॒ बृह॒स्पति॒र्ब्रह्मै॒व तेन॑ दाधार॒ यद्वै᳚ष्ण॒वस्त्रि॑कपा॒लो भव॑ति य॒ज्ञो वै विष्णु॑र्य॒ज्ञमे॒व तेन॑ दाधार॒ यत्तृ॒तीये॑ संवथ्स॒रे॑\-ऽभि॒जिता॒ यज॑ते॒\-ऽभिजि॑त्यै॒ यथ्सं॑वथ्स॒रमुख्यं॑ बि॒भर्ती॒ममे॒व~(२१)\ip

%5.6.5.3
तेन॑ लो॒कꣴ स्पृ॑णोति॒ यद्द्वि॒तीये॑ संवथ्स॒रे᳚\-ऽग्निं चि॑नु॒ते᳚\-ऽन्तरि॑क्षमे॒व तेन॑ स्पृणोति॒ यत्तृ॒तीये॑ संवथ्स॒रे यज॑ते॒\-ऽमुमे॒व तेन॑ लो॒कꣴ स्पृ॑णोत्ये॒तं वै पर॑ आट्णा॒रः क॒क्षीवाꣳ॑ औशि॒जो वी॒तह॑व्यः श्राय॒सस्त्र॒सद॑स्युः पौरुकु॒थ्स्यः प्र॒जाका॑मा अचिन्वत॒ ततो॒ वै ते स॒हस्रꣳ॑सहस्रं पु॒त्रान॑विन्दन्त॒ प्रथ॑ते प्र॒जया॑ प॒शुभि॒स्तां मात्रा॑माप्नोति॒ यां ते\-ऽग॑च्छ॒न्॒ य ए॒वं वि॒द्वाने॒तम॒ग्निं चि॑नु॒ते॥~(२२)\ip

{\anuvakamend[{दा॒धा॒र॒ त्रि॒ष्टुभ॑मि॒ममे॒वैवं च॒त्वारि॑ च}]}

%5.6.6.1
प्र॒जा\-प॑तिर॒ग्निम॑चिनुत॒ स क्षु॒रप॑विर्भू॒त्वाति॑ष्ठ॒त्तं दे॒वा बिभ्य॑तो॒ नोपा॑य॒न्ते छन्दो॑भिरा॒त्मानं॑ छादयि॒त्वोपा॑य॒न्तच्छन्द॑सां छन्द॒स्त्वं ब्रह्म॒ वै छन्दाꣳ॑सि॒ ब्रह्म॑ण ए॒तद्रू॒पं यत्कृ॑ष्णाजि॒नङ्कार्ष्णी॑ उपा॒नहा॒वुप॑ मुञ्चते॒ छन्दो॑भिरे॒वात्मानं॑ छादयि॒त्वाग्निमुप॑ चरत्या॒त्मनो\-ऽहिꣳ॑सायै देवनि॒धिर्वा ए॒ष नि धी॑यते॒ यद॒ग्निः~(२३)\ip

%5.6.6.2
अ॒न्ये वा॒ वै नि॒धिमगु॑प्तं वि॒न्दन्ति॒ न वा॒ प्रति॒ प्र जा॑नात्यु॒खामा क्रा॑मत्या॒त्मान॑मे॒वाधि॒पां कु॑रुते॒ गुप्त्या॒ अथो॒ खल्वा॑हु॒र्नाक्रम्येति॑ नैर्\mbox{}ऋ॒त्यु॑खा यदा॒क्रामे॒न्निर्\mbox{}ऋ॑त्या आ॒त्मान॒मपि॑ दध्या॒त्तस्मा॒न्नाक्रम्या॑ पुरुषशी॒र्॒\mbox{}षमुप॑ दधाति॒ गुप्त्या॒ अथो॒ यथा᳚ ब्रू॒यादे॒तन्मे॑ गोपा॒येति॑ ता॒दृगे॒व तत्~(२४)\ip

%5.6.6.3
प्र॒जा\-प॑ति॒र्वा अथ॑र्वा॒ग्निरे॒व द॒ध्यङ्ङा॑थर्व॒णस्तस्येष्ट॑का अ॒स्थान्ये॒तꣳ ह॒ वाव तदृषि॑र॒भ्यनू॑वा॒चेन्द्रो॑ दधी॒चो अ॒स्थभि॒रिति॒ यदिष्ट॑काभिर॒ग्निं चि॒नोति॒ सात्मा॑नमे॒वाग्निं चि॑नुते॒ सात्मा॒मुष्मिँ॑ल्लो॒के भ॑वति॒ य ए॒वं वेद॒ शरी॑रं॒ वा ए॒तद॒ग्नेर्यच्चित्य॑ आ॒त्मा वै᳚श्वान॒रो यच्चि॒ते वै᳚श्वान॒रं जु॒होति॒ शरी॑रमे॒व स॒ꣴ॒स्कृत्य॑~(२५)\ip

%5.6.6.4
अ॒भ्यारो॑हति॒ शरी॑रं॒ वा ए॒तद्यज॑मानः॒ सꣴस्कु॑रुते॒ यद॒ग्निं चि॑नु॒ते यच्चि॒ते वै᳚श्वान॒रं जु॒होति॒ शरी॑रमे॒व स॒ꣴ॒स्कृत्या॒त्मना॒भ्यारो॑हति॒ तस्मा॒त्तस्य॒ नाव॑ द्यन्ति॒ जीव॑न्ने॒व दे॒वानप्ये॑ति वैश्वान॒र्यर्चा पुरी॑ष॒मुप॑ दधाती॒यं वा अ॒ग्निर्वै᳚श्वान॒रस्तस्यै॒षा चिति॒र्यत्पुरी॑षम॒ग्निमे॒व वै᳚श्वान॒रं चि॑नुत ए॒षा वा अ॒ग्नेः प्रि॒या त॒नूर्यद्वै᳚श्वान॒रः प्रि॒यामे॒वास्य॑ त॒नुव॒मव॑ रुन्धे॥~(२६)\ip

{\anuvakamend[{अ॒ग्निस्तथ्स॒ꣴ॒स्कृत्या॒ग्नेर्दश॑ च}]}

%5.6.7.1
अ॒ग्नेर्वै दी॒क्षया॑ दे॒वा वि॒राज॑माप्नुवन्ति॒स्रो रात्री᳚र्दीक्षि॒तः स्या᳚त् त्रि॒पदा॑ वि॒राड्वि॒राज॑माप्नोति॒ षड्रात्री᳚र्दीक्षि॒तः स्या॒त् षड्वा ऋ॒तवः॑ संवथ्स॒रः सं॑वथ्स॒रो वि॒राड्वि॒राज॑माप्नोति॒ दश॒ रात्री᳚र्दीक्षि॒तः स्या॒द्दशा᳚क्षरा वि॒राड्वि॒राज॑माप्नोति॒ द्वाद॑श॒ रात्री᳚र्दीक्षि॒तः स्या॒द्द्वाद॑श॒ मासाः᳚ संवथ्स॒रः सं॑वथ्स॒रो वि॒राड्वि॒राज॑माप्नोति॒ त्रयो॑दश॒ रात्री᳚र्दीक्षि॒तः स्या॒त् त्रयो॑दश~(२७)\ip

%5.6.7.2
मासाः᳚ संवथ्स॒रः सं॑वथ्स॒रो वि॒राड्वि॒राज॑माप्नोति॒ पञ्च॑दश॒ रात्री᳚र्दीक्षि॒तः स्या॒त्पञ्च॑दश॒ वा अ॑र्धमा॒सस्य॒ रात्र॑यो\-ऽर्धमास॒शः सं॑वथ्स॒र आ᳚प्यते संवथ्स॒रो वि॒राड्वि॒राज॑माप्नोति स॒प्तद॑श॒ रात्री᳚र्दीक्षि॒तः स्या॒द्द्वाद॑श॒ मासाः॒ पञ्च॒र्तवः॒ स सं॑वथ्स॒रः सं॑वथ्स॒रो वि॒राड्वि॒राज॑माप्नोति॒ चतु॑र्विꣳशति॒ꣳ॒ रात्री᳚र्दीक्षि॒तः स्या॒च्चतु॑र्विꣳशतिरर्धमा॒साः सं॑वथ्स॒रः सं॑वथ्स॒रो वि॒राड्वि॒राज॑माप्नोति त्रि॒ꣳ॒शत॒ꣳ॒ रात्री᳚र्दीक्षि॒तः स्या᳚त्~(२८)\ip

%5.6.7.3
त्रि॒ꣳ॒शद॑क्षरा वि॒राड्वि॒राज॑माप्नोति॒ मासं॑ दीक्षि॒तः स्या॒द्यो मासः॒ स सं॑वथ्स॒रः सं॑वथ्स॒रो वि॒राड्वि॒राज॑माप्नोति च॒तुरो॑ मा॒सो दी᳚क्षि॒तः स्या᳚च्च॒तुरो॒ वा ए॒तं मा॒सो वस॑वो\-ऽबिभरु॒स्ते पृ॑थि॒वीमाज॑यन्गाय॒त्रीं छन्दो॒\-ऽष्टौ रु॒द्रास्ते᳚\-ऽन्तरि॑क्ष॒माज॑यन्त्रि॒ष्टुभं॒ छन्दो॒ द्वाद॑शादि॒त्यास्ते दिव॒माज॑य॒ञ्जग॑तीं॒ छन्द॒स्ततो॒ वै ते व्या॒वृत॑मगच्छ॒ञ्छ्रैष्ठ्यं॑ दे॒वानां॒ तस्मा॒द्द्वाद॑श मा॒सो भृ॒त्वाग्निं चि॑न्वीत॒ द्वाद॑श॒ मासाः᳚ संवथ्स॒रः सं॑वथ्स॒रो᳚\-ऽग्निश्चित्य॒स्तस्या॑होरा॒त्राणीष्ट॑का आ॒प्तेष्ट॑कमेनं चिनु॒ते\-ऽथो᳚ व्या॒वृत॑मे॒व ग॑च्छति॒ श्रैष्ठ्यꣳ॑ समा॒नाना᳚म्॥~(२९)\ip

{\anuvakamend[{स्या॒त् त्रयो॑दश त्रि॒ꣳ॒शत॒ꣳ॒ रात्री᳚र्दीक्षि॒तः स्या॒द्वै ते᳚\-ऽष्टाविꣳ॑शतिश्च}]}

%5.6.8.1
सु॒व॒र्गाय॒ वा ए॒ष लो॒काय॑ चीयते॒ यद॒ग्निस्तं यन्नान्वा॒रोहे᳚थ्सुव॒र्गाल्लो॒काद्यज॑मानो हीयेत पृथि॒वीमाक्र॑मिषं प्रा॒णो मा॒ मा हा॑सीद॒न्तरि॑क्ष॒माक्र॑मिषं प्र॒जा मा॒ मा हा॑सी॒द्दिव॒माक्र॑मिष॒ꣳ॒ सुव॑रग॒न्मेत्या॑है॒ष वा अ॒ग्नेर॑न्वारो॒हस्तेनै॒वैन॑\-म॒न्वारो॑हति सुव॒र्गस्य॑ लो॒कस्य॒ सम॑ष्ट्यै॒ यत्प॒क्षस॑म्मितां मिनु॒यात्~(३०)\ip

%5.6.8.2
कनी॑याꣳसं यज्ञक्र॒तुमुपे॑या॒त्पापी॑यस्यस्या॒त्मनः॑ प्र॒जा स्या॒द्वेदि॑सम्मितां मिनोति॒ ज्यायाꣳ॑समे॒व य॑ज्ञक्र॒तुमुपै॑ति॒ नास्या॒त्मनः॒ पापी॑यसी प्र॒जा भ॑वति साह॒स्रं चि॑न्वीत प्रथ॒मं चि॑न्वा॒नः स॒हस्र॑सम्मितो॒ वा अ॒यं लो॒क इ॒ममे॒व लो॒कम॒भि ज॑यति॒ द्विषा॑हस्रं चिन्वीत द्वि॒तीयं॑ चिन्वा॒नो द्विषा॑हस्रं॒ वा अ॒न्तरि॑क्षम॒न्तरि॑क्षमे॒वाभि ज॑यति॒ त्रिषा॑हस्रं चिन्वीत तृ॒तीयं॑ चिन्वा॒नः~(३१)\ip

%5.6.8.3
त्रिषा॑हस्रो॒ वा अ॒सौ लो॒को॑\-ऽमुमे॒व लो॒कम॒भि ज॑यति जानुद॒घ्नं चि॑न्वीत प्रथ॒मं चि॑न्वा॒नो गा॑यत्रि॒यैवेमं लो॒कम॒भ्यारो॑हति नाभिद॒घ्नं चि॑न्वीत द्वि॒तीयं॑ चिन्वा॒नस्त्रि॒ष्टुभै॒वान्तरि॑क्षम॒भ्यारो॑हति ग्रीवद॒घ्नं चि॑न्वीत तृ॒तीयं॑ चिन्वा॒नो जग॑त्यै॒वामुं लो॒कम॒भ्यारो॑हति॒ नाग्निं चि॒त्वा रा॒मामुपे॑यादयो॒नौ रेतो॑ धास्या॒मीति॒ न द्वि॒तीयं॑ चि॒त्वान्यस्य॒ स्त्रियम्᳚~(३२)\ip

%5.6.8.4
उपे॑या॒न्न तृ॒तीयं॑ चि॒त्वा कां च॒नोपे॑या॒द्रेतो॒ वा ए॒तन्नि ध॑त्ते॒ यद॒ग्निं चि॑नु॒ते यदु॑पे॒याद्रेत॑सा॒ व्यृ॑ध्ये॒ताथो॒ खल्वा॑हुरप्रज॒स्यं तद्यन्नोपे॒यादिति॒ यद्रे॑तः॒सिचा॑वुप॒दधा॑ति॒ ते ए॒व यज॑मानस्य॒ रेतो॑ बिभृत॒स्तस्मा॒दुपे॑या॒द्रेत॒सो\-ऽस्क॑न्दाय॒ त्रीणि॒ वाव रेताꣳ॑सि पि॒ता पु॒त्रः पौत्रः॑~(३३)\ip

%5.6.8.5
यद्द्वे रे॑तः॒सिचा॑वुपद॒ध्याद्रेतो᳚\-ऽस्य॒ विच्छि॑न्द्यात्ति॒स्र उप॑ दधाति॒ रेत॑सः॒ सन्त॑त्या इ॒यं वाव प्र॑थ॒मा रे॑तः॒सिग्वाग्वा इ॒यं तस्मा॒त्पश्य॑न्ती॒मां पश्य॑न्ति॒ वाचं॒ वद॑न्तीम॒न्तरि॑क्षं द्वि॒तीया᳚ प्रा॒णो वा अ॒न्तरि॑क्षं॒ तस्मा॒न्नान्तरि॑क्षं॒ पश्य॑न्ति॒ न प्रा॒णम॒सौ तृ॒तीया॒ चक्षु॒र्वा अ॒सौ तस्मा॒त्पश्य॑न्त्य॒मूं पश्य॑न्ति॒ चक्षु॒र्यजु॑षे॒मां च॑~(३४)\ip

%5.6.8.6
अ॒मूं चोप॑ दधाति॒ मन॑सा मध्य॒मामे॒षां लो॒कानां॒ कॢप्त्या॒ अथो᳚ प्रा॒णाना॑मि॒ष्टो य॒ज्ञो भृगु॑भिराशी॒र्दा वसु॑भि॒स्तस्य॑ त इ॒ष्टस्य॑ वी॒तस्य॒ द्रवि॑णे॒ह भ॑क्षी॒येत्या॑ह स्तुतश॒स्त्रे ए॒वैतेन॑ दुहे पि॒ता मा॑त॒रिश्वाच्छि॑द्रा प॒दा धा॒ अच्छि॑द्रा उ॒शिजः॑ प॒दानु॑ तक्षुः॒ सोमो॑ विश्व॒विन्ने॒ता ने॑ष॒द्बृह॒स्पति॑रुक्थाम॒दानि॑ शꣳसिष॒दित्या॑है॒तद्वा अ॒ग्नेरु॒क्थन्तेनै॒वैन॒मनु॑ शꣳसति॥~(३५)\ip

{\anuvakamend[{मि॒नु॒यात्तृ॒तीयं॑ चिन्वा॒नस्त्रियं॒ पौत्र॑श्च॒ वै स॒प्तद॑श च}]}

%5.6.9.1
सू॒यते॒ वा ए॒षो᳚\-ऽग्नी॒नां य उ॒खायां᳚ भ्रि॒यते॒ यद॒धः सा॒दये॒द्गर्भाः᳚ प्र॒पादु॑काः स्यु॒रथो॒ यथा॑ स॒वात्प्र॑त्यव॒रोह॑ति ता॒दृगे॒व तदा॑स॒न्दी सा॑दयति॒ गर्भा॑णां॒ धृत्या॒ अप्र॑पादा॒याथो॑ स॒वमे॒वैनं॑ करोति॒ गर्भो॒ वा ए॒ष यदुख्यो॒ योनिः॑ शि॒क्यं॑ यच्छि॒क्या॑दु॒खां नि॒रूहे॒द्योने॒र्गर्भं॒ निर्\mbox{}ह॑ण्या॒थ्षडु॑द्यामꣳ शि॒क्यं॑ भवति षोढाविहि॒तो वै~(३६)\ip

%5.6.9.2
पुरु॑ष आ॒त्मा च॒ शिर॑श्च च॒त्वार्यङ्गा᳚न्या॒त्मन्ने॒वैनं॑ बिभर्ति प्र॒जा\-प॑ति॒र्वा ए॒ष यद॒ग्निस्तस्यो॒खा चो॒लूख॑लं च॒ स्तनौ॒ ताव॑स्य प्र॒जा उप॑ जीवन्ति॒ यदु॒खां चो॒लूख॑लं चोप॒दधा॑ति॒ ताभ्या॑मे॒व यज॑मानो॒\-ऽमुष्मिँ॑ल्लो॒के᳚\-ऽग्निं दु॑हे संवथ्स॒रो वा ए॒ष यद॒ग्निस्तस्य॑ त्रेधाविहि॒ता इ॑ष्टकाः प्राजाप॒त्या वै᳚ष्ण॒वीः~(३७)\ip

%5.6.9.3
वै॒श्व॒क॒र्म॒णीर॑होरा॒त्राण्ये॒वास्य॑ प्राजाप॒त्या यदुख्यं॑ बि॒भर्ति॑ प्राजाप॒त्या ए॒व तदुप॑ धत्ते॒ यथ्स॒मिध॑ आ॒दधा॑ति वैष्ण॒वा वै वन॒स्पत॑यो वैष्ण॒वीरे॒व तदुप॑ धत्ते॒ यदिष्ट॑काभिर॒ग्निं चि॒नोती॒यं वै वि॒श्वक॑र्मा वैश्वकर्म॒णीरे॒व तदुप॑ धत्ते तस्मा॑दाहुस्त्रि॒वृद॒ग्निरिति॒ तं वा ए॒तं यज॑मान ए॒व चि॑न्वीत॒ यद॑स्या॒न्यश्चि॑नु॒याद्यत्तं दक्षि॑णाभि॒र्न रा॒धये॑द॒ग्निम॑स्य वृञ्जीत॒ यो᳚\-ऽस्या॒ग्निं चि॑नु॒यात्तं दक्षि॑णाभी राधयेद॒ग्निमे॒व तथ्स्पृ॑णोति॥~(३८)\ip

{\anuvakamend[{षो॒ढा॒वि॒हि॒तो वै वै᳚ष्ण॒वीर॒न्यो विꣳ॑श॒तिश्च॑}]}

%5.6.10.1
प्र॒जा\-प॑तिर॒ग्निम॑चिनुत॒र्तुभिः॑ संवथ्स॒रं व॑स॒न्तेनै॒वास्य॑ पूर्वा॒र्धम॑चिनुत ग्री॒ष्मेण॒ दक्षि॑णं प॒क्षं व॒र्॒\mbox{}षाभिः॒ पुच्छꣳ॑ श॒रदोत्त॑रं प॒क्षꣳ हे॑म॒न्तेन॒ मध्यं॒ ब्रह्म॑णा॒ वा अ॑स्य॒ तत्पू᳚र्वा॒र्धम॑चिनुत क्ष॒त्रेण॒ दक्षि॑णं प॒क्षं प॒शुभिः॒ पुच्छं॑ वि॒शोत्त॑रं प॒क्षमा॒शया॒ मध्यं॒ य ए॒वं वि॒द्वान॒ग्निं चि॑नु॒त ऋ॒तुभि॑रे॒वैनं॑ चिनु॒ते\-ऽथो॑ ए॒तदे॒व सर्व॒मव॑~(३९)\ip

%5.6.10.2
रु॒न्धे॒ शृ॒ण्वन्त्ये॑नम॒ग्निं चि॑क्या॒नमत्त्यन्न॒ꣳ॒ रोच॑त इ॒यं वाव प्र॑थ॒मा चिति॒रोष॑धयो॒ वन॒स्पत॑यः॒ पुरी॑षम॒न्तरि॑क्षं द्वि॒तीया॒ वयाꣳ॑सि॒ पुरी॑षम॒सौ तृ॒तीया॒ नक्ष॑त्राणि॒ पुरी॑षं य॒ज्ञश्च॑तु॒र्थी दक्षि॑णा॒ पुरी॑षं॒ यज॑मानः पञ्च॒मी प्र॒जा पुरी॑षं॒ यत् त्रिचि॑तीकं चिन्वी॒त य॒ज्ञं दक्षि॑णामा॒त्मानं॑ प्र॒जाम॒न्तरि॑या॒त्तस्मा॒त्पञ्च॑चितीकश्चेत॒व्य॑ ए॒तदे॒व सर्वꣴ॑ स्पृणोति॒ यत्ति॒स्रश्चित॑यः~(४०)\ip

%5.6.10.3
त्रि॒वृद्ध्य॑ग्निर्यद्द्वे द्वि॒पाद्यज॑मानः॒ प्रति॑ष्ठित्यै॒ पञ्च॒ चित॑यो भवन्ति॒ पाङ्क्तः॒ पुरु॑ष आ॒त्मान॑मे॒व स्पृ॑णोति॒ पञ्च॒ चित॑यो भवन्ति प॒ञ्चभिः॒ पुरी॑षैर॒भ्यू॑हति॒ दश॒ सं प॑द्यन्ते॒ दशा᳚क्षरो॒ वै पुरु॑षो॒ यावा॑ने॒व पुरु॑ष॒स्तꣴ स्पृ॑णो॒त्यथो॒ दशा᳚क्षरा वि॒राडन्नं॑ वि॒राड्वि॒राज्ये॒वान्नाद्ये॒ प्रति॑ तिष्ठति संवथ्स॒रो वै ष॒ष्ठी चिति॑र्\mbox{}ऋ॒तवः॒ पुरी॑ष॒ꣳ॒ षट्चित॑यो भवन्ति॒ षट्पुरी॑षाणि॒ द्वाद॑श॒ सं प॑द्यन्ते॒ द्वाद॑श॒ मासाः᳚ संवथ्स॒रः सं॑वथ्स॒र ए॒व प्रति॑ तिष्ठति॥~(४१)\ip

{\anuvakamend[{अव॒ चित॑यः॒ पुरी॑षं॒ पञ्च॑दश च}]}%॥10॥

%5.6.11.1
रोहि॑तो धू॒म्ररो॑हितः क॒र्कन्धु॑रोहित॒स्ते प्रा॑जाप॒त्या ब॒भ्रुर॑रु॒णब॑भ्रुः॒ शुक॑बभ्रु॒स्ते रौ॒द्राः श्येतः॑ श्येता॒क्षः श्येत॑ग्रीव॒स्ते पि॑तृदेव॒त्या᳚स्ति॒स्रः कृ॒ष्णा व॒शा वा॑रु॒ण्य॑स्ति॒स्रः श्वे॒ता व॒शाः सौ॒र्यो॑ मैत्राबार्\mbox{}हस्प॒त्या धू॒म्रल॑लामास्तूप॒राः॥~(४२)\ip

{\anuvakamend[{}]}%॥11॥

%5.6.12.1
पृश्ञि॑स्तिर॒श्चीन॑पृश्ञिरू॒र्ध्वपृ॑श्ञि॒स्ते मा॑रु॒ताः फ॒ल्गूर्लो॑हितो॒र्णी ब॑ल॒क्षी ताः सा॑रस्व॒त्यः॑ पृष॑ती स्थू॒लपृ॑षती क्षु॒द्रपृ॑षती॒ ता वै᳚श्वदे॒व्य॑स्ति॒स्रः श्या॒मा व॒शाः पौ॒ष्णिय॑स्ति॒स्रो रोहि॑णीर्व॒शा मै॒त्रिय॑ ऐन्द्राबार्\mbox{}हस्प॒त्या अ॑रु॒णल॑लामास्तूप॒राः॥~(४३)\ip

{\anuvakamend[{रोहि॑तः॒ पृश्ञिः॒ षड्विꣳ॑शतिः॒ षड्विꣳ॑शतिः}]}%॥12॥

%5.6.13.1
शि॒ति॒बा॒हुर॒न्यतः॑शितिबाहुः सम॒न्तशि॑तिबाहु॒स्त ऐ᳚न्द्रवाय॒वाः शि॑ति॒रन्ध्रो॒\-ऽन्यतः॑शितिरन्ध्रः सम॒न्तशि॑तिरन्ध्र॒स्ते मै᳚त्रावरु॒णाः शु॒द्धवा॑लः स॒र्वशु॑द्धवालो म॒णिवा॑ल॒स्त आ᳚श्वि॒नास्ति॒स्रः शि॒ल्पा व॒शा वै॑श्वदे॒व्य॑स्ति॒स्रः श्येनीः᳚ परमे॒ष्ठिने॑ सोमापौ॒ष्णाः श्या॒मल॑लामास्तूप॒राः॥~(४४)\ip

{\anuvakamend[{}]}%॥13॥

%5.6.14.1
उ॒न्न॒त ऋ॑ष॒भो वा॑म॒नस्त ऐ᳚न्द्रावरु॒णाः शिति॑ककुच्छितिपृ॒ष्ठः शिति॑भस॒त्त ऐ᳚न्द्राबार्\mbox{}हस्प॒त्याः शिति॒पाच्छि॒त्योष्ठः॑ शिति॒भ्रुस्त ऐ᳚न्द्रावैष्ण॒वास्ति॒स्रः सि॒ध्मा व॒शा वै᳚श्वकर्म॒ण्य॑स्ति॒स्रो धा॒त्रे पृ॑षोद॒रा ऐ᳚न्द्रापौ॒ष्णाः श्येत॑ललामास्तूप॒राः॥~(४५)\ip

{\anuvakamend[{शि॒ति॒बा॒हुरु॑न्न॒तः पञ्च॑विꣳशतिः॒ पञ्च॑विꣳशतिः}]}%॥14॥

%5.6.15.1
क॒र्णास्त्रयो॑ या॒माः सौ॒म्यास्त्रयः॑ श्विति॒ङ्गा अ॒ग्नये॒ यवि॑ष्ठाय॒ त्रयो॑ नकु॒लास्ति॒स्रो रोहि॑णी॒स्त्र्यव्य॒स्ता वसू॑नान्ति॒स्रो॑\-ऽरु॒णा दि॑त्यौ॒ह्य॑स्ता रु॒द्राणाꣳ॑ सोमै॒न्द्रा ब॒भ्रुल॑लामास्तूप॒राः॥~(४६)\ip

{\anuvakamend[{क॒र्णास्त्रयो॑विꣳशतिः}]}%॥15॥

%5.6.16.1
शु॒ण्ठास्त्रयो॑ वैष्ण॒वा अ॑धीलोध॒कर्णा॒स्त्रयो॒ विष्ण॑व उरुक्र॒माय॑ लफ्सु॒दिन॒स्त्रयो॒ विष्ण॑व उरुगा॒याय॒ पञ्चा॑वीस्ति॒स्र आ॑दि॒त्याना᳚न्त्रिव॒थ्सास्ति॒स्रो\-ऽङ्गि॑रसामैन्द्रावैष्ण॒वा गौ॒रल॑लामास्तूप॒राः॥~(४७)\ip

{\anuvakamend[{शु॒ण्ठा विꣳ॑श॒तिः}]}%॥16॥

%5.6.17.1
इन्द्रा॑य॒ राज्ञे॒ त्रयः॑ शितिपृ॒ष्ठा इन्द्रा॑याधिरा॒जाय॒ त्रयः॒ शिति॑ककुद॒ इन्द्रा॑य स्व॒राज्ञे॒ त्रयः॒ शिति॑भसदस्ति॒स्रस्तु॑र्यौ॒ह्यः॑ सा॒ध्याना᳚न्ति॒स्रः प॑ष्ठौ॒ह्यो॑ विश्वे॑षां दे॒वाना॑माग्ने॒न्द्राः कृ॒ष्णल॑लामास्तूप॒राः॥~(४८)\ip

{\anuvakamend[{इन्द्रा॑य॒ राज्ञे॒ द्वाविꣳ॑शतिः}]}%॥17॥

%5.6.18.1
अदि॑त्यै॒ त्रयो॑ रोहितै॒ता इ॑न्द्रा॒ण्यै त्रयः॑ कृष्णै॒ताः कु॒ह्वै᳚ त्रयो॑\-ऽरुणै॒तास्ति॒स्रो धे॒नवो॑ रा॒कायै॒ त्रयो॑\-ऽन॒ड्वाहः॑ सिनीवा॒ल्या आ᳚ग्नावैष्ण॒वा रोहि॑तललामास्तूप॒राः॥~(४९)\ip

{\anuvakamend[{अदि॑त्या अ॒ष्टाद॑श}]}%॥18॥

%5.6.19.1
सौ॒म्यास्त्रयः॑ पि॒शङ्गाः॒ सोमा॑य॒ राज्ञे॒ त्रयः॑ सा॒रङ्गाः᳚ पार्ज॒न्या नभो॑रूपास्ति॒स्रो॑\-ऽजा म॒ल्॒\mbox{}हा इ॑न्द्रा॒ण्यै ति॒स्रो मे॒ष्य॑ आदि॒त्या द्या॑वापृथि॒व्या॑ मा॒लङ्गा᳚स्तूप॒राः॥~(५०)\ip

{\anuvakamend[{सौ॒म्या एका॒न्नविꣳ॑शतिः}]}%॥19॥

%5.6.20.1
वा॒रु॒णास्त्रयः॑ कृ॒ष्णल॑लामा॒ वरु॑णाय॒ राज्ञे॒ त्रयो॒ रोहि॑तोललामा॒ वरु॑णाय रि॒शाद॑से॒ त्रयो॑\-ऽरु॒णल॑लामाः शि॒ल्पास्त्रयो॑ वैश्वदे॒वास्त्रयः॒ पृश्ञ॑यः सर्वदेव॒त्या॑ ऐन्द्रासू॒राः श्येत॑ललामास्तूप॒राः॥~(५१)\ip

{\anuvakamend[{वा॒रु॒णा विꣳ॑श॒तिः}]}%॥20॥

%5.6.21.1
सोमा॑य स्व॒राज्ञे॑\-ऽनोवा॒हाव॑न॒ड्वाहा॑विन्द्रा॒ग्निभ्या॑मोजो॒दाभ्या॒मुष्टा॑राविन्द्रा॒ग्नि\-भ्यां᳚ बल॒दाभ्याꣳ॑ सीरवा॒हाववी॒ द्वे धे॒नू भौ॒मी दि॒ग्भ्यो वड॑बे॒ द्वे धे॒नू भौ॒मी वै॑रा॒जी पु॑रु॒षी द्वे धे॒नू भौ॒मी वा॒यव॑ आरोहणवा॒हाव॑न॒ड्वाहौ॑ वारु॒णी कृ॒ष्णे व॒शे अ॑रा॒ड्यौ॑ दि॒व्यावृ॑ष॒भौ प॑रिम॒रौ॥~(५२)\ip

{\anuvakamend[{सोमा॑य स्व॒राज्ञे॒ चतु॑स्त्रिꣳशत्}]}%॥21॥

%5.6.22.1
एका॑\-दश प्रा॒तर्ग॒व्याः प॒शव॒ आ ल॑भ्यन्ते छग॒लः क॒ल्माषः॑ किकिदी॒विर्वि॑दी॒गय॒स्ते त्वा॒ष्ट्राः सौ॒रीर्नव॑ श्वे॒ता व॒शा अ॑नूब॒न्ध्या॑ भवन्त्याग्ने॒य ऐ᳚न्द्रा॒ग्न आ᳚श्वि॒नस्ते वि॑शालयू॒प आ ल॑भ्यन्ते॥~(५३)\ip

{\anuvakamend[{एका॑\-दश॒ पञ्च॑विꣳशतिः}]}%॥22॥

%5.6.23.1
पि॒शङ्गा॒स्त्रयो॑ वास॒न्ताः सा॒रङ्गा॒स्त्रयो॒ ग्रैष्माः॒ पृष॑न्त॒स्त्रयो॒ वार्\mbox{}षि॑काः॒ पृश्ञ॑य॒स्त्रयः॑ शार॒दाः पृ॑श्ञिस॒क्थास्त्रयो॒ हैम॑न्तिका अवलि॒प्तास्त्रयः॑ शैशि॒राः सं॑वथ्स॒राय॒ निव॑क्षसः~(५४)\ip

{\anuvakamend[{पि॒शङ्गा॑ विꣳश॒तिः}]}%॥23॥

\prashnaend{हिर॑ण्यवर्णा अ॒पां ग्रहा᳚न्भूतेष्ट॒काः स॒जूः सं॑वथ्स॒रं प्र॒जा\-प॑तिः॒ स क्षु॒रप॑विर॒ग्नेर्वै दी॒क्षया॑ सुव॒र्गाय॒ तं यन्न सू॒यते᳚ प्र॒जा\-प॑तिर्\mbox{}ऋ॒तुभी॒ रोहि॑तः॒ पृश्ञिः॑ शितिबा॒हुरु॑न्न॒तः क॒र्णाः शु॒ण्ठा इन्द्रा॒यादि॑त्यै सौ॒म्या वा॑रु॒णाः सोमा॒यैका॑\-दश पि॒शङ्गा॒स्त्रयो॑विꣳशतिः॥२३॥}{हिर॑ण्यवर्णा भूतेष्ट॒काश्छन्दो॒ यत्कनी॑याꣳसन्त्रि॒वृद्ध्य॑ग्निर्वा॑रु॒णाश्चतुः॑पञ्चाशत्॥५४॥}{हिर॑ण्यवर्णा॒ निव॑क्षसः॥}%%५-६
{हरिः॑ ॐ}{॥कृष्ण-यजुर्वेदीय-तैत्तिरीय-संहितायां पञ्चम्काण्डे षष्ठः प्रश्नः समाप्तः॥५-६॥}
%%% END PRASHNA
