% !TeX program = XeLaTeX
% !TeX root = ../vedamantrabook.tex
%ꣳꣳ॑ꣳ॒ꣴꣴ॒ꣴ॑
\setmainfont[Script=Devanagari,Mapping=tex-text,Mapping=devanagarinumerals,AutoFakeBold=2.0]{Siddhanta}
\title{\Huge कृष्णयजुर्वेद-ब्रह्मयज्ञः}
\date{}
\maketitle
\chapt{ब्रह्मयज्ञः}

\sect{सङ्कल्पः}

\textbf{आचमनम्।}

\twolineshloka*
{शुक्लाम्बरधरं विष्णुं शशिवर्णं चतुर्भुजम्}
{प्रसन्नवदनं ध्यायेत् सर्वविघ्नोपशान्तये}

\textbf{प्राणायामः।}

ममोपात्त-समस्त-दुरित-क्षय-द्वारा श्री-परमेश्वर-प्रीत्यर्थं ब्रह्मयज्ञं करिष्ये। ब्रह्मयज्ञेन यक्ष्ये।

\sect{यज्ञः}
विद्यु॑दसि॒ विद्य॑ मे पा॒पमान॑मृ॒तात् स॒त्यमुपै॑मि।

त्रिराचा॑मेत्।\\
द्विः प॑रि॒मृज्य॑।\\
स॒कृदु॑प॒स्पृश्य।\\
शि॒र॒श्चक्षु॑षी॒ नासि॑के॒ श्रोत्रे॒ हृद॑यमा॒लभ्य।

उपस्थं॑ कृ॒त्वा।

ॐ भूः । तत् स॑वि॒तुर्वरे॑॑ण्यम्।\
ॐ भुवः। भर्गो॑ दे॒वस्य॑ धीमहि।\\
ओꣳ सुवः। धियो॒ यो नः॑ प्रचो॒दया॑॑त्।\\
ॐ भूः तत् स॑वि॒तुर्वरे॑॑ण्यम्। भर्गो॑ दे॒वस्य॑ धीमहि।\\
ॐ भुवः। धियो॒ यो नः॑ प्रचो॒दया॑॑त्।\\
ओꣳसुवः। तत् स॑वि॒तुर्वरे॑॑ण्यम्। भर्गो॑ दे॒वस्य॑ धीमहि। धियो॒ यो नः॑ प्रचो॒दया॑॑त्।\\

\dnsub{वेदादयः}
हरिः ॐ। अ॒ग्निमी᳚ळे पु॒रोहि॑तं य॒ज्ञस्य॑ दे॒वमृ॒त्विजम्᳚। होता᳚रं रत्न॒-धात॑मम्॥ हरिः॑ ॐ॥\\

हरिः ॐ। इ॒षेत्वो॒र्जे त्वा॑ वा॒यवः॑ स्थो पा॒यवः॑ स्थ दे॒वो वः॑ सवि॒ता प्रार्प॑यतु॒ श्रेष्ठ॑तमाय॒ कर्म॑णे॥ हरिः॑ ॐ॥ \\

हरिः ॐ। अग्न॒ आया॑हि वी॒तये॑ गृणा॒नो ह॒व्यदा॑तये। नि होता॑ सथ्सि ब॒र्हिषि॑॥ हरिः॑ ॐ॥\\

हरिः ॐ। शन्नो॑ दे॒वीर॒भिष्ट॑य॒ आपो॑ भवन्तु पी॒तये᳚। शं योर॒भिस्र॑वन्तु नः॥ हरिः॑ ॐ॥\\

ॐ भूर्भुवः॒ सुवः॑।

सत्यं तपः श्रद्धायां॑ जुहो॒मि॥

\dnsub{परिधानीया}

ॐ॥ नमो॒ ब्रह्म॑णे॒ नमो॑ अस्त्व॒ग्नये॒ नमः॑ पृथि॒व्यै नम॒ ओष॑धीभ्यः।
नमो॑ वा॒चे नमो॑ वा॒चस्पत॑ये॒ नमो॒ विष्ण॑वे बृह॒ते क॑रोमि॥ (त्रिः)

वृष्टि॑रसि॒ वृश्च॑ मे पा॒प्मान॑मृ॒ताथ्स॒त्यमुपा॑गाम्॥

\sect{देवर्षिपितृ-तर्पणम्}
\vspace{-1ex}
{\scriptsize [उपवीती। सकृत् देवतीर्थेन।]}\\
ब्रह्मादयो ये देवाः तान् देवाꣳस्तर्पयामि।\\
सर्वान् देवाꣳस्तर्पयामि।\\
सर्वदेवगणाꣳस्तर्पयामि।\\
सर्वदेवपत्नीस्तर्पयामि।\\
सर्वदेवगणपत्नीस्तर्पयामि॥

{\scriptsize [निवीती। द्विः। ऋषितीर्थेन।]}\\
कृष्णद्वैपायनादयो ये ऋषयस्तान् ऋषीꣳस्तर्पयामि।\\
सर्वान् ऋषीꣳस्तर्पयामि।\\
सर्वर्षिगणाꣳस्तर्पयामि।\\
सर्वर्षिपत्नीस्तर्पयामि।\\
सर्वर्षिगणपत्नीस्तर्पयामि।\\
प्रजापतिं काण्डऋषिं तर्पयामि।\\
सोमं काण्डऋषिं तर्पयामि।\\
अग्निं काण्डऋषिं तर्पयामि।\\
विश्वान् देवान् काण्डऋषीꣳस्तर्पयामि।
\pagebreak[4]

{\scriptsize [सकृत् देवतीर्थेन।]}\\
साꣳहितीर्देवताः उपनिषदस्तर्पयामि।\\
याज्ञिकीर्देवताः उपनिषदस्तर्पयामि।\\
वारुणीर्देवताः उपनिषदस्तर्पयामि।\\
हव्यवाहं तर्पयामि।\\
विश्वान् देवान् काण्डऋषीꣳस्तर्पयामि।

{\scriptsize [द्विः। ब्रह्मतीर्थेन।]}\\
ब्रह्माणं स्वयम्भुवं तर्पयामि।

{\scriptsize [पुनः ऋषितीर्थेन। द्विः।]}\\
विश्वान् देवान् काण्डऋषीꣳस्तर्पयामि।\\
अरुणान् काण्डऋषीꣳस्तर्पयामि।

{\scriptsize [सकृत् देवतीर्थेन।]}\\
सदसस्पतिं तर्पयामि।\\
ऋग्वेदं तर्पयामि।\\
यजुर्वेदं तर्पयामि।\\
सामवेदं तर्पयामि।\\
अथर्ववेदं तर्पयामि।\\
इतिहासपुराणं तर्पयामि। कल्पं तर्पयामि।

{\scriptsize [प्राचीनावीती। त्रिः। पितृतीर्थेन।]}\\
सोमः पितृमान् यमोऽङ्गिरस्वान् अग्निः कव्यवाहनः इत्यादयो ये पितरस्तान् पितॄꣳस्तर्पयामि।\\
सर्वान् पितॄꣳस्तर्पयामि।\\
सर्वपितृगणाꣳस्तर्पयामि।\\
सर्व-पितृ-पत्नीस्तर्पयामि।\\
सर्व-पितृ-गण-पत्नीस्तर्पयामि।\\
ऊर्जं वहन्ती-रमृतं घृतं पयः कीलालं परिस्रुतं स्वधा स्थ तर्पयत मे पितॄन्। तृप्यत तृप्यत तृप्यत॥

{\scriptsize [उपवीती।]}\\
\dnsub{समर्पणम्}
\vspace{-3ex}
\fourlineindentedshloka*
{कायेन वाचा मनसेन्द्रियैर्वा}
{बुद्‌ध्याऽऽत्मना वा प्रकृतेः स्वभावात्}
{करोमि यद्यत् सकलं परस्मै}
{नारायणायेति समर्पयामि}

\textbf{आचमनम्।}
