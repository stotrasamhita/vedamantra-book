\chapt{काण्डम् १}
\sect{सप्तमः प्रश्नः}\setcounter{anuvakam}{0}
\dnsub{तैत्तिरीयसंहितायां प्रथमकाण्डे सप्तमः प्रश्नः}
%1.7.1.1
पा॒क॒य॒ज्ञं वा अन्वाहि॑ताग्नेः प॒शव॒ उप॑ तिष्ठन्त॒ इडा॒ खलु॒ वै पा॑कय॒ज्ञः सैषा\-ऽन्त॒रा प्र॑याजानूया॒जान् यज॑मानस्य लो॒के\-ऽव॑हिता॒ तामा᳚ह्रि॒यमा॑णाम॒भि म॑न्त्रयेत॒ सुरू॑पवर्\mbox{}षवर्ण॒ एहीति॑ प॒शवो॒ वा इडा॑ प॒शूने॒वोप॑ ह्वयते य॒ज्ञं वै दे॒वा अदु॑ह्रन् य॒ज्ञो\-ऽसु॑राꣳ अदुह॒त् ते\-ऽसु॑रा य॒ज्ञदु॑ग्धाः॒ परा॑\-ऽभव॒न्॒ यो वै य॒ज्ञस्य॒ दोहं॑ वि॒द्वान्~(१)

%1.7.1.2
यज॒ते\-ऽप्य॒न्यं यज॑मानं दुहे॒ सा मे॑ स॒त्या\-ऽऽ\-शीर॒स्य य॒ज्ञस्य॑ भूया॒दित्या॑है॒ष वै य॒ज्ञस्य॒ दोह॒स्तेनै॒वैनं॑ दुहे॒ प्रत्ता॒ वै गौर्दु॑हे॒ प्रत्तेडा॒ यज॑मानाय दुह ए॒ते वा इडा॑यै॒ स्तना॒ इडोप॑हू॒तेति॑ वा॒युर्व॒थ्सो यर्\mbox{}हि॒ होतेडा॑मुप॒ह्वये॑त॒ तर्\mbox{}हि॒ यज॑मानो॒ होता॑र॒मीक्ष॑माणो वा॒युं मन॑सा ध्यायेन्-~(२)

%1.7.1.3
मा॒त्रे व॒थ्समु॒पाव॑सृजति॒ सर्वे॑ण॒ वै य॒ज्ञेन॑ दे॒वाः सु॑व॒र्गं लो॒कमा॑यन् पाकय॒ज्ञेन॒ मनु॑रश्राम्य॒थ्सेडा॒ मनु॑मु॒पाव॑र्तत॒ तान्दे॑वासु॒रा व्य॑ह्वयन्त प्र॒तीचीं᳚ दे॒वाः परा॑ची॒मसु॑राः॒ सा दे॒वानु॒पाव॑र्तत प॒शवो॒ वै तद्दे॒वान॑वृणत प॒शवो\-ऽसु॑रानजहु॒र्यं का॒मये॑ताप॒शुः स्या॒दिति॒ परा॑चीं॒ तस्येडा॒मुप॑ह्वयेताप॒शुरे॒व भ॑वति॒ यं~(३)

%1.7.1.4
का॒मये॑त पशु॒मान्थ्स्या॒दिति॑ प्र॒तीचीं॒ तस्येडा॒मुप॑ह्वयेत पशु॒माने॒व भ॑वति ब्रह्मवा॒दिनो॑ वदन्ति॒ स त्वा इडा॒मुप॑ह्वयेत॒ य इडा॑मुप॒हूया॒\-ऽऽ\-त्मान॒मिडा॑यामुप॒ह्वये॒तेति॒ सा नः॑ प्रि॒या सु॒प्रतू᳚र्तिर्म॒घोनीत्या॒हेडा॑मे॒वोप॒हूया॒\-ऽऽ\-त्मान॒मिडा॑या॒मुप॑ ह्वयते॒ व्य॑स्तमिव॒ वा ए॒तद्य॒ज्ञस्य॒ यदिडा॑ सा॒मि प्रा॒श्ञन्ति॑~(४)

%1.7.1.5
सा॒मि मा᳚र्जयन्त ए॒तत् प्रति॒ वा असु॑राणां य॒ज्ञो व्य॑च्छिद्यत॒ ब्रह्म॑णा दे॒वाः सम॑दधु॒र्बृह॒स्पति॑स्तनुतामि॒मं न॒ इत्या॑ह॒ ब्रह्म॒ वै दे॒वानां॒ बृह॒स्पति॒र्ब्रह्म॑णै॒व य॒ज्ञꣳ सन्द॑धाति॒ विच्छि॑न्नं य॒ज्ञꣳ समि॒मं द॑धा॒त्वित्या॑ह॒ सन्त॑त्यै॒ विश्वे॑ दे॒वा इ॒ह मा॑दयन्ता॒मित्या॑ह स॒न्तत्यै॒व य॒ज्ञं दे॒वेभ्यो\-ऽनु॑ दिशति॒ यां वै~(५)


%1.7.1.6
य॒ज्ञे दक्षि॑णां॒ ददा॑ति॒ ताम॑स्य प॒शवो\-ऽनु॒ सङ्क्रा॑मन्ति॒ स ए॒ष ई॑जा॒नो॑\-ऽप॒शुर्भावु॑को॒ यज॑मानेन॒ खलु॒ वै तत्का॒र्य॑मित्या॑हु॒र्यथा॑ देव॒त्रा द॒त्तं कु॑र्वी॒ता\-ऽऽ\-त्मन् प॒शून् र॒मये॒तेति॒ ब्रध्न॒ पिन्व॒स्वेत्या॑ह य॒ज्ञो वै ब्र॒ध्नो य॒ज्ञमे॒व तन्म॑हय॒त्यथो॑ देव॒त्रैव द॒त्तं कु॑रुत आ॒त्मन् प॒शून् र॑मयते॒ दद॑तो मे॒ मा क्षा॒यीत्या॒हाक्षि॑तिमे॒वोपै॑ति कुर्व॒तो मे॒ मोप॑ दस॒दित्या॑ह भू॒मान॑मे॒वोपै॑ति॥~(६)

{\anuvakamend[{वि॒द्वान्ध्या॑ये॒ द्यं प्रा॒श्ञन्ति॒ यां वै म॒ एका॒न्नविꣳ॑श॒तिश्च॑}]}%~(१)

%1.7.2.1
सꣴश्र॑वा ह सौवर्चन॒सस्तुमि॑ञ्ज॒मौपो॑दितिमुवाच॒ यथ्स॒त्रिणा॒ꣳ॒ होता\-ऽभूः॒ कामिडा॒मुपा᳚\-ह्वथा॒ इति॒ तामुपा᳚ह्व॒ इति॑ होवाच॒ या प्रा॒णेन॑ दे॒वान् दा॒धार॑ व्या॒नेन॑ मनु॒ष्या॑नपा॒नेन॑ पि॒तॄनिति॑ छि॒नत्ति॒ सा न छि॑न॒त्ती~(३) इति॑ छि॒नत्तीति॑ होवाच॒ शरी॑रं॒ वा अ॑स्यै॒ तदुपा᳚ह्वथा॒ इति॑ होवाच॒ गौर्वा~(७)

%1.7.2.2
अ॑स्यै॒ शरी॑रं॒ गां वाव तौ तत्पर्य॑वदतां॒ या य॒ज्ञे दी॒यते॒ सा प्रा॒णेन॑ दे॒वान् दा॑धार॒ यया॑ मनु॒ष्या॑ जीव॑न्ति॒ सा व्या॒नेन॑ मनु॒ष्यान्॑ यां पि॒तृभ्यो॒ घ्नन्ति॒ सा\-ऽपा॒नेन॑ पि॒तॄन् य ए॒वं वेद॑ पशु॒मान् भ॑व॒त्यथ॒ वै तामुपा᳚ह्व॒ इति॑ होवाच॒ या प्र॒जाः प्र॒भव॑न्तीः॒ प्रत्या॒भव॒तीत्यन्नं॒ वा अ॑स्यै॒ त-~(८)

%1.7.2.3
दुपा᳚ह्वथा॒ इति॑ होवा॒चौष॑धयो॒ वा अ॑स्या॒ अन्न॒मोष॑धयो॒ वै प्र॒जाः प्र॒भव॑न्तीः॒ प्रत्या भ॑वन्ति॒ य ए॒वं वेदा᳚न्ना॒दो भ॑व॒त्यथ॒ वै तामुपा᳚ह्व॒ इति॑ होवाच॒ या प्र॒जाः प॑रा॒भव॑न्तीरनुगृ॒ह्णाति॒ प्रत्या॒भव॑न्तीर्गृ॒ह्णातीति॑ प्रति॒ष्ठां वा अ॑स्यै॒ तदुपा᳚ह्वथा॒ इति॑ होवाचे॒यं वा अ॑स्यै प्रति॒ष्ठे-~(९)

%1.7.2.4
यं वै प्र॒जाः प॑रा॒भव॑न्ती॒रनु॑गृह्णाति॒ प्रत्या॒भव॑न्तीर्गृह्णाति॒ य ए॒वं वेद॒ प्रत्ये॒व ति॑ष्ठ॒त्यथ॒ वै तामुपा᳚ह्व॒ इति॑ होवाच॒ यस्यै॑ नि॒क्रम॑णे घृ॒तं प्र॒जाः स॒ञ्जीव॑न्तीः॒ पिब॒न्तीति॑ छि॒नत्ति॒ सा न छि॑न॒त्ती~(३) इति॒ न छि॑न॒त्तीति॑ होवाच॒ प्र तु ज॑नय॒तीत्ये॒ष वा इडा॒मुपा᳚ह्वथा॒ इति॑ होवाच॒ वृष्टि॒र्वा इडा॒ वृष्ट्यै॒ वै नि॒क्रम॑णे घृ॒तं प्र॒जाः स॒ञ्जीव॑न्तीः पिबन्ति॒ य ए॒वं वेद॒ प्रैव जा॑यते\-ऽन्ना॒दो भ॑वति॥~(१०)

{\anuvakamend[{गौर्वा अ॑स्यै॒ तत् प्र॑ति॒ष्ठा\-ऽह्व॑था॒ इति॑ विꣳश॒तिश्च॑}]}%~(२)

%1.7.3.1
प॒रोक्षं॒ वा अ॒न्ये दे॒वा इ॒ज्यन्ते᳚ प्र॒त्यक्ष॑म॒न्ये यद्यज॑ते॒ य ए॒व दे॒वाः प॒रोक्ष॑मि॒ज्यन्ते॒ ताने॒व तद्य॑जति॒ यद॑न्वाहा॒र्य॑मा॒हर॑त्ये॒ते वै दे॒वाः प्र॒त्यक्षं॒ यद् ब्रा᳚ह्म॒णास्ताने॒व तेन॑ प्रीणा॒त्यथो॒ दक्षि॑णै॒वास्यै॒षा\-ऽथो॑ य॒ज्ञस्यै॒व छि॒द्रमपि॑ दधाति॒ यद्वै य॒ज्ञस्य॑ क्रू॒रं यद्विलि॑ष्टं॒ तद॑न्वाहा॒र्ये॑णा॒-~(११)

%1.7.3.2
न्वाह॑रति॒ तद॑न्वाहा॒र्य॑स्यान्वाहार्य॒त्वं दे॑वदू॒ता वा ए॒ते यदृ॒त्विजो॒ यद॑न्वाहा॒र्य॑मा॒हर॑ति देवदू॒ताने॒व प्री॑णाति प्र॒जा\-प॑तिर्दे॒वेभ्यो॑ य॒ज्ञान् व्यादि॑श॒थ्स रि॑रिचा॒नो॑\-ऽमन्यत॒ स ए॒तम॑न्वाहा॒र्य॑मभ॑क्तमपश्य॒त् तमा॒त्मन्न॑धत्त॒ स वा ए॒ष प्रा॑जाप॒त्यो यद॑न्वाहा॒र्यो॑ यस्यै॒वं वि॒दुषो᳚\-ऽन्वाहा॒र्य॑ आह्रि॒यते॑ सा॒क्षादे॒व प्र॒जा\-प॑तिमृध्नो॒त्यप॑रिमितो नि॒रुप्यो\-ऽप॑रिमितः प्र॒जा\-प॑तिः प्र॒जा\-प॑ते॒-~(१२)

%1.7.3.3
राप्त्यै॑ दे॒वा वै यद्य॒ज्ञे\-ऽकु॑र्वत॒ तदसु॑रा अकुर्वत॒ ते दे॒वा ए॒तं प्रा॑जाप॒त्यम॑न्वाहा॒र्य॑मपश्य॒न् तम॒न्वाह॑रन्त॒ ततो॑ दे॒वा अभ॑व॒न् परासु॑रा॒ यस्यै॒वं वि॒दुषो᳚\-ऽन्वाहा॒र्य॑ आह्रि॒यते॒ भव॑त्या॒त्मना॒ परा᳚स्य॒ भ्रातृ॑व्यो भवति य॒ज्ञेन॒ वा इ॒ष्टी प॒क्वेन॑ पू॒र्ती यस्यै॒वं वि॒दुषो᳚\-ऽन्वाहा॒र्य॑ आह्रि॒यते॒ स त्वे॑वेष्टा॑पू॒र्ती प्र॒जा\-प॑तेर्भा॒गो॑\-\mbox{\-ऽसी-~(१३)}

%1.7.3.4
त्या॑ह प्र॒जा\-प॑तिमे॒व भा॑ग॒धेये॑न॒ सम॑र्धय॒त्यूर्ज॑स्वा॒न् पय॑स्वा॒नित्या॒होर्ज॑मे॒वास्मि॒न् पयो॑ दधाति प्राणापा॒नौ मे॑ पाहि समानव्या॒नौ मे॑ पा॒हीत्या॑हा॒\-ऽऽ\-शिष॑मे॒वैतामा शा॒स्ते\-ऽक्षि॑तो॒\-ऽस्यक्षि॑त्यै त्वा॒ मा मे᳚ क्षेष्ठा अ॒मुत्रा॒मुष्मिँ॑ल्लो॒क इत्या॑ह॒ क्षीय॑ते॒ वा अ॒मुष्मिँ॑ल्लो॒के\-ऽन्न॑मि॒तः प्र॑दान॒ꣴ॒ ह्य॑मुष्मिँ॑ल्लो॒के प्र॒जा उ॑प॒जीव॑न्ति॒ यदे॒वम॑भिमृ॒शत्यक्षि॑तिमे॒वैन॑द्गमयति॒ नास्या॒मुष्मिँ॑ल्लो॒के\-ऽन्नं॑ क्षीयते॥~(१४)

{\anuvakamend[{अ॒न्वा॒हा॒र्ये॑ण प्र॒जा\-प॑तेरसि॒ ह्य॑मुष्मिँ॑ल्लो॒के पञ्च॑दश च}]}%~(३)

%1.7.4.1
ब॒र्॒\mbox{}हिषो॒\-ऽहं दे॑वय॒ज्यया᳚ प्र॒जावा᳚न् भूयास॒मित्या॑ह ब॒र्॒\mbox{}हिषा॒ वै प्र॒जा\-प॑तिः प्र॒जा अ॑सृजत॒ तेनै॒व प्र॒जाः सृ॑जते॒ नरा॒शꣳस॑स्या॒हं दे॑वय॒ज्यया॑ पशु॒मान् भू॑यास॒मित्या॑ह॒ नरा॒शꣳसे॑न॒ वै प्र॒जा\-प॑तिः प॒शून॑सृजत॒ तेनै॒व प॒शून्थ्सृ॑जते॒\-ऽग्नेः स्वि॑ष्ट॒कृतो॒\-ऽहं दे॑वय॒ज्यया\-ऽऽ\-यु॑ष्मान् य॒ज्ञेन॑ प्रति॒ष्ठां ग॑मेय॒मित्या॒हा\-ऽऽ\-यु॑रे॒वा\-ऽऽ\-त्मन् ध॑त्ते॒ प्रति॑ य॒ज्ञेन॑ तिष्ठति दर्\mbox{}श\-पूर्ण\-मा॒सयो॒र्-~(१५)

%1.7.4.2
र्वै दे॒वा उज्जि॑ति॒मनूद॑जयन् दर्\mbox{}श\-पूर्ण\-मा॒साभ्या॒मसु॑रा॒नपा॑\-नुदन्ता॒ग्नेर॒ह\-मुज्जि॑ति॒\-मनूज्जे॑ष॒मित्या॑ह दर्\mbox{}श\-पूर्ण\-मा॒सयो॑रे॒व दे॒वता॑नां॒ यज॑मान॒ उज्जि॑ति॒मनूज्ज॑यति दर्\mbox{}श\-पूर्ण\-मा॒साभ्यां॒ भ्रातृ॑व्या॒नप॑ नुदते॒ वाज॑वतीभ्यां॒ व्यू॑ह॒त्यन्नं॒ वै वाजो\-ऽन्न॑मे॒वाव॑\-रुन्धे॒ द्वाभ्यां॒ प्रति॑ष्ठित्यै॒ यो वै य॒ज्ञस्य॒ द्वौ दोहौ॑ वि॒द्वान् यज॑त उभ॒यत॑~-~(१६)

%1.7.4.3
ए॒व य॒ज्ञं दु॑हे पु॒रस्ता᳚च्चो॒परि॑ष्टाच्चै॒ष वा अ॒न्यो य॒ज्ञस्य॒ दोह॒ इडा॑याम॒न्यो यर्\mbox{}हि॒ होता॒ यज॑मानस्य॒ नाम॑ गृह्णी॒यात् तर्\mbox{}हि॑ ब्रूया॒देमा अ॑ग्मन्ना॒शिषो॒ दोह॑कामा॒ इति॒ सꣴस्तु॑ता ए॒व दे॒वता॑ दु॒हे\-ऽथो॑ उभ॒यत॑ ए॒व य॒ज्ञं दु॑हे पु॒रस्ता᳚च्चो॒परि॑ष्टाच्च॒ रोहि॑तेन त्वा॒\-ऽग्निर्दे॒वतां᳚ गमय॒त्वित्या॑है॒ते वै दे॑वा॒श्वा~-~(१७)

%1.7.4.4
यज॑मानः प्रस्त॒रो यदे॒तैः प्र॑स्त॒रं प्र॒हर॑ति देवा॒श्वैरे॒व यज॑मानꣳ सुव॒र्गं लो॒कं ग॑मयति॒ वि ते॑ मुञ्चामि रश॒ना वि र॒श्मीनित्या॑है॒ष वा अ॒ग्नेर्वि॑मो॒कस्तेनै॒वैनं॒ वि मु॑ञ्चति॒ विष्णोः᳚ शं॒योर॒हं दे॑वय॒ज्यया॑ य॒ज्ञेन॑ प्रति॒ष्ठां ग॑मेय॒मित्या॑ह य॒ज्ञो वै विष्णु॑र्य॒ज्ञ ए॒वान्त॒तः प्रति॑ तिष्ठति॒ सोम॑स्या॒हं दे॑वय॒ज्यया॑ सु॒रेता॒~-~(१८)

%1.7.4.5
रेतो॑ धिषी॒येत्या॑ह॒ सोमो॒ वै रे॑तो॒धास्तेनै॒व रेत॑ आ॒त्मन् ध॑त्ते॒ त्वष्टु॑र॒हं दे॑वय॒ज्यया॑ पशू॒नाꣳ रू॒पं पु॑षेय॒मित्या॑ह॒ त्वष्टा॒ वै प॑शू॒नां मि॑थु॒नानाꣳ॑ रूप॒कृत्तेनै॒व प॑शू॒नाꣳ रू॒पमा॒त्मन् ध॑त्ते दे॒वानां॒ पत्नी॑र॒ग्निर्गृ॒हप॑तिर्य॒ज्ञस्य॑ मिथु॒नं तयो॑र॒हं दे॑वय॒ज्यया॑ मिथु॒नेन॒ प्र भू॑यास॒मित्या॑है॒तस्मा॒द्वै मि॑थु॒नात्प्र॒जा\-प॑तिर्मिथु॒नेन॒~(१९)

%1.7.4.6
प्राजा॑यत॒ तस्मा॑दे॒व यज॑मानो मिथु॒नेन॒ प्र जा॑यते वे॒दो॑\-ऽसि॒ वित्ति॑रसि वि॒देयेत्या॑ह वे॒देन॒ वै दे॒वा असु॑राणां वि॒त्तं वेद्य॑मविन्दन्त॒ तद्वे॒दस्य॑ वेद॒त्वं यद्य॒द् भ्रातृ॑व्यस्याभि॒ध्याये॒त् तस्य॒ नाम॑ गृह्णीया॒त् तदे॒वास्य॒ सर्वं॑ वृङ्क्ते घृ॒तव॑न्तं कुला॒यिनꣳ॑ रा॒यस्पोषꣳ॑ सह॒स्रिणं॑ वे॒दो द॑दातु वा॒जिन॒मित्या॑ह॒ प्र स॒हस्रं॑ प॒शूना᳚प्नो॒त्यास्य॑ प्र॒जायां᳚ वा॒जी जा॑यते॒ य ए॒वं वेद॑॥~(२०)

{\anuvakamend[{द॒र्॒\mbox{}श॒पू॒र्ण॒मा॒सयो॑रुभ॒यतो॑ देवा॒श्वाः सु॒रेताः᳚ प्र॒जा\-प॑तिर्मिथु॒नेना᳚\-ऽऽ\-प्नोत्य॒ष्टौ च॑}]}%~(४)

%1.7.5.1
ध्रु॒वां वै रिच्य॑मानां य॒ज्ञो\-ऽनु॑ रिच्यते य॒ज्ञं यज॑मानो॒ यज॑मानं प्र॒जा ध्रु॒वामा॒प्याय॑मानां य॒ज्ञो\-ऽन्वा प्या॑यते य॒ज्ञं यज॑मानो॒ यज॑मानं प्र॒जा आ प्या॑यतां ध्रु॒वा घृ॒तेनेत्या॑ह ध्रु॒वामे॒वा\-ऽऽ\-प्या॑ययति॒ तामा॒प्याय॑मानां य॒ज्ञो\-ऽन्वा प्या॑यते य॒ज्ञं यज॑मानो॒ यज॑मानं प्र॒जाः प्र॒जा\-प॑तेर्वि॒भान्नाम॑ लो॒कस्तस्मिꣴ॑ स्त्वा दधामि स॒ह यज॑माने॒ने-~(२१)

%1.7.5.2
त्या॑हा॒यं वै प्र॒जा\-प॑तेर्वि॒भान्नाम॑ लो॒कस्तस्मि॑न्ने॒वैनं॑ दधाति स॒ह यज॑मानेन॒ रिच्य॑त इव॒ वा ए॒तद्यद्यज॑ते॒ यद्य॑जमानभा॒गं प्रा॒श्ञात्या॒त्मान॑मे॒व प्री॑णात्ये॒तावा॒न्॒ वै य॒ज्ञो यावान्॑ यजमानभा॒गो य॒ज्ञो यज॑मानो॒ यद्य॑जमानभा॒गं प्रा॒श्ञाति॑ य॒ज्ञ ए॒व य॒ज्ञं प्रति॑ष्ठापयत्ये॒तद्वै सू॒यव॑स॒ꣳ॒ सोद॑कं॒ यद्ब॒र्॒\mbox{}हिश्चा\-ऽऽ\-प॑श्चै॒तद्~(२२)

%1.7.5.3
यज॑मानस्या॒\-ऽऽ\-यत॑नं॒ यद्वेदि॒र्यत् पू᳚र्णपा॒त्रम॑न्तर्वे॒दि नि॒नय॑ति॒ स्व ए॒वा\-ऽऽ\-यत॑ने सू॒यव॑स॒ꣳ॒ सोद॑कं कुरुते॒ सद॑सि॒ सन्मे॑ भूया॒ इत्या॒हा\-ऽऽ\-पो॒ वै य॒ज्ञ आपो॒\-ऽमृतं॑ य॒ज्ञमे॒वामृत॑मा॒त्मन्ध॑त्ते॒ सर्वा॑णि॒ वै भू॒तानि॑ व्र॒तमु॑प॒यन्त॒मनूप॑ यन्ति॒ प्राच्यां᳚ दि॒शि दे॒वा ऋ॒त्विजो॑ मार्जयन्ता॒मित्या॑है॒ष वै द॑र्\mbox{}शपूर्णमा॒सयो॑रवभृ॒थो~(२३)

%1.7.5.4
यान्ये॒वैनं॑ भू॒तानि॑ व्र॒तमु॑प॒यन्त॑मनूप॒यन्ति॒ तैरे॒व स॒हाव॑भृ॒थमवै॑ति॒ विष्णु॑मुखा॒ वै दे॒वाश्छन्दो॑भिरि॒माँल्लो॒कान॑नप\-ज॒य्यम॒भ्य॑जय॒न्॒ यद्वि॑ष्णुक्र॒मान् क्रम॑ते॒ विष्णु॑रे॒व भू॒त्वा यज॑मान॒श्छन्दो॑भि\-रि॒माँल्लो॒का\-न॑नप\-ज॒य्य\-म॒भि ज॑यति॒ विष्णोः॒ क्रमो᳚\-ऽस्यभिमाति॒हेत्या॑ह गाय॒त्री वै पृ॑थि॒वी त्रैष्टु॑भम॒न्तरि॑क्षं॒ जाग॑ती॒ द्यौरानु॑ष्टुभी॒र्दिश॒श्छन्दो॑\-भिरे॒वेमाँल्लो॒कान् य॑थापू॒र्वम॒भि ज॑यति॥~(२४)

{\anuvakamend[{इत्ये॒तद॑वभृ॒थो दिशः॑ स॒प्त च॑}]}%~(५)

%1.7.6.1
अग॑न्म॒ सुवः॒ सुव॑रग॒न्मेत्या॑ह सुव॒र्गमे॒व लो॒कमे॑ति स॒न्दृश॑स्ते॒ मा छि॑थ्सि॒ यत्ते॒ तप॒स्तस्मै॑ ते॒ मा वृ॒क्षीत्या॑ह यथाय॒जुरे॒वैतथ्सु॒भूर॑सि॒ श्रेष्ठो॑ रश्मी॒नामा॑यु॒र्धा अ॒स्यायु॑र्मे धे॒हीत्या॑हा॒\-ऽऽ\-शिष॑मे॒वैतामा शा᳚स्ते॒ प्र वा ए॒षो᳚\-ऽस्माल्लो॒काच्च्य॑वते॒ यो~(२५)

%1.7.6.2
वि॑ष्णुक्र॒मान् क्रम॑ते सुव॒र्गाय॒ हि लो॒काय॑ विष्णुक्र॒माः क्र॒म्यन्ते᳚ ब्रह्मवा॒दिनो॑ वदन्ति॒ स त्वै वि॑ष्णुक्र॒मान् क्र॑मेत॒ य इ॒माँल्लो॒कान् भ्रातृ॑व्यस्य सं॒विद्य॒ पुन॑रि॒मं लो॒कं प्र॑त्यव॒रोहे॒दित्ये॒ष वा अ॒स्य लो॒कस्य॑ प्रत्यवरो॒हो यदाहे॒दम॒हम॒मुं भ्रातृ॑व्यमा॒भ्यो दि॒ग्भ्यो᳚\-ऽस्यै दि॒व इती॒माने॒व लो॒कान्भ्रातृ॑व्यस्य सं॒विद्य॒ पुन॑रि॒मं लो॒कं प्र॒त्यव॑रोहति॒ सं~(२६)

%1.7.6.3
ज्योति॑षा\-ऽभूव॒मित्या॑हा॒स्मिन्ने॒व लो॒के प्रति॑ तिष्ठत्यै॒न्द्रीमा॒\-वृत॑म॒न्वाव॑र्त॒ इत्या॑हा॒सौ वा आ॑दि॒त्य इन्द्र॒स्तस्यै॒वा\-ऽऽ\-\-वृत॒मनु॑ प॒र्याव॑र्तते दक्षि॒णा प॒र्याव॑र्तते॒ स्वमे॒व वी॒र्य॑मनु॑ प॒र्याव॑र्तते॒ तस्मा॒द्दक्षि॒णो\-ऽर्ध॑ आ॒त्मनो॑ वी॒र्या॑वत्त॒रो\-ऽथो॑ आदि॒त्यस्यै॒वा\-ऽऽ\-वृत॒मनु॑ प॒र्याव॑र्तते॒ सम॒हं प्र॒जया॒ सं मया᳚ प्र॒जेत्या॑हा॒\-ऽऽ\-शिष॑-~(२७)

%1.7.6.4
मे॒वैतामा शा᳚स्ते॒ समि॑द्धो अग्ने मे दीदिहि समे॒द्धा ते॑ अग्ने दीद्यास॒मित्या॑ह यथाय॒जुरे॒वैतद्वसु॑मान् य॒ज्ञो वसी॑यान् भूयास॒मित्या॑हा॒\-ऽऽ\-शिष॑मे॒वैतामा शा᳚स्ते ब॒हु वै गार्\mbox{}ह॑पत्य॒स्यान्ते॑ मि॒श्रमि॑व चर्यत आग्निपावमा॒नीभ्यां॒ गार्\mbox{}ह॑पत्य॒मुप॑ तिष्ठते पु॒नात्ये॒वाग्निं पु॑नी॒त आ॒त्मानं॒ द्वाभ्यां॒ प्रति॑ष्ठित्या॒ अग्ने॑ गृहपत॒ इत्या॑ह~(२८)

%1.7.6.5
यथाय॒जुरे॒वैतच्छ॒तꣳ हिमा॒ इत्या॑ह श॒तं त्वा॑ हेम॒न्तानि॑न्धिषी॒येति॒ वावैतदा॑ह पु॒त्रस्य॒ नाम॑ गृह्णात्यन्ना॒दमे॒वैनं॑ करोति॒ तामा॒शिष॒मा शा॑से॒ तन्त॑वे॒ ज्योति॑ष्मती॒मिति॑ ब्रूया॒द्यस्य॑ पु॒त्रो\-ऽजा॑तः॒ स्यात् ते॑ज॒स्व्ये॑वास्य॑ ब्रह्मवर्च॒सी पु॒त्रो जा॑यते॒ तामा॒शिष॒मा शा॑से॒\-ऽमुष्मै॒ ज्योति॑ष्मती॒मिति॑ ब्रूया॒द्यस्य॑ पु॒त्रो~(२९)

%1.7.6.6
जा॒तः स्यात् तेज॑ ए॒वास्मि॑न् ब्रह्मवर्च॒सं द॑धाति॒ यो वै य॒ज्ञं प्र॒युज्य॒ न वि॑मु॒ञ्चत्य॑प्रतिष्ठा॒नो वै स भ॑वति॒ कस्त्वा॑ युनक्ति॒ स त्वा॒ वि मु॑ञ्च॒त्वित्या॑ह प्र॒जा\-प॑ति॒र्वै कः प्र॒जा\-प॑तिनै॒वैनं॑ यु॒नक्ति॑ प्र॒जा\-प॑तिना॒ वि मु॑ञ्चति॒ प्रति॑ष्ठित्या ईश्व॒रं वै व्र॒तमवि॑सृष्टं प्र॒दहो\-ऽग्ने᳚ व्रतपते व्र॒तम॑चारिष॒मित्या॑ह व्र॒तमे॒व~(३०)

%1.7.6.7
वि सृ॑जते॒ शान्त्या॒ अप्र॑दाहाय॒ परा॒ङ्॒ वाव य॒ज्ञ ए॑ति॒ न नि व॑र्तते॒ पुन॒र्यो वै य॒ज्ञस्य॑ पुनराल॒म्भं वि॒द्वान् यज॑ते॒ तम॒भि नि व॑र्तते य॒ज्ञो ब॑भूव॒ स आ ब॑भू॒वेत्या॑है॒ष वै य॒ज्ञस्य॑ पुनराल॒म्भस्तेनै॒वैनं॒ पुन॒राल॑भ॒ते\-ऽन॑वरुद्धा॒ वा ए॒तस्य॑ वि॒राड्य आहि॑ताग्निः॒ सन्न॑स॒भः प॒शवः॒ खलु॒ वै ब्रा᳚ह्म॒णस्य॑ स॒भेष्ट्वा प्राङु॒त्क्रम्य॑ ब्रूया॒द्गोमाꣳ॑ अ॒ग्ने\-ऽवि॑माꣳ अ॒श्वी य॒ज्ञ इत्यव॑ स॒भाꣳ रु॒न्धे प्र स॒हस्रं॑ प॒शूना᳚प्नो॒त्यास्य॑ प्र॒जायां᳚ वा॒जी जा॑यते॥~(३१)

{\anuvakamend[{यः स मा॒शिषं॑ गृहपत॒ इत्या॑ह॒ यस्य॑ पु॒त्रो व्र॒तमे॒व खलु॒ वै चतु॑र्विꣳशतिश्च}]}%~(६)

%1.7.7.1
देव॑ सवितः॒ प्रसु॑व य॒ज्ञं प्रसु॑व य॒ज्ञप॑तिं॒ भगा॑य दि॒व्यो ग॑न्ध॒र्वः। के॒त॒पूः केतं॑ नः पुनातु वा॒चस्पति॒र्वाच॑म॒द्य स्व॑दाति नः॥ इन्द्र॑स्य॒ वज्रो॑\-ऽसि॒ वार्त्र॑घ्न॒स्त्वया॒\-ऽयं वृ॒त्रं व॑ध्यात्॥ वाज॑स्य॒ नु प्र॑स॒वे मा॒तरं॑ म॒हीमदि॑तिं॒ नाम॒ वच॑सा करामहे। यस्या॑मि॒दं विश्वं॒ भुव॑नमावि॒वेश॒ तस्यां᳚ नो दे॒वः स॑वि॒ता धर्म॑ साविषत्॥ अ॒-~(३२)

%1.7.7.2
फ्स्व॑न्तर॒मृत॑म॒फ्सु भे॑ष॒जम॒पामु॒त प्रश॑स्ति॒ष्वश्वा॑ भवथ वाजिनः॥ वा॒युर्वा᳚ त्वा॒ मनु॑र्वा त्वा गन्ध॒र्वाः स॒प्तविꣳ॑शतिः। ते अग्रे॒ अश्व॑मायुञ्ज॒न्ते अ॑स्मिञ्ज॒वमाद॑धुः॥ अपां᳚ नपादाशुहेम॒न्॒ य ऊ॒र्मिः क॒कुद्मा॒न् प्रतू᳚र्तिर्वाज॒सात॑म॒स्तेना॒यं वाजꣳ॑ सेत्॥ विष्णोः॒ क्रमो॑\-ऽसि॒ विष्णोः᳚ क्रा॒न्तम॑सि॒ विष्णो॒र्विक्रा᳚न्तमस्य॒ङ्कौ न्य॒ङ्काव॒भितो॒ रथं॒ यौ ध्वा॒न्तं वा॑ता॒ग्रमनु॑ स॒ञ्चर॑न्तौ दू॒रेहे॑तिरिन्द्रि॒यावा᳚न्पत॒त्री ते नो॒\-ऽग्नयः॒ पप्र॑यः पारयन्तु॥~(३३)

{\anuvakamend[{अ॒फ्सु न्य॒ङ्कौ पञ्च॑दश च}]}%~(७)

%1.7.8.1
दे॒वस्या॒हꣳ स॑वि॒तुः प्र॑स॒वे बृह॒स्पति॑ना वाज॒जिता॒ वाजं॑ जेषं दे॒वस्या॒हꣳ स॑वि॒तुः प्र॑स॒वे बृह॒स्पति॑ना वाज॒जिता॒ वर्\mbox{}षि॑ष्ठं॒ नाकꣳ॑ रुहेय॒मिन्द्रा॑य॒ वाचं॑ वद॒तेन्द्रं॒ वाजं॑ जापय॒तेन्द्रो॒ वाज॑मजयित्। अश्वा॑जनि वाजिनि॒ वाजे॑षु वाजिनीव॒त्यश्वा᳚न्थ्स॒मथ्सु॑ वाजय॥ अर्वा॑\-ऽसि॒ सप्ति॑रसि वा॒ज्य॑सि॒ वाजि॑नो॒ वाजं॑ धावत म॒रुतां᳚ प्रस॒वे ज॑यत॒ वि योज॑ना मिमीध्व॒मध्व॑नः स्कभ्नीत॒~(३४)

%1.7.8.2
काष्ठां᳚ गच्छत॒ वाजे॑वाजे\-ऽवत वाजिनो नो॒ धने॑षु विप्रा अमृता ऋतज्ञाः॥ अ॒स्य मध्वः॑ पिबत मा॒दय॑ध्वं तृ॒प्ता या॑त प॒थिभि॑र्देव॒यानैः᳚॥ ते नो॒ अर्व॑न्तो हवन॒श्रुतो॒ हवं॒ विश्वे॑ शृण्वन्तु वा॒जिनः॑॥ मि॒तद्र॑वः सहस्र॒सा मे॒धसा॑ता सनि॒ष्यवः॑। म॒हो ये रत्नꣳ॑ समि॒थेषु॑ जभ्रि॒रे शं नो॑ भवन्तु वा॒जिनो॒ हवे॑षु॥ दे॒वता॑ता मि॒तद्र॑वः स्व॒र्काः। ज॒म्भय॒न्तो\-ऽहिं॒ वृक॒ꣳ॒ रक्षाꣳ॑सि॒ सने᳚म्य॒स्मद्यु॑यव॒न्न-~(३५)

%1.7.8.3
मी॑वाः॥ ए॒ष स्य वा॒जी क्षि॑प॒णिं तु॑रण्यति ग्री॒वायां᳚ ब॒द्धो अ॑पिक॒क्ष आ॒सनि॑। क्रतुं॑ दधि॒क्रा अनु॑ स॒न्तवी᳚त्वत् प॒थामङ्का॒ꣴ॒स्यन्वा॒पनी॑फणत्॥ उ॒त स्मा᳚स्य॒ द्रव॑तस्तुरण्य॒तः प॒र्णं न वेरनु॑ वाति प्रग॒र्धिनः॑। श्ये॒नस्ये॑व॒ ध्रज॑तो अङ्क॒सं परि॑ दधि॒क्राव्ण्णः॑ स॒होर्जा तरि॑त्रतः॥ आ मा॒ वाज॑स्य प्रस॒वो ज॑गम्या॒दा द्यावा॑\-पृथि॒वी वि॒श्वश॑म्भू। आ मा॑ गन्तां पि॒तरा॑~(३६)

%1.7.8.4
मा॒तरा॒ चा\-ऽऽ\-मा॒ सोमो॑ अमृत॒त्वाय॑ गम्यात्॥ वाजि॑नो वाजजितो॒ वाजꣳ॑ सरि॒ष्यन्तो॒ वाजं॑ जे॒ष्यन्तो॒ बृह॒स्पते᳚र्भा॒गमव॑ जिघ्रत॒ वाजि॑नो वाजजितो॒ वाजꣳ॑ ससृ॒वाꣳसो॒ वाजं॑ जिगि॒वाꣳसो॒ बृह॒स्पते᳚र्भा॒गे नि मृ॑ढ्वमि॒यं वः॒ सा स॒त्या स॒न्धा\-ऽभू॒द्यामिन्द्रे॑ण स॒मध॑ध्व॒मजी॑जिपत वनस्पतय॒ इन्द्रं॒ वाजं॒ विमु॑च्यध्वम्॥~(३७)

{\anuvakamend[{स्क॒भ्नी॒त॒ यु॒य॒व॒न्पि॒तरा॒ द्विच॑त्वारिꣳशच्च}]}%~(८)

%1.7.9.1
क्ष॒त्रस्योल्ब॑मसि क्ष॒त्रस्य॒ योनि॑रसि॒ जाय॒ एहि॒ सुवो॒ रोहा॑व॒ रोहा॑व॒ हि सुव॑र॒हं ना॑वु॒भयोः॒ सुवो॑ रोक्ष्यामि॒ वाज॑श्च प्रस॒वश्चा॑पि॒जश्च॒ क्रतु॑श्च॒ सुव॑श्च मू॒र्धा च॒ व्यश्ञि॑यश्चा\-ऽऽ\-न्त्याय॒नश्चान्त्य॑श्च भौव॒नश्च॒ भुव॑न॒श्चाधि॑\-पतिश्च। आयु॑र्य॒ज्ञेन॑ कल्पतां प्रा॒णो य॒ज्ञेन॑ कल्पतामपा॒नो~-~(३८)

%1.7.9.2
य॒ज्ञेन॑ कल्पतां व्या॒नो य॒ज्ञेन॑ कल्पतां॒ चक्षु॑र्य॒ज्ञेन॑ कल्पता॒ꣴ॒ श्रोत्रं॑ य॒ज्ञेन॑ कल्पतां॒ मनो॑ य॒ज्ञेन॑ कल्पतां॒ वाग्य॒ज्ञेन॑ कल्पतामा॒त्मा य॒ज्ञेन॑ कल्पतां य॒ज्ञो य॒ज्ञेन॑ कल्पता॒ꣳ॒ सुव॑र्दे॒वाꣳ अ॑गन्मा॒मृता॑ अभूम प्र॒जा\-प॑तेः प्र॒जा अ॑भूम॒ सम॒हं प्र॒जया॒ सं मया᳚ प्र॒जा सम॒हꣳ रा॒यस्पोषे॑ण॒ सं मया॑ रा॒यस्पोषो\-ऽन्ना॑य त्वा॒\-ऽन्नाद्या॑य त्वा॒ वाजा॑य त्वा वाजजि॒त्यायै᳚ त्वा॒\-ऽमृत॑मसि॒ पुष्टि॑रसि प्र॒जन॑नमसि॥~(३९)

{\anuvakamend[{अ॒पा॒नो वाजा॑य॒ नव॑ च}]}%~(९)

%1.7.10.1
वाज॑स्ये॒मं प्र॑स॒वः सु॑षुवे॒ अग्रे॒ सोम॒ꣳ॒ राजा॑न॒मोष॑धीष्व॒फ्सु। ता अ॒स्मभ्यं॒ मधु॑मतीर्भवन्तु व॒यꣳ रा॒ष्ट्रे जा᳚ग्रियाम पु॒रोहि॑ताः॥ वाज॑स्ये॒दं प्र॑स॒व आ ब॑भूवे॒मा च॒ विश्वा॒ भुव॑नानि स॒र्वतः॑। स वि॒राजं॒ पर्ये॑ति प्रजा॒नन् प्र॒जां पुष्टिं॑ व॒र्धय॑मानो अ॒स्मे॥ वाज॑स्ये॒मां प्र॑स॒वः शि॑श्रिये॒ दिव॑मि॒मा च॒ विश्वा॒ भुव॑नानि स॒म्राट्। अदि॑थ्सन्तं दापयतु प्रजा॒नन् र॒यिं~(४०)

%1.7.10.2
च॑ नः॒ सर्व॑वीरां॒ नि य॑च्छतु॥ अग्ने॒ अच्छा॑ वदे॒ह नः॒ प्रति॑ नः सु॒मना॑ भव। प्र णो॑ यच्छ भुवस्पते धन॒दा अ॑सि न॒स्त्वम्॥ प्र णो॑ यच्छत्वर्य॒मा प्र भगः॒ प्र बृह॒स्पतिः॑। प्र दे॒वाः प्रोत सू॒नृता॒ प्र वाग्दे॒वी द॑दातु नः॥ अ॒र्य॒मणं॒ बृह॒स्पति॒मिन्द्रं॒ दाना॑य चोदय। वाचं॒ विष्णु॒ꣳ॒ सर॑स्वतीꣳ सवि॒तारं॑~(४१)

%1.7.10.3
च वा॒जिनम्᳚॥ सोम॒ꣳ॒ राजा॑नं॒ वरु॑णम॒ग्निम॒न्वार॑भामहे। आ॒दि॒त्यान् विष्णु॒ꣳ॒ सूर्यं॑ ब्र॒ह्माणं॑ च॒ बृह॒स्पतिम्᳚॥ दे॒वस्य॑ त्वा सवि॒तुः प्र॑स॒वे᳚\-ऽश्विनो᳚र्बा॒हु\-भ्यां᳚ पू॒ष्णो हस्ता᳚भ्या॒ꣳ॒ सर॑स्वत्यै वा॒चो य॒न्तुर्य॒न्त्रेणा॒ग्नेस्त्वा॒ साम्रा᳚ज्येना॒भिषि॑ञ्चा॒मीन्द्र॑स्य॒ बृह॒स्पते᳚स्त्वा॒ साम्रा᳚ज्येना॒भिषि॑ञ्चामि॥~(४२)

{\anuvakamend[{र॒यिꣳ स॑वि॒तार॒ꣳ॒ षट्त्रिꣳ॑शच्च}]}%॥10॥

%1.7.11.1
अ॒ग्निरेका᳚क्षरेण॒ वाच॒मुद॑जयद॒श्विनौ॒ द्व्य॑क्षरेण प्राणा\-पा॒ना\-वुद॑\-जयतां॒ विष्णु॒स्त्र्य॑क्षरेण॒ त्रीँल्लो॒कानुद॑जय॒थ्सोम॒श्चतु॑रक्षरेण॒ चतु॑ष्पदः प॒शूनुद॑जयत् पू॒षा पञ्चा᳚क्षरेण प॒ङ्क्तिमुद॑जयद्धा॒ता षड॑क्षरेण॒ षडृ॒तूनुद॑जयन्म॒रुतः॑ स॒प्ताक्ष॑रेण स॒प्तप॑दा॒ꣳ॒ शक्व॑री॒मुद॑जय॒न् बृह॒स्पति॑र॒ष्टाक्ष॑रेण गाय॒त्रीमुद॑जयन्मि॒त्रो नवा᳚क्षरेण त्रि॒वृत॒ꣴ॒ स्तोम॒मुद॑जय॒-~(४३)

%1.7.11.2
द्वरु॑णो॒ दशा᳚क्षरेण वि॒राज॒मुद॑जय॒दिन्द्र॒ एका॑\-दशा\-क्षरेण त्रि॒ष्टुभ॒मुद॑जय॒द् विश्वे॑ दे॒वा द्वाद॑शाक्षरेण॒ जग॑ती॒मुद॑जय॒न् वस॑व॒स्त्रयो॑\-दशा\-क्षरेण त्रयोद॒शꣴस्तोम॒मुद॑जयन् रु॒द्राश्चतु॑र्दशा\-क्षरेण चतुर्द॒शꣴ स्तोम॒मुद॑जयन्नादि॒त्याः पञ्च॑\-दशा\-क्षरेण पञ्चद॒शꣴ स्तोम॒मुद॑जय॒न्नदि॑तिः॒ षोड॑शाक्षरेण षोड॒शꣴ स्तोम॒मुद॑जयत् प्र॒जा\-प॑तिः स॒प्तद॑शाक्षरेण सप्तद॒शꣴ स्तोम॒मुद॑जयत्॥~(४४)

{\anuvakamend[{अ॒ज॒य॒त् षट्च॑त्वारिꣳशच्च}]}%॥11॥

%1.7.12.1
उ॒प॒या॒मगृ॑हीतो\-ऽसि नृ॒षदं॑ त्वा द्रु॒षदं॑ भुवन॒सद॒मिन्द्रा॑य॒ जुष्टं॑ गृह्णाम्ये॒ष ते॒ योनि॒रिन्द्रा॑य त्वोपया॒मगृ॑हीतो\-ऽस्यफ्सु॒षदं॑ त्वा घृत॒सदं॑ व्योम॒सद॒मिन्द्रा॑य॒ जुष्टं॑ गृह्णाम्ये॒ष ते॒ योनि॒रिन्द्रा॑य त्वोपया॒मगृ॑हीतो\-ऽसि पृथिवि॒षदं॑ त्वा\-ऽन्तरिक्ष॒सदं॑ नाक॒सद॒मिन्द्रा॑य॒ जुष्टं॑ गृह्णाम्ये॒ष ते॒ योनि॒रिन्द्रा॑य त्वा॥ ये ग्रहाः᳚ पञ्चज॒नीना॒ येषां᳚ ति॒स्रः प॑रम॒जाः। दैव्यः॒ कोशः॒~(४५)

%1.7.12.2
समु॑ब्जितः। तेषां॒ विशि॑प्रियाणा॒मिष॒मूर्ज॒ꣳ॒ सम॑ग्रभीमे॒ष ते॒ योनि॒रिन्द्रा॑य त्वा॥ अ॒पाꣳ रस॒मुद्व॑यस॒ꣳ॒ सूर्य॑रश्मिꣳ स॒माभृ॑तम्। अ॒पाꣳ रस॑स्य॒ यो रस॒स्तं वो॑ गृह्णाम्युत्त॒ममे॒ष ते॒ योनि॒रिन्द्रा॑य त्वा॥ अ॒या वि॒ष्ठा ज॒नय॒न्कर्व॑राणि॒ स हि घृणि॑रु॒रुर्वरा॑य गा॒तुः। स प्रत्युदै᳚द्ध॒रुणो॒ मध्वो॒ अग्र॒ꣴ॒ स्वायां॒ यत्त॒नुवां᳚ त॒नूमैर॑यत। उ॒प॒या॒मगृ॑हीतो\-ऽसि प्र॒जा\-प॑तये त्वा॒ जुष्टं॑ गृह्णाम्ये॒ष ते॒ योनिः॑ प्र॒जा\-प॑तये त्वा॥~(४६)

{\anuvakamend[{कोश॑स्त॒नुवां॒ त्रयो॑दश च}]}%॥12॥

%1.7.13.1
अन्वह॒ मासा॒ अन्विद्वना॒न्यन्वोष॑धी॒रनु॒ पर्व॑तासः। अन्विन्द्र॒ꣳ॒ रोद॑सी वावशा॒ने अन्वापो॑ अजिहत॒ जाय॑मानम्॥ अनु॑ ते दायि म॒ह इ॑न्द्रि॒याय॑ स॒त्रा ते॒ विश्व॒मनु॑ वृत्र॒हत्ये᳚। अनु॑ क्ष॒त्रमनु॒ सहो॑ यज॒त्रेन्द्र॑ दे॒वेभि॒रनु॑ ते नृ॒षह्ये᳚॥ इ॒न्द्रा॒णीमा॒सु नारि॑षु सु॒पत्नी॑म॒हम॑श्रवम्। न ह्य॑स्या अप॒रं च॒न ज॒रसा॒~(४७)

%1.7.13.2
मर॑ते॒ पतिः॑॥ नाहमि॑न्द्राणि रारण॒ सख्यु॑र्वृ॒षाक॑पेर्‌ऋ॒ते। यस्ये॒दमप्यꣳ॑ ह॒विः प्रि॒यं दे॒वेषु॒ गच्छ॑ति॥ यो जा॒त ए॒व प्र॑थ॒मो मन॑स्वान् दे॒वो दे॒वान् क्रतु॑ना प॒र्यभू॑षत्। यस्य॒ शुष्मा॒द्रोद॑सी॒ अभ्य॑सेतां नृ॒म्णस्य॑ म॒ह्ना स ज॑नास॒ इन्द्रः॑॥ आ ते॑ म॒ह इ॑न्द्रो॒त्यु॑ग्र॒ सम॑न्यवो॒ यथ्स॒मर॑न्त॒ सेनाः᳚। पता॑ति दि॒द्युन्नर्य॑स्य बाहु॒वोर्मा ते॒~(४८)

%1.7.13.3
मनो॑ विष्व॒द्रिय॒ग्विचा॑रीत्॥ मा नो॑ मर्धी॒रा भ॑रा द॒द्धि तन्नः॒ प्र दा॒शुषे॒ दात॑वे॒ भूरि॒ यत् ते᳚। नव्ये॑ दे॒ष्णे श॒स्ते अ॒स्मिन् त॑ उ॒क्थे प्र ब्र॑वाम व॒यमि॑न्द्र स्तु॒वन्तः॑॥ आ तू भ॑र॒ माकि॑रे॒तत् परि॑ष्ठाद्वि॒द्मा हि त्वा॒ वसु॑पतिं॒ वसू॑नाम्। इन्द्र॒ यत् ते॒ माहि॑नं॒ दत्र॒मस्त्य॒स्मभ्यं॒ तद्ध॑र्यश्व॒~(४९)

%1.7.13.4
प्र य॑न्धि॥ प्र॒दा॒तारꣳ॑ हवामह॒ इन्द्र॒मा ह॒विषा॑ व॒यम्। उ॒भा हि हस्ता॒ वसु॑ना पृ॒णस्वा\-ऽऽ\-प्र य॑च्छ॒ दक्षि॑णा॒दोत स॒व्यात्॥ प्र॒दा॒ता व॒ज्री वृ॑ष॒भस्तु॑रा॒षाट्छु॒ष्मी राजा॑ वृत्र॒हा सो॑म॒पावा᳚। अ॒स्मिन् य॒ज्ञे ब॒र्॒\mbox{}हिष्या नि॒षद्याथा॑ भव॒ यज॑मानाय॒ शं योः॥ इन्द्रः॑ सु॒त्रामा॒ स्ववा॒ꣳ॒ अवो॑भिः सुमृडी॒को भ॑वतु वि॒श्ववे॑दाः। बाध॑तां॒ द्वेषो॒ अभ॑यं कृणोतु सु॒वीर्य॑स्य॒~(५०)

%1.7.13.5
पत॑यः स्याम॥ तस्य॑ व॒यꣳ सु॑म॒तौ य॒ज्ञिय॒स्यापि॑ भ॒द्रे सौ॑मन॒से स्या॑म। स सु॒त्रामा॒ स्ववा॒ꣳ॒ इन्द्रो॑ अ॒स्मे आ॒राच्चि॒द्द्वेषः॑ सनु॒तर्यु॑योतु॥ रे॒वती᳚र्नः सध॒माद॒ इन्द्रे॑ सन्तु तु॒विवा॑जाः। क्षु॒मन्तो॒ याभि॒र्मदे॑म॥ प्रो ष्व॑स्मै पुरोर॒थमिन्द्रा॑य शू॒षम॑र्चत। अ॒भीके॑ चिदु लोक॒कृथ्स॒ङ्गे स॒मथ्सु॑ वृत्र॒हा। अ॒स्माकं॑ बोधि चोदि॒ता नभ॑न्तामन्य॒केषा᳚म्। ज्या॒का अधि॒ धन्व॑सु॥~(५१)

{\anuvakamend[{ज॒रसा॒ मा ते॑ हर्यश्व सु॒वीर्य॒स्याध्येकं॑ च}]}%॥13॥

\prashnaend{पा॒क॒य॒ज्ञꣳ सग्ग् श्र॑वाः प॒रोक्षं॑ ब॒र्हिषो॒हं ध्रु॒वामग॒न्मेत्या॑ह॒ देव॑ सवितर्दे॒वस्या॒हं क्ष॒त्रस्योल्बं॒ वाज॑स्ये॒मम॒ग्निरेका᳚क्षरेणोऽपया॒म गृ॑हीतो॒ऽस्यन्वह॒ मासा॒स्त्रयो॑दश॥१३॥}{पा॒क॒य॒ज्ञं प॒रोक्षं॑ ध्रु॒वां वि सृ॑जते च नः॒ सर्व॑वीरां॒ पत॑यः स्या॒मैक॑पञ्चा॒शत्॥५१॥}{पा॒क॒य॒ज्ञं धन्व॑सु॥}%%१-७
{हरिः॑ ॐ}{॥कृष्ण-यजुर्वेदीय-तैत्तिरीय-संहितायां प्रथमकाण्डे सप्तमः प्रश्नः समाप्तः॥१-७॥}
%%% END PRASHNA
