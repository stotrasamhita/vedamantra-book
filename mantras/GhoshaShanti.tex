% !TeX program = XeLaTeX
% !TeX root = ../vedamantrabook.tex


\chapt{घोष-शान्तिः}

शं नो॒ वातः॑ पवतां मात॒रिश्वा॒ शं न॑स्तपतु॒ सूर्यः॑। अहा॑नि॒शं भ॑वन्तु नः॒ शꣳ रात्रिः॒ प्रति॑धीयताम्। शमु॒षा नो॒ व्यु॑च्छतु॒ शमा॑दि॒त्य उदे॑तु नः। शि॒वा नः॒ शन्त॑मा भव सुमृडी॒का सर॑स्वति। मा ते॒ व्यो॑म स॒न्दृशि॑। इडा॑यै॒ वास्त्व॑सि वास्तु॒मद्वा᳚स्तु॒मन्तो॑ भूयास्म॒ मा वास्तो᳚श्छिथ्स्मह्यवा॒स्तुः स भू॑या॒द्यो᳚ऽस्मान्द्वेष्टि॒ यं च॑ व॒यं द्वि॒ष्मः। प्र॒ति॒ष्ठासि॑ प्रति॒ष्ठाव॑न्तो भूयास्म॒ मा प्र॑ति॒ष्ठाया᳚श्छिथ्स्मह्यप्रति॒ष्ठः स भू॑या॒द्यो᳚ऽस्मान्द्वेष्टि॒ यं च॑ व॒यं द्वि॒ष्मः। आ वा॑त वाहि भेष॒जं वि वा॑त वाहि॒ यद्रपः॑। त्वꣳ हि वि॒श्वभे॑षजो दे॒वानां᳚ दू॒त ईय॑से। द्वावि॒मौ वातौ॑ वात॒ आ सिन्धो॒रा प॑रा॒वतः॑॥

%७.४२.२
दक्षं॑ मे अ॒न्य आ॒वातु॒ परा॒न्यो वा॑तु॒ यद्रपः॑। यद॒दो वा॑तते गृ॒हे॑ऽमृत॑स्य नि॒धिर्\mbox{}हि॒तः। ततो॑ नो देहि जी॒वसे॒ ततो॑ नो धेहि भेष॒जम्। ततो॑ नो॒ मह॒ आव॑ह॒ वात॒ आवा॑तु भेष॒जम्। श॒म्भूर्म॑यो॒भूर्नो॑ हृ॒दे प्र ण॒ आयूꣳ॑षि तारिषत्। इन्द्र॑स्य गृ॒हो॑ऽसि॒ तं त्वा॒ प्रप॑द्ये॒ सगुः॒ साश्वः॑। स॒ह यन्मे॒ अस्ति॒ तेन॑। भूः प्रप॑द्ये॒ भुवः॒ प्रप॑द्ये॒ सुवः॒ प्रप॑द्ये॒ भूर्भुवः॒ सुवः॒ प्रप॑द्ये वा॒युं प्रप॒द्येऽना᳚र्तां दे॒वतां॒ प्रप॒द्येऽश्मा॑नमाख॒णं प्रप॑द्ये प्र॒जाप॑तेर्ब्रह्मको॒शं ब्रह्म॒ प्रप॑द्य॒ ओं प्रप॑द्ये। अ॒न्तरि॑क्षं म उ॒र्व॑न्तरं॑ बृ॒हद॒ग्नयः॒ पर्व॑ताश्च॒ यया॒ वातः॑ स्व॒स्त्या स्व॑स्ति॒मान्तया᳚ स्व॒स्त्या स्व॑स्ति॒मान॑सानि। प्राणा॑पानौ मृ॒त्योर्मा॑ पातं॒ प्राणा॑पानौ॒ मा मा॑ हासिष्टं॒ मयि॑ मे॒धां मयि॑ प्र॒जां मय्य॒ग्निस्तेजो॑ दधातु॒ मयि॑ मे॒धां मयि॑ प्र॒जां मयीन्द्र॑ इन्द्रि॒यं द॑धातु॒ मयि॑ मे॒धां मयि॑ प्र॒जां मयि॒ सूर्यो॒ भ्राजो॑ दधातु॥

%७.४२.३
द्यु॒भिर॒क्तुभिः॒ परि॑पातम॒स्मानरि॑ष्टेभिरश्विना॒ सौभ॑गेभिः। तन्नो॑ मि॒त्रो वरु॑णो मामहन्ता॒मदि॑तिः॒ सिन्धुः॑ पृथि॒वी उ॒त द्यौः। कया॑ नश्चि॒त्र आ भु॑वदू॒ती स॒दावृ॑धः॒ सखा᳚। कया॒ शचि॑ष्ठया वृ॒ता। कस्त्वा॑ स॒त्यो मदा॑नां॒ मꣳहि॑ष्ठो मथ्स॒दन्ध॑सः। दृ॒ढाचि॑दा॒रुजे॒ वसु॑। अ॒भी षु णः॒ सखी॑नामवि॒ता ज॑रितॄ॒णाम्। श॒तं भ॑वास्यू॒तिभिः॑। वयः॑ सुप॒र्णा उप॑सेदु॒रिन्द्रं॑ प्रि॒यमे॑धा॒ ऋष॑यो॒ नाध॑मानाः। अप॑ ध्वा॒न्तमू᳚र्णु॒हि पू॒र्धि चक्षु॑र्मुमु॒ग्ध्य॑स्मान्नि॒धये॑व ब॒द्धान्॥

%७.४२.४
शं नो॑ दे॒वीर॒भिष्ट॑य॒ आपो॑ भवन्तु पी॒तये᳚। शं योर॒भिस्र॑वन्तु नः। ईशा॑ना॒ वार्या॑णां॒ क्षय॑न्तीश्चर्\mbox{}षणी॒नाम्। अ॒पो या॑चामि भेष॒जम्। सु॒मि॒त्रा न॒ आप॒ ओष॑धयः सन्तु दुर्मि॒त्रास्तस्मै॑ भूयासु॒र्यो᳚ऽस्मान्द्वेष्टि॒ यं च॑ व॒यं द्वि॒ष्मः। आपो॒ हि ष्ठा म॑यो॒भुव॒स्ता न॑ ऊ॒र्जे द॑धातन। म॒हे रणा॑य॒ चक्ष॑से। यो वः॑ शि॒वत॑मो॒ रस॒स्तस्य॑ भाजयते॒ह नः॑। उ॒श॒तीरि॑व मा॒तरः॑। तस्मा॒ अरं॑ गमाम वो॒ यस्य॒ क्षया॑य॒ जिन्व॑थ॥

%७.४२.५
आपो॑ ज॒नय॑था च नः। पृ॒थि॒वी शा॒न्ता साग्निना॑ शा॒न्ता सा मे॑ शा॒न्ता शुचꣳ॑ शमयतु। अ॒न्तरि॑क्षꣳ शा॒न्तं तद्वा॒युना॑ शा॒न्तं तन्मे॑ शा॒न्तꣳ शुचꣳ॑ शमयतु। द्यौः  शा॒न्ता सादि॒त्येन॑ शा॒न्ता सा मे॑ शा॒न्ता शुचꣳ॑ शमयतु। पृ॒थि॒वी शान्ति॑र॒न्तरि॑क्ष॒ꣳ॒ शान्ति॒र्द्यौः  शान्ति॒र्दिशः॒ शान्ति॑रवान्तरदि॒शाः  शान्ति॑र॒ग्निः  शान्ति॑र्वा॒युः  शान्ति॑रादि॒त्यः  शान्ति॑श्च॒न्द्रमाः॒ शान्ति॒र्नक्ष॑त्राणि॒ शान्ति॒रापः॒ शान्ति॒रोष॑धयः॒ शान्ति॒र्वन॒स्पत॑यः॒ शान्ति॒र्गौः  शान्ति॑र॒जा शान्ति॒रश्वः॒ शान्तिः॒ पुरु॑षः॒ शान्ति॒र्ब्रह्म॒ शान्ति॑र्ब्राह्म॒णः  शान्तिः॒ शान्ति॑रे॒व शान्तिः॒ शान्ति॑र्मे अस्तु॒ शान्तिः॑। तया॒हꣳ शा॒न्त्या स॑र्वशा॒न्त्या मह्यं॑ द्वि॒पदे॒ चतु॑ष्पदे च॒ शान्तिं॑ करोमि॒ शान्ति॑र्मे अस्तु॒ शान्तिः॑। एह॒ श्रीश्च॒ ह्रीश्च॒ धृति॑श्च॒ तपो॑ मे॒धा प्र॑ति॒ष्ठा श्र॒द्धा स॒त्यं धर्म॑श्चै॒तानि॒ मोत्ति॑ष्ठन्त॒मनूत्ति॑ष्ठन्तु॒ मा मा॒ꣴ॒ श्रीश्च॒ ह्रीश्च॒ धृति॑श्च॒ तपो॑ मे॒धा प्र॑ति॒ष्ठा श्र॒द्धा स॒त्यं धर्म॑श्चै॒तानि॑ मा॒ मा हा॑सिषुः। उदायु॑षा स्वा॒युषोदोष॑धीना॒ꣳ॒ रसे॒नोत्प॒र्जन्य॑स्य॒ शुष्मे॒णोद॑स्थाम॒मृता॒ꣳ॒ अनु॑। तच्चक्षु॑र्दे॒वहि॑तं पु॒रस्ता᳚च्छु॒क्रमु॒च्चर॑त्। पश्ये॑म श॒रदः॑ श॒तं जीवे॑म श॒रदः॑ श॒तं नन्दा॑म श॒रदः॑ श॒तं मोदा॑म श॒रदः॑ श॒तं भवा॑म श॒रदः॑ श॒तꣳ शृ॒णवा॑म श॒रदः॑ श॒तं प्रब्र॑वाम श॒रदः॑ श॒तमजी॑ताः स्याम श॒रदः॑ श॒तं ज्योक्च॒ सूर्यं॑ दृ॒शे। य उद॑गान्मह॒तोऽर्णवा᳚द्वि॒भ्राज॑मानः सरि॒रस्य॒ मध्या॒थ्स मा॑ वृष॒भो लो॑हिता॒क्षः सूर्यो॑ विप॒श्चिन्मन॑सा पुनातु। ब्रह्म॑ण॒श्चोत॑न्यसि॒ ब्रह्म॑ण आ॒णी स्थो॒ ब्रह्म॑ण आ॒वप॑नमसि धारि॒तेयं पृ॑थि॒वी ब्रह्म॑णा म॒ही धा॑रि॒तमे॑नेन म॒हद॒न्तरि॑क्षं॒ दिवं॑ दाधार पृथि॒वीꣳ सदे॑वां॒ यद॒हं वेद॒ तद॒हं धा॑रयाणि॒ मा मद्वेदोऽधि॒विस्र॑सत्। मे॒धा॒म॒नी॒षे मावि॑शताꣳ स॒मीची॑ भू॒तस्य॒ भव्य॒स्याव॑रुध्यै॒ सर्व॒मायु॑रयाणि॒ सर्व॒मायु॑रयाणि। आ॒भिर्गी॒र्भिर्यदतो॑ न ऊ॒नमाप्या॑यय हरिवो॒ वर्ध॑मानः। य॒दा स्तो॒तृभ्यो॒ महि॑ गो॒त्रा रु॒जासि॑ भूयिष्ठ॒भाजो॒ अध॑ ते स्याम। ब्रह्म॒ प्रावा॑दिष्म॒ तन्नो॒ मा हा॑सीत्। ॐ शान्तिः॒ शान्तिः॒ शान्तिः॑॥
