\chapt{काण्डम् ४}
\sect{द्वितीयः प्रश्नः}\setcounter{anuvakam}{0}
\dnsub{तैत्तिरीयसंहितायां चतुर्थकाण्डे द्वितीयः प्रश्नः}
%4.2.1.1
विष्णोः॒ क्रमो᳚\-ऽस्यभिमाति॒हा गा॑य॒त्रं छन्द॒ आ रो॑ह पृथि॒वीमनु॒ वि क्र॑मस्व॒ निर्भ॑क्तः॒ स यं द्वि॒ष्मो विष्णोः॒ क्रमो᳚\-ऽस्यभिशस्ति॒हा त्रैष्टु॑भं॒ छन्द॒ आ रो॑हा॒न्तरि॑क्ष॒मनु॒ वि क्र॑मस्व॒ निर्भ॑क्तः॒ स यं द्वि॒ष्मो विष्णोः॒ क्रमो᳚\-ऽस्यरातीय॒तो ह॒न्ता जाग॑तं॒ छन्द॒ आ रो॑ह॒ दिव॒मनु॒ वि क्र॑मस्व॒ निर्भ॑क्तः॒ स यं द्वि॒ष्मो विष्णोः॒~(१)

%4.2.1.2
क्रमो॑\-ऽसि शत्रूय॒तो ह॒न्ताऽऽनु॑ष्टुभं॒ छन्द॒ आ रो॑ह॒ दिशो\-ऽनु॒ वि क्र॑मस्व॒ निर्भ॑क्तः॒ स यं द्वि॒ष्मः। अक्र॑न्दद॒ग्निः स्त॒नय॑न्निव॒ द्यौः क्षामा॒ रेरि॑हद्वी॒रुधः॑ सम॒ञ्जन्न्। स॒द्यो ज॑ज्ञा॒नो वि हीमि॒द्धो अख्य॒दा रोद॑सी भा॒नुना॑ भात्य॒न्तः। अग्ने᳚\-ऽभ्यावर्तिन्न॒भि न॒ आ व॑र्त॒स्वाऽऽयु॑षा॒ वर्च॑सा स॒न्या मे॒धया᳚ प्र॒जया॒ धने॑न। अग्ने॑~(२)

%4.2.1.3
अङ्गिरः श॒तं ते॑ सन्त्वा॒वृतः॑ स॒हस्रं॑ त उपा॒वृतः॑। तासां॒ पोष॑स्य॒ पोषे॑ण॒ पुन॑र्नो न॒ष्टमा कृ॑धि॒ पुन॑र्नो र॒यिमा कृ॑धि। पुन॑रू॒र्जा नि व॑र्तस्व॒ पुन॑रग्न इ॒षा\-ऽ\-ऽ\-यु॑षा। पुन॑र्नः पाहि वि॒श्वतः॑। स॒ह र॒य्या नि व॑र्त॒स्वाग्ने॒ पिन्व॑स्व॒ धार॑या। वि॒श्वफ्स्नि॑या वि॒श्वत॒स्परि॑। उदु॑त्त॒मं व॑रुण॒ पाश॑म॒स्मदवा॑ध॒मं~(३)

%4.2.1.4
वि म॑ध्य॒मꣴ श्र॑थाय। अथा॑ व॒यमा॑दित्य व्र॒ते तवाना॑गसो॒ अदि॑तये स्याम। आ त्वा॑हार्\mbox{}षम॒न्तर॑भूर्ध्रु॒वस्ति॒ष्ठा\-वि॑चाचलिः। विश॑स्त्वा॒ सर्वा॑ वाञ्छन्त्व॒स्मिन्रा॒ष्ट्रमधि॑ श्रय। अग्रे॑ बृ॒हन्नु॒षसा॑मू॒र्ध्वो अ॑स्थान्निर्जग्मि॒वान्तम॑सो॒ ज्योति॒षा\-ऽऽ\-ऽ\-गा᳚त्। अ॒ग्निर्भा॒नुना॒ रुश॑ता॒ स्वङ्ग॒ आ जा॒तो विश्वा॒ सद्मा᳚न्यप्राः। सीद॒ त्वं मा॒तुर॒स्या~-~(४)

%4.2.1.5
उ॒पस्थे॒ विश्वा᳚न्यग्ने व॒युना॑नि वि॒द्वान्। मैना॑म॒र्चिषा॒ मा तप॑सा॒\-ऽभि शू॑शुचो॒\-ऽन्तर॑स्याꣳ शु॒क्रज्यो॑ति॒र्वि भा॑हि। अ॒न्तर॑ग्ने रु॒चा त्वमु॒खायै॒ सद॑ने॒ स्वे। तस्या॒स्त्वꣳ हर॑सा॒ तप॒ञ्जात॑वेदः शि॒वो भ॑व। शि॒वो भू॒त्वा मह्य॑म॒ग्ने\-ऽथो॑ सीद शि॒वस्त्वम्। शि॒वाः कृ॒त्वा दिशः॒ सर्वाः॒ स्वं योनि॑मि॒हास॑दः। ह॒ꣳ॒सः शु॑चि॒षद्वसु॑रन्तरिक्ष॒सद्धोता॑ वेदि॒षदति॑थिर्दुरोण॒सत्। नृ॒षद्व॑र॒सदृ॑त॒सद्व्यो॑म॒सद॒ब्जा गो॒जा ऋ॑त॒जा अ॑द्रि॒जा ऋ॒तं बृ॒हत्॥~(५)

{\anuvakamend[{दिव॒मनु॒ वि क्र॑मस्व॒ निर्भ॑क्तः॒ स यं द्वि॒ष्मो विष्णो॒र्धने॒नाग्ने॑\-ऽध॒मम॒स्याः शु॑चि॒षथ्षोड॑श च}]}%~(१)

%4.2.2.1
दि॒वस्परि॑ प्रथ॒मं ज॑ज्ञे अ॒ग्निर॒स्मद्द्वि॒तीयं॒ परि॑ जा॒तवे॑दाः। तृ॒तीय॑म॒फ्सु नृ॒मणा॒ अज॑स्र॒मिन्धा॑न एनं जरते स्वा॒धीः। वि॒द्मा ते॑ अग्ने त्रे॒धा त्र॒याणि॑ वि॒द्मा ते॒ सद्म॒ विभृ॑तं पुरु॒त्रा। वि॒द्मा ते॒ नाम॑ पर॒मं गुहा॒ यद्वि॒द्मा तमुथ्सं॒ यत॑ आज॒गन्थ॑। स॒मु॒द्रे त्वा॑ नृ॒मणा॑ अ॒फ्स्व॑न्तर्नृ॒चक्षा॑ ईधे दि॒वो अ॑ग्न॒ ऊधन्न्॑। तृ॒तीये᳚ त्वा॒~(६)

%4.2.2.2
रज॑सि तस्थि॒वाꣳस॑मृ॒तस्य॒ योनौ॑ महि॒षा अ॑हिन्वन्न्। अक्र॑न्दद॒ग्निः स्त॒नय॑न्निव॒ द्यौः क्षामा॒ रेरि॑हद्वी॒रुधः॑ सम॒ञ्जन्न्। स॒द्यो ज॑ज्ञा॒नो वि हीमि॒द्धो अख्य॒दा रोद॑सी भा॒नुना॑ भात्य॒न्तः। उ॒शिक्पा॑व॒को अ॑र॒तिः सु॑मे॒धा मर्ते᳚ष्व॒ग्निर॒मृतो॒ निधा॑यि। इय॑र्ति धू॒मम॑रु॒षं भरि॑भ्र॒दुच्छु॒क्रेण॑ शो॒चिषा॒ द्यामिन॑क्षत्। विश्व॑स्य के॒तुर्भुव॑नस्य॒ गर्भ॒ आ~(७)

%4.2.2.3
रोद॑सी अपृणा॒ज्जाय॑मानः। वी॒डुं चि॒दद्रि॑मभिनत्परा॒यञ्जना॒ यद॒ग्निमय॑जन्त॒ पञ्च॑। श्री॒णामु॑दा॒रो ध॒रुणो॑ रयी॒णां म॑नी॒षाणां॒ प्रार्प॑णः॒ सोम॑गोपाः। वसोः᳚ सू॒नुः सह॑सो अ॒फ्सु राजा॒ वि भा॒त्यग्र॑ उ॒षसा॑मिधा॒नः। यस्ते॑ अ॒द्य कृ॒णव॑द्भद्रशोचे\-ऽपू॒पं दे॑व घृ॒तव॑न्तमग्ने। प्र तं न॑य प्रत॒रां वस्यो॒ अच्छा॒भि द्यु॒म्नं दे॒वभ॑क्तं यविष्ठ। आ~(८)

%4.2.2.4
तम्भ॑ज सौश्रव॒सेष्व॑ग्न उ॒क्थउ॑क्थ॒ आ भ॑ज श॒स्यमा॑ने। प्रि॒यः सूर्ये᳚ प्रि॒यो अ॒ग्ना भ॑वा॒त्युज्जा॒तेन॑ भि॒नद॒दुज्जनि॑त्वैः। त्वाम॑ग्ने॒ यज॑माना॒ अनु॒ द्यून् विश्वा॒ वसू॑नि दधिरे॒ वार्या॑णि। त्वया॑ स॒ह द्रवि॑णमि॒च्छमा॑ना व्र॒जं गोम॑न्तमु॒शिजो॒ वि व॑व्रुः। दृ॒शा॒नो रु॒क्म उ॒र्व्या व्य॑द्यौद्दु॒र्मर्\mbox{}ष॒मायुः॑ श्रि॒ये रु॑चा॒नः। अ॒ग्निर॒मृतो॑ अभव॒द्वयो॑भि॒र्यदे॑नं॒ द्यौरज॑नयथ्सु॒रेताः᳚॥~(९)

{\anuvakamend[{तृ॒तीये᳚ त्वा॒ गर्भ॒ आ य॑वि॒ष्ठा यच्च॒त्वारि॑ च}]}%~(२)

%4.2.3.1
अन्न॑प॒ते\-ऽन्न॑स्य नो देह्यनमी॒वस्य॑ शु॒ष्मिणः॑। प्रप्र॑दा॒तारं॑ तारिष॒ ऊर्जं॑ नो धेहि द्वि॒पदे॒ चतु॑ष्पदे। उदु॑ त्वा॒ विश्वे॑ दे॒वा अग्ने॒ भर॑न्तु॒ चित्ति॑भिः। स नो॑ भव शि॒वत॑मः सु॒प्रती॑को वि॒भाव॑सुः। प्रेद॑ग्ने॒ ज्योति॑ष्मान् याहि शि॒वेभि॑र॒र्चिभि॒स्त्वम्। बृ॒हद्भि॑र्भा॒नुभि॒र्भास॒न्मा हिꣳ॑सीस्त॒नुवा᳚ प्र॒जाः। स॒मिधा॒ग्निं दु॑वस्यत घृ॒तैर्बो॑धय॒ताति॑थिम्। आ~(१०)

%4.2.3.2
ऽस्मि॑न् ह॒व्या जु॑होतन। प्रप्रा॒यम॒ग्निर्भ॑र॒तस्य॑ शृण्वे॒ वि यथ्सूर्यो॒ न रोच॑ते बृ॒हद्भाः। अ॒भि यः पू॒रुं पृत॑नासु त॒स्थौ दी॒दाय॒ दैव्यो॒ अति॑थिः शि॒वो नः॑। आपो॑ देवीः॒ प्रति॑ गृह्णीत॒ भस्मै॒तथ्स्यो॒ने कृ॑णुध्वꣳ सुर॒भावु॑ लो॒के। तस्मै॑ नमन्तां॒ जन॑यः सु॒पत्नी᳚र्मा॒तेव॑ पु॒त्रं बि॑भृ॒ता स्वे॑नम्। अ॒फ्स्व॑ग्ने॒ सधि॒ष्टव॒~(११)

%4.2.3.3
सौष॑धी॒रनु॑ रुध्यसे। गर्भे॒ सञ्जा॑यसे॒ पुनः॑। गर्भो॑ अ॒स्योष॑धीनां॒ गर्भो॒ वन॒स्पती॑नाम्। गर्भो॒ विश्व॑स्य भू॒तस्याग्ने॒ गर्भो॑ अ॒पाम॑सि। प्र॒सद्य॒ भस्म॑ना॒ योनि॑म॒पश्च॑ पृथि॒वीम॑ग्ने। स॒ꣳ॒सृज्य॑ मा॒तृभि॒स्त्वं ज्योति॑ष्मा॒न्पुन॒रास॑दः। पुन॑रा॒सद्य॒ सद॑नम॒पश्च॑ पृथि॒वीम॑ग्ने। शेषे॑ मा॒तुर्यथो॒पस्थे॒\-ऽन्तर॒स्याꣳ शि॒वत॑मः। पुन॑रू॒र्जा~(१२)

%4.2.3.4
नि व॑र्तस्व॒ पुन॑रग्न इ॒षा\-ऽ\-ऽ\-यु॑षा। पुन॑र्नः पाहि वि॒श्वतः॑। स॒ह र॒य्या नि व॑र्त॒स्वाग्ने॒ पिन्व॑स्व॒ धार॑या। वि॒श्वफ्स्नि॑या वि॒श्वत॒स्परि॑। पुन॑स्त्वादि॒त्या रु॒द्रा वस॑वः॒ समि॑न्धतां॒ पुन॑र्ब्र॒ह्माणो॑ वसुनीथ य॒ज्ञैः। घृ॒तेन॒ त्वं त॒नुवो॑ वर्धयस्व स॒त्याः स॑न्तु॒ यज॑मानस्य॒ कामाः᳚। बोधा॑ नो अ॒स्य वच॑सो यविष्ठ॒ मꣳहि॑ष्ठस्य॒ प्रभृ॑तस्य स्वधावः। पीय॑ति त्वो॒ अनु॑ त्वो गृणाति व॒न्दारु॑स्ते त॒नुवं॑ वन्दे अग्ने। स बो॑धि सू॒रिर्म॒घवा॑ वसु॒दावा॒ वसु॑पतिः। यु॒यो॒ध्य॑स्मद्द्वेषाꣳ॑सि॥~(१३)

{\anuvakamend[{आ तवो॒र्जा\-ऽनु॒ षोड॑श च}]}%~(३)

%4.2.4.1
अपे॑त॒ वीत॒ वि च॑ सर्प॒तातो॒ ये\-ऽत्र॒ स्थ पु॑रा॒णा ये च॒ नूत॑नाः। अदा॑दि॒दं य॒मो॑\-ऽव॒सानं॑ पृथि॒व्या अक्र॑न्नि॒मं पि॒तरो॑ लो॒कम॑स्मै। अ॒ग्नेर्भस्मा᳚स्य॒ग्नेः पुरी॑षमसि सं॒ज्ञान॑मसि काम॒धर॑णं॒ मयि॑ ते काम॒धर॑णं भूयात्। सं या वः॑ प्रि॒यास्त॒नुवः॒ सम्प्रि॒या हृद॑यानि वः। आ॒त्मा वो॑ अस्तु॒~(१४)

%4.2.4.2
सम्प्रि॑यः॒ सम्प्रि॑यास्त॒नुवो॒ मम॑। अ॒यꣳ सो अ॒ग्निर्यस्मि॒न्थ्सोम॒\-मिन्द्रः॑ सु॒तं द॒धे ज॒ठरे॑ वावशा॒नः। स॒ह॒स्रियं॒ वाज॒मत्यं॒ न सप्तिꣳ॑ सस॒वान्थ्सन्थ्स्तू॑यसे जातवेदः। अग्ने॑ दि॒वो अर्ण॒मच्छा॑ जिगा॒स्यच्छा॑ दे॒वाꣳ ऊ॑चिषे॒ धिष्णि॑या॒ ये। याः प॒रस्ता᳚द्रोच॒ने सूर्य॑स्य॒ याश्चा॒वस्ता॑दुप॒तिष्ठ॑न्त॒ आपः॑। अग्ने॒ यत्ते॑ दि॒वि वर्चः॑ पृथि॒व्यां यदोष॑धी-~(१५)

%4.2.4.3
ष्व॒फ्सु वा॑ यजत्र। येना॒न्तरि॑क्षमु॒र्वा॑त॒तन्थ॑ त्वे॒षः स भा॒नुर॑र्ण॒वो नृ॒चक्षाः᳚। पु॒री॒ष्या॑सो अ॒ग्नयः॑ प्राव॒णेभिः॑ स॒जोष॑सः। जु॒षन्ताꣳ॑ ह॒व्यमाहु॑तमनमी॒वा इषो॑ म॒हीः। इडा॑मग्ने पुरु॒दꣳसꣳ॑ स॒निं गोः श॑श्वत्त॒मꣳ हव॑मानाय साध। स्यान्नः॑ सु॒नुस्तन॑यो वि॒जावाग्ने॒ सा ते॑ सुम॒तिर्भू᳚त्व॒स्मे। अ॒यं ते॒ योनि॑र्\mbox{}ऋ॒त्वियो॒ यतो॑ जा॒तो अरो॑चथाः। तं जा॒न-~(१६)

%4.2.4.4
न्न॑ग्न॒ आ रो॒हाथा॑ नो वर्धया र॒यिम्। चिद॑सि॒ तया॑ दे॒वत॑याङ्गिर॒स्वद्ध्रु॒वा सी॑द परि॒चिद॑सि॒ तया॑ दे॒वत॑या\-ऽङ्गिर॒स्वद्ध्रु॒वा सी॑द लो॒कं पृ॑ण छि॒द्रं पृ॒णाथो॑ सीद शि॒वा त्वम्। इ॒न्द्रा॒ग्नी त्वा॒ बृह॒स्पति॑र॒स्मिन् योना॑वसीषदन्न्। ता अ॑स्य॒ सूद॑दोहसः॒ सोमꣴ॑ श्रीणन्ति॒ पृश्ञ॑यः। जन्मं॑ दे॒वानां॒ विश॑स्त्रि॒ष्वा रो॑च॒ने दि॒वः॥~(१७)

{\anuvakamend[{अ॒स्त्वोष॑धीषु जा॒नन्न॒ष्टाच॑त्वारिꣳशच्च}]}%~(४)

%4.2.5.1
समि॑त॒ꣳ॒ सं क॑ल्पेथा॒ꣳ॒ सम्प्रि॑यौ रोचि॒ष्णू सु॑मन॒स्यमा॑नौ। इष॒मूर्ज॑म॒भि सं॒वसा॑नौ॒ सं वां॒ मनाꣳ॑सि॒ सं व्र॒ता समु॑ चि॒त्तान्याक॑रम्। अग्ने॑ पुरीष्याधि॒पा भ॑वा॒ त्वं नः॑। इष॒मूर्जं॒ यज॑मानाय धेहि। पु॒री॒ष्य॑स्त्वम॑ग्ने रयि॒मान्पु॑ष्टि॒माꣳ अ॑सि। शि॒वाः कृ॒त्वा दिशः॒ सर्वाः॒ स्वां योनि॑मि॒हास॑दः। भव॑तं नः॒ सम॑नसौ॒ समो॑कसौ~(१८)

%4.2.5.2
अ॒रे॒पसौ᳚। मा य॒ज्ञꣳ हिꣳ॑सिष्टं॒ मा य॒ज्ञप॑तिं जातवेदसौ शि॒वौ भ॑वतम॒द्य नः॑। मा॒तेव॑ पु॒त्रं पृ॑थि॒वी पु॑री॒ष्य॑म॒ग्निꣴ स्वे योना॑वभारु॒खा। तां विश्वै᳚र्दे॒वैर्\mbox{}ऋ॒तुभिः॑ संविदा॒नः प्र॒जा\-प॑तिर्वि॒श्वक॑र्मा॒ वि मु॑ञ्चतु। यद॒स्य पा॒रे रज॑सः शु॒क्रं ज्योति॒रजा॑यत। तन्नः॑ पर्\mbox{}ष॒दति॒ द्विषो\-ऽग्ने॑ वैश्वानर॒ स्वाहा᳚। नमः॒ सु ते॑ निर्\mbox{}ऋते विश्वरूपे~(१९)

%4.2.5.3
अ॒य॒स्मयं॒ वि चृ॑ता ब॒न्धमे॒तम्। य॒मेन॒ त्वं य॒म्या॑ संविदा॒नोत्त॒मं नाक॒मधि॑ रोहये॒मम्। यत्ते॑ दे॒वी निर्\mbox{}ऋ॑तिरा ब॒बन्ध॒ दाम॑ ग्री॒वास्व॑विच॒र्त्यम्। इ॒दं ते॒ तद्वि ष्या॒म्यायु॑षो॒ न मध्या॒दथा॑ जी॒वः पि॒तुम॑द्धि॒ प्रमु॑क्तः। यस्या᳚स्ते अ॒स्याः क्रू॒र आ॒सञ्जु॒होम्ये॒षां ब॒न्धाना॑मव॒सर्ज॑नाय। भूमि॒रिति॑ त्वा॒ जना॑ वि॒दुर्निर्\mbox{}ऋ॑तिः~(२०)

%4.2.5.4
इति॑ त्वा॒हं परि॑ वेद वि॒श्वतः॑। असु॑न्वन्त॒मय॑जमानमिच्छ स्ते॒नस्ये॒त्यां तस्क॑र॒स्यान्वे॑षि। अ॒न्यम॒स्मदि॑च्छ॒ सा त॑ इ॒त्या नमो॑ देवि निर्\mbox{}ऋते॒ तुभ्य॑मस्तु। दे॒वीम॒हं निर्\mbox{}ऋ॑तिं॒ वन्द॑मानः पि॒तेव॑ पु॒त्रं द॑सये॒ वचो॑भिः। विश्व॑स्य॒ या जाय॑मानस्य॒ वेद॒ शिरः॑शिरः॒ प्रति॑ सू॒री वि च॑ष्टे। नि॒वेश॑नः सं॒गम॑नो॒ वसू॑नां॒ विश्वा॑ रू॒पाभि च॑ष्टे~(२१)

%4.2.5.5
शची॑भिः। दे॒व इ॑व सवि॒ता स॒त्यध॒र्मेन्द्रो॒ न त॑स्थौ सम॒रे प॑थी॒नाम्। सं व॑र॒त्रा द॑धातन॒ निरा॑हा॒वान्कृ॑णोतन। सि॒ञ्चाम॑हा अव॒टमु॒द्रिणं॑ व॒यं विश्वाहाद॑स्त॒मक्षि॑तम्। निष्कृ॑ताहावमव॒टꣳ सु॑वर॒त्रꣳ सु॑षेच॒नम्। उ॒द्रिणꣳ॑ सिञ्चे॒ अक्षि॑तम्। सीरा॑ युञ्जन्ति क॒वयो॑ यु॒गा वि त॑न्वते॒ पृथ॑क्। धीरा॑ दे॒वेषु॑ सुम्न॒या। यु॒नक्त॒ सीरा॒ वि यु॒गा त॑नोत कृ॒ते योनौ॑ वपते॒ह~(२२)

%4.2.5.6
बीजम्᳚। गि॒रा च॑ श्रु॒ष्टिः सभ॑रा॒ अस॑न्नो॒ नेदी॑य॒ इथ्सृ॒ण्या॑ प॒क्वमाय॑त्। लाङ्ग॑लं॒ पवी॑रवꣳ सु॒शेवꣳ॑ सुम॒तिथ्स॑रु। उदित्कृ॑षति॒ गामविं॑ प्रफ॒र्व्यं॑ च॒ पीव॑रीम्। प्र॒स्थाव॑द्रथ॒वाह॑नम्। शु॒नं नः॒ फाला॒ वि तु॑दन्तु॒ भूमिꣳ॑ शु॒नं की॒नाशा॑ अ॒भि य॑न्तु वा॒हान्। शु॒नं प॒र्जन्यो॒ मधु॑ना॒ पयो॑भिः॒ शुना॑सीरा शु॒नम॒स्मासु॑ धत्तम्। कामं॑ कामदुघे धुक्ष्व मि॒त्राय॒ वरु॑णाय च। इन्द्रा॑या॒ग्नये॑ पू॒ष्ण ओष॑धीभ्यः प्र॒जाभ्यः॑। घृ॒तेन॒ सीता॒ मधु॑ना॒ सम॑क्ता॒ विश्वै᳚र्दे॒वैरनु॑मता म॒रुद्भिः॑। ऊर्ज॑स्वती॒ पय॑सा॒ पिन्व॑माना॒स्मान्थ्सी॑ते॒ पय॑सा॒भ्याव॑वृथ्स्व॥~(२३)

{\anuvakamend[{समो॑कसौ विश्वरूपे वि॒दुर्निर्\mbox{}ऋ॑तिर॒भि च॑ष्ट इ॒ह मि॒त्राय॒ द्वाविꣳ॑शतिश्च}]}%~(५)

%4.2.6.1
या जा॒ता ओष॑धयो दे॒वेभ्य॑स्त्रियु॒गं पु॒रा। मन्दा॑मि ब॒भ्रूणा॑महꣳ श॒तं धामा॑नि स॒प्त च॑। श॒तं वो॑ अम्ब॒ धामा॑नि स॒हस्र॑मुत वो॒ रुहः॑। अथा॑ शतक्रत्वो यू॒यमि॒मं मे॑ अग॒दं कृ॑त। पुष्पा॑वतीः प्र॒सूव॑तीः फ॒लिनी॑रफ॒ला उ॒त। अश्वा॑ इव स॒जित्व॑रीर्वी॒रुधः॑ पारयि॒ष्णवः॑। ओष॑धी॒रिति॑ मातर॒स्तद्वो॑ देवी॒रुप॑ ब्रुवे। रपाꣳ॑सि विघ्न॒तीरि॑त॒ रपः॑~(२४)

%4.2.6.2
चा॒तय॑मानाः। अ॒श्व॒त्थे वो॑ नि॒षद॑नं प॒र्णे वो॑ वस॒तिः कृ॒ता। गो॒भाज॒ इत्किला॑सथ॒ यथ्स॒नव॑थ॒ पूरु॑षम्। यद॒हं वा॒जय॑न्नि॒मा ओष॑धी॒र्॒\mbox{}हस्त॑ आद॒धे। आ॒त्मा यक्ष्म॑स्य नश्यति पु॒रा जी॑व॒गृभो॑ यथा। यदोष॑धयः सं॒गच्छ॑न्ते॒ राजा॑नः॒ समि॑ताविव। विप्रः॒ स उ॑च्यते भि॒षग्र॑क्षो॒हामी॑व॒चात॑नः। निष्कृ॑ति॒र्नाम॑ वो मा॒ताथा॑ यू॒यꣴ स्थ॒ सङ्कृ॑तीः। स॒राः प॑त॒त्रिणीः᳚~(२५)

%4.2.6.3
स्थ॒न॒ यदा॒मय॑ति॒ निष्कृ॑त। अ॒न्या वो॑ अ॒न्याम॑वत्व॒न्यान्यस्या॒ उपा॑वत। ताः सर्वा॒ ओष॑धयः संविदा॒ना इ॒दं मे॒ प्राव॑ता॒ वचः॑। उच्छुष्मा॒ ओष॑धीनां॒ गावो॑ गो॒ष्ठादि॑वेरते। धनꣳ॑ सनि॒ष्यन्ती॑नामा॒त्मानं॒ तव॑ पूरुष। अति॒ विश्वाः᳚ परि॒ष्ठाः स्ते॒न इ॑व व्र॒जम॑क्रमुः। ओष॑धयः॒ प्राचु॑च्यवु॒र्यत् किं च॑ त॒नुवा॒ꣳ॒ रपः॑। याः~(२६)

%4.2.6.4
त॒ आ॒त॒स्थुरा॒त्मानं॒ या आ॑विवि॒शुः परुः॑परुः। तास्ते॒ यक्ष्मं॒ वि बा॑धन्तामु॒ग्रो म॑ध्यम॒शीरि॑व। सा॒कं य॑क्ष्म॒ प्र प॑त श्ये॒नेन॑ किकिदी॒विना᳚। सा॒कं वात॑स्य॒ ध्राज्या॑ सा॒कं न॑श्य नि॒हाक॑या। अ॒श्वा॒व॒तीꣳ सो॑मव॒तीमू॒र्जय॑न्ती॒\-मुदो॑जसम्। आ वि॑थ्सि॒ सर्वा॒ ओष॑धीर॒स्मा अ॑रि॒ष्टता॑तये। याः फ॒लिनी॒र्या अ॑फ॒ला अ॑पु॒ष्पा याश्च॑ पु॒ष्पिणीः᳚। बृह॒स्पति॑प्रसूता॒स्ता नो॑ मुञ्च॒न्त्वꣳह॑सः। याः~(२७)

%4.2.6.5
ओष॑धयः॒ सोम॑राज्ञीः॒ प्रवि॑ष्टाः पृथि॒वीमनु॑। तासां॒ त्वम॑स्युत्त॒मा प्र णो॑ जी॒वात॑वे सुव। अ॒व॒पत॑न्तीरवदन्दि॒व ओष॑धयः॒ परि॑। यं जी॒वम॒श्ञवा॑महै॒ न स रि॑ष्याति॒ पूरु॑षः। याश्चे॒दमु॑पशृ॒ण्वन्ति॒ याश्च॑ दू॒रं परा॑गताः। इ॒ह सं॒गत्य॒ ताः सर्वा॑ अस्मै॒ सं द॑त्त भेष॒जम्। मा वो॑ रिषत्खनि॒ता यस्मै॑ चा॒हं खना॑मि वः। द्वि॒पच्चतु॑ष्पद॒स्माक॒ꣳ॒ सर्व॑म॒स्त्वना॑तुरम्। ओष॑धयः॒ सं व॑दन्ते॒ सोमे॑न स॒ह राज्ञा᳚। यस्मै॑ क॒रोति॑ ब्राह्म॒णस्तꣳ रा॑जन्पारयामसि॥~(२८)

{\anuvakamend[{रपः॑ पत॒त्रिणी॒र्या अꣳह॑सो॒ याः खना॑मि वो॒\-ऽष्टाद॑श च}]}%~(६)

%4.2.7.1
मा नो॑ हिꣳसीज्जनि॒ता यः पृ॑थि॒व्या यो वा॒ दिवꣳ॑ स॒त्यध॑र्मा ज॒जान॑। यश्चा॒पश्च॒न्द्रा बृ॑ह॒तीर्ज॒जान॒ कस्मै॑ दे॒वाय॑ ह॒विषा॑ विधेम। अ॒भ्याव॑र्तस्व पृथिवि य॒ज्ञेन॒ पय॑सा स॒ह। व॒पां ते॑ अ॒ग्निरि॑षि॒तो\-ऽव॑ सर्पतु। अग्ने॒ यत्ते॑ शु॒क्रं यच्च॒न्द्रं यत्पू॒तं यद्य॒ज्ञियम्᳚। तद्दे॒वेभ्यो॑ भरामसि। इष॒मूर्ज॑म॒हमि॒त आ~(२९)

%4.2.7.2
द॒द॒ ऋ॒तस्य॒ धाम्नो॑ अ॒मृत॑स्य॒ योनेः᳚। आ नो॒ गोषु॑ विश॒त्वौष॑धीषु॒ जहा॑मि से॒दिमनि॑रा॒ममी॑वाम्। अग्ने॒ तव॒ श्रवो॒ वयो॒ महि॑ भ्राजन्त्य॒र्चयो॑ विभावसो। बृह॑द्भानो॒ शव॑सा॒ वाज॑मु॒क्थ्यं॑ दधा॑सि दा॒शुषे॑ कवे। इ॒र॒ज्यन्न॑ग्ने प्रथयस्व ज॒न्तुभि॑र॒स्मे रायो॑ अमर्त्य। स द॑र्\mbox{}श॒तस्य॒ वपु॑षो॒ वि रा॑जसि पृ॒णक्षि॑ सान॒सिꣳ र॒यिम्। ऊर्जो॑ नपा॒ज्जात॑वेदः सुश॒स्तिभि॒र्मन्द॑स्व~(३०)

%4.2.7.3
धी॒तिभि॑र्हि॒तः। त्वे इषः॒ सं द॑धु॒र्भूरि॑रेतसश्चि॒त्रोत॑यो वा॒मजा॑ताः। पा॒व॒कव॑र्चाः शु॒क्रव॑र्चा॒ अनू॑नवर्चा॒ उदि॑यर्\mbox{}षि भा॒नुना᳚। पु॒त्रः पि॒तरा॑ वि॒चर॒न्नुपा॑वस्यु॒भे पृ॑णक्षि॒ रोद॑सी। ऋ॒तावा॑नं महि॒षं वि॒श्वच॑र्\mbox{}षणिम॒ग्निꣳ सु॒म्नाय॑ दधिरे पु॒रो जनाः᳚। श्रुत्क॑र्णꣳ स॒प्रथ॑स्तं᳚ त्वा गि॒रा दैव्यं॒ मानु॑षा यु॒गा। नि॒ष्क॒र्तार॑मध्व॒रस्य॒ प्रचे॑तसं॒ क्षय॑न्त॒ꣳ॒ राध॑से म॒हे। रा॒तिं भृगू॑णामु॒शिजं॑ क॒विक्र॑तुं पृ॒णक्षि॑ सान॒सिम्~(३१)

%4.2.7.4
र॒यिम्। चितः॑ स्थ परि॒चित॑ ऊर्ध्व॒चितः॑ श्रयध्वं॒ तया॑ दे॒वत॑याङ्गिर॒स्वद् ध्रु॒वाः सी॑दत। आ प्या॑यस्व॒ समे॑तु ते वि॒श्वतः॑ सोम॒ वृष्णि॑यम्। भवा॒ वाज॑स्य सङ्ग॒थे। सं ते॒ पयाꣳ॑सि॒ समु॑ यन्तु॒ वाजाः॒ सं वृष्णि॑यान्यभिमाति॒षाहः॑। आ॒प्याय॑मानो अ॒मृता॑य सोम दि॒वि श्रवाꣴ॑स्युत्त॒मानि॑ धिष्व॥~(३२)

{\anuvakamend[{आ मन्द॑स्व सान॒सिमेका॒न्नच॑त्वारि॒ꣳ॒शच्च॑}]}%~(७)

%4.2.8.1
अ॒भ्य॑स्था॒द्विश्वाः॒ पृत॑ना॒ अरा॑ती॒स्तद॒ग्निरा॑ह॒ तदु॒ सोम॑ आह। बृह॒स्पतिः॑ सवि॒ता तन्म॑ आह पु॒षा मा॑धाथ्सुकृ॒तस्य॑ लो॒के। यदक्र॑न्दः प्रथ॒मं जाय॑मान उ॒द्यन्थ्स॑मु॒द्रादुत वा॒ पुरी॑षात्। श्ये॒नस्य॑ प॒क्षा ह॑रि॒णस्य॑ बा॒हू उप॑स्तुतं॒ जनि॑म॒ तत्ते॑ अर्वन्न्। अ॒पां पृ॒ष्ठम॑सि॒ योनि॑र॒ग्नेः स॑मु॒द्रम॒भितः॒ पिन्व॑मानम्। वर्ध॑मानं म॒हः~(३३)

%4.2.8.2
आ च॒ पुष्क॑रं दि॒वो मात्र॑या व॒रिणा प्र॑थस्व। ब्रह्म॑ जज्ञा॒नं प्र॑थ॒मं पु॒रस्ता॒द्वि सी॑म॒तः सु॒रुचो॑ वे॒न आ॑वः। स बु॒ध्निया॑ उप॒मा अ॑स्य वि॒ष्ठाः स॒तश्च॒ योनि॒मस॑तश्च॒ विवः॑। हि॒र॒ण्य॒ग॒र्भः सम॑वर्त॒ताग्रे॑ भू॒तस्य॑ जा॒तः पति॒रेक॑ आसीत्। स दा॑धार पृथि॒वीं द्यामु॒तेमां कस्मै॑ दे॒वाय॑ ह॒विषा॑ विधेम। द्र॒फ्सश्च॑स्कन्द पृथि॒वीमनु॑~(३४)

%4.2.8.3
द्यामि॒मं च॒ योनि॒मनु॒ यश्च॒ पूर्वः॑। तृ॒तीयं॒ योनि॒मनु॑ स॒ञ्चर॑न्तं द्र॒फ्सं जु॑हो॒म्यनु॑ स॒प्त होत्राः᳚। नमो॑ अस्तु स॒र्पेभ्यो॒ ये के च॑ पृथि॒वीमनु॑। ये अ॒न्तरि॑क्षे॒ ये दि॒वि तेभ्यः॑ स॒र्पेभ्यो॒ नमः॑। ये॑\-ऽदो रो॑च॒ने दि॒वो ये वा॒ सूर्य॑स्य र॒श्मिषु॑। येषा॑म॒फ्सु सदः॑ कृ॒तं तेभ्यः॑ स॒र्पेभ्यो॒ नमः॑। या इष॑वो यातु॒धाना॑नां॒ ये वा॒ वन॒स्पती॒ꣳ॒ रनु॑। ये वा॑व॒टेषु॒ शेर॑ते॒ तेभ्यः॑ स॒र्पेभ्यो॒ नमः॑॥~(३५)

{\anuvakamend[{म॒हो\-ऽनु॑ यातु॒धाना॑ना॒मेका॑\-दश च}]}%~(८)

%4.2.9.1
ध्रु॒वासि॑ ध॒रुणास्तृ॑ता वि॒श्वक॑र्मणा॒ सुकृ॑ता। मा त्वा॑ समु॒द्र उद्व॑धी॒न्मा सु॑प॒र्णो\-ऽव्य॑थमाना पृथि॒वीं दृꣳ॑ह। प्र॒जा\-प॑तिस्त्वा सादयतु पृथि॒व्याः पृ॒ष्ठे व्यच॑स्वतीं॒ प्रथ॑स्वतीं॒ प्रथो॑\-ऽसि पृथि॒व्य॑सि॒ भूर॑सि॒ भूमि॑र॒स्यदि॑तिरसि वि॒श्वधा॑या॒ विश्व॑स्य॒ भुव॑नस्य ध॒र्त्री पृ॑थि॒वीं य॑च्छ पृथि॒वीं दृꣳ॑ह पृथि॒वीं मा हिꣳ॑सी॒र्विश्व॑स्मै प्रा॒णाया॑पा॒नाय॑ व्या॒नायो॑दा॒नाय॑ प्रति॒ष्ठायै᳚~(३६)

%4.2.9.2
च॒रित्रा॑या॒ग्निस्त्वा॒भि पा॑तु म॒ह्या स्व॒स्त्या छ॒र्दिषा॒ शन्त॑मेन॒ तया॑ दे॒वत॑याङ्गिर॒स्वद्ध्रु॒वा सी॑द। काण्डा᳚त्काण्डात् प्र॒रोह॑न्ती॒ परु॑षःपरुषः॒ परि॑। ए॒वा नो॑ दूर्वे॒ प्र त॑नु स॒हस्रे॑ण श॒तेन॑ च। या श॒तेन॑ प्रत॒नोषि॑ स॒हस्रे॑ण वि॒रोह॑सि। तस्या᳚स्ते देवीष्टके वि॒धेम॑ ह॒विषा॑ व॒यम्। अषा॑ढासि॒ सह॑माना॒ सह॒स्वारा॑तीः॒ सह॑स्वारातीय॒तः सह॑स्व॒ पृत॑नाः॒ सह॑स्व पृतन्य॒तः। स॒हस्र॑वीर्या~(३७)

%4.2.9.3
अ॒सि॒ सा मा॑ जिन्व। मधु॒ वाता॑ ऋताय॒ते मधु॑ क्षरन्ति॒ सिन्ध॑वः। माध्वी᳚र्नः स॒न्त्वोष॑धीः। मधु॒ नक्त॑मु॒तोषसि॒ मधु॑म॒त्पार्थि॑व॒ꣳ॒ रजः। मधु॒ द्यौर॑स्तु नः पि॒ता। मधु॑मान्नो॒ वन॒स्पति॒र्मधु॑माꣳ अस्तु॒ सूर्यः॑। माध्वी॒र्गावो॑ भवन्तु नः। म॒ही द्यौः पृ॑थि॒वी च॑ न इ॒मं य॒ज्ञं मि॑मिक्षताम्। पि॒पृ॒तां नो॒ भरी॑मभिः। तद्विष्णोः᳚ पर॒मम्~(३८)

%4.2.9.4
प॒दꣳ सदा॑ पश्यन्ति सू॒रयः॑। दि॒वीव॒ चक्षु॒रात॑तम्। ध्रु॒वासि॑ पृथिवि॒ सह॑स्व पृतन्य॒तः। स्यू॒ता दे॒वेभि॑र॒मृते॒नागाः᳚। यास्ते॑ अग्ने॒ सूर्ये॒ रुच॑ उद्य॒तो दिव॑मात॒न्वन्ति॑ र॒श्मिभिः॑। ताभिः॒ सर्वा॑भी रु॒चे जना॑य नस्कृधि। या वो॑ देवाः॒ सूर्ये॒ रुचो॒ गोष्वश्वे॑षु॒ या रुचः॑। इन्द्रा᳚ग्नी॒ ताभिः॒ सर्वा॑भी॒ रुचं॑ नो धत्त बृहस्पते। वि॒राट्~(३९)

%4.2.9.5
ज्योति॑रधारयथ्स॒म्राड्ज्योति॑रधारयथ्स्व॒राड्ज्योति॑रधारयत्। अग्ने॑ यु॒क्ष्वा हि ये तवाश्वा॑सो देव सा॒धवः॑। अरं॒ वह॑न्त्या॒शवः॑। यु॒क्ष्वा हि दे॑व॒हूत॑मा॒ꣳ॒ अश्वाꣳ॑ अग्ने र॒थीरि॑व। नि होता॑ पू॒र्व्यः स॑दः। द्र॒फ्सश्च॑स्कन्द पृथि॒वीमनु॒ द्यामि॒मं च॒ योनि॒मनु॒ यश्च॒ पूर्वः॑। तृ॒तीयं॒ योनि॒मनु॑ स॒ञ्चर॑न्तं द्र॒फ्सं जु॑हो॒म्यनु॑ स॒प्त~(४०)

%4.2.9.6
होत्राः᳚। अभू॑दि॒दं विश्व॑स्य॒ भुव॑नस्य॒ वाजि॑नम॒ग्नेर्वै᳚श्वान॒रस्य॑ च। अ॒ग्निर्ज्योति॑षा॒ ज्योति॑ष्मान्रु॒क्मो वर्च॑सा॒ वर्च॑स्वान्। ऋ॒चे त्वा॑ रु॒चे त्वा॒ समिथ्स्र॑वन्ति स॒रितो॒ न धेनाः᳚। अ॒न्तर्\mbox{}हृ॒दा मन॑सा पू॒यमा॑नाः। घृ॒तस्य॒ धारा॑ अ॒भि चा॑कशीमि। हि॒र॒ण्ययो॑ वेत॒सो मध्य॑ आसाम्। तस्मि᳚न्थ्सुप॒र्णो म॑धु॒कृत्कु॑ला॒यी भज॑न्नास्ते॒ मधु॑ दे॒वता᳚भ्यः। तस्या॑सते॒ हर॑यः स॒प्त तीरे᳚ स्व॒धां दुहा॑ना अ॒मृत॑स्य॒ धारा᳚म्॥~(४१)

{\anuvakamend[{प्र॒ति॒ष्ठायै॑ स॒हस्र॑वीर्या पर॒मं वि॒राट्थ्स॒प्त तीरे॑ च॒त्वारि॑ च}]}%~(९)

%4.2.10.1
आ॒दि॒त्यं गर्भं॒ पय॑सा सम॒ञ्जन्थ्स॒हस्र॑स्य प्रति॒मां वि॒श्वरू॑पम्। परि॑ वृङ्ग्धि॒ हर॑सा॒ माभि मृ॑क्षः श॒तायु॑षं कृणुहि ची॒यमा॑नः। इ॒मं मा हिꣳ॑सीर्द्वि॒पादं॑ पशू॒नाꣳ सह॑स्राक्ष॒ मेध॒ आ ची॒यमा॑नः। म॒युमा॑र॒ण्यमनु॑ ते दिशामि॒ तेन॑ चिन्वा॒नस्त॒नुवो॒ नि षी॑द। वात॑स्य॒ ध्राजिं॒ वरु॑णस्य॒ नाभि॒मश्वं॑ जज्ञा॒नꣳ स॑रि॒रस्य॒ मध्ये᳚। शिशुं॑ न॒दीना॒ꣳ॒ हरि॒मद्रि॑बुद्ध॒मग्ने॒ मा हिꣳ॑सीः~(४२)

%4.2.10.2
प॒र॒मे व्यो॑मन्न्। इ॒मं मा हिꣳ॑सी॒रेक॑शफं पशू॒नां क॑निक्र॒दं वा॒जिनं॒ वाजि॑नेषु। गौ॒रमा॑र॒ण्यमनु॑ ते दिशामि॒ तेन॑ चिन्वा॒नस्त॒नुवो॒ नि षी॑द। अज॑स्र॒मिन्दु॑मरु॒षं भु॑र॒ण्युम॒ग्निमी॑डे पू॒र्वचि॑त्तौ॒ नमो॑भिः। स पर्व॑भिर्\mbox{}ऋतु॒शः कल्प॑मानो॒ गां मा हिꣳ॑सी॒रदि॑तिं वि॒राजम्᳚। इ॒मꣳ स॑मु॒द्रꣳ श॒तधा॑र॒मुथ्सं॑ व्य॒च्यमा॑नं॒ भुव॑नस्य॒ मध्ये᳚। घृ॒तं दुहा॑ना॒मदि॑तिं॒ जना॒याग्ने॒ मा~(४३)

%4.2.10.3
हि॒ꣳ॒सीः॒ प॒र॒मे व्यो॑मन्न्। ग॒व॒यमा॑र॒ण्यमनु॑ ते दिशामि॒ तेन॑ चिन्वा॒नस्त॒नुवो॒ नि षी॑द। वरू᳚त्रिं॒ त्वष्टु॒र्वरु॑णस्य॒ नाभि॒मविं॑ जज्ञा॒नाꣳ रज॑सः॒ पर॑स्मात्। म॒हीꣳ सा॑ह॒स्रीमसु॑रस्य मा॒यामग्ने॒ मा हिꣳ॑सीः॒ पर॒मे व्यो॑मन्न्। इ॒मामू᳚र्णा॒युं वरु॑णस्य मा॒यां त्वचं॑ पशू॒नां द्वि॒पदां॒ चतु॑ष्पदाम्। त्वष्टुः॑ प्र॒जानां᳚ प्रथ॒मं ज॒नित्र॒मग्ने॒ मा हिꣳ॑सीः पर॒मे व्यो॑मन्न्। उष्ट्र॑मार॒ण्यमनु॑~(४४)

%4.2.10.4
ते॒ दि॒शा॒मि॒ तेन॑ चिन्वा॒नस्त॒नुवो॒ नि षी॑द। यो अ॒ग्निर॒ग्नेस्तप॒सो\-ऽधि॑ जा॒तः शोचा᳚त्पृथि॒व्या उ॒त वा॑ दि॒वस्परि॑। येन॑ प्र॒जा वि॒श्वक॑र्मा॒ व्यान॒ट्तम॑ग्ने॒ हेडः॒ परि॑ ते वृणक्तु। अ॒जा ह्य॑ग्नेरज॑निष्ट॒ गर्भा॒थ्सा वा अ॑पश्यज्जनि॒तार॒मग्रे᳚। तया॒ रोह॑माय॒न्नुप॒ मेध्या॑स॒स्तया॑ दे॒वा दे॒वता॒मग्र॑ आयन्न्। श॒र॒भमा॑र॒ण्यमनु॑ ते दिशामि॒ तेन॑ चिन्वा॒नस्त॒नुवो॒ नि षी॑द॥~(४५)

{\anuvakamend[{अग्ने॒ मा हिꣳ॑सी॒रग्ने॒ मोष्ट्र॑मार॒ण्यमनु॑ शर॒भं नव॑ च}]}%॥10॥ आ॒दि॒त्यमि॒मं द्वि॒पाद॑म्म॒युं वात॒स्याश्व॑मि॒ममेक॑शफङ्गौ॒रमज॑स्रङ्गव॒यं वरू᳚त्रि॒मवि॑मि॒मामू᳚र्णा॒युमुष्ट्रं॒ यो अ॒ग्निर॒ग्नेः श॑र॒भम्॥

%4.2.11.1
इन्द्रा᳚ग्नी रोच॒ना दि॒वः परि॒ वाजे॑षु भूषथः। तद्वां᳚ चेति॒ प्र वी॒र्यम्᳚। श्ञथ॑द्वृ॒त्रमु॒त स॑नोति॒ वाज॒मिन्द्रा॒ यो अ॒ग्नी सहु॑री सप॒र्यात्। इ॒र॒ज्यन्ता॑ वस॒व्य॑स्य॒ भूरेः॒ सह॑स्तमा॒ सह॑सा वाज॒यन्ता᳚। प्र च॑र्\mbox{}ष॒णिभ्यः॑ पृतना॒ हवे॑षु॒ प्र पृ॑थि॒व्या रि॑रिचाथे दि॒वश्च॑। प्र सिन्धु॑भ्यः॒ प्र गि॒रिभ्यो॑ महि॒त्वा प्रेन्द्रा᳚ग्नी॒ विश्वा॒ भुव॒नात्य॒न्या। मरु॑तो॒ यस्य॒ हि~(४६)

%4.2.11.2
क्षये॑ पा॒था दि॒वो वि॑महसः। स सु॑गो॒पात॑मो॒ जनः॑। य॒ज्ञैर्वा॑ यज्ञवाहसो॒ विप्र॑स्य वा मती॒नाम्। मरु॑तः शृणु॒ता हवम्᳚। श्रि॒यसे॒ कं भा॒नुभिः॒ सम्मि॑मिक्षिरे॒ ते र॒श्मिभि॒स्त ऋक्व॑भिः सुखा॒दयः॑। ते वाशी॑मन्त इ॒ष्मिणो॒ अभी॑रवो वि॒द्रे प्रि॒यस्य॒ मारु॑तस्य॒ धाम्नः॑। अव॑ ते॒ हेड॒ उदु॑त्त॒मम्। कया॑ नश्चि॒त्र आ भु॑वदू॒ती स॒दावृ॑धः॒ सखा᳚। कया॒ शचि॑ष्ठया वृ॒ता।~(४७)

%4.2.11.3
को अ॒द्य यु॑ङ्क्ते धु॒रि गा ऋ॒तस्य॒ शिमी॑वतो भा॒मिनो॑ दुर्\mbox{}हृणा॒यून्। आ॒सन्नि॑षून् हृ॒थ्स्वसो॑ मयो॒भून् य ए॑षां भृ॒त्यामृ॒णध॒थ्स जी॑वात्। अग्ने॒ नया दे॒वाना॒ꣳ॒ शं नो॑ भवन्तु॒ वाजे॑वाजे। अ॒फ्स्व॑ग्ने॒ सधि॒ष्टव॒ सौष॑धी॒रनु॑ रुध्यसे। गर्भे॒ सञ्जा॑यसे॒ पुनः॑। वृषा॑ सोम द्यु॒माꣳ अ॑सि॒ वृषा॑ देव॒ वृष॑व्रतः। वृषा॒ धर्मा॑णि दधिषे। इ॒मं मे॑ वरुण तत्त्वा॑ यामि॒ त्वं नो॑ अग्ने॒ स त्वं नो॑ अग्ने॥~(४८)

{\anuvakamend[{हि वृ॒ता म॒ एका॑\-दश च}]}%॥11॥

\prashnaend{विष्णोः॒ क्रमो॑\-ऽसि दि॒वस्पर्यन्न॑प॒ते\-ऽपे॑त॒ समि॑तं॒ या जा॒ता मा नो॑ हिꣳसीद्ध्रु॒वा\-ऽस्या॑दि॒त्यङ्गर्भ॒मिन्द्रा᳚ग्नी रोच॒नैका॑\-दश॥११॥}{विष्णो॑रस्मिन् ह॒व्येति॑ त्वा॒\-ऽहं धी॒तिभि॒र्॒\mbox{}होत्रा॑ अ॒ष्टाच॑त्वारिꣳशत्॥४८॥}{विष्णोः॒ क्रमो॑\-ऽसि॒ स त्वन्नो॑ अग्ने॥}%%४-२
{हरिः॑ ॐ}{॥कृष्ण-यजुर्वेदीय-तैत्तिरीय-संहितायां चतुर्थकाण्डे द्वितीयः प्रश्नः समाप्तः॥४-२॥}
%%% END PRASHNA
