% !TeX program = XeLaTeX
% !TeX root = vedamantrabook.tex
\pagenumbering{Roman}
\thispagestyle{empty}\clearpage
\begin{titlepage}
\vspace*{6.5cm}
\begin{center}
{\font\x="Sanskrit 2003:script=deva,mapping=tex-text" at 64pt \x \titletext}
\end{center}
\end{titlepage}

\begin{center}
\parbox{10cm}{
\fontspec{Candara}
{\Large \textbf{Colophon}}\\
\noindent This document was typeset using \XeLaTeX, and uses the Siddhanta font extensively. It also uses several \LaTeX\ macros designed by \textit{H.~L.~Prasād}. Practically all the encoding was done with the help of Ajit Krishnan's mudgala IME (\url{http://www.aupasana.com/}).
\vspace*{3cm}

{\large \textbf{Acknowledgements}}\\
The initial ITRANS encodings of some of these texts were obtained from \url{http://sanskritdocuments.org/} and \url{https://sa.wikisource.org/}. Thanks are also due to Ulrich Stiehl~ (\url{http://sanskritweb.de/}) ~for hosting a wonderful resource for Yajur Veda, and also generously sharing the original Kathaka texts edited by Subramania Sarma. \\
See also \url{http://stotrasamhita.github.io/about/}
\vspace*{1cm}
%
\begin{center}
{\scshape{For Personal Use Only\\
 Not For Commercial Printing/Distribution}}
\end{center}
}
\end{center}
\clearemptydoublepage %
\setcounter{page}{0} %
\pagenumbering{roman} %
\renewcommand{\chaptermark}[1]{%
\markboth{\large #1}{}} %
\pdfbookmark[1]{Contents}{Contents} %
%\begin{center}
\begin{large}
\tableofcontents
\end{large}
%\end{center}

\mbox{}
\clearpage
\thispagestyle{empty}
\clearemptydoublepage
