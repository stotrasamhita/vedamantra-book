% !TeX program = XeLaTeX
% !TeX root = ../AraNyakabook-kindle.tex
\sect{दशमः प्रश्नः --- महानारायणोपनिषत्}\setcounter{anuvakam}{0}

%६.१.०

%६.१.१
अम्भ॑स्यपा॒रे भुव॑नस्य॒ मध्ये॒ नाक॑स्य पृ॒ष्ठे म॑ह॒तो मही॑यान्। शु॒क्रेण॒ ज्योतीषि समनु॒प्रवि॑ष्टः प्र॒जाप॑तिश्चरति॒ गर्भे॑ अ॒न्तः॥ यस्मि॑न्नि॒द सं च॒ विचैति॒ सर्वं॒ यस्मि॑न्दे॒वा अधि॒ विश्वे॑ निषे॒दुः। तदे॒व भू॒तं तदु॒ भव्य॑मा इ॒दं तद॒क्षरे॑ पर॒मे व्यो॑मन्॥ येना॑वृ॒तं खं च॒ दिवं॑ म॒हीं च॒ येना॑दि॒त्यस्तप॑ति॒ तेज॑सा॒ भ्राज॑सा च। यम॒न्तः स॑मु॒द्रे क॒वयो॒ वय॑न्ति॒ यद॒क्षरे॑ पर॒मे प्र॒जाः॥ यत॑ प्रसू॒ता ज॒गत॑ प्रसूती॒ तोये॑न जी॒वान्व्यस॑सर्ज॒ भूम्याम्। यदोष॑धीभिः पु॒रुषान्प॒शूश्च॒ विवे॑श भू॒तानि॑ चराच॒राणि॑॥ अत॑ परं॒ नान्य॒दणी॑यस हि॒ परात्परं॒ यन्मह॑तो म॒हान्तम्। यदे॑कम॒व्यक्त॒मन॑न्तरूपं॒  विश्वं॑ पुरा॒णं तम॑सः॒ पर॑स्तात्॥१॥

%६.१.२
तदे॒वर्तं तदु॑ स॒त्यमा॑हु॒स्तदे॒व ब्रह्म॑ पर॒मं क॑वी॒नाम्। इ॒ष्टा॒पू॒र्तं ब॑हु॒धा जा॒तं जाय॑मानं  वि॒श्वं बि॑भर्ति॒ भुव॑नस्य॒ नाभि॑॥ तदे॒वाग्निस्तद्वा॒युस्तत्सूर्य॒स्तदु॑ च॒न्द्रमा। तदे॒व शु॒क्रम॒मृतं॒ तद्ब्रह्म॒ तदापः॒ स प्र॒जाप॑तिः॥ सर्वे॑ निमे॒षा ज॒ज्ञिरे॑ वि॒द्युतः॒ पुरु॑षा॒दधि॑। क॒ला मु॑हू॒र्ताः काष्ठाश्चाहोरा॒त्राश्च॑ सर्व॒शः॥ अ॒र्द्ध॒मा॒सा मासा॑ ऋ॒तव॑ संवत्स॒रश्च॑ कल्पताम्। स आप॑ प्रदु॒घे उ॒भे इ॒मे अ॒न्तरि॑क्ष॒मथो॒ सुव॑॥ नैन॑मू॒र्ध्वं न ति॒र्यञ्चं॒ न मध्ये॒ परि॑जग्रभत्। न तस्ये॑शे॒ कश्च॒न तस्य॑ नाम म॒हद्यश॑॥२॥

%६.१.३
न स॒न्दृशे॑ तिष्ठति॒ रूप॑मस्य॒ न चक्षु॑षा पश्यति॒ कश्च॒नैनम्। हृ॒दा म॑नी॒षा मन॑सा॒ऽभिक्लृ॑प्तो॒ य ए॑नं  वि॒दुरमृ॑ता॒स्ते भ॑वन्ति॥ अ॒द्भ्यः सम्भू॑तो हिरण्यग॒र्भ इत्य॒ष्टौ॥ ए॒ष हि दे॒वः प्र॒दिशोनु॒ सर्वाः॒ पूर्वो॑ हि जा॒तः स उ॒ गर्भे॑ अ॒न्तः। स वि॒जाय॑मानः स जनि॒ष्यमा॑णः प्र॒त्यङ्मुखास्तिष्ठति वि॒श्वतो॑मुखः॥ वि॒श्वत॑श्चक्षुरु॒त वि॒श्वतो॑मुखो वि॒श्वतो॑हस्त उ॒त वि॒श्वत॑स्पात्। सं बा॒हुभ्यां॒ नम॑ति॒ सं पत॑त्रै॒र्द्यावा॑पृथि॒वी ज॒नय॑न्दे॒व एक॑॥ वे॒नस्तत्पश्य॒न्विश्वा॒ भुव॑नानि वि॒द्वान् यत्र॒ विश्वं॒ भव॒त्येक॑नीळम्। यस्मि॑न्नि॒द सं च॒ विचैक॒ स ओतः॒ प्रोत॑श्च वि॒भुः प्र॒जासु॑। प्र तद्वो॑चे अ॒मृतं॒ नु वि॒द्वान्ग॑न्ध॒र्वो नाम॒ निहि॑त॒ङ्गुहा॑सु॥३॥

%६.१.४
त्रीणि॑ प॒दा निहि॑ता॒ गुहा॑सु॒ यस्तद्वेद॑ सवि॒तुः पि॒ताऽस॑त्। स नो॒ बन्धु॑र्जनि॒ता स वि॑धा॒ता धामा॑नि॒ वेद॒ भुव॑नानि॒ विश्वा। यत्र॑ दे॒वा अ॒मृत॑मानशा॒नास्तृ॒तीये॒ धामान्य॒भ्यैर॑यन्त। परि॒ द्यावा॑पृथि॒वी य॑न्ति स॒द्यः परि॑ लो॒कान् परि॒ दिशः॒ परि॒ सुव॑। ऋ॒तस्य॒ तन्तुं॑  विततं  वि॒चृत्य॒ तद॑पश्य॒त्तद॑भवत् प्र॒जासु॑। प॒रीत्य॑ लो॒कान्प॒रीत्य॑ भू॒तानि॑ प॒रीत्य॒ सर्वा प्र॒दिशो॒ दिश॑श्च। प्र॒जाप॑तिः प्रथम॒जा ऋ॒तस्य॒\aav{}\aav{}त्मन॒\aav{}\aav{}त्मान॑म॒भिसम्ब॑भूव। सद॑स॒स्पति॒मद्भु॑तं प्रि॒यमिन्द्र॑स्य॒ काम्यम्। सनिं॑ मे॒धाम॑यासिषम्। उद्दीप्यस्व जातवेदोऽप॒घ्नन्निर्\mbox{}ऋ॑तिं॒ मम॑॥४॥

%६.१.५
प॒शूश्च॒ मह्य॒माव॑ह॒ जीव॑नं च॒ दिशो॑ दिश। मा नो॑ हिसीज्जातवेदो॒ गामश्वं॒ पुरु॑षं॒ जग॑त्। अबि॑भ्र॒दग्न॒ आग॑हि श्रि॒या मा॒ परि॑पातय। पुरु॑षस्य विद्म सहस्रा॒क्षस्य॑ महादे॒वस्य॑ धीमहि। तन्नो॑ रुद्रः प्रचो॒दयात्। तत्पुरु॑षाय वि॒द्महे॑ महादे॒वाय॑ धीमहि। तन्नो॑ रुद्रः प्रचो॒दयात्। तत्पुरु॑षाय वि॒द्महे॑ वक्रतु॒ण्डाय॑ धीमहि। तन्नो॑ दन्तिः प्रचो॒दयात्। तत्पुरु॑षाय वि॒द्महे॑ सुवर्णप॒क्षाय॑ धीमहि॥५॥

%६.१.६
तन्नो॑ गरुडः प्रचो॒दयात्। का॒त्या॒य॒नाय॑ वि॒द्महे॑ कन्यकु॒मारि॑ धीमहि। तन्नो॑ दुर्गिः प्रचो॒दयात्। ना॒रा॒य॒णाय॑ वि॒द्महे॑ वासुदे॒वाय॑ धीमहि। तन्नो॑ विष्णुः प्रचो॒दयात्। स॒ह॒स्र॒पर॑मा दे॒वी॒ श॒तमू॑ला श॒ताङ्कु॑रा। सर्व हरतु॑ मे पा॒प॒न्दू॒र्वा दु॑स्वप्न॒नाशि॑नी। अश्व॑क्रा॒न्ते र॑थक्रा॒न्ते॒ वि॒ष्णुक्रान्ते व॒सुन्ध॑रा। शिरसा॑ धारि॑ता दे॒वी॒ र॒क्ष॒स्व मां पदे॒ पदे। उ॒द्धृता॑सि व॑राहे॒ण॒ कृ॒ष्णे॒न श॑तबा॒हुना॥६॥

%६.१.७
भूमिर्द्धेनुर्धरणी लो॑कधा॒रिणी। मृ॒त्तिके॑ हन॑ मे पा॒पं॒ य॒न्म॒या दु॑ष्कृतं॒ कृतम्। त्वया॑ ह॒तेन॑ पापे॒न॒ जी॒वा॒मि श॑रदः॒ शतम्। मृ॒त्तिके॑ देहि॑ मे पु॒ष्टिं॒ त्व॒यि स॑र्वं प्र॒तिष्ठि॑तम्। ग॒न्ध॒द्वा॒रान्दु॑राध॒र्\mbox{}षान्नि॒त्यपु॑ष्टां करी॒षिणीम्। ई॒श्वरी सर्व॑भूता॒नां॒ त्वामि॒होप॑ह्वये॒ श्रियम्। हिर॑ण्यशृङ्गं॒ वरु॑णं॒ प्रप॑द्ये ती॒र्थं मे देहि॒ याचि॑तः। य॒न्मया॑ भु॒क्तम॒साधू॑नां पा॒पेभ्य॑श्च प्र॒तिग्र॑हः। यन्मे॒ मन॑सा वा॒चा॒ क॒र्म॒णा वा दु॑ष्कृतं॒ कृतम्। तन्न॒ इन्द्रो॒ वरु॑णो॒ बृह॒स्पति॑ सवि॒ता च॑ पुनन्तु॒ पुन॑ पुनः॥७॥

%६.१.८
सु॒मि॒त्रा न॒ आप॒ ओष॑धयः सन्तु दुर्मि॒त्रास्तस्मै॑ भूयासु॒र्योऽस्मान्द्वेष्टि॒ यं च॑ व॒यं द्वि॒ष्मः। नमो॒ऽग्नयेऽप्सु॒मते॒ नम॒ इन्द्रा॑य॒ नमो॒ वरु॑णाय॒ नमो वारुण्यै॑ नमो॒ऽद्भ्यः। यद॒पां क्रू॒रं यद॑मे॒ध्यं यद॑शा॒न्तं तदप॑गच्छतात्। अ॒त्या॒श॒नाद॑तीपा॒ना॒द्य॒च्च उ॒ग्रात् प्र॑ति॒ग्रहात्। तन्मे॒ वरु॑णो रा॒जा॒ पा॒णिना ह्यव॒मर्\mbox{}श॑तु। सो॑ऽहम॑पा॒पो वि॒रजो॒ निर्मु॒क्तो मु॑क्तकि॒ल्बिषः। नाक॑स्य पृ॒ष्ठमारु॑ह्य॒ गच्छे॒द्ब्रह्म॑सलो॒कताम्। इ॒मं मे॑ गङ्गे यमुने सरस्वति॒ शुतु॑द्रि॒ स्तोम सचता॒ परु॒ष्णिया। अ॒सि॒क्नि॒या म॑रुद्वृधे वि॒तस्त॒याऽऽर्जी॑कीये श्रुणु॒ह्या सु॒षोम॑या। ऋ॒तं च॑ स॒त्यं चा॒भीद्धा॒त्तप॒सोऽध्य॑जायत॥८॥

%६.१.९
ततो॒ रात्रि॑रजायत॒ तत॑ समु॒द्रो अ॑र्ण॒वः। स॒मु॒द्राद॑र्ण॒वादधि॑ संवत्स॒रो अ॑जायत। अ॒हो॒रा॒त्राणि॑ वि॒दध॒द्विश्व॑स्य मिष॒तो व॒शी। सू॒र्या॒च॒न्द्र॒मसौ॑ धा॒ता य॑थापू॒र्वम॑कल्पयत्। दिवं॑ च पृथि॒वीं चा॒न्तरि॑क्ष॒मथो॒ सुव॑। यत्पृ॑थि॒व्या रज॑स्व॒ मान्तरि॑क्षे वि॒रोद॑सी। इ॒मास्तदा॒पो व॑रुणः पु॒नात्व॑घमर्\mbox{}ष॒णः। ए॒ष भू॒तस्य॑ भ॒व्ये भुव॑नस्य गो॒प्ता। ए॒ष पु॒ण्यकृ॑तां लो॒का॒ने॒ष मृ॒त्योर्\mbox{}हि॑र॒ण्मयम्। द्यावा॑पृथि॒व्योर्\mbox{}हि॑र॒ण्मय॒ सश्रि॑त॒ सुव॑॥९॥

%६.१.१०
स नः॒ सुवः॒ सशि॑शाधि। आर्द्रं॒ ज्वल॑ति॒ ज्योति॑र॒हम॑स्मि। ज्योति॒र्ज्वल॑ति॒ ब्रह्मा॒हम॑स्मि। यो॑ऽहम॑स्मि॒ ब्रह्मा॒हम॑स्मि। अ॒हमे॒वाहं मां जु॑होमि॒ स्वाहा। अ॒का॒र्य॒का॒र्य॑वकी॒र्णी स्ते॒नो भ्रू॑ण॒हा गु॑रुत॒ल्पगः। वरु॑णो॒ऽपाम॑घमर्\mbox{}ष॒णस्तस्मात्पा॒पात् प्रमु॑च्यते। र॒जो भूमि॑स्त्व॒मा रोद॑यस्व॒ प्रव॑दन्ति॒ धीरा। पु॒नन्तु॒ ऋष॑यः पु॒नन्तु॒ वस॑वः पु॒नातु॒ वरु॑णः पु॒नात्व॑घमर्\mbox{}ष॒णः। आक्रान्त्समु॒द्रः प्र॑थ॒मे विध॑र्मं ज॒नय॑न्प्र॒जा भुव॑नस्य॒ राजा॥१०॥

%६.१.११
वृषा॑ प॒वित्रे॒ अधि॒ सानो॒ अव्ये॑ बृ॒हत्सोमो॑ वावृधे सुवा॒न इन्दु॑। जा॒तवे॑दसे सुनवाम॒ सोम॑मरातीय॒तो निज॑हाति॒ वेद॑। स न॑ पर्\mbox{}ष॒दति॑ दु॒र्गाणि॒ विश्वा॑ ना॒वेव॒ सिन्धुं॑ दुरि॒ताऽत्य॒ग्निः। ताम॒ग्निव॑र्णां॒ तप॑सा ज्वल॒न्तीं॒ वै॑रोच॒नीं क॑र्मफ॒लेषु॒ जुष्टाम्। दु॒र्गान्दे॒वी शर॑णम॒हं प्रप॑द्ये सु॒तर॑सि तरसे॒ नम॑। अग्ने॒ त्वं पा॑रया॒ नव्यो॑ अ॒स्मान्त्स्व॒स्तिभि॒रति॑ दु॒र्गाणि॒ विश्वा। पूश्च॑ पृ॒थ्वी ब॑हु॒ला न॑ उ॒र्वी भवा॑ तो॒काय॒ तन॑याय॒ शँय्योः। विश्वा॑नि नो दु॒र्गहा॑ जातवेदः॒ सिन्धुं॒ न ना॒वा दु॑रि॒ताति॑ पर्\mbox{}षि। अग्ने॑ अत्रि॒वन्मन॑सा गृणा॒नोऽस्माकं॑ बोध्यवि॒ता त॒नूनाम्। पृ॒त॒ना॒जित॒ सह॑मानम॒ग्निमु॒ग्र हु॑वेम पर॒मात्स॒धस्थात्। स न॑ पर्\mbox{}ष॒दति॑ दु॒र्गाणि॒ विश्वा॒ क्षाम॑द्दे॒वो अति॑ दुरि॒ताऽत्य॒ग्निः। प्र॒त्नोषि॑ क॒मीड्यो॑ अध्व॒रेषु॑ स॒नाच्च॒ होता॒ नव्य॑श्च॒ सत्सि॑। स्वाञ्चाग्ने त॒नुवं॑ पि॒प्रय॑स्वा॒स्मभ्यं॑ च॒ सौभ॑ग॒माय॑जस्व॥११॥
\anuvakamend[पर॑स्ता॒द्यशो॒ गुहा॑सु॒ मम॑ सुवर्णप॒क्षाय॑ धीमहि शतबा॒हुना पुन॑ पुनरजायत॒ सुवो॒ राजा॑ स॒धस्था॒त्त्रीणि॑ च। १।]



%६.२.१
भूर॒ग्नये॑ पृथि॒व्यै स्वाहा॒ भुवो॑ वा॒यवे॒ऽन्तरि॑क्षाय॒ स्वाहा॒ सुव॑रादि॒त्याय॑ दि॒वे स्वाहा॒ भूर्भुवः॒ सुव॑श्च॒न्द्रम॑से दि॒ग्भ्यः स्वाहा॒ नमो॑ दे॒वेभ्य॑ स्व॒धा पि॒तृभ्यो॒ भूर्भुवः॒ सुव॒रोम्॥१२॥
\anuvakamend


%६.३.१
भूरन्न॑म॒ग्नये॑ पृथि॒व्यै स्वाहा॒ भुवोऽन्नं॑ वा॒यवे॒ऽन्तरि॑क्षाय॒ स्वाहा॒ सुव॒रन्न॑मादि॒त्याय॑ दि॒वे स्वाहा॒ भूर्भुवः॒ सुव॒रन्नं॑ च॒न्द्रम॑से दि॒ग्भ्यः स्वाहा॒ नमो॑ दे॒वेभ्य॑ स्व॒धा पि॒तृभ्यो॒ भूर्भुवः॒ सुव॒रन्न॒मोम्॥१३॥
\anuvakamend


%६.४.१
भूर॒ग्नये॑ च पृथि॒व्यै च॑ मह॒ते च॒ स्वाहा॒ भुवो॑ वा॒यवे॑ चा॒न्तरि॑क्षाय च मह॒ते च॒ स्वाहा॒ सुव॑रादि॒त्याय॑ च दि॒वे च॑ मह॒ते च॒ स्वाहा॒ भूर्भुवः॒ सुव॑श्च॒न्द्रम॑से च॒ नक्ष॑त्रेभ्यश्च दि॒ग्भ्यश्च॑ मह॒ते च॒ स्वाहा॒ नमो॑ दे॒वेभ्य॑ स्व॒धा पि॒तृभ्यो॒ भूर्भुवः॒ सुव॒र्मह॒रोम्॥१४॥ %६.५.०
\anuvakamend


%६.५.१
पाहि नो अग्न एन॑से स्वा॒हा। पाहि नो विश्ववेद॑से स्वा॒हा। यज्ञं पाहि विभाव॑सो स्वा॒हा। सर्वं पाहि शतक्र॑तो स्वा॒हा॥१५॥
%६.६.०
\anuvakamend


%६.६.१
यश्छन्द॑सामृष॒भो वि॒श्वरू॑प॒श्छन्दोभ्य॒श्छन्दास्यावि॒वेश॑। सता शिक्यः पुरोवाचो॑पनि॒षदिन्द्रो ज्ये॒ष्ठ इ॑न्द्रि॒याय॒ ऋषि॑भ्यो॒ नमो॑ दे॒वेभ्य॑ स्व॒धा पि॒तृभ्यो॒ भूर्भुवः॒ सुव॒रोम्॥१६॥
%६.७.०
\anuvakamend


%६.७.१
नमो॒ ब्रह्म॑णे धा॒रणं॑ मे अ॒स्त्वनि॑राकरणन्धा॒रयि॑ता भूयासं॒ कर्ण॑योः  श्रु॒तं मा च्योढ्वं॒ ममा॒मुष्य॒ ओम्॥१७॥
%६.८.०
\anuvakamend


%६.८.१
ऋ॒तं तप॑ स॒त्यं तप॑ श्रु॒तं तप॑ शा॒न्तं तपो॒ दानं॒ तपो॒ यज्ञ॒स्तपो॒ भूर्भुवः॒ सुव॒र्ब्रह्मै॒तदुपास्यै॒तत्तप॑॥१८॥
\anuvakamend


%६.९.०

%६.९.१
यथा॑ वृ॒क्षस्य॑ सं॒पुष्पि॑तस्य दू॒राद्ग॒न्धो वात्ये॒वं पुण्य॑स्य क॒र्मणो॑ दू॒राद्ग॒न्धो वा॑ति॒ यथा॑ऽसिधा॒रां क॒र्तेऽव॑हितामव॒क्रामे॒द्यद्युवे॒ युवे॒ ह वा॑ वि॒ह्वदि॑ष्यामि क॒र्तं प॑तिष्या॒मीत्ये॒वम॒नृता॑दा॒त्मानं॑ जु॒गुप्सेत्॥१९॥

%६.१०.०
अजोन्यः सुव॒ नाभिः॒ सर्व॑म॒ष्टौ च॑॥ १०।
\anuvakamend


%६.१०.१
अ॒णोरणी॑यान्मह॒तो मही॑याना॒त्मा गुहा॑यां॒ निहि॑तोऽस्य ज॒न्तोः। तम॑क्रतुं पश्यति वीतशो॒को धा॒तुः प्र॒सादान्महि॒मान॑मीशम्। स॒प्त प्रा॒णाः प्र॒भव॑न्ति॒ तस्मात्स॒प्तार्चिष॑ स॒मिध॑ स॒प्त जि॒ह्वाः। स॒प्त इ॒मे लो॒का येषु॒ चर॑न्ति प्रा॒णा गु॒हाश॑यां॒ निहि॑ताः स॒प्तस॑प्त। अत॑ समु॒द्रा गि॒रय॑श्च॒ सर्वे॒ऽस्मात्स्यन्द॑न्ते॒ सिन्ध॑वः॒ सर्व॑रूपाः। अत॑श्च॒ विश्वा॒ ओष॑धयो॒ रसाश्च॒ येनै॑ष भू॒तस्ति॑ष्ठत्यन्तरा॒त्मा। ब्र॒ह्मा दे॒वानां पद॒वीः क॑वी॒नामृषि॒र्विप्रा॑णां महि॒षो मृ॒गाणाम्। श्ये॒नो गृध्रा॑णा॒ स्वधि॑ति॒र्वना॑ना॒ सोम॑ प॒वित्र॒मत्ये॑ति॒ रेभ\sn{}। अ॒जामेकाँ॒ल्लोहि॑तशुक्लकृ॒ष्णां ब॒ह्वीं प्र॒जां ज॒नय॑न्ती॒ सरू॑पाम्। अ॒जो ह्येको॑ जु॒षमा॑णोऽनु॒शेते॒ जहात्येनां भु॒क्तभो॑गा॒मजोऽन्यः॥२०॥

%६.१०.२
ह॒सः  शु॑चि॒षद्वसु॑रन्तरिक्ष॒सद्धोता॑ वेदि॒षदति॑थिर्दुरोण॒सत्। नृ॒षद्व॑र॒सदृ॑त॒सद्व्यो॑म॒सद॒ब्जा गो॒जा ऋ॑त॒जा अ॑द्रि॒जा ऋ॒तं बृ॒हत्। यस्माज्जा॒ता न प॒रा नैव॒ किं च॒नास॒ य आ॑वि॒वेश॒ भुव॑नानि॒ विश्वा। प्र॒जाप॑तिः प्र॒जया॑ संविदा॒नस्त्रीणि॒ ज्योतीषि सचते॒ स षो॑ड॒शी। वि॒ध॒र्तार हवामहे॒ वसो कु॒विद्व॒नाति॑ नः। स॒वि॒तार॑न्नृ॒चक्ष॑सम्। अ॒द्या नो॑ देव सवितः प्र॒जाव॑त्सावीः॒ सौभ॑गम्। परा॑ दु॒ष्वप्नि॑य सुव। विश्वा॑नि देव सवितर्दुरि॒तानि॒ परा॑ सुव। यद्भ॒द्रं तन्म॒ आ सु॑व॥२१॥

%६.१०.३
मधु॒ वाता॑ ऋताय॒ते मधु॑ क्षरन्ति॒ सिन्ध॑वः। माध्वीर्नः स॒न्त्वोष॑धीः। मधु॒ नक्त॑मु॒तोषसि॒ मधु॑म॒त्पार्थि॑व॒ रज॑। मधु॒ द्यौर॑स्तु नः पि॒ता। मधु॑मान्नो॒ वन॒स्पति॒र्मधु॑मा अस्तु॒ सूर्य॑। माध्वी॒र्गावो॑ भवन्तु नः। घृ॒तं मि॑मिक्षे घृ॒तम॑स्य॒ योनि॑र्घृ॒ते श्रि॒तो घृ॒तमु॑वस्य॒ धाम॑। अ॒नु॒ष्व॒धमाव॑ह मा॒दय॑स्व॒ स्वाहा॑कृतं॑ वृषभ वक्षि ह॒व्यम्। स॒मु॒द्रादू॒र्मिमधु॑मा॒ उदा॑रदुपा॒शुना॒ सम॑मृत॒त्वमा॑नट्। घृ॒तस्य॒ नाम॒ गुह्यं॒ यदस्ति॑ जि॒ह्वा दे॒वाना॑म॒मृत॑स्य॒ नाभि॑॥२२॥

%६.१०.४
व॒यं नाम॒ प्रब्र॑वामा घृ॒तेना॒स्मिन् य॒ज्ञे धा॑रयामा॒ नमो॑भिः। उप॑ ब्र॒ह्माशृ॑णवच्छ॒स्यमा॑नं॒ चतु॑ शृङ्गोवमीद्गौ॒र ए॒तत्। च॒त्वारि॒ शृङ्गा॒ त्रयो॑ अस्य॒ पादा॒ द्वे शी॒र्\mbox{}षे स॒प्त हस्ता॑सो अ॒स्य। त्रिधा॑ ब॒द्धो वृ॑ष॒भो रो॑रवीति म॒हो दे॒वो मर्त्या॒ आवि॑वेश। त्रिधा॑ हि॒तं प॒णिभि॑र्गु॒ह्यमा॑न॒ङ्गवि॑ दे॒वासो॑ घृ॒तमन्व॑विन्दन्। इन्द्र॒ एक॒ सूर्य॒ एकं॑ जजान वे॒नादेक स्व॒धया॒ निष्ट॑तक्षुः। यो दे॒वानां प्रथ॒मं पु॒रस्ता॒द्विश्वा॒धिको॑ रु॒द्रो म॒हर्\mbox{}षि॑। हि॒र॒ण्य॒ग॒र्भं प॑श्यत॒ जाय॑मान॒ स नो॑ दे॒वः  शु॒भया॒ स्मृत्या॒ संयु॑नक्तु। यस्मा॒त्परं॒ नाप॑र॒मस्ति॒ किञ्चि॒द्यस्मां॒ नाणी॑यो॒ न ज्यायोऽस्ति॒ कश्चि॑त्। वृ॒क्ष इ॑व स्तब्धो दि॒वि ति॑ष्ठ॒त्येक॒स्तेने॒दं पू॒र्णं पुरु॑षेण॒ सर्वम्॥२३॥

%६.१०.५
न कर्म॑णा न प्र॒जया॒ धने॑न॒ त्यागे॑नैके अमृत॒त्वमा॑न॒शुः। परे॑ण॒ नाकं॒ निहि॑त॒ङ्गुहा॑यां  वि॒भ्राज॑ते॒ यद्यत॑यो वि॒शन्ति॑। वे॒दा॒न्त॒वि॒ज्ञान॒सुनि॑श्चिता॒र्थाः सन्न्या॑सयो॒गाद्यत॑यः  शुद्ध॒सत्वा। ते ब्र॑ह्मलो॒के तु॒ परान्तकाले॒ परा॑मृता॒त्परि॑मुच्यन्ति॒ सर्वे। द॒ह्रं॒  वि॒पा॒पं प॒रमेश्मभूतं॒ यत्पु॑ण्डरी॒कं पु॒रम॑ध्यस॒स्थम्। त॒त्रा॒पि द॒ह्रङ्ग॒गनं॑  विशोक॒स्तस्मि॑न् यद॒न्तस्तदुपा॑सित॒व्यम्। यद्वेदादौ स्व॑रः प्रो॒क्तो॒ वे॒दान्ते॑ च प्र॒तिष्ठि॑तः। तस्य॑ प्र॒कृति॑लीन॒स्य॒ यः॒ पर॑ स म॒हेश्व॑रः॥२४॥

%६.११.०
ना॒रा॒य॒णः स्थि॑तो व्य॒वस्थि॑तश्च॒त्वारि॑ च॥ ११॥
\anuvakamend


%६.११.१
स॒ह॒स्र॒शीर्॑षं  दे॒वं॒  वि॒श्वाक्षं॑  वि॒श्वश॑म्भुवम्। विश्वं॑ ना॒राय॑णं दे॒व॒म॒क्षरं॑ पर॒मं प॒दम्। वि॒श्वतः॒ पर॑मान्नि॒त्यं॒  वि॒श्वं ना॑राय॒ण ह॑रिम्। विश्व॑मे॒वेदं पुरु॑ष॒स्तद्विश्व॒मुप॑जीवति। पतिं॒  विश्व॑स्या॒त्मेश्व॑र॒ शाश्व॑त शि॒वम॑च्युतम्। ना॒राय॒णं म॑हाज्ञे॒यं॒  वि॒श्वात्मा॑नं प॒राय॑णम्। ना॒राय॒णप॑रं ब्र॒ह्म॒ त॒त्वं ना॑राय॒णः प॑रः। ना॒राय॒णप॑रो ज्यो॒ति॒रा॒त्मा ना॑राय॒णः प॑रः। यच्च॑ किं॒ चिज्ज॑गत्य॒स्मि॒न्दृ॒श्यते श्रूय॒तेऽपि॑ वा। अन्त॑र्ब॒हिश्च॑ तत्स॒र्व॒व्व्याँ॒प्य ना॑राय॒णः स्थि॑तः॥२५॥

%६.११.२
अन॑न्त॒मव्य॑यं क॒वि स॑मु॒द्रेन्तं॑  वि॒श्वश॑म्भुवम्। प॒द्म॒को॒शप्र॑तीका॒श॒ हृ॒दयं॑ चाप्य॒धोमु॑खम्। अधो॑ नि॒ष्ट्या वि॑तस्त्या॒न्ते॒ ना॒भ्यामु॑परि॒ तिष्ठ॑ति। हृ॒दयं॑ तद्वि॑जानी॒या॒द्वि॒श्वस्या॑यत॒नं म॑हत्। सन्त॑त सि॒राभि॑स्तु॒ लम्ब॑त्याकोश॒सन्नि॑भम्। तस्यान्ते॑ सुषि॒र सू॒क्ष्मं तस्मिन्त्स॒र्वं प्रति॑ष्ठितम्। तस्य॒ मध्ये॑ म॒हान॑ग्निर्वि॒श्वार्चि॑र्वि॒श्वतो॑मुखः। सोऽग्र॑भु॒ग्विभ॑जन्ति॒ष्ठं॒ नाहा॑रमज॒रः क॒विः। स॒न्ता॒पय॑ति स्वन्दे॒हमापा॑दतल॒मस्त॑कम्। तस्य॒ मध्ये॒ वह्नि॑शिखा अ॒णीयोर्ध्वा व्य॒वस्थि॑तः। नी॒लतो॑यद॑मध्य॒स्था॒ वि॒द्युल्ले॑खेव॒ भास्व॑रा। नी॒वार॒शूक॑वत्त॒न्वी॒ पी॒ताभास्यात्त॒नूप॑मा। तस्या  शिखा॒या म॑ध्ये प॒रमात्मा व्य॒वस्थि॑तः। स ब्रह्मा॒ स शिवः॒ सेन्द्रः॒ सोऽक्ष॑रः पर॒मः स्व॒राट्॥२६॥
\anuvakamend


%६.१२.१
ऋ॒त स॒त्यं प॑रं ब्र॒ह्म॒ पु॒रुष॑ङ्कृष्ण॒पिङ्ग॑लम्। ऊ॒र्ध्वरे॑तं  वि॑रूपा॒क्षं॒  वि॒श्वरू॑पाय॒ वै नम॑॥२७॥%१२।
%६.१३.०
\anuvakamend


%६.१३.१
आ॒दि॒त्यो वा ए॒ष ए॒तन्म॒ण्डलं॒ तप॑ति॒ तत्र॒ ता ऋच॒स्तदृ॒चां म॒ण्डल॒ स ऋ॒चां लो॒कोऽथ॒ य ए॒ष ए॒तस्मि॑न्म॒ण्डले॒ऽर्चिर्दी॒प्यते॒ तानि॒ सामा॑नि॒ स सा॒म्नाम्म॒ण्डल॒ स सा॒म्नां लो॒कोऽथ॒ य ए॒ष ए॒तस्मि॑न्म॒ण्डले॒ऽर्चिषि॒ पुरु॑ष॒स्तानि॒ यजूषि॒ स यजु॑षां म॒ण्डल॒ स यजु॑षां लो॒कः सैषा त्र॒य्येव॑ वि॒द्या त॑पति॒ य ए॒षोऽन्तरा॑दि॒त्ये हि॑र॒ण्मयः॒ पुरु॑षः॥२८॥
%६.१४.०
\anuvakamend


%६.१४.१
आ॒दि॒त्यो वै तेज॒ ओजो॒ बलं॒ यश॒श्चक्षुः॒ श्रोत्र॑मा॒त्मा मनो॑ म॒न्युर्मनु॑र्मृ॒त्युः स॒त्यो मि॒त्रो वा॒युरा॑का॒शः प्रा॒णो लो॑कपा॒लः कः किं कं तत्स॒त्यमन्न॒मायु॑र॒मृतो॑ जी॒वो विश्व॑ कत॒मः स्व॑य॒म्भुः प्र॒जाप॑ति संवत्स॒र इति॑ संवत्स॒रो॑ऽसावा॑दि॒त्यो य ए॒ष पुरु॑ष ए॒ष भू॒ताना॒मधि॑पति॒र्ब्रह्म॑णः॒ सायु॑ज्य सलो॒कता॑माप्नोत्ये॒तासा॑मे॒व दे॒वता॑ना॒ सायु॑ज्य सा॒र्ष्टिता समानलो॒कता॑माप्नोति॒ य ए॒वं वेदेत्युप॒निषत्॥२९॥
%६.१५.०
\anuvakamend


%६.१५.१
घृणिः॒ सूर्य॑ आदि॒त्योम॑र्चयन्ति॒ तप॑ स॒त्यं मधु॑ क्षरन्ति॒ तद्ब्रह्म॒ तदाप॒ आपो॒ ज्योती॒रसो॒ऽमृतं॒ ब्रह्म॒ भूर्भु॑वः॒ सुव॒रोम्॥३०॥
%६.१६.०
%६.१६.०
\anuvakamend


%६.१६.१
सर्वो॒ वै रु॒द्रस्तस्मै॑ रु॒द्राय॒ नमो॑ अस्तु। पुरु॑षो॒ वै रु॒द्रः सन्म॒हो नमो॒ नम॑। विश्वं॑ भू॒तं भुव॑नं चि॒त्रं ब॑हु॒धा जा॒तं जाय॑मानं च॒ यत्। सर्वो॒ ह्ये॑ष रु॒द्रस्तस्मै॑ रु॒द्राय॒ नमो॑ अस्तु॥३१॥
\anuvakamend


%६.१७.१
कद्रु॒द्राय॒ प्रचे॑तसे मी॒ढुष्ट॑माय॒ तव्य॑से। वो॒चेम॒ शन्त॑म हृ॒दे। सर्वो॒ ह्ये॑ष रु॒द्रस्तस्मै॑ रु॒द्राय॒ नमो॑ अस्तु॥३२॥
%६.१८.०
\anuvakamend


%६.१८.१
नमो हिरण्यबाहवे हिरण्यपतयेऽम्बिकापतय उमापतये॑ नमो॒ नमः॥३३॥
%६.१९.०
\anuvakamend


%६.१९.१
यस्य॒ वैक॑ङ्कत्यग्निहोत्र॒हव॑णी भवति॒ प्रति॑ष्ठिताः॒ प्रत्ये॒वास्याहु॑तयस्तिष्ठ॒न्त्यथो॒ प्रति॑ष्ठित्यै॥३४॥

%६.२०.०
। २०।
\anuvakamend


%६.२०.१
कृ॒णु॒ष्व पाज॒ इति॒ पञ्च॑॥३५॥
%६.२१.०
\anuvakamend


%६.२१.१
अदि॑तिर्दे॒वा ग॑न्ध॒र्वा म॑नु॒ष्या पि॒तरोऽसु॑रा॒स्तेषा सर्वभू॒तानां मा॒ता मे॒दिनी॑ मह॒ती म॒ही सा॑वि॒त्री गा॑य॒त्री जग॑त्यु॒र्वी पृ॒थ्वी ब॑हु॒ला विश्वा॑ भू॒ता क॑त॒मा का या सा स॒त्येत्य॒मृतेति॑ वसि॒ष्ठः॥३६॥
%६.२२.०
\anuvakamend


%६.२२.१
आपो॒ वा इ॒द सर्वं॒  विश्वा॑ भू॒तान्याप॑ प्रा॒णा वा आप॑ प॒शव॒ आपो॒ऽमृत॒मापोऽन्न॒माप॑ स॒म्राडापो॑ वि॒राडाप॑ स्व॒राडाप॒श्छन्दा॒स्यापो॒ ज्योती॒ष्याप॑ स॒त्यमापः॒ सर्वा॑ दे॒वता॒ आपो॒ भूर्भुवः॒ सुव॒राप॒ ओम्॥३७॥
%६.२३.०
\anuvakamend


%६.२३.१
आप॑ पुनन्तु पृथि॒वीं पृ॑थि॒वी पू॒ता पु॑नातु॒ माम्। पु॒नन्तु॒ ब्रह्म॑ण॒स्पति॒र्ब्रह्म॑पू॒ता पु॑नातु॒ माम्। यदुच्छि॑ष्ट॒मभोज्यं॒ यद्वा॑ दु॒श्चरि॑तं॒ मम॑। सर्वं॑ पुनन्तु॒ मामापो॑ऽस॒तां च॑ प्रति॒ग्रह॒ स्वाहा॥३८€॥
%६.२४.०
\anuvakamend


%६.२४.१
अग्निश्च मा मन्युश्च मन्युपतयश्च मन्यु॑कृते॒भ्यः। पापेभ्यो॑ रक्ष॒न्ताम्। यदह्ना पाप॑मका॒र्\mbox{}षम्। मनसा वाचा॑ हस्ता॒भ्याम्। पद्भ्यामुदरे॑ण शि॒श्ञा। अह॒स्तद॑वलु॒म्पतु। यत्किं च॑ दुरि॒तं मयि॑। इदमहं माममृ॑तयो॒नौ। सत्ये ज्योतिषि जुहो॑मि स्वा॒हा॥३९॥
%६.२५.०
\anuvakamend


%६.२५.१
सूर्यश्च मा मन्युश्च मन्युपतयश्च मन्यु॑कृते॒भ्यः। पापेभ्यो॑ रक्ष॒न्ताम्। यद्रात्रिया पाप॑मका॒र्\mbox{}षम्। मनसा वाचा॑ हस्ता॒भ्याम्। पद्भ्यामुदरे॑ण शि॒श्ञा। रात्रि॒स्तद॑वलु॒म्पतु। यत्किं च॑ दुरि॒तं मयि॑। इदमहं माममृ॑तयो॒नौ। सूर्ये ज्योतिषि जुहो॑मि स्वा॒हा॥४०॥
%६.२६.०
\anuvakamend


%६.२६.१
आया॑तु॒ वर॑दा दे॒वी॒ अ॒क्षरं॑ ब्रह्म॒संमि॑तम्। गा॒य॒त्रीञ्छन्द॑सां मा॒तेदं ब्र॑ह्म जु॒षस्व॑ नः। ओजो॑ऽसि॒ सहो॑ऽसि॒ बल॑मसि॒ भ्राजो॑ऽसि दे॒वाना॒न्धाम॒ नामा॑सि॒ विश्व॑मसि वि॒श्वायुः॒ सर्व॑मसि स॒र्वायुरभिभूरोङ्गायत्रीमावा॑हया॒मि॥४१॥
%६.२७.०
\anuvakamend


%६.२७.१
ओं भूः। ओं भुव॑। ओ सुव॑। ओं मह॑। ओं जन॑। ओं तप॑। ओ स॒त्यम्। ओं तत्स॑वि॒तुर्वरेण्यं॒ भर्गो॑ दे॒वस्य॑ धीमहि। धियो॒ यो न॑ प्रचो॒दयात्। ओमापो॒ ज्योती॒रसो॒ऽमृतं॒ ब्रह्म॒ भूर्भुवः॒ सुव॒रोम्॥४२॥
%६.२८.०
। २८।
\anuvakamend


%६.२८.१
ओं भूर्भुवः॒ सुव॒र्मह॒र्जन॒स्तप॑ स॒त्यं तद्ब्रह्म॒ तदाप॒ आपो॒ ज्योती॒रसो॒ऽमृतं॒ ब्रह्म॒ भूर्भुवः॒ सुव॒रोम्॥४३॥
%६.२९.०
\anuvakamend


%६.२९.१
ओं तद्ब्र॒ह्म। ओं तद्वा॒युः। ओं तदा॒त्मा। ओं तत्सर्वम्। ओं तत्पुरो॒र्नम॑॥४४॥
%६.३०.०

%६.३०.१
उ॒त्तमे॑ शिख॑रे दे॒वी॒ भू॒म्यां प॑र्वत॒मूर्ध॑नि। ब्रा॒ह्म॒णेभ्यो ह्य॑नुज्ञा॒नं॒ गच्छ दे॑वि य॒थासु॑खम्॥४५॥
\anuvakamend


%६.३१.०
। ३१।

%६.३१.१
ओमन्तश्चरति॑ भूते॒षु॒ गुहायां वि॑श्वमू॒र्तिषु। त्वं यज्ञस्त्वं  विष्णुस्त्वं व॑षट्का॒र॒स्त्व रुद्रस्त्वं ब्रह्मा त्वं॑ प्रजा॒पति॑॥४६॥
%६.३२.०
\anuvakamend


%६.३२.१
अ॒मृ॒तो॒प॒स्तर॑णमसि॥४७॥
%६.३३.०
\anuvakamend


%६.३३.१
प्रा॒णे निवि॑ष्टो॒ऽमृतं॑ जुहोमि। प्रा॒णाय॒ स्वाहा। अ॒पा॒ने निवि॑ष्टो॒ऽमृतं॑ जुहोमि। अ॒पा॒नाय॒ स्वाहा। व्या॒ने निवि॑ष्टो॒ऽमृतं॑ जुहोमि। व्यानाय॒ स्वाहा। उ॒दा॒ने निवि॑ष्टो॒ऽमृतं॑ जुहोमि। उ॒दा॒नाय॒ स्वाहा। स॒मा॒ने निवि॑ष्टो॒ऽमृतं॑ जुहोमि। स॒मा॒नाय॒ स्वाहा। ब्रह्म॑णि म आ॒त्माऽमृ॑त॒त्वाय॑॥४८॥
\anuvakamend


%६.३४.१
प्रा॒णे निवि॑ष्टो॒ऽमृतं॑ जुहोमि शि॒वो मा॑ वि॒शा प्र॑दाहाय प्रा॒णाय॒ स्वाहा। अ॒पा॒ने निवि॑ष्टो॒ऽमृतं॑ जुहोमि शि॒वो मा॑ वि॒शा प्र॑दाहायापा॒नाय॒ स्वाहा। व्या॒ने निवि॑ष्टो॒ऽमृतं॑ जुहोमि शि॒वो मा॑ वि॒शा प्र॑दाहाय व्या॒नाय॒ स्वाहा। उ॒दा॒ने निवि॑ष्टो॒ऽमृतं॑ जुहोमि शि॒वो मा॑ वि॒शा प्र॑दाहायोदा॒नाय॒ स्वाहा। स॒मा॒ने निवि॑ष्टो॒ऽमृतं॑ जुहोमि शि॒वो मा॑ वि॒शा प्र॑दाहाय समा॒नाय॒ स्वाहा। ब्रह्म॑णि म आ॒त्माऽमृ॑त॒त्वाय॑॥४९॥
%६.३५.०
\anuvakamend


%६.३५.१
अ॒मृ॒ता॒पि॒धा॒नम॑सि॥५०॥
%६.३६.०
\anuvakamend


%६.३६.१
श्र॒द्धायां प्रा॒णे निवि॑श्या॒मृत हु॒तम्प्रा॒णमन्ने॑नाप्यायस्व। अ॒पा॒ने निवि॑श्या॒मृत हु॒तम॑पा॒नमन्ने॑नाप्यायस्व। व्या॒ने निवि॑श्या॒मृत हु॒तव्व्याँ॒नमन्ने॑नाप्यायस्व। उ॒दा॒ने निवि॑श्या॒मृत हु॒तमु॑दा॒नमन्ने॑नाप्यायस्व। स॒मा॒ने निवि॑श्या॒मृत हु॒त स॑मा॒नमन्ने॑नाप्यायस्व। ब्रह्म॑णि म आ॒त्माऽमृ॑त॒त्वाय॑॥५१॥
%६.३७.०
\anuvakamend


%६.३७.१
प्राणानाङ्ग्रन्थिरसि रुद्रो मा॑ऽऽविशा॒न्तकस्तेनान्नेनाप्याय॒स्व॥५२॥
%६.३८.०
\anuvakamend


%६.३८.१
अङ्गुष्ठमात्रः पुरुषोऽङ्गुष्ठं च॑ समा॒श्रितः। ईशः सर्वस्य जगतः प्रभुः प्रीणाति॑ विश्व॒भुक्॥५३॥
%६.३९.०
\anuvakamend


%६.३९.१
मे॒धा दे॒वी जु॒षमा॑णा न॒ आगाद्वि॒श्वाची॑ भ॒द्रा सु॑मन॒स्यमा॑ना। त्वया॒ जुष्टा॑ जु॒षमा॑णा दु॒रुक्तान्बृ॒हद्व॑देम वि॒दथे॑ सु॒वीरा॥ त्वया॒ जुष्ट॑ ऋ॒षिर्भ॑वति देवि॒ त्वया॒ ब्रह्मा॑ऽऽग॒तश्री॑रु॒त त्वया। त्वया॒ जुष्ट॑श्चि॒त्रं  वि॑न्दते वसु॒ सा नो॑ जुषस्व॒ द्रवि॑णो न मेधे॥५४॥
%६.४०.०
\anuvakamend


%६.४०.१
मे॒धां म॒ इन्द्रो॑ ददातु मे॒धान्दे॒वी सर॑स्वती। मे॒धां मे॑ अ॒श्विनौ॑ दे॒वावाध॑त्तां॒ पुष्क॑रस्रजा॥५५॥
%६.४१.०
\anuvakamend


%६.४१.१
अ॒प्स॒रासु॑ च॒ या मे॒धा ग॑न्ध॒र्वेषु॑ च॒ यन्मन॑। दैवी॑ मे॒धा म॑नुष्य॒जा सा मां मे॒धा सु॒रभि॑र्जुषताम्॥५६॥
%६.४२.०
\anuvakamend


%६.४२.१
आ मां मे॒धा सु॒रभि॑र्वि॒श्वरू॑पा॒ हिर॑ण्यवर्णा॒ जग॑ती जग॒म्या। ऊर्ज॑स्वती॒ पय॑सा॒ पिन्व॑माना॒ सा मां मे॒धा सु॒प्रती॑का जुषताम्॥५७॥
%६.४३.०
\anuvakamend


%६.४३.१
स॒द्योजा॒तं प्र॑पद्या॒मि॒ स॒द्योजा॒ताय॒ वै नम॑। भ॒वेभ॑वे॒ नाति॑भवे भजस्व॒ मां भ॒वोद्भ॑वाय॒ नम॑॥५८॥
%६.४४.०
\anuvakamend


%६.४४.१
वा॒म॒दे॒वाय॒ नमो ज्ये॒ष्ठाय॒ नमो॑ रु॒द्राय॒ नमः॒ काला॑य॒ नमः॒ कल॑विकरणाय॒ नमो॒ बल॑विकरणाय॒ नमो॒ बल॑प्रमथनाय॒ नमः॒ सर्व॑भूतदमनाय॒ नमो॑ म॒नोन्म॑नाय॒ नम॑॥५९॥
%६.४५.०
\anuvakamend


%६.४५.१
अ॒घोरेभ्योऽथ॒ घोरेभ्यो॒ घोर॒घोर॑तरेभ्यः स॒र्वत॑ शर्व॒ सर्वेभ्यो॒ नम॑स्ते अस्तु रु॒द्ररू॑पेभ्यः॥६०॥
%६.४६.०
\anuvakamend


%६.४६.१
तत्पुरु॑षाय वि॒द्महे॑ महादे॒वाय॑ धीमहि। तन्नो॑ रुद्रः प्रचो॒दयात्॥६१॥

%६.४७.०
। ४७।
\anuvakamend


%६.४७.१
ईशानः सर्व॑विद्या॒ना॒मीश्वरः सर्व॑भूता॒नां॒ ब्रह्माधि॑पति॒र्ब्रह्म॒णोऽधि॑पति॒र्ब्रह्मा॑ शि॒वो मे॑ अस्तु सदाशि॒वोम्॥६२॥

%६.४८.०

%६.४८.१
ब्रह्म॑मेतु॒ माम्। मधु॑मेतु॒ माम्। ब्रह्म॑मे॒व मधु॑मेतु॒ माम्। यास्ते॑ सोम प्र॒जाव॒त्सोभि॒ सो अ॒हम्। दुस्व॑प्न॒हन्दु॑रुष्व॒हा। यास्ते॑ सोम प्रा॒णास्तां जु॑होमि। त्रिसु॑पर्ण॒मया॑चितं ब्राह्म॒णाय॑ दद्यात्। ब्र॒ह्म॒ह॒त्यां वा ए॒ते घ्न॑न्ति। ये ब्राह्म॒णास्त्रिसु॑पर्णं॒ पठ॑न्ति। ते सोमं॒ प्राप्नु॑वन्ति। आ॒स॒ह॒स्रात्प॒ङ्क्तिं पुन॑न्ति। ओम्॥६३॥

%६.४९.०
। ४९।
\anuvakamend


%६.४९.१
ब्रह्म॑ मे॒धया। मधु॑ मे॒धया। ब्रह्म॑मे॒व मधु॑ मे॒धया। अ॒द्या नो॑ देव सवितः प्र॒जाव॑त्सावीः॒ सौभ॑गम्। परा॑ दु॒ष्वप्नि॑य सुव। विश्वा॑नि देव सवितर्दुरि॒तानि॒ परा॑ सुव। यद्भ॒द्रं तन्म॒ आ सु॑व। मधु॒ वाता॑ ऋताय॒ते मधु॑ क्षरन्ति॒ सिन्ध॑वः। माध्वीर्नः स॒न्त्वोष॑धीः। मधु॒ नक्त॑मु॒तोषसि॒ मधु॑म॒त्पार्थि॑व॒ रज॑। मधु॒ द्यौर॑स्तु नः पि॒ता। मधु॑मान्नो॒ वन॒स्पति॒र्मधु॑मा अस्तु॒ सूर्य॑। माध्वी॒र्गावो॑ भवन्तु नः। य इ॒मं त्रिसु॑पर्ण॒मया॑चितं ब्राह्म॒णाय॑ दद्यात्। भ्रू॒ण॒ह॒त्यां वा ए॒ते घ्न॑न्ति। ये ब्राह्म॒णास्त्रिसु॑पर्णं॒ पठ॑न्ति। ते सोमं॒ प्राप्नु॑वन्ति। आ॒स॒ह॒स्रात्प॒ङ्क्तिं पुन॑न्ति। ओम्॥६४॥
%६.५०.०
\anuvakamend


%६.५०.१
ब्रह्म॑ मे॒धवा। मधु॑ मे॒धवा। ब्रह्म॑मे॒व मधु॑ मे॒धवा। ब्र॒ह्मा दे॒वानां पद॒वीः क॑वी॒नामृषि॒र्विप्रा॑णां महि॒षो मृ॒गाणाम्। श्ये॒नो गृध्रा॑णा॒ स्वधि॑ति॒र्वना॑ना॒ सोम॑ प॒वित्र॒मत्ये॑ति॒ रेभ\sn{}। ह॒सः  शु॑चि॒षद्वसु॑रन्तरिक्ष॒सद्धोता॑ वेदि॒षदति॑थिर्दुरोण॒सत्। नृ॒षद्व॑र॒सदृ॑त॒सद्व्यो॑म॒सद॒ब्जा गो॒जा ऋ॑त॒जा अ॑द्रि॒जा ऋ॒तं बृ॒हत्। य इ॒मं त्रिसु॑पर्ण॒मया॑चितं ब्राह्म॒णाय॑ दद्यात्। वी॒र॒ह॒त्यां वा ए॒ते घ्न॑न्ति। ये ब्राह्म॒णास्त्रिसु॑पर्णं॒ पठ॑न्ति। ते सोमं॒ प्राप्नु॑वन्ति। आ॒स॒ह॒स्रात्प॒ङ्क्तिं पुन॑न्ति। ओम्॥६५॥

%६.५१.०
। ५१।
\anuvakamend


%६.५१.१
प्राणापानव्यानोदानसमाना मे॑ शुद्ध्य॒न्तां॒ ज्योति॑र॒हं  वि॒रजा॑ विपा॒प्मा भू॑यास॒ स्वाहा॥६६॥
%६.५२.०
\anuvakamend


%६.५२.१
वाङ्मनश्चक्षुःश्रोत्रजिह्वाघ्राणरेतोबुध्याकूतिसङ्कल्पा मे॑ शुद्ध्य॒न्तां॒ ज्योति॑र॒हं  वि॒रजा॑ विपा॒प्मा भू॑यास॒ स्वाहा॥६७॥
%६.५३.०
\anuvakamend


%६.५३.१
शिरःपाणिपादपार्श्वपृष्ठोदरजङ्घशिश्ञोपस्थपायवो मे॑ शुद्ध्य॒न्तां॒ ज्योति॑र॒हं  वि॒रजा॑ विपा॒प्मा भू॑यास॒ स्वाहा॥६८॥
%६.५४.०
\anuvakamend


%६.५४.१
त्वक्चर्ममासरुधिरमेदोऽस्थिमज्जा मे॑ शुद्ध्य॒न्तां॒ ज्योति॑र॒हं  वि॒रजा॑ विपा॒प्मा भू॑यास॒ स्वाहा॥६९॥
%६.५५.०
\anuvakamend


%६.५५.१
शब्दस्पर्शरूपरसगन्धा मे॑ शुद्ध्य॒न्तां॒ ज्योति॑र॒हं  वि॒रजा॑ विपा॒प्मा भू॑यास॒ स्वाहा॥७०॥
%६.५६.०
\anuvakamend


%६.५६.१
पृथिव्यप्तेजोवाय्वाकाशा मे॑ शुद्ध्य॒न्तां॒ ज्योति॑र॒हं  वि॒रजा॑ विपा॒प्मा भू॑यास॒ स्वाहा॥७१॥
%६.५७.०
\anuvakamend


%६.५७.१
अन्नमयप्राणमयमनोमयविज्ञानमयानन्दमया मे॑ शुद्ध्य॒न्तां॒ ज्योति॑र॒हं  वि॒रजा॑ विपा॒प्मा भू॑यास॒ स्वाहा॥७२॥
%६.५८.०
\anuvakamend


%६.५८.१
विवि॑ट्टि॒ स्वाहा॥७३॥
%६.५९.०
\anuvakamend


%६.५९.१
घ॒षोत्काय॒ स्वाहा॥७४॥

%६.६०.०
। ६०।
\anuvakamend


%६.६०.१
उत्तिष्ठ पुरुषा हरी लोहितपिङ्गलाक्षि देहि देहि ददापयिता मे॑ शुद्ध्य॒न्तां॒ ज्योति॑र॒हं  वि॒रजा॑ विपा॒प्मा भू॑यास॒ स्वाहा॥७५॥
%६.६१.०
\anuvakamend


%६.६१.१
ओ स्वाहा॥७६॥
%६.६२.०
\anuvakamend


%६.६२.१
स॒त्यं परं॒ पर स॒त्य स॒त्येन॒ न सु॑व॒र्गाल्लो॒काच्च्य॑वन्ते क॒दाच॒न स॒ता हि स॒त्यं तस्मात्स॒त्ये र॑मन्ते॒ तप॒ इति॒ तपो॒ नानश॑ना॒त्परं॒ यद्धि परं॒ तप॒स्तद्दुर्द्ध॑र्\mbox{}षं॒ तद्दुरा॑धर्\mbox{}षं॒ तस्मा॒त्तप॑सि रमन्ते॒ दम॒ इति॒ निय॑तं ब्रह्मचा॒रिण॒स्तस्मा॒द्दमे॑ रमन्ते॒ शम॒ इत्यर॑ण्ये मु॒नय॒स्तस्मा॒च्छमे॑ रमन्ते दा॒नमिति॒ सर्वा॑णि भू॒तानि॑ प्र॒शस॑न्ति दा॒नान्नाति॑ दु॒ष्करं॒ तस्माद्दा॒ने र॑मन्ते ध॒र्म इति॒ धर्मे॑ण॒ सर्व॑मि॒दं परि॑गृहीतं ध॒र्मान्नाति॑ दु॒श्चरं॒ तस्माद्ध॒र्मे र॑मन्ते प्र॒जन॒ इति॒ भूयास॒स्तस्मा॒द्भूयि॑ष्ठाः॒ प्रजा॑यन्ते॒ तस्मा॒द्भूयि॑ष्ठाः प्र॒जन॑ने रमन्ते॒ऽग्नय॒ इत्या॑ह॒ तस्मा॑द॒ग्नय॒ आधा॑तव्या अग्निहो॒त्रमित्या॑ह॒ तस्मा॑दग्निहो॒त्रे र॑मन्ते य॒ज्ञ इति॑ य॒ज्ञो हि दे॒वानां य॒ज्ञेन॒ हि दे॒वा दिव॑ङ्ग॒तास्तस्माद्य॒ज्ञे र॑मन्ते मान॒समिति॑ वि॒द्वास॒स्तस्मा॑द्वि॒द्वास॑ ए॒व मा॑न॒से र॑मन्ते न्या॒स इति॑ ब्र॒ह्मा ब्र॒ह्मा हि परः॒ परो॑ हि ब्र॒ह्मा तानि॒ वा ए॒तान्यव॑राणि॒ तपासि न्या॒स ए॒वात्य॑रेचय॒द्य ए॒वं वेदेत्युप॒निषत्॥७७॥
%६.६३.०
\anuvakamend


%६.६३.१
प्रा॒जा॒प॒त्यो हारु॑णिः सुप॒र्णेय॑ प्र॒जाप॑तिं पि॒तर॒मुप॑ससार॒ किं भ॑गव॒न्तः प॑र॒मं व॑द॒न्तीति॒ तस्मै॒ प्रो॑वाच स॒त्येन॑ वा॒युरावा॑ति स॒त्येना॑दि॒त्यो रो॑चते दि॒वि स॒त्यं वा॒चः प्र॑ति॒ष्ठा स॒त्ये स॒र्वं प्रति॑ष्ठितं॒ तस्मात्स॒त्यं प॑र॒मं वद॑न्ति॒ तप॑सा दे॒वा दे॒वता॒मग्र॑ आय॒न्तप॒सर्\mbox{}ष॑यः॒ सुव॒रन्व॑विन्दं॒ तप॑सा स॒पत्ना॒प्रणु॑दा॒मारा॑ती॒स्तप॑सि स॒र्वं प्रति॑ष्ठितं॒ तस्मा॒त्तप॑ पर॒मं वद॑न्ति॒ दमे॑न दा॒न्ताः कि॒ल्बिष॑मवधू॒न्वन्ति॒ दमे॑न ब्रह्मचा॒रिणः॒ सुव॑रगच्छ॒न्दमो॑ भू॒तानान्दुरा॒धर्\mbox{}ष॒न्दमे॑ स॒र्वं प्रति॑ष्ठितं॒ तस्मा॒द्दम॑ पर॒मं वद॑न्ति॒ शमे॑न शा॒न्ताः  शि॒वमा॒चर॑न्ति॒ शमे॑न ना॒कं मु॒नयो॒ऽन्ववि॑न्द॒ञ्छमो॑ भू॒तानान्दुरा॒धर्\mbox{}ष॒ञ्छमे॑ स॒र्वं प्रति॑ष्ठितं॒ तस्मा॒च्छम॑ पर॒मं वद॑न्ति दा॒नं य॒ज्ञानां॒ वरू॑थ॒न्दक्षि॑णा लो॒के दा॒तार सर्वभू॒तान्यु॑पजी॒वन्ति॑ दा॒नेनारा॑ती॒रपा॑नुदन्त दा॒नेन॑ द्विष॒न्तो मि॒त्रा भ॑वन्ति दा॒ने स॒र्वं प्रति॑ष्ठितं॒ तस्माद्दा॒नं प॑र॒मं वद॑न्ति ध॒र्मो विश्व॑स्य॒ जग॑तः प्रति॒ष्ठा लो॒के ध॒र्मिष्ठं॑ प्र॒जा उ॑पस॒र्पन्ति॑ ध॒र्मेण॑ पा॒पम॑प॒नुद॑ति ध॒र्मे स॒र्वं प्रति॑ष्ठितं॒ तस्माद्ध॒र्मं प॑र॒मं वद॑न्ति प्र॒जन॑नं॒ वै प्र॑ति॒ष्ठा लो॒के सा॒धु प्र॒जायास्त॒न्तुं त॑न्वा॒नः पि॑तृ॒णाम॑नृ॒णो भव॑ति॒ तदे॑व त॒स्यानृ॑णं॒ तस्मात् प्र॒जन॑नं पर॒मं वद॑न्त्य॒ग्नयो॒ वै त्रयी॑ वि॒द्या दे॑व॒यानः॒ पन्था॑ गार्\mbox{}हप॒त्य ऋक्पृ॑थि॒वी र॑थन्त॒रम॑न्वाहार्य॒पच॑नो॒ यजु॑र॒न्तरि॑क्षं वामदे॒व्यमा॑हव॒नीयः॒ साम॑ सुव॒र्गो लो॒को बृ॒हत्तस्मा॑द॒ग्नीन्प॑र॒मं वद॑न्त्यग्निहो॒त्र सा॑यं प्रा॒तर्गृ॒हाणां॒ निष्कृ॑तिः॒ स्वि॑ष्ट सुहु॒तं य॑ज्ञक्रतू॒नां प्राय॑ण सुव॒र्गस्य॑ लो॒कस्य॒ ज्योति॒स्तस्मा॑दग्निहो॒त्रं प॑र॒मं वद॑न्ति य॒ज्ञ इति॑ य॒ज्ञो हि दे॒वानां य॒ज्ञेन॒ हि दे॒वा दिव॑ङ्ग॒ता य॒ज्ञेनासु॑रा॒नपा॑नुदन्त य॒ज्ञेन॑ द्विष॒न्तो मि॒त्रा भ॑वन्ति य॒ज्ञे स॒र्वं प्रति॑ष्ठितं॒ तस्माद्य॒ज्ञं प॑र॒मं वद॑न्ति मान॒सं वै प्रा॑जाप॒त्यं प॒वित्रं॑ मान॒सेन॒ मन॑सा सा॒धु प॑श्यति मान॒सा ऋष॑यः प्र॒जा अ॑सृजन्त मान॒से स॒र्वं प्रति॑ष्ठितं॒ तस्मान्मान॒सं प॑र॒मं वद॑न्ति न्या॒स इ॒त्याहु॑र्मनी॒षिणो ब्र॒ह्माणं॑ ब्र॒ह्मा विश्व॑ कत॒मः स्व॑य॒म्भुः प्र॒जाप॑तिः संवत्स॒र इति॑ संवत्स॒रो॑ऽसावा॑दि॒त्यो य ए॒ष आ॑दि॒त्ये पुरु॑षः॒ स प॑रमे॒ष्ठी ब्रह्मा॒त्मा याभि॑रादि॒त्यस्तप॑ति र॒श्मिभि॒स्ताभि॑ प॒र्जन्यो॑ वर्\mbox{}षति प॒र्जन्ये॑नौषधिवनस्प॒तयः॒ प्रजा॑यन्त ओषधिवनस्प॒तिभि॒रन्नं॑ भव॒त्यन्ने॑न प्रा॒णाः प्रा॒णैर्बलं॒ बले॑न॒ तप॒स्तप॑सा श्र॒द्धा श्र॒द्धया॑ मे॒धा मे॒धया॑ मनी॒षा म॑नी॒षया॒ मनो॒ मन॑सा॒ शान्तिः॒ शान्त्या॑ चि॒त्तं चि॒त्तेन॒ स्मृति॒ स्मृत्या॒ स्मार॒ स्मारे॑ण वि॒ज्ञानं॑  वि॒ज्ञाने॑ना॒त्मानं॑ वेदयति॒ तस्मा॑द॒न्नं दद॒न्त्सर्वाण्ये॒तानि॑ ददा॒त्यन्नात् प्रा॒णा भ॑वन्ति भू॒तानां प्रा॒णैर्मनो॒ मन॑सश्च वि॒ज्ञानं॑  वि॒ज्ञाना॑दान॒न्दो ब्र॑ह्मयो॒निः स वा ए॒ष पुरु॑षः पञ्च॒धा प॑ञ्चा॒त्मा येन॒ सर्व॑मि॒दं प्रोतं॑ पृथि॒वी चा॒न्तरि॑क्षं च॒ द्यौश्च॒ दिश॑श्चावान्तरदि॒शाश्च॒ स वै सर्व॑मि॒दं जग॒त्स च॒ भूत स भ॒व्यं जि॑ज्ञासकॢ॒प्त ऋ॑त॒जा रयि॑ष्ठाः  श्र॒द्धा स॒त्यो मह॑स्वान्त॒मसो॒परि॑ष्टा॒द्ज्ञात्वा॑ तमे॒वं मन॑सा हृ॒दा च॒ भूयो॑ न मृ॒त्युमुप॑याहि वि॒द्वान्तस्मान्न्या॒समे॒षां तप॑सामतिरिक्त॒माहु॑र्वसुर॒ण्यो॑ वि॒भूर॑सि प्रा॒णे त्वमसि॑ सन्धा॒ता ब्रह्मं॑ त्वम॑सि विश्व॒सृक्ते॑जो॒दास्त्वम॑स्य॒ग्नेर्व॑र्चो॒दास्त्वम॑सि॒ सूर्य॑स्य द्युम्नो॒दास्त्वम॑सि च॒न्द्रम॑स उपया॒मगृ॑हीतोऽसि ब्र॒ह्मणे त्वा॒ महस॒ ओमित्या॒त्मानं॑ युञ्जीतै॒तद्वै म॑होप॒निष॑दन्दे॒वाना॒ङ्गुह्यं॒ य ए॒वं वेद॑ ब्र॒ह्मणो॑ महि॒मान॑माप्नोति॒ तस्माद्ब्र॒ह्मणो॑ महि॒मान॑मित्युप॒निष॑त्॥७८॥
\anuvakamend


%६.६४.१
तस्यै॒वं  वि॒दुषो॑ य॒ज्ञस्या॒त्मा यज॑मानः श्र॒द्धा पत्नी॒ शरी॑रमि॒ध्ममुरो॒ वेदि॒र्लोमा॑नि ब॒\ar{}हिर्वे॒दः  शिखा॒ हृद॑यं॒ यूपः॒ काम॒ आज्यं॑ म॒न्युः प॒शुस्तपो॒ऽग्निः  श॑मयि॒ता दक्षि॑णा॒ वाग्घोता प्रा॒ण उ॑द्गा॒ता चक्षु॑रध्व॒र्युर्मनो॒ ब्रह्मा॒ श्रोत्र॑म॒ग्नीद्याव॒द्ध्रिय॑ते॒ सा दी॒क्षा यदश्ञा॑ति॒ यत्पिब॑ति॒ तद॑स्य सोमपा॒नं यद्रम॑ते॒ तदु॑प॒सदो॒ यत्स॒ञ्चर॑त्युप॒विश॑त्यु॒त्तिष्ठ॑ते च॒ स प्र॑व॒र्ग्यो॑ यन्मुखं॒ तदा॑हव॒नीयो॒ यद॑स्य वि॒ज्ञानं॒ तज्जु॒होति॒ यत्सा॒यं प्रा॒तर॑त्ति॒ तत्स॒मिधो॒ यत्सा॒यं प्रा॒तर्म॒ध्यन्दि॑नं च॒ तानि॒ सव॑नानि॒ ये अ॑होरा॒त्रे ते द॑र्\mbox{}शपूर्णमा॒सौ येऽर्द्धमा॒साश्च॒ मासाश्च॒ ते चा॑तुर्मा॒स्यानि॒ य ऋ॒तव॒स्ते प॑शुब॒न्धा ये सं॑वत्स॒राश्च॑ परिवत्स॒राश्च॒ तेऽह॑र्ग॒णाः स॑र्ववेद॒सं वा ए॒तत्स॒त्रं यन्मर॑णं॒ तद॑व॒भृथ॑ ए॒तद्वै ज॑रामर्यमग्निहो॒त्र स॒त्रं य ए॒वं  वि॒द्वानु॑द॒गय॑ने प्र॒मीय॑ते दे॒वाना॑मे॒व म॑हि॒मान॑ङ्ग॒त्वाऽऽदि॒त्यस्य॒ सायु॑ज्यं गच्छ॒त्यथ॒ यो द॑क्षि॒णे प्र॒मीय॑ते पितृ॒णामे॒व म॑हि॒मान॑ङ्ग॒त्वा च॒न्द्रम॑सः॒ सायु॑ज्यं गच्छत्ये॒तौ वै सूर्याचन्द्र॒मसोर्महि॒मानौ ब्राह्म॒णो वि॒द्वान॒भिज॑यति॒ तस्माद्ब्र॒ह्मणो॑ महि॒मान॑माप्नोति॒ तस्माद्ब्र॒ह्मणो॑ महि॒मान॑मित्युप॒निष॑त्॥७९॥
\anuvakamend

%अम्भ॑सि॒ भूर॒ग्नये॒ भूरन्नं॒ भूर॒ग्नये॑ च॒ पाहि नो यश्छन्द॑सां॒ नमो॒ ब्रह्म॑ण ऋ॒तं तपो॒ यथा॑ वृ॒क्षस्या॒णोरणी॑यान्त्सहस्र॒शीर्\mbox{}ष॑मृ॒तमा॑दि॒त्यो वा ए॒ष आ॑दि॒त्यो वै तेज॒ ओजो॒ घृणिः॒ सर्वो॒ वै कद्रु॒द्राय॒ नमो हिरण्यबाहवे यस्य॒ वैक॑ङ्कती कृणु॒ष्व पाजोऽदि॑ति॒रापो॒ वा इद सर्व॒माप॑ पुन॒न्त्वग्निश्च सूर्यश्चाया॒त्वों भूरों भूर्भुवः॒ सुव॒रोन्तदु॒त्तम॒ ओमन्तश्चरत्य॑मृतोप॒स्तर॑णमसि प्रा॒णे निवि॑ष्टः प्रा॒णे नि॑विष्टः  शि॒वो॑ऽमृतापिधा॒नम॑सि श्र॒द्धायां प्रा॒णे निवि॑श्य॒ प्राणानामङ्गुष्ठमात्रो मे॒धा दे॒वी मे॒धां म॒ इन्द्रो॑ ददात्वप्स॒रास्वामां मे॒धा स॒द्यो वा॑मदे॒वाया॒घोरेभ्य॒स्तत्पुरु॑षा॒येशानो ब्रह्म॑ मेतु॒ ब्रह्म॑ मे॒धया॒ ब्रह्म॑ मे॒धवा॒ प्राणापानवाङ्मनः  शिरःपाणित्वक्चर्मशब्दस्पर्शपृथिव्यन्नमय विवि॑ट्टि घ॒षोक्ता॒योत्तिष्ठो स॒त्यं परं॑ प्राजाप॒त्यस्तस्यै॒वञ्चतु॑षष्टिः॥

%अम्भ॑सि॒ वृषा॑ ह॒सः सर्वो॒ वा आया॑तु श्र॒द्धायां॒ तत्पुरु॑षाय॒ पृथिव्यप्तेजो नव॑सप्ततिः॥ ७९। अम्भ॒सीत्यु॑प॒निष॑त्॥

\centerline{॥ॐ शान्ति॒ शान्ति॒ शान्ति॑॥  हरि॑ ओम्॥}

\closesection
\clearpage
