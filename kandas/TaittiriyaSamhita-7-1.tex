\chapt{काण्डम् ७}
\sect{प्रथमः प्रश्नः}\setcounter{anuvakam}{0}
\dnsub{तैत्तिरीयसंहितायां सप्तमकाण्डे प्रथमः प्रश्नः}
%7.1.1.1
प्र॒जन॑नं॒ ज्योति॑र॒ग्निर्दे॒वता॑नां॒ ज्योति॑र्वि॒राट्छन्द॑सां॒ ज्योति॑र्वि॒राड्वा॒चो᳚\-ऽग्नौ सं ति॑ष्ठते वि॒राज॑म॒भि सम्प॑द्यते॒ तस्मा॒\-त्तज्ज्योति॑रुच्यते॒ द्वौ स्तोमौ᳚ प्रातःसव॒नं व॑हतो॒ यथा᳚ प्रा॒णश्चा॑पा॒नश्च॒ द्वौ माध्य॑न्दिन॒ꣳ॒ सव॑नं॒ यथा॒ चक्षु॑श्च॒ श्रोत्रं॑ च॒ द्वौ तृ॑तीयसव॒नं यथा॒ वाक्च॑ प्रति॒ष्ठा च॒ पुरु॑षसम्मितो॒ वा ए॒ष य॒ज्ञो\-ऽस्थू॑रिः~(१)

%7.1.1.2
यं कामं॑ का॒मय॑ते॒ तमे॒तेना॒भ्य॑श्ञुते॒ सर्व॒ꣴ॒ ह्यस्थू॑रिणाभ्यश्ञु॒ते᳚\-ऽग्निष्टो॒मेन॒ वै प्र॒जा\-प॑तिः प्र॒जा अ॑सृजत॒ ता अ॑ग्निष्टो॒मेनै॒व पर्य॑गृह्णा॒त्तासां॒ परि॑गृहीतानामश्वत॒रो\-ऽत्य॑प्रवत॒ तस्या॑नु॒हाय॒ रेत॒ आद॑त्त॒ तद्ग॑र्द॒भे न्य॑मा॒र्ट्तस्मा᳚द्गर्द॒भो द्वि॒रेता॒ अथो॑ आहु॒र्वड॑बायां॒ न्य॑मा॒र्डिति॒ तस्मा॒द्वड॑बा द्वि॒रेता॒ अथो॑ आहु॒रोष॑धीषु~(२)

%7.1.1.3
न्य॑मा॒र्डिति॒ तस्मा॒दोष॑ध॒यो\-ऽन॑भ्यक्ता रेभ॒न्त्यथो॑ आहुः प्र॒जासु॒ न्य॑मा॒र्डिति॒ तस्मा᳚द्य॒मौ जा॑येते॒ तस्मा॑दश्वत॒रो न प्र जा॑यत॒ आत्त॑रेता॒ हि तस्मा᳚द्ब॒र्॒\mbox{}हिष्यन॑वकॢप्तः सर्ववेद॒से वा॑ स॒हस्रे॒ वाव॑कॢ॒प्तो\-ऽति॒ ह्यप्र॑वत॒ य ए॒वं वि॒द्वान॑ग्निष्टो॒मेन॒ यज॑ते॒ प्राजा॑ताः प्र॒जा ज॒नय॑ति॒ परि॒ प्रजा॑ता गृह्णाति॒ तस्मा॑दाहुर्ज्येष्ठय॒ज्ञ इति॑~(३)

%7.1.1.4
प्र॒जा\-प॑ति॒र्वाव ज्येष्ठः॒ स ह्ये॑तेनाग्रे\-ऽय॑जत प्र॒जा\-प॑तिरकामयत॒ प्र जा॑ये॒येति॒ स मु॑ख॒तस्त्रि॒वृतं॒ निर॑मिमीत॒ तम॒ग्नि\-र्दे॒वतान्व॑सृज्यत गाय॒त्री छन्दो॑ रथन्त॒रꣳ साम॑ ब्राह्म॒णो म॑नु॒ष्या॑णाम॒जः प॑शू॒नान्तस्मा॒त्ते मुख्या॑ मुख॒तो ह्यसृ॑ज्य॒न्तोर॑सो बा॒हु\-भ्यां᳚ पञ्चद॒शं निर॑मिमीत॒ तमिन्द्रो॑ दे॒वतान्व॑सृज्यत त्रि॒ष्टुप्छन्दो॑ बृ॒हत्~(४)

%7.1.1.5
साम॑ राज॒न्यो॑ मनु॒ष्या॑णा॒मविः॑ पशू॒नान्तस्मा॒त्ते वी॒र्या॑वन्तो वी॒र्या᳚द्ध्यसृ॑ज्यन्त मध्य॒तः स॑प्तद॒शं निर॑मिमीत॒ तं विश्वे॑ दे॒वा दे॒वता॒ अन्व॑सृज्यन्त॒ जग॑ती॒ छन्दो॑ वैरू॒पꣳ साम॒ वैश्यो॑ मनु॒ष्या॑णां॒ गावः॑ पशू॒नान्तस्मा॒त्त आ॒द्या॑ अन्न॒धाना॒\-द्ध्यसृ॑ज्यन्त॒ तस्मा॒द्भूयाꣳ॑सो॒\-ऽन्येभ्यो॒ भूयि॑ष्ठा॒ हि दे॒वता॒ अन्वसृ॑ज्यन्त प॒त्त ए॑कवि॒ꣳ॒शं निर॑मिमीत॒ तम॑नु॒ष्टुप्छन्दः॑ [5[]

%7.1.1.6
अन्व॑सृज्यत वैरा॒जꣳ साम॑ शू॒द्रो म॑नु॒ष्या॑णा॒मश्वः॑ पशू॒नान्तस्मा॒त्तौ भू॑तसङ्क्रा॒मिणा॒वश्व॑श्च शू॒द्रश्च॒ तस्मा᳚च्छू॒द्रो य॒ज्ञे\-ऽन॑वकॢप्तो॒ न हि दे॒वता॒ अन्वसृ॑ज्यत॒ तस्मा॒त्पादा॒वुप॑ जीवतः प॒त्तो ह्यसृ॑ज्येतां प्रा॒णा वै त्रि॒वृद॑र्धमा॒साः प॑ञ्चद॒शः प्र॒जा\-प॑तिः सप्तद॒शस्त्रय॑ इ॒मे लो॒का अ॒सावा॑दि॒त्य ए॑कवि॒ꣳ॒श ए॒तस्मि॒न्वा ए॒ते श्रि॒ता ए॒तस्मि॒न्प्रति॑ष्ठिता॒ य ए॒वं वेदै॒तस्मि॑न्ने॒व श्र॑यत ए॒तस्मि॒न्प्रति॑ तिष्ठति॥~(६)

{\anuvakamend[{अस्थू॑रि॒रोष॑धीषु ज्येष्ठय॒ज्ञ इति॑ बृ॒हद॑नु॒ष्टुप्छन्दः॒ प्रति॑ष्ठिता॒ नव॑ च}]}%~(१)

%7.1.2.1
प्रा॒तः॒स॒व॒ने वै गा॑य॒त्रेण॒ छन्द॑सा त्रि॒वृते॒ स्तोमा॑य॒ ज्योति॒र्दध॑देति त्रि॒वृता᳚ ब्रह्मवर्च॒सेन॑ पञ्चद॒शाय॒ ज्योति॒र्दध॑देति पञ्चद॒शेनौज॑सा वी॒र्ये॑ण सप्तद॒शाय॒ ज्योति॒र्दध॑देति सप्तद॒शेन॑ प्राजाप॒त्येन॑ प्र॒जन॑नेनैकवि॒ꣳ॒शाय॒ ज्योति॒र्दध॑देति॒ स्तोम॑ ए॒व तथ्स्तोमा॑य॒ ज्योति॒र्दध॑दे॒त्यथो॒ स्तोम॑ ए॒व स्तोम॑म॒भि प्र ण॑यति॒ याव॑न्तो॒ वै स्तोमा॒स्ताव॑न्तः॒ कामा॒स्ताव॑न्तो लो॒कास्ताव॑न्ति॒ ज्योतीꣴ॑ष्ये॒ताव॑त ए॒व स्तोमा॑ने॒ताव॑तः॒ कामा॑ने॒ताव॑तो लो॒काने॒ताव॑न्ति॒ ज्योती॒ꣳ॒ष्यव॑ रुन्धे॥~(७)

{\anuvakamend[{ताव॑न्तो लो॒कास्त्रयो॑दश च}]}%~(२)

%7.1.3.1
ब्र॒ह्म॒वा॒दिनो॑ वदन्ति॒ स त्वै य॑जेत॒ यो᳚\-ऽग्निष्टो॒मेन॒ यज॑मा॒नो\-ऽथ॒ सर्व॑स्तोमेन॒ यजे॒तेति॒ यस्य॑ त्रि॒वृत॑मन्त॒र्यन्ति॑ प्रा॒णाꣴस्तस्या॒न्तर्य॑न्ति प्रा॒णेषु॒ मे\-ऽप्य॑स॒दिति॒ खलु॒ वै य॒ज्ञेन॒ यज॑मानो यजते॒ यस्य॑ पञ्चद॒शम॑न्त॒र्यन्ति॑ वी॒र्यं॑ तस्या॒न्तर्य॑न्ति वी॒र्ये॑ मे\-ऽप्य॑स॒दिति॒ खलु॒ वै य॒ज्ञेन॒ यज॑मानो यजते॒ यस्य॑ सप्तद॒शम॑न्त॒र्यन्ति॑~(८)

%7.1.3.2
प्र॒जां तस्या॒न्तर्य॑न्ति प्र॒जायां॒ मे\-ऽप्य॑स॒दिति॒ खलु॒ वै य॒ज्ञेन॒ यज॑मानो यजते॒ यस्यै॑कवि॒ꣳ॒शम॑न्त॒र्यन्ति॑ प्रति॒ष्ठां तस्या॒न्तर्य॑न्ति प्रति॒ष्ठायां॒ मे\-ऽप्य॑स॒दिति॒ खलु॒ वै य॒ज्ञेन॒ यज॑मानो यजते॒ यस्य॑ त्रिण॒वम॑न्त॒र्यन्त्यृ॒तूꣴश्च॒ तस्य॑ नक्ष॒त्रियां᳚ च वि॒राज॑म॒न्तर्य॑न्त्यृ॒तुषु॒ मे\-ऽप्य॑सन्नक्ष॒त्रिया॑यां च वि॒राजीति॑~(९)

%7.1.3.3
खलु॒ वै य॒ज्ञेन॒ यज॑मानो यजते॒ यस्य॑ त्रयस्त्रि॒ꣳ॒शम॑न्त॒र्यन्ति॑ दे॒वता॒स्तस्या॒न्तर्य॑न्ति दे॒वता॑सु॒ मे\-ऽप्य॑स॒दिति॒ खलु॒ वै य॒ज्ञेन॒ यज॑मानो यजते॒ यो वै स्तोमा॑नामव॒मं प॑र॒मतां॒ गच्छ॑न्तं॒ वेद॑ पर॒मता॑मे॒व ग॑च्छति त्रि॒वृद्वै स्तोमा॑नामव॒म\-स्त्रि॒वृत्प॑र॒मो य ए॒वं वेद॑ पर॒मता॑मे॒व ग॑च्छति॥~(१०)

{\anuvakamend[{स॒प्त॒द॒शम॑न्त॒र्यन्ति॑ वि॒राजीति॒ चतु॑श्चत्वारिꣳशच्च}]}%~(३)

%7.1.4.1
अङ्गि॑रसो॒ वै स॒त्रमा॑सत॒ ते सु॑व॒र्गं लो॒कमा॑य॒न्तेषाꣳ॑ ह॒विष्माꣴ॑श्च हवि॒ष्कृच्चा॑हीयेता॒न्ताव॑कामयेताꣳ सुव॒र्गं लो॒कमि॑या॒वेति॒ तावे॒तं द्वि॑रा॒त्रम॑पश्यतां॒ तमाह॑रतां॒ तेना॑यजेतां॒ ततो॒ वै तौ सु॑व॒र्गं लो॒कमै॑तां॒ य ए॒वं वि॒द्वान्द्वि॑रा॒त्रेण॒ यज॑ते सुव॒र्गमे॒व लो॒कमे॑ति॒ तावैतां॒ पूर्वे॒णा\-ऽह्ना\-ऽग॑च्छता॒मुत्त॑रेण~(११)

%7.1.4.2
अ॒भि॒प्ल॒वः पूर्व॒मह॑र्भवति॒ गति॒रुत्त॑रं॒ ज्योति॑ष्टोमो\-ऽग्निष्टो॒मः पूर्व॒मह॑र्भवति॒ तेज॒स्तेनाव॑ रुन्धे॒ सर्व॑स्तोमो\-ऽतिरा॒त्र उत्त॑र॒ꣳ॒ सर्व॒स्याप्त्यै॒ सर्व॒स्याव॑रुद्ध्यै गाय॒त्रं पूर्वेह॒न्थ्साम॑ भवति॒ तेजो॒ वै गा॑य॒त्री गा॑य॒त्री ब्र॑ह्मवर्च॒सं तेज॑ ए॒व ब्र॑ह्मवर्च॒स\-मा॒त्मन्ध॑त्ते॒ त्रैष्टु॑भ॒मुत्त॑र॒ ओजो॒ वै वी॒र्यं॑ त्रि॒ष्टुगोज॑ ए॒व वी॒र्य॑मा॒त्मन्ध॑त्ते रथन्त॒रं पूर्वे᳚~(१२)

%7.1.4.3
अह॒न्थ्साम॑ भवती॒यं वै र॑थन्त॒रम॒स्यामे॒व प्रति॑ तिष्ठति बृ॒हदुत्त॑रे॒\-ऽसौ वै बृ॒हद॒मुष्या॑मे॒व प्रति॑ तिष्ठति॒ तदा॑हुः॒ क्व॑ जग॑ती चानु॒ष्टुप्चेति॑ वैखान॒सं पूर्वे\-ऽह॒न्थ्साम॑ भवति॒ तेन॒ जग॑त्यै॒ नैति॑ षोड॒श्युत्त॑रे॒ तेना॑नु॒ष्टुभो\-ऽथा॑हु॒र्यथ्स॑मा॒ने᳚\-ऽर्धमा॒से स्याता॑मन्यत॒रस्याह्नो॑ वी॒र्य॑मनु॑ पद्ये॒तेत्य॑मावा॒स्या॑यां॒ पूर्व॒मह॑र्भव॒त्युत्त॑रस्मि॒न्नुत्त॑र॒न्नानै॒वार्ध॑मा॒सयो᳚र्भवतो॒ नाना॑वीर्ये भवतो ह॒विष्म॑न्निधनं॒ पूर्व॒मह॑र्भवति हवि॒ष्कृन्नि॑धन॒मुत्त॑रं॒ प्रति॑ष्ठित्यै॥~(१३)

{\anuvakamend[{उत्त॑रेण रथन्त॒रं पूर्वे\-ऽन्वेक॑विꣳशतिश्च}]}%~(४)

%7.1.5.1
आपो॒ वा इ॒दमग्रे॑ सलि॒लमा॑सी॒त्तस्मि॑न्प्र॒जा\-प॑तिर्वा॒युर्भू॒त्वाच॑र॒थ्स इ॒माम॑पश्य॒त्तां व॑रा॒हो भू॒त्वाह॑र॒त्तां वि॒श्वक॑र्मा भू॒त्वा व्य॑मा॒र्ट्थ्साप्र॑थत॒ सा पृ॑थि॒व्य॑भव॒त्तत्पृ॑थि॒व्यै पृ॑थिवि॒त्वन्तस्या॑मश्राम्यत्प्र॒जा\-प॑तिः॒ स दे॒वान॑सृजत॒ वसू᳚न्रु॒द्राना॑दि॒त्यान्ते दे॒वाः प्र॒जा\-प॑तिमब्रुव॒न्प्र जा॑यामहा॒ इति॒ सो᳚\-ऽब्रवीत्~(१४)

%7.1.5.2
यथा॒हं यु॒ष्माꣴस्तप॒सासृ॑क्ष्ये॒वं तप॑सि प्र॒जन॑नमिच्छध्व॒मिति॒ तेभ्यो॒\-ऽग्निमा॒यत॑नं॒ प्राय॑च्छदे॒तेना॒यत॑नेन श्राम्य॒तेति॒ ते᳚\-ऽग्निना॒यत॑नेनाश्राम्य॒न्ते सं॑वथ्स॒र एकां॒ गाम॑सृजन्त॒ तां वसु॑भ्यो रु॒द्रेभ्य॑ आदि॒त्येभ्यः॒ प्राय॑च्छन्ने॒ताꣳ र॑क्षध्व॒मिति॒ तां वस॑वो रु॒द्रा आ॑दि॒त्या अ॑रक्षन्त॒ सा वसु॑भ्यो रु॒द्रेभ्य॑ आदि॒त्येभ्यः॒ प्राजा॑यत॒ त्रीणि॑ च~(१५)

%7.1.5.3
श॒तानि॒ त्रय॑स्त्रिꣳशतं॒ चाथ॒ सैव स॑हस्रत॒म्य॑भव॒त्ते दे॒वाः प्र॒जा\-प॑तिमब्रुवन्थ्स॒हस्रे॑ण नो याज॒येति॒ सो᳚\-ऽग्निष्टो॒मेन॒ वसू॑नयाजय॒त्त इ॒मं लो॒कम॑जय॒न्तच्चा॑ददुः॒ स उ॒क्थ्ये॑न रु॒द्रान॑याजय॒त्ते᳚\-ऽन्तरि॑क्षमजय॒न्तच्चा॑ददुः॒ सो॑\-ऽतिरा॒त्रेणा॑\-दि॒त्यान॑याजय॒त्ते॑\-ऽमुं लो॒कम॑जय॒न्तच्चा॑ददु॒स्तद॒न्तरि॑क्षम्~(१६)

%7.1.5.4
व्यवै᳚र्यत॒ तस्मा᳚द्रु॒द्रा घातु॑का अनायत॒ना हि तस्मा॑दाहुः शिथि॒लं वै म॑ध्य॒ममह॑स्त्रिरा॒त्रस्य॒ वि हि तद॒वैर्य॒तेति॒ त्रैष्टु॑भं मध्य॒मस्याह्न॒ आज्य॑म्भवति सं॒याना॑नि सू॒क्तानि॑ शꣳसति षोड॒शिनꣳ॑ शꣳस॒त्यह्नो॒ धृत्या॒ अशि॑थिलं भावाय॒ तस्मा᳚त्त्रिरा॒त्रस्या᳚ग्निष्टो॒म ए॒व प्र॑थ॒ममहः॑ स्या॒दथो॒क्थ्यो\-ऽथा॑तिरा॒त्र ए॒षां लो॒कानां॒ विधृ॑त्यै॒ त्रीणि॑त्रीणि श॒तान्य॑नूचीना॒हमव्य॑वच्छिन्नानि ददाति~(१७)

%7.1.5.5
ए॒षां लो॒काना॒मनु॒ सन्त॑त्यै द॒शतं॒ न विच्छि॑न्द्याद्वि॒राजं॒ नेद्वि॑च्छि॒नदा॒नीत्यथ॒ या स॑हस्रत॒म्यासी॒त्तस्या॒मिन्द्र॑श्च॒ विष्णु॑श्च॒ व्याय॑च्छेता॒ꣳ॒ स इन्द्रो॑\-ऽमन्यता॒नया॒ वा इ॒दं विष्णुः॑ स॒हस्रं॑ वर्क्ष्यत॒ इति॒ तस्या॑मकल्पेतां॒ द्विभा॑ग॒ इन्द्र॒स्तृती॑ये॒ विष्णु॒स्तद्वा ए॒षाभ्यनू᳚च्यत उ॒भा जि॑ग्यथु॒रिति॒ तां वा ए॒ताम॑च्छावा॒कः~(१८)

%7.1.5.6
ए॒व शꣳ॑स॒त्यथ॒ या स॑हस्रत॒मी सा होत्रे॒ देयेति॒ होता॑रं॒ वा अ॒भ्यति॑रिच्यते॒ यद॑ति॒रिच्य॑ते॒ होताना᳚प्तस्यापयि॒ता\-था॑हुरुन्ने॒त्रे देयेत्यति॑रिक्ता॒ वा ए॒षा स॒हस्र॒स्याति॑रिक्त उन्ने॒तर्त्विजा॒मथा॑हुः॒ सर्वे᳚भ्यः सद॒स्ये᳚भ्यो॒ देयेत्यथा॑हुरुदा॒कृत्या॒ सा वशं॑ चरे॒दित्यथा॑हुर्ब्र॒ह्मणे॑ चा॒ग्नीधे॑ च॒ देयेति॑~(१९)

%7.1.5.7
द्विभा॑गं ब्र॒ह्मणे॒ तृती॑यम॒ग्नीध॑ ऐ॒न्द्रो वै ब्र॒ह्मा वै᳚ष्ण॒वो᳚\-ऽग्नीद्यथै॒व तावक॑ल्पेता॒मित्यथा॑हु॒र्या क॑ल्या॒णी ब॑हुरू॒पा सा देयेत्यथा॑हु॒र्या द्वि॑रू॒पोभ॒यत॑एनी॒ सा देयेति॑ स॒हस्र॑स्य॒ परि॑गृहीत्यै॒ तद्वा ए॒तथ्स॒हस्र॒स्याय॑नꣳ स॒हस्रꣴ॑ स्तो॒त्रीयाः᳚ स॒हस्रं॒ दक्षि॑णाः स॒हस्र॑सम्मितः सुव॒र्गो लो॒कः सु॑व॒र्गस्य॑ लो॒कस्या॒भिजि॑त्यै॥~(२०)

{\anuvakamend[{अ॒ब्र॒वी॒च्च॒ तद॒न्तरि॑क्षन्ददात्यच्छावा॒कश्च॒ देयेति॑ स॒प्तच॑त्वारिꣳशच्च}]}%~(५)

%7.1.6.1
सोमो॒ वै स॒हस्र॑मविन्द॒त्तमिन्द्रो\-ऽन्व॑विन्द॒त्तौ य॒मो न्याग॑च्छ॒त्ताव॑ब्रवी॒दस्तु॒ मे\-ऽत्रापीत्यस्तु॒ ही(३) इत्य॑ब्रूता॒ꣳ॒ स य॒म एक॑स्यां वी॒र्यं॑ पर्य॑पश्यदि॒यं वा अ॒स्य स॒हस्र॑स्य वी॒र्यं॑ बिभ॒र्तीति॒ ताव॑ब्रवीदि॒यं ममास्त्वे॒तद्यु॒वयो॒रिति॒ ताव॑ब्रूता॒ꣳ॒ सर्वे॒ वा ए॒तदे॒तस्यां᳚ वी॒र्यम्᳚~(२१)

%7.1.6.2
परि॑ पश्या॒मो\-ऽꣳश॒मा ह॑रामहा॒ इति॒ तस्या॒मꣳश॒माह॑रन्त॒ ताम॒फ्सु प्रावे॑शय॒न्थ्सोमा॑यो॒देहीति॒ सा रोहि॑णी पिङ्ग॒लैक॑हायनी रू॒पं कृ॒त्वा त्रय॑स्त्रिꣳशता च त्रि॒भिश्च॑ श॒तैः स॒होदैत्तस्मा॒द्रोहि॑ण्या पिङ्ग॒लयैक॑हायन्या॒ सोमं॑ क्रीणीया॒द्य ए॒वं वि॒द्वान्रोहि॑ण्या पिङ्ग॒लयैक॑हायन्या॒ सोमं॑ क्री॒णाति॒ त्रय॑स्त्रिꣳशता चै॒वास्य॑ त्रि॒भिश्च॑~(२२)

%7.1.6.3
श॒तैः सोमः॑ क्री॒तो भ॑वति॒ सुक्री॑तेन यजते॒ ताम॒फ्सु प्रावे॑शय॒न्निन्द्रा॑यो॒देहीति॒ सा रोहि॑णी लक्ष्म॒णा प॑ष्ठौ॒ही वार्त्र॑घ्नी रू॒पं कृ॒त्वा त्रय॑स्त्रिꣳशता च त्रि॒भिश्च॑ श॒तैः स॒होदैत्तस्मा॒द्रोहि॑णीं लक्ष्म॒णां प॑ष्ठौ॒हीं वार्त्र॑घ्नीं दद्या॒द्य ए॒वं वि॒द्वान्रोहि॑णीं लक्ष्म॒णां प॑ष्ठौ॒हीं वार्त्र॑घ्नीं॒ ददा॑ति॒ त्रय॑स्त्रिꣳशच्चै॒वास्य॒ त्रीणि॑ च श॒तानि॒ सा द॒त्ता~(२३)

%7.1.6.4
भ॒व॒ति॒ ताम॒फ्सु प्रावे॑शयन् य॒मायो॒देहीति॒ सा जर॑ती मू॒र्खा त॑ज्जघ॒न्या रू॒पं कृ॒त्वा त्रय॑स्त्रिꣳशता च त्रि॒भिश्च॑ श॒तैः स॒होदैत्तस्मा॒ज्जर॑तीं मू॒र्खां त॑ज्जघ॒न्याम॑नु॒स्तर॑णीं कुर्वीत॒ य ए॒वं वि॒द्वाञ्जर॑तीं मू॒र्खां त॑ज्जघ॒न्याम॑नु॒स्तर॑णीं कुरु॒ते त्रय॑स्त्रिꣳशच्चै॒वास्य॒ त्रीणि॑ च श॒तानि॒ सामुष्मिँ॑ल्लो॒के भ॑वति॒ वागे॒व स॑हस्रत॒मी तस्मा᳚त्~(२४)

%7.1.6.5
वरो॒ देयः॒ सा हि वरः॑ स॒हस्र॑मस्य॒ सा द॒त्ता भ॑वति॒ तस्मा॒द्वरो॒ न प्र॑ति॒गृह्यः॒ सा हि वरः॑ स॒हस्र॑मस्य॒ प्रति॑गृहीतं भवती॒यं वर॒ इति॑ ब्रूया॒दथा॒न्यां ब्रू॑यादि॒यं ममेति॒ तथा᳚स्य॒ तथ्स॒हस्र॒मप्र॑तिगृहीतं भवत्युभयतए॒नी स्या॒त्तदा॑हुरन्यत\-ए॒नी स्या᳚थ्स॒हस्रं॑ प॒रस्ता॒देत॒मिति॒ यैव वरः॑~(२५)

%7.1.6.6
क॒ल्या॒णी रू॒पस॑मृद्धा॒ सा स्या॒थ्सा हि वरः॒ समृ॑द्ध्यै॒ तामुत्त॑रे॒णाग्नी᳚ध्रं पर्या॒णीया॑हव॒नीय॒स्यान्ते᳚ द्रोणकल॒शमव॑ घ्रापये॒दा जि॑घ्र क॒लशं॑ मह्यु॒रुधा॑रा॒ पय॑स्व॒त्या त्वा॑ विश॒न्त्विन्द॑वः समु॒द्रमि॑व॒ सिन्ध॑वः॒ सा मा॑ स॒हस्र॒ आ भ॑ज प्र॒जया॑ प॒शुभिः॑ स॒ह पुन॒र्मा वि॑शताद्र॒यिरिति॑ प्र॒जयै॒वैनं॑ प॒शुभी॑ र॒य्या सम्~(२६)

%7.1.6.7
अ॒र्ध॒य॒ति॒ प्र॒जावा᳚न्पशु॒मान्र॑यि॒मान्भ॑वति॒ य ए॒वं वेद॒ तया॑ स॒हाग्नी᳚ध्रं प॒रेत्य॑ पु॒रस्ता᳚त्प्र॒तीच्यां॒ तिष्ठ॑न्त्यां जुहुयादु॒भा जि॑ग्यथु॒र्न परा॑ जयेथे॒ न परा॑ जिग्ये कत॒रश्च॒नैनोः᳚। इन्द्र॑श्च विष्णो॒ यदप॑स्पृधेथां त्रे॒धा स॒हस्रं॒ वि तदै॑रयेथा॒मिति॑ त्रेधाविभ॒क्तं वै त्रि॑रा॒त्रे स॒हस्रꣳ॑ साह॒स्रीमे॒वैनां᳚ करोति स॒हस्र॑स्यै॒वैनां॒ मात्रा᳚म्~(२७)

%7.1.6.8
क॒रो॒ति॒ रू॒पाणि॑ जुहोति रू॒पैरे॒वैना॒ꣳ॒ सम॑र्धयति॒ तस्या॑ उपो॒त्थाय॒ कर्ण॒मा ज॑पे॒दिडे॒ रन्ते\-ऽदि॑ते॒ सर॑स्वति॒ प्रिये॒ प्रेय॑सि॒ महि॒ विश्रु॑त्ये॒तानि॑ ते अघ्निये॒ नामा॑नि सु॒कृतं॑ मा दे॒वेषु॑ ब्रूता॒दिति॑ दे॒वेभ्य॑ ए॒वैन॒मा वे॑दय॒त्यन्वे॑नं दे॒वा बु॑ध्यन्ते॥~(२८)

{\anuvakamend[{ए॒तदे॒तस्यां᳚ वी॒र्य॑मस्य त्रि॒भिश्च॑ द॒त्ता स॑हस्रत॒मी तस्मा॑दे॒व वरः॒ सं मात्रा॒मेका॒न्नच॑त्वारि॒ꣳ॒शच्च॑}]}%~(६)

%7.1.7.1
स॒ह॒स्र॒त॒म्या॑ वै यज॑मानः सुव॒र्गं लो॒कमे॑ति॒ सैनꣳ॑ सुव॒र्गं लो॒कं ग॑मयति॒ सा मा॑ सुव॒र्गं लो॒कं ग॑म॒येत्या॑ह सुव॒र्गमे॒वैनं॑ लो॒कं ग॑मयति॒ सा मा॒ ज्योति॑ष्मन्तं लो॒कं ग॑म॒येत्या॑ह॒ ज्योति॑ष्मन्तमे॒वैनं॑ लो॒कं ग॑मय॒ति सा मा॒ सर्वा॒न्पुण्याँ᳚ल्लो॒कान्ग॑म॒येत्या॑ह॒ सर्वा॑ने॒वैनं॒ पुण्याँ᳚ल्लो॒कान्ग॑मयति॒ सा~(२९)

%7.1.7.2
मा॒ प्र॒ति॒ष्ठां ग॑मय प्र॒जया॑ प॒शुभिः॑ स॒ह पुन॒र्मा वि॑शताद्र॒यिरिति॑ प्र॒जयै॒वैनं॑ प॒शुभी॑ र॒य्यां प्रति॑\-ष्ठापयति प्र॒जावा᳚न्पशु॒मान्र॑यि॒मान्भ॑वति॒ य ए॒वं वेद॒ ताम॒ग्नीधे॑ वा ब्र॒ह्मणे॑ वा॒ होत्रे॑ वोद्गा॒त्रे वा᳚ध्व॒र्यवे॑ वा दद्याथ्स॒हस्र॑मस्य॒ सा द॒त्ता भ॑वति स॒हस्र॑मस्य॒ प्रति॑गृहीतं भवति॒ यस्तामवि॑द्वान्~(३०)

%7.1.7.3
प्र॒ति॒गृ॒ह्णाति॒ तां प्रति॑ गृह्णीया॒देका॑सि॒ न स॒हस्र॒मेकां᳚ त्वा भू॒तां प्रति॑ गृह्णामि॒ न स॒हस्र॒मेका॑ मा भू॒ता वि॑श॒ मा स॒हस्र॒मित्येका॑मे॒वैनां᳚ भू॒तां प्रति॑ गृह्णाति॒ न स॒हस्रं॒ य ए॒वं वेद॑ स्यो॒नासि॑ सु॒षदा॑ सु॒शेवा᳚ स्यो॒ना मा वि॑श सु॒षदा॒ मा वि॑श सु॒शेवा॒ मा वि॑श~(३१)

%7.1.7.4
इत्या॑ह स्यो॒नैवैनꣳ॑ सु॒षदा॑ सु॒शेवा॑ भू॒ता वि॑शति॒ नैनꣳ॑ हिनस्ति ब्रह्मवा॒दिनो॑ वदन्ति स॒हस्र॑ꣳ सहस्रत॒म्यन्वे॒ती(३) स॑हस्रत॒मीꣳ स॒हस्रा(३)मिति॒ यत्प्राची॑मुथ्सृ॒जेथ्स॒हस्रꣳ॑ सहस्रत॒म्यन्वि॑या॒त्तथ्स॒हस्र॑मप्रज्ञा॒त्रꣳ सु॑व॒र्गं लो॒कं न प्र जा॑नीयात्प्र॒तीची॒मुथ्सृ॑जति॒ ताꣳ स॒हस्र॒मनु॑ प॒र्याव॑र्तते॒ सा प्र॑जान॒ती सु॑व॒र्गं लो॒कमे॑ति॒ यज॑मानम॒भ्युथ्सृ॑जति क्षि॒प्रे स॒हस्रं॒ प्र जा॑यत उत्त॒मा नी॒यते᳚ प्रथ॒मा दे॒वान्ग॑च्छति॥~(३२)

{\anuvakamend[{लो॒कान्ग॑मयति॒ सावि॑द्वान्थ्सु॒शेवा॒ मावि॑श॒ यज॑मानं॒ द्वाद॑श च}]}%~(७)

%7.1.8.1
अत्रि॑रददा॒दौर्वा॑य प्र॒जां पु॒त्रका॑माय॒ स रि॑रिचा॒नो॑\-ऽमन्यत॒ निर्वी᳚र्यः शिथि॒लो या॒तया॑मा॒ स ए॒तं च॑तूरा॒त्रम॑पश्य॒त् तमाह॑र॒त्तेना॑यजत॒ ततो॒ वै तस्य॑ च॒त्वारो॑ वी॒रा आजा॑यन्त॒ सुहो॑ता॒ सू᳚द्गाता॒ स्व॑ध्वर्युः॒ सुस॑भेयो॒ य ए॒वं वि॒द्वाꣴश्च॑तूरा॒त्रेण॒ यज॑त॒ आस्य॑ च॒त्वारो॑ वी॒रा जा॑यन्ते॒ सुहो॑ता॒ सू᳚द्गाता॒ स्व॑ध्वर्युः॒ सुस॑भेयो॒ ये च॑तुर्वि॒ꣳ॒शाः पव॑माना ब्रह्मवर्च॒सं तत्~(३३)

%7.1.8.2
य उ॒द्यन्तः॒ स्तोमाः॒ श्रीः सात्रिꣴ॑ श्र॒द्धादे॑वं॒ यज॑मानं च॒त्वारि॑ वीर्याणि॒ नोपा॑नम॒न्तेज॑ इन्द्रि॒यं ब्र॑ह्मवर्च॒सम॒न्नाद्य॒ꣳ॒ स ए॒ताꣴश्च॒तुर॒श्चतु॑ष्टोमा॒न्थ्सोमा॑नपश्य॒त्तानाह॑र॒त्तैर॑यजत॒ तेज॑ ए॒व प्र॑थ॒मेनावा॑रुन्धेन्द्रि॒यं द्वि॒तीये॑न ब्रह्मवर्च॒सं तृ॒तीये॑ना॒न्नाद्यं॑ चतु॒र्थेन॒ य ए॒वं वि॒द्वाꣴश्च॒तुर॒श्चतु॑ष्टोमा॒न्थ्सोमा॑ना॒हर॑ति॒ तैर्यज॑ते॒ तेज॑ ए॒व प्र॑थ॒मेनाव॑ रुन्ध इन्द्रि॒यं द्वि॒तीये॑न ब्रह्मवर्च॒सं तृ॒तीये॑ना॒न्नाद्यं॑ चतु॒र्थेन॒ यामे॒वात्रि॒र्॒\mbox{}ऋद्धि॒मार्ध्नो॒त्तामे॒व यज॑मान ऋध्नोति॥~(३४)

{\anuvakamend[{तत्तेज॑ ए॒वाष्टाद॑श च}]}%~(८)

%7.1.9.1
ज॒मद॑ग्निः॒ पुष्टि॑कामश्चतूरा॒त्रेणा॑यजत॒ स ए॒तान्पोषाꣳ॑ अपुष्य॒त्तस्मा᳚त्पलि॒तौ जाम॑दग्नियौ॒ न सं जा॑नाते ए॒ताने॒व पोषा᳚न्पुष्यति॒ य ए॒वं वि॒द्वाꣴश्च॑तूरा॒त्रेण॒ यज॑ते पुरोडा॒शिन्य॑ उप॒सदो॑ भवन्ति प॒शवो॒ वै पु॑रो॒डाशः॑ प॒शूने॒वाव॑ रु॒न्धे\-ऽन्नं॒ वै पु॑रो॒डाशो\-ऽन्न॑मे॒वाव॑ रुन्धे\-ऽन्ना॒दः प॑शु॒मान्भ॑वति॒ य ए॒वं वि॒द्वाꣴश्च॑तूरा॒त्रेण॒ यज॑ते॥~(३५)

{\anuvakamend[{ज॒मद॑ग्निर॒ष्टाच॑त्वारिꣳशत्}]}%~(९)

%7.1.10.1
सं॒व॒थ्स॒रो वा इ॒दमेक॑ आसी॒थ्सो॑\-ऽकामयत॒र्तून्थ्सृ॑जे॒येति॒ स ए॒तं प॑ञ्चरा॒त्रम॑पश्य॒त्तमाह॑र॒त्तेना॑यजत॒ ततो॒ वै स ऋ॒तून॑सृजत॒ य ए॒वं वि॒द्वान्प॑ञ्चरा॒त्रेण॒ यज॑ते॒ प्रैव जा॑यते॒ त ऋ॒तवः॑ सृ॒ष्टा न व्याव॑र्तन्त॒ त ए॒तं प॑ञ्चरा॒त्रम॑पश्य॒न् तमाह॑र॒न्तेना॑यजन्त॒ ततो॒ वै ते व्याव॑र्तन्त~(३६)

%7.1.10.2
य ए॒वं वि॒द्वान्प॑ञ्चरा॒त्रेण॒ यज॑ते॒ वि पा॒प्मना॒ भ्रातृ॑व्ये॒णा व॑र्तते॒ सार्व॑सेनिः शौचे॒यो॑\-ऽकामयत पशु॒मान्थ्स्या॒मिति॒ स ए॒तं प॑ञ्चरा॒त्रमाह॑र॒त्तेना॑यजत॒ ततो॒ वै स स॒हस्रं॑ प॒शून्प्राप्नो॒द्य ए॒वं वि॒द्वान्प॑ञ्चरा॒त्रेण॒ यज॑ते॒ प्र स॒हस्रं॑ प॒शूना᳚प्नोति बब॒रः प्रावा॑हणिरकामयत वा॒चः प्र॑वदि॒ता स्या॒मिति॒ स ए॒तं प॑ञ्चरा॒त्रमा~(३७)

%7.1.10.3
अ॒ह॒र॒त्तेना॑यजत॒ ततो॒ वै स वा॒चः प्र॑वदि॒ताभ॑व॒द्य ए॒वं वि॒द्वान्प॑ञ्चरा॒त्रेण॒ यज॑ते प्रवदि॒तैव वा॒चो भ॑व॒त्यथो॑ एनं वा॒चस्पति॒रित्या॑हु॒रना᳚प्तश्चतूरा॒त्रो\-ऽति॑रिक्तः षड्रा॒त्रो\-ऽथ॒ वा ए॒ष सं॑ प्र॒ति य॒ज्ञो यत्प॑ञ्चरा॒त्रो य ए॒वं वि॒द्वान्प॑ञ्चरा॒त्रेण॒ यज॑ते सम्प्र॒त्ये॑व य॒ज्ञेन॑ यजते पञ्चरा॒त्रो भ॑वति॒ पञ्च॒ वा ऋ॒तवः॑ संवथ्स॒रः~(३८)

%7.1.10.4
ऋ॒तुष्वे॒व सं॑वथ्स॒रे प्रति॑ तिष्ठ॒त्यथो॒ पञ्चा᳚क्षरा प॒ङ्क्तिः पाङ्क्तो॑ य॒ज्ञो य॒ज्ञमे॒वाव॑ रुन्धे त्रि॒वृद॑ग्निष्टो॒मो भ॑वति॒ तेज॑ ए॒वाव॑ रुन्धे पञ्चद॒शो भ॑वतीन्द्रि॒यमे॒वाव॑ रुन्धे सप्तद॒शो भ॑वत्य॒न्नाद्य॒स्याव॑रुद्ध्या॒ अथो॒ प्रैव तेन॑ जायते पञ्चवि॒ꣳ॒शो᳚\-ऽग्निष्टो॒मो भ॑वति प्र॒जा\-प॑ते॒राप्त्यै॑ महाव्र॒तवा॑न॒न्नाद्य॒स्याव॑रुद्ध्यै विश्व॒जिथ्सर्व॑पृष्ठो\-ऽतिरा॒त्रो भ॑वति॒ सर्व॑स्या॒भिजि॑त्यै॥~(३९)

{\anuvakamend[{ते व्याव॑र्तन्त प्रवदि॒ता स्या॒मिति॒ स ए॒तं प॑ञ्चरा॒त्रमा सं॑वथ्स॒रो॑\-ऽभिजि॑त्यै}]}%॥10॥

%7.1.11.1
दे॒वस्य॑ त्वा सवि॒तुः प्र॑स॒वे᳚\-ऽश्विनो᳚र्बा॒हु\-भ्यां᳚ पू॒ष्णो हस्ता᳚भ्या॒मा द॑द इ॒माम॑गृभ्णन्रश॒नामृ॒तस्य॒ पूर्व॒ आयु॑षि वि॒दथे॑षु क॒व्या। तया॑ दे॒वाः सु॒तमा ब॑भूवुर्\mbox{}ऋ॒तस्य॒ साम᳚न्थ्स॒रमा॒रप॑न्ती। अ॒भि॒धा अ॑सि॒ भुव॑नमसि य॒न्तासि॑ ध॒र्तासि॒ सो᳚\-ऽग्निं वै᳚श्वान॒रꣳ सप्र॑थसं गच्छ॒ स्वाहा॑कृतः पृथि॒व्यां य॒न्ता राड्य॒न्तासि॒ यम॑नो ध॒र्तासि॑ ध॒रुणः॑ कृ॒ष्यै त्वा॒ क्षेमा॑य त्वा र॒य्यै त्वा॒ पोषा॑य त्वा पृथि॒व्यै त्वा॒ऽन्तरि॑क्षाय त्वा दि॒वे त्वा॑ स॒ते त्वास॑ते त्वा॒द्भ्यस्त्वौष॑धीभ्यस्त्वा॒ विश्वे᳚भ्यस्त्वा भू॒तेभ्यः॑॥~(४०)

{\anuvakamend[{ध॒रुणः॒ प़ञ्च॑विꣳशतिश्च}]}%॥11॥

%7.1.12.1
वि॒भूर्मा॒त्रा प्र॒भूः पि॒त्राश्वो॑\-ऽसि॒ हयो॒\-ऽस्यत्यो॑\-ऽसि॒ नरो॒\-ऽस्यर्वा॑सि॒ सप्ति॑रसि वा॒ज्य॑सि॒ वृषा॑सि नृ॒मणा॑ असि॒ ययु॒र्नामा᳚स्यादि॒त्यानां॒ पत्वान्वि॑ह्य॒ग्नये॒ स्वाहा॒ स्वाहे᳚न्द्रा॒ग्निभ्या॒ꣴ॒ स्वाहा᳚ प्र॒जा\-प॑तये॒ स्वाहा॒ विश्वे᳚भ्यो दे॒वेभ्यः॒ स्वाहा॒ सर्वा᳚भ्यो दे॒वेता᳚भ्य इ॒ह धृतिः॒ स्वाहे॒ह विधृ॑तिः॒ स्वाहे॒ह रन्तिः॒ स्वाहे॒ह रम॑तिः॒ स्वाहा॒ भूर॑सि भु॒वे त्वा॒ भव्या॑य त्वा भविष्य॒ते त्वा॒ विश्वे᳚भ्यस्त्वा भू॒तेभ्यो॒ देवा॑ आशापाला ए॒तं दे॒वेभ्यो\-ऽश्वं॒ मेधा॑य॒ प्रोक्षि॑तं गोपायत॥~(४१)

{\anuvakamend[{रन्तिः॒ स्वाहा॒ द्वाविꣳ॑शतिश्च}]}%॥12॥

%7.1.13.1
आय॑नाय॒ स्वाहा॒ प्राय॑णाय॒ स्वाहो᳚द्द्रा॒वाय॒ स्वाहोद्द्रु॑ताय॒ स्वाहा॑ शूका॒राय॒ स्वाहा॒ शूकृ॑ताय॒ स्वाहा॒ पला॑यिताय॒ स्वाहा॒\-ऽऽ\-पला॑यिताय॒ स्वाहा॒\-ऽऽ\-वल्ग॑ते॒ स्वाहा॑ परा॒वल्ग॑ते॒ स्वाहा॑\-ऽऽ\-य॒ते स्वाहा᳚ प्रय॒ते स्वाहा॒ सर्व॑स्मै॒ स्वाहा᳚॥~(४२)

{\anuvakamend[{आय॑ना॒योत्त॑रमा॒पला॑यिताय॒ षड्विꣳ॑शतिः}]}%॥13॥

%7.1.14.1
अ॒ग्नये॒ स्वाहा॒ सोमा॑य॒ स्वाहा॑ वा॒यवे॒ स्वाहा॒पां मोदा॑य॒ स्वाहा॑ सवि॒त्रे स्वाहा॒ सर॑स्वत्यै॒ स्वाहेन्द्रा॑य॒ स्वाहा॒ बृह॒स्पत॑ये॒ स्वाहा॑ मि॒त्राय॒ स्वाहा॒ वरु॑णाय॒ स्वाहा॒ सर्व॑स्मै॒ स्वाहा᳚॥~(४३)

{\anuvakamend[{}]}

%7.1.15.1
पृ॒थि॒व्यै स्वाहा॒\-ऽन्तरि॑क्षाय॒ स्वाहा॑ दि॒वे स्वाहा॒ सूर्या॑य॒ स्वाहा॑ च॒न्द्रम॑से॒ स्वाहा॒ नक्ष॑त्रेभ्यः॒ स्वाहा॒ प्राच्यै॑ दि॒शे स्वाहा॒ दक्षि॑णायै दि॒शे स्वाहा᳚ प्र॒तीच्यै॑ दि॒शे स्वाहोदी᳚च्यै दि॒शे स्वाहो॒र्ध्वायै॑ दि॒शे स्वाहा॑ दि॒ग्भ्यः स्वाहा॑\-ऽ\-वान्तरदि॒शाभ्यः॒ स्वाहा॒ समा᳚भ्यः॒ स्वाहा॑ श॒रद्भ्यः॒ स्वाहा॑\-ऽहोरा॒त्रेभ्यः॒ स्वाहा᳚\-ऽर्धमा॒सेभ्यः॒ स्वाहा॒ मासे᳚भ्यः॒ स्वाह॒र्तुभ्यः॒ स्वाहा॑ संवथ्स॒राय॒ स्वाहा॒ सर्व॑स्मै॒ स्वाहा᳚॥~(४४)

{\anuvakamend[{}]}

%7.1.16.1
अ॒ग्नये॒ स्वाहा॒ सोमा॑य॒ स्वाहा॑ सवि॒त्रे स्वाहा॒ सर॑स्वत्यै॒ स्वाहा॑ पू॒ष्णे स्वाहा॒ बृह॒स्पत॑ये॒ स्वाहा॒\-ऽपां मोदा॑य॒ स्वाहा॑ वा॒यवे॒ स्वाहा॑ मि॒त्राय॒ स्वाहा॒ वरु॑णाय॒ स्वाहा॒ सर्व॑स्मै॒ स्वाहा᳚॥~(४५)

{\anuvakamend[{}]}

%7.1.17.1
पृ॒थि॒व्यै स्वाहा॒\-ऽन्तरि॑क्षाय॒ स्वाहा॑ दि॒वे स्वाहा॒\-ऽग्नये॒ स्वाहा॒ सोमा॑य॒ स्वाहा॒ सूर्या॑य॒ स्वाहा॑ च॒न्द्रम॑से॒ स्वाहा\-ऽह्ने॒ स्वाहा॒ रात्रि॑यै॒ स्वाह॒र्जवे॒ स्वाहा॑ सा॒धवे॒ स्वाहा॑ सुक्षि॒त्यै स्वाहा᳚ क्षु॒धे स्वाहा॑\-ऽऽ\-शिति॒म्ने स्वाहा॒ रोगा॑य॒ स्वाहा॑ हि॒माय॒ स्वाहा॑ शी॒ताय॒ स्वाहा॑\-ऽऽ\-त॒पाय॒ स्वाहा\-ऽर॑ण्याय॒ स्वाहा॑ सुव॒र्गाय॒ स्वाहा॑ लो॒काय॒ स्वाहा॒ सर्व॑स्मै॒ स्वाहा᳚॥~(४६)

{\anuvakamend[{}]}

%7.1.18.1
भुवो॑ दे॒वानां॒ कर्म॑णा॒पस॒र्तस्य॑ प॒थ्या॑सि॒ वसु॑भिर्दे॒वेभि॑र्दे॒वत॑या गाय॒त्रेण॑ त्वा॒ छन्द॑सा युनज्मि वस॒न्तेन॑ त्व॒र्तुना॑ ह॒विषा॑ दीक्षयामि रु॒द्रेभि॑र्दे॒वेभि॑र्दे॒वत॑या॒ त्रैष्टु॑भेन त्वा॒ छन्द॑सा युनज्मि ग्री॒ष्मेण॑ त्व॒र्तुना॑ ह॒विषा॑ दीक्षयाम्यादि॒त्येभि॑\-र्दे॒वेभि॑र्दे॒वत॑या॒ जाग॑तेन त्वा॒ छन्द॑सा युनज्मि व॒र्॒\mbox{}षाभि॑स्त्व॒र्तुना॑ ह॒विषा॑ दीक्षयामि॒ विश्वे॑भिर्दे॒वेभि॑र्दे॒वत॒यानु॑ष्टुभेन त्वा॒ छन्द॑सा युनज्मि~(४७)

%7.1.18.2
श॒रदा᳚ त्व॒र्तुना॑ ह॒विषा॑ दीक्षया॒म्यङ्गि॑रोभिर्दे॒वेभि॑र्दे॒वत॑या॒ पाङ्क्ते॑न त्वा॒ छन्द॑सा युनज्मि हेमन्तशिशि॒रा\-भ्यां᳚ त्व॒र्तुना॑ ह॒विषा॑ दीक्षया॒म्याहं दी॒क्षाम॑रुहमृ॒तस्य॒ पत्नीं᳚ गाय॒त्रेण॒ छन्द॑सा॒ ब्रह्म॑णा च॒र्तꣳ स॒त्ये॑\-ऽधाꣳ स॒त्यमृ॒ते॑\-ऽधाम्। म॒हीमू॒ षु सु॒त्रामा॑णमि॒ह धृतिः॒ स्वाहे॒ह विधृ॑तिः॒ स्वाहे॒ह रन्तिः॒ स्वाहे॒ह रम॑तिः॒ स्वाहा᳚॥~(४८)

{\anuvakamend[{}]}

%7.1.19.1
ई॒ङ्का॒राय॒ स्वाहें कृ॑ताय॒ स्वाहा॒ क्रन्द॑ते॒ स्वाहा॑\-ऽव॒क्रन्द॑ते॒ स्वाहा॒ प्रोथ॑ते॒ स्वाहा᳚ प्र॒प्रोथ॑ते॒ स्वाहा॑ ग॒न्धाय॒ स्वाहा᳚ घ्रा॒ताय॒ स्वाहा᳚ प्रा॒णाय॒ स्वाहा᳚ व्या॒नाय॒ स्वाहा॑\-ऽपा॒नाय॒ स्वाहा॑ सन्दी॒यमा॑नाय॒ स्वाहा॒ सन्दि॑ताय॒ स्वाहा॑ विचृ॒त्यमा॑नाय॒ स्वाहा॒ विचृ॑त्ताय॒ स्वाहा॑ पलायि॒ष्यमा॑णाय॒ स्वाहा॒ पला॑यिताय॒ स्वाहो॑परꣴस्य॒ते स्वाहोप॑रताय॒ स्वाहा॑ निवेक्ष्य॒ते स्वाहा॑ निवि॒शमा॑नाय॒ स्वाहा॒ निवि॑ष्टाय॒ स्वाहा॑ निषथ्स्य॒ते स्वाहा॑ नि॒षीद॑ते॒ स्वाहा॒ निष॑ण्णाय॒ स्वाहा॑~(४९)



%7.1.19.2
आ॒सि॒ष्य॒ते स्वाहा\-ऽऽ\-सी॑नाय॒ स्वाहा॑\-ऽऽ\-सि॒ताय॒ स्वाहा॑ निपथ्स्य॒ते स्वाहा॑ नि॒पद्य॑मानाय॒ स्वाहा॒ निप॑न्नाय॒ स्वाहा॑ शयिष्य॒ते स्वाहा॒ शया॑नाय॒ स्वाहा॑ शयि॒ताय॒ स्वाहा॑ सम्मीलिष्य॒ते स्वाहा॑ स॒म्मील॑ते॒ स्वाहा॒ सम्मी॑लिताय॒ स्वाहा᳚ स्वफ्स्य॒ते स्वाहा᳚ स्वप॒ते स्वाहा॑ सु॒प्ताय॒ स्वाहा᳚ प्रभोथ्स्य॒ते स्वाहा᳚ प्र॒बुध्य॑मानाय॒ स्वाहा॒ प्रबु॑द्धाय॒ स्वाहा॑ जागरिष्य॒ते स्वाहा॒ जाग्र॑ते॒ स्वाहा॑ जागरि॒ताय॒ स्वाहा॒ शुश्रू॑षमाणाय॒ स्वाहा॑ शृण्व॒ते स्वाहा᳚ श्रु॒ताय॒ स्वाहा॑ वीक्षिष्य॒ते स्वाहा᳚~(५०)

%7.1.19.3
वीक्ष॑माणाय॒ स्वाहा॒ वीक्षि॑ताय॒ स्वाहा॑ सꣳहास्य॒ते स्वाहा॑ स॒ञ्जिहा॑नाय॒ स्वाहो॒ज्जिहा॑नाय॒ स्वाहा॑ विवर्थ्स्य॒ते स्वाहा॑ वि॒वर्त॑मानाय॒ स्वाहा॒ विवृ॑त्ताय॒ स्वाहो᳚त्थास्य॒ते स्वाहो॒त्तिष्ठ॑ते॒ स्वाहोत्थि॑ताय॒ स्वाहा॑ विधविष्य॒ते स्वाहा॑ विधून्वा॒नाय॒ स्वाहा॒ विधू॑ताय॒ स्वाहो᳚त्क्रꣴस्य॒ते स्वाहो॒त्क्राम॑ते॒ स्वाहोत्क्रा᳚न्ताय॒ स्वाहा॑ चङ्क्रमिष्य॒ते स्वाहा॑ चङ्क्र॒म्यमा॑णाय॒ स्वाहा॑ चङ्क्रमि॒ताय॒ स्वाहा॑ कण्डूयिष्य॒ते स्वाहा॑ कण्डू॒यमा॑नाय॒ स्वाहा॑ कण्डूयि॒ताय॒ स्वाहा॑ निकषिष्य॒ते स्वाहा॑ नि॒कष॑माणाय॒ स्वाहा॒ निक॑षिताय॒ स्वाहा॒ यदत्ति॒ तस्मै॒ स्वाहा॒ यत्पिब॑ति॒ तस्मै॒ स्वाहा॒ यन्मेह॑ति॒ तस्मै॒ स्वाहा॒ यच्छकृ॑त्क॒रोति॒ तस्मै॒ स्वाहा॒ रेत॑से॒ स्वाहा᳚ प्र॒जाभ्यः॒ स्वाहा᳚ प्रजन॑नाय॒ स्वाहा॒ सर्व॑स्मै॒ स्वाहा᳚॥~(५१)

{\anuvakamend[{}]}

%7.1.20.1
अ॒ग्नये॒ स्वाहा॑ वा॒यवे॒ स्वाहा॒ सूर्या॑य॒ स्वाह॒र्तम॑स्यृ॒तस्य॒र्तम॑सि स॒त्यम॑सि स॒त्यस्य॑ स॒त्यम॑स्यृ॒तस्य॒ पन्था॑ असि दे॒वानां᳚ छा॒यामृ॑तस्य॒ नाम॒ तथ्स॒त्यं यत्त्वं प्र॒जा\-प॑ति॒रस्यधि॒ यद॑स्मिन्वा॒जिनी॑व॒ शुभः॒ स्पर्ध॑न्ते॒ दिवः॒ सूर्ये॑ण॒ विशो॒\-ऽपो वृ॑णा॒नः प॑वते क॒व्यन्प॒शुं न गो॒पा इर्यः॒ परि॑ज्मा~(५२)


\prashnaend{प्र॒जन॑नं प्रातः सव॒ने वै ब्र॑ह्मवा॒दिनः॒ स त्वा अङ्गि॑रस॒ आपो॒ वै सोमो॒ वै स॑हस्रत॒म्याऽत्रि॑र्ज॒मद॑ग्निः संवथ्स॒रो दे॒वस्य॑ वि॒भूराय॑नाया॒ग्नये॑ पृथि॒व्या अ॒ग्नये॑ पृथि॒व्यै भुव॑ ईङ्का॒राया॒ऽग्नये॑ वा॒यवे॒ सूर्या॑य विꣳश॒तिः॥२०॥}{प्र॒जन॑न॒मङ्गि॑रसः सोमो॒ वै प्र॑तिगृ॒ह्णाति॑ वी॒भूर्वीक्ष॑माणाय॒ द्विप॑ञ्चा॒शत्॥५२॥}{प्र॒जन॑नं॒ परि॑ज्मा॥}%%७-१
{हरिः॑ ॐ}{॥कृष्ण-यजुर्वेदीय-तैत्तिरीय-संहितायां सप्तमकाण्डे प्रथमः प्रश्नः समाप्तः॥७-१॥}
%%% END PRASHNA
