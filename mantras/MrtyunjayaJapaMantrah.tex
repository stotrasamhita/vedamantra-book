% !TeX program = XeLaTeX
% !TeX root = ../vedamantrabook.tex
\let\oldparskip\parskip
\parskip=0ex
\chapt{मृत्युञ्जयहोम-मन्त्राः}
%१
अपै॑तु मृ॒त्युर॒मृतं॑ न॒ आग॑न्वैवस्व॒तो नो॒ अभ॑यं कृणोतु।
प॒र्णं वन॒स्पते॑रिवा॒भिनः॑ शीयताꣳ र॒यिः सच॑तां नः॒ शची॒पतिः॑॥१॥

%२
परं॑ मृत्यो॒ अनु॒ परे॑हि॒ पन्थां॒ यस्ते॒ स्व इत॑रो देव॒याना᳚त्।
चक्षु॑ष्मते शृण्व॒ते ते᳚ ब्रवीमि॒ मा नः॑ प्र॒जाꣳ री॑रिषो॒ मोत वी॒रान्॥२॥

%३
वातं॑ प्रा॒णं मन॑सा॒ऽन्वा र॑भामहे प्र॒जाप॑तिं॒ यो भुव॑नस्य गो॒पाः।
स नो॑ मृ॒त्योस्त्रा॑यतां॒ पात्वꣳह॑सो॒ ज्योग्जी॒वा ज॒राम॑शीमहि॥३॥

%४
अ॒मु॒त्र॒ भूया॒दध॒ यद्य॒मस्य॒ बृह॑स्पते अ॒भिश॑स्ते॒रमु॑ञ्चः।
प्रत्यौ॑हताम॒श्विना॑ मृ॒त्युम॑स्माद्दे॒वाना॑मग्ने भि॒षजा॒ शची॑भिः॥४॥

%५
हरि॒ꣳ॒ हर॑न्त॒मनु॑यन्ति दे॒वा विश्व॒स्येशा॑नं वृष॒भं म॑ती॒नाम्।
ब्रह्म॒ सरू॑प॒मनु॑मे॒दमागा॒दय॑नं॒ मा विव॑धी॒र्विक्र॑मस्व॥५॥

%६
शल्कै॑र॒ग्निमि॑न्धा॒न उ॒भौ लो॒कौ स॑नेम॒हम्।
उ॒भयो᳚र्लो॒कयोर्॑-ऋ॒ध्वाऽति॑ मृ॒त्युं त॑राम्य॒हम्॥६॥

%७
मा छि॑दो मृत्यो॒ मा व॑धी॒र्मा मे॒ बलं॒ विवृ॑हो॒ मा प्रमो॑षीः।
प्र॒जां मा मे॑ रीरिष॒ आयु॑रुग्र नृ॒चक्ष॑सं त्वा ह॒विषा॑ विधेम॥७॥

%८
मा नो॑ म॒हान्त॑मु॒त मा नो॑ अर्भ॒कं मा न॒ उक्ष॑न्तमु॒त मा न॑ उक्षि॒तम्।
मा नो॑ वधीः पि॒तरं॒ मोत मा॒तरं॑ प्रि॒या मा न॑स्त॒नुवो॑ रुद्र रीरिषः॥८॥

%९
मा न॑स्तो॒के तन॑ये॒ मा न॒ आयु॑षि॒ मा नो॒ गोषु॒ मा नो॒ अश्वे॑षु रीरिषः।
वी॒रान्मा नो॑ रुद्र भामि॒तो व॑धीर्‌ह॒विष्म॑न्तो॒ नम॑सा विधेम ते॥९॥

%१०
प्रजा॑पते॒ न त्वदे॒तान्य॒न्यो विश्वा॑ जा॒तानि॒ परि॒ता ब॑भूव।
यत्का॑मास्ते जुहु॒मस्तन्नो॑ अस्तु व॒यꣴ स्या॑म॒ पत॑यो रयी॒णाम्॥१०॥

%११
यत॑ इन्द्र॒ भया॑महे॒ ततो॑ नो॒ अभ॑यं कृधि।
मघ॑वन्छ॒ग्धि तव॒ तन्न॑ ऊ॒तये॒ विद्विषो॒ विमृधो॑ जहि॥११॥

%१२
स्व॒स्ति॒दा वि॒शस्पति॑र्वृत्र॒हा विमृधो॑ व॒शी।
वृषेन्द्रः॑ पु॒र ए॑तु नः स्वस्ति॒दा अ॑भयङ्क॒रः॥१२॥

%१३
त्र्य॑म्बकं यजामहे सुग॒न्धिं पु॑ष्टि॒वर्ध॑नम्।
 उ॒र्वा॒रु॒कमि॑व॒ बन्ध॑नान्मृ॒त्योर्मु॑क्षीय॒ माऽमृता᳚त्॥१३॥

%१४
अप॑मृ॒त्युमप॒क्षुधम्᳚।   अपे॒तः श॒पथं॑ जहि।
अधा॑ नो अग्न॒ आव॑ह।   रा॒यस्पोषꣳ॑ सह॒स्रिणम्᳚॥१४॥

%१५
ये ते॑ स॒हस्र॑म॒युतं॒ पाशाः᳚।   मृत्यो॒ मर्त्या॑य॒ हन्त॑वे।
तान् य॒ज्ञस्य॑ मा॒यया᳚।   सर्वा॒नव॑यजामहे॥१५॥

%१६
जा॒तवे॑दसे सुनवाम॒ सोम॑ मरातीय॒तो निद॑हाति॒ वेदः॑।
स नः॑ पर्‌ष॒दति॑ दु॒र्गाणि॒ विश्वा॑ ना॒वेव॒ सिन्धुं॑ दुरि॒ताऽत्य॒ग्निः॥१६॥

%१७
भूर्भुवः॒ स्वः॑।
ओजो॒ बलम्᳚।
ब्रह्म॑ क्ष॒त्रम्।
यशो॑ म॒हत्।
स॒त्यं तपो॒ नाम॑।
रू॒पम॒मृतम्᳚।
चक्षुः॒ श्रोत्रम्᳚।
मन॒ आयुः॑।
विश्वं॒ यशो॑ म॒हः।
स॒मं तपो॒ हरो॒ भाः।
जा॒तवे॑दा॒ यदि॑ वा पाव॒कोऽसि॑।
वै॒श्वा॒न॒रो यदि॑ वा वैद्यु॒तोऽसि॑।
शं प्र॒जाभ्यो॒ यज॑मानाय लो॒कम्।
ऊर्जं॒ पुष्टिं॒ दद॑द॒भ्याव॑वृथ्स्व॥१७॥
   
मृ॒त्युर्न॑श्य॒त्वायु॑र्वर्धतां॒ भूः॥१८॥

मृ॒त्युर्न॑श्य॒त्वायु॑र्वर्धतां॒ भुवः॑॥१९॥

मृ॒त्युर्न॑श्य॒त्वायु॑र्वर्धता॒ꣳ॒ सुवः॑॥२०॥

मृ॒त्युर्न॑श्य॒त्वायु॑र्वर्धतां॒ भूर्भुवः॒ सुवः॑॥२१॥ मृ॒त्युर्न॑श्य॒त्वायु॑र्वर्धताम्॥

\centerline{॥ॐ शान्तिः॒ शान्तिः॒ शान्तिः॑॥}
\let\parskip\oldparskip

\closesection
\centerline{\scriptsize (तैत्तिरीयारण्यकम् – ४/प्रपाठकः – ३/अनुवाकः – १५)}
%१५.१
हरि॒ꣳ॒ हर॑न्त॒मनु॑यन्ति दे॒वाः। विश्व॒स्येशा॑नं वृष॒भं म॑ती॒नाम्। ब्रह्म॒ सरू॑प॒मनु॑मे॒दमागा᳚त्। अय॑नं॒ मा विव॑धी॒र्विक्र॑मस्व। मा छि॑दो मृत्यो॒ मा व॑धीः। मा मे॒ बलं॒  विवृ॑हो॒ मा प्रमो॑षीः। प्र॒जां मा मे॑ रीरिष॒ आयु॑रुग्र। नृ॒चक्ष॑सं त्वा ह॒विषा॑ विधेम। स॒द्यश्च॑कमा॒नाय॑। प्र॒वे॒पा॒नाय॑ मृ॒त्यवे᳚॥१॥

%१५.२
प्रास्मा॒ आशा॑ अशृण्वन्। कामे॑नाजनय॒न्पुनः॑। कामे॑न मे॒ काम॒ आगा᳚त्। हृद॑या॒द्धृद॑यं मृ॒त्योः। यद॒मीषा॑म॒दः प्रि॒यम्। तदैतूप॒माम॒भि। परं॑ मृत्यो॒ अनु॒ परे॑हि॒ पन्था᳚म्। यस्ते॒ स्व इत॑रो देव॒याना᳚त्। चक्षु॑ष्मते शृण्व॒ते ते᳚ ब्रवीमि। मा नः॑ प्र॒जाꣳ री॑रिषो॒ मोत वी॒रान्। प्र पू॒र्व्यं मन॑सा॒ वन्द॑मानः। नाध॑मानो वृष॒भं च॑र्\mbox{}षणी॒नाम्। यः प्र॒जाना॑मेक॒राण्मानु॑षीणाम्। मृ॒त्युं य॑जे प्रथम॒जामृ॒तस्य॑॥२॥