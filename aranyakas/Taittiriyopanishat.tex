% !TeX program = XeLaTeX
% !TeX root = ../AraNyakabook-kindle.tex
\sect{सप्तमः प्रश्नः --- शीक्षावल्ली}\setcounter{anuvakam}{0}
\label{sec:start_taittiriyopanishat}
%\setmainfont[Script=Devanagari]{Siddhanta}
%॥ तैत्तिरीयीरण्यके पञ्चमप्रश्नः प्रारम्भः॥ तैत्तिरीयोपनिषत्॥
%शं नः॒ शीक्षाꣳ स॒ह नौ॒ यश्छन्द॑सां॒ भूः स यः पृथिव्योमित्यृतं चा॒हं वेदमनूच्य॒ शं नः॑ स॒ह ना॑ववतु॒ ब्रह्म॒विद्भृगुः॒ पञ्च॑दश॥ १५॥ शं नो॒ मह॒ इति॒ ये तत्र॑ भी॒षाऽस्मा॒दन्नं॑ ब॒हु कु॑र्वीत॒ द्विच॑त्वारिꣳशत्॥ ४२॥ शं न॒ इत्यु॑प॒निष॑त्॥ ॐ शान्तिः॒ शान्तिः॒ शान्तिः॑॥ हरिः॑ ओम्। 

ॐ। शं नो॑ मि॒त्रः शं वरु॑णः। 
शं नो॑ भवत्वर्य॒मा। 
शं न॒ इन्द्रो॒ बृह॒स्पतिः॑। 
शं नो॒ विष्णु॑रुरुक्र॒मः। 
नमो॒ ब्रह्म॑णे। 
नम॑स्ते वायो। 
त्वमे॒व प्र॒त्यक्षं॒ ब्रह्मा॑सि। 
त्वमे॒व प्र॒त्यक्षं॒ ब्रह्म॑ वदिष्यामि। 
ऋ॒तं व॑दिष्यामि। 
स॒त्यं व॑दिष्यामि। 
तन्माम॑वतु। 
तद्व॒क्तार॑मवतु। 
अव॑तु॒ माम्। 
अव॑तु व॒क्तारम्᳚। 
ॐ शान्तिः॒ शान्तिः॒ शान्तिः॑॥१॥%
\anuvakamend[स॒त्यं व॑दिष्यामि॒ पञ्च॑ च]

शीक्षां व्या᳚ख्यास्या॒मः। 
वर्णः॒ स्वरः। 
मात्रा॒ बलम्। 
साम॑ सन्ता॒नः। 
इत्युक्तः शी᳚क्षाध्या॒यः॥२॥
\anuvakamend[शीक्षां पञ्च॑]

स॒ह नौ॒ यशः। 
स॒ह नौ ब्र॑ह्मव॒र्चसम्। 
अथातः सꣳहिताया उप\-निषदं व्या᳚ख्यास्या॒मः। 
पञ्चस्वधिक॑रणे॒षु। 
अधिलोकमधि\-ज्यौतिषमधि\-विद्यमधि\-प्रज॑मध्या॒त्मम्। 
ता महा\-सꣳहिता इ॑त्या\-च॒क्षते। 
अथा॑धिलो॒कम्। 
पृथिवी पू᳚र्व\-रू॒पम्। 
द्यौरुत्त॑र\-रू॒पम्। 
आका॑शः स॒न्धिः॥३॥

वायुः॑ सन्धा॒नम्। 
इत्य॑धि\-लो॒कम्। 
अथा॑धिज्यौ॒तिषम्। 
अग्निः पू᳚र्व\-रू॒पम्। 
आदित्य उत्त॑र\-रू॒पम्। 
आ॑पः स॒न्धिः। 
वैद्युतः॑ सन्धा॒नम्। 
इत्य॑धि\-ज्यौ॒तिषम्। 
अथा॑धिवि॒द्यम्। 
आचार्यः पू᳚र्वरू॒पम्॥४॥

अन्तेवास्युत्त॑र\-रू॒पम्। 
वि॑द्या स॒न्धिः। 
प्रवचनꣳ॑ सन्धा॒नम्। 
इत्य॑धि\-वि॒द्यम्। 
अथाधि॒प्रजम्। 
माता पू᳚र्व\-रू॒पम्। 
पितोत्त॑र\-रू॒पम्। 
प्र॑जा स॒न्धिः। 
प्रजननꣳ॑ सन्धा॒नम्। 
इत्यधि॒प्रजम्॥५॥

अथाध्या॒त्मम्। 
अधराहनुः पू᳚र्व\-रू॒पम्। 
उत्तराहनुरुत्त॑र\-रू॒पम्। 
वाख्स॒न्धिः। 
जिह्वा॑ सन्धा॒नम्। 
इत्यध्या॒त्मम्। 
इतीमा म॑हा\-स॒ꣳ॒हिताः। 
य एवमेता महा\-सꣳहिता व्याख्या॑ता वे॒द। 
सन्धीयते प्रज॑या प॒शुभिः। 
ब्रह्मवर्चसेनान्नाद्येन सुवर्ग्येण॑ लोके॒न॥६॥
\anuvakamend[स॒न्धिराचार्यः पू᳚र्वरू॒पमित्यधि॒प्रजं लो॑के॒न]

यश्छन्द॑सामृष॒भो वि॒श्वरू॑पः। 
छन्दो॒भ्योऽध्य॒मृता᳚थ्सम्ब॒भूव॑। 
स मेन्द्रो॑ मे॒धया᳚ स्पृणोतु। 
अ॒मृत॑स्य देव॒ धार॑णो भूयासम्। 
शरी॑रं मे॒ विच॑र्\mbox{}षणम्। 
जि॒ह्वा मे॒ मधु॑मत्तमा। 
कर्णा᳚भ्यां॒ भूरि॒ विश्रु॑वम्। 
ब्रह्म॑णः को॒शो॑ऽसि मे॒धयाऽपि॑हितः। 
श्रु॒तं मे॑ गोपाय। 
आ॒वह॑न्ती वितन्वा॒ना॥७॥

कु॒र्वा॒णा चीर॑मा॒त्मनः॑। 
वासाꣳ॑सि॒ मम॒ गाव॑श्च। 
अ॒न्न॒पा॒ने च॑ सर्व॒दा। 
ततो॑ मे॒ श्रिय॒माव॑ह। 
लो॒म॒शां प॒शुभिः॑ स॒ह स्वाहा᳚। 
आ मा॑ यन्तु ब्रह्मचा॒रिणः॒ स्वाहा᳚। 
वि मा॑ऽऽयन्तु ब्रह्मचा॒रिणः॒ स्वाहा᳚। 
प्र मा॑ऽऽयन्तु ब्रह्मचा॒रिणः॒ स्वाहा᳚। 
दमा॑ऽऽयन्तु ब्रह्मचा॒रिणः॒ स्वाहा᳚। 
शमा॑ऽऽयन्तु ब्रह्मचा॒रिणः॒ स्वाहा᳚॥८॥

यशो॒ जने॑ऽसानि॒ स्वाहा᳚। 
श्रेया॒न्॒ वस्य॑सोऽसानि॒ स्वाहा᳚। 
तं त्वा॑ भग॒ प्रवि॑शानि॒ स्वाहा᳚। 
स मा॑ भग॒ प्रवि॑श॒ स्वाहा᳚। 
तस्मि᳚न्थ्स॒हस्र॑शाखे। 
निभ॑गा॒हं त्वयि॑ मृजे॒ स्वाहा᳚। 
यथाऽऽपः॒ प्रव॑ता॒ यन्ति॑। 
यथा॒ मासा॑ अहर्ज॒रम्। 
ए॒वं मां ब्र॑ह्मचा॒रिणः॑। 
धात॒राय॑न्तु स॒र्वतः॒ स्वाहा᳚। 
प्र॒ति॒वे॒शो॑ऽसि॒ प्र मा॑ भाहि॒ प्र मा॑ पद्यस्व॥९॥
\anuvakamend[वि॒त॒न्वा॒ना शमा॑ऽऽयन्तु ब्रह्मचा॒रिणः॒ स्वाहा॒ धात॒राय॑न्तु स॒र्वतः॒ स्वाहैकं॑ च]


भूर्भुवः॒ सुव॒रिति॒ वा ए॒तास्ति॒स्रो व्याहृ॑तयः। 
तासा॑मुहस्मै॒ तां च॑तु॒र्थीम्। 
माहा॑चमस्यः॒ प्रवे॑दयते। 
मह॒ इति॑। 
तद्ब्रह्म॑। 
स आ॒त्मा। 
अङ्गा᳚न्य॒न्या दे॒वताः᳚। 
भूरिति॒ वा अ॒यं लो॒कः। 
भुव॒ इत्य॒न्तरि॑क्षम्। 
सुव॒रित्य॒सौ लो॒कः॥१०॥

मह॒ इत्या॑दि॒त्यः। 
आ॒दि॒त्येन॒ वाव सर्वे॑ लो॒का मही॑यन्ते। 
भूरिति॒ वा अ॒ग्निः। 
भुव॒ इति॑ वा॒युः। 
सुव॒रित्या॑दि॒त्यः। 
मह॒ इति॑ च॒न्द्रमाः᳚। 
च॒न्द्रम॑सा॒ वाव सर्वा॑णि॒ ज्योतीꣳ॑षि॒ मही॑यन्ते। 
भूरिति॒ वा ऋचः॑। 
भुव॒ इति॒ सामा॑नि। 
सुव॒रिति॒ यजूꣳ॑षि॥११॥

मह॒ इति॒ ब्रह्म॑। 
ब्रह्म॑णा॒ वाव सर्वे॑ वे॒दा मही॑यन्ते। 
भूरिति॒ वै प्रा॒णः। 
भुव॒ इत्य॑पा॒नः। 
सुव॒रिति॑ व्या॒नः। 
मह॒ इत्यन्नम्᳚। 
अन्ने॑न॒ वाव सर्वे᳚ प्रा॒णा मही॑यन्ते। 
ता वा ए॒ताश्चत॑स्रश्चतु॒र्धा। 
चत॑स्रश्चतस्रो॒ व्याहृ॑तयः। 
ता यो वेद॑। 
स वे॑द॒ ब्रह्म॑। 
सर्वे᳚ऽस्मै दे॒वा ब॒लिमाव॑हन्ति॥१२॥
\anuvakamend[अ॒सौ लो॒को यजूꣳ॑षि॒ वेद॒ द्वे च॑]

स य ए॒षो᳚ऽन्तर्\mbox{}हृ॑दय आका॒शः। 
तस्मि॑न्न॒यं पुरु॑षो मनो॒मयः॑। 
अमृ॑तो हिर॒ण्मयः॑। 
अन्त॑रेण॒ तालु॑के। 
य ए॒ष स्तन॑ इवाव॒\-लम्ब॑ते। 
से᳚न्द्रयो॒निः। 
यत्रा॒सौ के॑शा॒न्तो वि॒वर्त॑ते। 
व्य॒पोह्य॑ शीर्\mbox{}षकपा॒ले। 
भूरित्य॒ग्नौ प्रति॑\-तिष्ठति। 
भुव॒ इति॑ वा॒यौ॥१३॥

सुव॒रित्या॑दि॒त्ये। 
मह॒ इति॒ ब्रह्म॑णि। 
आ॒प्नोति॒ स्वारा᳚ज्यम्। 
आ॒प्नोति॒ मन॑स॒स्पतिम्᳚। 
वाक्प॑ति॒श्चक्षु॑ष्पतिः। 
श्रोत्र॑पतिर्वि॒\-ज्ञान॑पतिः। 
ए॒तत्ततो॑ भवति। 
आ॒का॒शश॑रीरं॒ ब्रह्म॑। 
स॒त्यात्म॑प्रा॒णारा॑मं॒ मन॑ आनन्दम्। 
शान्ति॑समृद्धम॒मृतम्᳚। 
इति॑ प्राचीनयो॒ग्योपा᳚स्व॥१४॥
\anuvakamend[वा॒याव॒मृत॒मेकं॑ च]

पृ॒थि॒व्य॑न्तरि॑क्षं॒ द्यौर्दिशो॑ऽवान्तरदि॒शाः। 
अ॒ग्निर्वा॒युरा॑दि॒त्य\-श्च॒न्द्रमा॒ नक्ष॑त्राणि। 
आप॒ ओष॑धयो॒ वन॒स्पत॑य आका॒श आ॒त्मा। 
इत्य॑धि\-भू॒तम्। 
अथाध्या॒त्मम्। 
प्रा॒णो व्या॒नो॑ऽपा॒न उ॑दा॒नः स॑मा॒नः। 
चक्षुः॒ श्रोत्रं॒ मनो॒ वाक्त्वक्। 
चर्म॑ मा॒ꣳ॒सꣴ स्नावास्थि॑ म॒ज्जा। 
ए॒तद॑धि वि॒धाय॒\-र्\mbox{}षि॒\-रवो॑चत्। 
पाङ्क्तं॒ वा इ॒दꣳ सर्वम्᳚। 
पाङ्क्ते॑नै॒व पाङ्क्तꣴ॑ स्पृणो॒तीति॑॥१५॥
\anuvakamend[सर्व॒मेकं॑ च]

ओमिति॒ ब्रह्म॑। 
ओमिती॒दꣳ सर्वम्᳚। 
ओमित्ये॒तद॑नुकृति ह स्म॒ वा अ॒प्योश्रा॑व॒येत्या\-श्रा॑व\-यन्ति। 
ओमिति॒ सामा॑नि गायन्ति। 
ओꣳ\-शोमिति॑ श॒स्त्राणि॑ शꣳसन्ति। 
ओमित्य॑ध्व॒र्युः प्र॑तिग॒रं प्रति॑\-गृणाति। 
ओमिति॒ ब्रह्मा॒ प्रसौ॑ति। 
ओमित्य॑ग्निहो॒त्रमनु॑जानाति। 
ओमिति॑ ब्राह्म॒णः प्र॑व॒क्ष्यन्ना॑ह॒ ब्रह्मोपा᳚प्नवा॒नीति॑। 
ब्रह्मै॒वो\-पा᳚प्नोति॥१६॥%
\anuvakamend[ओन्दश॑]

ऋतं च स्वाध्याय\-प्रव॑चने॒ च। 
सत्यं च स्वाध्याय\-प्रव॑चने॒ च। 
तपश्च स्वाध्याय\-प्रव॑चने॒ च। 
दमश्च स्वाध्याय\-प्रव॑चने॒ च। 
शमश्च स्वाध्याय\-प्रव॑चने॒ च। 
अग्नयश्च स्वाध्याय\-प्रव॑चने॒ च। 
अग्निहोत्रं च स्वाध्याय\-प्रव॑चने॒ च। 
अतिथयश्च स्वाध्याय\-प्रव॑चने॒ च। 
मानुषं च स्वाध्याय\-प्रव॑चने॒ च। 
प्रजा च स्वाध्याय\-प्रव॑चने॒ च। 
प्रजनश्च स्वाध्याय\-प्रव॑चने॒ च। 
प्रजातिश्च स्वाध्याय\-प्रव॑चने॒ च। 
सत्यमिति सत्य\-वचा॑ राथी॒तरः। 
तप इति तपो\-नित्यः पौ॑रुशि॒ष्टिः। 
स्वाध्याय\-प्रवचने एवेति नाको॑ मौद्ग॒ल्यः। 
तद्धि तप॑स्तद्धि॒ तपः॥१७॥
\anuvakamend[प्रजा च स्वाध्याय\-प्रव॑चने॒ च षट्च॑]

अ॒हं वृ॒क्षस्य॒ रेरि॑वा। 
की॒र्तिः पृ॒ष्ठं गि॒रेरि॑व। 
ऊ॒र्ध्वप॑वित्रो वा॒जिनी॑व स्व॒मृत॑मस्मि। 
द्रवि॑ण॒ꣳ॒ सव॑र्चसम्। 
सुमेधा अ॑मृतो॒क्षितः। 
इति त्रिशङ्कोर्वेदा॑नुव॒चनम्॥१८॥
\anuvakamend[अ॒हꣳ षट्]

वेदमनूच्याऽऽचार्योऽन्तेवासिनम॑नुशा॒स्ति। 
सत्यं॒ वद। 
धर्मं॒ चर। 
स्वाध्याया᳚न्मा प्र॒मदः। 
आचार्याय प्रियं धनमाहृत्य प्रजातन्तुं मा व्य॑वच्छे॒थ्सीः। 
सत्यान्न प्रम॑दित॒व्यम्। 
धर्मान्न प्रम॑दित॒व्यम्। 
कुशलान्न प्रम॑दित॒व्यम्। 
भूत्यै न प्रम॑दित॒व्यम्। 
स्वाध्याय\-प्रवचनाभ्यां न प्रम॑दित॒व्यम्॥१९॥

देवपितृकार्याभ्यां न प्रम॑दित॒व्यम्। 
मातृ॑देवो॒ भव। 
पितृ॑देवो॒ भव। 
आचार्य॑देवो॒ भव। 
अतिथि॑देवो॒ भव। 
यान्यनवद्यानि॑ कर्मा॒णि। 
तानि सेवि॑तव्या॒नि। 
नो इ॑तरा॒णि। 
यान्यस्माकꣳ सुच॑रिता॒नि। 
तानि त्वयो॑पास्या॒नि॥२०॥

नो इ॑तरा॒णि। 
ये के चास्मच्छ्रेयाꣳ॑सो ब्रा॒ह्मणाः। 
तेषां त्वयाऽऽसनेन प्रश्व॑सित॒व्यम्। 
श्रद्ध॑या दे॒यम्। 
अश्रद्ध॑याऽदे॒यम्। 
श्रि॑या दे॒यम्। 
ह्रि॑या दे॒यम्। 
भि॑या दे॒यम्। 
संवि॑दा दे॒यम्। 
अथ यदि ते कर्मविचिकिथ्सा वा वृत्तविचिकि॑थ्सा वा॒ स्यात्॥२१॥

ये तत्र ब्राह्मणाः᳚ सम्म॒र्॒\mbox{}शिनः। 
युक्ता॑ आयु॒क्ताः। 
अलूक्षा॑ धर्म॑कामाः॒ स्युः। 
यथा ते॑ तत्र॑ वर्ते॒रन्। 
तथा तत्र॑ वर्ते॒थाः। 
अथाभ्या᳚ख्याते॒षु। 
ये तत्र ब्राह्मणाः᳚ सम्म॒र्॒\mbox{}शिनः। 
युक्ता॑ आयु॒क्ताः। 
अलूक्षा॑ धर्म॑कामाः॒ स्युः। 
यथा ते॑ तेषु॑ वर्ते॒रन्। 
तथा तेषु॑ वर्ते॒थाः। 
एष॑ आदे॒शः। 
एष उ॑पदे॒शः। 
एषा वे॑दोप॒निषत्। 
एतद॑नुशा॒सनम्। 
एवमुपा॑सित॒व्यम्। 
एवमु चैत॑दुपा॒स्यम्॥२२॥
\anuvakamend[स्वाध्यायप्रवचनाभ्यान्न प्रम॑दित॒व्यं तानि त्वयो॑पास्या॒नि स्यात्तेषु॑ वर्ते॒रन्थ्स॒प्त च॑]

ॐ। शं नो॑ मि॒त्रः शं वरु॑णः। 
शं नो॑ भवत्वर्य॒मा। 
शं न॒ इन्द्रो॒ बृह॒स्पतिः॑। 
शं नो॒ विष्णु॑रुरुक्र॒मः। 
नमो॒ ब्रह्म॑णे। 
नम॑स्ते वायो। 
त्वमे॒व प्र॒त्यक्षं॒ ब्रह्मा॑सि। 
त्वामे॒व प्र॒त्यक्षं॒ ब्रह्मावा॑दिषम्। 
ऋ॒तम॑वादिषम्। 
स॒त्यम॑वादिषम्। 
तन्मामा॑वीत्। 
तद्व॒क्तार॑मावीत्। 
आवी॒न्माम्। 
आवी᳚द्व॒क्तारम्᳚। 
ॐ शान्तिः॒ शान्तिः॒ शान्तिः॑॥२३॥
\anuvakamend[स॒त्यम॑वादिषं॒ पञ्च॑ च]

\clearpage

\sect{अष्टमः प्रश्नः --- ब्रह्मानन्दवल्ली}\setcounter{anuvakam}{0}
ॐ। स॒ह ना॑ववतु। 
स॒ह नौ॑ भुनक्तु। 
स॒ह वी॒र्यं॑ करवावहै। 
ते॒ज॒स्वि ना॒वधी॑तमस्तु॒ मा वि॑द्विषा॒वहै᳚। 
ॐ शान्तिः॒ शान्तिः॒ शान्तिः॑॥%२३॥
%\anuvakamend%[स॒ह ना॑ववतु॒ पञ्च॑]

%(ब्र॒ह्म॒विदिदमयमिदमेक॑विꣳशतिः। 
% अन्ना॒दन्न॑रस॒मयात् प्राणो॒ व्यानोऽपान आका॑शः॒ पृथिवी पुच्छ॒ꣳ॒ षड्विꣳ॑शतिः। 
% प्रा॒णं प्रा॑ण॒मयान्मनो॒ यजु॒र्॒‌ऋख्सामादे॒शोऽथर्वाङ्गिरसः पुच्छ॒न्द्वाविꣳ॑शतिः। 
% यतः॑ श्र॒द्धर्त्तꣳ सत्यं यो॑गो॒ महो᳚ऽष्टाद॑श। 
% वि॒ज्ञानं॒ प्रियं॒ मोदः प्रमोद आन॑न्दो॒ ब्रह्म पुच्छ॒न्द्वाविꣳ॑शतिः। 
% अस॑न्ने॒वाथाष्टाविꣳ॑शतिः। 
% अस॒थ्षोड॑श। 
% भी॒षाऽस्मा॒न्मानुषो॒ मनुष्यगन्धर्वाणां॒ देवगन्धर्वाणां॒ पितृणां चिरलोकलोकाना॒माजानजानां कर्मदेवानां ये कर्मणा देवाना॒मिन्द्र॑स्य॒ बृहस्पतेः॒ प्रजापते॒र्ब्रह्म॑णः॒ स यश्च॑ सङ्क्रा॒मत्येक॑पञ्चा॒शत्। 
% यतः॒ कुत॑श्च॒नैतमेका॑दश॒ नव॑॥ ९॥ ॥

ब्र॒ह्म॒विदा᳚प्नोति॒ परम्᳚। 
तदे॒षाभ्यु॑क्ता। 
स॒त्यं ज्ञा॒नम॑\-न॒न्तं ब्रह्म॑। 
यो वेद॒ निहि॑तं॒ गुहा॑यां पर॒मे व्यो॑मन्। 
सो᳚ऽश्ञुते॒ सर्वा॒न्कामा᳚न्थ्स॒ह। 
ब्रह्म॑णा वि\-प॒श्चितेति॑। 
तस्मा॒द्वा ए॒तस्मा॑\-दा॒त्मन॑ आका॒शः सम्भू॑तः। 
आ॒का॒शाद्वा॒युः। 
वा॒योर॒ग्निः। 
अ॒ग्नेरापः॑। 
अ॒द्भ्यः पृ॑थि॒वी। 
पृ॒थि॒व्या ओष॑धयः। 
ओष॑धी॒भ्यो\-ऽ\-न्नम्᳚। 
अन्ना॒त्पुरु॑षः। 
स वा एष पुरुषो\-ऽन्न॑\-रस॒मयः। 
तस्येद॑\-मेव॒ शिरः। 
अयं दक्षि॑णः प॒क्षः। 
अयमुत्त॑रः प॒क्षः। 
अयमात्मा᳚। 
इदं पुच्छं॑ प्रति\-॒ष्ठा। 
तदप्येष श्लो॑को भ॒वति॥१॥

अन्ना॒द्वै प्र॒जाः प्र॒जाय॑न्ते। 
याः काश्च॑ पृथि॒वीꣴ श्रि॒ताः। 
अथो॒ अन्ने॑नै॒व जी॑वन्ति। 
अथै॑न॒दपि॑ यन्त्यन्त॒तः। 
अन्न॒ꣳ॒ हि भू॒तानां॒ ज्येष्ठम्᳚। 
तस्मा᳚थ्सर्वौष॒धमु॑च्यते। 
सर्वं॒ वै तेऽन्न॑माप्नुवन्ति। 
येऽन्नं॒ ब्रह्मो॒पास॑ते। 
अन्न॒ꣳ॒ हि भू॒तानां॒ ज्येष्ठम्᳚। 
तस्मा᳚थ्सर्वौष॒धमु॑च्यते। 
अन्ना᳚द्भू॒तानि॒ जाय॑न्ते। 
जाता॒न्यन्ने॑न वर्धन्ते। 
अद्यतेऽत्ति च॑ भूता॒नि। 
तस्मादन्नं तदुच्य॑त इ॒ति। 
तस्माद्वा एतस्मादन्न॑रस॒मयात्। 
अन्योऽन्तर आत्मा᳚ प्राण॒मयः। 
तेनै॑ष पू॒र्णः। 
स वा एष पुरुषवि॑ध ए॒व। 
तस्य पुरु॑षवि॒धताम्। 
अन्वयं॑ पुरुष॒विधः। 
तस्य प्राण॑ एव॒ शिरः। 
व्यानो दक्षि॑णः प॒क्षः। 
अपान उत्त॑रः प॒क्षः। 
आका॑श आ॒त्मा। 
पृथिवी पुच्छं॑ प्रति\-॒ष्ठा। 
तदप्येष श्लो॑को भ॒वति॥२॥

प्रा॒णं दे॒वा अनु॒ प्राण॑न्ति। 
म॒नु॒ष्याः᳚ प॒शव॑श्च॒ ये। 
प्रा॒णो हि भू॒ताना॒मायुः॑। 
तस्मा᳚थ्सर्वायु॒षमु॑च्यते। 
सर्व॑मे॒व त॒ आयु॑र्\mbox{}यन्ति। 
ये प्रा॒णं ब्रह्मो॒पास॑ते। 
प्राणो हि भूता॑नामा॒युः। 
तस्माथ्सर्वायुषमुच्य॑त इ॒ति। 
तस्यैष एव शारी॑र आ॒त्मा। 
यः॑ पूर्व॒स्य। 
तस्माद्वा एतस्मा᳚त् प्राण॒मयात्। 
अन्योऽन्तर आत्मा॑ मनो॒मयः। 
तेनै॑ष पू॒र्णः। 
स वा एष पुरुषवि॑ध ए॒व। 
तस्य पुरु॑षवि॒धताम्। 
अन्वयं॑ पुरुष॒विधः। 
तस्य यजु॑रेव॒ शिरः। 
ऋग्दक्षि॑णः प॒क्षः। 
सामोत्त॑रः प॒क्षः। 
आदे॑श आ॒त्मा। 
अथर्वाङ्गिरसः पुच्छं॑ प्रति\-॒ष्ठा। 
तदप्येष श्लो॑को भ॒वति॥३॥

यतो॒ वाचो॒ निव॑र्तन्ते। 
अप्रा᳚प्य॒ मन॑सा स॒ह। 
आनन्दं ब्रह्म॑णो वि॒द्वान्। 
न बिभेति कदा॑चने॒ति। 
तस्यैष एव शारी॑र आ॒त्मा। 
यः॑ पूर्व॒स्य। 
तस्माद्वा एतस्मा᳚न्मनो॒मयात्। 
अन्योऽन्तर आत्मा वि॑ज्ञान॒मयः। 
तेनै॑ष पू॒र्णः। 
स वा एष पुरुषवि॑ध ए॒व। 
तस्य पुरु॑षवि॒धताम्। 
अन्वयं॑ पुरुष॒विधः। 
तस्य श्र॑द्धैव॒ शिरः। 
ऋतं दक्षि॑णः प॒क्षः। 
सत्यमुत्त॑रः प॒क्षः। 
यो॑ग आ॒त्मा। 
महः पुच्छं॑ प्रति\-॒ष्ठा। 
तदप्येष श्लो॑को भ॒वति॥४॥

वि॒ज्ञानं॑ य॒ज्ञं त॑नुते। 
कर्मा॑णि तनु॒तेऽपि॑ च। 
वि॒ज्ञानं॑ दे॒वाः सर्वे᳚। 
ब्रह्म॒ ज्येष्ठ॒मुपा॑सते। 
वि॒ज्ञानं॒ ब्रह्म॒ चेद्वेद॑। 
तस्मा॒च्चेन्न प्र॒माद्य॑ति। 
शरीरे॑ पाप्म॑नो हि॒त्वा। 
सर्वान्कामान्थ्समश्ञु॑त इ॒ति। 
तस्यैष एव शारी॑र आ॒त्मा। 
यः॑ पूर्व॒स्य। 
तस्माद्वा एतस्माद्वि॑ज्ञान॒मयात्। 
अन्योऽन्तर आत्मा॑ऽऽनन्द॒मयः। 
तेनै॑ष पू॒र्णः। 
स वा एष पुरुषवि॑ध ए॒व। 
तस्य पुरु॑षवि॒धताम्। 
अन्वयं॑ पुरुष॒विधः। 
तस्य प्रिय॑मेव॒ शिरः। 
मोदो दक्षि॑णः प॒क्षः। 
प्रमोद उत्त॑रः प॒क्षः। 
आन॑न्द आ॒त्मा। 
ब्रह्म पुच्छं॑ प्रति\-॒ष्ठा। 
तदप्येष श्लो॑को भ॒वति॥५॥

अस॑न्ने॒व स॑ भवति। 
अस॒द्ब्रह्मेति॒ वेद॒ चेत्। 
अस्ति ब्रह्मेति॑ चेद्वे॒द। 
सन्तमेनं ततो वि॑दुरि॒ति। 
तस्यैष एव शारी॑र आ॒त्मा। 
यः॑ पूर्व॒स्य। 
अथातो॑ऽनुप्र॒श्ञाः। 
उ॒ता वि॒द्वान॒मुं लो॒कं प्रेत्य॑। 
कश्च॒न ग॑च्छ॒ती(३)॥ आहो॑ वि॒द्वान॒मुं लो॒कं प्रेत्य॑। 
कश्चि॒थ्सम॑श्ञु॒ता(३) उ॒। 
सो॑ऽकामयत। 
ब॒हु स्यां॒ प्रजा॑ये॒येति॑। 
स तपो॑ऽतप्यत। 
स तप॑स्त॒प्त्वा। 
इ॒दꣳ सर्व॑मसृजत। 
यदि॒दं किं च॑। 
तथ्सृ॒ष्ट्वा। 
तदे॒वानु॒ प्रावि॑शत्। 
तद॑नुप्र॒विश्य॑। 
सच्च॒ त्यच्चा॑भवत्। 
नि॒रुक्तं॒ चानि॑रुक्तं च। 
नि॒लय॑नं॒ चानि॑लयनं च। 
वि॒ज्ञानं॒ चावि॑ज्ञानं च। 
सत्यं चानृतं च स॑त्यम॒भवत्। 
यदि॑दं किं॒ च। 
तथ्सत्यमि॑त्याच॒क्षते। 
तदप्येष श्लो॑को भ॒वति॥६॥

अस॒द्वा इ॒दमग्र॑ आसीत्। 
ततो॒ वै सद॑जायत। 
तदात्मानꣴ स्वय॑मकु॒रुत। 
तस्मात्तथ्सुकृतमुच्य॑त इ॒ति। 
यद्वै॑ तथ्सु॒कृतम्। 
र॑सो वै॒ सः। 
रसꣴ ह्येवायं लब्ध्वाऽऽन॑न्दी भ॒वति। 
को ह्येवान्या᳚त्कः प्रा॒ण्यात्। 
यदेष आकाश आन॑न्दो न॒ स्यात्। 
एष ह्येवान॑न्दया॒ति। 
य॒दा ह्ये॑वैष॒ एतस्मिन्नदृश्येऽनात्म्येऽनिरुक्तेऽनिलयनेऽभयं प्रति॑\-ष्ठां वि॒न्दते। 
अथ सोऽभयं ग॑तो भ॒वति। 
य॒दा ह्ये॑वैष॒ एतस्मिन्नुदरमन्त॑रं कु॒रुते। 
अथ तस्य भ॑यं भ॒वति। 
तत्त्वेव भयं विदुषोऽम॑न्वान॒स्य। 
तदप्येष श्लो॑को भ॒वति॥७॥

भी॒षाऽस्मा॒द्वातः॑ पवते। 
भी॒षोदे॑ति॒ सूर्यः॑। 
भीषाऽस्मादग्नि॑\-श्चेन्द्र॒श्च। 
मृत्युर्धावति पञ्च॑म इ॒ति। 
सैषाऽऽनन्दस्य मीमाꣳ॑सा भ॒वति। 
युवा स्याथ्साधु यु॑वाऽध्या॒यकः। 
आशिष्ठो दृढिष्ठो॑ बलि॒ष्ठः। 
तस्येयं पृथिवी सर्वा वित्तस्य॑ पूर्णा॒ स्यात्। 
स एको मानुष॑ आन॒न्दः। 
ते ये शतं मानुषा॑ आन॒न्दाः। 

स एको मनुष्यगन्धर्वाणा॑\-मान॒न्दः। 
श्रोत्रियस्य चाकाम॑हत॒स्य। 
ते ये शतं मनुष्यगन्धर्वाणा॑मान॒न्दाः। 

स एको देवगन्धर्वाणा॑\-मान॒न्दः। 
श्रोत्रियस्य चाकाम॑हत॒स्य। 
ते ये शतं देवगन्धर्वाणा॑मान॒न्दाः। 

स एकः पितृणां चिरलोकलोकाना॑\-मान॒न्दः। 
श्रोत्रियस्य चाकाम॑हत॒स्य। 
ते ये शतं पितृणां चिरलोकलोकाना॑\-मान॒न्दाः। 

स एक आजानजानां देवाना॑\-मान॒न्दः। 
श्रोत्रियस्य चाकाम॑हत॒स्य। 
ते ये शतमाजानजानां देवाना॑मान॒न्दाः। 

स एकः कर्मदेवानां देवाना॑\-मान॒न्दः। 
ये कर्मणा देवान॑पिय॒न्ति। 
श्रोत्रियस्य चाकाम॑हत॒स्य। 
ते ये शतं कर्मदेवानां देवाना॑मान॒न्दाः। 

स एको देवाना॑\-मान॒न्दः। 
श्रोत्रियस्य चाकाम॑हत॒स्य। 
ते ये शतं देवाना॑मान॒न्दाः। 

स एक इन्द्र॑स्याऽऽन॒न्दः। 
श्रोत्रियस्य चाकाम॑हत॒स्य। 
ते ये शतमिन्द्र॑स्याऽऽन॒न्दाः। 

स एको बृहस्पते॑रान॒न्दः। 
श्रोत्रियस्य चाकाम॑हत॒स्य। 
ते ये शतं बृहस्पते॑रान॒न्दाः। 

स एकः प्रजापते॑रान॒न्दः। 
श्रोत्रियस्य चाकाम॑हत॒स्य। 
ते ये शतं प्रजापते॑रान॒न्दाः। 

स एको ब्रह्मण॑ आन॒न्दः। 
श्रोत्रियस्य चाकाम॑हत॒स्य। 
स यश्चा॑यं पु॒रुषे। 
यश्चासा॑वादि॒त्ये। 
स एकः॑। 
स य॑ एवं॒वित्। 
अस्माल्लो॑कात्प्रे॒त्य। 
एतमन्न\-मय\-मात्मानमुप॑सङ्क्रा॒मति। 
एतं प्राण\-मय\-मात्मानमुप॑सङ्क्रा॒मति। 
एतं मनो\-मय\-मात्मानमुप॑\-सङ्क्रा॒मति। 
एतं विज्ञान\-मय\-मात्मानमुप॑\-सङ्क्रा॒मति। 
एतमानन्द\-मय\-मात्मानमुप॑\-सङ्क्रा॒मति। 
तदप्येष श्लो॑को भ॒वति॥८॥

यतो॒ वाचो॒ निव॑र्तन्ते। 
अप्रा᳚प्य॒ मन॑सा स॒ह। 
आनन्दं ब्रह्म॑णो वि॒द्वान्। 
न बिभेति कुत॑श्चने॒ति। 
एतꣳ ह वाव॑ न त॒पति। 
किमहꣳ साधु॑ नाक॒रवम्। 
किमहं पापमकर॑वमि॒ति। 
स य एवं विद्वानेते आत्मा॑नꣴ स्पृ॒णुते। 
उ॒भे ह्ये॑वैष॒ एते आत्मा॑नꣴ स्पृ॒णुते। 
य ए॒वं वेद॑। 
इत्यु॑प॒निष॑त्॥९॥

ॐ। स॒ह ना॑ववतु। 
स॒ह नौ॑ भुनक्तु। 
स॒ह वी॒र्यं॑ करवावहै। 
ते॒ज॒स्वि ना॒वधी॑तमस्तु॒ मा वि॑द्विषा॒वहै᳚। 
ॐ शान्तिः॒ शान्तिः॒ शान्तिः॑॥

\sect{नवमः प्रश्नः --- भृगुवल्ली}\setcounter{anuvakam}{0}
ॐ। स॒ह ना॑ववतु। 
स॒ह नौ॑ भुनक्तु। 
स॒ह वी॒र्यं॑ करवावहै। 
ते॒ज॒स्वि ना॒वधी॑तमस्तु॒ मा वि॑द्विषा॒वहै᳚। 
ॐ शान्तिः॒ शान्तिः॒ शान्तिः॑॥


भृगु॒र्वै वा॑रु॒णिः। 
वरु॑णं॒ पित॑र॒मुप॑ससार। 
अधी॑हि भगवो॒ ब्रह्मेति॑। 
तस्मा॑ ए॒तत्प्रो॑वाच। 
अन्नं॑ प्रा॒णं चक्षुः॒ श्रोत्रं॒ मनो॒ वाच॒मिति॑। 
तꣳ हो॑वाच। 
यतो॒ वा इ॒मानि॒ भूता॑नि॒ जाय॑न्ते। 
येन॒ जाता॑नि॒ जीव॑न्ति। 
यत्प्रय॑न्त्य॒भि संवि॑शन्ति। 
तद्विजि॑ज्ञासस्व। 
तद्ब्रह्मेति॑। 
स तपो॑ऽतप्यत। 
स तप॑स्त॒प्त्वा॥१॥

अन्नं॒ ब्रह्मेति॒ व्य॑जानात्। 
अ॒न्नाद्ध्ये॑व खल्वि॒मानि॒ भूता॑नि॒ जाय॑न्ते। 
अन्ने॑न॒ जाता॑नि॒ जीव॑न्ति। 
अन्नं॒ प्रय॑न्त्य॒भि संवि॑श॒न्तीति॑। 
तद्वि॒ज्ञाय॑। 
पुन॑रे॒व वरु॑णं॒ पित॑र॒मुप॑ससार। 
अधी॑हि भगवो॒ ब्रह्मेति॑। 
तꣳ हो॑वाच। 
तप॑सा॒ ब्रह्म॒ विजि॑ज्ञासस्व। 
तपो॒ ब्रह्मेति॑। 
स तपो॑ऽतप्यत। 
स तप॑स्त॒प्त्वा॥२॥

प्रा॒णो ब्र॒ह्मेति॒ व्य॑जानात्। 
प्रा॒णाद्ध्ये॑व खल्वि॒मानि॒ भूता॑नि॒ जाय॑न्ते। 
प्रा॒णेन॒ जाता॑नि॒ जीव॑न्ति। 
प्रा॒णं प्रय॑न्त्य॒भि संवि॑श॒न्तीति॑। 
तद्वि॒ज्ञाय॑। 
पुन॑रे॒व वरु॑णं॒ पित॑र॒मुप॑ससार। 
अधी॑हि भगवो॒ ब्रह्मेति॑। 
तꣳ हो॑वाच। 
तप॑सा॒ ब्रह्म॒ विजि॑ज्ञासस्व। 
तपो॒ ब्रह्मेति॑। 
स तपो॑ऽतप्यत। 
स तप॑स्त॒प्त्वा॥३॥

मनो॒ ब्रह्मेति॒ व्य॑जानात्। 
मन॑सो॒ ह्ये॑व खल्वि॒मानि॒ भूता॑नि॒ जाय॑न्ते। 
मन॑सा॒ जाता॑नि॒ जीव॑न्ति। 
मनः॒ प्रय॑न्त्य॒भि संवि॑श॒न्तीति॑। 
तद्वि॒ज्ञाय॑। 
पुन॑रे॒व वरु॑णं॒ पित॑र॒मुप॑ससार। 
अधी॑हि भगवो॒ ब्रह्मेति॑। 
तꣳ हो॑वाच। 
तप॑सा॒ ब्रह्म॒ विजि॑ज्ञासस्व। 
तपो॒ ब्रह्मेति॑। 
स तपो॑ऽतप्यत। 
स तप॑स्त॒प्त्वा॥४॥

वि॒ज्ञानं॒ ब्रह्मेति॒ व्य॑जानात्। 
वि॒ज्ञाना॒द्ध्ये॑व खल्वि॒मानि॒ भूता॑नि॒ जाय॑न्ते। 
वि॒ज्ञाने॑न॒ जाता॑नि॒ जीव॑न्ति। 
वि॒ज्ञानं॒ प्रय॑न्त्य॒भि संवि॑श॒न्तीति॑। 
तद्वि॒ज्ञाय॑। 
पुन॑रे॒व वरु॑णं॒ पित॑र॒मुप॑ससार। 
अधी॑हि भगवो॒ ब्रह्मेति॑। 
तꣳ हो॑वाच। 
तप॑सा॒ ब्रह्म॒ विजि॑ज्ञासस्व। 
तपो॒ ब्रह्मेति॑। 
स तपो॑ऽतप्यत। 
स तप॑स्त॒प्त्वा॥५॥

आ॒न॒न्दो ब्र॒ह्मेति॒ व्य॑जानात्। 
आ॒नन्दा॒द्ध्ये॑व खल्वि॒मानि॒ भूता॑नि॒ जाय॑न्ते। 
आ॒न॒न्देन॒ जाता॑नि॒ जीव॑न्ति। 
आ॒न॒न्दं प्रय॑न्त्य॒भि संवि॑श॒न्तीति॑। 
सैषा भा᳚र्ग॒वी वा॑रु॒णी वि॒द्या। 
प॒र॒मे व्यो॑म॒न् प्रति॑\-ष्ठिता। 
य ए॒वं वेद॒ प्रति॑\-तिष्ठति। 
अन्न॑वानन्ना॒दो भ॑वति। 
म॒हान्भ॑वति प्र॒जया॑ प॒शुभि॑र्ब्रह्मवर्च॒सेन॑। 
म॒हान्की॒र्त्या॥६॥

अन्नं॒ न नि॑न्द्यात्। 
तद्व्र॒तम्। 
प्रा॒णो वा अन्नम्᳚। 
शरी॑रमन्ना॒दम्। 
प्रा॒णे शरी॑रं॒ प्रति॑\-ष्ठितम्। 
शरी॑रे प्रा॒णः प्रति॑\-ष्ठितः। 
तदे॒तदन्न॒मन्ने॒ प्रति॑\-ष्ठितम्। 
स य ए॒तदन्न॒मन्ने॒ प्रति॑\-ष्ठितं॒ वेद॒ प्रति॑\-तिष्ठति। 
अन्न॑वानन्ना॒दो भ॑वति। 
म॒हान्भ॑वति प्र॒जया॑ प॒शुभि॑र्ब्रह्मवर्च॒सेन॑। 
म॒हान्की॒र्त्या॥७॥

अन्नं॒ न परि॑चक्षीत। 
तद्व्र॒तम्। 
आपो॒ वा अन्नम्᳚। 
ज्योति॑रन्ना॒दम्। 
अ॒फ्सु ज्योतिः॒ प्रति॑\-ष्ठितम्। 
ज्योति॒ष्यापः॒ प्रति॑\-ष्ठिताः। 
तदे॒तदन्न॒मन्ने॒ प्रति॑\-ष्ठितम्। 
स य ए॒तदन्न॒मन्ने॒ प्रति॑\-ष्ठितं॒ वेद॒ प्रति॑\-तिष्ठति। 
अन्न॑वानन्ना॒दो भ॑वति। 
म॒हान्भ॑वति प्र॒जया॑ प॒शुभि॑र्ब्रह्मवर्च॒सेन॑। 
म॒हान्की॒र्त्या॥८॥

अन्नं॑ ब॒हु कु॑र्वीत। 
तद्व्र॒तम्। 
पृ॒थि॒वी वा अन्नम्᳚। 
आ॒का॒शो᳚ऽन्ना॒दः। 
पृ॒थि॒व्यामा॑का॒शः प्रति॑\-ष्ठितः। 
आ॒का॒शे पृ॑थि॒वी प्रति॑\-ष्ठिता। 
तदे॒तदन्न॒मन्ने॒ प्रति॑\-ष्ठितम्। 
स य ए॒तदन्न॒मन्ने॒ प्रति॑\-ष्ठितं॒ वेद॒ प्रति॑\-तिष्ठति। 
अन्न॑वानन्ना॒दो भ॑वति। 
म॒हान्भ॑वति प्र॒जया॑ प॒शुभि॑र्ब्रह्मवर्च॒सेन॑। 
म॒हान्की॒र्त्या॥९॥

न कञ्चन वसतौ प्रत्या॑चक्षी॒त। 
तद्व्र॒तम्। 
तस्माद्यया कया च विधया बह्व॑न्नं प्रा॒प्नुयात्। 
अराध्यस्मा अन्नमि॑त्याच॒क्षते। 
एतद्वै मुखतो᳚\-ऽन्नꣳ रा॒द्धम्। 
मुखतोऽस्मा अ॑न्नꣳ रा॒ध्यते। 
एतद्वै मध्यतो᳚\-ऽन्नꣳ रा॒द्धम्। 
मध्यतोऽस्मा अ॑न्नꣳ रा॒ध्यते। 
एतद्वा अन्ततो᳚\-ऽन्नꣳ रा॒द्धम्। 
अन्ततोऽस्मा अ॑न्नꣳ रा॒ध्यते। 
य ए॑वं वे॒द। 
क्षेम इ॑ति वा॒चि। 
योगक्षेम इति प्रा॑णापा॒नयोः। 
कर्मे॑ति ह॒स्तयोः। 
गतिरि॑ति पा॒दयोः। 
विमुक्तिरि॑ति पा॒यौ। 
इति मानुषीः᳚ समा॒ज्ञाः। 
अथ दै॒वीः। 
तृप्तिरि॑ति वृ॒ष्टौ। 
बलमि॑ति वि॒द्युति। 
यश इ॑ति प॒शुषु। 
ज्योतिरिति न॑क्षत्रे॒षु। 
प्रजातिरमृतमानन्द इ॑त्युप॒स्थे। 
सर्वमि॑त्याका॒शे। 
तत्प्रति\-ष्ठेत्यु॑पासी॒त। 
प्रति\-ष्ठा॑वान्भ॒वति। 
तन्मह इत्यु॑पासी॒त। 
म॑हान्भ॒वति। 
तन्मन इत्यु॑पासी॒त। 
मान॑वान्भ॒वति। 
तन्नम इत्यु॑पासी॒त। 
नम्यन्ते᳚ऽस्मै का॒माः। 
तद्ब्रह्मेत्यु॑पासी॒त। 
ब्रह्म॑वान्भ॒वति। 
तद्ब्रह्मणः परिमर इत्यु॑पासी॒त। 
पर्येणं म्रियन्ते द्विषन्तः॑ सप॒त्नाः। 
परि ये᳚ऽप्रिया᳚ भ्रातृ॒व्याः। 
स यश्चा॑यं पु॒रुषे। 
यश्चासा॑वादि॒त्ये। 
स एकः॑। 
स य॑ एवं॒वित्। 
अस्माल्लो॑कात्प्रे॒त्य। 
एतमन्न\-मय\-मात्मानमुप॑सङ्क्र॒म्य। 
एतं प्राण\-मय\-मात्मानमुप॑सङ्क्र॒म्य। 
एतं मनो\-मय\-मात्मानमुप॑\-सङ्क्र॒म्य। 
एतं विज्ञान\-मय\-मात्मानमुप॑\-सङ्क्र॒म्य। 
एतमानन्द\-मय\-मात्मानमुप॑सङ्क्र॒म्य। 
इमाँल्लोकान्कामान्नी कामरूप्य॑नु\-स॒ञ्चरन्। 
एतथ्साम गा॑यन्ना॒स्ते। 
हा(३) वु॒ हा(३) वु॒ हा(३) वु॑। 
अ॒हमन्नम॒हमन्नम॒हमन्नम्। 
अ॒हमन्ना॒दो(२)\-ऽ॒हमन्ना॒दो(२)\-ऽ॒हमन्ना॒दः। 
अ॒हꣴ श्लोक॒कृद॒हꣴ श्लोक॒कृद॒हꣴ श्लोक॒कृत्। 
अहमस्मि प्रथमजा ऋता(३) स्य॒। 
पूर्वं देवेभ्यो अमृतस्य ना(३) भा॒इ॒। 
यो मा ददाति स इदेव मा(३) वाः॒। 
अ॒हमन्न॒मन्न॑म॒दन्त॒मा(३) द्मि॒। 
अ॒हं विश्वं॒ भुव॑न॒मभ्य॑भ॒वाम्। 
सुव॒र्न ज्योतीः᳚। 
य ए॒वं वेद॑। 
इत्यु॑प॒निष॑त्॥१०॥

ॐ। स॒ह ना॑ववतु। 
स॒ह नौ॑ भुनक्तु। 
स॒ह वी॒र्यं॑ करवावहै। 
ते॒ज॒स्वि ना॒वधी॑तमस्तु॒ मा वि॑द्विषा॒वहै᳚। 
ॐ शान्तिः॒ शान्तिः॒ शान्तिः॑॥


\noindent\hyperref[sec:start_taittiriyopanishat]{\closesection}


