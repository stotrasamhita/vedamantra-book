% !TeX program = XeLaTeX
% !TeX root = ../vedamantrabook.tex
%ꣴꣴ॒ꣴ॑ꣴ᳚ 

\chapt{उदकशान्ति-मन्त्राः}
कृ॒णु॒ष्व पाजः॒ प्रसि॑तिं॒ न पृ॒थ्वीं या॒हि राजे॒वाम॑वा॒ꣳ॒ इभे॑न। तृ॒ष्वीमनु॒ प्रसि॑तिं द्रूणा॒नोऽस्ता॑सि॒ विध्य॑ र॒क्षस॒स्तपि॑ष्ठैः। तव॑ भ्र॒मास॑ आशु॒या प॑त॒न्त्यनु॑ स्पृश धृष॒ता शोशु॑चानः। तपूꣴ॑ष्यग्ने जु॒ह्वा॑ पत॒ङ्गानस॑न्दितो॒ वि सृ॑ज॒ विष्व॑गु॒ल्काः। प्रति॒ स्पशो॒ वि सृ॑ज॒ तूर्णि॑तमो॒ भवा॑ पा॒युर्वि॒शो अ॒स्या अद॑ब्धः। यो नो॑ दू॒रे अ॒घशꣳ॑सो॒  यो अन्त्यग्ने॒ माकि॑ष्टे॒ व्यथि॒रा द॑धर्षीत्॥१॥

उद॑ग्ने तिष्ठ॒ प्रत्याऽऽत॑नुष्व॒ न्य॑मित्राꣳ॑ ओषतात्तिग्महेते। यो नो॒ अरा॑तिꣳ समिधान च॒क्रे नी॒चा तं ध॑क्ष्यत॒सं न शुष्कम्᳚। ऊ॒र्ध्वो भ॑व॒ प्रति॑ वि॒ध्याध्य॒स्मदा॒विष्कृ॑णुष्व॒ दैव्या᳚न्यग्ने। अव॑ स्थि॒रा त॑नुहि यातु॒जूनां᳚ जा॒मिमजा॑मिं॒ प्र मृ॑णीहि॒ शत्रून्॑। स ते॑ जानाति सुम॒तिं य॑विष्ठ॒ य ईव॑ते॒ ब्रह्म॑णे गा॒तुमैर॑त्॥२॥

विश्वा᳚न्यस्मै सु॒दिना॑नि रा॒यो द्यु॒म्नान्य॒र्यो वि दुरो॑ अ॒भि द्यौ᳚त्। सेद॑ग्ने अस्तु सु॒भगः॑ सु॒दानु॒र्यस्त्वा॒ नित्ये॑न ह॒विषा॒ य उ॒क्थैः। पिप्री॑षति॒ स्व आयु॑षि दुरो॒णे विश्वेद॑स्मै सु॒दिना॒ साऽस॑दि॒ष्टिः। अर्चा॑मि ते सुम॒तिं घोष्य॒र्वाख्सं ते॑ वा॒वाता॑ जरतामि॒यङ्गीः॥३॥
 
स्वश्वा᳚स्त्वा सु॒रथा॑ मर्जयेमा॒स्मे क्ष॒त्राणि॑ धारये॒रनु॒ द्यून्। इ॒ह त्वा॒ भूर्या च॑रे॒दुप॒ त्मन्दोषा॑\-वस्तर्दीदि॒वाꣳ\-स॒मनु॒ द्यून्। कीड॑न्तस्त्वा सु॒मन॑सः सपेमा॒भि द्यु॒म्ना त॑स्थि॒वाꣳसो॒ जना॑नाम्। यस्त्वा॒ स्वश्वः॑ सुहिर॒ण्यो अ॑ग्न उप॒याति॒ वसु॑मता॒ रथे॑न। तस्य॑ त्रा॒ता भ॑वसि॒ तस्य॒ सखा॒ यस्त॑ आति॒थ्यमा॑नु॒षग्जुजो॑षत्। म॒हो रु॑जामि ब॒न्धुता॒ वचो॑भि॒स्तन्मा॑ पि॒तुर्गोत॑मा॒दन्वि॑याय॥४॥

त्वं नो॑ अ॒स्य वच॑सश्चिकिद्धि॒ होत॑र्यविष्ठ सुक्रतो॒ दमू॑नाः। अस्व॑प्नजस्त॒रण॑यः सु॒शेवा॒ अत॑न्द्रासोऽवृ॒का अश्र॑मिष्ठाः। ते पा॒यवः॑ स॒ध्रिय॑ञ्चो नि॒षद्याऽग्ने॒ तव॑ नः पान्त्वमूर। ये पा॒यवो॑ मामते॒यं ते॑ अग्ने॒ पश्य॑न्तो अ॒न्धं दु॑रि॒तादर॑क्षन्। र॒रक्ष॒ तान्थ्सु॒कृतो॑ वि॒श्ववे॑दा॒ दिफ्स॑न्त॒ इद्रि॒पवो॒ ना ह॑ देभुः॥५॥
 
त्वया॑ व॒यꣳ स॑ध॒न्य॑स्त्वोता॒स्तव॒ प्रणी᳚त्यश्याम॒ वाजान्॑। उ॒भा शꣳसा॑ सूदय सत्यतातेऽनुष्ठु॒या कृ॑णुह्यह्रयाण। अ॒या ते॑ अग्ने स॒मिधा॑ विधेम॒ प्रति॒ स्तोमꣳ॑ श॒स्यमा॑नं गृभाय। दहा॒ऽ॒शसो॑ र॒क्षसः॑ पा॒ह्य॑स्मान्द्रु॒हो नि॒दोऽमि॑त्रमहो अव॒द्यात्। र॒क्षो॒हणं॑ वा॒जिन॒माऽऽजि॑घर्मि मि॒त्रं प्रथि॑ष्ठ॒मुप॑ यामि॒ शर्म॑। शिशा॑नो अ॒ग्निः क्रतु॑भिः॒ समि॑द्धः॒ स नो॒ दिवा॒ स रि॒षः पा॑तु॒ नक्तम्᳚॥६॥

 वि ज्योति॑षा बृह॒ता भा᳚त्य॒ग्निरा॒विर्विश्वा॑नि कृणुते महि॒त्वा। प्रादे॑वीर्मा॒याः स॑हते दु॒रेवाः॒ शिशी॑ते॒ शृङ्गे॒ रक्ष॑से वि॒निक्षे᳚। उ॒त स्वा॒नासो॑ दि॒विष॑न्त्व॒ग्नेस्ति॒ग्मायु॑धा॒ रक्ष॑से॒ हन्त॒वा उ॑। मदे॑ चिदस्य॒ प्ररु॑जन्ति॒ भामा॒ न व॑रन्ते परि॒बाधो॒ अदे॑वीः॥७॥[१.२.१४]

इन्द्रं॑ वो वि॒श्वत॒स्परि॒ हवा॑महे॒ जने᳚भ्यः। अ॒स्माक॑मस्तु॒ केव॑लः। इन्द्रं॒ नरो॑ ने॒मधि॑ता हवन्ते॒ यत्पार्या॑ यु॒नज॑ते॒ धिय॒स्ताः। शूरो॒ नृषा॑ता॒ शव॑सश्चका॒न आ गोम॑ति व्र॒जे भ॑जा॒ त्वं नः॑। इ॒न्द्रि॒याणि॑ शतक्रतो॒ या ते॒ जने॑षु प॒ञ्चसु॑। इन्द्र॒ तानि॑ त॒ आ वृ॑णे। अनु॑ ते दायि म॒ह इ॑न्द्रि॒याय॑ स॒त्रा ते॒ विश्व॒मनु॑ वृत्र॒हत्ये᳚। अनु॑ क्ष॒त्रमनु॒ सहो॑ यज॒त्रेन्द्र॑ दे॒वेभि॒रनु॑ ते नृ॒षह्ये᳚॥८॥

आ यस्मि᳚न्थ्स॒प्त वा॑स॒वास्तिष्ठ॑न्ति स्वा॒रुहो॑ यथा। ऋषि॑र्\mbox{}ह दीर्घ॒श्रुत्त॑म॒ इन्द्र॑स्य घ॒र्मो अति॑थिः। आ॒मासु॑ प॒क्वमैर॑य॒ आ सूर्यꣳ॑ रोहयो दि॒वि। घ॒र्मं न साम॑न्तपता सुवृ॒क्तिभि॒र्जुष्टं॒ गिर्व॑णसे॒ गिरः॑। इन्द्र॒मिद्गा॒थिनो॑ बृ॒हदिन्द्र॑म॒र्केभि॑र॒र्किणः॑। इन्द्रं॒ वाणी॑रनूषत। गाय॑न्ति त्वा गाय॒त्रिणोऽर्च॑न्त्य॒र्कम॒र्किणः॑॥९॥

 ब्र॒ह्माण॑स्त्वा शतक्रत॒वुद्व॒ꣳ॒शमि॑व येमिरे। अ॒ꣳ॒हो॒मुचे॒ प्र भ॑रेमा मनी॒षामो॑षिष्ठ॒दावन्ने॑ सुम॒तिं गृ॑णा॒नाः। इ॒दमि॑न्द्र॒ प्रति॑ ह॒व्यं गृ॑भाय स॒त्याः स॑न्तु॒ यज॑मानस्य॒ कामाः᳚। वि॒वेष॒ यन्मा॑ धि॒षणा॑ ज॒जान॒ स्तवै॑ पु॒रा पार्या॒दिन्द्र॒मह्नः॑। अꣳह॑सो॒ यत्र॑ पी॒पर॒द्यथा॑ नो ना॒वेव॒ यान्त॑मु॒भये॑ हवन्ते। प्र स॒म्राजं॑ प्रथ॒मम॑ध्व॒राणा॑मꣳहो॒मुचं॑ वृष॒भं य॒ज्ञिया॑नाम्॥१०॥
 
अ॒पां नपा॑तमश्विना॒ हय॑न्तम॒स्मिन्न॑र इन्द्रि॒यं ध॑त्त॒मोजः॑। वि न॑ इन्द्र॒ मृधो॑ जहि नी॒चा य॑च्छ पृतन्य॒तः। अ॒ध॒स्प॒दं तमीं᳚ कृधि॒ यो अ॒स्माꣳ अ॑भि॒दास॑ति। इन्द्र॑ क्ष॒त्रम॒भि वा॒ममोजोऽजा॑यथा वृषभ चर्\mbox{}षणी॒नाम्। अपा॑ऽनुदो॒ जन॑ममित्र॒यन्त॑मु॒रुं दे॒वेभ्यो॑ अकृणोरु लो॒कम्। मृ॒गो न भी॒मः कु॑च॒रो गि॑रि॒ष्ठाः प॑रा॒वत॒ आ ज॑गामा॒ पर॑स्याः॥११॥
 
   सृ॒कꣳ स॒ꣳ॒शाय॑ प॒विमि॑न्द्र ति॒ग्मं वि शत्रू᳚न् ताढि॒ वि मृधो॑ नुदस्व। वि शत्रू॒न्॒ वि मृधो॑ नुद॒ वि वृ॒त्रस्य॒ हनू॑ रुज। वि म॒न्युमि॑न्द्र भामि॒तो॑ऽमित्र॑स्याभि॒दास॑तः। त्रा॒तार॒मिन्द्र॑मवि॒तार॒मिन्द्र॒ꣳ॒ हवे॑हवे सु॒हव॒ꣳ॒ शूर॒मिन्द्रम्᳚। हु॒वे नु श॒क्रं पु॑रुहू॒तमिन्द्रꣴ॑ स्व॒स्ति नो॑ म॒घवा॑ धा॒त्विन्द्रः॑। मा ते॑ अ॒स्याꣳ स॑हसाव॒न् परि॑ष्टाव॒धाय॑ भूम हरिवः परा॒दै॥१२॥ 
   
   त्राय॑स्व नोऽवृ॒केभि॒र्वरू॑थै॒स्तव॑ प्रि॒यासः॑ सू॒रिषु॑ स्याम। अन॑वस्ते॒ रथ॒मश्वा॑य तक्ष॒न् त्वष्टा॒ वज्रं॑ पुरुहूत द्यु॒मन्तम्᳚। ब्र॒ह्माण॒ इन्द्रं॑ म॒हय॑न्तो अ॒र्कैरव॑र्धय॒न्नह॑ये॒ हन्त॒वा उ॑। वृष्णे॒ यत्ते॒ वृष॑णो अ॒र्कमर्चा॒निन्द्र॒ ग्रावा॑णो॒ अदि॑तिः स॒जोषाः᳚। अ॒न॒श्वासो॒ ये प॒वयो॑ऽर॒था इन्द्रे॑षिता अ॒भ्यव॑र्तन्त॒ दस्यून्॑॥१७॥[१.६.१२]


यत॑ इन्द्र॒ भया॑महे॒ ततो॑ नो॒ अभ॑यं कृधि।
मघ॑वञ्छ॒ग्धि तव॒ तन्न॑ ऊ॒तये॒ विद्विषो॒ विमृधो॑ जहि।
स्व॒स्ति॒दा वि॒शस्पति॑र्वृत्र॒हा विमृधो॑ व॒शी।
वृषेन्द्रः॑ पु॒र एतु॑ नः स्वस्ति॒दा अ॑भयङ्क॒रः।
म॒हाꣳ इन्द्रो॒ वज्र॑बाहुः षोड॒शी शर्म॑ यच्छतु।
स्व॒स्ति नो॑ म॒घवा॑ करोतु॒ हन्तु॑ पा॒प्मानं॒ यो᳚ऽस्मान्‌ द्वेष्टि॑।
स॒जोषा॑ इन्द्र॒ सग॑णो म॒रुद्भिः॒ सोमं॑ पिब वृत्रहञ्छूर वि॒द्वान्‌।
ज॒हि शत्रू॒ꣳ॒रप॒मृधो॑नुद॒स्वाथाभ॑यं कृणु हि वि॒श्वतो॑ नः॥१८॥

ये दे॒वाः पु॑रः॒ सदो॒ऽग्नि ने᳚त्रा रक्षो॒हण॒स्ते नः॑ पान्तु॒\\
ते नो॑ऽवन्तु॒  तेभ्यो॒ नम॒स्तेभ्यः॒ स्वाहा॒\\
ये दे॒वा द॑क्षिण॒सदो॑ य॒मने᳚त्रा रक्षो॒हण॒स्ते नः॑ पान्तु॒\\
ते नो॑ऽवन्तु॒  तेभ्यो॒ नम॒स्तेभ्यः॒ स्वाहा॒\\
ये दे॒वाः प॑श्चा॒त् सदः॑ सवि॒तृ ने᳚त्रा रक्षो॒हण॒स्ते नः॑ पान्तु॒\\
ते नो॑ऽवन्तु॒  तेभ्यो॒ नम॒स्तेभ्यः॒ स्वाहा॒\\
ये दे॒वा उ॑त्तर॒सदो॒ वरु॑णनेत्रा रक्षो॒हण॒स्ते नः॑ पान्तु॒\\
ते नो॑ऽवन्तु॒  तेभ्यो॒ नम॒स्तेभ्यः॒ स्वाहा॒\\
ये दे॒वा उ॑परि॒षदो॒ बृह॒स्पति॑नेत्रा रक्षो॒हण॒स्ते नः॑ पान्तु॒\\
ते नो॑ऽवन्तु॒  तेभ्यो॒ नम॒स्तेभ्यः॒ स्वाहा॒ऽग्नये॑ रक्षो॒घ्ने स्वाहा॑\\
य॒माय॑ रक्षो॒घ्ने स्वाहा॑ सवि॒त्रे र॑क्षो॒घ्ने स्वाहा॒\\
वरु॑णाय रक्षो॒घ्ने स्वाहा॒ बृह॒स्पत॑ये॒ दुव॑स्पते रक्षो॒घ्ने स्वाहा᳚॥१९॥

अ॒ग्निरायु॑ष्मा॒न्थ्स वन॒स्पति॑भि॒रायु॑ष्मा॒न्तेन॒ त्वाऽऽयु॒षा\-ऽऽयु॑ष्मन्तं करोमि॒
सोम॒ आयु॑ष्मा॒न्थ्स ओष॑धीभि॒रायु॑ष्मा॒न्तेन॒ त्वाऽऽयु॒षा\-ऽऽयु॑ष्मन्तं करोमि
य॒ज्ञ आयु॑ष्मा॒न्थ्स दक्षि॑णाभि॒रायु॑ष्मा॒न्तेन॒ त्वाऽऽयु॒षा\-ऽऽयु॑ष्मन्तं करोमि॒
ब्रह्मायु॑ष्म॒त्तद् ब्रा᳚ह्म॒णैरायु॑ष्म॒त्तेन॒ त्वाऽऽयु॒षा\-ऽऽयु॑ष्मन्तं करोमि
दे॒वा आयु॑ष्मन्त॒स्ते॑ऽमृते॒नायु॑ष्मन्त॒स्तेन॒ त्वा\-ऽऽयु॒षाऽऽयु॑ष्मन्तं करोमि॥२०॥

या वा॑मिन्द्रा वरुणा यत॒व्या॑ त॒नूस्तये॒ममꣳ ह॑सो मुञ्चतम्।\\
या वा॑मिन्द्रा वरुणा सह॒स्या॑ त॒नूस्तये॒ममꣳ ह॑सो मुञ्चतम्।\\
या वा॑मिन्द्रा वरुणा रक्ष॒स्या॑ त॒नूस्तये॒ममꣳ ह॑सो मुञ्चतम्।\\
या वा॑मिन्द्रा वरुणा तेज॒स्या॑ त॒नूस्तये॒ममꣳ ह॑सो मुञ्चतम्॥२१॥

यो वा॑मिन्द्रा वरुणाव॒ग्नौ स्राम॒स्तं वा॑मे॒तेनाव॑ यजे॒\\
यो वा॑मिन्द्रा वरुणा द्वि॒पाथ्सु॑ प॒शुषु॒ स्राम॒स्तं वा॑मे॒तेनाव॑ यजे॒\\
यो वा॑मिन्द्रा वरुणा॒ चतु॑ष्पाथ्सु॒ स्राम॒स्तं वा॑मे॒तेनाव॑ यजे॒\\
यो वा॑मिन्द्रा वरुणा गो॒ष्ठे स्राम॒स्तं वा॑मे॒तेनाव॑ यजे॒\\
यो वा॑मिन्द्रा वरुणा गृ॒हेषु॒ स्राम॒स्तं वा॑मे॒तेनाव॑ यजे॒\\
यो वा॑मिन्द्रा वरुणा॒फ्सु स्राम॒स्तं वा॑मे॒तेनाव॑ यजे॒\\
यो वा॑मिन्द्रा वरु॒णौष॑धीषु॒ स्राम॒स्तं वा॑मे॒तेनाव॑ यजे॒\\
यो वा॑मिन्द्रा वरुणा॒ वन॒स्पति॑षु॒ स्राम॒स्तं वा॑मे॒तेनाव॑ यजे॥२२॥

अग्ने॑ यशस्वि॒न् यश॑से॒ मम॑र्प॒येन्द्रा॑वती॒मप॑चितीमि॒हाव॑ह।
अ॒यं मू॒र्धा प॑रमे॒ष्ठी सु॒वर्चाः᳚ समा॒नाना॑मुत्त॒मश्लो॑को अस्तु॥
भ॒द्रं पश्य॑न्त॒ उप॑सेदु॒रग्रे॒ तपो॑ दी॒क्षामृष॑यः सुव॒र्विदः॑।
ततः॑ क्ष॒त्रं बल॒मोज॑श्च जा॒तं तद॒स्मै दे॒वा अ॒भि सन्न॑मन्तु॥
धा॒ता वि॑धा॒ता प॑र॒मोतस॒न्धृक् प्र॒जाप॑तिः परमे॒ष्ठी वि॒राजा᳚।
स्तोमा॒श्छन्दाꣳ॑सि नि॒विदो॑म आहुरे॒तस्मै॑ रा॒ष्ट्रम॒भिसन्न॑माम॥
अ॒भ्याव॑र्तध्व॒मुप॒मेत॑सा॒क म॒यꣳ शा॒स्ताऽधि॑पतिर्वो अस्तु।
अ॒स्य वि॒ज्ञान॒मनु॒सꣳर॑भध्वमि॒मं प॒श्चादनु॑ जीवाथ॒ सर्वे᳚॥२३॥


\sect{राष्ट्रभृतम्}
ऋ॒ता॒षाडृ॒तधा॑मा॒ग्निर्ग॑न्ध॒र्वस्तस्यौष॑धयोऽफ्स॒रस॒ ऊर्जो॒ नाम॒\\
स इ॒दं ब्रह्म॑क्ष॒त्रं पा॑तु॒ ता इ॒दं ब्रह्म॑क्ष॒त्रं पा᳚न्तु॒ तस्मै॒ स्वाहा॒ ताभ्यः॒ स्वाहा॑\\
सहि॒तो वि॒श्वसा॑मा॒ सूर्यो॑ गन्ध॒र्वस्तस्य॒ मरी॑चयोऽफ्स॒रस॑ आ॒युवो॒ नाम॒\\
स इ॒दं ब्रह्म॑क्ष॒त्रं पा॑तु॒ ता इ॒दं ब्रह्म॑क्ष॒त्रं पा᳚न्तु॒ तस्मै॒ स्वाहा॒ ताभ्यः॒ स्वाहा॑\\
सुषु॒म्णः सूर्य॑रश्मिश्च॒न्द्रमा॑ गन्ध॒र्वस्तस्य॒ नक्ष॑त्राण्यऽफ्स॒रसो॑ बे॒कुर॑यो॒ नाम॒\\
स इ॒दं ब्रह्म॑क्ष॒त्रं पा॑तु॒ ता इ॒दं ब्रह्म॑क्ष॒त्रं पा᳚न्तु॒ तस्मै॒ स्वाहा॒ ताभ्यः॒ स्वाहा॑\\
भु॒ज्युः सुु॑प॒र्णो य॒ज्ञो ग॑न्ध॒र्वस्तस्य॒ दक्षि॑णा अफ्स॒रस॑स्त॒वा नाम॒\\
स इ॒दं ब्रह्म॑क्ष॒त्रं पा॑तु॒ ता इ॒दं ब्रह्म॑क्ष॒त्रं पा᳚न्तु॒ तस्मै॒ स्वाहा॒ ताभ्यः॒ स्वाहा᳚\\
प्र॒जाप॑तिर्वि॒श्वक॑र्मा॒ मनो॑ गन्ध॒र्वस्तस्य॑र्ख्सा॒मान्य॑फ्स॒रसो॒ वह्न॑यो॒  नाम॒\\
स इ॒दं ब्रह्म॑क्ष॒त्रं पा॑तु॒ ता इ॒दं ब्रह्म॑क्ष॒त्रं पा᳚न्तु॒ तस्मै॒ स्वाहा॒ ताभ्यः॒ स्वाहे॑षि॒रो\\
वि॒श्वव्य॑चा॒ वातो॑ गन्ध॒र्वस्तस्यापो᳚ऽफ्स॒रसो॑ मु॒दा नाम॒\\
स इ॒दं ब्रह्म॑क्ष॒त्रं पा॑तु॒ ता इ॒दं ब्रह्म॑क्ष॒त्रं पा᳚न्तु॒ तस्मै॒ स्वाहा॒ ताभ्यः॒ स्वाहा᳚।

भुव॑नस्य पते॒ यस्य॑ त उ॒परि॑ गृ॒हा इ॒ह च॑।
स नो॑ रा॒स्वाज्या॑निꣳ रा॒यस्पोषꣳ॑ सु॒वीर्यꣳ॑ संवथ्स॒रीणाꣴ॑ स्व॒स्तिम्। (स्वाहा॑)।

प॒र॒मे॒ष्ठ्यधि॑पतिर्मृ॒त्युर्ग॑न्ध॒र्वस्तस्य॒ विश्व॑मफ्स॒रसो॒ भुवो॒  नाम॒\\
स इ॒दं ब्रह्म॑क्ष॒त्रं पा॑तु॒ ता इ॒दं ब्रह्म॑क्ष॒त्रं पा᳚न्तु॒ तस्मै॒ स्वाहा॒ ताभ्यः॒ स्वाहा॑\\
सुक्षि॒तिः सुुभू॑तिर्भद्र॒कृथ्सुव॑र्वान्प॒र्जन्यो॑ गन्ध॒र्वस्तस्य॑ वि॒द्युतो᳚ऽफ्स॒रसो॒ रुचो॒ नाम॒\\
स इ॒दं ब्रह्म॑क्ष॒त्रं पा॑तु॒ ता इ॒दं ब्रह्म॑क्ष॒त्रं पा᳚न्तु॒ तस्मै॒ स्वाहा॒ ताभ्यः॒ स्वाहा॑\\
दू॒रे हे॑तिरमृड॒यो मृ॒त्युर्ग॑न्ध॒र्वस्तस्य॑ प्र॒जा अ॑फ्स॒रसो॑ भी॒रुवो॒ नाम॒\\
स इ॒दं ब्रह्म॑क्ष॒त्रं पा॑तु॒ ता इ॒दं ब्रह्म॑क्ष॒त्रं पा᳚न्तु॒ तस्मै॒ स्वाहा॒ ताभ्यः॒ स्वाहा॒\\
चारुः॑ कृपणका॒शी कामो॑ गन्ध॒र्वस्तस्या॒धयो᳚ऽफ्स॒रसः॑ शो॒चय॑न्ती॒र्नाम॒\\
स इ॒दं ब्रह्म॑क्ष॒त्रं पा॑तु॒ ता इ॒दं ब्रह्म॑क्ष॒त्रं पा᳚न्तु॒ तस्मै॒ स्वाहा॒ ताभ्यः॒ स्वाहा॒\\
स नो॑ भुवनस्य पते॒ यस्य॑ त उ॒परि॑ गृ॒हा इ॒ह च॑।
उ॒रु ब्रह्म॑णे॒ऽस्मै क्ष॒त्राय॒ महि॒ शर्म॑ यच्छ (स्वाहा᳚)॥\\
नमो॑ अस्तु स॒र्पेभ्यो॒ ये के च॑ पृथि॒वीमनु॑।
ये अ॒न्तरि॑क्षे॒ ये दि॒वि  तेभ्यः॑ स॒र्पेभ्यो॒ नमः॑॥
ये॑ऽदो रो॑च॒ने दि॒वो ये वा॒ सूर्य॑स्य र॒श्मिषु॑।
येषा॑म॒फ्सु सदः॑ कृ॒तं तेभ्यः॑ स॒र्पेभ्यो॒ नमः॑॥
या इष॑वो यातु॒धाना॑नां॒ ये वा॒ वन॒स्पती॒ꣳ॒रनु॑।
ये वा॑ऽव॒टेषु॒ शेर॑ते॒ तेभ्यः॑ स॒र्पेभ्यो॒ नमः॑॥

\sect{पञ्चचोडाः}
अ॒यं पु॒रो हरि॑केशः॒ सूर्य॑रश्मि॒स्तस्य॑ रथगृ॒थ्सश्च॒ रथौ॑जाश्च सेनानिग्राम॒ण्यौ॑
पुञ्जिकस्थ॒ला च॑ कृतस्थ॒ला चा᳚फ्स॒रसौ॑ यातु॒धाना॑ हे॒ती रक्षाꣳ॑सि॒ 
प्रहे॑ति॒स्तेभ्यो॒ नम॒स्ते नो॑ मृडयन्तु॒ ते यं द्वि॒ष्मो यश्च॑ नो॒ द्वेष्टि॒ तं वो॒  जम्भे॑ दधामि।

अ॒यं द॑क्षि॒णा वि॒श्वक॑र्मा॒ तस्य॑ रथस्व॒नश्च॒ रथे॑ चित्रश्च  सेनानिग्राम॒ण्यौ॑
मेन॒का च॑ सहज॒न्या चा᳚फ्स॒रसौ॑ द॒ङ्क्ष्णवः॑ प॒शवो॑ हे॒तिः पौरु॑षेयोव॒धः
प्रहे॑ति॒स्तेभ्यो॒ नम॒स्ते नो॑ मृडयन्तु॒ ते यं द्वि॒ष्मो यश्च॑ नो॒ द्वेष्टि॒ तं वो॒  जम्भे॑ दधामि।

अ॒यं प॒श्चाद्वि॒श्वव्य॑चा॒स्तस्य॒ रथ॑प्रोत॒श्चास॑मरथश्च सेनानिग्राम॒ण्यौ᳚
प्र॒म्लोच॑न्ती चानु॒म्लोच॑न्ती चाफ्स॒रसौ॑ स॒र्पा हे॒तिर्व्या॒घ्राः 
प्रहे॑ति॒स्तेभ्यो॒ नम॒स्ते नो॑ मृडयन्तु॒ ते यं द्वि॒ष्मो यश्च॑ नो॒ द्वेष्टि॒ तं वो॒  जम्भे॑ दधामि।

अ॒यमु॑त्त॒राथ्सं॒यद्व॑सु॒स्तस्य॑ सेन॒जिच्च॑ सु॒षेण॑श्च सेनानिग्राम॒ण्यौ॑
वि॒श्वाची॑ च घृ॒ताची॑ चाफ्स॒रसा॒वापो॑ हे॒तिर्वातः॒ 
प्रहे॑ति॒स्तेभ्यो॒ नम॒स्ते नो॑ मृडयन्तु॒ ते यं द्वि॒ष्मो यश्च॑ नो॒ द्वेष्टि॒ तं वो॒  जम्भे॑ दधामि।

अ॒यमु॒पर्य॒र्वाग्व॑सु॒स्तस्य॒ तार्क्ष्य॒श्चारि॑ष्टनेमिश्च  सेनानि\-ग्राम॒ण्या॑वु॒र्वशी॑ च
पू॒र्वचि॑त्तिश्चाफ्स॒रसौ॑ वि॒द्युद्धे॒तिर॑व॒\-स्फूर्ज॒न्प्रहे॑ति॒स्तेभ्यो॒
नम॒स्ते नो॑ मृडयन्तु॒ ते यं द्वि॒ष्मो यश्च॑ नो॒ द्वेष्टि॒ तं वो॒  जम्भे दधामि।


\sect{अप्रतिरथम्}
\centerline{\scriptsize (तैत्तिरीयसंहितायां काण्डः – ४/प्रश्नः – ६/अनुवाकः – ४)}

आ॒शुः शिशा॑नो वृष॒भो न यु॒ध्मो घ॑नाघ॒नः क्षोभ॑णश्चर्‌षणी॒नाम्। स॒ङ्क्रन्द॑नोऽनिमि॒ष ए॑कवी॒रः श॒तꣳ सेना॑ अजयत् सा॒कमिन्द्रः॑। स॒ङ्क्रन्द॑नेनानिमि॒षेण॑ जि॒ष्णुना॑ युत्का॒रेण॑ दुश्च्यव॒नेन॑ धृ॒ष्णुना᳚। तदिन्द्रे॑ण जयत॒ तथ्स॑हध्वं॒ युधो॑ नर॒ इषु॑हस्तेन॒ वृष्णा᳚। स इषु॑हस्तैः॒ सनि॑ष॒ङ्गिभि॑र्व॒शी सꣴस्र॑ष्टा॒ स युध॒ इन्द्रो॑ ग॒णेन॑। स॒ꣳ॒सृ॒ष्ट॒जिथ्सो॑म॒पा बा॑हुश॒ध्यू᳚र्ध्वध॑न्वा॒ प्रति॑हिताभि॒रस्ता᳚।

बृह॑स्पते॒ परि॑ दीया॒ रथे॑न रक्षो॒हाऽमित्राꣳ॑ अप॒बाध॑मानः। प्र॒भ॒ञ्जन्थ्सेनाः᳚ प्रमृ॒णो यु॒धा जय॑न्न॒स्माक॑मेध्यवि॒ता रथा॑नाम्। गो॒त्र॒भिदं॑ गो॒विदं॒ वज्र॑बाहुं॒ जय॑न्त॒मज्म॑ प्रमृ॒णन्त॒मोज॑सा। इ॒मꣳ स॑जाता॒ अनु॑ वीरयध्व॒मिन्द्रꣳ॑ सखा॒योऽनु॒ सꣳ र॑भध्वम्। ब॒ल॒वि॒ज्ञा॒यः स्थवि॑रः॒ प्रवी॑रः॒ सह॑स्वान् वा॒जी सह॑मान उ॒ग्रः। अ॒भिवी॑रो अ॒भिस॑त्त्वा सहो॒जा जैत्र॑मिन्द्र॒ रथ॒मा ति॑ष्ठ गो॒वित्। अ॒भि गो॒त्राणि॒ सह॑सा॒ गाह॑मानोऽदा॒यो वी॒रः श॒तम॑न्यु॒रिन्द्रः॑।

दु॒श्च्य॒व॒न पृ॑तना॒षाड॑यु॒ध्यो᳚ऽस्माक॒ꣳ॒ सेना॑ अवतु॒ प्र यु॒थ्सु। इन्द्र॑ आसां ने॒ता बृह॒स्पति॒र्दक्षि॑णा य॒ज्ञः पु॒र ए॑तु॒ सोमः॑। दे॒व॒से॒नाना॑मभिभञ्जती॒नां जय॑न्तीनां म॒रुतो॑ य॒न्त्वग्रे᳚। इन्द्र॑स्य॒ वृष्णो॒ वरु॑णस्य॒ राज्ञ॑ आदि॒त्यानां᳚ म॒रुता॒ꣳ॒ शर्ध॑ उ॒ग्रम्। म॒हाम॑नसां भुवनच्य॒वानां॒ घोषो॑ दे॒वानां॒ जय॑ता॒मुद॑स्थात्। अ॒स्माक॒मिन्द्रः॒ समृ॑तेषु ध्व॒जेष्व॒स्माकं॒ या इष॑व॒स्ता ज॑यन्तु।

अ॒स्माकं॑ वी॒रा उत्त॑रे भवन्त्व॒स्मानु॑ देवा अवता॒ हवे॑षु। उद्ध॑र्‌षय मघव॒न्नायु॑धा॒न्युत् सत्त्व॑नां माम॒कानां॒ महाꣳ॑सि। उद्वृ॑त्रहन् वा॒जिनां॒ वाजि॑ना॒न्युद्रथा॑नां॒ जय॑तामेतु॒ घोषः॑। उप॒ प्रेत॒ जय॑ता नरः स्थि॒रा वः॑ सन्तु बा॒हवः॑। इन्द्रो॑ वः॒ शर्म॑ यच्छत्त्वनाधृ॒ष्या यथाऽस॑थ। अव॑सृष्टा॒ परा॑ पत॒ शर॑व्ये॒ ब्रह्म॑सꣳशिता।

गच्छा॒मित्रा॒न् प्रवि॑श॒ मैषां॒ कं च॒नोच्छि॑षः। मर्मा॑णि ते॒ वर्म॑भिश्छादयामि॒ सोम॑स्त्वा॒ राजा॒ऽमृते॑ना॒भिव॑स्ताम्। उ॒रोर्वरी॑यो॒ वरि॑वस्ते अस्तु॒ जय॑न्तं॒ त्वामनु॑ मदन्तु दे॒वाः। यत्र॑ बा॒णाः स॒म्पत॑न्ति कुमा॒रा वि॑शि॒खा इ॑व। इन्द्रो॑ न॒स्तत्र॑ वृत्र॒हा वि॑श्वा॒हा शर्म॑ यच्छतु॥

शं च॑ मे॒ मय॑श्च मे प्रि॒यं च॑ मेऽनुका॒मश्च॑ मे॒ काम॑श्च मे सौमन॒सश्च॑ मे भ॒द्रं च॑ मे॒ श्रेय॑श्च मे॒ वस्य॑श्च मे॒ यश॑श्च मे॒ भग॑श्च मे॒ द्रवि॑णं च मे य॒न्ता च॑ मे ध॒र्ता च॑ मे॒ क्षेम॑श्च मे॒ धृति॑श्च मे॒ विश्वं॑ च मे॒ मह॑श्च मे सं॒विच्च॑ मे॒ ज्ञात्रं॑ च मे॒ सूश्च॑ मे प्र॒सूश्च॑ मे॒ सीरं॑ च मे ल॒यश्च॑ म ऋ॒तं च॑ मे॒ऽमृतं॑ च मेऽय॒क्ष्मं च॒ मेऽना॑मयच्च मे जी॒वातु॑श्च मे दीर्घायु॒त्वं च॑ मेऽनमि॒त्रं च॒ मेऽभ॑यं च मे सु॒गं च॑ मे॒ शय॑नं च मे सू॒षा च॑ मे सु॒दिनं॑ च मे॥

\sect{विहव्यम्}
ममा᳚ग्ने॒ वर्चो॑ विह॒वेष्व॑स्तु व॒यं त्वेन्धा॑नास्त॒नुवं॑ पुषेम। मह्यं॑ नमन्तां प्र॒दिश॒श्चत॑स्र॒स्त्वयाऽध्य॑क्षेण॒ पृत॑ना जयेम। मम॑ दे॒वा वि॑ह॒वे स॑न्तु॒ सर्व॒ इन्द्रा॑वन्तो म॒रुतो॒ विष्णु॑र॒ग्निः। ममा॒न्तरि॑क्षमु॒रु गो॒पम॑स्तु॒ मह्यं॒ वातः॑ पवतां॒ कामे॑ अ॒स्मिन्। मयि॑ दे॒वा द्रवि॑ण॒मा य॑जन्तां॒ मय्या॒शीर॑स्तु॒ मयि॑ दे॒वहू॑तिः। दैव्या॒ होता॑रा वनिषन्त॒ पूर्वेऽरि॑ष्टाः स्याम त॒नुवा॑ सु॒वीराः᳚। मह्यं॑ यजन्तु॒ मम॒ यानि॑ ह॒व्याऽऽकू॑तिः स॒त्या मन॑सो मे अस्तु। एनो॒ मा नि गां᳚ कत॒मच्च॒नाहं विश्वे॑ देवासो॒ अधि॑ वोचता मे। देवीः᳚ षडुर्वीरु॒रु णः॑ कृणोत॒ विश्वे॑ देवास इ॒ह वी॑रयध्वम्। मा हा᳚स्महि प्र॒जया॒ मा त॒नूभि॒र्मा र॑धाम द्विष॒ते सो॑म राजन्। अ॒ग्निर्म॒न्युं प्र॑तिनु॒दन् पु॒रस्ता॒दद॑ब्धो गो॒पाः परि॑ पाहि न॒स्त्वम्। प्र॒त्यञ्चो॑ यन्तु नि॒गुतः॒ पुन॒स्ते॑ऽमैषां᳚ चि॒त्तं प्र॒बुधा॒ वि ने॑शत्। धा॒ता धा॑तृ॒णां भुव॑नस्य॒ यस्पति॑र्दे॒वꣳ स॑वि॒तार॑मभिमाति॒षाऽहम्᳚। इ॒मं य॒ज्ञम॒श्विनो॒भा बृह॒स्पति॑र्दे॒वाः पा᳚न्तु॒ यज॑मानं न्य॒र्थात्। उ॒रु॒व्यचा॑ नो महि॒षः शर्म॑ यꣳसद॒स्मिन् हवे॑ पुरुहू॒तः पु॑रु॒क्षु। स नः॑ प्र॒जायै॑ हर्यश्व मृड॒येन्द्र॒ मा नो॑ रीरिषो॒ मा परा॑ दाः। ये नः॑ स॒पत्ना॒ अप॒ ते भ॑वन्त्विन्द्रा॒ग्निभ्या॒मव॑ बाधामहे॒ तान्। वस॑वो रु॒द्रा आ॑दि॒त्या उ॑परि॒स्पृशं॑ मो॒ग्रं चेत्ता॑रमधिरा॒जम॑क्रन्। अ॒र्वाञ्च॒मिन्द्र॑म॒मुतो॑ हवामहे॒ यो गो॒जिद्ध॑न॒जिद॑श्व॒जिद्यः। इ॒मं नो॑ य॒ज्ञं वि॑ह॒वे जु॑षस्वा॒स्य कु॑र्मो हरिवो मे॒दिनं॑ त्वा॥

\sect{मृगारम्}
अ॒ग्नेर्म॑न्वे प्रथ॒मस्य॒ प्रचे॑तसो॒ यं पाञ्च॑जन्यं ब॒हवः॑ समि॒न्धते᳚। विश्व॑स्यां वि॒शि प्र॑विविशि॒वाꣳस॑मीमहे॒ स नो॑ मुञ्च॒त्वꣳह॑सः। यस्ये॒दं प्रा॒णं नि॑मि॒षद्यदेज॑ति॒ यस्य॑ जा॒तं जन॑मानं च॒ केव॑लम्। स्तौम्य॒ग्निं ना॑थि॒तो जो॑हवीमि॒ स नो॑ मुञ्च॒त्वꣳह॑सः।\\
इन्द्र॑स्य मन्ये प्रथ॒मस्य॒ प्रचे॑तसो वृत्र॒घ्नः स्तोमा॒ उप॒ मामु॒पागुः॑। यो दा॒शुषः॑ सु॒कृतो॒ हव॒मुप॒ गन्ता॒ स नो॑ मुञ्च॒त्वꣳह॑सः। यः स॑ङ्ग्रा॒मं नय॑ति॒ संव॒शी यु॒धे यः पु॒ष्टानि॑ सꣳसृ॒जति॑ त्र॒याणि॑। स्तौमीन्द्रं॑ नाथि॒तो जो॑हवीमि॒ स नो॑ मुञ्च॒त्वꣳह॑सः।\\
म॒न्वे वां᳚ मित्रावरुणा॒ तस्य॑ वित्त॒ꣳ॒ सत्यौ॑जसा दृꣳहणा॒ यं नु॒देथे᳚। या राजा॑नꣳ स॒रथं॑ या॒थ उ॑ग्रा॒ ता नो॑ मुञ्चत॒माग॑सः। यो वा॒ꣳ॒ रथ॑ ऋ॒जुर॑श्मिः स॒त्यध॑र्मा॒ मिथु॒श्चर॑न्तमुप॒याति॑ दू॒षयन्॑। स्तौमि॑ मि॒त्रावरु॑णा नाथि॒तो जो॑हवीमि॒ तौ नो॑ मुञ्चत॒माग॑सः।\\
वा॒योः स॑वि॒तुर्वि॒दथा॑नि मन्महे॒ यावा᳚त्म॒न्वद्बि॑भृ॒तो यौ च॒ रक्ष॑तः। यौ विश्व॑स्य परि॒भू ब॑भू॒वतु॒स्तौ नो॑ मुञ्चत॒माग॑सः। उप॒ श्रेष्ठा॑ न आ॒शिषो॑ दे॒वयो॒र्धर्मे॑ अस्थिरन्। स्तौमि॑ वा॒युꣳ स॑वि॒तारं॑ नाथि॒तो जो॑हवीमि॒ तौ नो॑ मुञ्चत॒माग॑सः।\\
र॒थीत॑मौ रथी॒नाम॑ह्व ऊ॒तये॒ शुभं॒ गमि॑ष्ठौ सु॒यमे॑भि॒रश्वैः᳚। ययो᳚र्वां देवौ दे॒वेष्वनि॑शित॒मोज॒स्तौ नो॑ मुञ्चत॒माग॑सः। यदया॑तं वह॒तुꣳ सू॒र्याया᳚स्त्रिच॒क्रेण॑ स॒ꣳ॒सद॑मि॒च्छमा॑नौ। स्तौमि॑ दे॒वाव॒श्विनौ॑ नाथि॒तो जो॑हवीमि॒ तौ नो॑ मुञ्चत॒माग॑सः।\\
म॒रुतां᳚ मन्वे॒ अधि॑ नो ब्रुवन्तु॒ प्रेमां वाचं॒ विश्वा॑मवन्तु॒ विश्वे᳚। आ॒शून् हु॑वे सु॒यमा॑नू॒तये॒ ते नो॑ मुञ्च॒न्त्वेन॑सः। ति॒ग्ममायु॑धं वीडि॒तꣳ सह॑स्वद्दि॒व्यꣳ शर्धः॒ पृत॑नासु जि॒ष्णु। स्तौमि॑ दे॒वान्म॒रुतो॑ नाथि॒तो जो॑हवीमि॒ ते नो॑ मुञ्च॒न्त्वेन॑सः।\\
दे॒वानां᳚ मन्वे॒ अधि॑ नो ब्रुवन्तु॒ प्रेमां वाचं॒ विश्वा॑मवन्तु॒ विश्वे᳚। आ॒शून् हु॑वे सु॒यमा॑नू॒तये॒ ते नो॑ मुञ्च॒न्त्वेन॑सः। यदि॒दं मा॑ऽभि॒शोच॑ति॒ पौरु॑षेयेण॒ दैव्ये॑न। स्तौमि॒ विश्वा᳚न्दे॒वान्ना॑थि॒तो जो॑हवीमि॒ ते नो॑ मुञ्च॒न्त्वेन॑सः।\\
अनु॑ नो॒ऽद्यानु॑मतिर्य॒ज्ञं दे॒वेषु॑ मन्यताम्। अ॒ग्निश्च॑ हव्य॒वाह॑नो॒ भव॑तां दा॒शुषे॒ मयः॑। अन्विद॑नुमते॒ त्वं मन्या॑सै॒ शं च॑ नः कृधि। क्रत्वे॒ दक्षा॑य नो हि नु॒ प्रण॒ आयूꣳ॑षि तारिषः।
वै॒श्वा॒न॒रो न॑ ऊ॒त्या प्र या॑तु  परा॒वतः॑। अ॒ग्निरु॒क्थेन॒ वाह॑सा॥
पृ॒ष्टो दि॒वि पृ॒ष्टो अ॒ग्निः पृ॑थि॒व्यां पृ॒ष्टो विश्वा॒ ओष॑धी॒रावि॑वेश। वै॒श्वा॒न॒रसह॑सा पृ॒ष्टो अ॒ग्निः स नो॒ दिवा॒ स रि॒षः पा॑तु॒ नक्तम्᳚।  ये अप्र॑थेता॒ममि॑तेभि॒रोजो॑भि॒र्ये प्र॑ति॒ष्ठे अभ॑वतां॒ वसू॑नाम्। 

स्तौमि॒ द्यावा॑पृथि॒वी ना॑थि॒तो जो॑हवीमि॒ ते नो॑ मुञ्चत॒मꣳह॑सः। उर्वी॑ रोदसी॒ वरि॑वः कृणोतं॒ क्षेत्र॑स्य पत्नी॒ अधि॑ नो ब्रूयातम्। 
स्तौमि॒ द्यावा॑पृथि॒वी ना॑थि॒तो जो॑हवीमि॒ ते नो॑ मुञ्चत॒मꣳह॑सः। यत्ते॑ व॒यं पु॑रुष॒त्रा य॑वि॒ष्ठाऽवि॑द्वाꣳसश्चकृ॒मा कच्च॒नाऽऽगः॑। 
कृ॒धी स्व॑स्माꣳ अदि॑ते॒रना॑गा॒ व्येनाꣳ॑सि शिश्रथो॒ विष्व॑गग्ने। यथा॑ ह॒ तद्व॑सवो गौ॒र्यं॑ चित्प॒दि षि॒ताममु॑ञ्चता यजत्राः। ए॒वा त्वम॒स्मत्प्र मु॑ञ्चा॒ व्यꣳहः॒ प्राता᳚र्यग्ने प्रत॒रां न॒ आयुः॑॥

\sect{सर्पाहुतिः}
स॒मीची॒ नामा॑सि॒ प्राची॒ दिक्तस्या᳚स्ते॒ऽग्निरधि॑\-पतिरसि॒तो र॑क्षि॒ता 
यश्चाधि॑पति॒र्यश्च॑ गो॒प्ता ताभ्यां॒ नम॒स्तौ नो॑ मृडयतां॒ ते 
यं द्वि॒ष्मो यश्च॑ नो॒ द्वेष्टि॒ तं वां॒ जम्भे॑\\
दधाम्योज॒स्विनी॒ नामा॑सि दक्षि॒णा दिक्तस्या᳚स्त॒ इन्द्रोऽधि॑-पतिः॒ पृदा॑कू रक्षि॒ता 
यश्चाधि॑पति॒र्यश्च॑ गो॒प्ता ताभ्यां॒ नम॒स्तौ नो॑ मृडयतां॒ ते 
यं द्वि॒ष्मो यश्च॑ नो॒ द्वेष्टि॒ तं वां॒ जम्भे॑ दधामि॒\\ 
प्राची॒ नामा॑सि प्र॒तीची॒ दिक्तस्या᳚स्ते॒ सोमोऽधि॑\-पतिः स्व॒जो र॑क्षि॒ता 
यश्चाधि॑पति॒र्यश्च॑ गो॒प्ता ताभ्यां॒ नम॒स्तौ नो॑ मृडयतां॒ ते 
यं द्वि॒ष्मो यश्च॑ नो॒ द्वेष्टि॒ तं वां॒ जम्भे॑\\
दधाम्यव॒स्थावा॒ नामा॒स्युदी॑ची॒ दिक्तस्या᳚स्ते॒ वरु॒णोऽधि॑\-पतिस्ति॒रश्च॑राजी रक्षि॒ता 
यश्चाधि॑पति॒र्यश्च॑ गो॒प्ता ताभ्यां॒ नम॒स्तौ नो॑ मृडयतां॒ ते 
यं द्वि॒ष्मो यश्च॑ नो॒ द्वेष्टि॒ तं वां॒ जम्भे॑\\
दधा॒म्यधि॑पत्नी॒ नामा॑सि बृह॒ती दिक्तस्या᳚स्ते॒ बृह॒स्पति॒रधि॑पतिः श्वि॒त्रो र॑क्षि॒ता 
यश्चाधि॑पति॒र्यश्च॑ गो॒प्ता ताभ्यां॒ नम॒स्तौ नो॑ मृडयतां॒ ते 
यं द्वि॒ष्मो यश्च॑ नो॒ द्वेष्टि॒ तं वां॒ जम्भे॑ दधामि\\
व॒शिनी॒ नामा॑सी॒यं दिक्तस्या᳚स्ते य॒मोऽधि॑पतिः क॒ल्माष॑ग्रीवो रक्षि॒ता 
यश्चाधि॑पति॒र्यश्च॑ गो॒प्ता ताभ्यां॒ नम॒स्तौ नो॑ मृडयतां॒ ते 
यं द्वि॒ष्मो यश्च॑ नो॒ द्वेष्टि॒ तं वां॒ जम्भे॑ दधामि॥

\sect{गन्धर्वाहुतीः}
हे॒तयो॒ नाम॑ स्थ॒ तेषां᳚ वः पु॒रो गृ॒हा अ॒ग्निर्व॒ इष॑वः सलि॒लो\\ 
वा॑तना॒मं तेभ्यो॑ वो॒ नम॒स्ते नो॑ मृडयत॒ ते यं द्वि॒ष्मो यश्च॑ नो॒ द्वेष्टि॒ तं वो॒ जम्भे॑ दधामि\\
निलि॒म्पा नाम॑ स्थ॒ तेषां᳚ वो दक्षि॒णा गृ॒हाः पि॒तरो॑ व॒ इष॑वः॒ सग॑रो\\ 
वातना॒मं तेभ्यो॑ वो॒ नम॒स्ते नो॑ मृडयत॒ ते यं द्वि॒ष्मो यश्च॑ नो॒ द्वेष्टि॒ तं वो॒ जम्भे॑ दधामि\\
व॒ज्रिणो॒ नाम॑ स्थ॒ तेषां᳚ वः प॒श्चाद्गृ॒हाः स्वप्नो॑ व॒ इष॑वो॒ गह्व॑रो\\ 
वातना॒मं तेभ्यो॑ वो॒ नम॒स्ते नो॑ मृडयत॒ ते यं द्वि॒ष्मो यश्च॑ नो॒ द्वेष्टि॒ तं वो॒ जम्भे॑\\
दधाम्यव॒स्थावा॑नो॒ नाम॑ स्थ॒ तेषां᳚ व उत्त॒राद्गृ॒हा आपो॑ व॒ इष॑वः समु॒द्रो\\ 
वा॑तना॒मं तेभ्यो॑ वो॒ नम॒स्ते नो॑ मृडयत॒ ते यं द्वि॒ष्मो यश्च॑ नो॒ द्वेष्टि॒ तं वो॒ जम्भे॑\\
दधा॒म्यधि॑पतयो॒ नाम॑ स्थ॒ तेषां᳚ व उ॒परि॑ गृ॒हा व॒र्\mbox{}षं व॒ इष॒वोऽव॑स्वान्\\ 
वातना॒मं तेभ्यो॑ वो॒ नम॒स्ते नो॑ मृडयत॒ ते यं द्वि॒ष्मो यश्च॑ नो॒ द्वेष्टि॒ तं वो॒ जम्भे॑ दधामि\\
क्र॒व्या नाम॑ स्थ॒ पार्थि॑वा॒स्तेषां᳚ व इ॒ह गृ॒हा अन्नं॑ व॒ इष॑वो निमि॒षो\\ 
वा॑तना॒मं तेभ्यो॑ वो॒ नम॒स्ते नो॑ मृडयत॒ ते यं द्वि॒ष्मो यश्च॑ नो॒ द्वेष्टि॒ तं वो॒ जम्भे॑ दधामि।

\sect{अज्यानि}
श॒तायु॑धाय श॒तवी᳚र्याय श॒तोत॑येऽभिमाति॒षाहे᳚।
श॒तं यो नः॑ श॒रदो॒ अजी॑ता॒निन्द्रो॑ नेष॒दति॑ दुरि॒तानि॒ विश्वा᳚॥

ये च॒त्वारः॑ प॒थयो॑ देव॒याना॑ अन्त॒रा द्यावा॑पृथि॒वी वि॒यन्ति॑।
तेषां॒ यो अज्या॑नि॒मजी॑तिमा॒ वहा॒त्तस्मै॑ नो देवाः॒ परि॑ दत्ते॒ह सर्वे᳚॥

ग्री॒ष्मो हे॑म॒न्त उ॒त नो॑ वस॒न्तः श॒रद्व॒र्षाः सु॑वि॒तं नो॑ अस्तु।
तेषा॑मृतू॒नाꣳ श॒तशा॑रदानां निवा॒त ए॑षा॒मभ॑ये स्याम॥

इ॒दु॒व॒थ्स॒राय॑ परिवथ्स॒राय॑ संवथ्स॒राय॑ कृणुता बृ॒हन्नमः॑।
तेषां᳚ व॒यꣳ सु॑म॒तौ य॒ज्ञिया॑नां॒ ज्योगजी॑ता॒ अह॑ताः स्याम॥

भ॒द्रान्नः॒ श्रेयः॒ सम॑नैष्ट देवा॒स्त्वया॑ऽव॒सेन॒ सम॑शीमहि त्वा।
स नो॑ मयो॒भूः पि॑तो॒ आवि॑शस्व॒ शन्तो॒काय॑ त॒नुवे᳚ स्यो॒ नः॥

भू॒तं भव्यं॑ भवि॒ष्यद्वष॒ट् स्वाहा॒\\
नम॒ ऋख्साम॒यजु॒र्वष॒ट् स्वाहा॒\\
नमो॑ गाय॒त्रीत्रि॒ष्टुब्जग॑ती॒ वष॒ट् स्वाहा॒\\
नमः॑ पृथि॒व्य॑न्तरि॑क्षं॒ द्यौर्वष॒ट् स्वाहा॒\\
नमो॒ऽग्निर्वा॒युः सूर्यो॒ वष॒ट् स्वाहा॒\\
नमः॑ प्रा॒णो व्या॒नो॑ऽपा॒नो वष॒ट् स्वाहा॒\\
नमोऽन्नं॑ कृ॒षिर्वृष्टि॒र्वष॒ट् स्वाहा॒\\
नमः॑ पि॒ता पु॒त्रः पौत्रो॒ वष॒ट् स्वाहा॒\\
नमो॒ भूर्भुवः॒ सुव॒र्वष॒ट् स्वाहा॒ नमः॑।

\sect{अथर्वशिरसम्}
\centerline{\scriptsize (तैत्तिरीयब्राह्मणे अष्टकम् – १/प्रश्नः – ५/अनुवाकः – ८)}

इन्द्रो॑ दधी॒चो अ॒स्थभिः॑।
वृ॒त्राण्यप्र॑तिष्कुतः।
ज॒घान॑ नव॒तीर्नव॑।
इ॒च्छन्नश्व॑स्य॒ यच्छिरः॑।
पर्व॑ते॒ष्वप॑श्रितम्।
तद्वि॑दच्छर्य॒णाव॑ति।
अत्राह॒ गोरम॑न्वत।
नाम॒ त्वष्टु॑रपी॒च्यम्᳚।
इ॒त्था च॒न्द्रम॑सो गृ॒हे।
इन्द्र॒मिद्गा॒थिनो॑ बृ॒हत्॥

%1.5.8.2
इन्द्र॑म॒र्केभि॑र॒र्किणः॑।
इन्द्रं॒ वाणी॑रनूषत।
इन्द्र॒ इद्धर्योः॒ सचा᳚।
सम्मि॑श्ल॒ आव॑चो॒ युजा᳚।
इन्द्रो॑ व॒ज्री हि॑र॒ण्ययः॑।
इन्द्रो॑ दी॒र्घाय॒ चक्ष॑से।
आ सूर्यꣳ॑ रोहयद्दि॒वि।
वि गोभि॒रद्रि॑मैरयत्।
इन्द्र॒ वाजे॑षु नो अव।
स॒हस्र॑प्रधनेषु च॥

%1.5.8.3
उ॒ग्र उ॒ग्राभि॑रू॒तिभिः॑।
तमिन्द्रं॑ वाजयामसि।
म॒हे वृ॒त्राय॒ हन्त॑वे।
स वृषा॑ वृष॒भो भु॑वत्।
इन्द्रः॒ स दाम॑ने कृ॒तः।
ओजि॑ष्ठः॒ स बले॑ हि॒तः।
द्यु॒म्नी श्लो॒की स सौ॒म्यः॑।
गि॒रा वज्रो॒ न सम्भृ॑तः।
सब॑लो॒ अन॑पच्युतः।
व॒व॒क्षुरु॒ग्रो अस्तृ॑तः॥

\sect{प्रत्यङ्गिरसम्}
\centerline{\scriptsize (तैत्तिरीयब्राह्मणे अष्टकम् – २/प्रश्नः – ३/अनुवाकः – २)}
%2.4.2.1
चक्षु॑षो हेते॒ मन॑सो हेते।
वाचो॑ हेते॒ ब्रह्म॑णो हेते।
यो मा॑ऽघा॒युर॑भि॒दास॑ति।
तम॑ग्ने मे॒न्या मे॒निं कृ॑णु।
यो मा॒ चक्षु॑षा॒ यो मन॑सा।
यो वा॒चा ब्रह्म॑णाऽघा॒युर॑भि॒दास॑ति।
तया᳚ऽग्ने॒ त्वं मे॒न्या।
अ॒मुम॑मे॒निं कृ॑णु।
यत्किञ्चा॒सौ मन॑सा॒ यच्च॑ वा॒चा।
य॒ज्ञैर्जु॒होति॒ यजु॑षा ह॒विर्भिः॑॥

%2.4.2.2
तन्मृ॒त्युर्निर्\mbox{}ऋ॑त्या संविदा॒नः।
पु॒रादि॒ष्टादाहु॑तीरस्य हन्तु।
या॒तु॒धाना॒ निर्\mbox{}ऋ॑ति॒रादु॒रक्षः॑।
ते अ॑स्य घ्न॒न्त्वनृ॑तेन स॒त्यम्।
इन्द्रे॑षिता॒ आज्य॑मस्य मथ्नन्तु।
मा तथ्समृ॑द्धि॒ यद॒सौ क॒रोति॑।
हन्मि॑ ते॒ऽहं कृ॒तꣳ ह॒विः।
यो मे॑ घो॒रमची॑कृतः।
अपा᳚ञ्चौ त उ॒भौ बा॒हू।
अप॑नह्याम्या॒स्यम्᳚॥

%2.4.2.3
अप॑ नह्यामि ते बा॒हू।
अप॑ नह्याम्या॒स्यम्᳚।
अ॒ग्नेर्दे॒वस्य॒ ब्रह्म॑णा।
सर्वं॑ तेऽवधिषं कृ॒तम्।
पु॒राऽमुष्य॑ वषट्का॒रात्।
य॒ज्ञं दे॒वेषु॑ नस्कृधि।
स्वि॑ष्टम॒स्माकं॑ भूयात्।
माऽस्मान्प्राप॒न्न\-रा॑तयः।
अन्ति॑ दू॒रे स॒तो अ॑ग्ने।
भ्रातृ॑व्यस्याभि॒दास॑तः॥१४॥

%2.4.2.4
व॒ष॒ट्का॒रेण॒ वज्रे॑ण।
कृ॒त्याꣳ ह॑न्मि कृ॒ताम॒हम्।
यो मा॒ नक्तं॒ दिवा॑ सा॒यम्।
प्रा॒तश्चाह्नो॑ नि॒पीय॑ति।
अ॒द्या तमि॑न्द्र॒ वज्रे॑ण।
भातृ॑व्यं पादयामसि।

\centerline{\scriptsize (तैत्तिरीयब्राह्मणे अष्टकम् – २/प्रश्नः – ५/अनुवाकः – १)}

%2.5.1.1
प्रा॒णो र॑क्षति॒ विश्व॒मेज॑त्।
इर्यो॑ भू॒त्वा ब॑हु॒धा ब॒हूनि॑।
स इथ्सर्वं॒ व्या॑नशे।
यो दे॒वो दे॒वेषु॑ वि॒भूर॒न्तः।
आवृ॑दू॒दात् क्षेत्रिय॑ध्व॒गद्वृषा᳚।
तमित्प्रा॒णं मन॒सोप॑ शिक्षत।
अग्रं॑ दे॒वाना॑मि॒दम॑त्तु नो ह॒विः।
मन॑स॒श्चित्ते॒दम्।
भू॒तं भव्यं॑ च गुप्यते।
तद्धि दे॒वेष्व॑ग्रि॒यम्॥

%2.5.1.2
आ न॑ एतु पुरश्च॒रम्।
स॒ह दे॒वैरि॒मꣳ हवम्᳚।
मनः॒ श्रेय॑सिश्रेयसि।
कर्म॑न् य॒ज्ञप॑तिं॒ दध॑त्।
जु॒षतां᳚ मे॒ वागि॒दꣳ ह॒विः।
वि॒राड्दे॒वी पु॒रोहि॑ता।
ह॒व्य॒वाडन॑पायिनी।
यया॑ रू॒पाणि॑ बहु॒धा वद॑न्ति।
पेशाꣳ॑सि दे॒वाः प॑र॒मे ज॒नित्रे᳚।
सा नो॑ वि॒राडन॑पस्फुरन्ती॥

%2.5.1.3
वाग्दे॒वी जु॑षतामि॒दꣳ ह॒विः।
चक्षु॑र्दे॒वानां॒ ज्योति॑र॒मृते॒ न्य॑क्तम्।
अ॒स्य वि॒ज्ञाना॑य बहु॒धा निधी॑यते।
तस्य॑ सु॒म्नम॑शीमहि।
मा नो॑ हासीद्विचक्ष॒णम्।
आयु॒रिन्नः॒ प्रती᳚र्यताम्।
अन॑न्धा॒श्चक्षु॑षा व॒यम्।
जी॒वा ज्योति॑रशीमहि।
सुव॒र्ज्योति॑रु॒तामृतम्᳚।
श्रोत्रे॑ण भ॒द्रमु॒त शृ॑ण्वन्ति स॒त्यम्।
श्रोत्रे॑ण॒ वाचं॑ बहु॒धोद्यमा॑नाम्।
श्रोत्रे॑ण॒ मोद॑श्च॒ मह॑श्च श्रूयते।
श्रोत्रे॑ण॒ सर्वा॒ दिश॒ आ शृ॑णोमि।
येन॒ प्राच्या॑ उ॒त द॑क्षि॒णा।
प्र॒तीच्यै॑ दि॒शः शृ॒ण्वन्त्यु॑त्त॒रात्।
तदिच्छ्रोत्रं॑ बहु॒धोद्यमा॑नम्।
अ॒रान्न ने॒मिः परि॒ सर्वं॑ बभूव॥

\centerline{\scriptsize (तैत्तिरीयब्राह्मणे अष्टकम् – २/प्रश्नः – ७/अनुवाकः – ७)}
सि॒ꣳ॒हे व्या॒घ्र उ॒त या पृदा॑कौ।
त्विषि॑र॒ग्नौ ब्रा᳚ह्म॒णे सूर्ये॒ या।
इन्द्रं॒ या दे॒वी सु॒भगा॑ ज॒जान॑।
सा न॒ आग॒न्वर्च॑सा संविदा॒ना।
या रा॑ज॒न्ये॑ दुन्दु॒भावाय॑तायाम्।
अश्व॑स्य॒ क्रन्द्ये॒ पुरु॑षस्य मा॒यौ।
इन्द्रं॒ या दे॒वी सु॒भगा॑ ज॒जान॑।
सा न॒ आग॒न्वर्च॑सा संविदा॒ना।
या ह॒स्तिनि॑ द्वी॒पिनि॒ या हिर॑ण्ये।
त्विषि॒रश्वे॑षु॒ पुरु॑षेषु॒ गोषु॑॥

इन्द्रं॒ या दे॒वी सु॒भगा॑ ज॒जान॑।
सा न॒ आग॒न्वर्च॑सा संविदा॒ना।
रथे॑ अ॒क्षेषु॑ वृष॒भस्य॒ वाजे᳚।
वाते॑ प॒र्जन्ये॒ वरु॑णस्य॒ शुष्मे᳚।
इन्द्रं॒ या दे॒वी सु॒भगा॑ ज॒जान॑।
सा न॒ आग॒न्वर्च॑सा संविदा॒ना।
राड॑सि वि॒राड॑सि।
स॒म्राड॑सि स्व॒राड॑सि।
इन्द्रा॑य त्वा॒ तेज॑स्वते॒ तेज॑स्वन्तꣴ श्रीणामि।
इन्द्रा॑य॒ त्वौज॑स्वत॒ ओज॑स्वन्तꣴ श्रीणामि॥

इन्द्रा॑य त्वा॒ पय॑स्वते॒ पय॑स्वन्तꣴ श्रीणामि।
इन्द्रा॑य॒ त्वाऽऽयु॑ष्मत॒ आयु॑ष्मन्तꣴ श्रीणामि।
तेजो॑ऽसि।
तत्ते॒ प्र य॑च्छामि।
तेज॑स्वदस्तु मे॒ मुखम्᳚।
तेज॑स्व॒च्छिरो॑ अस्तु मे।
तेज॑स्वान् वि॒श्वतः॑ प्र॒त्यङ्ङ्।
तेज॑सा॒ सं पि॑पृग्धि मा।
ओजो॑ऽसि।
तत्ते॒ प्र य॑च्छामि॥

ओज॑स्वदस्तु मे॒ मुखम्᳚।
ओज॑स्व॒च्छिरो॑ अस्तु मे।
ओज॑स्वान् वि॒श्वतः॑ प्र॒त्यङ्ङ्।
ओज॑सा॒ सं पि॑पृग्धि मा।
पयो॑ऽसि।
तत्ते॒ प्र य॑च्छामि।
पय॑स्वदस्तु मे॒ मुखम्᳚।
पय॑स्व॒च्छिरो॑ अस्तु मे।
पय॑स्वान् वि॒श्वतः॑ प्र॒त्यङ्ङ्।
पय॑सा॒ सं पि॑पृग्धि मा॥

आयु॑रसि।
तत्ते॒ प्र य॑च्छामि।
आयु॑ष्मदस्तु मे॒ मुखम्᳚।
आयु॑ष्म॒च्छिरो॑ अस्तु मे।
आयु॑ष्मान् वि॒श्वतः॑ प्र॒त्यङ्ङ्।
आयु॑षा॒ सं पि॑पृग्धि मा।
इ॒मम॑ग्न॒ आयु॑षे॒ वर्च॑से कृधि।
प्रि॒यꣳ रेतो॑ वरुण सोम राजन्।
मा॒तेवा᳚स्मा अदिते॒ शर्म॑ यच्छ।
विश्वे॑ देवा॒ जर॑दष्टि॒र्यथाऽस॑त्॥

आयु॑रसि वि॒श्वायु॑रसि।
स॒र्वायु॑रसि॒ सर्व॒मायु॑रसि।
यतो॒ वातो॒ मनो॑जवाः।
यतः॒ क्षर॑न्ति॒ सिन्ध॑वः।
तासां᳚ त्वा॒ सर्वा॑साꣳ रु॒चा।
अ॒भिषि॑ञ्चामि॒ वर्च॑सा।
स॒मु॒द्र इ॑वासि ग॒ह्मना᳚।
सोम॑ इवा॒स्यदा᳚भ्यः।
अ॒ग्निरि॑व वि॒श्वतः॑ प्र॒त्यङ्ङ्।
सूर्य॑ इव॒ ज्योति॑षा वि॒भूः॥


अ॒पां यो द्रव॑णे॒ रसः॑।
तम॒हम॒स्मा आ॑मुष्याय॒णाय॑।
तेज॑से ब्रह्मवर्च॒साय॑ गृह्णामि।
अ॒पां य ऊ॒र्मौ रसः॑।
तम॒हम॒स्मा आ॑मुष्याय॒णाय॑।
ओज॑से वी॒र्या॑य गृह्णामि।
अ॒पां यो म॑ध्य॒तो रसः॑।
तम॒हम॒स्मा आ॑मुष्याय॒णाय॑।
[पुष्ट्यै᳚ प्र॒जन॑नाय गृह्णामि।
अ॒पां यो य॒ज्ञियो॒ रसः॑।
तम॒हम॒स्मा आ॑मुष्याय॒णाय॑।]
आयु॑षे दीर्घायु॒त्वाय॑ गृह्णामि॥

\centerline{\scriptsize (तैत्तिरीयब्राह्मणे अष्टकम् – २/प्रश्नः – ८/अनुवाकः – ८)}
%2.8.8.1
अ॒हम॑स्मि प्रथम॒जा ऋ॒तस्य॑।
पूर्वं॑ दे॒वेभ्यो॑ अ॒मृत॑स्य॒ नाभिः॑।
यो मा॒ ददा॑ति॒ स इदे॒व माऽऽवाः᳚।
अ॒हमन्न॒मन्न॑म॒दन्त॑मद्मि।
पूर्व॑म॒ग्नेरपि॑ दह॒त्यन्नम्᳚।
य॒त्तौ हा॑साते अहमुत्त॒रेषु॑।
व्यात्त॑मस्य प॒शवः॑ सु॒जम्भम्᳚।
पश्य॑न्ति॒ धीराः॒ प्रच॑रन्ति॒ पाकाः᳚।
जहा᳚म्य॒न्यन्न ज॑हाम्य॒न्यम्।
अ॒हमन्नं॒ वश॒मिच्च॑रामि॥

%2.8.8.2
स॒मा॒नमर्थं॒ पर्ये॑मि भु॒ञ्जत्।
को मामन्नं॑ मनु॒ष्यो॑ दयेत।
परा॑के॒ अन्नं॒ निहि॑तं लो॒क ए॒तत्।
विश्वै᳚र्दे॒वैः पि॒तृभि॑र्गु॒प्तमन्नम्᳚।
यद॒द्यते॑ लु॒प्यते॒ यत्प॑रो॒प्यते᳚।
श॒त॒त॒मी सा त॒नूर्मे॑ बभूव।
म॒हान्तौ॑ च॒रू स॑कृद्दु॒ग्धेन॑ पप्रौ।
दिवं॑ च॒ पृश्ञि॑ पृथि॒वीं च॑ सा॒कम्।
तथ्स॒म्पिब॑न्तो॒ न मि॑नन्ति वे॒धसः॑।
नैतद्भूयो॒ भव॑ति॒ नो कनी॑यः॥

%2.8.8.3
अन्नं॑ प्रा॒णमन्न॑मपा॒नमा॑हुः।
अन्नं॑ मृ॒त्युं तमु॑ जी॒वातु॑माहुः।
अन्नं॑ ब्र॒ह्माणो॑ ज॒रसं॑  वदन्ति।
अन्न॑माहुः प्र॒जन॑नं प्र॒जाना᳚म्।
मोघ॒मन्नं॑ विन्दते॒ अप्र॑चेताः।
स॒त्यं ब्र॑वीमि व॒ध इथ्स तस्य॑।
नार्य॒मणं॒ पुष्य॑ति॒ नो सखा॑यम्।
केव॑लाघो भवति केवला॒दी।
अ॒हं मे॒घः स्त॒नय॒न्वर्\mbox{}ष॑न्नस्मि।
माम॑दन्त्य॒हम॑द्म्य॒न्यान्॥

%2.8.8.4
अ॒हꣳ सद॒मृतो॑ भवामि।
मदा॑दि॒त्या अधि॒ सर्वे॑ तपन्ति।

\dnsub{वाक्सूक्तम्}
\centerline{\scriptsize (तैत्तिरीयब्राह्मणे अष्टकम् – २/प्रश्नः – ४/अनुवाकः – ५९–६०)}

दे॒वीं वाच॑मजनयन्त दे॒वाः।
तां वि॒श्वरू॑पाः प॒शवो॑ वदन्ति।
सा नो॑ म॒न्द्रेष॒मूर्जं॒ दुहा॑ना।
धे॒नुर्वाग॒स्मानुप॒ सुष्टु॒तैतु॑॥६०॥


यद्वाग्वद॑न्त्यविचेत॒नानि॑।
राष्ट्री॑ दे॒वानां᳚ निष॒साद॑ म॒न्द्रा।
चत॑स्र॒ ऊर्जं॑ दुदुहे॒ पयाꣳ॑सि।
क्व॑ स्विदस्याः पर॒मं ज॑गाम।

\closesection

अ॒न॒न्तामन्ता॒दधि॒ निर्मि॑तां म॒हीम्।
यस्यां᳚ दे॒वा अ॑दधु॒र्भोज॑नानि।
एका᳚क्षरां द्वि॒पदा॒ꣳ॒ षट्प॑दां च।
वाचं॑ दे॒वा उप॑ जीवन्ति॒ विश्वे᳚।
वाचं॑ दे॒वा उप॑ जीवन्ति॒ विश्वे᳚।
वाचं॑ गन्ध॒र्वाः प॒शवो॑ मनु॒ष्याः᳚।
वा॒चीमा विश्वा॒ भुव॑ना॒न्यर्पि॑ता॥

%2.8.8.5
सा नो॒ हवं॑ जुषता॒मिन्द्र॑पत्नी।
वाग॒क्षरं॑ प्रथम॒जा ऋ॒तस्य॑।
वेदा॑नां मा॒ताऽमृत॑स्य॒ नाभिः॑।
सा नो॑ जुषा॒णोप॑ य॒ज्ञमागा᳚त्।
अव॑न्ती दे॒वी सु॒हवा॑ मे अस्तु।
यामृष॑यो मन्त्र॒कृतो॑ मनी॒षिणः॑।
अ॒न्वैच्छं॑ दे॒वास्तप॑सा॒ श्रमे॑ण।
तान्दे॒वीं वाचꣳ॑ ह॒विषा॑ यजामहे।
सा नो॑ दधातु सुकृ॒तस्य॑ लो॒के।
च॒त्वारि॒ वाक्परि॑मिता प॒दानि॑॥

%2.8.8.6
तानि॑ विदुर्ब्राह्म॒णा ये म॑नी॒षिणः॑।
गुहा॒ त्रीणि॒ निहि॑ता॒ नेङ्ग॑यन्ति।
तु॒रीयं॑ वा॒चो म॑नु॒ष्या॑ वदन्ति।

\dnsub{श्रद्धा सूक्तम्}

श्र॒द्धया॒ऽग्निः समि॑ध्यते।
श्र॒द्धया॑ विन्दते ह॒विः।
श्र॒द्धां भग॑स्य मू॒र्धनि॑।
वच॒सा वे॑दयामसि।
प्रि॒यꣴ श्र॑द्धे॒ दद॑तः।
प्रि॒यꣴ श्र॑द्धे॒ दिदा॑सतः।
प्रि॒यं भो॒जेषु॒ यज्व॑सु॥

%2.8.8.7
इ॒दं म॑ उदि॒तं कृ॑धि।
यथा॑ दे॒वा असु॑रेषु।
श्र॒द्धामु॒ग्रेषु॑ चक्रि॒रे।
ए॒वं भो॒जेषु॒ यज्व॑सु।
अ॒स्माक॑मुदि॒तं कृ॑धि।
श्र॒द्धां दे॑वा॒ यज॑मानाः।
वा॒युगो॑पा॒ उपा॑सते।
श्र॒द्धाꣳ हृ॑द॒य्य॑याऽऽकू᳚त्या।
श्र॒द्धया॑ हूयते ह॒विः।
श्र॒द्धां प्रा॒तर्\mbox{}ह॑वामहे॥

%2.8.8.8
श्र॒द्धां म॒ध्यन्दि॑नं॒ परि॑।
श्र॒द्धाꣳ सूर्य॑स्य नि॒म्रुचि॑।
श्रद्धे॒ श्रद्धा॑पये॒ह मा᳚।
श्र॒द्धा दे॒वानधि॑ वस्ते।
श्र॒द्धा विश्व॑मि॒दं जग॑त्।
श्र॒द्धां काम॑स्य मा॒तरम्᳚।
ह॒विषा॑ वर्धयामसि।

\dnsub{ब्रह्मा सूक्तम्}
ब्रह्म॑ जज्ञा॒नं प्र॑थ॒मं पु॒रस्ता᳚त्।
वि सी॑म॒तः सु॒रुचो॑ वे॒न आ॑वः।
स बु॒ध्निया॑ उप॒ मा अ॑स्य वि॒ष्ठाः॥६६॥

%2.8.8.9
स॒तश्च॒ योनि॒मस॑तश्च॒ विवः॑।
पि॒ता वि॒राजा॑मृष॒भो र॑यी॒णाम्।
अ॒न्तरि॑क्षं वि॒श्वरू॑प॒ आवि॑वेश।
तम॒र्कैर॒भ्य॑र्चन्ति व॒थ्सम्।
ब्रह्म॒ सन्तं॒ ब्रह्म॑णा व॒र्धय॑न्तः।
ब्रह्म॑ दे॒वान॑जनयत्।
ब्रह्म॒ विश्व॑मि॒दं जग॑त्।
ब्रह्म॑णः क्ष॒त्रं निर्मि॑तम्।
ब्रह्म॑ ब्राह्म॒ण आ॒त्मना᳚।
अ॒न्तर॑स्मिन्नि॒मे लो॒काः॥६७॥

%2.8.8.10
अ॒न्तर्विश्व॑मि॒दं जग॑त्।
ब्रह्मै॒व भू॒तानां॒ ज्येष्ठम्᳚।
तेन॒ को॑ऽर्\mbox{}हति॒ स्पर्धि॑तुम्।
ब्रह्म॑न्दे॒वास्त्रय॑स्त्रिꣳशत्।
ब्रह्म॑न्निन्द्रप्रजाप॒ती।
ब्रह्म॑न् ह॒ विश्वा॑ भू॒तानि॑।
ना॒वीवा॒न्तः स॒माहि॑ता।
चत॑स्र॒ आशाः॒ प्रच॑रन्त्व॒ग्नयः॑।
इ॒मं नो॑ य॒ज्ञं न॑यतु प्रजा॒नन्।
घृ॒तं पिन्व॑न्न॒जरꣳ॑ सु॒वीरम्᳚॥

%2.8.8.12
ब्रह्म॑ स॒मिद्भ॑व॒त्याहु॑तीनाम्।

\dnsub{गो-सूक्तम्}
आ गावो॑ अग्मन्नु॒त भ॒द्रम॑क्रन्।
सीद॑न्तु गो॒ष्ठे र॒णय॑न्त्व॒स्मे।
प्र॒जाव॑तीः पुरु॒रूपा॑ इ॒ह स्युः।
इन्द्रा॑य पू॒र्वीरु॒षसो॒ दुहा॑नाः।
इन्द्रो॒ यज्व॑ने पृण॒ते च॑ शिक्षति।
उपेद्द॑दाति॒ न स्वं मु॑षायति।
भूयो॑भूयो र॒यिमिद॑स्य व॒र्धयन्॑।
अभि॑न्ने खि॒ल्ले नि द॑धाति देव॒युम्।

%2.4.6.9
न ता न॑शन्ति॒ न द॑भाति॒ तस्क॑रः।
नैना॑ अमि॒त्रो व्यथि॒राद॑धर्‌षति।
दे॒वाꣴश्च॒ याभि॒र्यज॑ते॒ ददा॑ति च।
ज्योगित्ताभिः॑ सचते॒ गोप॑तिः स॒ह।
न ता अर्वा॑ रे॒णुक॑काटो अश्ञुते।
न सꣴ॑स्कृत॒त्रमुप॑ यन्ति॒ ता अ॒भि।
उ॒रु॒गा॒यमभ॑यं॒ तस्य॒ ता अनु॑।
गावो॒ मर्त्य॑स्य॒ वि च॑रन्ति॒ यज्व॑नः॥

%2.8.8.13
गावो॒ भगो॒ गाव॒ इन्द्रो॑ मे अच्छात्।
गावः॒ सोम॑स्य प्रथ॒मस्य॑ भ॒क्षः।
इ॒मा या गावः॒ सज॑नास॒ इन्द्रः॑।
इ॒च्छामीद्धृ॒दा मन॑सा चि॒दिन्द्रम्᳚।
यू॒यं गा॑वो मेदयथा कृ॒शञ्चि॑त्।
अ॒श्ली॒लञ्चि॑त्कृणुथा सु॒प्रती॑कम्।
भ॒द्रं गृ॒हं कृ॑णुथ भद्रवाचः।
बृ॒हद्वो॒ वय॑ उच्यते स॒भासु॑।
प्र॒जाव॑तीः सू॒यव॑सꣳ रि॒शन्तीः᳚।
शु॒द्धा अ॒पः सु॑प्रपा॒णे पिब॑न्तीः।
मा वः॑ स्ते॒न ई॑शत॒ माऽघशꣳ॑सः।
परि॑ वो हे॒ती रु॒द्रस्य॑ वृञ्ज्यात्।
उपे॒दमु॑प॒पर्च॑नम्।
आ॒सु गोषूप॑पृच्यताम्।
उप॑र्\mbox{}ष॒भस्य॒ रेत॑सि।
उपे᳚न्द्र॒ तव॑ वी॒र्ये᳚॥

%2.8.9.1
ता सू᳚र्याचन्द्र॒मसा॑ विश्व॒भृत्त॑मा म॒हत्।
तेजो॒ वसु॑मद्राजतो दि॒वि।
सामा᳚त्माना चरतः सामचा॒रिणा᳚।
ययो᳚र्व्र॒तं न म॒मे जातु॑ दे॒वयोः᳚।
उ॒भावन्तौ॒ परि॑ यात॒ अर्म्या᳚।
दि॒वो न र॒श्मीꣴस्त॑नु॒तो व्य॑र्ण॒वे।
उ॒भा भु॑व॒न्ती भुव॑ना क॒विक्र॑तू।
सूर्या॒ न च॒न्द्रा च॑रतो ह॒ताम॑ती।
पती᳚ द्यु॒मद्वि॑श्व॒विदा॑ उ॒भा दि॒वः।
सूर्या॑ उ॒भा च॒न्द्रम॑सा विचक्ष॒णा॥

%2.8.9.2
वि॒श्ववा॑रा वरिवो॒भा वरे᳚ण्या।
ता नो॑ऽवतं मति॒मन्ता॒ महि॑व्रता।
वि॒श्व॒वप॑री प्र॒तर॑णा तर॒न्ता।
सु॒व॒र्विदा॑ दृ॒शये॒ भूरि॑रश्मी।
सूर्या॒ हि च॒न्द्रा वसु॑ त्वे॒षद॑र्\mbox{}शता।
म॒न॒स्विनो॒भानु॑चर॒तोनु॒ सन्दिवम्᳚।
अ॒स्य श्रवो॑ न॒द्यः॑ स॒प्त बि॑भ्रति।
द्यावा॒ क्षामा॑ पृथि॒वी द॑र्\mbox{}श॒तं वपुः॑।
अ॒स्मे सू᳚र्याचन्द्र॒मसा॑ऽभि॒चक्षे᳚।
श्र॒द्धेकमि॑न्द्र चरतो विचर्तु॒रम्॥

%2.8.9.3
पू॒र्वा॒प॒रं च॑रतो मा॒ययै॒तौ।
शिशू॒ क्रीड॑न्तौ॒ परि॑ यातो अध्व॒रम्।
विश्वा᳚न्य॒न्यो भुव॑नाऽभि॒ चष्टे᳚।
ऋ॒तून॒न्यो वि॒दध॑ज्जायते॒ पुनः॑।

हिर॑ण्यवर्णाः॒ शुच॑यः पाव॒का यासु॑ जा॒तः क॒श्यपो॒ यास्विन्द्रः॑।
अ॒ग्निं या गर्भं॑ दधि॒रे विरू॑पा॒स्ता न॒ आपः॒ शꣴ स्यो॒ना भ॑वन्तु॥ 
यासा॒ꣳ॒ राजा॒ वरु॑णो॒ याति॒ मध्ये॑ सत्यानृ॒ते अ॑व॒पश्यं॒ जना॑नाम्।
म॒धु॒श्चुतः॒ शुच॑यो॒ याः पा॑व॒कास्ता न॒ आपः॒ शꣴ स्यो॒ना भ॑वन्तु॥ 
यासां᳚ दे॒वा दि॒वि कृ॒ण्वन्ति॑ भ॒क्षं या अ॒न्तरि॑क्षे बहु॒धा भव॑न्ति।
याः पृ॑थि॒वीं पय॑सो॒न्दन्ति॑ शु॒क्रास्ता न॒ आपः॒ शꣴ स्यो॒ना भ॑वन्तु॥ 
शि॒वेन॑ मा॒ चक्षु॑षा पश्यताऽऽपः शि॒वया॑ त॒नुवोप॑ स्पृशत॒ त्वचं॑ मे।
सर्वाꣳ॑ अ॒ग्नीꣳ र॑फ्सु॒षदो॑ हुवे वो॒ मयि॒ वर्चो॒ बल॒मोजो॒ नि ध॑त्त॥

आपो॑ भ॒द्रा घृ॒तमिदाप॑ आसुर॒ग्नीषोमौ॑ बिभ्र॒त्याप॒ इत्ताः।
ती॒व्रो रसो॑ मधु॒पृचा॑मरङ्ग॒म आ मा᳚ प्रा॒णेन॑ स॒ह वर्च॑ सागन्॥

आदित्प॑श्याम्यु॒त वा॑ शृणो॒म्यामा॒ घोषो॑ गच्छति॒ वाङ् न॑ आसाम्।
मन्ये॑भेजा॒नो अ॒मृत॑स्य॒ तर्\mbox{}हि॒ हिर॑ण्यवर्णा॒ अतृ॑पं य॒दावः॑॥


नास॑दासी॒न्नो सदा॑सीत्त॒दानी᳚म्।
नासी॒द्रजो॒ नो व्यो॑मा प॒रो यत्।
किमाव॑रीवः॒ कुह॒ कस्य॒ शर्मन्॑॥

%2.8.9.4
अम्भः॒ किमा॑सी॒द्गह॑नं गभी॒रम्।
न मृ॒त्युर॒मृतं॒ तर्\mbox{}हि॒ न।
रात्रि॑या॒ अह्न॑ आसीत्प्रके॒तः।
आनी॑दवा॒तꣴ स्व॒धया॒ तदेकम्᳚।
तस्मा᳚द्धा॒न्यं न प॒रः किञ्च॒नास॑।
तम॑ आसी॒त्तम॑सा गू॒ढमग्रे᳚ प्रके॒तम्।
स॒लि॒लꣳ सर्व॑मा इ॒दम्।
तु॒च्छेना॒भ्वपि॑हितं॒ यदासी᳚त्।
तम॑स॒स्तन्म॑हि॒ना जा॑य॒तैकम्᳚।
काम॒स्तदग्रे॒ सम॑वर्त॒ताधि॑॥

%2.8.9.5
मन॑सो॒ रेतः॑ प्रथ॒मं यदासी᳚त्।
स॒तो बन्धु॒मस॑ति॒ निर॑विन्दन्।
हृ॒दि प्र॒तीष्या॑ क॒वयो॑ मनी॒षा।
ति॒र॒श्चीनो॒ वित॑तो र॒श्मिरे॑षाम्।
अ॒धः स्वि॑दा॒सी(३)दु॒परि॑ स्विदासी(३)त्।
रे॒तो॒धा आ॑सन्महि॒मान॑ आसन्।
स्व॒धा अ॒वस्ता॒त्प्रय॑तिः प॒रस्ता᳚त्।
को अ॒द्धा वे॑द॒ क इ॒ह प्र वो॑चत्।
कुत॒ आजा॑ता॒ कुत॑ इ॒यं विसृ॑ष्टिः।
अ॒र्वाग्दे॒वा अ॒स्य वि॒सर्ज॑नाय॥

%2.8.9.6
अथा॒ को वे॑द॒ यत॑ आब॒भूव॑।
इ॒यं विसृ॑ष्टि॒र्यत॑ आब॒भूव॑।
यदि॑ वा द॒धे यदि॑ वा॒ न।
यो अ॒स्याध्य॑क्षः पर॒मे व्यो॑मन्।
सो अ॒ङ्ग वे॑द॒ यदि॑ वा॒ न वेद॑।
किꣴस्वि॒द्वन॒ङ्क उ॒ स वृ॒क्ष आ॑सीत्।
यतो॒ द्यावा॑पृथि॒वी नि॑ष्टत॒क्षुः।
मनी॑षिणो॒ मन॑सा पृ॒च्छतेदु॒तत्।
यद॒ध्यति॑ष्ठ॒द्भुव॑नानि धा॒रयन्॑।
ब्रह्म॒ वनं॒ ब्रह्म॒ स वृ॒क्ष आ॑सीत्॥


%2.8.9.7
यतो॒ द्यावा॑पृथि॒वी नि॑ष्टत॒क्षुः।
मनी॑षिणो॒ मन॑सा॒ विब्र॑वीमि वः।
ब्रह्मा॒ध्यति॑ष्ठ॒द्भुव॑नानि धा॒रयन्॑।
प्रा॒तर॒ग्निं प्रा॒तरिन्द्रꣳ॑ हवामहे।
प्रा॒तर्मि॒त्रावरु॑णा प्रा॒तर॒श्विना᳚।
प्रा॒तर्भगं॑ पू॒षणं॒ ब्रह्म॑ण॒स्पतिम्᳚।
प्रा॒तः सोम॑मु॒त रु॒द्रꣳ हु॑वेम।
प्रा॒त॒र्जितं॒ भग॑मु॒ग्रꣳ हु॑वेम।
व॒यं पु॒त्रमदि॑ते॒र्यो वि॑ध॒र्ता।
आ॒ध्रश्चि॒द्यं मन्य॑मानस्तु॒रश्चि॑त्॥

%2.8.9.8
राजा॑ चि॒द्यं भगं॑ भ॒क्षीत्याह॑।
भग॒ प्रणे॑त॒र्भग॒ सत्य॑राधः।
भगे॒मां धिय॒मुद॑व॒ दद॑न्नः।
भग॒ प्र णो॑ जनय॒ गोभि॒रश्वैः᳚।
भग॒ प्र नृभि॑र्नृ॒वन्तः॑ स्याम।
उ॒तेदानीं॒ भग॑वन्तः स्याम।
उ॒त प्रपि॒त्व उ॒त मध्ये॒ अह्ना᳚म्।
उ॒तोदि॑ता मघव॒न्थ्सूर्य॑स्य।
व॒यं दे॒वानाꣳ॑ सुम॒तौ स्या॑म।
भग॑ ए॒व भग॑वाꣳ अस्तु देवाः॥

%2.8.9.9
तेन॑ व॒यं भग॑वन्तः स्याम।
तं त्वा॑ भग॒ सर्व॒ इज्जो॑हवीमि।
स नो॑ भग पुर ए॒ता भ॑वे॒ह।
सम॑ध्व॒रायो॒षसो॑ नमन्त।
द॒धि॒क्रावे॑व॒ शुच॑ये प॒दाय॑।
अ॒र्वा॒ची॒नं व॑सु॒विदं॒ भगं॑ नः।
रथ॑मि॒वाश्वा॑ वा॒जिन॒ आव॑हन्तु।
अश्वा॑वती॒र्गोम॑तीर्न उ॒षासः॑।
वी॒रव॑तीः॒ सद॑मुच्छन्तु भ॒द्राः।
घृ॒तं दुहा॑ना वि॒श्वतः॒ प्रपी॑नाः।
यू॒यं पा॑त स्व॒स्तिभिः॒ सदा॑ नः॥

\sect{नक्षत्रसूक्तम्}
\centerline{\scriptsize(तैत्तिरीय ब्राह्मणे अष्टकम् -- ३/प्रश्नः -- १)}\mbox{}\\[-2em]
\centerline{\scriptsize(तैत्तिरीय संहितायां काण्डम् -- ३/प्रपाठकः -- ५/अनुवाकः --१)}

अ॒ग्निर्नः॑ पातु॒ कृत्ति॑काः।
नक्ष॑त्रं दे॒वमि॑न्द्रि॒यम्।
इ॒दमा॑सां विचक्ष॒णम्।
ह॒विरा॒सं जु॑होतन।
यस्य॒ भान्ति॑ र॒श्मयो॒ यस्य॑ के॒तवः॑।
यस्ये॒मा विश्वा॒ भुव॑नानि॒ सर्वा᳚।
स कृत्ति॑काभि\-र॒भिसं॒वसा॑नः।
अ॒ग्निर्नो॑ दे॒वः सु॑वि॒ते द॑धातु॥१॥ 

प्र॒जाप॑ते रोहि॒णी वे॑तु॒ पत्नी᳚।
वि॒श्वरू॑पा बृह॒ती चि॒त्रभा॑नुः।
सा नो॑ य॒ज्ञस्य॑ सुवि॒ते द॑धातु।
यथा॒ जीवे॑म श॒रदः॒ सवी॑राः।
रो॒हि॒णी दे॒व्युद॑गात्पु॒रस्ता᳚त्।
विश्वा॑ रू॒पाणि॑ प्रति॒मोद॑माना।
प्र॒जाप॑तिꣳ ह॒विषा॑ व॒र्धय॑न्ती।
प्रि॒या दे॒वाना॒मुप॑यातु य॒ज्ञम्॥२॥ 

सोमो॒ राजा॑ मृगशी॒र्‌षेण॒ आगन्॑।
शि॒वं नक्ष॑त्रं प्रि॒यम॑स्य॒ धाम॑।
आ॒प्याय॑मानो बहु॒धा जने॑षु।
रेतः॑ प्र॒जां यज॑माने दधातु।
यत्ते॒ नक्ष॑त्रं मृगशी॒र्‌षमस्ति॑।
प्रि॒यꣳ रा॑जन् प्रि॒यत॑मं प्रि॒याणा᳚म्।
तस्मै॑ ते सोम ह॒विषा॑ विधेम।
शं न॑ एधि द्वि॒पदे॒ शं चतु॑ष्पदे॥३॥ 

आ॒र्द्रया॑ रु॒द्रः प्रथ॑मा न एति।
श्रेष्ठो॑ दे॒वानां॒ पति॑रघ्नि॒याना᳚म्।
नक्ष॑त्रमस्य ह॒विषा॑ विधेम।
मा नः॑ प्र॒जाꣳ री॑रिष॒न्मोत वी॒रान्।
हे॒ती रु॒द्रस्य॒ परि॑ णो वृणक्तु।
आ॒र्द्रा नक्ष॑त्रं जुषताꣳ ह॒विर्नः॑।
प्र॒मु॒ञ्चमा॑नौ दुरि॒तानि॒ विश्वा᳚।
अपा॒घशꣳ॑ सन्नुदता॒मरा॑तिम्॥४॥ 

पुन॑र्नो दे॒व्यदि॑तिः स्पृणोतु।
पुन॑र्वसू नः॒ पुन॒रेतां᳚ य॒ज्ञम्।
पुन॑र्नो दे॒वा अ॒भिय॑न्तु॒ सर्वे᳚।
पुनः॑ पुनर्वो ह॒विषा॑ यजामः।
ए॒वा न दे॒व्यदि॑तिरन॒र्वा।
विश्व॑स्य भ॒र्त्री जग॑तः प्रति॒ष्ठा।
पुन॑र्वसू ह॒विषा॑ व॒र्धय॑न्ती।
प्रि॒यं दे॒वाना॒मप्ये॑तु॒ पाथः॑॥५॥ 

बृह॒स्पतिः॑ प्रथ॒मं जाय॑मानः।
ति॒ष्यं॑ नक्ष॑त्रम॒भि सम्ब॑भूव।
श्रेष्ठो॑ दे॒वानां॒ पृत॑नासु जि॒ष्णुः।
दि॒शोऽनु॒ सर्वा॒ अभ॑यं नो अस्तु।
ति॒ष्यः॑ पु॒रस्ता॑दु॒त म॑ध्य॒तो नः॑।
बृह॒स्पति॑र्नः॒ परि॑ पातु प॒श्चात्।
बाधे॑तां॒ द्वेषो॒ अभ॑यं कृणुताम्।
सु॒वीर्य॑स्य॒ पत॑यः स्याम॥६॥ 

इ॒दꣳ स॒र्पेभ्यो॑ ह॒विर॑स्तु॒ जुष्टम्᳚।
आ॒श्रे॒षा येषा॑मनु॒यन्ति॒ चेतः॑।
ये अ॒न्तरि॑क्षं पृथि॒वीं क्षि॒यन्ति॑।
ते नः॑ स॒र्पासो॒ हव॒माग॑मिष्ठाः।
ये रो॑च॒ने सूर्य॒स्यापि॑ स॒र्पाः।
ये दिवं॑ दे॒वीमनु॑ स॒ञ्चर॑न्ति।
येषा॑माश्रे॒षा अ॑नु॒यन्ति॒ कामम्᳚।
तेभ्यः॑ स॒र्पेभ्यो॒ मधु॑मज्जुहोमि॥७॥ 

उप॑हूताः पि॒तरो॒ ये म॒घासु॑।
मनो॑जवसः सु॒कृतः॑ सुकृ॒त्याः।
ते नो॒ नक्ष॑त्रे॒ हव॒माग॑मिष्ठाः।
स्व॒धाभि॑र्य॒ज्ञं प्रय॑तं जुषन्ताम्।
ये अ॑ग्निद॒ग्धा येऽन॑ग्निदग्धाः।
ये॑ऽमुं लो॒कं पि॒तरः॑ क्षि॒यन्ति॑।
याꣴश्च॑ वि॒द्म याꣳ उ॑ च॒ न प्र॑वि॒द्म।
म॒घासु॑ य॒ज्ञꣳ सुकृ॑तं जुषन्ताम्॥८॥ 

गवां॒ पतिः॒ फल्गु॑नीनामसि॒ त्वम्।
तद॑र्यमन्वरुणमित्र॒ चारु॑।
तं त्वा॑ व॒यꣳ स॑नि॒तारꣳ॑ सनी॒नाम्।
जी॒वा जीव॑न्त॒मुप॒ संवि॑शेम।
येने॒मा विश्वा॒ भुव॑नानि॒ सञ्जि॑ता।
यस्य॑ दे॒वा अ॑नु सं॒ यन्ति॒ चेतः॑।
अ॒र्य॒मा राजा॒\-ऽजर॒स्तुवि॑ष्मान्।
फल्गु॑नीनामृष॒भो रो॑रवीति॥९॥ 

श्रेष्ठो॑ दे॒वानां᳚ भगवो भगासि।
तत्त्वा॑ विदुः॒ फल्गु॑नी॒स्तस्य॑ वित्तात्।
अ॒स्मभ्यं॑ क्ष॒त्रम॒जरꣳ॑ सु॒वीर्यम्᳚।
गोम॒दश्व॑व॒दुप॒ सन्नु॑\-दे॒ह।
भगो॑ ह दा॒ता भग॒ इत्प्र॑दा॒ता।
भगो॑ दे॒वीः फल्गु॑नी॒रा वि॑वेश।
भग॒स्येत्तं प्र॑स॒वं ग॑मेम।
यत्र॑ दे॒वैः स॑ध॒मादं॑ मदेम॥१०॥ 

आया॑तु दे॒वः स॑वि॒तोप॑यातु।
हि॒र॒ण्यये॑न सु॒वृता॒ रथे॑न।
वह॒न्॒ हस्तꣳ॑ सु॒भगं॑ विद्म॒नाप॑सम्।
प्र॒यच्छ॑न्तं॒ पपु॑रिं॒ पुण्य॒मच्छ॑।
हस्तः॒ प्रय॑च्छत्व॒मृतं॒ वसी॑यः।
दक्षि॑णेन॒ प्रति॑गृभ्णीम एनत्।
दा॒तार॑म॒द्य स॑वि॒ता वि॑देय।
यो नो॒ हस्ता॑य प्रसु॒वाति॑ य॒ज्ञम्॥११॥ 

त्वष्टा॒ नक्ष॑त्रम॒भ्ये॑ति चि॒त्राम्।
सु॒भꣳ स॑सं युव॒तिꣳ रोच॑मानाम्।
नि॒वे॒शय॑न्न॒\-मृता॒न्मर्त्याꣴ॑श्च।
रू॒पाणि॑ पि॒ꣳ॒शन् भुव॑नानि॒ विश्वा᳚।
तन्न॒स्त्वष्टा॒ तदु॑ चि॒त्रा विच॑ष्टाम्।
तन्नक्ष॑त्रं भूरि॒दा अ॑स्तु॒ मह्यम्᳚।
तन्नः॑ प्र॒जां वी॒रव॑तीꣳ सनोतु।
गोभि॑र्नो॒ अश्वैः॒ सम॑नक्तु य॒ज्ञम्॥१२॥ 

वा॒युर्नक्ष॑त्रम॒भ्ये॑ति॒ निष्ट्या᳚म्।
ति॒ग्मशृ॑ङ्गो वृष॒भो रोरु॑वाणः।
स॒मी॒रय॒न् भुव॑ना मात॒रिश्वा᳚।
अप॒ द्वेषाꣳ॑सि नुदता॒मरा॑तीः।
तन्नो॑ वा॒युस्तदु॒ निष्ट्या॑ शृणोतु।
तन्नक्ष॑त्रं भूरि॒दा अ॑स्तु॒ मह्यम्᳚।
तन्नो॑ दे॒वासो॒ अनु॑जानन्तु॒ कामम्᳚।
यथा॒ तरे॑म दुरि॒तानि॒ विश्वा᳚॥१३॥ 

दू॒रम॒स्मच्छत्र॑वो यन्तु भी॒ताः।
तदि॑न्द्रा॒ग्नी कृ॑णुतां॒ तद्विशा॑खे।
तन्नो॑ दे॒वा अनु॑मदन्तु य॒ज्ञम्।
प॒श्चात् पु॒रस्ता॒दभ॑यं नो अस्तु।
नक्ष॑त्राणा॒मधि॑पत्नी॒ विशा॑खे।
श्रेष्ठा॑विन्द्रा॒ग्नी भुव॑नस्य गो॒पौ।
विषू॑चः॒ शत्रू॑नप॒ बाध॑मानौ।
अप॒ क्षुधं॑ नुदता॒मरा॑तिम्॥१४॥ 

पू॒र्णा प॒श्चादु॒त पू॒र्णा पु॒रस्ता᳚त्।
उन्म॑ध्य॒तः पौ᳚र्णमा॒सी जि॑गाय।
तस्यां᳚ दे॒वा अधि॑ सं॒वस॑न्तः।
उ॒त्त॒मे नाक॑ इ॒ह मा॑दयन्ताम्।
पृ॒थ्वी सु॒वर्चा॑ युव॒तिः स॒जोषाः᳚।
पौ॒र्ण॒मा॒स्युद॑गा॒च्छोभ॑माना।
आ॒प्या॒यय॑न्ती दुरि॒तानि॒ विश्वा᳚।
उ॒रुं दुहां॒ यज॑मानाय य॒ज्ञम्॥१५॥ 

ऋ॒द्ध्यास्म॑ ह॒व्यैर्नम॑सोप॒सद्य॑।
मि॒त्रं दे॒वं मि॑त्र॒धेयं॑ नो अस्तु।
अ॒नू॒रा॒धान् ह॒विषा॑ व॒र्धय॑न्तः।
श॒तं जी॑वेम श॒रदः॒ सवी॑राः।
चि॒त्रं नक्ष॑त्र॒मुद॑गात्पु॒रस्ता᳚त्।
अ॒नू॒रा॒धास॒ इति॒ यद्वद॑न्ति।
तन्मि॒त्र ए॑ति प॒थिभि॑र्देव॒यानैः᳚।
हि॒र॒ण्ययै॒र्वित॑तै\-र॒न्तरि॑क्षे॥१६॥ 

इन्द्रो᳚ ज्ये॒ष्ठामनु॒ नक्ष॑त्रमेति।
यस्मि॑न्वृ॒त्रं वृ॑त्र॒तूर्ये॑ त॒तार॑।
तस्मि॑न्व॒यम॒मृतं॒ दुहा॑नाः।
क्षुधं॑ तरेम॒ दुरि॑तिं॒ दुरि॑ष्टिम्।
पु॒र॒न्द॒राय॑ वृष॒भाय॑ धृ॒ष्णवे᳚।
अषा॑ढाय॒ सह॑मानाय मी॒ढुषे᳚।
इन्द्रा॑य ज्ये॒ष्ठा मधु॑म॒द्दुहा॑ना।
उ॒रुं कृ॑णोतु॒ यज॑मानाय लो॒कम्॥१७॥ 

मूलं॑ प्र॒जां वी॒रव॑तीं विदेय।
परा᳚च्येतु॒ निर्‌ऋ॑तिः परा॒चा।
गोभि॒र्नक्ष॑त्रं प॒शुभिः॒ सम॑क्तम्।
अह॑र्भूया॒द्यज॑मानाय॒ मह्यम्᳚।
अह॑र्नो अ॒द्य सु॑वि॒ते द॑धातु।
मूलं॒ नक्ष॑त्र॒मिति॒ यद्वद॑न्ति।
परा॑चीं वा॒चा निर्‌ऋ॑तिं नुदामि।
शि॒वं प्र॒जायै॑ शि॒वम॑स्तु॒ मह्यम्᳚॥१८॥ 

या दि॒व्या आपः॒ पय॑सा सम्बभू॒वुः।
या अ॒न्तरि॑क्ष उ॒त पार्थि॑वी॒र्याः।
यासा॑मषा॒ढा अ॑नु॒यन्ति॒ कामम्᳚।
ता न॒ आपः॒ शꣴ स्यो॒ना भ॑वन्तु।
याश्च॒ कूप्या॒ याश्च॑ ना॒द्याः᳚ समु॒द्रियाः᳚।
याश्च॑ वैश॒न्तीरु॒त प्रा॑स॒चीर्याः।
यासा॑मषा॒ढा मधु॑ भ॒क्षय॑न्ति।
ता न॒ आपः॒ शꣴ स्यो॒ना भ॑वन्तु॥१९॥ 

तन्नो॒ विश्वे॒ उप॑ शृण्वन्तु दे॒वाः।
तद॑षा॒ढा अ॒भिसंय॑न्तु य॒ज्ञम्।
तन्नक्ष॑त्रं प्रथतां प॒शुभ्यः॑।
कृ॒षिर्वृ॒ष्टिर्यज॑मानाय कल्पताम्।
शु॒भ्राः क॒न्या॑ युव॒तयः॑ सु॒पेश॑सः।
क॒र्म॒कृतः॑ सु॒कृतो॑ वी॒र्या॑वतीः।
विश्वा᳚न् दे॒वान् ह॒विषा॑ व॒र्धय॑न्तीः।
अ॒षा॒ढाः काम॒मुप॑ यान्तु य॒ज्ञम्॥२०॥ 

यस्मि॒न् ब्रह्मा॒\-ऽभ्यज॑य॒थ्सर्व॑मे॒तत्।
अ॒मुं च॑ लो॒कमि॒दमू॑ च॒ सर्वम्᳚।
तन्नो॒ नक्ष॑त्रमभि॒जिद्वि॒जित्य॑।
श्रियं॑ दधा॒त्वहृ॑णीय\-मानम्।
उ॒भौ लो॒कौ ब्रह्म॑णा॒ सञ्जि॑ते॒मौ।
तन्नो॒ नक्ष॑त्रमभि॒जिद्विच॑ष्टाम्।
तस्मि॑न्व॒यं पृत॑नाः॒ सञ्ज॑येम।
तन्नो॑ दे॒वासो॒ अनु॑जानन्तु॒ कामम्᳚॥२१॥ 

शृ॒ण्वन्ति॑ श्रो॒णाम॒मृत॑स्य गो॒पाम्।
पुण्या॑मस्या॒ उप॑शृणोमि॒ वाचम्᳚।
म॒हीं दे॒वीं विष्णु॑पत्नीमजू॒र्याम्।
प्र॒तीची॑मेनाꣳ ह॒विषा॑ यजामः।
त्रे॒धा विष्णु॑रुरुगा॒यो विच॑क्रमे।
म॒हीं दिवं॑ पृथि॒वीम॒न्तरि॑क्षम्।
तच्छ्रो॒णैति॒ श्रव॑ इ॒च्छमा॑ना।
पुण्य॒ꣴ॒ श्लोकं॒ यज॑मानाय कृण्व॒ती॥२२॥ 

अ॒ष्टौ दे॒वा वस॑वः सो॒म्यासः॑।
चत॑स्रो दे॒वीर॒जराः॒ श्रवि॑ष्ठाः।
ते य॒ज्ञं पा᳚न्तु॒ रज॑सः प॒रस्ता᳚त्।
सं॒व॒थ्स॒रीण॑म॒मृतꣴ॑ स्व॒स्ति।
य॒ज्ञं नः॑ पान्तु॒ वस॑वः पु॒रस्ता᳚त्।
द॒क्षि॒ण॒तो॑ऽभिय॑न्तु॒ श्रवि॑ष्ठाः।
पुण्यं॒ नक्ष॑त्रम॒भि संवि॑शाम।
मा नो॒ अरा॑तिर॒घश॒ꣳ॒साऽगन्॑{}॥२३॥ 

क्ष॒त्रस्य॒ राजा॒ वरु॑णोऽधिरा॒जः।
नक्ष॑त्राणाꣳ श॒तभि॑ष॒ग्वसि॑ष्ठः।
तौ दे॒वेभ्यः॑ कृणुतो दी॒र्घमायुः॑।
श॒तꣳ स॒हस्रा॑ भेष॒जानि॑ धत्तः।
य॒ज्ञं नो॒ राजा॒ वरु॑ण॒ उप॑यातु।
तन्नो॒ विश्वे॑ अ॒भि संय॑न्तु दे॒वाः।
तन्नो॒ नक्ष॑त्रꣳ श॒तभि॑षग्जुषा॒णम्।
दी॒र्घमायुः॒ प्रति॑रद्भेष॒जानि॑॥२४॥ 

अ॒ज एक॑पा॒दुद॑गात्पु॒रस्ता᳚त्।
विश्वा॑ भू॒तानि॑ प्रति॒ मोद॑मानः।
तस्य॑ दे॒वाः प्र॑स॒वं य॑न्ति॒ सर्वे᳚।
प्रो॒ष्ठ॒प॒दासो॑ अ॒मृत॑स्य गो॒पाः।
वि॒भ्राज॑मानः समिधा॒न उ॒ग्रः।
आऽन्तरि॑क्षमरुह॒दग॒न्द्याम्।
तꣳ सूर्यं॑ दे॒वम॒जमेक॑पादम्।
प्रो॒ष्ठ॒प॒दासो॒ अनु॑यन्ति॒ सर्वे᳚॥२५॥ 

अहि॑र्बु॒ध्नियः॒ प्रथ॑मान एति।
श्रेष्ठो॑ दे॒वाना॑मु॒त मानु॑षाणाम्।
तं ब्रा᳚ह्म॒णाः सो॑म॒पाः सो॒म्यासः॑।
प्रो॒ष्ठ॒प॒दासो॑ अ॒भि र॑क्षन्ति॒ सर्वे᳚।
च॒त्वार॒ एक॑म॒भि कर्म॑ दे॒वाः।
प्रो॒ष्ठ॒प॒दास॒ इति॒ यान् वद॑न्ति।
ते बु॒ध्नियं॑ परि॒षद्यꣴ॑ स्तु॒वन्तः॑।
अहिꣳ॑ रक्षन्ति॒ नम॑सोप॒सद्य॑॥२६॥ 

पू॒षा रे॒वत्यन्वे॑ति॒ पन्था᳚म्।
पु॒ष्टि॒पती॑ पशु॒पा वाज॑बस्त्यौ।
इ॒मानि॑ ह॒व्या प्रय॑ता जुषा॒णा।
सु॒गैर्नो॒ यानै॒रुप॑यातां य॒ज्ञम्।
क्षु॒द्रान् प॒शून् र॑क्षतु रे॒वती॑ नः।
गावो॑ नो॒ अश्वा॒ꣳ॒ अन्वे॑तु पू॒षा।
अन्न॒ꣳ॒ रक्ष॑न्तौ बहु॒धा विरू॑पम्।
वाजꣳ॑ सनुतां॒ यज॑मानाय य॒ज्ञम्॥२७॥ 

तद॒श्विना॑वश्व॒युजोप॑याताम्।
शुभ॒ङ्गमि॑ष्ठौ सु॒यमे॑भि॒रश्वैः᳚।
स्वं नक्ष॑त्रꣳ ह॒विषा॒ यज॑न्तौ।
मध्वा॒ सम्पृ॑क्तौ॒ यजु॑षा॒ सम॑क्तौ।
यौ दे॒वानां᳚ भि॒षजौ॑ हव्यवा॒हौ।
विश्व॑स्य दू॒ताव॒मृत॑स्य गो॒पौ।
तौ नक्ष॑त्रं जुजुषा॒णोप॑याताम्।
नमो॒ऽश्विभ्यां᳚ कृणुमोऽश्व॒युग्भ्या᳚म्॥२८॥ 

अप॑ पा॒प्मानं॒ भर॑णीर्भरन्तु।
तद्य॒मो राजा॒ भग॑वा॒न्॒ विच॑ष्टाम्।
लो॒कस्य॒ राजा॑ मह॒तो म॒हान् हि।
सु॒गं नः॒ पन्था॒मभ॑यं कृणोतु।
यस्मि॒न्नक्ष॑त्रे य॒म एति॒ राजा᳚।
यस्मि॑न्नेनम॒भ्यषि॑ञ्चन्त दे॒वाः।
तद॑स्य चि॒त्रꣳ ह॒विषा॑ यजाम।
अप॑ पा॒प्मानं॒ भर॑णीर्भरन्तु॥२९॥ 

नि॒वेश॑नी स॒ङ्गम॑नी॒ वसू॑नां॒ विश्वा॑ रू॒पाणि॒ वसू᳚न्यावे॒शय॑न्ती। 
स॒ह॒स्र॒पो॒षꣳ सु॒भगा॒ ररा॑णा॒ सा न॒ आग॒न्वर्च॑सा संविदा॒ना॥ यत्ते॑ दे॒वा अद॑धुर्भाग॒धेय॒ममा॑वास्ये सं॒वस॑न्तो महि॒त्वा। 
सा नो॑ य॒ज्ञं पि॑पृहि विश्ववारे र॒यिं नो॑ धेहि सुभगे सु॒वीरम्᳚॥३०॥ 

\closesection


नवो॑ नवो भवति॒ जाय॑मा॒नोऽह्नां᳚ के॒तुरु॒षसा॑मे॒त्यग्रे᳚।
भा॒गं दे॒वेभ्यो॒ विद॑धात्या॒यन् प्रच॒न्द्रमा᳚स्तिरति दी॒र्घमायुः॑॥

यमा॑दि॒त्या अ॒ꣳ॒शु॒मा᳚प्या॒यय॑न्ति॒ यमक्षि॑त॒मक्षि॑तयः॒ पिब॑न्ति।
तेन॑ नो॒ राजा॒ वरु॑णो॒ बृह॒स्पति॒रा प्या॑ययन्तु॒ भुव॑नस्य गो॒पाः॥

ये विरू॑पे॒ सम॑नसा सं॒व्यय॑न्ती। स॒मा॒नं तन्तुं॑ परितात॒नाते᳚।
वि॒भू प्र॒भू अ॑नु॒भू वि॒श्वतो॑ हुवे। ते नो॒ नक्ष॑त्रे॒ हव॒माग॑मेतम्।
व॒यं दे॒वी ब्रह्म॑णा संविदा॒नाः। सु॒रत्ना॑सो दे॒ववी॑तिं॒ दधा॑नाः।
अ॒हो॒रा॒त्रे ह॒विषा॑ व॒र्धय॑न्तः। अति॑ पा॒प्मान॒मति॑मुक्त्यागमेम।
प्रत्यु॑वदृश्याय॒ती। व्यु॒च्छन्ती॑ दुहि॒ता दि॒वः।
अ॒पो म॒ही वृ॑णुते॒ चक्षु॑षा। तमो॒ ज्योति॑ष्कृणोति सू॒नरी᳚।
उदु॒स्त्रियाः᳚ सचते॒ सूर्यः॑। सचा॑ उ॒द्यन्नक्ष॑त्रमर्चि॒मत्।
तवेदु॑षो॒ व्युषि॒ सूर्य॑स्य च। सं भ॒क्तेन॑ गमेमहि।
तन्नो॒ नक्ष॑त्रमर्चि॒मत्। भा॒नुमत्तेज॑ उ॒च्चर॑त्।
उप॑य॒ज्ञमि॒हाग॑मत्।
प्र नक्ष॑त्राय दे॒वाय॑। इन्द्रा॒येन्दुꣳ॑ हवामहे।
स नः॑ सवि॒ता सु॑वथ्स॒निम्। पु॒ष्टि॒दां वी॒रव॑त्तमम्।


उदु॒त्यं जा॒तवे॑दसं दे॒वं व॑हन्ति के॒तवः॑। दृ॒शे विश्वा॑य॒ सूर्यम्᳚।
चि॒त्रं दे॒वाना॒मुद॑गा॒दनी॑कं॒ चक्षु॑र्मि॒त्रस्य॒ वरु॑णस्या॒ग्नेः।
आऽप्रा॒ द्यावा॑ पृथि॒वी अ॒न्तरि॑क्ष॒ꣳ॒ सूर्य॑ आ॒त्मा जग॑तस्त॒स्थुष॑श्च।


आदि॑तिर्न उरुष्य॒त्वदि॑तिः॒ शर्म॑ यच्छतु। अदि॑तिः पा॒त्वꣳह॑सः।
म॒हीमू॒षु मा॒तरꣳ॑ सुव्र॒ताना॑मृ॒तस्य॒ पत्नी॒मव॑से हुवेम।
तु॒वि॒क्ष॒त्राम॒जर॑न्तीमुरू॒चीꣳ सु॒शर्मा॑ण॒\-मदि॑तिꣳ सु॒प्रणी॑तिम्।


इ॒दं विष्णु॒र्विच॑क्रमे त्रे॒धा निद॑धे प॒दम्। समू॑ढमस्य पाꣳसु॒रे।
प्रतद्विष्णुः॑ स्तवते वी॒र्या॑य। मृ॒गो न भी॒मः कु॑च॒रो गि॑रि॒ष्ठाः।
यस्यो॒रुषु॑ त्रि॒षु वि॒क्रम॑णेषु। अधि॑क्षि॒यन्ति॒ भुव॑नानि॒ विश्वा᳚।

अ॒ग्निर्मू॒र्धा दि॒वः क॒कुत्पतिः॑ पृथि॒व्या अ॒यम्। अ॒पाꣳ रेताꣳ॑सि जिन्वति।
भुवो॑ य॒ज्ञस्य॒ रज॑सश्च ने॒ता यत्रा॑नि॒युद्भिः॒ सच॑से शि॒वाभिः॑।
दि॒वि मू॒र्धानं॑ दधिषे सुव॒र्\mbox{}षां जि॒ह्वाम॑ग्ने चकृषे हव्य॒वाहम्᳚।
अनु॑नो॒ऽद्यानु॑मतिर्य॒ज्ञं दे॒वेषु॑ मन्यताम्।
अ॒ग्निश्च॑ हव्य॒वाह॑नो॒ भव॑तां दा॒शुषे॒ मयः॑।
अन्विद॑नुमते॒ त्वं मन्या॑सै॒ शं च॑ नः कृधि।
क्रत्वे॒ दक्षा॑य नोहि नु॒ प्रण॒ आयूꣳ॑षि तारिषः।
ह॒व्य॒वाह॑मभिमाति॒षाहम्᳚। र॒क्षो॒हणं॒ पृत॑नासु जि॒ष्णुम्।
ज्योति॑ष्मन्तं॒ दीद्य॑तं॒ पुर॑न्धिम्। अ॒ग्निꣴ स्वि॑ष्ट॒कृत॒माहु॑वेम।
स्वि॑ष्टमग्ने अ॒भितत्पृ॑णाहि। विश्वा॑ देव॒ पृत॑ना अ॒भिष्य।
उ॒रुं नः॒ पन्थां᳚ प्रदि॒शन्विभा॑हि। ज्योति॑ष्मद्धेह्य॒जरं॑ न॒ आयुः॑॥

अ॒ग्नये॒ स्वाहा॒ कृत्ति॑काभ्यः॒ स्वाहा᳚।\\
अ॒म्बायै॒ स्वाहा॑ दु॒लायै॒ स्वाहा᳚।\\
नि॒त॒त्न्यै स्वाहा॒ऽभ्रय॑न्त्यै॒ स्वाहा᳚।\\
मे॒घय॑न्त्यै॒ स्वाहा॑ व॒र्ष॒य॑न्त्यै॒ स्वाहा᳚।\\
चु॒पु॒णीका॑यै॒ स्वाहा᳚।\\
प्र॒जाप॑तये॒ स्वाहा॑ रोहि॒ण्यै स्वाहा᳚।\\
रोच॑मानायै॒ स्वाहा᳚ प्र॒जाभ्यः॒ स्वाहा᳚।\\
सोमा॑य॒ स्वाहा॑ मृगशी॒र्षाय॒ स्वाहा᳚।\\
इ॒न्व॒काभ्यः॒ स्वाहौष॑धीभ्यः॒ स्वाहा᳚।\\
रा॒ज्याय॒ स्वाहा॒ऽभिजि॑त्यै॒ स्वाहा᳚।\\
रु॒द्राय॒ स्वाहा॒ऽऽर्द्रायै॒ स्वाहा᳚।\\
पिन्व॑मानायै॒ स्वाहा॑ प॒शुभ्यः॒ स्वाहा᳚।\\
अदि॑त्यै॒ स्वाहा॒ पुन॑र्वसुभ्याम्।\\
स्वाहा भू᳚त्यै॒ स्वाहा॒ प्रजा᳚त्यै॒ स्वाहा᳚।\\
बृह॒स्पत॑ये॒ स्वाहा॑ ति॒ष्या॑य॒ स्वाहा᳚।\\
ब्र॒ह्म॒व॒र्च॒साय॒ स्वाहा᳚।\\
स॒र्पेभ्यः॒ स्वाहा᳚ऽऽश्रे॒षाभ्यः॒ स्वाहा᳚।\\
द॒न्द॒शूके᳚भ्यः॒ स्वाहा᳚।\\
पि॒तृभ्यः॒ स्वाहा॑ म॒घाभ्यः॑।\\
स्वाहा॑ऽन॒घाभ्यः॒ स्वाहा॑ऽग॒दाभ्यः॑।\\
स्वाहा॑ऽरुन्ध॒तीभ्यः॒ स्वाहा᳚।\\
अ॒र्य॒म्णे स्वाहा॒ फल्गु॑नीभ्या॒ꣴ॒ स्वाहा᳚।\\
प॒शुभ्यः॒ स्वाहा᳚।\\
भगा॑य॒ स्वाहा॒ फल्गु॑नीभ्या॒ꣴ॒ स्वाहा᳚।\\
श्रैष्ठ्या॑य॒ स्वाहा᳚।\\
स॒वि॒त्रे स्वाहा॒ हस्ता॑य।\\
स्वाहा॑ दद॒ते स्वाहा॑ पृण॒ते।\\
स्वाहा᳚ प्र॒यच्छ॑ते॒ स्वाहा᳚ प्रतिगृभ्ण॒ते स्वाहा᳚।\\
त्वष्ट्रे॒ स्वाहा॑ चि॒त्रायै॒ स्वाहा᳚।\\
चैत्रा॑य॒ स्वाहा᳚ प्र॒जायै॒ स्वाहा᳚।\\
वा॒यवे॒ स्वाहा॒ निष्ट्या॑यै॒ स्वाहा᳚।\\
का॒म॒चारा॑य॒ स्वाहा॒ऽभिजि॑त्यै॒ स्वाहा᳚।\\
इ॒न्द्रा॒ग्निभ्या॒ꣴ॒ स्वाहा॒ विशा॑खाभ्या॒ꣴ॒ स्वाहा᳚।\\
श्रैष्ठ्याय॒ स्वाहा॒ऽभिजि॑त्यै॒ स्वाहा᳚।\\
पौ॒र्ण॒मा॒स्यै स्वाहा॒ कामा॑य॒ स्वाहाऽऽग॑त्यै॒ स्वाहा᳚।\\
मि॒त्राय॒ स्वाहा॑ऽनूरा॒धेभ्यः॒ स्वाहा᳚।\\
मि॒त्र॒धेया॑य॒ स्वाहा॒ऽभिजि॑त्यै॒ स्वाहा᳚।\\
इन्द्रा॑य॒ स्वाहा᳚ ज्ये॒ष्ठायै॒ स्वाहा᳚।\\
ज्यैष्ठ्या॑य॒ स्वाहा॒ऽभिजि॑त्यै॒ स्वाहा᳚।\\
प्र॒जाप॑तये॒ स्वाहा॒ मूला॑य॒ स्वाहा᳚।\\
प्र॒जायै॒ स्वाहा᳚।\\
अ॒द्भ्यः स्वाहा॑ऽषा॒ढाभ्यः॒ स्वाहा᳚।\\
स॒मु॒द्राय॒ स्वाहा॒ कामा॑य॒ स्वाहा᳚।\\
अ॒भिजि॑त्यै॒ स्वाहा᳚।\\
विश्वे᳚भ्यो दे॒वेभ्यः॒ स्वाहा॑ऽषा॒ढाभ्यः॒ स्वाहा᳚।\\
अ॒न॒प॒ज॒य्याय॒ स्वाहा॒ जित्यै॒ स्वाहा᳚।\\
ब्रह्म॑णे॒ स्वाहा॑ऽभि॒जिते॒ स्वाहा᳚।\\
ब्र॒ह्म॒लो॒काय॒ स्वाहा॒ऽभिजि॑त्यै॒ स्वाहा᳚।\\
विष्ण॑वे॒ स्वाहा᳚ श्रो॒णायै॒ स्वाहा᳚।\\
श्लोका॑य॒ स्वाहा᳚ श्रु॒ताय॒ स्वाहा᳚।\\
वसु॑भ्यः॒ स्वाहा॒ श्रवि॑ष्ठाभ्यः॒ स्वाहा᳚।\\
अग्रा॑य॒ स्वाहा॒ परी᳚त्यै॒ स्वाहा᳚।\\
वरु॑णाय॒ स्वाहा॑ श॒तभि॑षजे॒ स्वाहा᳚।\\
भे॒ष॒जेभ्यः॒ स्वाहा᳚।\\
अ॒जायैक॑पदे॒ स्वाहा᳚ प्रोष्ठप॒देभ्यः॒ स्वाहा᳚।\\
तेज॑से॒ स्वाहा᳚ ब्रह्मवर्च॒साय॒ स्वाहा᳚।\\
अह॑ये बु॒ध्निया॑य॒ स्वाहा᳚ प्रोष्ठप॒देभ्यः॒ स्वाहा᳚।\\
प्र॒ति॒ष्ठायै॒ स्वाहा᳚।\\
पू॒ष्णे स्वाहा॑ रे॒वत्यै॒ स्वाहा᳚।\\
प॒शुभ्यः॒ स्वाहा᳚।\\
अ॒श्विभ्या॒ꣴ॒ स्वाहा᳚ऽश्व॒युग्भ्या॒ꣴ॒ स्वाहा᳚।\\
श्रोत्रा॑य॒ स्वाहा॒ श्रुत्यै॒ स्वाहा᳚।\\
य॒माय॒ स्वाहा॑ऽपभ॒रणीभ्यः॒ स्वाहा᳚।\\
रा॒ज्याय॒ स्वाहा॒ऽभिजि॑त्यै॒ स्वाहा᳚।\\
अ॒मा॒वा॒स्या॑यै॒ स्वाहा॒ कामा॑य॒ स्वाहाऽऽग॑त्यै॒ स्वाहा᳚।\\
च॒न्द्रम॑से॒ स्वाहा᳚ प्रती॒दृश्या॑यै॒ स्वाहा᳚।\\
अ॒हो॒रा॒त्रेभ्यः॒ स्वाहा᳚ऽर्धमा॒सेभ्यः॒ स्वाहा᳚।\\
मासे᳚भ्यः॒ स्वाह॒र्तुभ्यः॒ स्वाहा᳚।\\
सं॒व॒थ्स॒राय॒ स्वाहा᳚।\\
अह्ने॒ स्वाहा॒ रात्रि॑यै॒ स्वाहा᳚।\\
अति॑मुक्त्यै॒ स्वाहा᳚।\\
उ॒षसे॒ स्वाहा॒ व्यु॑ष्ट्यै॒ स्वाहा᳚।\\
व्यू॒षुष्यै॒ स्वाहा᳚ व्यु॒च्छन्त्यै॒ स्वाहा᳚।\\
व्यु॑ष्टायै॒ स्वाहा᳚।\\
नक्ष॑त्राय॒ स्वाहो॑देष्य॒ते स्वाहा᳚।\\
उ॒द्य॒ते स्वाहोदि॑ताय॒ स्वाहा᳚।\\
हर॑से॒ स्वाहा॒ भर॑से॒ स्वाहा᳚।\\
भ्राज॑से॒ स्वाहा॒ तेज॑से॒ स्वाहा᳚।\\
तप॑से॒ स्वाहा᳚ ब्रह्मवर्च॒साय॒ स्वाहा᳚।\\
सूर्या॑य॒ स्वाहा॒ नक्ष॑त्रेभ्यः॒ स्वाहा᳚।\\
प्र॒ति॒ष्ठायै॒ स्वाहा᳚।\\
अदि॑त्यै॒ स्वाहा᳚ प्रति॒ष्ठायै॒ स्वाहा᳚।\\
विष्ण॑वे॒ स्वाहा॑ य॒ज्ञाय॒ स्वाहा᳚।\\
प्र॒ति॒ष्ठायै॒ स्वाहा᳚॥\\

द॒धि॒क्राव्ण्णो॑ अकारिषं जि॒ष्णोरश्व॑स्य वा॒जिनः॑।
सु॒र॒भि नो॒ मुखा॑कर॒त् प्रण॒ आयूꣳ॑षि तारिषत्॥

आपो॒ हि ष्ठा म॑यो॒ भुव॒स्तान॑ ऊ॒र्जे द॑धातन। म॒हेरणा॑य चक्ष॑से॥

यो वः॑ शि॒वत॑मो॒ रस॒स्तस्य॑ भाजयते॒ह नः॑। उ॒श॒तीरि॑व मा॒तरः॑॥

तस्मा॒ अरं॑ गमाम वो॒ यस्य॒ क्षया॑य॒ जिन्व॑थ। आपो॑ ज॒नय॑था च नः॥

उदु॑त्त॒मं व॑रुण॒पाश॑म॒स्मदवा॑ध॒मं विम॑ध्य॒मꣴ श्र॑थाय।
अथा॑ व॒यमा॑दित्यव्र॒ते तवाना॑गसो॒ अदि॑तये स्याम।
अस्त॑भ्ना॒द्॒ द्यामृ॑ष॒भो अ॒न्तरि॑क्ष॒ममि॑मीत वरि॒माणं॑ पृथि॒व्या
आसी॑द॒द्विश्वा॒ भुव॑नानि स॒म्राड्विश्वेत्तानि॒ वरु॑णस्य व्र॒तानि॑।

यत्किं चे॒दं व॑रुण॒ दैव्ये॒ जने॑भिद्रो॒हं मनु॒ष्या᳚श्चरा॑मसि। 
अचि॑त्ती॒यत्तव॒ धर्मा॑ युयोपि॒म मा न॒स्तस्मा॒देन॑सो देव रीरिषः॥

कि॒त॒वासो॒ यद्रि॑रि॒पुर्नदी॒वि यद्वा॑ घा स॒त्यमु॒त यं न वि॒द्म। 
सर्वा॒ताविष्य॑ शिथि॒रेव॑ दे॒वाथा॑ ते स्याम वरुण प्रि॒यासः॑॥

अव॑ ते॒ हेडो॑ वरुण॒ नमो॑भि॒रव॑य॒ज्ञेभि॑रीमहे ह॒विर्भिः॑।
क्षय॑न्न॒स्मभ्य॑मसुरप्रचेतो॒ राज॒न्नेनाꣳ॑सिशिश्रथः कृ॒तानि॑॥

तत्त्वा॑ यामि॒ ब्रह्म॑णा॒ वन्द॑मान॒स्तदाशा᳚स्ते॒ यज॑मानो ह॒विर्भिः॑। 
अहे॑डमानो वरुणे॒ह बो॒ध्युरु॑शꣳस॒ मा न॒ आयुः॒ प्रमो॑षीः॥

हिर॑ण्यवर्णाः॒ शुच॑यः पाव॒का यासु॑ जा॒तः क॒श्यपो॒ यास्विन्द्रः॑।
अ॒ग्निं या गर्भं॑ दधि॒रे विरू॑पा॒स्ता न॒ आपः॒ शꣴ स्यो॒ना भ॑वन्तु॥ 
यासा॒ꣳ॒ राजा॒ वरु॑णो॒ याति॒ मध्ये॑ सत्यानृ॒ते अ॑व॒पश्यं॒ जना॑नाम्।
म॒धु॒श्चुतः॒ शुच॑यो॒ याः पा॑व॒कास्ता न॒ आपः॒ शꣴ स्यो॒ना भ॑वन्तु॥ 
यासां᳚ दे॒वा दि॒वि कृ॒ण्वन्ति॑ भ॒क्षं या अ॒न्तरि॑क्षे बहु॒धा भव॑न्ति।
याः पृ॑थि॒वीं पय॑सो॒न्दन्ति॑ शु॒क्रास्ता न॒ आपः॒ शꣴ स्यो॒ना भ॑वन्तु॥ 
शि॒वेन॑ मा॒ चक्षु॑षा पश्यताऽऽपः शि॒वया॑ त॒नुवोप॑ स्पृशत॒ त्वचं॑ मे।
सर्वाꣳ॑ अ॒ग्नीꣳ र॑फ्सु॒षदो॑ हुवे वो॒ मयि॒ वर्चो॒ बल॒मोजो॒ नि ध॑त्त॥

पव॑मानः॒ सुव॒र्जनः॑। प॒वित्रे॑ण॒ विच॑\ur{}षणिः। यः पोता॒ स पु॑नातु मा। पु॒नन्तु॑ मा देवज॒नाः।
पु॒नन्तु॒ मन॑वो धि॒या। पु॒नन्तु॒ विश्व॑ आ॒यवः॑। जात॑वेदः प॒वित्र॑वत्। प॒वित्रे॑ण पुनाहि मा।
शु॒क्रेण॑ देव॒दीद्य॑त्। अग्ने॒ क्रत्वा॒ क्रतू॒ꣳ॒ रनु॑। यत्ते॑ प॒वित्र॑म॒र्चिषि॑। अग्ने॒ वित॑तमन्त॒रा।
ब्रह्म॒ तेन॑ पुनीमहे। उ॒भाभ्यां᳚ देवसवितः। प॒वित्रे॑ण स॒वेन॑ च। इ॒दं ब्रह्म॑ पुनीमहे।
वै॒श्व॒दे॒वी पु॑न॒ती दे॒व्यागा᳚त्। यस्यै॑ ब॒ह्वीस्त॒नुवो॑ वी॒तपृ॑ष्ठाः।
तया॒ मद॑न्तः सध॒माद्ये॑षु। व॒यꣴ स्या॑म॒ पत॑यो रयी॒णाम्।
वै॒श्वा॒न॒रो र॒श्मिभि॑र्मा पुनातु। वातः॑ प्रा॒णेने॑षि॒रो म॑यो॒ भूः।
द्यावा॑पृथि॒वी पय॑सा॒ पयो॑भिः। ऋ॒ताव॑री य॒ज्ञिये॑ मा पुनीताम्।
बृ॒हद्भिः॑ सवित॒स्तृभिः॑। वर्{}षि॑ष्ठैर्देव॒मन्म॑भिः।
अग्ने॒ दक्षैः᳚ पुनाहि मा। येन॑ दे॒वा अपु॑नत।
येनाऽऽपो॑ दि॒व्यं कशः॑। तेन॑ दि॒व्येन॒ ब्रह्म॑णा। इ॒दं ब्रह्म॑ पुनीमहे। यः पा॑वमा॒नीर॒ध्येति॑।
ऋषि॑भिः॒ सम्भृ॑त॒ꣳ॒ रसम्᳚। सर्व॒ꣳ॒ स पू॒तम॑श्नाति।
स्व॒दि॒तं मा॑त॒रिश्व॑ना। पा॒व॒मा॒नीर्यो अ॒ध्येति॑।
ऋषि॑भिः॒ सम्भृ॑त॒ꣳ॒ रसम्᳚। तस्मै॒ सर॑स्वती दुहे। क्षी॒रꣳ स॒र्पिर्मधू॑द॒कम्॥
पा॒व॒मा॒नीः स्व॒स्त्यय॑नीः। सु॒दुघा॒हि पय॑स्वतीः।
ऋषि॑भिः॒ सम्भृ॑तो॒ रसः॑। ब्रा॒ह्म॒णेष्व॒मृतꣳ॑ हि॒तम्।
पा॒व॒मा॒नीर्दि॑शन्तु नः। इ॒मं लो॒कमथो॑ अ॒मुम्।
कामा॒न्थ्सम॑र्धयन्तु नः। दे॒वीर्दे॒वैः स॒माभृ॑ताः।
पा॒व॒मा॒नीः स्व॒स्त्यय॑नीः। सु॒दुघा॒हि घृ॑त॒श्चुतः॑।
ऋषि॑भिः॒ सम्भृ॑तो॒ रसः॑। ब्रा॒ह्म॒णेष्व॒मृतꣳ॑ हि॒तम्।
येन॑ दे॒वाः प॒वित्रे॑ण। आ॒त्मानं॑ पु॒नते॒ सदा᳚।
तेन॑ स॒हस्र॑धारेण। पा॒व॒मा॒न्यः पु॑नन्तु मा।
प्रा॒जा॒प॒त्यं प॒वित्रम्᳚। श॒तोद्या॑मꣳ हिर॒ण्मयम्᳚।
तेन॑ ब्रह्म॒ विदो॑ व॒यम्। पू॒तं ब्रह्म॑ पुनीमहे।
इन्द्रः॑ सुनी॒ती स॒हमा॑ पुनातु। सोमः॑ स्व॒स्त्या वरु॑णः स॒मीच्या᳚।
य॒मो राजा᳚ प्रमृ॒णाभिः॑ पुनातु मा। जा॒तवे॑दा मो॒र्जय॑न्त्या पुनातु। भूर्भुवः॒ सुवः॑।

तच्छं॒ योरावृ॑णीमहे। गा॒तुं य॒ज्ञाय॑।
गा॒तुं यज्ञप॑तये। दैवीः᳚ स्व॒स्तिर॑स्तु नः।
स्व॒स्तिर्मानु॑षेभ्यः। ऊ॒र्ध्वं जि॑गातु भेष॒जम्।
शं नो॑ अस्तु द्वि॒पदे᳚। शं चतु॑ष्पदे।
ॐ शान्तिः॒ शान्तिः॒ शान्तिः॑॥

नमो॒ ब्रह्म॑णे॒ नमो॑ अस्त्व॒ग्नये॒ नमः॑ पृथि॒व्यै नम॒ ओष॑धीभ्यः।
नमो॑ वा॒चे नमो॑ वा॒चस्पत॑ये॒ विष्ण॑वे बृह॒ते क॑रोमि॥

\centerline{॥ॐ शान्तिः॒ शान्तिः॒ शान्तिः॑॥}

\sect{प्रोक्षण मन्त्राः}

आपो॒ हि ष्ठा म॑यो॒ भुव॒स्ता न॑ ऊ॒र्जे द॑धातन।
म॒हेरणा॑य॒ चक्ष॑से। यो वः॑ शि॒वत॑मो॒ रस॒स्तस्य॑ भाजयते॒ह नः॑।
उ॒श॒तीरि॑व मा॒तरः॑। तस्मा॒ अरं॑ गमाम वो॒ यस्य॒ क्षया॑य॒ जिन्व॑थ।
आपो॑ ज॒नय॑था च नः॥

दे॒वस्य॑ त्वा सवि॒तुः प्र॑स॒वे। अ॒श्विनो᳚र्बा॒हुभ्या᳚म्। पू॒ष्णो हस्ता᳚भ्याम्।
अ॒श्विनो॒र्भैष॑ज्येन। तेज॑से ब्रह्मवर्च॒साया॒भिषि॑ञ्चामि॥

दे॒वस्य॑ त्वा सवि॒तुः प्र॑स॒वे। अ॒श्विनो᳚र्बा॒हुभ्या᳚म्। पू॒ष्णो हस्ता᳚भ्याम्।
सर॑स्वत्यै॒ भैष॑ज्येन। वी॒र्या॑या॒न्नाद्या॑या॒भिषि॑ञ्चामि॥

दे॒वस्य॑ त्वा सवि॒तुः प्र॑स॒वे। अ॒श्विनो᳚र्बा॒हुभ्या᳚म्। पू॒ष्णो हस्ता᳚भ्याम्।
इन्द्र॑स्येन्द्रि॒येण॑। श्रि॒ये यश॑से॒ बला॑या॒भिषि॑ञ्चामि॥

दे॒वस्य॑ त्वा सवि॒तुः प्र॑स॒वे᳚ऽश्विनो᳚र्बा॒हुभ्यां᳚ पू॒ष्णो हस्ता᳚भ्या॒ꣳ॒
सर॑स्वत्यै॒ वा॒चो य॒न्तुर्य॒न्त्रेणा॒ग्नेस्त्वा॒ साम्रा᳚ज्येना॒भिषि॑ञ्चामि॥

दे॒वस्य॑ त्वा सवि॒तुः प्र॑स॒वे᳚ऽश्विनो᳚र्बा॒हुभ्यां᳚ पू॒ष्णो हस्ता᳚भ्या॒ꣳ॒
सर॑स्वत्यै॒ वा॒चो य॒न्तुर्य॒न्त्रेण॒ बृह॒स्पते᳚स्त्वा॒ साम्रा᳚ज्येना॒भिषि॑ञ्चामि॥

द्रु॒प॒दादि॑व॒ मुञ्च॑तु। द्रु॒प॒दादि॒वेन्मु॑मुचा॒नः।
स्वि॒न्नः स्ना॒त्वी मला॑दिव। पू॒तं प॒वित्रे॑णे॒वाज्यम्᳚।
आपः॑ शुन्धन्तु॒ मैन॑सः।

आपो॒ वा इ॒दꣳ सर्वं॒ विश्वा॑ भू॒तान्यापः॑ प्रा॒णा वा आपः॑ प॒शव॒ आपोऽन्न॒मापोऽमृ॑त॒मापः॑ स॒म्राडापो॑ वि॒राडापः॑ स्व॒राडाप॒श्छन्दा॒ꣴ॒स्यापो॒ ज्योती॒ꣴ॒ष्यापो॒ यजू॒ꣴ॒ष्यापः॑ स॒त्यमापः॒ सर्वा॑ दे॒वता॒ आपो॒ भूर्भुवः॒ सुव॒राप॒ ओम्॥

\centerline{॥ॐ शान्तिः॒ शान्तिः॒ शान्तिः॑॥}
